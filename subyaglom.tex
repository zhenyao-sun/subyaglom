% * Commits
% ! 2018.12.24 the extinction rate of subcritical superprocesses - 3.tex by R. Liu: Original ideas and proofs.
% ! 2019.01.01 subSuper0101.tex by Z. Sun: changed the tex file into amsart style.
% ! 2019.01.07 subSuper0107.tex by Z. Sun: recommended some more general settings.
% ! 2019.01.21 subSuper0121.tex by R. Liu
% ! 2019.01.28 subSuper0128.tex by Z. Sun: gave more details about the reverse spine decomposition.
% ! 2019.01.31 subSuper0131.tex by R. Liu
% ! 2019.02.20 subSuper0220.tex by Z. Sun: there is a problem in Section 3. 
% * The preamble
% ! This TeX file uses 'amsart' for its document class. 
\documentclass[12pt,a4paper]{amsart}
\setlength{\textwidth}{\paperwidth}
\addtolength{\textwidth}{-2in}
\calclayout
% ! 'amsmath', 'amsthm', 'amsfonts' package is automatically loaded by 'amsart' class.
% ! Below is the configuration of 'amsmath' package.
\numberwithin{equation}{section}
\allowdisplaybreaks
% ! Below is the configuration of 'amsthm' package.
\theoremstyle{plain}
\newtheorem{thm}{Theorem}[section]
\newtheorem{lem}[thm]{Lemma}
\newtheorem{prop}[thm]{Proposition}
\newtheorem{cor}[thm]{Corollaray}
\newtheorem{conj}[thm]{Conjecture}
\theoremstyle{definition}
\newtheorem{defi}[thm]{Definition}
\newtheorem{exa}[thm]{Example}
\newtheorem{asp}{Assumption}
\newtheorem{iss}{Issue}
\newtheorem{rem}[thm]{Remark}
\newtheorem*{ack}{Acknowledgment}
% ! 'inputenc' package is used to specify the set of characters allowed to input into this TeX file.
% \usepackage[utf8]{inputenc}
% ! 'foutenc' package is used to specify the set of characters allowed to output into the generated pdf file.
% \usepackage[T1]{fontenc}
% ! 'amssymb', 'mathrsfs' and 'mathtools' packages provide additional math symbols from the AMS default symbol fonts. They are compatible with the AMS class and is recommended to be used.
\usepackage{amssymb}
\usepackage{mathtools}
\mathtoolsset{showonlyrefs}
\usepackage{mathrsfs}
% ! 'hyperref' package is used to handle cross-referencing in the generated pdf file.
\usepackage[backref]{hyperref}
% ! Other packages
\usepackage{xcolor}
\usepackage{comment}
% * Top matter
\begin{document}
\title
% ! [Subcritical super-diffusions]
{
% ! \large The extinction probability for a class of subcritical super-diffusions
Subcritical Superprocesses
}
\author
% ! [R. Liu, Y.-X. Ren, and Z. Sun]
{
  Rongli Liu, Yan-Xia Ren, Renming Song and Zhenyao Sun}
\address{
  Yan-Xia Ren\\
  School of Mathematical Sciences\\
  Peking University\\
  Beijing, P. R. China, 100871}
\email{
  yxren@math.pku.edu.cn}
\thanks{
  The research of Yan-Xia Ren is supported in part by NSFC (Grant Nos. 11671017  and 11731009).}
\address{
  Rongli Liu\\
  {\bf Information about Rongli Liu}}
\email{
  rlliu@bjtu.edu.cn }
\thanks{
  The research of Rongli Liu is supported in part by NSFC
  (Grant No. 11301261), and the Fundamental Research Funds for the Central Universities (Grant No.  2017RC007)}
\address{
  Zhenyao Sun\\
  School of Mathematical Sciences\\
  Peking University\\
  Beijing, P. R. China, 100871}
\email{
  zhenyao.sun@pku.edu.cn}
\begin{abstract}
\begin{comment}
	In this paper, we discussed the decay rate of the extinction probability of a class of superdiffusions.
  Two different conditional limits, the Yaglom probability and the Q process, are considered.
  We characterize the Yaglom distribution by some function $G(f)$, which satisfies $G(v^t(0,\cdot))=t$ for each $t>0$ and where $v^t(0,x)=-\log(\mathbb P_x(\zeta<t))$.
  It is proved that the Yaglom distribution is the minimal quasi-stationary distribution.
  As for the Q process, we show that under finite LlogL moment assumptions, the Q process has the equilibrium distribution.
  And it is a size-biased Yaglom distribution.
  Moreover, we point out that under some conditional distribution the equilibrium  distribution is an infinitely divisible distribution.
\end{comment}
TBD
\end{abstract}
\maketitle
% * Document body
% ** Section: Introduction
% *** Backgroud
\section{Introduction}
\subsection{Backgroud}
\begin{comment}
  % Suppose $(Z_n, n\ge 1)$ is a Galton-Watson branching process with offspring distribution $\{p_n\}$. 
  Suppose that $(Z_n)_{n \in \mathbb Z_+}$ is a Galton-Watson process with offspring distribution $(p_n)_{n\in \mathbb Z_+}$. 
  % Let $m:=\sum^{\infty}_{n=1}np_n$ be the mean number of children per particle.
  Let $m:=\sum^{\infty}_{n=0}np_n$ be the mean number of children per particle.
  % It is well known that when $m<1$ the extinction probability $q:=\lim_{n\rightarrow\infty}P\left(Z_n=0\right)$ is equal to $1$.
  % That is to say the process $(Z_n)$ is extinct in finite time almost surely.
  % A natural question is what the right decay rate of the probability is.
  A natural question is what the right decay rate of those non-extinction probability $P(Z_n >0)$ is.
  % In 1967, Heathcot, Seneta and Vere-Jone \cite{HeathcoteSenetaVere-Jones1967Refinement} answered the question, giving an LlogL criteria.
  In 1967, Heathcot, Seneta and Vere-Jone \cite{HeathcoteSenetaVere-Jones1967Refinement} answered this question with an LlogL criteria.
  % Let $L$ stand for a random variable with distribution $\{p_n\}$.
  Let $L$ stands for a random variable with distribution $(p_n)_{n \in \mathbb Z_+}$.
  \begin{thm}[\cite{HeathcoteSenetaVere-Jones1967Refinement}]
    The sequence $\{ P(Z_n>0)/m^n\}$ is decreasing. 
    If $m<1$, then the following are equivalent:
    \begin{enumerate}
    \item $\lim_{n\rightarrow\infty}P(Z_n>0)/m^n>0$,
    \item $\sup E[Z_n|Z_n>0]<\infty$,
    \item $E\left[L\log^+ L\right]<\infty$.
    \end{enumerate}
  \end{thm}
  % In 1995, Lyons, Pemantle and Peres developed a martingale change of measure method in \cite{LyonsPemantlePeres1995Conceptual} to give a new proof for this $L\log L$ theorem.
  In 1995, Lyons, Pemantle and Peres developed a martingale change of measure method in \cite{LyonsPemantlePeres1995Conceptual} to give a new proof for this $L\log L$ theorem.

  In \cite{Lambert2001Arbres,Lambert2003Coalescence}, Lambert discusses the similar equivalencies for continuous time continuous state branching process $(Z_t)$.
  Its branching mechanism function $\psi$ is specified by the L\'evy-Khinchin formula
\[
	\psi(\lambda)=-\beta\lambda+\sigma^2\lambda^2+\int_0^\infty \left(e^{-\lambda r}-1+\lambda r\right)n(dr),
\]
where $n(dr)$ is a positive measure on $(0,\infty)$ such that $\int_0^\infty(r^2\wedge r) n(dr)<\infty$ and where
\begin{equation}\label{eq:_extinc_assump_for_continuous}
	\int^\infty\frac{1}{\psi(\lambda)}d\lambda   <   \infty.
\end{equation}
We denote by $\mathbb P_x$ the law of $(Z_t)$ with initial value $Z_0=x>0$.
Then for any $\lambda, t\geq 0$,
\[
	\mathbb P_x\left(\exp\{-\lambda Z_t\}\right)=\exp(-xu_t(\lambda)),
\]
where $u_t(\lambda)$ is the unique solution to the following equation
\begin{equation}
  \begin{cases}
    \dfrac{d u_t(\lambda)}{dt}=-\psi(u_t(\lambda));\\
    u_0(\lambda)=\lambda.
  \end{cases}
\end{equation}
Let $\zeta:=\inf\{t\geq 0; Z_t=0\}$ be the extinction time of $Z$. There is a
nonnegative function $\varphi(t)$ such that for any $x>0$,
\[
	\mathbb P_x(Z_t>0)=\mathbb P_x(\zeta>t)=1-\exp(-x\varphi(t)), \qquad t>0.
\]
When $\beta<0$ (c.f.\cite{Grey1974Asymptotic}),
\[
	\mathbb P_x(\zeta<\infty)=1.
\]
Thus the decay rate of $\mathbb P_x(\zeta>t)$ is determined by the decay rate of
$\varphi(t)$. It is shown (c.f.\cite{Lambert2007Quasistationary}) that for any
$\lambda>0$,
\[
	\lim_{t\rightarrow\infty}\frac{u_t(\lambda)}{\varphi(t)}=G(\lambda),
\]
where $G(\lambda)=\exp(\beta\int_{\lambda}^\infty1/\psi(u)du)$. Moreover, the
following LlogL criteria holds.
\begin{thm}
  \label{thm:equivalent_for_cbp}
	When $\beta<0$, the following items are equivalent
  \begin{itemize}
  \item[$(i).$] $G'(0+)<\infty$;
  \item[$(ii).$] There is a positive constant $c$ such that $\varphi(t)\sim
    c\exp(\beta t)$;
  \item[$(iii).$] $\int_1^\infty r\log r n(dr)<\infty$.
  \end{itemize}
  In this case, $c^{-1}=G'(0+)$.
\end{thm}
As far as the two type dynamics running over large amounts of time are
concerned, special attentions are given to two conditional limits. One is called
Yaglom distribution and another one is called $Q$ process. Here we take the
continuous state branching process $Z$ for example to give the definitions of
the two limits. We can find the corresponding investigations for GW branching
processes in \cite{AthreyaNey1972Branching}.

The Yaglom distribution $\nu_{\beta}$ is the limit probability that for any
$x>0$ and Borel set $A$,
\[
	\lim_{t\rightarrow\infty}\mathbb P_x(Z_t\in A\big|\zeta>t)=\nu_{\beta}(A).
\]
The conditional convergence can be found in \cite{Li2000Asymptotic} where it is
also generalized to conditioning of the type $\{\zeta>t+r\}$ for any finite
$r>0$ instead of $\{\zeta>t\}$. A quasi-stationary distribution (QSD) is a
subinvariant distribution $\nu$ on $(0,\infty)$ satisfying
\[
	\mathbb P_{\nu}(Z_t\in A|\zeta>t)=\nu(A).
\]
It is shown in \cite{Lambert2007Quasistationary} that the Yaglom distribution
$\nu_{\beta}$ is the minimal QSD of $Z$ in the following sense. For any
$\gamma\in[\beta,0)$, there is a unique QSD $\nu_{\gamma}$ associated to the
rate of mass decay $-\gamma$, and there is no QSD associated to $\gamma<\beta$.


Let $\mathcal F_t=\sigma(Z_s,s\leq t), t\geq 0$ be the natural
$\sigma-$filtration generated by $Z$. Then due to
\cite{RenSongZhang2018Williams}, another conditional limit $\widetilde{\mathbb
  P}_x$ is defined in the sense that for any $x,t>0$ and $A\in\mathcal F_t$,
\begin{align}
	\lim_{s\rightarrow\infty}\mathbb P_x(A\big|\zeta>s)=\widetilde{\mathbb P}_x(A).
\end{align}
It can be expressed as an $h-$transform of $\mathbb P_x$ with martingale
$M_t=Z_te^{-\beta t}$ that
\[
	\left.\dfrac{d\widetilde{\mathbb P}_x}{d\mathbb P_x}\right|_{\mathcal F_t}=\frac{M_t}{x}.
\]
Moreover, under $\widetilde{\mathbb P}_x$, the process $Z$ is a branching
process with immigration called $Q$-process. Lambert established the connection
between these two limit distributions in \cite{Lambert2007Quasistationary}. Let
$\Upsilon$ be a random variable whose distribution is the Yaglom distribution
$\nu_\beta$. If $\int_1^\infty r\log r n(dr)<\infty$, then under
$\widetilde{\mathbb P}_x$, $Z_t$ converges in distribution to a positive random
variable $Z_\infty$ as $t\to\infty$, which has the distribution of the
size-biased Yaglom distribution
\[
	\widetilde{\mathbb P}_x(Z_\infty\in dr)=\frac{r}{\mathbb E\Upsilon}\mathbb P(\Upsilon\in dr).
\]
The studies on the Q processes,the Yaglom distributions and the quasi-stationary
distributions for more models we refer to the survey
\cite{MeleardVillemonais2012Quasistationary}, the thesis
\cite{Penisson2010Conditional} and the references therein.

Superdiffusions are a class of branching processes with spatial motions. Similar
to branching processes we introduced above, $0$ is also their absorbing state.
In this paper, we will discuss a class of superdiffusions which will be extinct
in finite time. We establish an $L\log L$ criterion taking use of the spine
method and investigate the Yaglom distributions and $Q$ processes. Moreover, we
study the QSD for our model as well. To state our main results, we need to
introduce the setup we are going to work with first.
\end{comment}
% *** Main results
% **** Definition of Models
% ***** Notations before the definition of superprocesses
\subsection{Main results}
\label{sec:main-results}
Let $E$ be a locally compact separable metric space. 
Denote by $\mathcal B_b(E,\mathbb R)$ the collection of all bounded measurable functions on
$E$. 
Denote by $\mathcal B_b(E, [0,\infty))$ the collection of all non-negative bounded measurable functions on $E$. 
Denote by $\mathcal M(E)$ the space of all finite measures on $E$ equipped with the weak topology. 
Write $\langle f,\mu\rangle$ or $\mu(f)$ for the integral $\int_E f(x)\mu(dx)$ whenever it makes sense.

Let the \emph{spatial motion} $Y=\{(Y_t)_{t\geq 0};(\Pi_x)_{x\in E}\}$ be an $E$-valued Hunt process with its lifetime denoted by $\tau$ and its transition semigroup denoted by $(P_t)_{t\geq
  0}$. 
Let the \emph{branching mechanism} $\psi$ be a function on $E\times[0,\infty)$ given by
\[
	\psi(x,z)
	=-\beta(x)z + \alpha(x)^2 z^2+ \int_{(0,\infty)} (e^{-rz}-1+zr )n(x, dr),\qquad x\in E, z\geq0,
\]
where $\beta, \alpha\in \mathcal B_b(E,\mathbb R)$ and $n$ is a $\sigma$-finite kernel from $E$ to $(0,\infty)$ with
\[
	\sup_{x\in E}\int_0^\infty (r\wedge r^2)n(x,dr)
	<\infty.
\]

% ***** Definition of superprocesses
\par
In this paper, we consider a \emph{$(Y,\psi)$-superprocess} $X$ which is defined as an $\mathcal M(E)$-valued Hunt process $X=\{(X_t)_{t\geq 0}; (\mathbb
P_\mu)_{\mu \in \mathcal M(E)}\}$ satisfying that for any $\mu \in \mathcal M(E), f\in \mathcal B_b(E, [0,\infty))$,
\begin{equation}
  \label{eq:_def_of_vtf}
  \mathbb P_\mu [e^{-\langle f,X_t\rangle}] = e^{-\langle V_tf, \mu\rangle},
  \quad t\geq 0,
\end{equation}
where, for each $f\in\mathcal B_b(E, [0,\infty))$, function $(t,x) \mapsto V_tf(x)$ on $[0,\infty) \times E$ is the unique locally bounded positive solution to the
equation
\begin{equation}\label{eq:FKPP_in_definition}
  V_t f(x) +   \Pi_x\left[\int_0^{t\wedge \tau} \psi \big(\xi_s,V_{t-s} f(\xi_s)\big) ds\right]
	= P_t f(x),
	\quad x \in E,\,\, t \geq 0.
\end{equation}
Here, we say a function $(t,x)\mapsto u(t,x)$ is \emph{locally bounded} on $[0,\infty) \times E$ if for each $T\geq 0$ we have $\sup_{t\in [0,T],x\in E} |u(t,x)| < \infty$. 
See \cite[Theorem 5.11]{Li2011Measurevalued} for the existence of such processes.
% **** Assumptions
% ***** Assumption of spatial motion
% ****** Notations
\par
To simplify the notation, we also write $\mathbb P_x$ for $\mathbb P_{\delta_x}$.
Define the \emph{Feynman-Kac semigroup} $(P^\beta_t)_{t\geq 0}$ such that for any $ f\in \mathcal B_b(E,\mathbb R)$,
\begin{align}
	P^\beta_tf(x)
	:= \Pi_x \big[e^{\int_0^{t} \beta(Y_r)dr} f(Y_t)\mathbf 1_{\{t<\tau\}}\big],
	\quad t\geq 0, x\in E.
\end{align}
It is known, see \cite[Proposition 2.27]{Li2011Measurevalued} for example, $(P^\beta_t)_{t\geq 0}$ is \emph{the mean semigroup} of the superprocess $X$, in the sense that for any $\mu \in \mathcal M(E)$, and $f \in \mathcal B_b(E,\mathbb R)$,
\begin{align} \label{eq:Yaglom_type_result_without_2rd} 
\mathbb P_\mu [\langle f,X_t\rangle] = \mu(P^\beta_t f), \quad t \geq 0.
\end{align}
% ****** Statements
\par
For the spatial motion $Y$, we always assume the following:
\begin{asp}\label{asp:1}
  There exist a $\sigma$-finite measure $m$ with full support on $E$ and a family
  of strictly positive, bounded continuous functions $\{ p_t(\cdot,\cdot): t > 0
  \}$ on $E \times E$ such that
  \begin{align}
  	P_tf(x)
  	= \int_E p_t(x,y) f(y) m(dy),
  	&\quad t>0, x \in E,f \in \mathcal B_b(E,\mathbb R);
  	\\ \int_E p_t(x,y)m(dx)\leq 1, &\quad t>0,y\in E;
  	\\ \int_E \int_E p_t(x,y)^2 m(dx) m(dy)
  	<\infty,
  	&\quad t> 0;
  \end{align}
  and that the functions $x \mapsto \int_E p_t(x,y)^2 m(dy)$ and $y \mapsto \int_E
  p_t(x,y)^2 m(dx)$ are both continuous.
\end{asp}
% ****** Remarks
\par
Under Assumption \ref{asp:1}, it is proved in \cite{RenSongZhang2015Limit,RenSongZhang2017Central} that there exists a family
of strictly positive, bounded continuous functions $\{ p^\beta_t(\cdot,\cdot): t
> 0 \}$ on $E \times E$ such that
\begin{align}
	P^\beta_t f(x)
	= \int_E p_t^\beta (x,y) f(y) m(dy),
	\quad \quad t>0, x \in E,f \in \mathcal B_b(E,\mathbb R).
\end{align}
Its corresponding dual semigroup $(\widehat P^{\beta}_t)_{t \geq 0}$ is given by
\begin{align}
	\widehat P^{\beta}_0 = I;
	\quad \widehat P^{\beta}_t f(y)
	:= \int_E p^\beta_t (x,y) f(x) m(dx),
	\quad t>0, y\in E, f\in \mathcal B_b(E,\mathbb R).
\end{align}
It is shown in \cite{RenSongZhang2015Limit, RenSongZhang2017Central} that both $(P^\beta_t)_{t \geq 0}$ and $(\widehat P_t^{\beta})_{t\geq 0}$ are strongly continuous semigroups of compact operators on $L^2(E,m)$. 
Let $A$ and $\widehat
A$ be the generators of the semigroups $(P^\beta_t)_{t \geq 0}$ and $(\widehat
P^\beta_t)_{t \geq 0}$, respectively. 
Denote by $\sigma(A)$ and $\sigma(\widehat
A)$ the spectra of $A$ and $\widehat A$, respectively. 
According to \cite[Theorem V.6.6]{Schaefer1974Banach}, $\lambda := \sup \text{Re}(\sigma(A))
= \sup \text{Re}(\sigma(\widehat A))$ is a common eigenvalue of multiplicity $1$ for both $A$ and $\widehat A$. 
It is also proved in \cite{RenSongZhang2015Limit,RenSongZhang2017Central} that the eigenfunctions $\phi$ of $A$ and $\widehat\phi$ of $\widehat A$ associated with the common eigenvalue $\lambda$ can be chosen to be strictly positive and continuous everywhere on $E$. 
Normalize $\phi$ and $\widehat\phi$ by
\[	
	\int_E \phi(x)^2 m(dx) = \int_E \phi(x) \widehat \phi(x) m(dx) = 1
\]
so that they are unique.
% ***** Assumtion of mean behavior
% ****** Notations
\par
Notice that, for each $t \geq 0$ and $\mu \in \mathcal M(E)$, we have $ \mathbb
P_\mu[X_t(\phi)] = \mu(P^\beta_t \phi) = e^{\lambda t} \mu(\phi). $ 
If $\lambda
> 0$, the mean of $X_t(\phi)$ will increase exponentially; 
if $\lambda < 0$, the mean of $X_t(\phi)$ will decrease exponentially; 
and if $\lambda = 0$, the mean of $X_t(\phi)$ will be a constant. 
Therefore, we say $X$ is \emph{supercritical, critical} or \emph{subcritical}, according to $\lambda > 0$, $\lambda = 0$ or
$\lambda < 0$, respectively.
% ****** Statements
\par
Throughout this paper, we assume the following for the mean semigroup $(P_t^\beta)_{t\geq 0}$ in addition:
\begin{asp}~
\label{asp:IU}
\begin{enumerate}
\item The superprocess $X$ is subcritical, i.e. $\lambda < 0$.
\item The eigenfunctions $\phi$ and $\widehat\phi$ are bounded on $E$.
\item \label{subasp:IU} 
  The mean semigroup $(P_t^\beta)_{t\geq 0}$ is \emph{intrinsically ultracontractive}, that is, for each $t>0$, there is a constant $c_t >0$ such that for each $x,y\in E$, $p^\beta_t(x,y) \leq c_t \phi(x) \widehat\phi(y)$.
\end{enumerate}
\end{asp}
% ***** Assumption of non-persistent
% ****** Statement
\par
It follows from \cite[Proposition 2.5]{KimSong2008Intrinsic} that, under Assumption \ref{asp:IU}.\eqref{subasp:IU}, for each $t>0$, there exists $c'_t > 0$ such that 
\begin{align}
	p^\beta_t (x,y) 
  \geq c_t' \phi(x) \widehat \phi(y),
  \quad x,y \in E.
\end{align}
Define $\nu(dy):= \widehat \phi(y) m(dy)$. 
Then from above, $\nu$ is a finite measure on $E$.
Therefore we can consider the superprocess $X$ with initial configuration $\nu$.
  
For each measure $\mu$ on $E$, we write $\|\mu\|:= \left\langle \mathbf 1_E,\mu \right\rangle$.
% Let $\zeta=\inf\{t>0: \langle \mathbf 1_E,X_t\rangle=0\}$ be the extinction time of the superprocess $X$.
% Similar to \cite{RenSongSun2019Spine,RenSongZhang2018Williams}, we add the following assumption to assure the process will be extinct in finite time almost surely.
Similar to \cite{RenSongSun2019Spine,RenSongZhang2018Williams}, we add the following non-persistent assumption:
\begin{asp}~ 
  \label{asp:3}
  \begin{enumerate}
  \item \label{subsup:point_non_presistence}
    % $\mathbb P_{x}(\zeta < t)>0$ for each $x\in E$ and $t>0$.
    $\mathbb P_{x}(\|X_t\| = 0)>0$ for each $x\in E$ and $t>0$.
  \item \label{subasp:measure_non_presistence}
    % $\mathbb P_{\nu}(\zeta< t)>0$ for some $t>0$ with $\nu(dx):=\widehat\phi(x)m(dx)$.
    $\mathbb P_{\nu}(\|X_t\| = 0)>0$ for some $t>0$.
  \end{enumerate}
\end{asp}
 Let $\zeta=\inf\{t>0: \|X_t\| = 0\}$ be the extinction time of the superprocess $X$.
 The above assumption says that the process will extinct in finite time with a positive probability provided the initial configuration is $\nu$ or $\delta_x$ with $x\in E$.

% **** Results in this paper
% ***** Results about extinction time zeta
% ****** Notation: definition of the n-phi kernel
\par
Define a new kernel $n^\phi(x, dr)$ from $E$ to $(0,\infty)$ such that
\begin{equation} \label{eq:phi_change}
	\int_0^\infty f(r)n^\phi(x,dr)=\int_0^\infty f(r\phi(x))n(x, dr),
	\quad x\in E, f\in \mathcal B_b((0,\infty), \mathbb R).
\end{equation}
Our first result is about the asymptotic behavior of the extinction time $\zeta$:

% ****** statement of the result
\begin{thm}\label{thm:distribution_of_zeta}
	Suppose that the superprocess $X$ satisfies Assumptions \ref{asp:1},
  \ref{asp:IU}, and \ref{asp:3}. Then,
  \begin{enumerate}
  \item
    \label{subthm:extinct_almost_sure}
    for each $\mu \in \mathcal M(E)$, we have  $\mathbb P_\mu(\zeta<\infty)=1$;
  \item
    for each $\mu,\widetilde\mu\in \mathcal M(E)\setminus\{0\}$ and $s>0$, we have
    \[
      \lim_{t\rightarrow\infty}\dfrac{\mathbb P_{\mu}(\zeta>t+s)}{\mathbb P_{\widetilde\mu}(\zeta>t)}=\frac{\langle \phi,\mu\rangle }{\langle \phi,\widetilde\mu\rangle }e^{\lambda s};
    \]
  \item
  \label{subthm:LlogL}
    there exists a constant $k\in [0,\infty)$, such that for any $x\in E$,
    \begin{equation}\label{eq:decay_rate}
      \lim_{t\rightarrow\infty} e^{-\lambda t}\mathbb P_x(\zeta>t)=k\phi(x).
    \end{equation}
    Moreover, the constant $k>0$ if and only if $\int_E \widehat\phi(y)l(y)m(dy)<\infty$ where
    \begin{equation}\label{eq:m}
      l(y):=\int_1^\infty r\log r~n^\phi(y, dr),\quad y \in E.
    \end{equation}
  \end{enumerate}
\end{thm}

% ****** some explanation
\begin{comment}
In particular, for any $x,y\in E$ and $s\geq 0$, the second result in the above theorem can be written as
\begin{equation}\label{eq:ratio_result}
 	\lim_{t\rightarrow\infty}\frac{\mathbb P_x(\zeta>t+s)}{\mathbb P_y(\zeta>t)}=\frac{\phi(x)}{\phi(y)}e^{\lambda s}.
\end{equation}
So we can see that the effect of the position of the initial mass on the decay of the mass is a ratio of $\phi(\cdot)$ generally.
\end{comment}
\begin{comment}
% ***** Result about QSD and Yaglom theorem
% ****** Concept of QSD
For each probability ${\mathbf P}$ on $\mathcal M(E)$, we define
\[
	( {\mathbf P} \mathbb P)(\cdot) := \int_{\mathcal M(E)} \mathbb P_\mu(\cdot) {\mathbf P}(d\mu).
\]
Then $\{(X_t)_{t\geq 0}; ({\mathbf P}\mathbb P)\}$ can be considered as a $(Y,\psi)$-superprocess with a random initial value $X_0$ whose distribution is ${\mathbf P}$.
We say a probability ${\mathbf P}$ on $\mathcal M(E)$ is a \emph{quise-stationary distribution (QSD)} of the superprocess $X$ if  for each $t\geq 0$,
\[
	({\mathbf P}\mathbb P)(X_t \in \cdot | \zeta > t) ={\mathbf P}(\cdot).
\]
According to the standard theory of QSD (see \cite{MeleardVillemonais2012Quasistationary}), if ${\mathbf P}$ is a QSD of $X$, then under $({\mathbf P}\mathbb P)$, the lifetime $\zeta$ has an exponential distribution with some constant $r > 0$, that is
\[
	( {\mathbf P}\mathbb P)(\zeta > t) = e^{-r t}.
\]
We refer to $r$ the \emph{rate of mass decay} associated to the QSD $\mathbf P$.

% ****** Concept of Yaglom distribution
We say a probability ${\mathbf P}$ on $\mathcal M(E)$ is the \emph{Yaglom distribution} of the superprocess $X$ if for any $\mu\in \mathcal M(E)\setminus\{0\}$ we have
\[
	\mathbb P_\mu(X_t \in \cdot | \zeta > t) \xrightarrow[t\to \infty]{w} {\mathbf P}(\cdot).
\]
If the Yaglom distribution ${\mathbf P}$ of the superprocess $X$ exists, then it must be a QSD of $X$ (see \cite{MeleardVillemonais2012Quasistationary}).

% ****** Statement of the result
Our second theorem is about the QSD and the Yaglom distribution of the superpocess $X$:
\begin{thm}\label{thm:qsd_thm}
  Suppose that the superprocess $X$ satisfies Assumptions \ref{asp:1}, \ref{asp:IU}, and \ref{asp:3}.
  Then,
  \begin{enumerate}
  \item \label{thm:qsd_thm_1}
    for each $\gamma\in[\lambda,0)$, there is a unique probability measure ${\mathbf P}^{\gamma}$ on $\mathcal M(E)$ such that $ {\mathbf P}^\gamma$ is a $QSD$ of the superprocess $X$ with rate of mass decay $-\gamma$.
    Let $\nu(dx):=\widehat\phi(x) m(dx)$.
    Then ${\mathbf P}^\gamma$ is the distribution of the random measure $M^{(\gamma)}\nu(dx)$ where $M^{(\gamma)}$ is a non-negative random variable with Laplace transform
    \[
      E[e^{-\theta M^{(\gamma)}}]
      = 1 - e^{\gamma B(\theta)},
      \quad \theta \in (0,\infty).
    \]
    Here, map $B: \theta \mapsto B(\theta)$ with $\theta \in (0,\infty)$ is defined as the inverse of the map
    \[
      t
      \mapsto -\log \mathbb P_\nu(\zeta \leq t),
      \quad t\in (0,\infty).
    \]
  \item
    ${\mathbf P}^\lambda$ is the Yaglom distribution of $X$.
    In another word, for any  $f\in\mathcal B_b(E,[0,\infty))$ and any $\mu\in \mathcal M(E)\setminus\{0\}$,
    \[
      \lim_{t\rightarrow\infty}\mathbb P_{\mu}[e^{-\langle f,X_t\rangle} | \zeta>t]
      % = E [e^{- \langle f,\nu\rangle M^{(\lambda)}}],
      = E [e^{- \langle f,\nu\rangle M^{(\lambda)}}].
    \]
    In particular, for each $\mu \in \mathcal M(E)$, $\{\langle \phi, X_t\rangle; \mathbb P_{\mu}(\cdot| \zeta > t) \}$ converges to $M^{(\lambda)}$ in distribution.
  \item
    There is no QSD with rate of mass decay $-\gamma$ for each $\gamma\in(-\infty , \lambda)$.
  \end{enumerate}
\end{thm}

% ***** Result: QSD has infinite mean, and Yaglom law has finite mean iff the llogl condition is satisfied
The following Proposition \ref{eq:exp_prop} is an analog to Theorem \ref{thm:equivalent_for_cbp}. 
\begin{prop}\label{eq:exp_prop}
	Suppose that the superprocess $X$ satisfies Assumptions \ref{asp:1},
  \ref{asp:IU}, and \ref{asp:3}.
	Let $M^{(\gamma)}$ be the random variables given by Theorem \ref{thm:qsd_thm} (1) where $\gamma \in [\lambda, 0)$.
  Then
  \begin{enumerate}
  \item
    $E[M^{(\gamma)}] = \infty$ for each $\gamma \in (\lambda, 0)$, and
  \item
    $E[M^{(\lambda)}] < \infty$ if and only if $\int_E \widehat\phi(y)l(y)m(dy)<\infty$.
  \end{enumerate}
  Moreover, if $\int_E \widehat\phi(y)l(y)m(dy)<\infty$ then the constant $k$ in \eqref{eq:decay_rate} is equal to $E[M^{(\lambda)}]^{-1}$.
\end{prop}
% Recall the LlogL criteria of Theorem \ref{thm:equivalent_for_cbp}, we can say the above equivalency in proposition \ref{eq:exp_prop} is the same to that between items $(i)$ and $(iii)$ there.
% ***** Result: Q-process of the superprocess
% ****** Notations: a martingale change of measure 
\par
Define the process
\[
	M_t=e^{-\lambda t}  \langle \phi, X_t\rangle, \quad t\geq 0.
\]
It is well known that the process  $(M_t)_{t\geq 0}$ is a martingale with respect to the natural filtration $(\mathscr F_t)_{t\geq 0}$ of the superprocess $X$.
% For each $\mu \in \mathcal M(E)$, define probability $\widetilde{\mathbb P}_\mu$ as Doob's $h-$transform of $\mathbb P_\mu$
For each $\mu \in \mathcal M(E)$, let probability $\widetilde{\mathbb P}_\mu$ be Doob's $h-$transform of $\mathbb P_\mu$ such that
\begin{equation} \label{eq:martingale_transformation}
	\frac{d\widetilde{\mathbb P}_\mu|_{\mathscr F_t}}{d\mathbb P_\mu|_{\mathscr F_t}}
	=\frac{M_t}{\langle\phi,\mu\rangle },
	\quad t\geq 0.
\end{equation}
This kind of martingale measure transformation for branching processes and measure-valued processes have been widely studied.
We refer to the early papers \cite{EnglanderKyprianou2004Local,Evans1993Two,RoellyRouault1989Processus}, the thesis \cite{Penisson2010Conditional} and the references therein, and the recent papers \cite{ChampagnatRoelly2008Limit,RenSongSun2019Spine,RenSongZhang2018Williams}.
It is well known that the process $\{(X_t)_{t\geq 0}; \widetilde{\mathbb P}_{\mu}\}$ can be characterized by the so called spine decomposition theorem.
We will recall this decomposition in details for our model in section $2$.

% ****** Statement of the result
Our third theorem says that $\{(X_t)_{t\geq 0}; \widetilde{\mathbb P}_{\mu}\}$ can be considered as the Q-processs of $X$, i.e. the process $\{(X_t)_{t\geq 0}; \mathbb P_{\mu}\}$ conditioned to be never extinct:
\begin{thm}\label{thm:Q_process}
	Under the assumptions \ref{asp:1},\ref{asp:IU} and \ref{asp:3}, for each  $\mu \in \mathcal M(E), t\geq 0$ and $A\in\mathscr F_t$, we have $\lim_{s\rightarrow\infty}\mathbb P_\mu(A |\zeta>s)=\widetilde{\mathbb P}_\mu(A). $
\end{thm}

% ***** Result: Asymptotic behavior of Q-process
% ****** Concept: invariant probability of Q-process 
It would be interesting to study the asymptotic behavior of this Q-process $\{(X_t)_{t\geq 0}; (\widetilde{\mathbb P}_\mu)_{\mu \in \mathcal M(E)}\}$.
Samilar to the definition of $\mathbf P\mathbb P$, for each probability $\mathbf
P$ on $\mathcal M(E)$, we define probability $\mathbf P\widetilde{\mathbb P}$.
Then $\{(X_t)_{t\geq 0}; (\mathbf P\widetilde{\mathbb P})\}$ can be considered as the Q-process with a random initial value $X_0$ whose distribution is $\mathbf P$.
We say a probability $\mathbf P$ on $\mathcal M(E)$ is an \emph{invariant probability} of the Q-process $\{(X_t)_{t\geq 0}; (\widetilde{\mathbb P}_\mu)_{\mu\in\mathcal M(E)}\}$ if
\[
	(\mathbf P\widetilde{\mathbb P})(X_t \in \cdot ) =\mathbf P(\cdot),	\quad t\geq 0.
\]
% ****** Statement of the main result
Our fourth result is the following:
\begin{thm}\label{thm:structure_of_Qprocess}
	Under the assumptions \ref{asp:1},\ref{asp:IU} and \ref{asp:3}:
  \begin{enumerate}
  \item
    If $\int_E\widehat\phi(x)l(x)m(dx)<\infty$, then $\{(X_t)_{t\geq 0};(\widetilde{\mathbb P}_\mu)_{\mu\in\mathcal M(E)}\}$ has an invariant probability ${\mathbf P}$.
    For any $\mu\in\mathcal M(E)$, we have
\begin{align}
\label{eq:uniqueness_of_invariant_probability}
\widetilde{\mathbb P}_\mu(X_t \in \cdot ) \xrightarrow[t\to \infty]{w} {\mathbf P}(\cdot).
\end{align}
    The invariant probability $\mathbf P$ is the distribution of the random measure $M\widehat\phi(x)m(dx)$, where the non-negative random variable $M$ has Laplace transform
    \[
      E[e^{-\theta  M}] = \dfrac{E[M^{(\lambda)}e^{-\theta M^{(\lambda)}}]}{E[M^{(\lambda)}]},\quad \theta > 0.
    \]
  \item
    If $\int_E\widehat\phi(x)l(x)m(dx)<\infty$, then there is a random measure $K$ on $(0,\infty)$ such that
    \[
      E[e^{-\theta \widetilde M}] = E\Big[\exp\Big\{- \int_{(0,\infty)} (1-e^{-\theta z }) K(dz) \Big\}\Big].
    \]
  \item
    If $\int_E\widehat\phi(x)l(x)m(dx)=\infty$, then for each $\mu \in \mathcal M(E)$, we have $\lim_{t\rightarrow\infty}\langle \phi, X_t\rangle =\infty$ in probability with respect to $\widetilde{\mathbb P}_\mu$.
  \end{enumerate}
\end{thm}
\begin{iss}[TODO]~
  \begin{itemize}
  \item[ZS:]
    Which notation for the random variable in the above theorem is better? $\widetilde M$ or $M$? There is a notation inconsistency in the above theorem, we should fix it. I recommend using ``widetilde'' for ``size-biased-related'' stuff.
  \item[Ren:] 
Agree.
\item[ZS:]
We will use  $M$ instead of $\widetilde M$
  \end{itemize}
\end{iss}
\begin{iss}[TODO]~
  \begin{itemize}
  \item[ZS:]
I have a question about the terminology. Why we call that probability in the above theorem the ``equilibrium probability''? Why not call it ``invariant probability'' or ``invariant measure''?
\item[Ren:] Prefer "invariant probability"
\item[ZS:] We will use ``invariant probability'' instead of ``equilibrium probability''.
  \end{itemize}
\end{iss}
\begin{iss}[open]~
  \begin{itemize}
  \item[ZS:]
Why we only considered the existence of that equilibrium probability in the above theorem? Is it unique?
  \end{itemize}
\end{iss}
\begin{iss}[open]~
  \begin{itemize}
  \item[ZS:]
    Can I have this conjecture: Under $\mathbf P \widetilde{\mathbb P}$, $(X_t)_{t\geq 0}$ is a stationary (or even better, an ergodic) measure-valued process?
\item[Ren:]
  It is safe to say stationary because we have not prove uniqueness.
  \end{itemize}
\end{iss}
% **** Some comment about all the results
\begin{iss}[TODO]~
  \begin{itemize}
  \item[ZS:]
Should we extend the following paragraph into a new subsection with maybe more details about the spine-decomposition?
 \item[Ren:] 
  Yes, we should introduce more on it .  
\end{itemize}
 \end{iss}
One important technique used to prove the above theorems is a ``spine-decomposition'' for the super-diffusion $X$ under a martingale change of measure.
This decomposition was used by Englander and Kyprianou in \cite{EnglanderKyprianou2004Local} to investigate the local extinction of super-diffusions, in which the branching mechanism is $\psi(x,z)-\beta(x)z=\alpha(x)^2z^2-\beta(x)z$.
This technique is usually used to investigate the properties of supercritical superdiffusions ($\lambda>0$).
Here we use it to analyze the subcritical case.
\end{comment}
% ** Section: Preliminaries
% *** Spine process and its time reverse
% **** Spine as the FK-transform of the spatial motion
% ***** Dfinition of FK-transform
\section{Preliminaries}
\subsection{Spine process and its time reverse}
Let $\{(Y_t)_{t\geq 0}; (\Pi_x)_{x\in E}\}$ be the spatial motion introduced in Section 1 with assumption \ref{asp:1} and \ref{asp:IU}.
For each $x\in E$, let the probability $\widetilde \Pi_{x}$ be Doob's $h$-transform of $\Pi_x$ such that
\begin{align}
	\dfrac{d\widetilde{\Pi}_x|_{\mathscr F^Y_t}}{d\Pi_x|_{\mathscr F^Y_t}}= \frac{e^{\int_0^t \beta(Y_s)ds}\phi(Y_t) \mathbf 1_{\{t<\tau\}}}{e^{\lambda t}\phi(x)},
	\quad t\geq 0,
\end{align}
where $(\mathscr F_t^Y)_{t\geq 0}$ is the natural filtration of process $(Y_t)_{t\geq 0}$.
For each $\mu \in \mathcal M(E)$, define
\[
	\Pi_{\mu}(\cdot)
	:= \mu(E)^{-1}\int_{E} \Pi_x(\cdot)\mu(dx),
\]
and
\[
	\widetilde\Pi_{\mu}(\cdot):= \mu(E)^{-1} \int_E\widetilde\Pi_x(\cdot)\mu(dx).
\]
For each function $f \in \mathcal B_b(E,[0,\infty))$ and measure $\mu \in \mathcal M(E)$, define measure $( f \cdot\mu)$ such that
\[
  (f \cdot \mu)(dx)
  := f(x)\mu(dx),
  \quad x\in E.
\]
% ***** Lemma: FK-transform will influence the initial configuration.
\begin{lem}
	For each $\mu\in \mathcal M(E)$, we have
 \[
    \dfrac{\widetilde \Pi_{\phi\cdot\mu}|_{\mathscr F_t^Y}}{\Pi_{\mu}|_{\mathscr F_t^Y}}
  	= \frac{e^{\int_0^t \beta(Y_s)ds}\phi(Y_t) \mathbf 1_{\{t<\tau\}}}{\mu(E)^{-1}e^{\lambda t} \mu(\phi)},
  	\quad t\geq 0.
  \]
\end{lem}
\begin{proof}
	Fix an arbitrary time $t\geq 0$. Fix an arbitrary event $A \in \mathscr
  F_t^Y$. Then we have
  \begin{align}
    &\widetilde{\Pi}_{\phi\cdot\mu}(A)
      = \mu(\phi)^{-1} \int_E \widetilde \Pi_x(A)\phi(x)\mu(dx)
    \\&= \mu(\phi)^{-1} \int_E  \Pi_x\left[\frac{e^{\int_0^t \beta(Y_s)ds}\phi(Y_t) \mathbf 1_{\{t<\tau\}}}{e^{\lambda t}\phi(x)} \mathbf 1_A\right]\phi(x)\mu(dx)
    \\&= \mu(E)^{-1}\int_E  \Pi_x\left[\frac{e^{\int_0^t \beta(Y_s)ds}\phi(Y_t) \mathbf 1_{\{t<\tau\}}}{\mu(E)^{-1}e^{\lambda t} \mu(\phi)} \mathbf 1_A\right]\mu(dx)
    \\&= \Pi_{\mu}\left[ \frac{e^{\int_0^t \beta(Y_s)ds}\phi(Y_t) \mathbf 1_{\{t<\tau\}}}{\mu(E)^{-1} e^{\lambda t}\mu(\phi)} \mathbf 1_A \right].
    \qedhere
  \end{align}
\end{proof}

% ***** Fact: this process is exponential-ergodic
It can be verified (see \cite{KimSong2008Intrinsic} for example) that process $\{(Y_t)_{t\geq 0}; (\widetilde\Pi_x)_{x\in E}\}$ is a time homogeneous Markov process.
Its' transition density with respect to measure $m$ is given by
\begin{equation}
  \label{eq:tilde_p}
  \widetilde p_t(x, y)
  :=\frac{\mbox{e}^{-\lambda t}}{\phi(x)}\ p^\beta_t(x, y)\phi(y),
  \quad x,y \in E,t>0.
\end{equation}
It can also be verified that $\phi(y)\widehat{\phi}(y)m(dy)$ is the unique invariant measure of $\{(Y_t)_{t\geq 0}; (\widetilde\Pi_x)_{x\in E}\}$.
It follows from \cite[Theorem 2.7]{KimSong2008Intrinsic} that for each $\epsilon > 0$, there exists $c, \rho > 0$ such that 
\begin{align}
\label{eq:IU}
	\sup_{x,y\in E}\left|\frac{\widetilde p_t(x,y)}{\phi(y) \widehat\phi(y)}- 1\right|
	=\sup_{x,y\in E}\left|\frac{e^{-\lambda t}p^\beta_t(x,y)}{\phi(x) \widehat\phi(y)}- 1\right|
	\leq c\,e^{-\rho t},
	\quad t\geq \epsilon.
\end{align}
% **** Reverse of the Spine process
% ***** Construction
\par
Let $\{(\widehat{Y}_t)_{t\geq 0}; (\widehat{\Pi}_x)_{x\in E}\}$ be an $E$-valued Hunt process whose transition density with respect to measure $m$ is given by
\[
  \widehat{p}_t(x,y)
  =e^{-\lambda t}p^\beta_t(y,x)\frac{{\widehat\phi}(y)}{{\widehat\phi}(x)}
  =\widetilde p_t(y,x)\frac{\phi(y){\widehat\phi}(y)}{\phi(x){\widehat\phi}(x)},
  \quad x,y \in E,\,\, t> 0.
\]
It is easy to check that $(\widehat Y_t)_{t\geq 0}$ has the unique invariant measure $\phi(x)\widehat\phi(x)m(dx)$.
({\bf Sun: We should verify that $\widetilde Y$ is a well defined Hunt process.})
% ***** Proof of the time reverse property.
\par
Recall that $\nu(dx):=\widehat\phi(x)m(dx)$.
\begin{lem}
  \label{lem:referse_of_the_spine}
	For each $T > 0$, we have
  \[
    \left \{(Y_{T-t})_{0\leq t\leq T}; \widetilde \Pi_{\phi \cdot \nu} \right\}
    \overset{d}{=} \left\{ ( \widehat Y_{t} )_{0\leq t\leq T}; \widehat \Pi_{\phi \cdot \nu} \right\}
  \]
\end{lem}
\begin{proof}
	Fix arbitrary $T>0$, $n \in \mathbb N$ and $0= t_1\leq \dots \leq t_n = T$.
	For each $i=1,\dots, n$, choose an arbitrary $B_i \in \mathscr B(E)$.
	We only need to show that
  \[
    \widetilde \Pi_{\phi \cdot \nu} \left(Y_{T-t_i}\in B_i,\forall i=1,\dots, n \right)
    =\widehat \Pi_{\phi \cdot \nu} \left( \widehat Y_{t_i}\in B_i,\forall i=1,\dots, n \right)
  \]
	In fact, on one hand, we have
  \begin{align}
    &\widetilde \Pi_{\phi \cdot \nu} \left( Y_{T-t_i}\in B_i,\forall i=1,\dots, n \right)
    \\&\begin{multlined} =\int_{y_n\in B_n} \phi(y_n)\widehat\phi(y_n) m(dy_n)\int_{y_{n-1}\in B_{n-1}} \widetilde p_{t_n - t_{n-1}}(y_n,y_{n-1})m(dy_{n-1})
    \\ \dots \int_{y_1\in B_1} \widetilde p_{t_2 - t_1}(y_2,y_1)m(dy_1) \end{multlined}
    \\&= \int_{E^n} \left(\prod_{i=1}^n \mathbf 1_{\{y_i\in B_i\}}\right) \cdot \left( \prod_{i=1}^{n-1} \widetilde p_{t_{i+1}-t_i}(y_{i+1},y_i) \right) \cdot \phi(y_n) \widehat \phi(y_n) \cdot \left( \prod_{i=1}^nm(dy_i) \right).
  \end{align}
	On the other hand, we have
  \begin{align}
    &\widehat \Pi_{\phi\cdot\nu}\{Y_{t_i}\in B_i, \forall i = 1,\dots, n\}
    \\&\begin{multlined} = \int_{y_1\in B_1} \phi(y_1)\widehat \phi(y_1)m(dy_1) \int_{y_2\in B_2} \widehat p_{t_2-t_1}(y_1,y_2)m(dy_2)
    \\ \dots\int_{y_n\in B_n}\widehat p_{t_n-t_{n-1}}(y_{n-1},y_n)m(dy_n) \end{multlined}
    \\&\begin{multlined} = \int_{y_1\in  B_1} \phi(y_1)\widehat \phi(y_1)m(dy_1) \int_{y_2\in B_2} \widetilde p_{t_2-t_1}(y_2,y_1)\frac{\phi(y_2)\widehat \phi(y_2)}{\phi(y_1)\widehat \phi(y_1)}m(dy_2)
    \\ \dots\int_{y_n\in B_n}\widetilde p_{t_n-t_{n-1}}(y_n,y_{n-1})\frac{\phi(y_n)\widehat\phi(y_n)}{\phi(y_{n-1})\widehat\phi(y_{n-1})}m(dy_n) \end{multlined}
    \\&= \int_{E^n} \left( \prod_{i=1}^n \mathbf 1_{\{y_i\in B_i\}}\right) \cdot \left(\prod_{i=1}^{n-1} \widetilde p_{t_{i+1}-t_i}(y_{i+1},y_i) \right) \cdot \phi(y_n) \widehat \phi(y_n) \cdot \left( \prod_{i=1}^n m(dy_i)\right).
    \qedhere
  \end{align}
\end{proof}

% *** Subsection: Kuznestuv Measure
\subsection{Kuznestuv measure}
Suppose that $X$ is the superprocess introduced in Section \ref{sec:main-results} which satisfies that $\mathbb P_{x}(\|X_t\| = 0)>0$ for each $x\in E$ and $t>0$. Denote by
\begin{align} 
\begin{multlined} \mathbb D:=\{ w= (w_t)_{t\geq 0}: w \text{ is an $\mathcal M(E)$-valued c\`{a}dl\`{a}g function on $[0,\infty)$ }
	\\ \text{ with the null measure as a trap} \} \end{multlined}
\end{align}
the Skorokhod space of measure-valued excursion paths.
({\bf Sun: There might be  a tiny problem. In Li's book, he focuses on the right-continuous paths. 
But we always uses the Cadlag paths.}) 

According to \cite[Section 8.4]{Li2011Measurevalued}, there is a unique family of $\sigma$-finite measures $(\mathbb N_x)_{x\in E}$ on $\mathbb D$ such that
\begin{itemize}
\item
  $\mathbb N_x \{\forall t > 0, \|w_t\| =0\} =0$ for each $x\in E$;
\item
  $\mathbb N_x \{ \|w_0\| > 0\} = 0$ for each $x\in E$;
\item
  for each $\mu \in \mathcal M(E)$, if $\mathcal N$ is a Poisson random measure on $\mathbb D$ with intensity
  \[
    \mathbb N_\mu(dw):= \int_E \mathbb N_x(dw)\mu(dx), \quad w\in \mathbb D.
  \]
	then
  \[
    \left(\int_{\mathbb D} w_t~\mathcal N(dw)\right)_{t> 0}
    \overset{f.d.d.}{=} \{(X_t)_{t> 0};\mathbb P_\mu\}.
\]
\end{itemize}
This family of measure $(\mathbb N_x)_{x\in E}$ is known as the \emph{Kuznetsov measures} of $X$.

% In the remainder of this paper, we will always use $w = (w_t)_{t\geq 0}$ to denote a generic element in $\mathbb D$.
% For each $f\in \mathcal B_b^+(D)$, from Campbell's formula it can be verified that for each $x\in E$ and $t>0$, we have
% \begin{align}\label{eq: kuznetsov Laplace}
%   \mathbb N_x[1-e^{-w_t(f) }]
%   &=-\log \mathbb P_x[e^{-X_t(f)}] = V_t f(x),
%   \\ \mathbb N_x[w_t(f)]
%   &=P_t^{\beta}f(x),
%   \\\mathbb N_x\{w_t(\mathbf 1_E) \neq 0\}
%   &=-\log\mathbb P_x\{X_t(\mathbf 1_E) = 0\}.
% \end{align}

In the remainder of this paper, we will always use $w = (w_t)_{t\geq 0}$ to denote a generic element in $\mathbb D$.
For each $f\in \mathcal B_b(\mathbb D, [0,\infty))$, from Campbell's formula it can be verified that for each $\mu\in \mathcal M(E)$ and $t>0$, we have
\begin{align}\label{eq: kuznetsov Laplace}
 	\mathbb N_\mu \left[1-e^{-w_t(f) }\right]
 	&=-\log \mathbb P_\mu \left[ e^{-X_t(f)} \right] 
    = \mu(V_t f),
 	\\ \mathbb N_\mu [w_t(f)]
 	&= \mathbf P_\mu [X_t(f)]
    =\mu(P_t^{\beta}f),
\end{align}
and
\begin{align}
  \label{eq:Nmeasure_survive_is_superprocess_extinction}
 	\mathbb N_\mu( \|w_t\| > 0)
 	=-\log\mathbb P_\mu( \| X_t\|= 0).
\end{align}
% *** section: Spine decomposition
\subsection{Spine decomposition}
Suppose that $X$ is the superprocess introduced in Section \ref{sec:main-results} which satisfies Assumption \ref{asp:1} and \ref{asp:3} \eqref{subsup:point_non_presistence}.
Fix an arbitrary $\mu\in \mathcal M(E)$.
Define the probability $\widetilde {\mathbb P}_\mu$ using \eqref{eq:martingale_transformation}.

{\bf (Sun: I need to introduce the generalized spine decomposition in my paper here. Because the size-biased transform of the Kuznetsov measure will be used in the proof.)}

For each $\mu \in \mathcal M(E)$, we say $\{(Y)_{t\geq 0}, (X^{\mathrm n,
  \sigma})_{\sigma\in \mathcal D^\mathrm n}, (X^{\mathrm m, \sigma})_{\sigma \in
  \mathcal D^\mathrm m}, (X_t)_{t\geq 0}; \mathbb Q_{\mu}\}$ is a \emph{spine representation} of $\{(X_t)_{t\geq 0}; \widetilde {\mathbb P}_\mu\}$ if the followings are true:
\begin{itemize}
\item
  The \emph{spine process} $\{(Y_t)_{t\geq 0}; \mathbb Q_\mu\}$ is a copy of
  $\{(Y_t)_{t\geq 0}; \widetilde \Pi_{\phi\cdot\mu}\}$.
\item
  Given $\{(Y_t)_{t\geq 0}; \mathbb Q_\mu\}$, \emph{the continuum immigration} $\{ (X^{\mathrm n,\sigma})_{\sigma \in \mathcal D^\mathrm n};
  \mathbb Q_\mu(\cdot |Y)\}$ is a $\mathbb D$-valued point process such that
  \[
    \mathrm n(ds,dw) := \sum_{\sigma\in \mathcal D^{\mathrm n}} \delta_{(\sigma, X^{\mathrm n,\sigma})}(ds,dw)
  \]
  is a Poission random measure on $[0,\infty)\times \mathbb D$ with intensity
  \[
    \mathbf n(ds,dw):= 2 \alpha(Y_s) ds \cdot \mathbb N_{Y_s}(dw).
  \]
\item
  Given $\{(Y_t)_{t\geq 0}; \mathbb Q_\mu\}$, \emph{the discrete immigration} $\{(X^{\mathrm m,\sigma})_{\sigma\in \mathcal D^{\mathrm m}};
  \mathbb Q_\mu(\cdot |Y)\}$ is a $\mathbb D$-valued point process such that
  \[
    \mathrm m(ds,dw) := \sum_{\sigma\in \mathcal D^{\mathrm m}} \delta_{(\sigma, X^{\mathrm m,\sigma})}(ds,dw)
  \]
  is a Poisson random measure on $[0,\infty ) \times \mathbb D$ with intensity
  \begin{align}\label{eq:meanMeasImmigr}
    \mathbf m(ds,dw):= ds \cdot \int_{(0,\infty)} y \mathbb P_{y\delta_{Y_s}}(X\in dw) n(Y_s,dy);
  \end{align}
\item
  Given $\{(Y_t)_{t\geq 0}; \mathbb Q_\mu\}$, the continuum immigration $(X^{\mathrm n,\sigma})_{\sigma \in \mathcal D^n}$ and the discrete immigration $(X^{\mathrm m,\sigma})_{\sigma\in \mathcal D^{\mathrm m}}$ are independent of each other.
\item
	% $\{(X_t)_{t\geq 0}; \mathbb Q_\mu\}$ is a copy of the superprocess
	% $\{(X_t)_{t\geq 0}; \mathbb P_\mu\}$ which is independent of the spine
	% process $(Y_t)_{t\geq 0}$, the continuum immigration $(X^{\mathrm
	% n,\sigma})_{\sigma \in \mathcal D^n}$ and the discrete immigration
	% $(X^{\mathrm m,\sigma})_{\sigma\in \mathcal D^{\mathrm m}}$.
	$\{(X_t)_{t\geq 0}; \mathbb Q_\mu\}$ is a copy of the superprocess $\{(X_t)_{t\geq 0}; \mathbb P_\mu\}$, and is independent of the spine process $(Y_t)_{t\geq 0}$, the continuum immigration $(X^{\mathrm n,\sigma})_{\sigma
    \in \mathcal D^\mathrm n}$ and the discrete immigration $(X^{\mathrm
    m,\sigma})_{\sigma\in \mathcal D^{\mathrm m}}$.
\end{itemize}

% For each $\mu \in \mathcal M(E)$,
To simplify the notations, for each $\mu \in \mathcal M(E)$, $t\geq 0$ and each
$B \in \mathscr B([0,t))$, with respect to probability $\mathbb Q_\mu$, define
the following random measures:
\begin{align}
	Z^{\mathrm n,B}_t
	&:= \int_{B\times \mathbb D} w_{t-s} ~\mathrm n (ds,dw)
   = \sum_{\sigma \in \mathcal D^\mathrm n \cap B} X^{\mathrm n,\sigma}_{t-\sigma},
	\\ Z^{\mathrm m,B}_t
	&:= \int_{B\times \mathbb D} w_{t-s} ~\mathrm m (ds,dw)
   = \sum_{\sigma \in \mathcal D^\mathrm m \cap B} X^{\mathrm m,\sigma}_{t-\sigma}.
\end{align}

({\bf Sun: Measurability issue?})

The spine decomposition theorem (see \cite{RenSongSun2019Spine} for the general
cases) says that
\begin{lem}\label{lem:spine_structure}
	Suppose that $\{(Y)_{t\geq 0}, (X^{\mathrm n, \sigma})_{\sigma\in \mathcal
    D^\mathrm n}, (X^{\mathrm m, \sigma})_{\sigma \in \mathcal D^\mathrm m},
  (X_t)_{t\geq 0}; \mathbb Q_{\mu}\}$ is a spine representation of
  $\{(X_t)_{t\geq 0}; \widetilde {\mathbb P}_\mu\}$. Then
  \begin{align}
    \left\{(X_t)_{t\geq 0}; \widetilde{\mathbb P}_\mu\right\}
    \overset{f.d.d.}{=}
    \left\{ \left(X_t + Z^{\mathrm n, [0,t)}_{t} + Z^{\mathrm m, [0,t)}_{t} \right)_{t\geq 0}; \mathbb Q_\mu \right\}.
  \end{align}
\end{lem}

% *** Reverse spine representation
\subsection{Reverse spine representation}

Suppose that $X$ is the superprocess introduced in Section \ref{sec:main-results} which satisfies Assumption \ref{asp:1} and \ref{asp:3} \eqref{subsup:point_non_presistence}.

Recall that $\nu := \widehat \phi \cdot m \in \mathcal M(E)$.
Define the probability $\widetilde {\mathbb P}_\nu$ using \eqref{eq:martingale_transformation}.
We say $\{(Y)_{t\geq 0}, (X^{\mathrm n, \sigma})_{\sigma\in \mathcal D^\mathrm n}, (X^{\mathrm m, \sigma})_{\sigma \in \mathcal D^\mathrm m}, (X_t)_{t\geq 0}; \widehat {\mathbb Q}_{\nu}\}$ is a \emph{reverse spine representation} of $\{(X_t)_{t\geq 0}; \widetilde {\mathbb P}_\nu\}$ if the followings are true:
\begin{itemize}
\item
  The \emph{reverse spine process} $\{(Y_t)_{t\geq 0}; \widehat {\mathbb Q}_\nu\}$ is a copy of $\{(Y_t)_{t\geq 0}; \widehat \Pi_{\phi\cdot\nu}\}$.
\item
  Conditioned on $\{(Y_t)_{t\geq 0}; \widehat{\mathbb Q}_\nu\}$, \emph{the reverse continuum immigration} $\{ (X^{\mathrm n,\sigma})_{\sigma \in \mathcal D^\mathrm n}; \widehat{\mathbb Q}_\nu(\cdot |Y)\}$ is a $\mathbb D$-valued point process such that
  \[
    \mathrm n(ds,dw)
    = \sum_{\sigma\in \mathcal D^{\mathrm n}} \delta_{(\sigma, X^{\mathrm n,\sigma})}(ds,dw)
  \]
	is a Poission random measure on $[0,T]\times \mathcal W$ with density
  \[
    \mathbf n(ds,dw)= 2\alpha(Y_s) ds \cdot \mathbb N_{Y_s}(dw).
  \]
\item
  Conditioned on $\{(Y_t)_{t\geq 0}; \widehat{\mathbb Q}_\nu\}$, \emph{the reverese discrete immigration} $\{(X^{\mathrm m,\sigma})_{\sigma\in \mathcal D^{\mathrm m}}; \widehat{\mathbb Q}_\nu(\cdot |Y)\}$ is a $\mathbb D$-valued point process such that
  \[
    \mathrm m(ds,dw)
    = \sum_{\sigma\in \mathcal D^{\mathrm m}} \delta_{(\sigma, X^{\mathrm m,\sigma})}(ds,dw)
  \]
	is a Poisson random measure on $[0,\infty ) \times \mathbb D$ with intensity
  \begin{align}\label{eq:meanMeasImmigr}
    \mathbf m(ds,dw)= ds \cdot \int_{(0,\infty)} y \mathbb P_{y\delta_{Y_s}}(X\in dw) n(Y_s,dy);
  \end{align}
\item
	Given $\{(Y_t)_{t\geq 0}; \widehat{\mathbb Q}_\nu\}$, the reverse continuum immigration $(X^{\mathrm n,\sigma})_{\sigma \in \mathcal D^n}$ and the reverse discrete immigration $(X^{\mathrm m,\sigma})_{\sigma\in \mathcal D^{\mathrm m}}$ are independent of each other.
\item
	$\{(X_t)_{t\geq 0}; \widehat {\mathbb Q}_\nu\}$ is a copy of the superprocess $\{(X_t)_{t\geq 0}; \mathbb P_\mu\}$ which is independent of the reverse spine process $(Y_t)_{t\geq 0}$, the reverse continuum immigration $(X^{\mathrm n,\sigma})_{\sigma \in \mathcal D^n}$ and the reverse discrete immigration $(X^{\mathrm m,\sigma})_{\sigma\in \mathcal D^{\mathrm m}}$.
\end{itemize}

To simplyfy the notations, for each $t\geq 0$, with respect to probability $\widehat{\mathbb Q}_\nu$, define the following random measures:
\[\begin{split}
    \widehat Z^{\mathrm n}_t
    &:= \int_{[0,t)\times \mathbb D} w_{s} ~\mathrm n (ds,dw)
    = \sum_{\sigma \in \mathcal D^\mathrm n \cap [0,t)} X^{\mathrm n,\sigma}_{\sigma},
    \\ \widehat Z^{\mathrm m}_t
    &:= \int_{[0,t)\times \mathbb D} w_{s} ~\mathrm m (ds,dw)
    = \sum_{\sigma \in \mathcal D^\mathrm m \cap [0,t)} X^{\mathrm m,\sigma}_{\sigma}.
  \end{split}\]
\begin{lem}
\label{lem:usage_of_reverse_spine_decomposition}
	Suppose that $\{X, Y, \mathrm n, \mathrm m; \widehat{\mathbb Q}_{\nu}\}$ is a reverse spine representation of $\{X; \widetilde {\mathbb P}_\nu\}$.
	Then for each $t\geq 0$,
  $
	\{X_t; \widetilde{\mathbb P}_\nu\}
	\overset{d}{=}
	\{X_t + \widehat Z^{\mathrm n}_{t} + \widehat Z^{\mathrm m}_{t}; \widehat{\mathbb Q}_\nu\}.
  $
\end{lem}
\begin{proof}
	Fix an arbitrary time $t\geq 0$. According to Lemma \ref{lem:spine_structure}, we only have to proof that
  \[
    \{Z^{\mathrm n,[0,t)}_{t} + Z^{\mathrm m,[0,t)}_{t}; \mathbb Q_\nu\}
    \overset{d}{=}
    \{\widehat Z^{\mathrm n}_{t} + \widehat Z^{\mathrm m}_{t}; \widehat{\mathbb Q}_\nu\}.
  \]
	In fact, for each $f\in \mathcal B_b(E,[0,\infty))$, from campbell's formula, we have
  \begin{align}
    &-\log \mathbb Q_\nu \left [\left. e^{-\langle f, Z^{\mathrm n,[0,t)}_{t} + Z^{\mathrm m,[0,t)}_{t}\rangle}\right |(Y_t)_{t\geq 0}\right]
    \\&= \int_{[0,t)\times \mathbb D} \left(1-e^{- \langle f, w_{t-s}\rangle}\right)\left(\mathbf n(ds,dw) + \mathbf m(ds,dw)\right)
    \\&= \int_{[0,t)} \left(2\alpha(Y_s) \cdot \mathbb N_{Y_s}\left(1-e^{-w_{t-s}(f)}\right) + \int_{(0,\infty)} y \mathbb P_{y\delta_{Y_s}}\left(1-e^{-X_{t-s}(f)}\right)n(Y_s,dy)\right) ds
    \\&= \int_{[0,t)} \left(2\alpha(Y_s) \cdot (V_{t-s}f)(Y_s) + \int_{(0,\infty)} y \left(1-e^{-y\cdot(V_{t-s}f)(Y_s)}\right)n(Y_s,dy)\right) ds
    \\&= \int_{[0,t)} \psi_0'\left( Y_s, V_{t-s}f(Y_s)\right)ds
  \end{align}
	and
  \begin{align}
    &-\log \widehat{\mathbb Q}_\nu \left [\left. e^{-(\widehat Z^{\mathrm n}_{t} + \widehat Z^{\mathrm m}_{t})(f)}\right |(Y_t)_{t\geq 0}\right]
    \\&= \int_{[0,t)\times \mathbb D} \left(1-e^{- w_s(f)}\right)\left(\mathbf n(ds,dw) + \mathbf m(ds,dw)\right)
    \\&= \int_{[0,t)} \left(2\alpha(Y_s) \cdot \mathbb N_{Y_s}\left(1-e^{-w_{s}(f)}\right) + \int_{(0,\infty)} y \mathbb P_{y\delta_{Y_s}}\left(1-e^{-X_{s}(f)}\right)n(Y_s,dy)\right) ds
    \\&= \int_{[0,t)} \left(2\alpha(Y_s) \cdot (V_{t}f)(Y_s) + \int_{(0,\infty)} y \left(1-e^{-y\cdot(V_{t}f)(Y_s)}\right)n(Y_s,dy)\right) ds
    \\&= \int_{[0,t)} \psi_0'\left( Y_s, V_{t}f(Y_s)\right)ds.
  \end{align}
	Therefore, according to Lemma \ref{lem:referse_of_the_spine}, for each $f\in \mathcal B_b(E,[0,\infty))$, we have
  \begin{align}
  	&\mathbb Q_\nu  \big[e^{-(Z^{\mathrm n,[0,t)}_{t} + Z^{\mathrm m,[0,t)}_{t})(f)}\big]
     = \widetilde \Pi_{\phi\cdot\nu} \big[e^{-\int_{[0,t)} \psi_0'( Y_s, V_{t-s}f(Y_s))ds}\big]
  	\\&= \widetilde \Pi_{\phi\cdot\nu} \big[e^{-\int_{[0,t)} \psi_0'( Y_{t-s}, V_{s}f(Y_{t-s}))ds}\big]
  	= \widehat \Pi_{\phi\cdot\nu} \big[e^{-\int_{[0,t)} \psi_0'( Y_{s}, V_{s}f(Y_{s}))ds}\big]
  	\\&= \widehat{\mathbb Q}_\nu \big [e^{-(\widehat Z^{\mathrm n}_{t} + \widehat Z^{\mathrm m}_{t})(f)}\big].
  	\qedhere
  \end{align}
\end{proof}

% *** subsection: Llog L critierion
\subsection{LlogL criterion.}

{\bf (Sun: Settings of this subsection.)}
Both of the processes $\{(Y)_{t\geq 0}, \widetilde\Pi_x\}$ and $\{(Y)_{t\geq 0}, \widehat{\Pi}_x\}$ are ergodic and have the same invariant probability $\phi(x)\widehat\phi(x)m(dx)$.
The transition probability of these two processes both have uniform convergence properties \eqref{eq:IU} and \eqref{eq:IU'}.
Therefore we can repeat the arguments for Lemma $3.2$ in \cite{LiuRenSong2009Log} and obtain the following results.
The proofs will be omitted.

\begin{lem}\label{lem:import_lemma}
	Let $\mu \in \mathcal M(E)\setminus \{0\}$.
	Suppose that \[\{(Y)_{t\geq 0}, (X^{\mathrm n, \sigma})_{\sigma\in \mathcal D^\mathrm n}, (X^{\mathrm m, \sigma})_{\sigma \in \mathcal D^\mathrm m}, (X_t)_{t\geq 0}; \mathbb Q_{\mu}\}\] is a spine representation of $\{(X_t)_{t\geq 0}; \widetilde {\mathbb P}_\mu\}$.
	With respect to probability $\mathbb Q_\mu$, let $(m_\sigma)_{\sigma\in \mathcal D^{\mathrm m}}$ be the $\mathbb R^+$-valued point process defined by
  \[
  	m_\sigma = X^{\mathrm m, \sigma}_0(\mathbf 1_E),
  	\quad \sigma \in \mathcal D^{\mathrm m}.
  \]
	Then the following sequence of random variables
  \[
    \sigma_0=0,\quad \sigma_i=\inf\{s\in\mathcal D^{\mathrm m}:\ s>\sigma_{i-1},\ m_s\phi(Y_s)>1\}, \quad\eta_i=m_{\sigma_i},\quad i=1,2,\cdots.
  \]
	are well defined with respect to probability $\mathbb Q_\mu$.
	Furthermore, 
  \begin{itemize}
  \item
    if $\int_E\widehat{\phi}(y)l(y)m(dy)<\infty$ then
  \[
    \sum_{s\in\mathcal D^{\mathrm m}}\mbox{e}^{\lambda s}m_s\phi(Y_s) < \infty, \quad
    % \mathbb Q_{\mu}-\mbox{a.s.}
    \mathbb Q_{\mu}\text{-a.s.};
  \]
\item
  if $ \int_E\widehat{\phi}(y)l(y)m(dy)=\infty$, then
  \[
    \limsup_{i\rightarrow\infty}e^{\lambda \sigma_i}\eta_i
    \phi(Y_{\sigma_i})=\infty,
    \quad \mathbb Q_{\mu}\text{-a.s.}.
  \]
\end{itemize}
\end{lem}

{\bf (Sun: I'm not so sure about this Lemma.)}
\begin{lem}\label{lem:import_lemma}
Let $\nu := \widehat \phi \cdot m$.
Suppose that
\[\{(Y)_{t\geq 0}, (X^{\mathrm n, \sigma})_{\sigma\in \mathcal D^\mathrm n}, (X^{\mathrm m, \sigma})_{\sigma \in \mathcal D^\mathrm m}, (X_t)_{t\geq 0}; \widehat{\mathbb Q}_{\nu}\}\]
is a reverse spine representation of $\{(X_t)_{t\geq 0}; \widetilde {\mathbb P}_\nu\}$.
With respect to probability $\widehat{\mathbb Q}_\nu$, let $(m_\sigma)_{\sigma\in \mathcal D^{\mathrm m}}$ be the $\mathbb R^+$-valued point process defined by
  \[
    m_\sigma
    = X^{\mathrm m, \sigma}_0(\mathbf 1_E),
    \quad \sigma \in \mathcal D^{\mathrm m}.
  \]
Then the following sequence of random variables
  \[
    \sigma_0=0,\quad \sigma_i=\inf\{s\in\mathcal D^{\mathrm m}:\ s>\sigma_{i-1},\ m_s\phi(Y_s)>1\}, \quad\eta_i=m_{\sigma_i},\quad i=1,2,\cdots
  \]
  are well defined.
  Furthermore, 
  \begin{itemize}
  \item
    if $\int_E\widehat{\phi}(y)l(y)m(dy)<\infty$ then
  \[
    \sum_{s\in\mathcal D^{\mathrm m}} e^{\lambda s}m_s\phi(Y_s)
    < \infty,
    \quad \widehat{\mathbb Q}_{\mu}\text{-a.s.};
  \]
\item
  if $ \int_E\widehat{\phi}(y)l(y)m(dy)=\infty$, then
  \[
    \limsup_{i\rightarrow\infty} e^{\lambda \sigma_i}\eta_i \phi(Y_{\sigma_i})
    =\infty,
    \quad \widehat{\mathbb Q}_{\mu}\text{-a.s.}.
  \]
\end{itemize}
\end{lem}

% ** Section: The proofs of main results
% *** Subsection about a Partial differential equation
% **** Determine the asymptotic behavior of <v(t,*),\mu>
% ***** Definition of v(t,x)
\section{The proofs of main results}
\subsection{Some properties of the solutions of partial differential equations}
\begin{iss}[Open]~
  \begin{itemize}
  \item[ZS:]
    In this section, I will use a new set of notations called the ``Big O notation''. More precisely, let $\mathcal F$ be a family of non-negative function on space $E$; consider a functional $A:\mathcal F \mapsto [0,\infty]$; we say a function $f \in O(A)$ iff
    \begin{align}
A |f| < \infty;
    \end{align}
and say $f \in o(A)$ iff
\begin{align}
A |f|  = 0.
\end{align}
\item[ZS:]
For example, we can write the following:
\begin{align}
a + e^{-x} &= O(\limsup_{x\to \infty}), \quad a\geq 0,
\\ a e^{-x} &= o(\sup_{a> 0} \limsup_{x\to \infty}),
\\ e^{-x+a} &= o(\limsup_{x\to \infty} \sup_{a>0}). 
\end{align}
Here, the equal sign ``$=$'' should be understood as ``belongs to'', i.e. ``$\in $''.
This is a common convention for Big-O notations. 
We can even write
\begin{align}
  o(\limsup_{x\to \infty} \sup_{a> 0}) O(\limsup_{x\to \infty} \sup_{a > 0}) = o(\limsup_{x\to \infty} \sup_{a>0})
  \end{align}
  where the equal sign should be understood as ``$\subset$''.
\item[ZS:]
  One advantage I love most for using big O notation is that it avoids of using a lot of constants $C$. 
Sometimes, it is very annoying to have too many constants, because you have to track the dependency of them with each parameters.
\item[ZS:]
  You may notice that the Big O notations used here are not the standard one.
  One disadvantage for the commonly used standard big O notations, i.e. $O(1)$ and $o(1)$, or $O(a_n)$ and $o(a_n)$, is that, it sometimes makes you feel not rigorous.
  Some authors write $f = O(g)$ meaning that $|f/g|$ is bounded;
  Some authors write $f_t = O(g_t)$ meaning that $\limsup_{t\to \infty}|f_t/g_t| < \infty$; 
  Some authors don't even bother mentioning the exact meaning of their Big-O notations.
  Some authors simply don't use big O notations, but they have a very long list of constants whose dependencies are not clearly stated.
\item[ZS:]
  This is a new set of notations. 
  We need to introduce them formally in this paper.
  \end{itemize}
\end{iss}
We first give a Lemma which will be used several times:

\begin{lem}
\label{lem:Pf_and_fnu}
  Denote by $L_1^+(\nu)$ the class of non-negative measurable functions on $E$ which are integrable with respect to measure $\nu$.
For each $\epsilon > 0$, we have
  \begin{align}
    P^\beta_t f(x) = e^{\lambda t} O(\sup_{t\geq \epsilon, x\in E, f\in L_1^+(\nu)}) \left\langle f, \nu \right\rangle.  
  \end{align}
\end{lem}
\begin{proof}
  It can be verified that, for each $\epsilon > 0$, there exists $\rho > 0$, such that
  \begin{align}
    & P^\beta_t f(x) = \int_{y\in E} p^\beta_t (x,y)f(y) m(dy)
    \\&  = \int_{y\in E}  e^{\lambda t}\left (1+ e^{-\rho t} O(\sup_{t \geq \epsilon, x,y\in E, f\in L_1^+(\nu)})\right)\phi(x) \widehat \phi(y) f(y) m(dy) ,\quad\text{by~\eqref{eq:IU}}
    \\&  =   e^{\lambda t}\left(1+ e^{-\rho t} O(\sup_{t \geq \epsilon, x\in E, f\in L_1^+(\nu)})\right)\phi(x) \left\langle f,\nu \right\rangle \label{eq:IU_for_Ptf} 
    \\&= e^{\lambda t} O(\sup_{t\geq \epsilon, x\in E, f\in L_1^+(\nu)}) \left\langle f, \nu \right\rangle. 
\qedhere 
  \end{align}
\end{proof}

Recall the operator $(V_t)_{t\geq 0}$ defined in~\eqref{eq:_def_of_vtf}.
For each $f\in \mathcal B(E, [0,\infty]),t\geq 0$ and $x\in E$, using the monotone convergence theorem we can define
\begin{align}
\label{eq:Vtf_is_finite}
  V_tf(x)
  := \lim_{n\to \infty}V_t(f\wedge n)(x)
  = - \log \mathbb P_x[e^{-\left\langle f, X_t \right\rangle}]
  \leq -\log \mathbb P_x(\|X_t\| = 0).
\end{align}
Then $V_0$ is the identity map from $\mathcal B(E,[0,\infty])$ to $\mathcal B(E,[0,\infty])$; and if $t > 0$, $V_t$ is a map from $\mathcal B(E,[0,\infty])$ to $\mathcal B(E, [0,\infty))$, since according to Assumption \ref{asp:3},
\begin{align}
V_tf(x) \leq - \log \mathbb P_x(\|X_t\| = 0 ) < \infty, 
\quad t >0, x\in E.
\end{align}
We call $(V_t)_{t\geq 0}$ the cumulant semigroup of the superprocess $X$, because it holds that $V_tV_sf(x) = V_{t+s}f(x)$ for all $f\in \mathcal B(E,[0,\infty])$, $t,s \geq 0$ and $x\in E$. 
In particular, we write
\[
	v_t(x):= V_t(\infty)(x)= -\log \mathbb P_x(\| X_t\| = 0) < \infty,
	% \quad t > 0, x\in \mathbb R^d.
  \quad t > 0, x\in E.
\]
\begin{iss}[TODO]~
  \begin{itemize}
  \item[ZS:]
Latter, it will be used that $-\log \mathbb P_\mu(\|X_t\| = 0)$ is linear in measure $\mu$ at several places.
I think it is better to reasoning this here.
  \end{itemize}
\end{iss}
\begin{iss}[TODO]~
  \begin{itemize}
  \item[ZS:]
It would be better to mention Li's book as a good reference on this cumulant semigroups
  \end{itemize}
\end{iss}
% ***** Lemma about the asymptotic behavior of <\mu, v_t>
\begin{iss}[OPEN]~
  \begin{itemize}
  \item[ZS:]
I'm confident with the proof of the following Lemma now. 
I would like to ask Rongli to double check the proof of this key Lemma.
\item[ZS:]
There are some new features I want to mention here:
a. The testing function $f$ now can take extended value $\infty$. So this Lemma also covers $v_t$ which can be considered as $V_t(\infty)$.
b. Instead of finding the upper bound and the lower bound separately, we proof the assertion (3) of this Lemma in one stroke with the help of the Big-O notations. 
  \end{itemize}
\end{iss}
\begin{lem}
\label{lem:asmptotic_of_Vtf}
  Suppose that Assumptions \ref{asp:1}, \ref{asp:IU} and \ref{asp:3} hold. 
  Let $f\in \mathcal B(E, [0,\infty])$.
  Then, the followings are hold:
  \begin{enumerate}
  \item \label{subVtf_vanish}
    For each $\mu \in \mathcal M(E)$, we have
    \[
      \lim_{t\rightarrow\infty}\langle V_tf,\mu\rangle=0.
    \]
   \item \label{lem:extinct_2_1}
For each $s>0$, we have
     \begin{align}
       \lim_{t\to \infty} \frac{\left\langle V_{t+s}f,\nu\right\rangle}{\left\langle V_t f,\nu\right\rangle} 
= e^{\lambda s}.
     \end{align}
  \item \label{lem:extinct:3}
    For each $s\geq 0$, we have
    \begin{equation} \label{eq:ont_point_ratio_limit}
      \lim_{t\to \infty} \sup_{x\in E}\Big|\frac{V_{t+s}f(x)}{\langle V_tf,\nu\rangle } - \phi(x)e^{\lambda s} \Big|
      =0.
    \end{equation}
  \end{enumerate}
\end{lem}
% ***** Proof of assertion (1) 
\begin{proof}[Proof of Lemma~\ref{lem:asmptotic_of_Vtf}.(\ref{subVtf_vanish})]
Define function $\psi_0$ by
\[
  \psi_0(x,\lambda) = \psi(x,\lambda)+ \beta(x) \lambda,
  \quad x\in E, \lambda \geq 0,
\]
and operator $\Psi_0$ by
\begin{align}
\Psi_0f(x):=\psi_0(x,f(x)), \quad f\in \mathcal B(E,[0,\infty)).
\end{align}
It is known from \cite[Theorem 2.23]{Li2011Measurevalued} and monotonicity that for each $g \in \mathcal B(E,[0,\infty])$, $(s,x)\mapsto V_sg(x)$ satisfies the following equation
\[
  V_sg + \int_0^s P_{s-u}^\beta \Psi_0V_u g~du = P_s^\beta g,
  \quad s\geq 0.
\]
\begin{iss}[Open]~
  \begin{itemize}
  \item[ZS:]
Note that in the above $g$ is not assumed to be bounded. So the both sides might taking the extended value $+\infty$.
\item[ZS:]
Because of this, it is allowed to write
\begin{align}
V_s g + \int_0^s P_{s-u}^\beta \Psi_0 V_u g~du = P_s^\beta g
\end{align}
but is not allowed to write
\begin{align}
  V_s g  = P_s^\beta g - \int_0^s P_{s-u}^\beta \Psi_0 V_u g~du
\end{align}
since the right side of the latter equation might be of the form $\infty - \infty$.
  \end{itemize}
\end{iss}
In particular, taking $g=V_tf$ we have for each $t,s \geq 0$,
\begin{equation}
  \label{eq:equation_for_Vtf}
  V_{t+s}f + \int_0^s P^\beta_{s-u}\Psi_0V_{t+u}f~du
  =P^\beta_s V_t f.
\end{equation}

According to Assumption \ref{asp:3}.\eqref{subasp:measure_non_presistence} and \eqref{eq:Vtf_is_finite}, for large $t$, we have that $V_tf \in L^+_1(\nu)$.
Therefore, there exists a $\rho > 0$ such that
\begin{align}
  &P_s^\beta V_t f(x) = e^{\lambda s} \phi(x) \left\langle V_tf, \nu \right\rangle \left (1+e^{-\rho s}O(\limsup_{t\to \infty}\sup_{s\geq 1,x\in E})\right),\quad\text{by~\eqref{eq:IU_for_Ptf}} \label{eq:asmptotic_for_PsVtf}
  \\&= O(\limsup_{t\to \infty}\sup_{s\geq 1, x\in E}) \cdot e^{\lambda s} \langle V_tf,\nu \rangle \label{eq:Pv_and_vnu}
  \\&\leq O(\limsup_{t\to \infty} \sup_{s\geq 1, x\in E}) \cdot e^{\lambda s} \langle v_t,\nu\rangle,\quad\text{by~\eqref{eq:Vtf_is_finite}}
  \\&= O(\limsup_{t\to \infty}\sup_{s\geq 1, x\in E}) e^{\lambda s},\quad\text{by~monotonicity}. \label{eq:Pv_convergence_exponentially}
\end{align}
From \eqref{eq:equation_for_Vtf}, we know that $V_{t+s} f \leq P_s^\beta V_t f$, so if we take $s = 1$, we have
\begin{align}
\label{eq:VtPlusOnef_leq_Vtfnu}
V_{t+1}f(x) \leq P_1^\beta V_t f(x) = O(\limsup_{t
  \to \infty}\sup_{x\in E}) \langle V_tf,\nu\rangle,\quad\text{by~\eqref{eq:Pv_and_vnu}} 
\end{align}
and if we take $s = t$, we have
\begin{align}
\label{eq:Vtf_convergence_to_zero}
V_{2t}f(x) 
\leq P_t^\beta V_tf(x)
= e^{\lambda t} O(\limsup_{t\to \infty} \sup_{x\in E})
  = o(\limsup_{t\to \infty} \sup_{x\in E}),\quad\text{by~\eqref{eq:Pv_convergence_exponentially}}.
\end{align}
This implies the desired result.
\end{proof}

% ***** Proof of assertion (3) version B 
\begin{proof}[Proof of Lemma \ref{cor:extinct}.\eqref{lem:extinct_2_1}]

For each $x\in E$, define 
\begin{align}
\label{eq:definition_of_psi'0}
 \psi_0'(x,\lambda):=\frac{\partial}{\partial \lambda}\psi_0(x,\lambda)
    =2\alpha(x)^2\lambda+\int_0^{\infty}\left(1-e^{-r\lambda}\right)rn(x,dr), 
    \quad \lambda \geq 0,
\end{align}
which is a nonnegative and increasing function of $\lambda$. 
Therefore, for each $x\in E$, $\psi_0(x,\lambda)$ is a convex function in $\lambda$, and it holds that
\begin{align}
\psi(x,\lambda) \leq \lambda \psi_0'(x,\lambda),
\quad \lambda \geq 0.
\label{eq:6}
\end{align}
To simplify the notation, define an operator $\Psi'_0$ on $\mathcal B(E,[0,\infty))$ such that
\begin{align}
\Psi'_0 f(x) := \psi_0(x,f(x)), \quad x\in E.
\end{align}
For each $x\in E$, from \eqref{eq:Vtf_convergence_to_zero}, we have $\lim_{t\to \infty} V_tf(x) = 0$, therefore according to \eqref{eq:definition_of_psi'0} we have from monotonicity that 
\begin{align}
\label{eq:psi0'vt_converges_to_0}
\lim_{t\to \infty}\Psi_0' V_tf(x) = 0.
\end{align}
\begin{iss}[Open]~
  \begin{itemize}
  \item[ZS:]
Here is a tiny problem about the above argument. $V_t f(x)$ may not be monotone in $t$. In the old proof, we only considered $v_t(x) = V_t(\infty)(x)$ which is monotone in $t$. 
  \item[ZS:]
It seems that what really should be used here is the continuity of $\psi_0'(x,z)$.
  \end{itemize}
\end{iss}

It can also be verified from \eqref{eq:definition_of_psi'0} and \eqref{eq:Vtf_convergence_to_zero} that
\begin{align}
&\Psi_0'V_tf(x) \leq 2\alpha(x)^2 V_tf(x)+ \left(\int_0^1 r^2~n(x, dr)\right) V_tf(x) + 2 \int_1^\infty r~n(x,dr)
\\&= O(\sup_{x\in E,t\geq 0})V_tf(x)+ O(\sup_{x\in E,t\geq 0})
\\&=  O(\limsup_{t\to \infty} \sup_{x\in E}). \label{eq:psi0'vtx_is_Big_O}
\end{align}
So by dominated convergece theorem, \eqref{eq:psi0'vt_converges_to_0} and \eqref{eq:psi0'vtx_is_Big_O} we have
\begin{align}
\label{eq:Psi0'Vtfnu_is_smallo}
\left\langle \Psi_0' V_tf, \nu \right\rangle = o(\limsup_{t\to \infty}).
\end{align}
Another fact can be derived from \eqref{eq:psi0'vtx_is_Big_O} is that
\begin{align}
&\Psi_0 V_tf(x) 
\leq V_tf(x) \cdot (\Psi'_0 V_tf)(x) ,\quad\text{by~\eqref{eq:6}}
  \\&= V_tf(x)\cdot O(\limsup_{t\to \infty} \sup_{x\in E}),\quad\text{by~\eqref{eq:psi0'vtx_is_Big_O}}.\label{eq:Psi0_Vtf_leq_Vtf}
\end{align}

Now,  integrating the both sides of \eqref{eq:equation_for_Vtf} with respect to measure $\nu$, we get
\begin{equation}
\label{eq:equation_for_nuvt}
    \langle V_{t+s}f, \nu\rangle + e^{\lambda s}\int_0^s e^{-\lambda u}\left\langle \Psi_0 V_{t+u} f,\nu\right\rangle~du
    = e^{\lambda s}\left\langle V_tf,\nu \right\rangle,
\quad t,s \geq 0.
\end{equation}  
\begin{rem}
\label{rem:Vtfnu_is_finite_when_t_is_large}
Note we already know from \eqref{eq:Vtf_convergence_to_zero} that $\left\langle V_tf,\nu \right\rangle$ is finite for $t$ large enough.
\begin{iss}[TODO]~
  \begin{itemize}
  \item[ZS:]
The above remark is not entirely honest. 
Actually, in order to obtain \eqref{eq:Vtf_convergence_to_zero}, we already used the fact that $V_tf\in L^1(\nu)$.
See the line above \eqref{eq:asmptotic_for_PsVtf}.
\item[ZS:]
This remark is also used below.
So maybe a better practice would be isolating the result of this remark as a preliminary Lemma.
  \end{itemize}
\end{iss}
\end{rem}
Therefore, $H(t):=e^{-\lambda t}\left\langle V_tf,\nu \right\rangle<\infty$ for $t$ large enough.
From \eqref{eq:equation_for_nuvt}, for large $t$ and each $s\geq 0$, we have 
\begin{align}
H(t)
&= H(t+s) + e^{-\lambda t}\int_0^{s} e^{-\lambda u} \left\langle\Psi_0 V_{t+u}f,\nu\right\rangle~du
\\&=H(t+s) + \int_0^{s}  \frac{\left\langle\Psi_0 V_{t+u} f,\nu\right\rangle}{ \left\langle V_{t+u} f,\nu \right\rangle } H(t+u)~du   
\\&=H(t+s) + \int_t^{t+s}\frac{\langle\Psi_0 V_uf,\nu\rangle}{\langle V_uf,\nu\rangle}H(u)~du. 
\end{align}
This implies that 
\begin{align}
\label{eq:Ht}
H(t)
=H(t_0) \exp\left\{-\int_{t_0}^t\frac{\langle\Psi_0 V_uf,\nu\rangle}{\langle V_uf,\nu\rangle}du\right\},
\end{align}
for both $t$ and $t_0$ large enough.  

In particular we have, for large $t$,
\begin{align}
\label{eq:the_magic_formula}
\frac{\langle V_tf,\nu\rangle}{\langle V_{t+1}f,\nu\rangle}
=\exp\Big\{-\lambda+\int_{t}^{t+1}\frac{\langle \Psi_0V_sf,\nu\rangle}{\langle V_sf,\nu\rangle}ds\Big\}.
\end{align}
This and \eqref{eq:Psi0_Vtf_leq_Vtf} imply that
\begin{align}
\label{eq:Vtfnu_leq_Vt1fnu}
\langle V_tf,\nu\rangle = O(\limsup_{t\to \infty}) \langle V_{t+1} f,\nu \rangle.
\end{align}
\begin{iss}[open]~
  \begin{itemize}
  \item[ZS:]
    Do we have $V_tf(x)>0$ for any possible $t,f,x$? If not, the fraction $\frac{1}{\left\langle V_tf,\mu \right\rangle}$ is not well-defined.
\item[ZS:]
It seems that we need to show that $\mathbf P_x (\|X_t\| = 0) \not\equiv 1 $.
  \end{itemize}
\end{iss}
Therefore, we can verify that
\begin{align}
\label{eq:Psi0vt_is_integrable}
&\left\langle \Psi_0 V_tf, \nu \right\rangle 
\leq \langle (V_tf) \cdot (\Psi'_0 V_tf), \nu \rangle,\quad\text{by~\eqref{eq:6}}
\\& \leq \langle \Psi'_0 V_tf,\nu\rangle  \cdot \sup_{x\in E} V_tf(x)
\\&= o(\limsup_{t\to \infty}) \cdot O(\limsup_{t\to \infty}) \langle V_{t-1}f,\nu\rangle,\quad\text{by~\eqref{eq:Psi0'Vtfnu_is_smallo} and \eqref{eq:VtPlusOnef_leq_Vtfnu}}
  \\&= o(\limsup_{t\to \infty}) \frac{\langle V_{t-1}f,\nu \rangle}{\langle V_{t}f ,\nu \rangle} \langle V_tf,\nu \rangle 
  \\& = o(\limsup_{t\to \infty})\langle V_tf,\nu \rangle,\quad\text{by~\eqref{eq:Vtfnu_leq_Vt1fnu}}. \label{eq:Psi0Vtfnu_lleq_Vtfnu}
\end{align}
Put this back into \eqref{eq:Ht}, we get
\begin{align}
&\frac{\langle V_tf,\nu\rangle}{\langle V_{t+s}f,\nu\rangle}
                = \exp \left\{-\lambda s+  \int_t^{t+s} \frac{\left\langle \Psi_0 V_u f,\nu \right\rangle}{ \left\langle V_uf, \nu \right\rangle } ~du \right\},\quad\text{by~\eqref{eq:Ht}}
  \\&=\exp\left\{-s\left(\lambda+o(\limsup_{t\to \infty} \sup_{s \geq  0})\right)\right\},\quad\text{by~\eqref{eq:Psi0Vtfnu_lleq_Vtfnu}}
\\ &= e^{-s\lambda} \left( 1+o(\sup_{s \geq 0}\limsup_{t\to \infty}) \right). 
     \label{eq:compare_with_Vtfnu_and_VtPlussfnu}
\end{align}
This implies the desired result.
\end{proof}
% ***** Proof of assertion (3.2) version C
\begin{proof}[Proof of Lemma \ref{cor:extinct}.\eqref{lem:extinct:3}]
From~\eqref{eq:equation_for_Vtf}, for each $t>0$, $s>0$ and $\epsilon \in (0,s)$, we have
\begin{align}
\label{eq:Vf_plus_I_plus_J_equals_PF}
  V_{t+s} f + I_{t,s,\epsilon} + J_{t,s,\epsilon}= P_s^\beta V_tf,
\end{align}
where
\begin{align}
I_{t,s,\epsilon}
&:= \int_0^{s-\epsilon} P_{s-u}^\beta \Psi_0 V_{t+u}f du,
\label{eq:definition_of_Its_epsilon}
\\ J_{t,s,\epsilon}
&:= \int_{s-\epsilon}^s P_{s-u}^\beta \Psi_0 V_{t+u}f du.
\label{eq:definition_of_Jtsepsilon}
\end{align}
Now we can verify that 
\begin{align}
  &\frac{ P_s^\beta V_tf(x)}{\left\langle V_{t+s} f,\nu \right\rangle } - \phi(x)
\\& \begin{multlined}= e^{\lambda s} \phi(x)\frac{  \left\langle V_tf,\nu \right\rangle   }{\left\langle V_{t+s}f,\nu \right\rangle } \left( 1+ e^{-\rho s} O(\limsup_{t\to \infty}\sup_{s\geq 1, x\in E}) \right)- \phi(x),
\\ \text{by~\eqref{eq:asmptotic_for_PsVtf}} \end{multlined}
  \\ &\begin{multlined} = \phi(x) \left( 1+o(\sup_{s \geq 0} \limsup_{t\to \infty} \sup_{x\in E}) \right)  \left( 1+ e^{-\rho s} O(\limsup_{t\to \infty}\sup_{s\geq 1, x\in E}) \right)- \phi(x),
\\ \text{by~\eqref{eq:compare_with_Vtfnu_and_VtPlussfnu}} \end{multlined}
\\&= o(\limsup_{s\to \infty}\limsup_{t\to \infty} \sup_{x\in E}). 
\label{eq:9}
\end{align}

\begin{iss}[Todo]~
\label{iss:PsiVtf_is_L1nu}
  \begin{itemize}
  \item[ZS:]
We need to show that $\Psi_0 V_tf(x) \in L_1^+(\nu)$ for $t$ large enough in order to make the following arguments valid.
\item[ZS:]
OK, I already know how to show this. I will do it in a near future. 
  \end{itemize}
\end{iss}
We can also verify that, for each $\epsilon > 0$,
\begin{align}
&\frac{I_{t,s,\epsilon}(x)}{\left\langle V_{t+s}f,\nu \right\rangle} 
= \frac{1}{\left\langle V_{t+s}f,\nu \right\rangle }  \int_0^{s-\epsilon} P_{s-u}^\beta \Psi_0 V_{t+u}f(x) du,\quad\text{by~\eqref{eq:definition_of_Its_epsilon}} 
\\& \begin{multlined}
= \frac{1}{\left\langle V_{t+s}f,\nu \right\rangle}  \int_0^{s-\epsilon} du \cdot e^{\lambda (s- u)} O(\limsup_{t\to \infty}\sup_{s > \epsilon, u\in (0, s-\epsilon), x\in E}) \left\langle \Psi_0 V_{t+u}f,\nu \right\rangle ,
\\\quad\text{by~Lemma \ref{lem:Pf_and_fnu} and Issue \ref{iss:PsiVtf_is_L1nu}} 
\end{multlined}
\\& = \frac{1}{\left\langle V_{t+s}f,\nu \right\rangle} O(\limsup_{t\to \infty}\sup_{s > \epsilon, x\in E}) \int_0^{s-\epsilon} e^{\lambda (s- u)}  \left\langle \Psi_0 V_{t+u}f,\nu \right\rangle ~du
\\& \leq \frac{1}{\left\langle V_{t+s}f,\nu \right\rangle} O(\limsup_{t\to \infty}\sup_{s > \epsilon, x\in E}) \int_0^{s} e^{\lambda (s- u)}  \left\langle \Psi_0 V_{t+u}f,\nu \right\rangle ~ du 
\\& \begin{multlined}
= \frac{1}{\left\langle V_{t+s}f,\nu \right\rangle} O(\limsup_{t\to \infty}\sup_{s > \epsilon, x\in E}) \left( e^{\lambda s} \left\langle V_tf,\nu \right\rangle - \left\langle V_{t+s}f,\nu \right\rangle \right),\\\quad\text{by~\eqref{eq:equation_for_nuvt} and Remark \ref{rem:Vtfnu_is_finite_when_t_is_large}}
\end{multlined}
\\& =O(\limsup_{t\to \infty}\sup_{s > \epsilon, x\in E}) \left( e^{\lambda s}\frac{ \left\langle V_tf,\nu \right\rangle }{\left \langle V_{t+s}f,\nu \right\rangle}- 1 \right) 
  \\& = O(\limsup_{t\to \infty}\sup_{s > \epsilon,x\in E}) o(\sup_{s>0}\limsup_{t\to \infty} \sup_{x\in E}),
\quad\text{by~\eqref{eq:compare_with_Vtfnu_and_VtPlussfnu},}
\\&= o(\sup_{s>\epsilon}\limsup_{t\to \infty} \sup_{x\in E}) .
\end{align}
Therefore, we have
\begin{align}
  \frac{I_{t,s,\epsilon}(x)}{\left\langle V_{t+s}f,\nu \right\rangle} 
= o(\sup_{s, \epsilon >0} \limsup_{t\to \infty}\sup_{x\in E}).
  \label{eq:10}
\end{align}
On the other hand, we have
\begin{align}
& \frac{J_{t,s,\epsilon}(x)}{\left\langle V_{t+s}f,\nu \right\rangle} 
= \frac{1}{\left\langle V_{t+s}f,\nu \right\rangle}\int_{s-\epsilon}^s P_{s-u}^\beta \Psi_0 V_{t+u}f (x)~du ,\quad\text{by~\eqref{eq:definition_of_Jtsepsilon}}
\\&= \frac{1}{\left\langle V_{t+s}f,\nu \right\rangle} \int_0^\epsilon P_u^\beta \Psi_0 V_{t+s-u}f(x)~du
  \\ & \leq  \frac{1}{\left\langle V_{t+s}f,\nu \right\rangle} \int_0^\epsilon \left(P_u^\beta \mathbf 1_E\right)(x) \cdot \left( \sup_{y\in E} \Psi_0 V_{t+s - u} f(y) \right) du
  \\ & \leq  \frac{1}{\left\langle V_{t+s}f,\nu \right\rangle}\cdot \left( \sup_{y \in E, u \in (0, \epsilon)}\Psi_0 V_{t+s - u} f(y) \right) \cdot \int_0^\epsilon e^{\|\beta\|u}du 
\end{align}
From \eqref{eq:Psi0_Vtf_leq_Vtf}, we have
\begin{align}
  \sup_{u \in (0,\epsilon),y\in E}\Psi_0 V_{t+s-u}f(y) = \left(\sup_{u \in (0,\epsilon), y\in E} V_{t+s-u}f(y) \right)\cdot O(\sup_{\epsilon > 0}\limsup_{t+s\to \infty} ).
\end{align}

Therefore, we have
\begin{align}
&\frac{J_{t,s,\epsilon}(x)}{\left \langle V_{t+s}f,\nu\right\rangle}
  \\ & =\left(\sup_{u\in (0,\epsilon), y \in E}\frac{V_{t+s-u}f(y)}{\left\langle V_{t+s}f,\nu \right\rangle}\right)\cdot O(\sup_{\epsilon>0}\limsup_{t+s\to \infty} \sup_{x\in E})\cdot o(\limsup_{\epsilon \to 0}\sup_{t,s>0,x\in E})
  \\ & = \left(\sup_{u\in (0,\epsilon), y \in E}\frac{V_{t+s-u}f(y)}{\left\langle V_{t+s}f,\nu \right\rangle}\right)\cdot o(\limsup_{\epsilon \to 0}\limsup_{t+s\to \infty}\sup_{x\in E})
  \\ & \begin{multlined} = \left(\sup_{u\in (0,\epsilon)}\frac{\left\langle V_{t+s - u - 1} f, \nu \right\rangle}{\left\langle V_{t+s}f,\nu \right\rangle} \right) \cdot O(\sup_{\epsilon >  0}\limsup_{t+s \to \infty}\sup_{x\in E}) \cdot o(\limsup_{\epsilon \to 0}\limsup_{t+s\to \infty}\sup_{x\in E}),
\\\quad\text{by~\eqref{eq:VtPlusOnef_leq_Vtfnu}} \end{multlined}
  \\ & \begin{multlined}
= \left(\sup_{u\in (0,\epsilon)}e^{-(1+u) \lambda} \right) O(\sup_{\epsilon > 0} \limsup_{t+s\to \infty}\sup_{x\in E}) o(\limsup_{\epsilon \to 0}\limsup_{t+s\to \infty}\sup_{x\in E}),
\\ \quad\text{by~\eqref{eq:compare_with_Vtfnu_and_VtPlussfnu}}
\end{multlined}
\\ & = o(\limsup_{\epsilon \to 0}\limsup_{t+s\to \infty}\sup_{x\in E}). \label{eq:8}
\end{align}

To summarize, we have
\begin{align}
& \frac{V_{t+s}f(x)}{\left\langle V_{t+s}f,\nu \right\rangle} - \phi(x) 
  \\&= \left(\frac{P_s^\beta V_t f (x)}{\left\langle V_{t+s} f,\nu \right\rangle} - \phi(x)\right) - \frac{I_{t,s,\epsilon}(x)}{\left\langle V_{t+s} f,\nu \right\rangle} -\frac{ J_{t,s,\epsilon}(x)}{\left\langle V_{t+s}f,\nu \right\rangle},\quad\text{by~\eqref{eq:Vf_plus_I_plus_J_equals_PF}}
  \\& \begin{multlined} = o(\limsup_{s\to \infty} \limsup_{t\to \infty} \sup_{x\in E, \epsilon > 0}) + o(\sup_{s,\epsilon>0} \limsup_{t\to \infty} \sup_{x\in E}) + o(\limsup_{\epsilon \to 0} \limsup_{t+s \to \infty} \sup_{x\in E}),
\\ \text{by~\eqref{eq:9}, \eqref{eq:10} and \eqref{eq:8}} \end{multlined}
\\& = o(\limsup_{\epsilon \to 0} \limsup_{s\to \infty}\limsup_{t\to \infty} \sup_{x\in E}). 
\end{align}
Therefore, we have
\begin{align}
\frac{V_{t} f(x)}{\left\langle V_t f,\nu \right\rangle} - \phi(x) 
= o(\limsup_{t\to \infty} \sup_{x\in E}).
\end{align}
\begin{iss}[TODO]~
  \begin{itemize}
  \item[ZS:]
It seems that I forget to discuss $V_{t+s}f(x)$.
It won't be a problem. I will do it shortly.
  \end{itemize}
\end{iss} 
\end{proof}

Taking $f \equiv \infty$ in the above Lemma, we get
\begin{cor}
\label{cor:extinct}
    Suppose that Assumptions \ref{asp:1}, \ref{asp:IU} and \ref{asp:3} hold.
    Then we have the followings:
\begin{enumerate}
\item 
\label{sub:extinct_2}
    For each $\mu \in \mathcal M(E)$, we have
    \[
      \lim_{t\rightarrow\infty}\langle v_t,\mu\rangle=0.
    \]
\item 
\label{sub:extinct_2_1}
    For each $s\geq 0$, we have
     \begin{align}
       \lim_{t\to \infty} \frac{\left\langle v_{t+s},\nu\right\rangle}{\left\langle v_t,\nu\right\rangle} 
= e^{\lambda s}.
     \end{align}
\item 
\label{sub:extinct_3}
    For each $s\geq 0$, we have
\begin{equation} 
    \lim_{t\to \infty} \sup_{x\in E}\Big|\frac{v_{t+s}(x)}{\langle v_t,\nu\rangle } - \phi(x)e^{\lambda s} \Big|
    =0.
\end{equation}
\end{enumerate}
\end{cor}
% ***** Proof of assertion (1) and (2)
% ****** Concept of cumulant semigroup
\begin{comment}
\begin{proof}[Proof of Lemma~\ref{cor:extinct}.\eqref{lem:extinct_2}]
Recall the operator $(V_t)_{t\geq 0}$ defined in~\eqref{eq:_def_of_vtf}.
Using the monotone convergence theorem we can define
\begin{align}
  V_tf(x)
  := \lim_{n\to \infty}V_t(f\wedge n)(x)
  = - \log \mathbb P_x[e^{-\left\langle f, X_t \right\rangle}] 
  \leq - \log \mathbb P_x(X_t \equiv 0)
  < \infty,
\end{align}
for each $f\in \mathcal B(E, [0,\infty]),t> 0$ and $x\in E$.
We call $(V_t)_{t> 0}$ the cumulant semigroup of the superprocess $X$, because it holds that $V_tV_sf(x) = V_{t+s}f(x)$ for all $f\in \mathcal B(E,[0,\infty])$, $t,s > 0$ and $x\in E$. 
In particular, we have $V_t(\infty) (x)= v_t(x)$ holds for each $t,x>0$ and $x\in E$.
% ****** An equation about the cumulant semigroup  
Define function $\psi_0$ by
\[
  \psi_0(x,\lambda) = \psi(x,\lambda)+ \beta(x) \lambda,
  \quad x\in E, \lambda \geq 0,
\]
and operator $\Psi_0$ by
\begin{align}
\Psi_0f(x):=\psi_0(x,f(x)), \quad f\in \mathcal B(E,[0,\infty)).
\end{align}
It is known from \cite[Theorem 2.23]{Li2011Measurevalued} and monotonicity that for each $f \in \mathcal B(E,[0,\infty])$, $(s,x)\mapsto V_sf(x)$ satisfies the following equation
\[
  V_sf + \int_0^s P_{s-u}^\beta \Psi_0V_u f~du = P_s^\beta f,
  \quad s\geq 0.
\]
In particular, taking $f=v_t$ we have for each $t,s>0$,
\begin{equation}
  \label{eq:equation_for_vt}
  % v_{t+s}(x) + \int_0^sP^\beta_{s-u}\Big(\psi_0\big(\cdot, v_{t+u}(\cdot)\big)\Big)(x)~du
  v_{t+s} + \int_0^s P^\beta_{s-u}\Psi_0v_{t+u}~du
  =P^\beta_s v_t,
  \quad t,s > 0.
\end{equation}

According to Assumption \ref{asp:3}.\eqref{subasp:measure_non_presistence}, for large $t$, we have that $v_t \in L^+_1(\nu)$.
Therefore, there exists a $\rho > 0$ such that
\begin{align}
  &P_s^\beta v_t(x) = \int p^\beta_s(x,y) v_t(y) m(dy)
  \\ &= e^{\lambda s} \phi(x) \left\langle v_t, \nu \right\rangle \left (1+e^{-\rho s}O(\limsup_{t\to \infty}\sup_{s\geq 1,x\in E})\right),\quad\text{by~\eqref{eq:IU_for_Ptf}} \label{eq:asmptotic_for_vs+tx}
  \\&= O(\limsup_{t\to \infty}\sup_{s\geq 1, x\in E}) \cdot e^{\lambda s} \langle v_t,\nu \rangle \label{eq:Pv_and_vnu}
  \\&= O(\limsup_{t\to \infty}\sup_{s\geq 1, x\in E}) e^{\lambda s},\quad\text{by~monotonicity}. \label{eq:Pv_convergence_exponentially}
\end{align}
From \eqref{eq:equation_for_vt}, we know that $v_{t+s} \leq P_s^\beta v_t$, so if we take $s = 1$, we have
\begin{align}
\label{eq:v1+t_is_controlled_by_vtnu}
v_{1+t}(x) \leq P_1^\beta v_t(x) = O(\limsup_{t
  \to \infty}\sup_{x\in E}) \langle v_t,\nu\rangle,\quad\text{by~\eqref{eq:Pv_and_vnu}} 
\end{align}
and if we take $s = t$, we have
\begin{align}
\label{eq:vt_convergence_to_zero}
v_{2t}(x) 
\leq P_t^\beta v_t(x)
= e^{\lambda t} O(\limsup_{t\to \infty} \sup_{x\in E})
  = o(\limsup_{t\to \infty} \sup_{x\in E}),\quad\text{by~\eqref{eq:Pv_convergence_exponentially}}.
\end{align}
This implies the desired result.
\end{proof}

% ***** Proof of assertion (3) version B 
\begin{proof}[Proof of Lemma \ref{cor:extinct}.\eqref{lem:extinct_2_1}]

For each $x\in E$, define 
\begin{align}
\label{eq:definition_of_psi'0}
 \psi_0'(x,\lambda):=\frac{\partial}{\partial \lambda}\psi_0(x,\lambda)
    =2\alpha(x)^2\lambda+\int_0^{\infty}\left(1-e^{-r\lambda}\right)rn(x,dr), 
    \quad \lambda \geq 0,
\end{align}
which is a nonnegative and increasing function of $\lambda$. 
Note that for each $x\in E$, $\psi_0(x,\lambda)$ is a convex function in $\lambda$, and therefore
\begin{align}
\psi(x,\lambda) \leq \lambda \psi_0'(x,\lambda),
\quad \lambda \geq 0.
\label{eq:6}
\end{align}
To simplify the notation, define an operator $\Psi'_0$ on $\mathcal B(E,[0,\infty))$ such that
\begin{align}
\Psi'_0 f(x) := \psi_0(x,f(x)), \quad x\in E.
\end{align}
For each $x\in E$, from \eqref{eq:vt_convergence_to_zero}, we have $\lim_{s\to \infty} v_s(x) = 0$, therefore according to \eqref{eq:definition_of_psi'0} we have from monotonicity that 
\begin{align}
\label{eq:psi0'vt_converges_to_0}
\lim_{s\to \infty}\Psi_0' v_s(x) = 0.
\end{align}

It can also be verified from \eqref{eq:definition_of_psi'0} and \eqref{eq:vt_convergence_to_zero} that
\begin{align}
&\Psi_0'v_s(x) \leq 2\alpha(x)^2 v_s(x)+ \left(\int_0^1 r^2~n(x, dr)\right) v_s(x) + 2 \int_1^\infty r~n(x,dr)
\\&= O(\sup_{x\in E,s>0})v_s(x)+ O(\sup_{x\in E,s>0})
\\&=  O(\limsup_{s\to \infty} \sup_{x\in E}).
\end{align}
So by dominated convergece theorem, \eqref{eq:psi0'vt_converges_to_0} and \eqref{eq:psi0'vtx_is_Big_O} we have
\begin{align}
\label{eq:psi0'vsnu_is_small_o}
\left\langle \Psi_0' v_s, \nu \right\rangle = o(\limsup_{s\to \infty}).
\end{align}
Another fact can be derived from \eqref{eq:psi0'vtx_is_Big_O} is that
\begin{align}
&\Psi_0 v_t(x) 
\leq v_t(x) \cdot (\Psi'_0 v_t)(x) ,\quad\text{by~\eqref{eq:6}}
  \\&= v_t(x) O(\limsup_{t\to \infty} \sup_{x\in E}),\quad\text{by~\eqref{eq:psi0'vtx_is_Big_O}}.  
\end{align}

Now,  integrating the both sides of \eqref{eq:equation_for_vt} with respect to measure $\nu$, we get
\begin{equation}
\langle v_{t+s}, \nu\rangle + e^{\lambda s}\int_0^s e^{-\lambda u}\left\langle \Psi_0 v_{t+u},\nu\right\rangle~du
= e^{\lambda s}\left\langle v_t,\nu \right\rangle,
\quad t,s > 0.
\end{equation}  
Note we already know from \eqref{eq:vt_convergence_to_zero} that $\left\langle v_t,\nu \right\rangle$ is finite for $t$ large enough.
Therefore, $g(t):=e^{-\lambda t}\left\langle v_t,\nu \right\rangle<\infty$ for $t$ large enough.
From \eqref{eq:equation_for_nuvt}, for large $t$ and each $s\geq 0$, we have 
\begin{align}
g(t+s)
&=g(t)-e^{-\lambda t}\int_0^{s} e^{-\lambda u} \left\langle\Psi_0 v_{t+u},\nu\right\rangle~du
\\&=g(t)-\int_0^{s}  \frac{\left\langle\Psi_0 v_{t+u},\nu\right\rangle}{ \left\langle v_{t+u},\nu \right\rangle } g(t+u)~du   
\\&=g(t)-\int_t^{t+s}\frac{\langle\Psi_0 v_u,\nu\rangle}{\langle v_u,\nu\rangle}g(u)~du. 
\end{align}
This implies that 
\begin{align}
\label{eq:ratio_of_nuvt_and_nuvtplus1}
g(t)
=g(t_0)\exp\left\{-\int_{t_0}^t\frac{\langle\Psi_0 v_u,\nu\rangle}{\langle v_u,\nu\rangle}du\right\},
\end{align}
for both $t$ and $t_0$ large enough.  
% ******* This expression says that vt can be controlled by v_{t+1} 
In particular we have, for large $t$,
\begin{align}
\label{eq:the_magic_formula}
\frac{\langle v_t,\nu\rangle}{\langle v_{t+1},\nu\rangle}
=\exp\Big\{-\lambda+\int_{t}^{t+1}\frac{\langle \Psi_0v_s,\nu\rangle}{\langle v_s,\nu\rangle}ds\Big\}.
\end{align}
This and \eqref{eq:7} imply that
\begin{align}
\langle v_t,\nu\rangle = O(\limsup_{t\to \infty}) \langle v_{t+1},\nu \rangle.
\end{align}
Therefore, from \eqref{eq:6}, \eqref{eq:psi0'vsnu_is_small_o}, \eqref{eq:v1+t_is_controlled_by_vtnu} and above, we have
\begin{align}
\label{eq:Psi0vt_is_integrable}
&\left\langle \Psi_0 v_t, \nu \right\rangle 
\leq \langle v_t \Psi'_0 v_t, \nu \rangle
\leq \langle \Psi'_0 v_t,\nu\rangle  \cdot \sup_{x\in E} v_t(x)
= o(\limsup_{t\to \infty}) \cdot O(\limsup_{t\to \infty}) \langle v_{t-1},\nu\rangle
\\&= o(\limsup_{t\to \infty}) \frac{\langle v_{t-1},\nu \rangle}{\langle v_{t} ,\nu \rangle} \langle v_t,\nu \rangle = o(\limsup_{t\to \infty})\langle v_t,\nu \rangle.
\end{align}
Put this back into \eqref{eq:ratio_of_nuvt_and_nuvtplus1}, we get
\begin{align}
&\frac{\langle v_t,\nu\rangle}{\langle v_{t+s},\nu\rangle}
=\exp\left\{-s\left(\lambda+o(\limsup_{t\to \infty} \sup_{s > 0})\right)\right\}
\\ &= e^{-s\lambda} \left( 1+o(\sup_{s>0}\limsup_{t\to \infty}) \right). 
\end{align}
This implies the desired result.
\end{proof}
% ***** Proof of assertion (3.2) version C
\begin{proof}[Proof of Lemma \ref{cor:extinct}.\eqref{lem:extinct:3} ]
From~\eqref{eq:equation_for_vt}, for each $t>0$, $s>0$ and $\epsilon \in (0,s)$, we have
\begin{align}
  v_{t+s} + I_{t,s,\epsilon} + J_{t,s,\epsilon}= P_s^\beta v_t,
\end{align}
where
\begin{align}
I_{t,s,\epsilon}
&:= \int_0^{s-\epsilon} P_{s-u}^\beta \Psi_0 v_{t+u} du,
\\ J_{t,s,\epsilon}
&:= \int_{s-\epsilon}^s P_{s-u}^\beta \Psi_0 v_{t+u} du.
\end{align}
According to \eqref{eq:asmptotic_for_vs+tx} and \eqref{eq:compare_with_vtnu_and_vt+snu}, we have 
\begin{align}
&\frac{ P_s^\beta v_t}{\left\langle v_{t+s},\nu \right\rangle } - \phi(x)
= e^{\lambda s} \phi(x)\frac{  \left\langle v_t,\nu \right\rangle   }{\left\langle v_{t+s},\nu \right\rangle } \left( 1+ e^{-\rho s} O(\limsup_{t\to \infty}\sup_{s\geq 1, x\in E}) \right)- \phi(x)
\\&=  \phi(x) \left( 1+o(\sup_{s>0} \limsup_{t\to \infty} \sup_{x\in E}) \right)  \left( 1+ e^{-\rho s} O(\limsup_{t\to \infty}\sup_{s\geq 1, x\in E}) \right)- \phi(x)
\\&= o(\limsup_{s\to \infty}\limsup_{t\to \infty} \sup_{x\in E}). 
\end{align}
\begin{iss}[open]~
  \begin{itemize}
  \item[ZS:]
I need to check if $\Psi_0 v_t(x) \in L_1^+(\nu)$ here.
  \end{itemize}
\end{iss}
We can also verify that, for each $\epsilon > 0$,
\begin{align}
  &\frac{I_{t,s,\epsilon}(x)}{\left\langle v_{t+s},\nu \right\rangle} 
    = \frac{1}{\left\langle v_{t+s},\nu \right\rangle }  \int_0^{s-\epsilon} P_{s-u}^\beta \Psi_0 v_{t+u}(x) du,\quad\text{by~\eqref{eq:definition_of_Its_epsilon}} 
  \\& = \frac{1}{\left\langle v_{t+s},\nu \right\rangle}  \int_0^{s-\epsilon} du \cdot e^{\lambda (s- u)} O(\sup_{s > \epsilon, x\in E, t>0, u\in (0, s-\epsilon)}) \left\langle \Psi_0 v_{t+u},\nu \right\rangle ,\quad\text{by~Lemma \ref{lem:Pf_and_fnu}} 
  \\& = \frac{1}{\left\langle v_{t+s},\nu \right\rangle} O(\sup_{s > \epsilon, x\in E, t>0}) \int_0^{s-\epsilon} du \cdot e^{\lambda (s- u)}  \left\langle \Psi_0 v_{t+u},\nu \right\rangle  
\\& \leq \frac{1}{\left\langle v_{t+s},\nu \right\rangle} O(\sup_{s > \epsilon, x\in E, t>0}) \int_0^{s} du \cdot e^{\lambda (s- u)}  \left\langle \Psi_0 v_{t+u},\nu \right\rangle  
\\& = \frac{1}{\left\langle v_{t+s},\nu \right\rangle} O(\sup_{s\geq \epsilon, t\geq 0, x\in E}) \left( e^{\lambda s} \left\langle v_t,\nu \right\rangle - \left\langle v_{t+s},\nu \right\rangle \right),\quad\text{by~\eqref{eq:equation_for_nuvt}}
\\& =O(\sup_{s\geq \epsilon, t\geq 0, x\in E}) \left( e^{\lambda s}\frac{ \left\langle v_t,\nu \right\rangle }{\left \langle v_{t+s},\nu \right\rangle}- 1 \right) 
\\& = O(\sup_{s\geq \epsilon, t\geq 0, x\in E}) o(\sup_{s>0}\limsup_{t\to \infty} \sup_{x\in E}) ,\quad\text{by~\eqref{eq:compare_with_vtnu_and_vt+snu}}
 \\&= o(\sup_{s>0}\limsup_{t\to \infty} \sup_{x\in E}) .
\end{align}
Therefore, we have
\begin{align}
  \frac{I_{t,s,\epsilon}(x)}{\left\langle v_{t+s},\nu \right\rangle} 
= o(\sup_{s, \epsilon >0} \limsup_{t\to \infty}\sup_{x\in E}).
\end{align}
On the other hand, we have
\begin{align}
& \frac{J_{t,s,\epsilon}(x)}{\left\langle v_{t+s},\nu \right\rangle} 
= \frac{1}{\left\langle v_{t+s},\nu \right\rangle}\int_{s-\epsilon}^s P_{s-u}^\beta \Psi_0 v_{t+u} (x)du ,\quad\text{by~\eqref{eq:definition_of_Jtsepsilon}}
\\&= \frac{1}{\left\langle v_{t+s},\nu \right\rangle} \int_0^\epsilon P_u^\beta \Psi_0 v_{t+s-u}(x) du
  \\ & \leq  \frac{1}{\left\langle v_{t+s},\nu \right\rangle} \int_0^\epsilon P_u^\beta \Psi_0 v_{t+s-\epsilon}(x) du,\quad\text{by~monotonicity}
  \\ & \leq  \frac{\sup_{x\in E} \Psi_0 v_{t+s-\epsilon}(x)}{\left\langle v_{t+s},\nu \right\rangle} \int_0^\epsilon \left(P_u^\beta \mathbf 1_E\right)(x) du
  \\ & = O(\limsup_{t+s-\epsilon\to \infty} \sup_{x\in E})  \frac{\sup_{x\in E}v_{t+s-\epsilon}(x)}{\left\langle v_{t+s},\nu \right\rangle} o(\limsup_{\epsilon \to 0}\sup_{t,s>0,x\in E}),\quad\text{by~\eqref{eq:7}}
  \\ & = \frac{\sup_{x\in E}v_{t+s-\epsilon}(x)}{\left\langle v_{t+s},\nu \right\rangle} o(\limsup_{\epsilon \to 0}\limsup_{t+s\to \infty}\sup_{x\in E})
  \\ & = \frac{\left\langle v_{t+s - \epsilon - 1}, \nu \right\rangle}{\left\langle v_{t+s},\nu \right\rangle} O(\sup_{t+s \geq \epsilon + 1, x\in E}) o(\limsup_{\epsilon \to 0}\limsup_{t+s\to \infty}\sup_{x\in E}),\quad\text{by~\eqref{eq:v1+t_is_controlled_by_vtnu}}
  \\ & = \frac{\left\langle v_{t+s - \epsilon - 1}, \nu \right\rangle}{\left\langle v_{t+s},\nu \right\rangle} o(\limsup_{\epsilon \to 0}\limsup_{t+s\to \infty}\sup_{x\in E})
  \\ & = e^{-(1+\epsilon) \lambda} O(\sup_{\epsilon > 0} \limsup_{t+s - \epsilon  - 1 \to \infty}\sup_{x\in E}) o(\limsup_{\epsilon \to 0}\limsup_{t+s\to \infty}\sup_{x\in E}),\quad\text{by~\eqref{eq:compare_with_vtnu_and_vt+snu}}
  \\ & = o(\limsup_{\epsilon \to 0}\limsup_{t+s\to \infty}\sup_{x\in E}).
\end{align}

To summarize, we have
\begin{align}
  & \frac{v_{t+s}(x)}{\left\langle v_{t+s},\nu \right\rangle} - \phi(x) 
    = \frac{P_s^\beta v_t(x)}{\left\langle v_{t+s},\nu \right\rangle} - \phi(x) 
    -\frac{I_{t,s,\epsilon}(x)}{\left\langle v_{t+s},\nu \right\rangle} -\frac{ J_{t,s,\epsilon}(x)}{\left\langle v_{t+s},\nu \right\rangle}
  \\& = o(\limsup_{s\to \infty} \limsup_{t\to \infty} \sup_{x\in E, \epsilon > 0}) + o(\sup_{s,\epsilon>0} \limsup_{t\to \infty} \sup_{x\in E}) + o(\limsup_{\epsilon \to 0} \limsup_{t+s \to \infty} \sup_{x\in E}),
\\&\qquad\qquad\text{by~\eqref{eq:9},\eqref{eq:10} and \eqref{eq:8}}
\\& = o(\limsup_{\epsilon \to 0} \limsup_{s\to \infty}\limsup_{t\to \infty} \sup_{x\in E}). 
\end{align}
Therefore, we have
\begin{align}
\frac{v_{t}(x)}{\left\langle v_{t},\nu \right\rangle} - \phi(x) 
= o(\limsup_{t\to \infty} \sup_{x\in E}).
\end{align} 

\end{proof}

\end{comment}
% ***** Lemma about the asymptotic behavior of vtf

% * end
\begin{iss}[Todo]
  \begin{itemize}
  \item[ZS:]
I'm now revising the rest of the paper.
  \end{itemize}
\end{iss}

\begin{cor}\label{cor:general_rate}
	For each $\mu\in\mathcal M(E)\setminus\{0\}$, $f\in\mathcal B_b(E,[0,\infty])$ and $s>0$, it holds that
  \begin{equation}\label{eq:ratio_limits}
\lim_{t\rightarrow\infty}\frac{\langle V_{t+s}f,\mu\rangle }{\langle V_{t}f,\nu\rangle }
=\langle \phi,\mu\rangle e^{\lambda s}.
  \end{equation}
\end{cor}

\begin{lem}\label{eq:ratio_limits_2}
	For any $f\in\mathcal B_b(E,[0,\infty))$, let $V_tf(x)$ is the solution to \eqref{eq: diff equ}.  Then there is a nonnegative function $G(f)$ such that $G(v_t)=t$ and such that for any $\mu\in\mathcal M(E)\setminus\{0\}$,
  \begin{equation}\label{eq:app_of_G}
    \lim_{t\rightarrow\infty}\frac{\langle V_tf,\mu\rangle }{\langle v_t,\mu\rangle }=e^{\lambda G(f)}.
  \end{equation}
\end{lem}

\begin{proof}
  \begin{iss}[Open]~
    \begin{itemize}
    \item[ZS:]
There is a lot of fraction in this proof.
We want to check that each denominators are not zero.
    \end{itemize}
  \end{iss}
	Thanks to Corollaray \ref{cor:general_rate} we only need to prove
  \begin{equation}\label{eq:spec_ratio}
    \lim_{t\rightarrow\infty}\frac{\langle V_tf,\nu\rangle }{\langle v_t,\nu\rangle }=e^{\lambda G(f)}.
  \end{equation}
  According to \eqref{eq:equation_for_nuvt}, we have $\langle v_t,\nu\rangle$ is a strictly decreasing continuous function of $t$, since
  \begin{align}\label{eq:monotone}
    \left\langle v_{t+s},\nu \right\rangle 
\leq e^{\lambda s} \left\langle v_{t},\nu \right\rangle
< \left\langle v_t,\nu \right\rangle.  
\end{align}
Moreover, Lemma \ref{lem:asmptotic_of_Vtf}.\eqref{subVtf_vanish} says that
  \[
\lim_{t\rightarrow\infty}\langle v_t,\nu\rangle =0,
  \]
and from the right continuity of the path of the superprocess, we have
\begin{align}
  \lim_{t\to 0+} \left\langle v_t,\nu \right\rangle 
  = - \log \lim_{t\to 0+} \mathbb P_\nu\left( \|X_t\| = 0 \right)
= +\infty.
\end{align}
Therefore, $t \mapsto \left\langle v_t,\nu \right\rangle$ is an 1-1 map from $(0,\infty)$ to $(0,\infty)$. 
Denote by $\theta \mapsto B(\theta)$ the inverse of this map.
Then, we always have
\begin{align}
  \left\langle v_{B(\theta)},\nu \right\rangle 
  = \theta,
\quad \theta \in (0,\infty);
\qquad 
  B\left( \left\langle v_t,\nu \right\rangle \right) 
= t
\quad t\in (0,\infty).
\end{align}

\begin{lem}
  Suppose that $f,g\in L^1(\nu)$ satisfy that $\left\langle f,\nu \right\rangle = \left\langle g,\nu \right\rangle$. 
Then, for each $t\geq 0$, we have $ \left\langle V_tf,\nu \right\rangle = \left\langle V_tg,\nu \right\rangle $.
\end{lem}
\begin{proof}
  Write $w_t =\left\langle V_tf,\nu \right\rangle - \left\langle V_t g, \nu \right\rangle $. 
  Then from \eqref{eq:equation_for_nuvt} we have
  \begin{align}
    \langle V_tf,\nu\rangle + \int_0^te^{\lambda(t-s)}\langle \Psi_0V_sf,\nu\rangle ds
=e^{\lambda t}\langle f,\nu\rangle,
  \end{align}
and
  \begin{align}
    \langle V_tg,\nu\rangle + \int_0^te^{\lambda(t-s)}\langle \Psi_0V_sg,\nu\rangle ds
    =e^{\lambda t}\langle g,\nu\rangle .
  \end{align}
Therefore, 
\begin{align}
  w_t \leq \int_0^t e^{\lambda (t-s)} \left | \left\langle \Psi_0 V_s f - \Psi_0 V_s g, \nu \right\rangle \right| ~ds 
\end{align}
.......
\begin{iss}[Open]~
  \begin{itemize}
  \item[ZS:]
It seems that this lemma is not right.
\item[ZS:]
It might be reasonable to change the conclusion to 
\begin{align}
  \left\langle V_tf,\nu \right\rangle = \left\langle V_t g,\nu \right\rangle \left( 1+o(\limsup_{t\to \infty}) \right)
\end{align}
\item[ZS:]
  After a closer look, the above approximation is also wrong. One shouldn't expect that $ \left\langle V_tf,\nu \right\rangle$ and $\left\langle V_t g,\nu \right\rangle$ are asymptotic equivalent.
  \end{itemize}
\end{iss}
\end{proof}
  Now from the fact that 
  \begin{align}
    \left\langle f, \nu \right\rangle
    = \left\langle v_{B(\langle  f,\nu\rangle)},\nu \right\rangle;
\quad 
    V_t v_{B(\langle f,\nu \rangle)} = v_{t+ B(\langle f,\nu\rangle)};
  \end{align}
  and the above Lemma we have that
  \begin{align}
\label{eq:Vfnu_and_vnu}
    \left\langle V_tf,\nu \right\rangle 
    = \left\langle v_{t+B(\left\langle f,\nu \right\rangle)}, \nu \right\rangle,
\quad t\geq 0.
  \end{align}
As a result, for any $\mu\in\mathcal M(E)\setminus\{0\}$,

\begin{align}
  &\lim_{t\to \infty}\frac{\left\langle V_tf, \mu \right\rangle}{\left\langle v_t,\mu \right\rangle}
  = \lim_{t\to \infty} \frac{\left\langle V_tf,\mu \right\rangle}{\left\langle V_tf,\nu \right\rangle} \cdot \frac{\left\langle V_tf,\nu \right\rangle}{\left\langle v_t,\nu \right\rangle} \cdot \frac{\left\langle v_t,\nu \right\rangle}{\left\langle v_t,\mu \right\rangle}
  \\&= \left\langle \phi,\mu \right\rangle \cdot \left(\lim_{t\to \infty}\frac{\left\langle v_{t+B(\left\langle f,\nu \right\rangle)}, \nu \right\rangle}{ \left\langle v_t,\nu \right\rangle}\right) \cdot \left\langle \phi,\mu \right\rangle^{-1},\quad\text{by~Corollaray \ref{cor:general_rate} and \eqref{eq:Vfnu_and_vnu}}
  \\& = e^{\lambda B(\left\langle f, \nu \right\rangle)},\quad\text{by~Corollaray \ref{cor:extinct}.\eqref{subextinct_2_1}}
\end{align}
\end{proof}

\subsection{Proof of main theorems}
\begin{proof}[{\bf Proof of Theorem \ref{thm:distribution_of_zeta}}]
	For any finite measure $\mu\in \mathcal M(E)$,
  \[
    \mathbb P_\mu(\zeta=\infty)=\lim_{t\rightarrow\infty}\mathbb P_\mu(\zeta>t)=1-e^{-\lim_{t\rightarrow\infty}\langle v_t,\mu\rangle }=0,
  \]
	follows from Lemma\ref{cor:extinct}.
	Meanwhile, thanks to \eqref{ext con}, for any $\mu,\widetilde\mu\in \mathcal M(E)\setminus\{0\}$ and $s\in\mathbb R$,
  \[
    \lim_{t\rightarrow\infty}\frac{\mathbb P_{\mu}(\zeta>t+s)}{\mathbb P_{\widetilde\mu}(\zeta>t)}=\lim_{t\rightarrow\infty}\frac{1-e^{-\langle v(t+s,\cdot),\mu\rangle }}{1-e^{-\langle v_t,\widetilde\mu\rangle }}
    =\lim_{t\rightarrow\infty}\frac{\langle v(t+s,\cdot),\mu\rangle }{\langle v_t,\widetilde\mu\rangle }=\frac{\langle \phi,\mu\rangle }{\langle \phi,\widetilde\mu\rangle }e^{\lambda s}.
  \]
\end{proof}

\begin{proof}[Proof of Theorem \ref{thm:distribution_of_zeta}.(\ref{subthm:LlogL})]
    From \eqref{eq:equation_for_nuvt} we know that
  \begin{equation}\label{eq:ext_equ_int}
    e^{-\lambda(s+t)}\langle v_{t+s}, \mu\rangle + e^{-\lambda t}\int_0^s e^{-\lambda u}\langle \Psi_0v_{t+u},\mu\rangle du
=e^{-\lambda t}\langle v_t,\mu\rangle,
\quad t,s\geq 0. 
  \end{equation}
	Thus $e^{-\lambda t}\langle v_t, \mu\rangle$ is decreasing in $t$. 
  Let
\begin{equation}
\label{eq:def_of_k}
    k:=
    \lim_{t\to\infty}e^{-\lambda t}\langle v_t, \nu\rangle
\end{equation}
then $k\in[0,\infty)$. 
Moreover, for any $x\in E$,
\begin{align}
  &e^{-\lambda t} v_t(x) 
  = e^{-\lambda t} \left\langle v_t,\nu \right\rangle \cdot \frac{v_t(x)}{\left\langle v_t,\nu \right\rangle}
  \\& \begin{multlined}= \left(k + o(\limsup_{t\to \infty} \sup_{x\in E})\right) \cdot \left(\phi(x)+o(\limsup_{t\to \infty}\sup_{x\in E})\right),
\\\text{by \eqref{eq:def_of_k} and Corollary \ref{cor:extinct}.(\ref{sub:extinct_3})},
\end{multlined}
  \\&=k\cdot \phi(x) +o(\limsup_{t\to \infty}\sup_{x\in E}),
\quad\text{since $\phi$ is bounded.}
\label{eq:uniform_convergence_of_elambdatvtx}
\end{align}
	Consequently,
  \begin{align}
    &e^{-\lambda t}\mathbb P_x\left( \left\| X_t \right\| > 0 \right) 
    = e^{-\lambda t} \left( 1-e^{-v_t(x)} \right)
    \\&=\frac{1-e^{-v_t(x)}}{v_t(x)} \cdot \left( k\phi(x) + o(\limsup_{t\to \infty} \sup_{x\in E}) \right),\quad\text{by~\eqref{eq:uniform_convergence_of_elambdatvtx}}
    \\&= \left( 1+o(\limsup_{t\to \infty} \sup_{x\in E}) \right)\cdot \left( k\phi(x) + o(\limsup_{t\to \infty} \sup_{x\in E}) \right),\quad\text{by~\eqref{eq:Vtf_convergence_to_zero}}
    \\&= k\phi(x) + o(\limsup_{t\to \infty}\sup_{x\in E}),\quad\text{since $\phi$ is bounded.}  
\end{align}

	Next, we investigate the value of $k$.  By the definition of the probability $\widetilde{\mathbb P}_\nu$ for $\nu(dx)=\widehat\phi(x)m(dx)$, we obtain that for any $t\ge 0,$
  \begin{align}\label{eq:subcritical_equality}
    &e^{-\lambda t}\mathbb P_\nu\left( \left\| X_t \right\| >0 \right) 
    = e^{-\lambda t}\mathbb P_\nu\left[ \frac{X_t(\phi)}{X_t(\phi)}\right] 
    \\ &= \widetilde {\mathbb P}_\nu\left[ \frac{1}{X_t(\phi)} \right],\quad\text{by~\eqref{eq:martingale_transformation}}
    \\ &= \widetilde {\mathbb Q}_\nu \left[ \frac{1}{X_t(\phi)+\widehat Z_t^{\mathrm n}(\phi) + \widehat Z_t^{\mathrm m}(\phi)} \right],\quad\text{by~Lemma \ref{lem:usage_of_reverse_spine_decomposition}}  
\end{align}

From the monotonicity, we write
\begin{align}
\lim_{t\to \infty} X_t(\phi) &= 0,
\\ \lim_{t\to \infty} \widehat Z_t^{\mathrm n}(\phi) &=: Z_\infty^{\mathrm n}(\phi),
\\ \lim_{t\to \infty} \widehat Z_t^{\mathrm m}(\phi) &=: Z_\infty^{\mathrm m}(\phi). 
\end{align}
Note that for each $t>t_0 > 0 $ we have
\begin{align}
  \left( X_t(\phi) + \widehat Z_t^{\mathrm n}(\phi) + \widehat Z_t^{\mathrm m}(\phi) \right)^{-1} 
    \leq \left( \widehat Z_{t_0}^{\mathrm n}(\phi) + \widehat Z_{t_0}^{\mathrm m}(\phi) \right)^{-1} 
\end{align}
and that the right hand side of the above is integrable with respect to probability $\widetilde {\mathbb Q}_\nu$ provided $t_0$ is large enough, since according to (? Size-biased add-on of the superprocess is size-biased transform of the Kuznestov measure.) we have
\begin{align}
  \widetilde {\mathbb Q}_\nu\left[ \left( \widehat Z_{t_0}^{\mathrm n}(\phi) + \widehat Z_{t_0}^{\mathrm m}(\phi) \right)^{-1}  \right]
  = \mathbb N_\nu^{w_{t_0}(\phi)} \left[ w_{t_0}(\phi)^{-1} \right]
  = \frac{\mathbb N_\nu \left(w_{t_0}(\phi) > 0 \right)}{\mathbb N_\nu[w_{t_0}(\phi)]}
  = \mathbb N_\nu \left(w_{t_0}(\phi) > 0 \right)
\end{align}
Therefore, by dominated convergence theory,
  \begin{equation}\label{eq:cons}
    k=\widehat{\mathbb Q}_{\nu}\left[\frac{1}{\sum_{\sigma\in\mathcal D^{\mathrm m}}\langle \phi, \widehat X_{\sigma}^{{\mathrm m},\sigma}\rangle +\sum_{\sigma\in \mathcal D^{\mathrm n}}\langle \phi, \widehat X_{\sigma}^{{\mathrm n},\sigma}\rangle }\right].
  \end{equation}
  \begin{iss}[Open]~
    \begin{itemize}
    \item[ZS:]
I can't see the above. What is the dominating function?
\item[ZS:]
OK, Here is a dominatin function. I need to resove this quickly.
    \end{itemize}
  \end{iss}
	For the continuum immigration part,
  \[
    \widehat{\mathbb Q}_{\nu}\left(\sum_{\sigma\in \mathcal D^{\mathrm n}}\langle \phi, \widehat X_{\sigma}^{{\mathrm n},\sigma}\rangle \right)=\int_0^\infty e^{\lambda s}\langle 2\alpha\phi, \phi\nu\rangle  ds=\frac{\langle 2\alpha\phi, \phi\nu\rangle}{-\lambda}<\infty.
  \]
	Therefore, $\sum_{\sigma\in \mathcal D^{\mathrm n}}\langle \phi, \widehat X_{\sigma}^{{\mathrm n},\sigma}\rangle$ is finite almost surely.

	For the discrete immigration part, let $\mathcal G=\sigma(Y, {\color{red}( m_\sigma)_{\sigma\in\mathcal D^{\mathrm m}}})$.  When  $\int_E\widehat{\phi}(y)l(y)m(dy)<\infty$, by Lemma \ref{lem:import_lemma},
  \[
    \widehat{\mathbb Q}_{\nu}\left(\sum_{\sigma\in \mathcal D^{\mathrm m}}\langle \phi, \widehat X_{\sigma}^{{\mathrm m}}\rangle\Big|\mathcal G \right)
    =\sum_{\sigma<\infty}m_\sigma e^{\lambda \sigma}\phi(Y_{\sigma})<\infty,  \qquad\qquad \widehat{\mathbb Q}_{\nu}-{\mathrm a.s.}
  \]
	Thus $k>0$.

	We claim that when $\int_D\widehat\phi(y)l(y)m(dy)=\infty$,
  \begin{equation}\label{eq:infty}
    \sum_{\sigma\in \mathcal D^{\mathrm m}}\langle \phi, \widehat X_{\sigma}^{{\mathrm m},\sigma}\rangle =\infty,\quad\qquad  \widehat{\mathbb Q}_\nu-{\mathrm a.s.}
  \end{equation}
	In the proof of \cite[Lemma $3.2$]{LiuRenSong2009Log}.  It is shown that for any $N>0$,
  \begin{equation}\label{eq:inf}
    \int_0^\infty dt\int_{\phi(Y_t)^{-1}e^{Nt}}^\infty rn(Y_t,dr)
    =\infty,\quad \widehat{\mathbb Q}_\nu-{\mathrm a.s.}
  \end{equation}
	Fix a path of $(Y_t)$.  And define stochastic
	time sequence
  \[
    \tau_0:=\left\{t>0; m_t>\phi(Y_t)^{-1}e^{Nt}\right\},\,
    \tau_{i+1}:=\left\{t>\tau_i;\, m_t>\phi(Y_t)^{-1}e^{Nt}\right\},\, i=0,1,\cdots
  \]
	Then $\tau_i<\infty$, $i=1,2,\ldots$ almost surely.
	If we can prove that $\sum_{i=0}^\infty I_{\left\{\langle\phi, \widehat X^{{\mathrm m},\tau_i}_{\tau_i}\rangle  >\varepsilon\right\}}=\infty,$ for some $\varepsilon>0,$ then our claim holds.  Similar to the argument for the proof of the second assertion of \cite[Lemma $2.2$]{LiuRenSong2009Log}, we just need to prove
  \[
    \widehat{\mathbb Q}_\nu
    \left[\left.\sum_{i=0}^\infty I_{\left\{\langle\phi, \widehat X^{{\mathrm m},\tau_i}_{\tau_i}\rangle  >\varepsilon\right\}} \right| Y\right]=\infty,\quad {\color{red}\widehat\Pi_{\phi\cdot\nu}}{\mathrm -a.s.}
  \]
	Since given the spatial motion of the spine the immigration process is a Poisson point process, therefore,
  \[
    \widehat{\mathbb Q}_\nu\left[\left.\sum_{i=0}^\infty I_{\left\{\langle\phi, \widehat X^{{\mathrm m},\tau_i}_{\tau_i}\rangle  >\varepsilon\right\}}\right| Y\right]=\int_0^\infty dt\int_{\phi(Y_t)^{-1}e^{Nt}}^\infty rn(Y_t, dr)\mathbb{P}_{r\delta_{Y_t}}\big(\langle\phi, X_t \rangle >\varepsilon\big).
  \]
	If we can prove for any time $t>0$,
  \begin{eqnarray}\label{eq:last_point}
    \inf_{r\geq \phi(x)^{-1}e^{Nt}, x\in E}\mathbb P_{r\delta_x}\big(\langle\phi, X_t
    \rangle >\varepsilon\big)>0,
  \end{eqnarray}
	then from \eqref{eq:inf}, \eqref{eq:infty} is obtained.  By Chebyshev inequality,
  \begin{eqnarray*}
    &&\mathbb P_{r\delta_x}\big(\langle\phi, X_t\rangle >\varepsilon\big)=\mathbb P_{r\delta_x}\left(e^{-\langle\phi, X_t\rangle }<e^{-\varepsilon}\right)\\
    &=&1-\mathbb P_{r\delta_x}\left(e^{-\langle\phi, X_t
        \rangle }\geq e^{-\varepsilon}\right)\geq 1-e^{\varepsilon }\mathbb P_{r\delta_x}e^{-\langle\phi, X_t\rangle }\\
    &=&1-e^{\varepsilon }e^{-rV_t\phi(x)},
  \end{eqnarray*}
	where, $V_t\phi(x)$ is a classical solution to evolution equation
  \begin{equation}\label{eq:diff}
    \begin{cases}
      \dfrac{\partial U}{\partial t}=AU-\beta(x)U-\psi(x, U),& x\in E, t>0;\\
      U(0,x)=\phi(x),& x\in E.\\
    \end{cases}
  \end{equation}
	Define a function on $[0,\infty)\times E,$ $V(t,x)=e^{-Nt}\phi(x).$  Then
  \[
    (A-\beta)V(t,x)-\psi(x, V)-\frac{\partial V(t,x)}{\partial t}=(\lambda +N)V(t,x)-(\psi+\beta)(x,V).
  \]
	It is obvious that $V$ is bounded on $[0,\infty)\times E$.  So we can choose $N$ large enough such that $(\lambda+N)V(t,x)-\psi(x,V)\geq 0$ on $[0,\infty)\times E.$
	Meanwhile, $V(0,x)=\phi(x)$.  Applying comparison theorem for semilinear equation, we obtain that $V_t\phi(x)\geq V(t,x)$.  So
  \begin{eqnarray*}
    &&\inf_{r\geq \phi(x)^{-1}e^{Nt}, x\in E}\mathbb P_{r\delta_x}\big(\langle\phi, X_t \rangle >\varepsilon\big)\geq \inf_{r\geq \phi(x)^{-1}e^{Nt}, x\in D} 1-e^{\varepsilon}e^{-rV_t\phi(x)}\\
    &&\geq \inf_{r\geq \phi(x)^{-1}e^{Nt}, x\in E} 1-e^{\varepsilon}e^{-rV(t,x)}
       \geq 1-e^{\varepsilon-1}.
  \end{eqnarray*}
	Choose $0<\varepsilon<1.$  The \eqref{eq:last_point} is obtained. And
  \eqref{eq:infty} follows. Thus $k=0$.
\end{proof}

\begin{rem}
  From the theorem we can see that $v(t,x)$  has the asymptotic behavior as $t\to\infty$
  \[
    v(t,x)\sim \phi(x)a(t)e^{\lambda t},
  \]
  where $\lim_{t\rightarrow\infty}a(t+s)/a(t)=1$ for all $s>0$.  When $\int_E\widehat\phi(y)l(y)m(dy)<\infty$, $\lim_{t\rightarrow\infty}a(t)=k$, which is given in \eqref{eq:def_of_k} or \eqref{eq:cons}.  When $\int_E\widehat\phi(y)l(y)m(dy)=\infty$, $\lim_{t\rightarrow\infty}a(t)=0$.
\end{rem}
%% -------------------------------------------------------------------------------------------------------------------
% --------------------------------------------------------------------------------------------------------------------------


\begin{proof}[{\bf Proof of Theorem \ref{thm:qsd_thm}}]
  For any nonzero $\mu\in \mathcal M(E)$, and any $f\in\mathcal B_b(E,[0,\infty))$,
  \begin{eqnarray*}
    \mathbb P_\mu\left(\left.1-\exp\{-\langle f, X_t\rangle \}\right|\zeta>t\right)= \frac{\mathbb P_\mu\left(1-\exp\{-\langle f, X_t\rangle \};\zeta>t\right)}{\mathbb P_\mu(\zeta>t)}=\frac{1-e^{-\langle V_tf,\mu\rangle }}{1-e^{-\langle v_t,\mu}\rangle }.
  \end{eqnarray*}
  Because $\lim_{t\rightarrow\infty}\langle V_tf(\cdot),\mu\rangle =\lim_{t\rightarrow\infty}\langle v_t,\mu\rangle =0$, due to lemma \ref{eq:ratio_limits_2},
  \[
    \lim_{t\rightarrow\infty}\mathbb P_\mu\left(\left.1-\exp\{-\langle f, X_t\rangle \}\right|\zeta>t\right)=\lim_{t\rightarrow\infty}\frac{\langle V_tf(\cdot),\mu\rangle }{\langle v_t,\mu\rangle }=e^{\lambda G(f)}.
  \]
  Thus
  \[
    \lim_{t\rightarrow\infty}\mathbb P_\mu\left(\left.\exp\{-\langle f, X_t\rangle \}\right|\zeta>t\right)=1-e^{\lambda G(f)}.
  \]
  When $f=0$, $G(f)=\infty$. Thus $1-e^{\lambda G(f)}$ is a Laplace functional of probability on $\mathcal M(E)$. We denote by $\mathbf P^{\lambda}$ the corresponding probability, then
  \[
    \mathbf P^{\lambda}(e^{-\langle f,\omega\rangle })=1-e^{\lambda G(f)}.
  \]
  Let $M^{(\lambda)}\in (0,\infty)$ be the random variable satisfying for any $\theta>0$,
  \[
    \lim_{t\rightarrow\infty}\mathbb P_\mu\left(\left.\exp\{-\theta\langle \phi, X_t\rangle \}\right|\zeta>t\right)
    =\mathbf P^{\lambda}(e^{-\theta M^{(\lambda)}})=1-e^{\lambda G(\theta\phi)}.
  \]
  Then for any $f\in \mathcal B_b(E,[0,\infty))$,
  \[
    \mathbf P^{\lambda}(e^{-\langle f,M^{(\lambda)}\nu\rangle })
    =\mathbf P^{\lambda}(e^{-\langle f,\nu\rangle M^{(\lambda)}})
    =1-e^{\lambda G(\langle f,\nu\rangle \phi)}
  \]
  By the definition of the function $G$,
  \[
    \langle v(G(f),\cdot),\nu\rangle =\langle f,\nu\rangle =\langle f,\nu\rangle \langle \phi,\nu\rangle =\langle v(G(\langle f,\nu\rangle \phi),\cdot),\nu\rangle .
  \]
  Since $\langle v_t,\nu\rangle $ is strictly decreasing with respect to $t$, so
  \begin{equation}\label{eq:eq_iden_G}
    G(f)=G(\langle f,\nu\rangle \phi).
  \end{equation}
  As a consequence, $\mathbf P^\lambda$ is concentrated on the set
  $$
  \mathcal S:=\{\mu\in\mathcal M(E); \mu(dx)=a\widehat\phi(x)m(dx), a\geq 0\}.
  $$
  Or we can say  $X_t|_{\zeta>t}$ converges to  $M^{(\lambda)}\widehat\phi(x)m(dx)$ weakly under $\mathbb P_\mu$.


  Therefore, the distribution of $M^{(\lambda)}\widehat\phi(x)m(dx)$ under $\mathbf P^{\lambda }$ is the Yaglom distribution of $(X_t)$.  So it is the quasi-stationary distribution associated to $\lambda$.  The Laplace transform of $M^{(\lambda)}$ is given by
  \begin{equation}\label{eq:def_of_Y}
    \mathbf P^{\lambda}(e^{-\theta M^{(\lambda)}})=1-e^{\lambda G(\theta\phi)}:=1-e^{\lambda B(\theta)},\qquad\quad\theta>0.
  \end{equation}
  According to \cite[Lemma $3.2$]{Lambert2007Quasistationary}, $1-e^{\gamma B(\theta)}$ for $\lambda\leq \gamma<0$ is also a Laplace transform of some probability measure on $(0,\infty)$.  Denote by $M^{(\gamma)}$ be the corresponding random variable.  Let $\mathbf P^{\gamma}$ be the distribution of $M^{(\gamma)}\widehat\phi(x)m(dx)$.  We claim that $\mathbf P^{\gamma}$ is the quasi-stationary distribution of $X_t$ associated to the rate of mass decay $-\gamma$.  Since for any $f\in\mathcal B_b(E,[0,\infty))$,
  \[
    \mathbf P^{\gamma}(e^{-\langle f,M^{(\gamma)}\nu\rangle })=\mathbf P^{\gamma}(e^{-\langle f,\nu\rangle M^{(\gamma)}})=1-e^{\gamma B(\langle f,\nu\rangle )}=1-e^{\gamma G(\langle f,\nu\rangle \phi)}=1-e^{\gamma G(f)}.
  \]
  Therefore,
  \begin{eqnarray*}
    \mathbf P^{\gamma}\mathbb P(\zeta>t)=1-\mathbf P^{\gamma}e^{-M^{(\gamma)}\langle v_t,\nu\rangle }=1-(1-e^{\gamma G(v_t)})=e^{\gamma t}.
  \end{eqnarray*}
  The last identity above comes from Lemma \ref{eq:ratio_limits_2}. Meanwhile, it follows from \eqref{eq:eq_iden_G} that
  \[
    \mathbf P^{\gamma}\mathbb P\left(1-e^{\langle f, X_t\rangle }\right)=1-\mathbf P^{\gamma}\left(e^{-\langle V_tf,\nu\rangle M^{(\gamma)}}\right)
    =e^{\gamma G(V_tf)}=e^{\gamma G(\langle V_tf,\nu\rangle\phi)}.
  \]
  We have shown that
  \begin{equation}\label{eq:ident_Vv}
    \langle V_tf,\nu\rangle =\langle v(t+G(f),\cdot),\nu\rangle,
  \end{equation}
  and that
  \begin{equation}\label{eq:ident_Gv}
    G(\langle v(t+G(f),\cdot),\nu\rangle \phi)=G(v(t+G(f),\cdot))=t+G(f).
  \end{equation}
  Thus
  \[
    \mathbf P^{\gamma}\mathbb P\left(1-e^{\langle f, X_t\rangle }\right)=e^{\gamma(t+G(f))}.
  \]
  And furthermore,
  \begin{eqnarray*}
    \mathbf P^{\gamma}\mathbb P\left(e^{-\langle f, X_t\rangle }\big|\zeta>t\right)&=&1-\mathbf P^{\gamma}\mathbb P\left(1-e^{-\langle f, X_t\rangle }\big|\zeta>t\right)=1-e^{\gamma(t+G(f))}e^{-\gamma t}\\
                                                                                   &=&1-e^{\gamma G(f)}=\mathbf P^{\gamma}(e^{-\langle f,M^{(\gamma)}\nu\rangle }).
  \end{eqnarray*}
  Our claim is proved.  So far the proof of the existence of the QSD associated to $\gamma\in[\lambda,0)$ is complete.  Now let us discuss the uniqueness of the QSD.

  For $\gamma<0$, we assume there is a quasi-stationary distribution $\mathbf P$.  Then
  \[
    \mathbf P\mathbb P(\zeta>t)=1-\int_{\mathcal M(E)}e^{-\langle v_t,\omega\rangle }\mathbf P(d\omega)=e^{\gamma t}.
  \]
  And for any nonnegative bounded Borel function $f$ on $E$,
  \begin{eqnarray*}
    \mathbf P\mathbb P\left(e^{-\langle f, X_t\rangle }\big|\zeta>t\right)&=&1-e^{-\gamma t}\mathbf P\mathbb P\left(1-e^{-\langle f, X_t\rangle }\right)\\
                                                                          &=&1-e^{-\gamma t}\left(1-\int_{\mathcal M(E)}e^{-\langle V_tf,\omega\rangle }\mathbf P(d\omega)\right)\\
                                                                          &=&\int_{\mathcal M(E)}e^{-\langle f,\omega\rangle }\mathbf P(d\omega).
  \end{eqnarray*}
  Recall that $\lim_{t\rightarrow\infty}\dfrac{\langle v(t+s,\cdot),\omega\rangle }{\langle v_t,\nu\rangle }=\langle \phi,\omega\rangle e^{\lambda s} $ and that $\lim_{t\rightarrow\infty}\dfrac{\langle V_{t+s}f,\omega\rangle }{\langle V_tf,\nu\rangle }=\langle \phi,\omega\rangle e^{\lambda s}$ uniformly for all $\omega\in \mathcal M(E)\setminus\{0\}$.  For any $\varepsilon>0$, there is $T>0$, such that when $t>T$,
  \[
    (1-\varepsilon)\langle v_t,\nu\rangle \langle \phi,\omega\rangle\leq \langle v_t,\omega\rangle\leq (1+\varepsilon)\langle v_t,\nu\rangle \langle \phi,\omega\rangle,\qquad \omega\in\mathcal M(E).
  \]
  Therefore, on one hand,
  \begin{eqnarray*}
    1&=&e^{-\gamma t}\left(1-\int_{\mathcal M(E)}e^{-\langle v_t,\omega\rangle }\mathbf P(d\omega)\right)\\
     &\leq &e^{-\gamma t}\left(1-\int_{\mathcal M(E)}e^{-\langle v_t,\nu\rangle\langle\phi,\omega\rangle }\mathbf P(d\omega)\right)+e^{-\gamma t}\int_{\mathcal M(E)}\left(1-e^{-\varepsilon\langle v_t,\nu\rangle\langle\phi,\omega\rangle  }\right)\mathbf P(d\omega)\\
     &=&I_t+II_t.
  \end{eqnarray*}
  For the given $\varepsilon>0$, choose $s>0$, such that $e^{\lambda s}/2>\varepsilon$.  And there is $T_1>0$, such that $t>T_1$,
  \[
    \varepsilon\langle v_t,\nu\rangle\langle\phi,\omega\rangle\leq \langle v(t+s,\cdot),\omega\rangle \quad\mbox{for all}\, \omega\in\mathcal M(E).
  \]
  So
  \begin{eqnarray*}
    0\leq II_t&\leq& e^{-\gamma t}\int_{\mathcal M(E)}\left(1-e^{-\varepsilon\langle v_t,\nu\rangle\langle\phi,\omega\rangle  }\right)\mathbf P(d\omega)\\
              &\leq& e^{-\gamma t}\int_{\mathcal M(E)}\left(1-e^{-\langle v(t+s,\cdot),\omega\rangle  }\right)\mathbf P(d\omega)\\
              &=&e^{\gamma s}.
  \end{eqnarray*}
  We can see that when $\varepsilon\to0+$, $s$ can be chosen any large number.  Then
  \[
    \limsup_{t\to\infty}II_t\leq e^{\gamma s}.
  \]
  Let $s\to\infty$, $\lim_{t\to\infty}II_t=0$. Thus $\liminf_{t\to\infty}I_t\geq 1$.

  On the other hand,
  \begin{eqnarray*}
    1&=&e^{-\gamma t}\left(1-\int_{\mathcal M(E)}e^{-\langle v_t,\omega\rangle }\mathbf P(d\omega)\right)\\
     &\geq &e^{-\gamma t}\left(1-\int_{\mathcal M(E)}e^{-\langle v_t,\nu\rangle\langle\phi,\omega\rangle }\mathbf P(d\omega)\right)-e^{-\gamma t}\int_{\mathcal M(E)}\left(1-e^{-\varepsilon\langle v_t,\nu\rangle\langle\phi,\omega\rangle  }\right)\mathbf P(d\omega)\\
     &=&I_t-II_t.
  \end{eqnarray*}
  It follows that $\limsup_{t\to\infty}I_t\leq 1$. In conclusion,
  \begin{equation}\label{eq:limit1}
    \lim_{t\rightarrow\infty}e^{-\gamma t}\left(1-\int_{\mathcal M(E)}e^{-\langle v_t,\nu\rangle\langle\phi,\omega\rangle }\mathbf P(d\omega)\right)=1.
  \end{equation}
  Repeating the arguments for \eqref{eq:limit1}, we can deduce the following limit
  \begin{equation}\label{eq:limit2}
    \lim_{t\rightarrow\infty}e^{-\gamma t}\left(1-\int_{\mathcal M(E)}e^{-\langle V_tf,\nu\rangle \langle \phi,\omega\rangle }\mathbf P(d\omega)\right)=1-\int_{\mathcal M(E)}e^{-\langle f,\omega\rangle }\mathbf P(d\omega).
  \end{equation}
  Since $\langle V_tf,\nu\rangle =\langle v(t+G(f),\cdot),\nu\rangle $, combining \eqref{eq:limit1} and \eqref{eq:limit2}, we get
  \begin{equation}\label{eq:lap_qsd}
    \int_{\mathcal M(E)}e^{-\langle f,\omega\rangle }\mathbf P(d\omega)=1-e^{\gamma G(f)}.
  \end{equation}
  When  $\lambda\leq\gamma<0$, this Laplace functional is the same in form to that of the QSD $\mathbf P^{\gamma}$.  The uniqueness follows.  Now the assertions $(1)$ and $(2)$ are proved.

  If $G(\theta\phi)$ is seen as a function of $\theta$ denoted by $B(\theta)$, then the definition \eqref{eq:def_of_Y} of the
  random variable $M^{(\lambda)}$ can yield
  \[
    0<\mathbf P^{\lambda}(M^{(\lambda)})=-\lambda\lim_{\theta\rightarrow 0+} e^{\lambda B(\theta)}B'(\theta).
  \]
  Observe that $\lim_{\theta\rightarrow 0+} B(\theta)=+\infty$, therefore $\lim_{\theta\rightarrow 0+}B'(\theta)=+\infty$. Taking derivative in \eqref{eq:lap_qsd},
  and noting that $G(f)=G(\langle f,\nu\rangle\phi)$, for $r>0$,
  \[
    \int_{\mathcal M(E)}\langle f,\omega\rangle e^{-r\langle f,\omega\rangle }\mathbf P(d\omega)
    =-\gamma \langle f,\nu\rangle  e^{\gamma B(r\langle f,\nu\rangle)}B'(r\langle f,\nu\rangle).
  \]
  Therefore,
  \[
    \int_{\mathcal M(E)}\langle f,\omega\rangle \mathbf P(d\omega)=-\gamma \langle f,\nu\rangle  \lim_{r\rightarrow 0+}e^{\gamma B(r)}B'(r)
  \]
  When $\gamma<\lambda$,
  \[
    \lim_{r\rightarrow 0+}e^{\gamma B(r)}B'(r)=0.
  \]
  Therefor if the QSD $\mathbf P^{\gamma}$ exists in the cases $\gamma<\lambda$, then
  \[
    \int_{\mathcal M(E)}\langle f,\omega\rangle \mathbf P^{\gamma}(d\omega)=0.
  \]
  So there is no QSD associated to the rate of mass decay $-\gamma$, for $\gamma<\lambda$.
\end{proof}

%%% -------------------------------------------------------------------------------------------------------------------------------------------------
{\color{blue}According to the definition for QSD in \cite{ChampagnatVillemonais2018Convergence}, there should be some probability measure $\mathbf Q^\gamma$ on $\mathcal M(E)$, $\gamma\in[\lambda, 0)$ such that
  \[
    \mathbf Q^{\gamma}\mathbb P(X_t \in \cdot | \zeta > t) \xrightarrow[t\to \infty]{w} {\mathbf P^\gamma}(\cdot).
  \]
  From the proof above, we can find a class of $\mathbf Q^\gamma$, $\gamma\in (\lambda, 0)$, for $X$.
  \begin{prop}
    Let $Z^{(\alpha)}$ be a random variable whose Laplace exponent is proportional to $\Phi(\lambda)=\lambda^{\alpha}$ for $\alpha\in(0,1)$.  Let $\gamma=\alpha\lambda$ and $\mathbf Q^{\gamma}$ is the distribution of $Z^{(\alpha)}\widehat\phi(x)m(dx)$.  Then
    \[
      \mathbf Q^{\gamma}\mathbb P(X_t \in \cdot | \zeta > t) \xrightarrow[t\to \infty]{w} {\mathbf P^\gamma}(\cdot).
    \]
  \end{prop}
  \begin{proof}
    Since for any $f\in\mathcal B_b(E,[0,\infty))$,
    \[
      \mathbf Q^{\gamma}\mathbb P\left(1-e^{\langle f, X_t\rangle }\right)=1-\mathbf Q^{\gamma}\left(e^{-\langle V_tf,\nu\rangle Z^{(\alpha)}}\right)
      =1-e^{-c\langle V_tf,\nu\rangle^\alpha}=1-e^{-c\langle v(t+G(f),\cdot),\nu\rangle^\alpha},
    \]
    for some positive constant $c$. The last identity comes from \eqref{eq:ident_Vv}.  Meanwhile,
    \begin{eqnarray*}
      \mathbf Q^{\gamma}\mathbb P(\zeta>t)=1-\mathbf Q^{\gamma}e^{-Z^{(\alpha)}\langle v_t,\nu\rangle }=1-e^{-c\langle v_t,\nu\rangle^\alpha}.
    \end{eqnarray*}
    Therefore, Thanks to corollary \ref{cor:general_rate},
    \begin{eqnarray*}
      &&\lim_{t\to\infty}\mathbf Q^{\gamma}\mathbb P\left(1-e^{\langle f, X_t\rangle }\big|\zeta>t\right)=\lim_{t\to\infty}\dfrac{1-e^{-c\langle v(t+G(f),\cdot),\nu\rangle^\alpha}}{1-e^{-c\langle v_t,\nu\rangle^\alpha}}\\
      &=&\lim_{t\to\infty}\dfrac{\langle v(t+G(f),\cdot),\nu\rangle^\alpha}{\langle v_t,\nu\rangle^\alpha}= e^{\alpha\lambda G(f)}=1-(1-e^{\gamma G(f)})\\
      &=&1-\mathbf P^{\gamma}(e^{-\langle f,\omega\rangle}).
    \end{eqnarray*}
    The proof is completed.
  \end{proof}
}


%% --------------------------------------------------------------------------------------------------------------------------------------------------
%% ----------------------------------------------------------------------------------------------------------------------------
\begin{proof}[Proof of Theorem \ref{thm:Q_process}]
  Assume $s>t$.  For any $A\in\mathscr F_t$, by the Markov property of $X$,
  \[
    \mathbb P_\mu(A|\zeta>s)=\dfrac{\mathbb P_\mu(A, \zeta>s)}{\mathbb P_\mu(\zeta>s)}=\dfrac{\mathbb P_\mu\big(\mathbb P_{X_t}(\zeta>s-t);A\big)}{\mathbb P_\mu(\zeta>s)},
  \]
  Note that
  \begin{eqnarray*}
    \lim_{s\rightarrow\infty}\dfrac{\mathbb P_{X_t}(\zeta>s-t)}{\mathbb P_\mu(\zeta>s)}
    &=&\lim_{s\rightarrow\infty}\dfrac{1-e^{-\langle v(s-t,\cdot),X_t\rangle }}{1-e^{-\langle v(s,\cdot),\mu\rangle }}
        =\lim_{s\rightarrow\infty}\dfrac{\langle v(s-t,\cdot),X_t\rangle }{\langle v(s,\cdot),\mu\rangle }\\
    &=& \dfrac{e^{-\lambda t}\langle \phi, X_t\rangle }{\langle \phi,\mu\rangle }=\dfrac{M_t}{\langle \phi,\mu\rangle }.
  \end{eqnarray*}
  The third identity follows from \eqref{ext con}.  From \eqref{eq:upp} we can get that there is a constant $\widetilde C>0$, such that for any $s>T>0$ and $x\in E$,
  \[
    v(s,x)\leq \widetilde C\phi(x)e^{\lambda T}\langle v(s-T,\cdot),\nu\rangle .
  \]
  Using the fact that $\lim_{x\rightarrow 0+}\dfrac{1-e^{-x}}{x}=1$, we choose $s$ sufficiently large such that
  \[
    1-e^{-\langle v(s,\cdot),\mu\rangle }>\frac{1}{2}\langle v(s,\cdot),\mu\rangle .
  \]
  Meanwhile since $1-e^{-x}\leq x$ for $x>0$, choosing $0<T_0<s-t$
  \[
    \dfrac{1-e^{-\langle v(s-t,\cdot),X_t\rangle }}{1-e^{-\langle v(s,\cdot),\mu\rangle }}
    \leq \dfrac{2\langle v(s-t,\cdot),X_t\rangle }{\langle v(s,\cdot),\mu\rangle }\leq \dfrac{2\widetilde C\langle \phi,X_t\rangle e^{\lambda T_0}\langle v(s-t-T_0,\cdot),\nu\rangle }{\langle v(s,\cdot),\mu\rangle }.
  \]
  We already show in \eqref{ext con} that
  \[
    \lim_{s\rightarrow\infty}\dfrac{\langle v(s-t-T_0,\cdot),\nu\rangle }{\langle v(s,\cdot),\mu\rangle }
    =e^{-\lambda(t+T_0)}\langle \phi,\mu\rangle ^{-1}.
  \]
  And $\langle \phi,X_t\rangle $ is integrable with respect to $\mathbb P_\mu$.  Thus by dominated convergence theorem,
  \[
    \lim_{s\rightarrow\infty}\mathbb P_\mu(A|\zeta>s)=\mathbb P_\mu\left(\frac{M_t}{\langle\phi,\mu\rangle };A\right)=\widetilde{\mathbb P}_\mu(A).
  \]
\end{proof}
%%%%%%%%%%%%%%%%%%%%%%%%%%%%%%%%%% -----------------------------------------------------------------------------------------
\begin{proof}[Proof of Theorem \ref{thm:structure_of_Qprocess}]
  It has been shown in section $2.3$, under $\widetilde{\mathbb P}_\mu$,  $X_t$ has a spine representation
  (see Lemma \ref{lem:spine_structure}) for any $t>0$.  For any $f\in\mathcal B_b(E,[0,\infty))$,
  \[
    \widetilde {\mathbb P}_{\mu}\left(e^{-\langle f, X_t\rangle }\right)=\mathbb Q_{\mu}\left(e^{-\langle f, X_t\rangle+\langle f, Z^{{\mathrm m},[0,t)}_t+Z^{{\mathrm n},[0,t)}_t\rangle }\right).
  \]
  When $\mu(dx)=\nu(dx)=\widehat\phi(x)m(dx)$,
  \[
    \mathbb Q_{\nu}\left(e^{-\langle f, X_t\rangle+\langle f, Z^{{\mathrm m},[0,t)}_t+Z^{{\mathrm n},[0,t)}_t\rangle }\right)=\widehat{\mathbb Q}_{\nu}\left(e^{-\langle f, X_t\rangle-\langle f, \widehat Z^{{\mathrm m}}_t+\widehat Z^{{\mathrm n}}_t\rangle }\right).
  \]
  Since $\lim_{t\to\infty}\langle\phi, X_{t}\rangle=0$ in probability and $\sum_{\sigma\in(0, t]\bigcap\mathcal D^{\mathrm m}}\langle \phi, \widehat X_{\sigma}^{{\mathrm m},\sigma}\rangle +\sum_{\tau\in (0, t]\bigcap \mathcal D^{\mathrm n}}\langle \phi, \widehat X_{\tau}^{{\mathrm n},\tau}\rangle $ is increasing with respect to $t$ having almost sure limit as $t\to\infty$ as well.  Therefore, by dominated convergence theory,
  \[
    \lim_{t\to\infty}\widetilde {\mathbb P}_{\nu}\left(e^{-\langle f, X_t\rangle }\right)=\widehat{\mathbb Q}_{\nu}\left(e^{-\langle f,\sum_{\sigma\in\mathcal D^{\mathrm m}}\widehat X^{{\mathrm m},\sigma}_\sigma+\sum_{\tau\in\mathcal D^{\mathrm n}}\widehat X^{{\mathrm n},\tau}_\tau\rangle }\right)
  \]
  Denote the random measure $\sum_{\sigma\in\mathcal D^{\mathrm m}}\widehat X^{{\mathrm m},\sigma}_\sigma+\sum_{\tau\in\mathcal D^{\mathrm n}}\widehat X^{{\mathrm n},\tau}_\tau$ by $X_\infty$.
  Then
  \[
    \lim_{t\to\infty}\widetilde {\mathbb P}_{\nu}\left(e^{-\langle f, X_t\rangle }\right)=\widehat{\mathbb Q}_\nu \left(e^{-\langle f,X_\infty\rangle}\right).
  \]
  Compare $X_\infty$ with the random measure in the definition \eqref{eq:cons} of the constant $k$.  Then
  $$
  k=\widehat{\mathbb Q}_\nu\left(\dfrac{1}{\langle\phi, X_\infty\rangle}\right).
  $$
  According to the discussion for $k$, we know that $\langle\phi, X_\infty\rangle<\infty$ if and only if $\int_E\widehat\phi(x)l(x)m(dx)<\infty$.


  Now let us consider the cases for general initial value $\mu\in\mathcal M(E)$.  For $f\in\mathcal B_b(E,[0,\infty))$, define function
  \begin{equation}\label{def:H}
    H(x,t):={\mathbb Q}_x\left(e^{-\langle f, Z^{\mathrm n, [0,t)}_{t} + Z^{\mathrm m, [0,t)}_{t}\rangle }\right).
  \end{equation}
  Then
  \begin{eqnarray*}
    &&\mathbb Q_\nu\left(\exp\Big\{-\langle f, X_t\rangle-\langle f, Z^{{\mathrm m},[0,t)}_t+Z^{{\mathrm n},[0,t)}_t\rangle \Big\}\right)\\
    &=&\mathbb P_\nu\left(e^{-\langle f, X_t\rangle}\right)\int_E\phi(y)\widehat\phi(y)H(y,t)m(dy).
  \end{eqnarray*}
  %%%%%%%%%%%%%%%%%%%%%%%%%%%%%%%%%%%%%%%%%%%%%%%%%%%%%%%%%%%%%%%%%%%%%%%%%%%%%%%%%%%%%%%%%%%%%%%%%%%%%%%%%%%%%%%%%%%%%%%%%%%%%%%%%%%%%%%%%%%%%%%%%%%%%%%%%%%%%%%%%%% 
  Define the filtration generated by the spine process that $\mathcal{H}_t=\sigma\big(Y_s; s\leq t\big)$, $t\geq 0$.  Then for $T,t>0$,
  \begin{equation}\label{eq:subcritical_upper_bound}
    \begin{aligned}
      &H(x,t+T)\\
      =&\mathbb Q_{x}\mathbb Q_{x}\Big[\exp\Big\{-\sum_{\sigma\in (0, t+T]\bigcap \mathcal D^{\mathrm m}}\langle f, X_{t+T-\sigma}^{{\mathrm m},\sigma}\rangle -\sum_{\tau\in (0, t+T]\bigcap \mathcal D^{\mathrm n}}\langle f, X_{t+T-\tau}^{{\mathrm n}, \tau}\rangle \Big\}\Big| \mathcal H_t\Big]\\
      \leq&\widetilde\Pi_x\mathbb Q_{x}\Big[\exp\Big\{-\sum_{\sigma\in (t, t+T]\bigcap \mathcal D^{\mathrm m}}\langle f, X_{t+T-\sigma}^{{\mathrm m},\sigma}\rangle -\sum_{\tau\in (t, t+T]\bigcap \mathcal D^{\mathrm n}}\langle f, X_{t+T-\tau}^{{\mathrm n}, \tau}\rangle \Big\}\Big| \mathcal H_t\Big]\\
      =&
      \widetilde\Pi_x\mathbb Q_{Y_t}\Big[\exp\Big\{-\sum_{\sigma\in (0, T]\bigcap \mathcal D^{\mathrm m}}\langle \phi, X_{T-\sigma}^{{\mathrm m},\sigma}\rangle -\sum_{\tau\in (0, T]\bigcap \mathcal D^{\mathrm n}}\langle \phi, X_{T-\tau}^{{\mathrm n}, \tau}\rangle \Big\}\Big]\\
      =&\widetilde\Pi_x\left[ H(Y_t, T)\right].
    \end{aligned}
  \end{equation}
  From \eqref{eq:IU}, there is some constants $c,\nu>0$ such that when $t>1$,
  \[
    H(x,t+T)\leq \widetilde\Pi_x\left[ H(Y_t, T)\right]\leq (1+ce^{-\nu t})\int_E\phi(y)\widehat\phi(y)H(y,T)m(dy)<\infty.
  \]
  Since $H(x,t)\leq 1$.  Set $\overline \eta(x)$ to be the supremum limit of $H(x,t)$ as $t\to \infty$.
  Fix time $T$ and let $t\to \infty$ in inequality \eqref{eq:subcritical_upper_bound}. We can imply that
  \begin{equation}\label{eq:sub_super}
    \overline\eta(x)\leq \int_E\phi(y)\widehat \phi(y)H(y,T)m(dy).
  \end{equation}
  Using Fatou's lemma for supremum limit
  in \eqref{eq:sub_super}, for any $x\in E$,
  \begin{equation}\label{eq:sup_inequality}
    \overline\eta(x)\leq \limsup_{T\rightarrow\infty}\int_E\phi(y)\widehat \phi(y)H(y,T)m(dy)\leq \int_E\phi(y)\widehat\phi(y)\overline{\eta}(y)m(dy).
  \end{equation}
  Since $\overline{\eta}(\cdot)\leq 1$, $\overline\eta(\cdot)$ is a constant function by \eqref{eq:sup_inequality}.
  Denote $\overline\eta(\cdot)$ by $q(f)$.  If $q(f)\equiv 0,$ then
  \begin{equation}\label{eq:limit}
    \lim_{t\rightarrow\infty}H(x,t)=q(f),\qquad \mbox{for all}\,\, x\in E.
  \end{equation}
  So in the
  following, we assume $q(f)>0$.
  For any $\varepsilon_1>0$, let
  $$
  \mu_1(T)=\int_{\{x\in
    E;H(x,T)>(1+\varepsilon_1)q(f)\}}\phi(x)\widehat\phi(x)m(dx).
  $$
  Then $\lim_{T\rightarrow\infty}\mu_1(T)=0.$  For any $\varepsilon_2>0$, let
  $$
  \mu_2(T)=\int_{\{x\in
    E;H(x,T)<(1-\varepsilon_2)q(f)\}}\phi(x)\widehat\phi(x)m(dx).
  $$
  Then we can deduce from \eqref{eq:sup_inequality} that
  \begin{eqnarray}\label{eq:sub_limit_in_prob}
    q(f)&\leq&
               (1-\varepsilon_2)q(f)\mu_2(T)+\mu_1(T)+(1+\varepsilon_1)q(f)(1-\mu_1(T)-\mu_2(T))\\
        &\le
             &(1+\varepsilon_1)q(f)-(\varepsilon_1+\varepsilon_2)\mu_2(T)+C\mu_1(T),
  \end{eqnarray}
  where $C$ is some positive finite constant.  Hence
  \begin{eqnarray*}\label{eq:sublimt_inequality}
    q(f)&\leq&
               \liminf_{T\rightarrow\infty}(1+\varepsilon_1)q(f)-(\varepsilon_1+\varepsilon_2)\mu_2(T)+C\mu_1(T)\\
        &=&(1+\varepsilon_1)q(f)-(\varepsilon_1+\varepsilon_2)q(f)\limsup_{T\rightarrow\infty}\mu_1(T).
  \end{eqnarray*}
  Since $\varepsilon_1$ is an arbitrary positive constant.
  \[
    q(f)\leq q(f)-\varepsilon_1 q(f)\limsup_{T\rightarrow\infty}\mu_1(T).
  \]
  This is impossible unless $\limsup_{T\rightarrow\infty}\mu_1(T)=0.$
  Therefore, $H(\cdot,T)$ converges to $q(f)$ in probability under probability $\phi(x)\widehat{\phi}(x)m(dx)$ as $T\to\infty$.


  Meanwhile, from the definition
  \eqref{def:H} of function $H$, we can get the following inequality
  \begin{equation}\label{eq:subsub}
    \begin{aligned}
      H(x,t+T)\geq& \mathbb Q_{x}\prod_{\sigma\leq t}I_{\{ X_{t+T-\sigma}^{{\mathrm m},\sigma}=0\}}\prod_{\tau\leq t}I_{\{ X_{t+T-\tau}^{{\mathrm n},\tau}=0\}}\\
      &\cdot\mathbb Q_{Y_t}\Big[\exp\Big\{-\sum_{\sigma\in (0, T]\bigcap \mathcal D^{\mathrm m}}\langle \phi, X_{T-\sigma}^{{\mathrm m},\sigma}\rangle -\sum_{\tau\in (0, T]\bigcap \mathcal D^{\mathrm n}}\langle \phi, X_{T-\tau}^{{\mathrm n},\tau}\rangle \Big\}\Big]\\
      =& \mathbb Q_{x}\left[\prod_{\sigma\leq t}I_{\{ X_{t+T-\sigma}^{{\mathrm m},\sigma}=0\}}\prod_{\tau\leq t}I_{\{ X_{t+T-\tau}^{{\mathrm n},\tau}=0\}}H(Y_t, T)\right].
    \end{aligned}
  \end{equation}
  Consider the following probability,
  \begin{eqnarray*}
    \mathbb Q_{x}\left(\prod_{\sigma\leq t}I_{\{ X_{t+T-\sigma}^{{\mathrm m},\sigma}=0\}}=1\right)
    =\widetilde\Pi_x\exp\left\{-\int_0^tds\int_0^\infty r(1-\mathbb P_{r\delta_{Y_s}}(\zeta<T+t-s))n(Y_s,dr)\right\}.
  \end{eqnarray*}
  Since in the case of $\lambda<0,$ the $(Y,\psi)$-superprocess starting from any finite measure is extinct in
  finite time.  So by dominated convergence theorem,
  \begin{equation}\label{eq:1_infty_limit}
    \lim_{T\rightarrow\infty}\int_0^tds\int_1^\infty r(1-\mathbb P_{r\delta_{Y_s}}(\zeta<T+t-s))n(Y_s,dr)=0,
  \end{equation}
  $\widetilde\Pi_x$ almost surely.   Note that
  \[
    1-\mathbb P_{r\delta_{Y_s}}(\zeta<T+t-s)\leq 1-(1-\mathbb P_{Y_s}(\zeta>T))^r.
  \]
  And recall that there are $T_0>0$ and $\eta>0$ such that when $T>T_0$, for any $x\in E$,
  \[
    \mathbb P_x(\zeta>T)\leq \eta \phi(x)e^{\lambda T}.
  \]
  Since when $x\rightarrow 0+$, $1-(1-x)^r\sim rx$ for any $r>0$, we assume $T_0$ is sufficiently large such that
  $\eta \phi(x)e^{\lambda T}$ is small enough so that $1-(1-\mathbb P_{Y_s}(\zeta>T))^r\leq 2r\eta \phi(Y_s)e^{\lambda T}$.
  Therefore,
  \[
    \int_0^tds\int_0^1 r(1-\mathbb P_{r\delta_{ Y_s}}(\zeta<T+t-s))n(Y_s,dr)\leq 2\eta e^{\lambda T}\int_0^t\phi(Y_s)ds\int_0^1 r^2 n(Y_s,dr).
  \]
  As a result,
  \begin{equation}\label{eq:01_limit}
    \lim_{T\rightarrow\infty}\int_0^tds\int_0^1 r(1-\mathbb P_{r\delta_{Y_s}}(\zeta<T+t-s))n(Y_s,dr)=0,
  \end{equation}
  $\widetilde\Pi_x$ almost surely.  Combining \eqref{eq:1_infty_limit} and \eqref{eq:01_limit}, we get
  \[
    \lim_{T\rightarrow\infty}\mathbb Q_{x}\left(\prod_{\sigma\leq t}I_{\{ X_{t+T-\sigma}^{{\mathrm m},\sigma}=0\}}=1\right)=1.
  \]
  Meanwhile,
  \begin{eqnarray*}
    \mathbb Q_x\left(\prod_{\sigma\leq t}I_{\{ X_{t+T-\sigma}^{{\mathrm n},\sigma}=0\}}=1\right)
    &=&\widetilde\Pi_x\exp\left\{-\int_0^t2\alpha(Y_s)\mathbb N_{Y_s}(\zeta<T+t-s)ds\right\}\\
    &=&\widetilde\Pi_x\exp\left\{-\int_0^t2\alpha(Y_s)v(T+t-s,Y_s)ds\right\}.
  \end{eqnarray*}
  We also get
  \[
    \lim_{T\rightarrow\infty}\mathbb Q_x\left(\prod_{\tau\leq t}I_{\{ X_{t+T-\tau}^{{\mathrm n},\tau}=0\}}=1\right)=1,
  \]
  since $\lim_{T\rightarrow\infty} v(T+t-s,x)=0$ for any $x$, and it is bounded when time $T$ is sufficiently large.  By the inequality \eqref{eq:IU}, for any $\varepsilon>0$ and $t>1$, there are $c>0$ and $\nu>0$, such that for any $x\in E$,
  \begin{eqnarray*}
    &&\limsup_{T\rightarrow\infty}\widetilde\Pi_x\left(|H(Y_t, T)-q(f)|>\varepsilon\right)\\
    &\leq& \limsup_{T\rightarrow\infty}(1+ce^{-\nu t})\int_E\phi(y)\widehat\phi(y)m(dy)I_{\{|H(y, T)-q(f)|>\varepsilon\}}=0.
  \end{eqnarray*}
  Then from the inequality \eqref{eq:subsub}, we have for any $x\in E$,
  \begin{eqnarray*}
    \liminf_{T\rightarrow\infty}H(x, t+T)&\geq&  \liminf_{T\rightarrow\infty} \mathbb Q_x\left[\prod_{\sigma\leq t}I_{\{ X_{t+T-\sigma}^{{\mathrm m},\sigma}=0\}}\prod_{\tau\leq t}I_{\{ X_{t+T-\tau}^{{\mathrm n},\tau}=0\}}H(Y_t, T)\right]\\
                                         &\geq& q(f)=\limsup_{t\rightarrow\infty}H(x, t).
  \end{eqnarray*}
  Therefore  $\lim_{t\rightarrow\infty}H(x, t)=q(f)$ for any $x\in E$.
  %%%%%%%%%%%%%%%%%%%%%%%%%%%%%%%%%%%%%%%%%%%%%%%%%%%%%%%%%%%%%%%%%%%%%%%%%%%%%%%%%%%%%%%%%%%%%%%%%%%%%%%%%%%%%%%%%%%%%%%%%%%%%%%%%%%%%%%%%%%%%%%%%%%%%%%%%%%%%%%%%%% 
  Due to $0\leq H(x,t)\leq 1$,
  \begin{equation*}
    q(f)=\lim_{t\rightarrow\infty}\int_E\phi(x)\widehat\phi(x)H(x,t)m(dx)
    =\lim_{t\rightarrow\infty}\widetilde{\mathbb P}_{\nu}\left(e^{-\langle f, X_t\rangle }\right)
    =\widehat{\mathbb Q}_{\nu}\left(e^{-\langle f, X_{\infty}\rangle }\right).
  \end{equation*}
  Therefore for any $\mu\in\mathcal M(E)\setminus\{0\}$, and $f\in\mathcal B_b(E,[0,\infty))$,
  \begin{eqnarray*}
    \lim_{t\rightarrow\infty}\widetilde{\mathbb P}_\mu\left(e^{-\langle f, X_t\rangle}\right)&=&\lim_{t\rightarrow\infty}\mathbb P_\mu\left(e^{-\langle f, X_t\rangle}\right)
                                                                                                 \lim_{t\to\infty}\dfrac{1}{\mu(\phi)}\int_E\phi(x)H(x, t)\mu(dx)=q(f)\\
                                                                                             &=&\widehat{\mathbb Q}_{\nu}\left(e^{-\langle f, X_{\infty}\rangle }\right).
  \end{eqnarray*}
  %% ----------------------------------------------------------------------------------------------------------------------------
  Note that
  \[
    \widehat{\mathbb Q}_{\nu}\big(\sum_{\tau\in \mathcal D^{\mathrm n}}\langle f, \widehat X_{\tau}^{{\mathrm n},\tau} \rangle \big)=\int_0^\infty2\langle \alpha P^{\beta}_sf,\phi\nu\rangle ds
    \leq 2\|\alpha\phi\|_\infty\dfrac{\langle f,\nu\rangle }{-\lambda}<\infty,
  \]
  and that
  \begin{eqnarray*}
    &&\widehat{\mathbb Q}_{\nu}\Big(\sum_{\sigma\in [1,\infty)\bigcap\mathcal D^{\mathrm m}}\langle f, \widehat X_{\sigma}^{{\mathrm m},\sigma} \rangle|\mathcal G \Big)
       =\sum_{t\in [1,\infty)\bigcap\mathcal D^{\mathrm m}}m_tP^{\beta}_tf( Y_t)\\
    &&\leq \sum_{t\in \mathcal D^{\mathrm m}}(1+ce^{-\rho t})m_te^{\lambda t}
       \phi(Y_t)\int_E\widehat\phi(y)f(y)m(dy)\\
    &&=\int_E\widehat\phi(y)f(y)m(dy) \sum_{t\in \mathcal D^{\mathrm m}}(1+ce^{-\rho t})m_te^{\lambda t}\phi(Y_t)<\infty,
  \end{eqnarray*}
  $\mathbb Q_{\nu}-{\mathrm a.s.}$ when $\int_E\widehat\phi(x)l(x)m(dx)<\infty$ by Lemma \ref{lem:import_lemma}.
  Thus in this case, the limit measure $X_\infty\in \mathcal M(E)$.  Denote the distribution of $X_\infty$ by $\mathbf P$.  Then $\mathbf P$
  is the invariant probability of the $Q$-process.
  %% ----------------------------------------------------------------------------------------------------------------------------


  When $\int_E\widehat\phi(x)l(x)m(dx)<\infty$, we analyze the Yaglom distribution first. Since the Yaglom distribution is independent of the initial value
  $\mu$.  Without loss of generality, we suppose $\mu(dx)=\nu(dx)=\widehat\phi(x)m(dx)$.  For $f\in\mathcal B_b(E,[0,\infty))$, using the martingale change of probability $\mathbb P_\nu$, we obtain
  \begin{eqnarray*}
    &&\mathbb P_\nu\left(\exp\{-\langle f, X_t\rangle \};\zeta>t\right)\\
    &=&\mathbb P_\nu\left(\dfrac{M_t}{M_t}\exp\{-\langle f, X_t\rangle \};\zeta>t\right)\\
    &=&\widetilde{\mathbb P}_\nu\left(\dfrac{1}{M_t}\exp\{-\langle f, X_t\rangle \}\right)\\
    &=&e^{\lambda t}\mathbb Q_{\nu}\left(\dfrac{\exp\Big\{-\langle f, X_t\rangle -\langle f,  Z^{\mathrm m, [0,t)}_t+ Z^{\mathrm n, [0,t)}_t\rangle\Big \}}{\langle\phi, X_t\rangle +\langle\phi,  Z^{\mathrm m, [0,t)}_t+ Z^{\mathrm n, [0,t)}_t\rangle }\right)\\
    &=&e^{\lambda t}\widehat{\mathbb Q}_{\nu}\left(\dfrac{\exp\Big\{-\langle f, X_t\rangle -\langle f,  \widehat Z^{\mathrm m}_t+ \widehat Z^{\mathrm n}_t\rangle\Big \}}{\langle\phi, X_t\rangle +\langle\phi,  \widehat Z^{\mathrm m}_t+ \widehat Z^{\mathrm n}_t\rangle }
        \right).
  \end{eqnarray*}
  Since $\lim_{t\rightarrow\infty}X_t=0$ in probability and $\widehat Z^{\mathrm m}_t+ \widehat Z^{\mathrm n}_t$ is increasing having almost sure limit
  $X_\infty\in\mathcal M(E)$, as $t\to\infty$. Therefore
  \[
    \lim_{t\rightarrow\infty}\widehat{\mathbb Q}_{\nu}\left(\dfrac{\exp\Big\{-\langle f, X_t\rangle -\langle f,  \widehat Z^{\mathrm m}_t+ \widehat Z^{\mathrm n}_t\rangle\Big \}}{\langle\phi, X_t\rangle +\langle\phi,  \widehat Z^{\mathrm m}_t+ \widehat Z^{\mathrm n}_t\rangle }
    \right)=\mathbf P\left(\frac{1}{\langle\phi, X_\infty\rangle }\exp\{-\langle f, X_{\infty}\rangle \}\right).
  \]
  Meanwhile we note that
  \[
    \lim_{t\rightarrow\infty}e^{-\lambda t}\mathbb P_\nu(\zeta>t)=k={\mathbf P} \left(\frac{1}{\langle\phi, X_\infty\rangle }\right).
  \]
  Thus
  \begin{eqnarray*}
    &&\lim_{t\rightarrow\infty}\mathbb P_\nu\left(\exp\{-\langle f, X_t\rangle \}\Big|\zeta>t\right)=\lim_{t\rightarrow\infty}\dfrac{\mathbb P_\nu\left(\exp\{-\langle f, X_t\rangle \};\zeta>t\right)}{\mathbb P_\mu(\zeta>t)}\\
    &&=\dfrac{\lim_{t\rightarrow\infty}\widehat{\mathbb Q}_{\nu}\left(\dfrac{\exp\Big\{-\langle f, X_t\rangle -\langle f,  \widehat Z^{\mathrm m}_t+ \widehat Z^{\mathrm n}_t\rangle\Big \}}{\langle\phi, X_t\rangle +\langle\phi,  \widehat Z^{\mathrm m}_t+ \widehat Z^{\mathrm n}_t\rangle }
       \right)}{\lim_{t\rightarrow\infty}e^{-\lambda t}\mathbb P_\mu(\zeta>t)}\\
    &&=\dfrac{\mathbf P\left(\dfrac{1}{\langle\phi, X_\infty\rangle }\exp\{-\langle f, X_{\infty}\rangle \}\right)}{{\mathbf P}\left(\dfrac{1}{\langle\phi, X_\infty\rangle }\right)}.
  \end{eqnarray*}
  Therefore the Yaglom distribution can be written as
  \[
    \mathbf P^{\lambda}(\cdot)=\dfrac{1}{k}{\mathbf P}\left(\dfrac{1}{\langle\phi, X_\infty\rangle }; X_\infty\in\cdot\right).
  \]
  Since $\mathbf P^{\lambda}$ is supported on $\mathcal S$, so is $X_\infty$. Define the random variable $M=\langle\phi,X_\infty\rangle$.  Then
  $X_\infty(dx)=M\widehat\phi(x)m(dx)$.  Therefore,
  \begin{equation}\label{eq:_ident_k}
    \mathbf P^{\lambda}(M^{(\lambda)})=\dfrac{1}{k}{\mathbf P}\left(\dfrac{eq:m}{M }\right)=\dfrac{1}{k}.
  \end{equation}
  And for any $\theta>0$,
  \[
    Ee^{-\theta M}={\mathbf P}\left(\dfrac{eq:m}{M }e^{-\theta M}\right)=k\mathbf P^{\lambda}(M^{(\lambda)}e^{-\theta M^{(\lambda)}})=\dfrac{E(M^{(\lambda)}e^{-\theta M^{(\lambda)}})}{EM^{(\lambda)}}.
  \]
  %% ------------------------------------------------------------------------------------------------------------------------------
  %% ----------------------------------------------------------------------------------------------------------------------------
  This says the invariant probability of the $Q$ process is a size-biased distribution of the Yaglom probability with weight function
  $\dfrac{M^{(\lambda)}}{EM^{(\lambda)}}$.


  When $\int_E\widehat\phi(x)l(x)m(dx)=\infty$, it is shown in theorem \ref{thm:distribution_of_zeta} that
  $\sum_{s\in\mathcal D^{\mathrm m}} \langle \phi,\widehat X^{{\mathrm m},s}_s\rangle =\infty$, $\widehat{\mathbb Q}_\nu$ almost surely. Thus
  \[
    \langle \phi, X_{\infty}\rangle =\infty,\qquad \mathbf P-{\mathrm a.s.}
  \]
  In this case, the $Q$ process does not have invariant probability. And $\langle \phi, X_t\rangle $ converges to $\infty$ as $t\to\infty$ in probability with respect to $\widetilde{\mathbb P}_\mu$ for any $\mu\in \mathcal M(E)\setminus\{0\}$.
\end{proof}
%%%%%%%%%%%%%%%%%%%%%%%%%%%%% ----------------------------------------------------------------------------------------------------------------
\begin{proof}[Proof of Proposition \ref{eq:exp_prop}]
  The results in this proposition are obtained from \eqref{eq:_ident_k} and the result of theorem \ref{thm:distribution_of_zeta}.
\end{proof}
%% --------------------------------------------------------------------------------------------------------------------------------------------
\begin{prop}\label{eq:inf_div}
  Let $L_{\mu, t}$ be the distribution of $X^D_t$ under the probability $\mathbb P_\mu$.  When $\int_D\widetilde\phi(x)m(x)dx<\infty$, given the path $\widehat Y$ of the spine under $\mathbb P_{\widetilde\phi,\phi}$,  $X^{\infty,D}$ has a infinitely divisible distribution on $M_F(D)$ with Levy measure $\mathcal N(\widehat{Y}, dw)=\int_0^\infty ds\int_0^\infty r n(\widehat{Y}_s, dr)L_{r\delta_{\widehat{Y}_s}, s}(dw)$.  In other words, for any $f\in\mathcal B_b^+(D)$,
  \[
    \mathbb P_{\widetilde\phi,\phi}\left[e^{-\langle f,X^{\infty,D}\rangle }\big|\widehat{Y}\right]=\exp\left\{-\int_{M_F(D)}(1-e^{-\langle f,\omega\rangle })\mathcal N(\widehat{Y}, d\omega)\right\}.
  \]
\end{prop}
\begin{proof}
  From the definition \eqref{laplace Q} of $X^{\infty,D}$ and \eqref{time rev}, the Laplace functional of $X^{\infty,D}$ is given by for any nonnegative Borel measurable function $f$ on $D$,
  \begin{eqnarray*}
    &&\mathbb P_{\widetilde\phi,\phi}\left[e^{-\langle f,X^{\infty,D}\rangle }\big|\widehat{Y}\right]=\mathbb P_{\widetilde\phi,\phi}\left[e^{-\sum_{s<\infty}\langle f,M_s\rangle }\big|\widehat{Y}\right]\\
    & =&\lim_{t\rightarrow\infty}\exp\left\{\int_0^tds\int_0^\infty \psi'(\widehat Y_s, U^s(0,\widehat Y_s))n(\widehat Y, dr)\right\}\\
    &=&\exp\left\{\int_0^\infty \psi'(\widehat Y_s, U^s(0,\widehat Y_s))ds\right\},\qquad \widehat\Pi_{\widetilde\phi\phi}^\phi{\mathrm -a.s.}
  \end{eqnarray*}
  Since $\psi'(x,\lambda)=\int_0^\infty r(1-e^{-\lambda r})n(x,dr)$,  so
  \begin{eqnarray*}
    \psi'(\widehat Y_s, U^s(0,\widehat Y_s))&=&\int_0^\infty r
                                                (1-e^{-rU^s(f)(0,\widehat Y_s)})n(\widehat Y_s,dr)\\
                                            &=&\int_0^\infty r\int_{M_F(D)}(1-e^{-\langle f,\omega\rangle })L_{r\delta_{\widehat{Y}_s}, s}(d\omega)n(\widehat Y_s,dr).
  \end{eqnarray*}
  Therefore,
  \begin{eqnarray*}
    &&\mathbb P_{\widetilde\phi,\phi}\left[e^{-\langle f,X^{\infty,D}\rangle }\big|\widehat{Y}\right]\\
    &=&\exp\left\{-\int_{M_F(D)}(1-e^{-\langle f,\omega\rangle })\int_0^\infty ds\int_0^\infty rL_{r\delta_{\widehat{Y}_s}, s}(d\omega)n(\widehat Y_s,dr)ds\right\}\\
    &=&\exp\left\{-\int_{M_F(D)}(1-e^{-\langle f,\omega\rangle })\mathcal N(\widehat{Y}, d\omega)\right\},\qquad \widehat\Pi_{\widetilde\phi\phi}^\phi{\mathrm -a.s.}
  \end{eqnarray*}
  Then our conclusions follows from \cite[Theorem 3.4.1]{Dawson1992Infinitely} for instance.\qed
\end{proof}
% ----- Bibliographic references-------------
\bibliographystyle{amsplain}
\bibliography{orggtd/bib.bib}
\begin{comment}
\begin{thebibliography}{10}

\bibitem{AthreyaNey1972Branching}
  K.~B. Athreya and P.~E. Ney, \emph{Branching processes}, Springer-Verlag, New
  York-Heidelberg, 1972.

\bibitem{ChampagnatRoelly2008Limit}
  N.~Champagnat and S.~R{\oe}lly, \emph{Limit theorems for conditioned multitype
    {D}awson-{W}atanabe processes and {F}eller diffusions}, Electron. J. Probab.
  \textbf{13} (2008), no.~25, 777--810.

\bibitem{ChampagnatVillemonais2018Convergence}
  N.~Champagnat and D.~Villemonais, \emph{Convergence of the {F}leming-{V}iot
    process toward the minimal quasi-stationary distribution}, arXiv:1810.06849,
  2018.

\bibitem{Dawson1992Infinitely}
  D.~A. Dawson, \emph{Infinitely divisible random measures and superprocesses},
  Stochastic analysis and related topics ({S}ilivri, 1990), Progr. Probab.,
  vol.~31, Birkh\"{a}user Boston, Boston, MA, 1992, pp.~1--129.

\bibitem{Dynkin1993Superprocesses}
  E.~B. Dynkin, \emph{Superprocesses and partial differential equations}, Ann.
  Probab. \textbf{21} (1993), no.~3, 1185--1262.

\bibitem{EnglanderKyprianou2004Local}
  J.~Engl\"{a}nder and A.~E. Kyprianou, \emph{Local extinction versus local
    exponential growth for spatial branching processes}, Ann. Probab. \textbf{32}
  (2004), no.~1A, 78--99.

\bibitem{Evans1993Two}
  S.~N. Evans, \emph{Two representations of a conditioned superprocess}, Proc.
  Roy. Soc. Edinburgh Sect. A \textbf{123} (1993), no.~5, 959--971.

\bibitem{Grey1974Asymptotic}
  D.~R. Grey, \emph{Asymptotic behaviour of continuous time, continuous
    state-space branching processes}, J. Appl. Probability \textbf{11} (1974),
  669--677.

\bibitem{HeathcoteSenetaVere-Jones1967Refinement}
  C.~R. Heathcote, E.~Seneta, and D.~Vere-Jones, \emph{A refinement of two
    theorems in the theory of branching processes}, Teor. Verojatnost. i
  Primenen. \textbf{12} (1967), 341--346.

\bibitem{KimSong2008Intrinsic}
  P.~Kim and R.~Song, \emph{Intrinsic ultracontractivity of non-symmetric
    diffusion semigroups in bounded domains}, Tohoku Math. J. (2) \textbf{60}
  (2008), no.~4, 527--547.

\bibitem{Lambert2001Arbres}
  A.~Lambert, \emph{Arbres, excursions et processus de {L}\'{e}vy completement
    asym\'{e}triques}, Ph.D. thesis, Universit\'{e} Pierre et Marie Curie-Paris
  VI, 2001.

\bibitem{Lambert2003Coalescence}
  \bysame, \emph{Coalescence times for the branching process}, Adv. in Appl.
  Probab. \textbf{35} (2003), no.~4, 1071--1089.

\bibitem{Lambert2007Quasistationary}
  \bysame, \emph{Quasi-stationary distributions and the continuous-state
    branching process conditioned to be never extinct}, Electron. J. Probab.
  \textbf{12} (2007), no.~14, 420--446.

\bibitem{Li2000Asymptotic}
  Z.~Li, \emph{Asymptotic behaviour of continuous time and state branching
    processes}, J. Austral. Math. Soc. Ser. A \textbf{68} (2000), no.~1, 68--84.

\bibitem{Li2011Measurevalued}
  \bysame, \emph{Measure-valued branching {eq:m}arkov processes}, Probability and
  its Applications (New York), Springer, Heidelberg, 2011.

\bibitem{LiuRenSong2009Log}
  R.~Liu, Y.-X. Ren, and R.~Song, \emph{{$L\log L$} criterion for a class of
    superdiffusions}, J. Appl. Probab. \textbf{46} (2009), no.~2, 479--496.

\bibitem{LyonsPemantlePeres1995Conceptual}
  R.~Lyons, R.~Pemantle, and Y.~Peres, \emph{Conceptual proofs of {$L\log L$}
    criteria for mean behavior of branching processes}, Ann. Probab. \textbf{23}
  (1995), no.~3, 1125--1138.

\bibitem{MeleardVillemonais2012Quasistationary}
  S.~M\'{e}l\'{e}ard and D.~Villemonais, \emph{Quasi-stationary distributions and
    population processes}, Probab. Surv. \textbf{9} (2012), 340--410.

\bibitem{Penisson2010Conditional}
  S.~P\'{e}nisson, \emph{Conditional limit theorems for multitype branching
    processes and illustration in epidemiological risk analysis}, Ph.D. thesis,
  Universit\"{a}t Potsdam; Universit\'{e} Paris Sud-Paris XI, 2010.

\bibitem{RenSongSun2019Spine}
  Y.-X. Ren, R.~Song, and Z.~Sun, \emph{Spine decompositions and limit theorems
    for a class of critical superprocesses}, arXiv:1711.09188, 2017.

\bibitem{RenSongZhang2015Limit}
  Y.-X. Ren, R.~Song, and R.~Zhang, \emph{Limit theorems for some critical
    superprocesses}, Illinois J. Math. \textbf{59} (2015), no.~1, 235--276.

\bibitem{RenSongZhang2017Central}
  \bysame, \emph{Central limit theorems for supercritical branching nonsymmetric
    {M}arkov processes}, Ann. Probab. \textbf{45} (2017), no.~1, 564--623.

\bibitem{RenSongZhang2018Williams}
  \bysame, \emph{Williams decomposition for superprocesses}, Electron. J. Probab.
  \textbf{23} (2018), no.~23, 33 pp.

\bibitem{RoellyRouault1989Processus}
  S.~Roelly and A.~Rouault, \emph{Processus de {D}awson-{W}atanabe
    conditionn\'{e} par le futur lointain}, C. R. Acad. Sci. Paris S\'{e}r. I
  Math. \textbf{309} (1989), no.~14, 867--872.

\bibitem{Schaefer1974Banach}
  H.~H. Schaefer, \emph{Banach lattices and positive operators}, Springer-Verlag,
  New York-Heidelberg, 1974.

\end{thebibliography}
\end{comment}

% ! Below are some references which I'm not sure if is needed or not.
\begin{comment}
  \begin{thebibliography} {10}

  \bibitem{AthreyaNey1972Branching}
    Athreya, K. B. and Ney, P. E.:
    \emph{Branching processes.}
    Die Grundlehren der mathematischen Wissenschaften, Band 196. Springer-Verlag, New York-Heidelberg, 1972. xi+287 pp.
    \MR{0373040}

  \bibitem{BigginsKyprianou2004Measure}
    Biggins, J. D. and Kyprianou, A. E.:
    \emph{Measure change in multitype branching.}
    Adv. in Appl. Probab. \textbf{36} (2004), no. 2, 544--581.
    \MR{2058149}

  \bibitem{ChampagnatRoelly2008Limit}
    Champagnat, N. and Roelly, S.:
    \emph{Limit theorems for conditioned multitype Dawson-Watanabe processes and Feller diffusions.}
    Electron. J. Probab. \textbf{13} (2008), no. 25, 777–810.
    \MR{2399296}

  \bibitem{ChampagnatVillemonais2018Convergence}
    {\color{blue}Champagnat, N. and Villemonais, D.:
      \emph{Convergence of the Fleming-Viot process toward
        theminimal quasi-stationary distribution.}
      https://arxiv.org/pdf/1810.06849.pdf}

  \bibitem{ChenRenYang2017Skeleton}
    Chen, Z.-Q., Ren, Y.-X. and Yang, T.:
    \emph{Skeleton decomposition and law of large numbers for supercritical superprocesses.}
    Acta Appl. Math. (2017), 1--61.
    \ARXIV{1709.00847}

  \bibitem{Dawson1992Infinitely}
    Dawson, D. A.:
    \emph{Infinitely divisible random measures and superprocesses.}Stochastic analysis and related topics (Silivri, 1990), 1--129,
    Progr. Probab., 31, Birkh{\"a}user Boston, Boston, MA, 1992.
    \MR{1203373}

  \bibitem{DelmasHenard2013A-Williams}
    Delmas, J.-F. and H\'enard, O.:
    \emph{A Williams decomposition for spatially dependent super-processes. }
    Electron. J. Probab. \textbf{18} (2013), no. 37, 43 pp.
    \MR{3035765}

  \bibitem{Dynkin1993Superprocesses}
    Dynkin, E. B.:
    \emph{Superprocesses and partial differential equations.}
    Ann. Probab. \textbf{21} (1993), no. 3, 1185--1262.
    \MR{1235414}

  \bibitem{EnglanderKyprianou2004Local}
    Engl\"ander, J. and Kyprianou, A. E.:
    \emph{Local extinction versus local exponential growth for spatial branching processes.}
    Ann. Probab. \textbf{32} (2004), no. 1A, 78--99.
    \MR{2040776}

  \bibitem{Evans1993Two}
    Evans, S. N.:
    \emph{Two representations of a conditioned superprocess.}
    Proc. Roy. Soc. Edinburgh Sect. A \textbf{123} (1993), no. 5, 959--971.
    \MR{1249698}

  \bibitem{Grey1974Asymptotic}
    Grey, D. R.:
    \emph{Asymptotic behaviour of continuous time, continuous state-space branching processes.}
    J. Appl. Probability \textbf{11} (1974), 669--677.
    \MR{0408016}

  \bibitem{HeathcoteSenetaVere-Jones1967Refinement}
    Heathcote, C. R., Seneta, E. and Vere-Jones, D.:
    \emph{A refinement of two theorems in the theory of branching processes.} (Russian summary)
    Teor. Verojatnost. i Primenen. \textbf{12} 1967 341--346.
    \MR{0217889}

  \bibitem{KimSong2008Intrinsic}
    Kim, P. and Song, R.:
    \emph{Intrinsic ultracontractivity of non-symmetric diffusion semigroups in bounded domains.}
    Tohoku Math. J. (2) \textbf{60} (2008), no. 4, 527--547.
    \MR{2487824}

  \bibitem{KimSong2008Intrinsic2}
    Kim, P. and Song, R.:
    \emph{Intrinsic ultracontractivity of nonsymmetric diffusions with measure-valued drifts and potentials.}
    Ann. Probab. \textbf{36} (2008), no. 5, 1904–1945.
    \MR{2440927}

  \bibitem{KimSong2009Intrinsic}
    Kim, P. and Song, R.:
    \emph{Intrinsic ultracontractivity for non-symmetric Lévy processes.}
    Forum Math. \textbf{21} (2009), no. 1, 43–66.
    \MR{2494884}

  \bibitem{Lambert2001Arbres}
    Lambert, A.:
    \emph{Arbres, excursions et processus de L\'evy completement asym\'etriques.}
    Diss. Université Pierre et Marie Curie-Paris VI, 2001.

  \bibitem{Lambert2003Coalescence}
    Lambert, A.:
    \emph{Coalescence times for the branching process.}
    Adv. in Appl. Probab. \textbf{35} (2003), no. 4, 1071--1089.
    \MR{2014270}

  \bibitem{Lambert2007Quasistationary}
    Lambert, A.:
    \emph{Quasi-stationary distributions and the continuous-state branching process conditioned to be never extinct.}
    Electron. J. Probab. \textbf{12} (2007), no. 14, 420--446.
    \MR{2299923}

  \bibitem{Li2000Asymptotic}
    Li, Z.-H.:
    \emph{Asymptotic behaviour of continuous time and state branching processes.}
    J. Austral. Math. Soc. Ser. A \textbf{68} (2000), no. 1, 68--84.
    \MR{1727226}

  \bibitem{Li2011Measurevalued}
    Li, Z.:
    \emph{Measure-valued branching Markov processes.}
    Probability and its Applications (New York). Springer, Heidelberg, 2011. xii+350 pp. ISBN: 978-3-642-15003-6
    \MR{2760602}

  \bibitem{LiuRenSong2009Log}
    Liu, R.-L., Ren, Y.-X. and Song, R.:
    \emph{{$L \log L$} criterion for a class of superdiffusions.}
    J. Appl. Probab. \textbf{46} (2009), no. 2, 479–496.
    \MR{2535827}

  \bibitem{LyonsPemantlePeres1995Conceptual}
    Lyons, R., Pemantle, R. and Peres, Y.:
    \emph{Conceptual proofs of $L\log L$ criteria for mean behavior of branching processes.}
    Ann. Probab. \textbf{23} (1995), no. 3, 1125--1138.
    \MR{1349164}

  \bibitem{MeleardVillemonais2012Quasistationary}
    M\'el\'eard, S. and Villemonais, D.:
    \emph{Quasi-stationary distributions and population processes.}
    Probab. Surv. \textbf{9} (2012), 340–410.
    \MR{2994898}

  \bibitem{Nagasawa1964Time}
    Nagasawa, M.:
    \emph{Time reversions of Markov processes.}
    Nagoya Math. J. \textbf{24} (1964), 177--204.
    \MR{0169290}

  \bibitem{RenSongSun2019Spine}
    Ren, Y.-X., Song, R. and Sun, Z.:
    \emph{Spine decompositions and limit theorems for a class of critical superprocesses.}
    Preprint.
    \ARXIV{1711.09188}

  \bibitem{RenSongYang2016Spine}
    Ren, Y.-X., Song, R. and Yang, T.:
    \emph{Spine decomposition and {$ L\log L $} criterion for superprocesses with non-local branching mechanisms.}
    Preprint.
    \ARXIV{1609.02257}

  \bibitem{RenSongZhang2015Limit}
    Ren, Y.-X., Song, R. and Zhang, R.:
    \emph{Limit theorems for some critical superprocesses.}
    Illinois J. Math. \textbf{59} (2015), no. 1, 235–276.
    \MR{3459635}

  \bibitem{RenSongZhang2017Central}
    Ren, Y.-X., Song, R. and Zhang, R.:
    \emph{Central limit theorems for supercritical branching nonsymmetric Markov processes.}
    Ann. Probab. \textbf{45} (2017), no. 1, 564–623.
    \MR{3601657}

  \bibitem{RenSongZhang2018Williams}
    Ren, Y.-X., Song, R. and Zhang, R.:
    \emph{Williams decomposition for superprocesses.}
    Electron. J. Probab. \textbf{23} (2018), Paper No. 23, 33 pp.
    \MR{3771760}

  \bibitem{RoellyRouault1989Processus}
    Roelly, S. and Rouault, A.:
    \emph{Processus de Dawson-Watanabe conditionn\'e par le futur lointain.} (French. English summary) [A Dawson-Watanabe process conditioned by the remote future]
    C. R. Acad. Sci. Paris Sér. I Math. \textbf{309} (1989), no. 14, 867--872.
    \MR{1055211}

  \bibitem{Penisson2010Conditional}
    P\'enisson, S.:
    \emph{Conditional limit theorems for multitype branching processes and illustration in epidemiological risk analysis.}Diss. Universit?t Potsdam, Université Paris Sud-Paris XI, 2010.

  \bibitem{Schaefer1974Banach}
    Schaefer, H. H.:
    \emph{Banach lattices and positive operators.}
    Die Grundlehren der mathematischen Wissenschaften, Band 215. Springer-Verlag, New York-Heidelberg, 1974.
    \MR{0423039}

  \bibitem{Yaglom1947Certain}
    Yaglom, A. M.:
    \emph{Certain limit theorems of the theory of branching random processes.} (Russian)
    Doklady Akad. Nauk SSSR (N.S.) \textbf{56} (1947), 795--798.
    \MR{0022045}

  \end{thebibliography}
\end{comment}
\end{document}
