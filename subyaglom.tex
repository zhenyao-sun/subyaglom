% * Commit history
% ** 2018.12.24 the extinction rate of subcritical superprocesses - 3.tex by R. Liu: Original ideas and proofs.
% ** 2019.01.01 subSuper0101.tex by Z. Sun: changed the tex file into amsart style.
% ** 2019.01.07 subSuper0107.tex by Z. Sun: recommended some more general settings.
% ** 2019.01.21 subSuper0121.tex by R. Liu
% ** 2019.01.28 subSuper0128.tex by Z. Sun: gave more details about the reverse spine decomposition.
% ** 2019.01.31 subSuper0131.tex by R. Liu
% ** 2019.02.20 subSuper0220.tex by Z. Sun: there is a problem in Section 3. 
% * The preamble
\documentclass[12pt,a4paper]{amsart}
\setlength{\textwidth}{\paperwidth}
\addtolength{\textwidth}{-2in}
\calclayout
\usepackage[utf8]{inputenc}
\usepackage[T1]{fontenc}
\usepackage{mathtools}
\mathtoolsset{showonlyrefs}
\usepackage{stackrel}
\usepackage{mathrsfs}
\usepackage[backref]{hyperref}
\usepackage{comment}
\usepackage{xcolor}
%\usepackage[nobysame]{amsrefs}
\usepackage{amsthm}
\theoremstyle{plain}
\newtheorem{thm}{Theorem}[section]
\newtheorem{lem}[thm]{Lemma}
\newtheorem{prop}[thm]{Proposition}
\newtheorem{cor}[thm]{Corollaray}
\newtheorem{conj}[thm]{Conjecture}
\theoremstyle{definition}
\newtheorem{defi}[thm]{Definition}
\newtheorem{rem}[thm]{Remark}
\newtheorem{exa}[thm]{Example}
\newtheorem{asp}{Assumption}
\numberwithin{equation}{section}
\allowdisplaybreaks
%\def\MR#1{\href{http://www.ams.org/mathscinet-getitem?mr=#1}{MR#1}}
%\def\ARXIV#1{\href{https://arxiv.org/abs/#1}{arXiv:#1}}
% * Top matter
\begin{document}
\title
    [Subcritical super-diffusions]
    {\large The extinction probability for a class of subcritical super-diffusions}
\author
    [R. Liu, Y.-X. Ren and Z. Sun]
    { Rongli Liu, Yan-Xia Ren and Zhenyao Sun}
\address
    {Yan-Xia Ren\\
    School of Mathematical Sciences\\
    Peking University\\
    Beijing, P. R. China, 100871}
\email{yxren@math.pku.edu.cn}
\thanks{The research of Yan-Xia Ren is supported in part by NSFC (Grant Nos. 11671017  and 11731009).}
\address
   {Rongli Liu\\
   {\bf Information about Rongli Liu}}
   \email{rlliu@bjtu.edu.cn }
   \thanks{ The research of Rongli Liu is supported in part by NSFC
(Grant No. 11301261), and the Fundamental Research Funds for the Central Universities (Grant No.  2017RC007)}
\address
    {Zhenyao Sun\\
    School of Mathematical Sciences\\
    Peking University\\
    Beijing, P. R. China, 100871}
\email{zhenyao.sun@pku.edu.cn}
\begin{abstract}
	In this paper, we discussed the decay rate of the extinction probability of a class of superdiffusions. Two different conditional limits, the Yaglom probability and the Q process, are considered.  We characterize the Yaglom distribution by some function $G(f)$, which satisfies $G(v^t(0,\cdot))=t$ for each $t>0$ and where $v^t(0,x)=-\log(\mathbb P_x(\zeta<t))$.  It is proved that the Yaglom distribution is the minimal quasi-stationary distribution.  As for the Q process, we show that under finite LlogL moment assumptions, the Q process has the equilibrium distribution.  And it is a size-biased Yaglom distribution. Moreover, we point out that under some conditional distribution the equilibrium  distribution is an infinitely divisible distribution.
\end{abstract}
\maketitle
% * Document body
% * Introduction
\section{Introduction}
% ** Backgroud
\subsection{Backgroud}
Suppose $(Z_n, n\ge 1)$ is a Galton-Watson branching process with offspring
distribution $\{p_n\}$. Let $m:=\sum^{\infty}_{n=1}np_n$ be the mean number of
children per particle. It is well known that when $m<1$ the extinction
probability $q:=\lim_{n\rightarrow\infty}P\left(Z_n=0\right)$ is equal to $1$.
That is to say the process $(Z_n)$ is extinct in finite time almost surely. In
this case, the non-extinction probability $P(Z_n>0)$ decays to $0$. A natural
question is what the right decay rate of the probability is. In 1967,
Heathcot,Seneta and Vere-Jone \cite{HeathcoteSenetaVere-Jones1967A-refinement}
answered the question, giving an LlogL criteria. Let $L$ stand for a random
variable with distribution $\{p_n\}$.

\begin{thm}[Heathcote, Seneta and Vere-Jones (1967)]
  The sequence $\{ P(Z_n>0)/m^n\}$ is decreasing. If $m<1$, then the following
  are equivalent:
  \begin{itemize}
  \item[$(i).$] $\lim_{n\rightarrow\infty}P(Z_n>0)/m^n>0$,
  \item[$(ii).$] $\sup E[Z_n|Z_n>0]<\infty$,
  \item[$(iii).$] $E\left[L\log^+ L\right]<\infty$.
  \end{itemize}
\end{thm}
In 1995, Lyons, Pemantle and Peres developed a martingale change of measure
method in \cite{LyonsPemantlePeres1995Conceptual} to give a new proof for this
$L\log L$ theorem.

In \cite{Lambert2001Arbres,Lambert2003Coalescence}, Lambert discusses the
similar equivalencies for continuous time continuous state branching process
$(Z_t)$. Its branching mechanism function $\psi$ is specified by the
L\'evy-Khinchin formula
\[
	\psi(\lambda)=-\beta\lambda+\sigma^2\lambda^2+\int_0^\infty \left(e^{-\lambda r}-1+\lambda r\right)n(dr),
\]
where $n(dr)$ is a positive measure on $(0,\infty)$ such that $\int_0^\infty
(r^2\wedge r) n(dr)<\infty$ and where
\begin{equation}\label{extinc assump  for continuous}
	\int^\infty\frac{1}{\psi(\lambda)}d\lambda<\infty.
\end{equation}
We denote by $\mathbb P_x$ the law of $(Z_t)$ with initial value $Z_0=x>0$. Then
for any $\lambda, t\geq 0$,
\[
	\mathbb P_x\left(\exp\{-\lambda Z_t\}\right)=\exp(-xu_t(\lambda)),
\]
where $u_t(\lambda)$ is the unique solution to the following equation
\begin{equation}
\begin{cases}
	\dfrac{d u_t(\lambda)}{dt}=-\psi(u_t(\lambda));\\
u_0(\lambda)=\lambda.
\end{cases}
\end{equation}
Let $\zeta:=\inf\{t\geq 0; Z_t=0\}$ be the extinction time of $Z$. There is a
nonnegative function $\varphi(t)$ such that for any $x>0$,
\[
	\mathbb P_x(Z_t>0)=\mathbb P_x(\zeta>t)=1-\exp(-x\varphi(t)), \qquad t>0.
\]
When $\beta<0$ (c.f.\cite{Grey1974Asymptotic}),
\[
	\mathbb P_x(\zeta<\infty)=1.
\]
Thus the decay rate of $\mathbb P_x(\zeta>t)$ is determined by the decay rate of
$\varphi(t)$. It is shown (c.f.\cite{Lambert2007Quasi-stationary}) that for any
$\lambda>0$,
\[
	\lim_{t\rightarrow\infty}\frac{u_t(\lambda)}{\varphi(t)}=G(\lambda),
\]
where $G(\lambda)=\exp(\beta\int_{\lambda}^\infty1/\psi(u)du)$. Moreover, the
following LlogL criteria holds.
\begin{thm}
  \label{equivalent for cbp}
	When $\beta<0$, the following items are equivalent
\begin{itemize}
\item[$(i).$] $G'(0+)<\infty$;
\item[$(ii).$] There is a positive constant $c$ such that $\varphi(t)\sim
  c\exp(\beta t)$;
\item[$(iii).$] $\int_1^\infty r\log r n(dr)<\infty$.
\end{itemize}
In this case, $c^{-1}=G'(0+)$.
\end{thm}
As far as the two type dynamics running over large amounts of time are
concerned, special attentions are given to two conditional limits. One is called
Yaglom distribution and another one is called $Q$ process. Here we take the
continuous state branching process $Z$ for example to give the definitions of
the two limits. We can find the corresponding investigations for GW branching
processes in \cite{AthreyaNey1972Branching}.

The Yaglom distribution $\nu_{\beta}$ is the limit probability that for any
$x>0$ and Borel set $A$,
\[
	\lim_{t\rightarrow\infty}\mathbb P_x(Z_t\in A\big|\zeta>t)=\nu_{\beta}(A).
\]
The conditional convergence can be found in \cite{Li2000Asymptotic} where it is
also generalized to conditioning of the type $\{\zeta>t+r\}$ for any finite
$r>0$ instead of $\{\zeta>t\}$. A quasi-stationary distribution (QSD) is a
subinvariant distribution $\nu$ on $(0,\infty)$ satisfying
\[
	\mathbb P_{\nu}(Z_t\in A|\zeta>t)=\nu(A).
\]
It is shown in \cite{Lambert2007Quasi-stationary} that the Yaglom distribution
$\nu_{\beta}$ is the minimal QSD of $Z$ in the following sense. For any
$\gamma\in[\beta,0)$, there is a unique QSD $\nu_{\gamma}$ associated to the
rate of mass decay $-\gamma$, and there is no QSD associated to $\gamma<\beta$.


Let $\mathcal F_t=\sigma(Z_s,s\leq t), t\geq 0$ be the natural
$\sigma-$filtration generated by $Z$. Then due to
\cite{RenSongZhang2018Williams}, another conditional limit $\widetilde{\mathbb
  P}_x$ is defined in the sense that for any $x,t>0$ and $A\in\mathcal F_t$,
\[
	\lim_{s\rightarrow\infty}\mathbb P_x(A\big|\zeta>s)=\widetilde{\mathbb P}_x(A).
\]
It can be expressed as an $h-$transform of $\mathbb P_x$ with martingale
$M_t=Z_te^{-\beta t}$ that
\[
	\left.\dfrac{d\widetilde{\mathbb P}_x}{d\mathbb P_x}\right|_{\mathcal F_t}=\frac{M_t}{x}.
\]
Moreover, under $\widetilde{\mathbb P}_x$, the process $Z$ is a branching
process with immigration called $Q$-process. Lambert established the connection
between these two limit distributions in \cite{Lambert2007Quasi-stationary}. Let
$\Upsilon$ be a random variable whose distribution is the Yaglom distribution
$\nu_\beta$. If $\int_1^\infty r\log r n(dr)<\infty$, then under
$\widetilde{\mathbb P}_x$, $Z_t$ converges in distribution to a positive random
variable $Z_\infty$ as $t\to\infty$, which has the distribution of the
size-biased Yaglom distribution
\[
	\widetilde{\mathbb P}_x(Z_\infty\in dr)=\frac{r}{\mathbb E\Upsilon}\mathbb P(\Upsilon\in dr).
\]
The studies on the Q processes,the Yaglom distributions and the quasi-stationary
distributions for more models we refer to the survey
\cite{MeleardVillemonais2012Quasi-stationary}, the thesis
\cite{Penisson2010Conditional} and the references therein.

Superdiffusions are a class of branching processes with spatial motions. Similar
to branching processes we introduced above, $0$ is also their absorbing state.
In this paper, we will discuss a class of superdiffusions which will be extinct
in finite time. We establish an $L\log L$ criterion taking use of the spine
method and investigate the Yaglom distributions and $Q$ processes. Moreover, we
study the QSD for our model as well. To state our main results, we need to
introduce the setup we are going to work with first.

% ** Main results
\subsection{Main results}
\label{sec: main results}
% *** Model
% **** Notations before the definition of superprocesses
Let $E$ be a Locally compact separable metric space. Let \emph{spatial motion}
$Y=\{(Y_t)_{t\geq 0};(\Pi_x)_{x\in E}\}$ be an $E$-valued Hunt process with its
lifetime denoted by $\tau$ and its transition semigroup denoted by $(P_t)_{t\geq
  0}$. Let the \emph{branching mechanism} $\psi$ be defined as a function on
$E\times[0,\infty)$ by
\[
	\psi(x,z)
	=-\beta(x)z + \alpha(x)^2 z^2+ \int_{(0,\infty)} (e^{-rz}-1+zr )n(x, dr),\qquad x\in E, z\geq0,
\]
where $\beta, \alpha\in \mathcal B_b(E)$ and $n$ is a $\sigma$-finite kernel
from $E$ to $(0,\infty)$ with
\[
	\sup_{x\in E}\int_0^\infty (r\wedge r^2)n(x,dr)
	<\infty.
\]
Denote by $\mathcal B_b(E)$ the collection of all bounded measurable function on
$E$. Denote by $\mathcal B^+_b(E)$the collection of all non-negative bounded
measurable functions on $E$. Denote by $\mathcal M(E)$ the space of all finite
measures on $E$ equipped with the weak topology. Write $\langle f,\mu\rangle$ or
$\mu(f)$ for the integral $\int_E f(x)\mu(dx)$ for any $f\in\mathcal B(E)$ and
$\mu\in \mathcal M(E)$.

% **** Definition of superprocesses
In this paper, we consider a \emph{$(Y,\psi)$-superprocess} $X$ which is defined
as an $\mathcal M(E)$-valued Hunt process $X=\{(X_t)_{t\geq 0}; (\mathbb
P_\mu)_{\mu \in \mathcal M(E)}\}$ satisfying that for any $\mu \in \mathcal
M(E), f\in \mathcal B^+_b(E)$ ( see\cite{Dynkin1993Superprocesses}),
\begin{equation} \label{eq: def of vtf}
    \mathbb P_\mu [e^{-\langle f,X_t\rangle}] = e^{-\langle V_tf, \mu\rangle},
    \quad t\geq 0,
\end{equation}
where, for each $f\in\mathcal B^+_b(E)$, function $(t,x) \mapsto V_tf(x)$ on
$[0,\infty) \times E$ is the unique locally bounded positive solution to the
equation
\begin{equation}\label{eq:FKPP_in_definition}
    V_t f(x) +   \Pi_x\Big[\int_0^{t\wedge \tau} \psi \big(\xi_s,V_{t-s} f(\xi_s)\big) ds\Big]
	= P_t f(x),
	\quad x \in E,\,\, t \geq 0.
\end{equation}


% *** Assumptionse
% **** Assumption of spatial mostion
% ***** Notations
To simplify the notation, we also write $\mathbb P_x$ for $\mathbb
P_{\delta_x}$. Define the \emph{Feynman-Kac semigroup} $(P^\beta_t)_{t\geq 0}$
such that for any $ f\in \mathcal B_b(E)$,
\begin{align}
	P^\beta_tf(x)
	:= \Pi_x \big[e^{\int_0^{t} \beta(Y_r)dr} f(Y_t)\mathbf 1_{\{t<\tau\}}\big],
	\quad t\geq 0,\,\, x\in E.
\end{align}
It is known, see \cite[Proposition 2.27]{Li2011Measure-valued} for example,
$(P^\beta_t)$ is \emph{the mean semigroup} of the superprocess $X$, in the sense
that for any $\mu \in \mathcal M(E),f \in \mathcal B_b(E)$,
\begin{align} \label{eq: Ygalom type result without 2rd moment} \mathbb P_\mu
  [\langle f,X_t\rangle] = \mu(P^\beta_t f), \quad t \geq 0.
\end{align}
% ***** Statements
For the spatial motion $Y$, we always assume the following:
\begin{asp}
\label{asp: 1}
There exist a $\sigma$-finite measure $m$ with full support on $E$ and a family
of strictly positive, bounded continuous functions $\{ p_t(\cdot,\cdot): t > 0
\}$ on $E \times E$ such that
\begin{align}
  	P_tf(x)
  	= \int_E p_t(x,y) f(y) m(dy),
  	&\quad t>0, x \in E,f \in \mathcal B_b(E);
  	\\ \int_E p_t(x,y)m(dx)\leq 1, &\quad t>0,y\in E;
  	\\ \int_E \int_E p_t(x,y)^2 m(dx) m(dy)
  	<\infty,
  	&\quad t> 0;
\end{align}
and the functions $x \mapsto \int_E p_t(x,y)^2 m(dy)$ and $y \mapsto \int_E
p_t(x,y)^2 m(dx)$ are both continuous.
\end{asp}
% ***** Remarks
Under Assumption \ref{asp: 1}, it is proved in
\cite{RenSongZhang2015Limit,RenSongZhang2017Central} that there exists a family
of strictly positive, bounded continuous functions $\{ p^\beta_t(\cdot,\cdot): t
> 0 \}$ on $E \times E$ such that
\begin{align}
	P^\beta_t f(x)
	= \int_E p_t^\beta (x,y) f(y) m(dy),
	\quad \quad t>0, x \in E,f \in \mathcal B_b(E).
\end{align}
Define the dual semigroup $(\widehat P^{\beta}_t)_{t \geq 0}$ by
\begin{align}
	\widehat P^{\beta}_0 = I;
	\quad \widehat P^{\beta}_t f(y)
	:= \int_E p^\beta_t (x,y) f(x) m(dx),
	\quad t>0, y\in E, f\in \mathcal B_b(E).
\end{align}
It is shown in \cite{RenSongZhang2015Limit, RenSongZhang2017Central} that both
$(P^\beta_t)_{t \geq 0}$ and $(\widehat P_t^{\beta})_{t\geq 0}$ are strongly
continuous semigroups of compact operators on $L^2(E,m)$. Let $A$ and $\widehat
A$ be the generators of the semigroups $(P^\beta_t)_{t \geq 0}$ and $(\widehat
P^\beta_t)_{t \geq 0}$, respectively. Denote by $\sigma(A)$ and $\sigma(\widehat
A)$ the spectra of $A$ and $\widehat A$, respectively. According to
\cite[Theorem V.6.6]{Schaefer1974Banach}, $\lambda := \sup \text{Re}(\sigma(A))
= \sup \text{Re}(\sigma(\widehat A))$ is a common eigenvalue of multiplicity $1$
for both $A$ and $\widehat A$. It is also proved in
\cite{RenSongZhang2015Limit,RenSongZhang2017Central} that the eigenfunctions
$\phi$ of $A$ and $\hat\phi$ of $\widehat A$ associated with the common
eigenvalue $\lambda$ can be chosen to be strictly positive and continuous
everywhere on $E$. Normalize $\phi$ and $\hat\phi$ by
\[	
	\int_E \phi(x)^2 m(dx) = \int_E \phi(x) \hat \phi(x) m(dx) = 1.
\]
so that they are unique.

% **** Assumtion of mean behavior
% ***** Notations
Notice that, for each $t \geq 0$ and $\mu \in \mathcal M(E)$, we have $ \mathbb
P_\mu[X_t(\phi)] = \mu(P^\beta_t \phi) = e^{\lambda t} \mu(\phi). $ If $\lambda
> 0$, the mean of $X_t(\phi)$ will increase exponentially; if $\lambda < 0$, the
mean of $X_t(\phi)$ will decrease exponentially; and if $\lambda = 0$, the mean
of $X_t(\phi)$ will be a constant. Therefore, we say $X$ is \emph{supercritical,
  critical} or \emph{subcritical}, according to $\lambda > 0$, $\lambda = 0$ or
$\lambda < 0$, respectively.

% ***** Statements
Throughout this paper, we assume the following for the mean semigroup
$(P_t^\beta)$ in addition:
\begin{asp}
\label{asp:IU}
\begin{enumerate}
\item The superprocess $X$ is subcritical, i.e., $\lambda < 0$.
\item The eigenfunctions $\phi$ and $\hat\phi$ are bounded on $E$.
\item The mean semigroup $(P_t^\beta)$ is \emph{intrinsically ultracontractive},
  that is, for each $t>0$, there is a constant $c_t >0$ such that for each
  $x,y\in E$, $p^\beta_t(x,y) \leq c_t \phi(x) \hat\phi(y)$.
\end{enumerate}
\end{asp}

% **** Assumption of non-presistent
% ***** some words before the statement
Let $\zeta=\inf\{t>0: \langle \mathbf 1_E,X_t\rangle=0\}$ be the extinction time
of the superprocess $X$. In the cases of $\lambda\leq 0,$ it is proved by
\cite{EnglanderKyprianou2004Local} that $\lim_{t\rightarrow\infty} \langle
\mathbf 1_E,X_t\rangle=0 $ in probability. ({\bf ZS: Do we necessarily have to
  cite \cite{EnglanderKyprianou2004Local} here? I think
  \cite{EnglanderKyprianou2004Local} only considered the homogenious branching
  superdifussions, and it focused on the local extinction. It is hard to see
  where it proved this statement. If in our prove we don't use the result that
  $\langle \mathbf 1_E,X_t\rangle=0$, then we don't have to cite this paper
  here.}) {\color{red} I prefer to keeping this result here.} But this is not
equivalent to
\begin{equation}\label{p1extinc}
	\mathbb P_\mu(\zeta<\infty)=1.
\end{equation}
If the branching mechanism $\psi$ is independent of the location $x$, then
\eqref{p1extinc} holds provided condition \eqref{extinc assump  for continuous}
is true. When the branching mechanism depends on the location, similar to
\cite{RenSongSun2017Spine,RenSongZhang2018Williams}, we add the following
assumption to assure the process will be extinct in finite time almost surely.

% ***** Statement
\begin{asp} \label{asp: 3}
\begin{enumerate}
\item \label{subasp: point non-presistence}
	% $\mathbb P_{x}(\zeta < t)>0$ for each $x\in E$ and $t>0$.
	$\mathbb P_{x}(\zeta \leq t)>0$ for each $x\in E$ and $t>0$.
\item \label{subasp: measure non-presistence}
  % $\mathbb P_{\nu}(\zeta< t)>0$ for some $t>0$ with $\nu(dx):=\hat\phi(x)m(dx)$.
  $\mathbb P_{\nu}(\zeta\leq t)>0$ for some $t>0$ with $\nu(dx):=\hat\phi(x)m(dx)$.
\end{enumerate}
\end{asp}


% *** Results
Define a new kernel $n^\phi(x, dr)$ from $E$ to $(0,\infty)$ such that for any $f\in\mathcal B_b((0,\infty))$,
\begin{equation} \label{phi-change}
	\int_0^\infty f(r)n^\phi(x,dr)=\int_0^\infty f(r\phi(x))n(x, dr),
	\quad x\in E.
\end{equation}
Our first result is about the asymptotic behavior of the extinction time $\zeta$:

\begin{thm}\label{thm: distribution of zeta}
	Suppose that the superprocess $X$ satisfies Assumptions \ref{asp: 1},\ref{asp:IU},\ref{asp: 3}.
	Then,
\begin{enumerate}
\item \label{subthm: extinct as sure}
	for each $\mu \in \mathcal M(E)$, we have  $\mathbb P_\mu(\zeta<\infty)=1$;
\item
 	for each $\mu,\tilde\mu\in \mathcal M(E)\setminus\{0\}$ and $s>0$, we have
 \[
 	\lim_{t\rightarrow\infty}\dfrac{\mathbb P_{\mu}(\zeta>t+s)}{\mathbb P_{\tilde\mu}(\zeta>t)}=\frac{\langle \phi,\mu\rangle }{\langle \phi,\tilde\mu\rangle }e^{\lambda s};
 \]
 \item
 	there exists a constant $k\in [0,\infty)$, such that for any $x\in E$,
\begin{equation}\label{decay rate}
	\lim_{t\rightarrow\infty} e^{-\lambda t}\mathbb P_x(\zeta>t)=k\phi(x).
\end{equation}
	Moreover, the constant $k>0$ if and only if $\int_E \hat\phi(y)l(y)m(dy)<\infty$ where
\begin{equation}\label{m}
	l(y):=\int_1^\infty r\log r~n^\phi(y, dr),\quad y \in E.
\end{equation}
\end{enumerate}
\end{thm}

	In particular, for any $x,y\in E$ and $s\geq 0$, the second result in the above theorem is written as
\begin{equation}\label{ratioresult}
 	\lim_{t\rightarrow\infty}\frac{\mathbb P_x(\zeta>t+s)}{\mathbb P_y(\zeta>t)}=\frac{\phi(x)}{\phi(y)}e^{\lambda s}.
\end{equation}
	So we can see that the effect of the position of the initial mass on the decay of the mass is a ratio of $\phi(\cdot)$ generally.


	For each probability ${\mathbf P}$ on $\mathcal M(E)$, we define
\[
	( {\mathbf P} \mathbb P)(\cdot) := \int_{\mathcal M(E)} \mathbb P_\mu(\cdot) {\mathbf P}(d\mu).
\]
	Then $\{(X_t)_{t\geq 0}; ({\mathbf P}\mathbb P)\}$ can be considered as a $(Y, \psi)$-superprocess with a random initial value $X_0$ whose distribution is ${\mathbf P}$.
	We say a probability ${\mathbf P}$ on $\mathcal M(E)$ is a \emph{quise-stationary distribution (QSD)} of the superprocess $X$ if  for each $t\geq 0$,
\[
	({\mathbf P}\mathbb P)(X_t \in \cdot | \zeta > t) ={\mathbf P}(\cdot).
\]
	According to the standard theory of QSD (see \cite{MeleardVillemonais2012Quasi-stationary}), if ${\mathbf P}$ is a QSD of $X$, then under $({\mathbf P}\mathbb P)$, the lifetime $\zeta$ has an exponential distribution with some constant $r > 0$, that is
\[
	( {\mathbf P}\mathbb P)(\zeta > t) = e^{-r t}.
\]
	We refer to $r$ the \emph{rate of mass decay} associated to the QSD $\mathbf P$.

	We say a probability ${\mathbf P}$ on $\mathcal M(E)$ is the \emph{Yaglom distribution} of the superprocess $X$ if for any $\mu\in \mathcal M(E)\setminus\{0\}$ we have
\[
	\mathbb P_\mu(X_t \in \cdot | \zeta > t) \xrightarrow[t\to \infty]{w} {\mathbf P}(\cdot).
\]

	If the Yaglom distribution ${\mathbf P}$ of the superprocess $X$ exists, then it must be a QSD of $X$
	(see \cite{MeleardVillemonais2012Quasi-stationary}).
	Our second theorem is about the QSD and the Yaglom distribution of the superpocess $X$:

\begin{thm}
\label{thm: qsd thm}
	Suppose that the superprocess $X$ satisfies Assumptions \ref{asp: 1},\ref{asp:IU},\ref{asp: 3}.  Then,
\begin{enumerate}
\item \label{thm: qsd thm 1}
	for each $\gamma\in[\lambda,0)$, there is a unique probability measure ${\mathbf P}^{\gamma}$ on $\mathcal M(E)$ such that $ {\mathbf P}^\gamma$ is a $QSD$ of the superprocess $X$ with rate of mass decay $-\gamma$.
	Let $\nu(dx):=\hat\phi(x)\cdot m(dx)$.
	Then ${\mathbf P}^\gamma$ is the distribution of the random measure $M^{(\gamma)}\nu(dx)$ where $M^{(\gamma)}$ is a non-negative random variable with Laplace transform
\[
	E[e^{-\theta M^{(\gamma)}}]
	= 1 - e^{\gamma B(\theta)},
	\quad \theta \in (0,\infty).
\]
	Here, map $B: \theta \mapsto B(\theta)$ with $\theta \in (0,\infty)$ is defined as the inverse of the map
\[
	t
	\mapsto -\log \mathbb P_\nu(\zeta \leq t),
	\quad t\in (0,\infty).
\]
\item
	${\mathbf P}^\lambda$ is the Yaglom distribution of $X$.
	In another word, for any  $f\in\mathcal B_b^+(E)$ and any $\mu\in \mathcal M(E)\setminus\{0\}$,
\[
	\lim_{t\rightarrow\infty}\mathbb P_{\mu}[e^{-\langle f,X_t\rangle} | \zeta>t]
	%= E [e^{- \langle f,\nu\rangle M^{(\lambda)}}],
	= E [e^{- \langle f,\nu\rangle M^{(\lambda)}}].
\]
	In particular, for each $\mu \in \mathcal M(E)$, $\{\langle \phi, X_t\rangle; \mathbb P_{\mu}(\cdot| \zeta > t) \}$ converges to $M^{(\lambda)}$ in distribution.
\item
	There is no QSD with rate of mass decay $-\gamma$ for each $\gamma\in(-\infty , \lambda)$.
\end{enumerate}
\end{thm}

\begin{prop}\label{exp prop}
	Suppose that the superprocess $X$ satisfies Assumptions \ref{asp: 1},\ref{asp:IU},\ref{asp: 3}.
	Let $M^{(\gamma)}$ where $\gamma \in [\lambda,0)$ be the random variables given by Theorem \ref{thm: qsd thm} (1). Then
\begin{enumerate}
\item
	$E[M^{(\gamma)}] = \infty$ for each $\gamma \in (\lambda, 0)$.
\item
	$E[M^{(\lambda)}] < \infty$ if and only if $\int_E \hat\phi(y)l(y)m(dy)<\infty$.
\end{enumerate}
	Moreover, if $\int_E \hat\phi(y)l(y)m(dy)<\infty$ then the constant $k$ in \eqref{decay rate} is equal to $E[M^{(\lambda)}]^{-1}$.
\end{prop}
	Recall the LlogL criteria of Theorem \ref{equivalent for cbp}, we can say the above equivalency in proposition \ref{exp prop} is the same to that between items $(i)$ and $(iii)$ there.

	Define the process
\[
	M_t=e^{-\lambda t}  \langle \phi, X_t\rangle, \quad t\geq 0.
\]
	It is well known that the process  $(M_t)_{t\geq 0}$ is a martingale with respect to the natural filtration $(\mathscr F_t)_{t\geq 0}$ of the superprocess $X$.

	%For each $\mu \in \mathcal M(E)$, define probability $\widetilde{\mathbb P}_\mu$ as Doob's $h-$transform of $\mathbb P_\mu$
	For each $\mu \in \mathcal M(E)$, let probability $\widetilde{\mathbb P}_\mu$ be Doob's $h-$transform of $\mathbb P_\mu$ such that
\begin{equation} \label{eq: martingale transformation}
	\frac{d\widetilde{\mathbb P}_\mu|_{\mathscr F_t}}{d\mathbb P_\mu|_{\mathscr F_t}}
	=\frac{M_t}{\langle\phi,\mu\rangle },
	%\quad t\geq 0,
	\quad t\geq 0.
\end{equation}
	This kind of martingale measure transformation for branching processes and measure-valued processes have been widely studied.
	We refer to the early papers \cite{EnglanderKyprianou2004Local,Evans1993Two,RoellyRouault1989Processus}, the thesis \cite{Penisson2010Conditional} and the references therein, and the recent papers \cite{ChampagnatRoelly2008Limit,RenSongSun2017Spine,RenSongZhang2018Williams}.
	It is well known that the process $\{(X_t)_{t\geq 0}; \widetilde{\mathbb P}_{\mu}\}$ can be characterized by the so called spine decomposition theorem.
	We will recall this decomposition in details for our model in section $2$.

	Our third theorem says that $\{(X_t)_{t\geq 0}; \widetilde{\mathbb P}_{\mu}\}$ can be considered as the Q-processs of $X$, i.e. the process $\{(X_t)_{t\geq 0}; \mathbb P_{\mu}\}$ conditioned to be never extinct:

\begin{thm}\label{thm: Qprocess}
	Under the assumptions \ref{asp: 1},\ref{asp:IU} and \ref{asp: 3}, for each $\mu \in \mathcal M(E), t\geq 0$ and $A\in\mathscr F_t$, we have
$
	\lim_{s\rightarrow\infty}\mathbb P_\mu(A |\zeta>s)=\widetilde{\mathbb P}_\mu(A).
$
\end{thm}

	It would be interesting to study the asymptotic behavior of this Q-process.  Same to the definition of $\mathbf P\mathbb P$, for each probability $\mathbf P$ on $\mathcal M(E)$, we define $\mathbf P\widetilde{\mathbb P}$.  Then $\{(X_t)_{t\geq 0}; (\mathbf P\widetilde{\mathbb P})\}$ can be considered as the Q-process with a random initial value $X_0$ whose distribution is $\mathbf P$.
	We say a probability $\mathbf P$ on $\mathcal M(E)$ is an \emph{equilibrium probability} of the Q-process $\{(X_t)_{t\geq 0}; (\widetilde{\mathbb P}_\mu)_{\mu\in\mathcal M(E)}\}$ if
\[
	(\mathbf P\widetilde{\mathbb P})(X_t \in \cdot ) =\mathbf P(\cdot),	\quad t\geq 0.
\]
	Our fourth result is the following:

\begin{thm}\label{thm: structure of Qprocess}
	Under the assumptions \ref{asp: 1},\ref{asp:IU} and \ref{asp: 3}:
\begin{enumerate}
\item
	If $\int_E\hat\phi(x)l(x)m(dx)<\infty$, then $\{(X_t)_{t\geq 0};(\widetilde{\mathbb P}_\mu)_{\mu\in\mathcal M(E)}\}$ has equilibrium probability ${\mathbf P}$.
	For any $\mu\in\mathcal M(E)$, we have
\[
	\widetilde{\mathbb P}_\mu(X_t \in \cdot ) \xrightarrow[t\to \infty]{w} {\mathbf P}(\cdot).
\]
	The equilibrium probability $\mathbf P$ is the distribution of the random measure $M\widehat\phi(x)m(dx)$,
	where the non-negative random variable $M$ has Laplace transform
\[
  	E[e^{-\theta  M}] = \dfrac{E[M^{(\lambda)}e^{-\theta M^{(\lambda)}}]}{E[M^{(\lambda)}]},\quad \theta > 0.
\]
\item
  	If $\int_E\hat\phi(x)l(x)m(dx)<\infty$, then there is a random measure $K$ on $(0,\infty)$ such that
\[
	E[e^{-\theta \widetilde M}] = E\Big[\exp\Big\{- \int_{(0,\infty)} (1-e^{-\theta z }) K(dz) \Big\}\Big].
\]
\item
	If $\int_E\hat\phi(x)l(x)m(dx)=\infty$, then for each $\mu \in \mathcal M(E)$, we have $\lim_{t\rightarrow\infty}\langle \phi, X_t\rangle =\infty$ in probability with respect to $\widetilde{\mathbb P}_\mu$.
\end{enumerate}
\end{thm}

	One important technique used to prove the above theorems is a ``spine-decomposition" for the super-diffusion $X$ under a martingale change of measure. This decomposition was used by Englander and Kyprianou in \cite{EnglanderKyprianou2004Local} to investigate the local extinction of super-diffusions,
	in which the branching mechanism is $\psi(x,z)-\beta(x)z=\alpha(x)^2z^2-\beta(x)z$. 	
	This technique is usually used to investigate the properties of supercritical superdiffusions ($\lambda>0$).
	Here we use it to analyze the case of $\lambda<0$.


\section{Preliminaries}
\subsection{Spine process and its time reverse}
	Let $\{(Y_t)_{t\geq 0}; (\Pi_x)_{x\in E}\}$ be the spatial motion introduced in Section 1 with assumption \ref{asp: 1} and \ref{asp:IU}.
	For each $x\in E$,
	let the probability $\widetilde \Pi_{x}$ be Doob's $h$-transform of $\Pi_x$ such that
\begin{align}
	\dfrac{d\widetilde{\Pi}_x|_{\mathscr F^Y_t}}{d\Pi_x|_{\mathscr F^Y_t}}= \frac{e^{\int_0^t \beta(Y_s)ds}\phi(Y_t) \mathbf 1_{\{t<\tau\}}}{e^{\lambda t}\phi(x)},
	\quad t\geq 0,
\end{align}
	where $(\mathscr F_t^Y)_{t\geq 0}$ is the natural filtration of process $(Y_t)_{t\geq 0}$. 	For each $\mu \in \mathcal M(E)$, define
\[
	\Pi_{\mu}(\cdot)
	:= \mu(E)^{-1}\int_{E} \Pi_x(\cdot)\mu(dx),
\]
    and
\[
	\tilde\Pi_{\mu}(\cdot):= \mu(E)^{-1} \int_E\tilde\Pi_x(\cdot)\mu(dx).
\]
	For each function $f \in \mathcal B_b^+(E)$ and measure $\mu \in \mathcal M(E)$, define measure $(\phi \cdot\mu)$ such that
\[
    (\phi \cdot \mu)(dx)
    := \phi(x)\mu(dx),
    \quad x\in E.
\]
\begin{lem}
	For each $\mu\in \mathcal M(E)$, we have
\[
	\dfrac{\widetilde \Pi_{\phi\cdot\mu}|_{\mathscr F_t^Y}}{\Pi_{\mu}|_{\mathscr F_t^Y}}
  	:= \frac{e^{\int_0^t \beta(Y_s)ds}\phi(Y_t) \mathbf 1_{\{t<\tau\}}}{\mu(E)^{-1}e^{\lambda t}\langle \phi,\mu\rangle},
  	\quad t\geq 0.
\]
\end{lem}
\begin{proof}
	Fix an arbitrary time $t\geq 0$. Fix an arbitrary event $A \in \mathscr F_t^Y$.
	Then we have
\begin{align}
	&\tilde{\Pi}_{\phi\cdot\mu}(A)
	= \langle\phi, \mu\rangle^{-1} \int_E \tilde \Pi_x(A)\phi(x)\mu(dx)
	\\&=\langle\phi, \mu\rangle^{-1} \int_E  \Pi_x\Big[\frac{e^{\int_0^t \beta(Y_s)ds}\phi(Y_t) \mathbf 1_{\{t<\tau\}}}{e^{\lambda t}\phi(x)} \mathbf 1_A\Big]\phi(x)\mu(dx)
	\\&= \mu(E)^{-1}\int_E  \Pi_x\Big[\frac{e^{\int_0^t \beta(Y_s)ds}\phi(Y_t) \mathbf 1_{\{t<\tau\}}}{\mu(E)^{-1}e^{\lambda t}\langle \phi,\mu\rangle} \mathbf 1_A\Big]\mu(dx)
	\\&= \Pi_{\mu}\Big[\frac{e^{\int_0^t \beta(Y_s)ds}\phi(Y_t) \mathbf 1_{\{t<\tau\}}}{\mu(E)^{-1} e^{\lambda t}\langle \phi,\bar\mu\rangle} \mathbf 1_A\Big].
	\qedhere
\end{align}
\end{proof}
	It can be verified (see \cite{KimSong2008Intrinsic} for example) that process $\{(Y_t)_{t\geq 0}; (\widetilde\Pi_x)_{x\in E}\}$ is a time homogeneous Markov process.  Its transition density with respect to measure $m$ is given by
\begin{equation}
\label{eq: tilde p}
    \tilde p(t, x, y)
    :=\frac{\mbox{e}^{-\lambda t}}{\phi(x)}\ p^\beta(t, x, y)\phi(y),
    \quad x,y \in E,t>0.
\end{equation}
	It can also be verified that $\phi(y)\hat{\phi}(y)m(dy)$ is the unique invariant measure of $\{(Y_t)_{t\geq 0}; (\widetilde\Pi_x)_{x\in E}\}$.

	It follows from \cite[Theorem 2.7]{KimSong2008Intrinsic} that there exists $c, \rho > 0$ such that
\begin{equation}\label{IU}
	\sup_{x,y\in E}\Big|\frac{\tilde p(t,x,y)}{\phi(y) \hat\phi(y)}- 1\Big|
	=\sup_{x,y\in E}\Big|\frac{e^{-\lambda t}p^\beta(t,x,y)}{\phi(x) \hat\phi(y)}- 1\Big|
	\leq c\,e^{-\rho t},
	\quad t\geq 1.
\end{equation}

	Let $\{(\hat{Y}_t)_{t\geq 0}; (\hat{\Pi}_x)_{x\in E}\}$ be an $E$-valued Hunt process whose transition density with respect to measure $m$ is given by
\[
    \hat{p}(t,x,y)
    =e^{-\lambda t}p^\beta(t,y,x)\frac{{\hat\phi}(y)}{{\hat\phi}(x)}
    =\tilde p(t,y,x)\frac{\phi(y){\hat\phi}(y)}{\phi(x){\hat\phi}(x)},
    \quad x,y \in E,\,\, t> 0.
\]
	It is easy to check that $(\widehat Y_t)_{t\geq 0}$ has the unique invariant measure $\phi(x)\hat\phi(x)m(dx)$, and is exponentially ergodic in the sense that there exists $c, \rho > 0$ such that
\begin{equation}\label{IU'}
	\sup_{x,y\in E}\left|\frac{\hat{p}(t, x,y)}{\phi(y) \hat\phi(y)}- 1\right|\le c\,\mbox{e}^{-\rho t}, \quad t\geq 1.
\end{equation}
\begin{lem}
\label{lem:reverse of the spine}
	Let $\nu(dx):=\hat\phi(x)m(dx)$.
	For each $T > 0$, we have
\[
	\{(Y_{T-t})_{0\leq t\leq T}; \widetilde \Pi_{\phi \cdot \nu}\}
	\overset{d}{=} \{(\widehat Y_{t})_{0\leq t\leq T}; \widehat \Pi_{\phi \cdot \nu}\}
\]
\end{lem}
\begin{proof}
	Fix arbitrary $T>0$, $n \in \mathbb N$ and $0= t_1\leq \dots \leq t_n = T$.
	For each $i=1,\dots, n$, choose an arbitrary $B_i \in \mathscr B(E)$.
	We only need to show that
\[
	\widetilde \Pi_{\phi \cdot \nu}\{Y_{T-t_i}\in B_i,\forall i=1,\dots, n\}
	\overset{d}{=}\widehat \Pi_{\phi \cdot \nu}\{\hat Y_{t_i}\in B_i,\forall i=1,\dots, n\}
\]
	In fact, on one hand, we have
\begin{align}
	&\widetilde \Pi_{\phi \cdot \nu}\{Y_{T-t_i}\in B_i,\forall i=1,\dots, n\}
	\\&= \int_{y_n\in B_n} \phi(y_n)\hat\phi(y_n) m(dy_n)\int_{y_{n-1}\in B_{n-1}} \tilde p_{t_n - t_{n-1}}(y_n,y_{n-1})m(dy_{n-1})
	\\& \qquad \dots \int_{y_1\in B_1} \tilde p_{t_2 - t_1}(y_2,y_1)m(dy_1)
	\\&= \int_{E^n} \Big(\prod_{i=1}^n \mathbf 1_{\{y_i\in B_i\}}\Big)\cdot\Big(\prod_{i=1}^{n-1} \tilde p_{t_{i+1}-t_i}(y_{i+1},y_i)\Big)\cdot\phi(y_n)\hat\phi(y_n)\cdot \Big(\prod_{i=1}^nm(dy_i)\Big).
\end{align}
	On the other hand, we have
\begin{align}
	&\widehat \Pi_{\phi\cdot\nu}\{Y_{t_i}\in B_i, \forall i = 1,\dots, n\}
	\\&= \int_{y_1\in B_1} \phi(y_1)\hat \phi(y_1)m(dy_1) \int_{y_2\in B_2} \hat p_{t_2-t_1}(y_1,y_2)m(dy_2)
	\\&\qquad \dots\int_{y_n\in B_n}\hat p_{t_n-t_{n-1}}(y_{n-1},y_n)m(dy_n)
	\\&= \int_{y_1\in  B_1} \phi(y_1)\hat \phi(y_1)m(dy_1) \int_{y_2\in B_2} \tilde p_{t_2-t_1}(y_2,y_1)\frac{\phi(y_2)\hat \phi(y_2)}{\phi(y_1)\hat \phi(y_1)}m(dy_2)
	\\&\qquad \dots\int_{y_n\in B_n}\tilde p_{t_n-t_{n-1}}(y_n,y_{n-1})\frac{\phi(y_n)\hat\phi(y_n)}{\phi(y_{n-1})\hat\phi(y_{n-1})}m(dy_n)
	\\&= \int_{E^n} \Big(\prod_{i=1}^n \mathbf 1_{\{y_i\in B_i\}}\Big)\cdot\Big(\prod_{i=1}^{n-1} \tilde p_{t_{i+1}-t_i}(y_{i+1},y_i)\Big)\cdot\phi(y_n)\hat\phi(y_n)\cdot \Big(\prod_{i=1}^n m(dy_i)\Big).
\qedhere
\end{align}
\end{proof}

\subsection{Kuznestuv measure}
	Suppose that $X$ is the superprocess introduced in Section \ref{sec: main results} which satisfies that $\mathbb P_{x}(\zeta < t)>0$ for each $x\in E$ and $t>0$. Denote by
\begin{align}
	\mathbb D &:=\{ w= (w_t)_{t\geq 0}: w \text{ is an $\mathcal M(E)$-valued c\`{a}dl\`{a}g function on $[0,\infty)$ }
	\\ &\qquad \text{ with the null measure as a trap} \}
\end{align}
	the Skorokhod space of measure-valued excursion paths.

	According to \cite[Section 8.4]{Li2011Measure-valued}, there is a unique family of $\sigma$-finite measures $(\mathbb N_x)_{x\in E}$ on $\mathbb D$ such that
\begin{itemize}
\item
    $\mathbb N_x \{\forall t > 0, w_t(\mathbf 1_E)=0\} =0$ for each $x\in E$;
\item
    $\mathbb N_x \{ w_0(\mathbf 1_E) > 0\} = 0$ for each $x\in E$;
\item
    For each $\mu \in \mathcal M(E)$, if $\mathcal N$ is a Poisson random measure on $\mathbb D$ with intensity
\[
	\mathbb N_\mu(dw):= \int_E \mathbb N_x(dw)\mu(dx), \quad w\in \mathbb D.
\]
	then
\[
	\{(X_t)_{t> 0};\mathbb P_\mu\}
	\overset{f.d.d.}{=} \left(\int_{\mathbb D} w_t~\mathcal N(dw)\right)_{t> 0}.
\]
\end{itemize}
	This family of measure $(\mathbb N_x)_{x\in E}$ is known as the \emph{Kuznetsov measures} of $X$.

	%In the remainder of this paper, we will always use $w = (w_t)_{t\geq 0}$ to denote a generic element in $\mathbb D$.
	%For each $f\in \mathcal B_b^+(D)$, from Campbell's formula it can be verified that for each $x\in E$ and $t>0$, we have
%\begin{align}\label{eq: kuznetsov Laplace}
 	%\mathbb N_x[1-e^{-w_t(f) }]
 	%&=-\log \mathbb P_x[e^{-X_t(f)}] = V_t f(x),
 	%\\ \mathbb N_x[w_t(f)]
 	%&=P_t^{\beta}f(x),
 	%\\\mathbb N_x\{w_t(\mathbf 1_E) \neq 0\}
 	%&=-\log\mathbb P_x\{X_t(\mathbf 1_E) = 0\}.
%\end{align}

	In the remainder of this paper, we will always use $w = (w_t)_{t\geq 0}$ to denote a generic element in $\mathbb D$.
	For each $f\in \mathcal B_b^+(D)$, from Campbell's formula it can be verified that for each $\mu\in \mathcal M(E)$ and $t>0$, we have
\begin{align}\label{eq: kuznetsov Laplace}
 	\mathbb N_\mu[1-e^{-w_t(f) }]
 	&=-\log \mathbb P_\mu[e^{-X_t(f)}] 
 	= \mu(V_t f),
 	\\ \mathbb N_\mu[w_t(f)]
 	&=\mu(P_t^{\beta}f),
 	\\\mathbb N_\mu\{w_t(\mathbf 1_E) \neq 0\}
 	&=-\log\mathbb P_\mu\{X_t(\mathbf 1_E) = 0\}.
\end{align}


\subsection{Spine decomposition}
Suppose that $X$ is the superprocess introduced in Section \ref{sec: main
  results} which satisfies Assumption \ref{asp: 1} and \ref{asp: 3}
\eqref{subasp: point non-presistence}. Fix an arbitrary $\mu\in \mathcal M(E)$.
Define the probability $\widetilde {\mathbb P}_\mu$ using \eqref{eq: martingale
  transformation}.

For each $\mu \in \mathcal M(E)$, we say $\{(Y)_{t\geq 0}, (X^{\mathrm n,
  \sigma})_{\sigma\in \mathcal D^\mathrm n}, (X^{\mathrm m, \sigma})_{\sigma \in
  \mathcal D^\mathrm m}, (X_t)_{t\geq 0}; \mathbb Q_{\mu}\}$ is a \emph{spine
  representation} of $\{(X_t)_{t\geq 0}; \widetilde {\mathbb P}_\mu\}$ if the
followings are true:
\begin{itemize}
\item The \emph{spine process} $\{(Y_t)_{t\geq 0}; \mathbb Q_\mu\}$ is a copy of
  $\{(Y_t)_{t\geq 0}; \widetilde \Pi_{\phi\cdot\mu}\}$.
\item Given $\{(Y_t)_{t\geq 0}; \mathbb Q_\mu\}$, \emph{the continuum
    immigration} $\{ (X^{\mathrm n,\sigma})_{\sigma \in \mathcal D^\mathrm n};
  \mathbb Q_\mu(\cdot |Y)\}$ is a $\mathbb D$-valued point process such that
  \[
	\mathrm n(ds,dw) := \sum_{\sigma\in \mathcal D^{\mathrm n}} \delta_{(\sigma, X^{\mathrm n,\sigma})}(ds,dw)
\]

is a Poission random measure on $[0,T]\times \mathbb D$ with intensity
\[
	\mathbf n(ds,dw):= 2 \alpha(Y_s) ds \cdot \mathbb N_{Y_s}(dw).
\]
\item Given $\{(Y_t)_{t\geq 0}; \mathbb Q_\mu\}$, \emph{the discrete
    immigration} $\{(X^{\mathrm m,\sigma})_{\sigma\in \mathcal D^{\mathrm m}};
  \mathbb Q_\mu(\cdot |Y)\}$ is a $\mathbb D$-valued point process such that
\[
	\mathrm m(ds,dw) := \sum_{\sigma\in \mathcal D^{\mathrm n}} \delta_{(\sigma, X^{\mathrm n,\sigma})}(ds,dw)
\]
is a Poisson random measure on $[0,\infty ) \times \mathbb D$ with intensity
\begin{align}\label{eq:meanMeasImmigr}
	\mathbf m(ds,dw):= ds \cdot \int_{(0,\infty)} y \mathbb P_{y\delta_{Y_s}}(X\in dw) n(Y_s,dy);
\end{align}
\item Given $\{(Y_t)_{t\geq 0}; \mathbb Q_\mu\}$, the continuum immigration
  $(X^{\mathrm n,\sigma})_{\sigma \in \mathcal D^n}$ and the discrete
  immigration $(X^{\mathrm m,\sigma})_{\sigma\in \mathcal D^{\mathrm m}}$ are
  independent of each other.
\item
	% $\{(X_t)_{t\geq 0}; \mathbb Q_\mu\}$ is a copy of the superprocess
	% $\{(X_t)_{t\geq 0}; \mathbb P_\mu\}$ which is independent of the spine
	% process $(Y_t)_{t\geq 0}$, the continuum immigration $(X^{\mathrm
	% n,\sigma})_{\sigma \in \mathcal D^n}$ and the discrete immigration
	% $(X^{\mathrm m,\sigma})_{\sigma\in \mathcal D^{\mathrm m}}$.
	$\{(X_t)_{t\geq 0}; \mathbb Q_\mu\}$ is a copy of the superprocess
  $\{(X_t)_{t\geq 0}; \mathbb P_\mu\}$, and is independent of the spine process
  $(Y_t)_{t\geq 0}$, the continuum immigration $(X^{\mathrm n,\sigma})_{\sigma
    \in \mathcal D^\mathrm n}$ and the discrete immigration $(X^{\mathrm
    m,\sigma})_{\sigma\in \mathcal D^{\mathrm m}}$.
\end{itemize}

% For each $\mu \in \mathcal M(E)$,
To simplify the notations, for each $\mu \in \mathcal M(E)$, $t\geq 0$ and each
$B \in \mathscr B([0,t))$, with respect to probability $\mathbb Q_\mu$, define
the following random measures:
\begin{align}
	Z^{\mathrm n,B}_t
	&:= \int_{B\times \mathbb D} w_{t-s} ~\mathrm n (ds,dw)
	= \sum_{\sigma \in \mathcal D^\mathrm n \cap B} X^{\mathrm n,\sigma}_{t-\sigma},
	\\ Z^{\mathrm m,B}_t
	&:= \int_{B\times \mathbb D} w_{t-s} ~\mathrm m (ds,dw)
	= \sum_{\sigma \in \mathcal D^\mathrm m \cap B} X^{\mathrm m,\sigma}_{t-\sigma}.
\end{align}

The spine decomposition theorem (see \cite{RenSongSun2017Spine} for the general
cases) says that
\begin{lem}\label{spine structure}
	Suppose that $\{(Y)_{t\geq 0}, (X^{\mathrm n, \sigma})_{\sigma\in \mathcal
    D^\mathrm n}, (X^{\mathrm m, \sigma})_{\sigma \in \mathcal D^\mathrm m},
  (X_t)_{t\geq 0}; \mathbb Q_{\mu}\}$ is a spine representation of
  $\{(X_t)_{t\geq 0}; \widetilde {\mathbb P}_\mu\}$. Then
\begin{align}
	\{(X_t)_{t\geq 0}; \widetilde{\mathbb P}_\mu\}
	\overset{f.d.d.}{=}
	\{(X_t + Z^{\mathrm n, [0,t)}_{t} + Z^{\mathrm m, [0,t)}_{t} )_{t\geq 0}; \mathbb Q_\mu\}.
\end{align}
\end{lem}
\subsection{Reverse spine representation}
Suppose that $X$ is the superprocess introduced in Section \ref{sec: main
  results} which satisfies Assumption \ref{asp: 1} and \ref{asp: 3}
\eqref{subasp: point non-presistence}.

    Recall that $\nu := \widehat \phi \cdot m \in \mathcal M(E)$. Define the probability $\widetilde {\mathbb P}_\nu$ using \eqref{eq: martingale transformation}.
    We say $\{(Y)_{t\geq 0}, (X^{\mathrm n, \sigma})_{\sigma\in \mathcal D^\mathrm n}, (X^{\mathrm m, \sigma})_{\sigma \in \mathcal D^\mathrm m}, (X_t)_{t\geq 0}; \hat {\mathbb Q}_{\nu}\}$ is a \emph{reverse spine representation} of $\{(X_t)_{t\geq 0}; \widetilde {\mathbb P}_\nu\}$ if the followings are true:
\begin{itemize}
\item
    The \emph{reverse spine process} $\{(Y_t)_{t\geq 0}; \widehat {\mathbb Q}_\nu\}$ is a copy of $\{(Y_t)_{t\geq 0}; \widehat \Pi_{\phi\cdot\nu}\}$.
\item
    Conditioned on $\{(Y_t)_{t\geq 0}; \widehat{\mathbb Q}_\nu\}$, \emph{the reverse continuum immigration} $\{ (X^{\mathrm n,\sigma})_{\sigma \in \mathcal D^\mathrm n}; \widehat{\mathbb Q}_\nu(\cdot |Y)\}$ is a $\mathbb D$-valued point process such that
\[
    \mathrm n(ds,dw)
    = \sum_{\sigma\in \mathcal D^{\mathrm n}} \delta_{(\sigma, X^{\mathrm n,\sigma})}(ds,dw)
\]
	is a Poission random measure on $[0,T]\times \mathcal W$ with density
\[
	\mathbf n(ds,dw)= 2\alpha(Y_s) ds \cdot \mathbb N_{Y_s}(dw).
\]
\item
    Conditioned on $\{(Y_t)_{t\geq 0}; \widehat{\mathbb Q}_\nu\}$, \emph{the reverese discrete immigration} $\{(X^{\mathrm m,\sigma})_{\sigma\in \mathcal D^{\mathrm m}}; \widehat{\mathbb Q}_\nu(\cdot |Y)\}$ is a $\mathbb D$-valued point process such that
\[
    \mathrm m(ds,dw)
    = \sum_{\sigma\in \mathcal D^{\mathrm m}} \delta_{(\sigma, X^{\mathrm m,\sigma})}(ds,dw)
\]
	is a Poisson random measure on $[0,\infty ) \times \mathbb D$ with intensity
\begin{align}\label{eq:meanMeasImmigr}
	\mathbf m(ds,dw)= ds \cdot \int_{(0,\infty)} y \mathbb P_{y\delta_{Y_s}}(X\in dw) n(Y_s,dy);
\end{align}
\item
	Given $\{(Y_t)_{t\geq 0}; \widehat{\mathbb Q}_\nu\}$, the reverse continuum immigration $(X^{\mathrm n,\sigma})_{\sigma \in \mathcal D^n}$ and the reverse discrete immigration $(X^{\mathrm m,\sigma})_{\sigma\in \mathcal D^{\mathrm m}}$ are independent of each other.
\item
	$\{(X_t)_{t\geq 0}; \widehat {\mathbb Q}_\nu\}$ is a copy of the superprocess $\{(X_t)_{t\geq 0}; \mathbb P_\mu\}$ which is independent of the reverse spine process $(Y_t)_{t\geq 0}$, the reverse continuum immigration $(X^{\mathrm n,\sigma})_{\sigma \in \mathcal D^n}$ and the reverse discrete immigration $(X^{\mathrm m,\sigma})_{\sigma\in \mathcal D^{\mathrm m}}$.
\end{itemize}

	To simplyfy the notations, for each $t\geq 0$, with respect to probability $\widehat{\mathbb Q}_\nu$, define the following random measures:
\[\begin{split}
	\widehat Z^{\mathrm n}_t
	&:= \int_{[0,t)\times \mathbb D} w_{s} ~\mathrm n (ds,dw)
	= \sum_{\sigma \in \mathcal D^\mathrm n \cap [0,t)} X^{\mathrm n,\sigma}_{\sigma},
	\\ \widehat Z^{\mathrm m}_t
	&:= \int_{[0,t)\times \mathbb D} w_{s} ~\mathrm m (ds,dw)
	= \sum_{\sigma \in \mathcal D^\mathrm m \cap [0,t)} X^{\mathrm m,\sigma}_{\sigma}.
\end{split}\]
\begin{lem}
	Suppose that $\{X, Y, \mathrm n, \mathrm m; \widehat{\mathbb Q}_{\nu}\}$ is a reverse spine representation of $\{X; \widetilde {\mathbb P}_\nu\}$.
	Then for each $t\geq 0$,
$
	\{X_t; \widetilde{\mathbb P}_\nu\}
	\overset{d}{=}
	\{X_t + \widehat Z^{\mathrm n}_{t} + \widehat Z^{\mathrm m}_{t}; \widehat{\mathbb Q}_\nu\}.
$
\end{lem}
\begin{proof}
	Fix an arbitrary time $t\geq 0$. According to Lemma \ref{spine structure}, we only have to proof that
\[
	\{Z^{\mathrm n,[0,t)}_{t} + Z^{\mathrm m,[0,t)}_{t}; \mathbb Q_\nu\}
	\overset{d}{=}
	\{\widehat Z^{\mathrm n}_{t} + \widehat Z^{\mathrm m}_{t}; \widehat{\mathbb Q}_\nu\}.
\]
	In fact, for each $f\in \mathcal B_b^+(E)$, from campbell's formula, we have
\begin{align}
	&-\log \mathbb Q_\nu \left [\left. e^{-\langle f, Z^{\mathrm n,[0,t)}_{t} + Z^{\mathrm m,[0,t)}_{t}\rangle}\right |(Y_t)_{t\geq 0}\right]
	\\&= \int_{[0,t)\times \mathbb D} \left(1-e^{- \langle f, w_{t-s}\rangle}\right)\left(\mathbf n(ds,dw) + \mathbf m(ds,dw)\right)
	\\&= \int_{[0,t)} \left(2\alpha(Y_s) \cdot \mathbb N_{Y_s}\left(1-e^{-w_{t-s}(f)}\right) + \int_{(0,\infty)} y \mathbb P_{y\delta_{Y_s}}\left(1-e^{-X_{t-s}(f)}\right)n(Y_s,dy)\right) ds
	\\&= \int_{[0,t)} \left(2\alpha(Y_s) \cdot (V_{t-s}f)(Y_s) + \int_{(0,\infty)} y \left(1-e^{-y\cdot(V_{t-s}f)(Y_s)}\right)n(Y_s,dy)\right) ds
	\\&= \int_{[0,t)} \psi_0'\left( Y_s, V_{t-s}f(Y_s)\right)ds
\end{align}
	and
\begin{align}
	&-\log \widehat{\mathbb Q}_\nu \left [\left. e^{-(\widehat Z^{\mathrm n}_{t} + \widehat Z^{\mathrm m}_{t})(f)}\right |(Y_t)_{t\geq 0}\right]
	\\&= \int_{[0,t)\times \mathbb D} \left(1-e^{- w_s(f)}\right)\left(\mathbf n(ds,dw) + \mathbf m(ds,dw)\right)
	\\&= \int_{[0,t)} \left(2\alpha(Y_s) \cdot \mathbb N_{Y_s}\left(1-e^{-w_{s}(f)}\right) + \int_{(0,\infty)} y \mathbb P_{y\delta_{Y_s}}\left(1-e^{-X_{s}(f)}\right)n(Y_s,dy)\right) ds
	\\&= \int_{[0,t)} \left(2\alpha(Y_s) \cdot (V_{t}f)(Y_s) + \int_{(0,\infty)} y \left(1-e^{-y\cdot(V_{t}f)(Y_s)}\right)n(Y_s,dy)\right) ds
	\\&= \int_{[0,t)} \psi_0'\left( Y_s, V_{t}f(Y_s)\right)ds.
\end{align}
	lem:reverse of the spine
	Therefore, according to Lemma \ref{lem:reverse of the spine}, for each $f\in \mathcal B_b^+(E)$, we have
\begin{align}
  	&\mathbb Q_\nu  \big[e^{-(Z^{\mathrm n,[0,t)}_{t} + Z^{\mathrm m,[0,t)}_{t})(f)}\big]
  	= \widetilde \Pi_{\phi\cdot\nu} \big[e^{-\int_{[0,t)} \psi_0'( Y_s, V_{t-s}f(Y_s))ds}\big]
  	\\&= \widetilde \Pi_{\phi\cdot\nu} \big[e^{-\int_{[0,t)} \psi_0'( Y_{t-s}, V_{s}f(Y_{t-s}))ds}\big]
  	= \widehat \Pi_{\phi\cdot\nu} \big[e^{-\int_{[0,t)} \psi_0'( Y_{s}, V_{s}f(Y_{s}))ds}\big]
  	\\&= \widehat{\mathbb Q}_\nu \big [e^{-(\widehat Z^{\mathrm n}_{t} + \widehat Z^{\mathrm m}_{t})(f)}\big].
  	\qedhere
\end{align}
\end{proof}

\subsection{LlogL criterion.}
	Both of the processes $\{(Y)_{t\geq 0}, \widetilde\Pi_x\}$ and $\{(Y)_{t\geq 0}, \widehat{\Pi}_x\}$ are ergodic and have the same invariant probability $\phi(x)\hat\phi(x)m(dx)$.
	The transition probability of these two processes both have uniform convergence properties \eqref{IU} and \eqref{IU'}.
	Therefore we can repeat the arguments for Lemma $3.2$ in \cite{LiuRenSong2009Llog} and obtain the following results.
	The proofs will be omitted.

\begin{lem}\label{import lemma}
	Let $\mu \in \mathcal M(E)\setminus \{\mathbf 0\}$.
	Suppose that \[\{(Y)_{t\geq 0}, (X^{\mathrm n, \sigma})_{\sigma\in \mathcal D^\mathrm n}, (X^{\mathrm m, \sigma})_{\sigma \in \mathcal D^\mathrm m}, (X_t)_{t\geq 0}; \mathbb Q_{\mu}\}\] is a spine representation of $\{(X_t)_{t\geq 0}; \widetilde {\mathbb P}_\mu\}$.
	With respect to probability $\mathbb Q_\mu$, let $(m_\sigma)_{\sigma\in \mathcal D^{\mathrm m}}$ be the $\mathbb R^+$-valued point process defined by
\[
  	m_\sigma = X^{\mathrm m, \sigma}_0(\mathbf 1_E),
  	\quad \sigma \in \mathcal D^{\mathrm m}.
\]
	Then the following sequence of random variables
\[
	\sigma_0=0,\quad \sigma_i=\inf\{s\in\mathcal D^{\mathrm m};\ s>\sigma_{i-1},\ m_s\phi(Y_s)>1\}, \quad\eta_i=m_{\sigma_i},\quad i=1,2,\cdots.
\]
	are well defined with respect to probability $\mathbb Q_\mu$.
	Furthermore, for any $\varepsilon>0$:
	\\if $\int_E\hat{\phi}(y)l(y)m(dy)<\infty$ then
\[
	\sum_{s\in\mathcal
	D^{\mathrm m}}\mbox{e}^{-\varepsilon s}m_s\phi(Y_s) < \infty, \quad
	%\mathbb Q_{\mu}-\mbox{a.s.}
	\mathbb Q_{\mu}\text{-a.s.};
\]
	if $ \int_E\hat{\phi}(y)l(y)m(dy)=\infty$, then
\[
	\limsup_{i\rightarrow\infty}e^{-\varepsilon \sigma_i}\eta_i
	\phi(Y_{\sigma_i})=\infty,
	\quad \mathbb Q_{\mu}\text{-a.s.}.
\]
\end{lem}
\begin{lem}\label{import lemma}
	Let $\nu := \widehat \phi \cdot m$.
	Suppose that \[\{(Y)_{t\geq 0}, (X^{\mathrm n, \sigma})_{\sigma\in \mathcal D^\mathrm n}, (X^{\mathrm m, \sigma})_{\sigma \in \mathcal D^\mathrm m}, (X_t)_{t\geq 0}; \widehat{\mathbb Q}_{\nu}\}\] is a reverse spine representation of $\{(X_t)_{t\geq 0}; \widetilde {\mathbb P}_\nu\}$.
	With respect to probability $\widehat{\mathbb Q}_\nu$, let $(m_\sigma)_{\sigma\in \mathcal D^{\mathrm m}}$ be the $\mathbb R^+$-valued point process defined by
\[
	m_\sigma
	= X^{\mathrm m, \sigma}_0(\mathbf 1_E),
	\quad \sigma \in \mathcal D^{\mathrm m}.
\]
	Then the following sequence of random variables
\[
	\sigma_0=0,\quad \sigma_i=\inf\{s\in\mathcal D^{\mathrm m};\ s>\sigma_{i-1},\ m_s\phi(Y_s)>1\}, \quad\eta_i=m_{\sigma_i},\quad i=1,2,\cdots.
\]
	Furthermore, for each $\varepsilon>0$:
	\\if $\int_E\hat{\phi}(y)l(y)m(dy)<\infty$ then
\[
	%\sum_{s\in\mathcal D^{\mathrm m}} \mbox{e}^{-\varepsilon s}m_s\phi(Y_s) 
	\sum_{s\in\mathcal D^{\mathrm m}} e^{-\varepsilon s}m_s\phi(Y_s)
	< \infty,
	%\quad \widehat{\mathbb Q}_{\mu}-\mbox{a.s.}
	\quad \widehat{\mathbb Q}_{\mu}\text{-a.s.};
\]
	if $ \int_E\hat{\phi}(y)l(y)m(dy)=\infty$, then
\[
	\limsup_{i\rightarrow\infty} e^{-\varepsilon\sigma_i}\eta_i \phi(Y_{\sigma_i})
	=\infty,
	\quad \widehat{\mathbb Q}_{\mu}\text{-a.s.}.
\]
\end{lem}

\section{The proofs of main results}
\subsection{Some properties of the solutions to partial differential equations}
	Thanks to Assumption \ref{asp: 3} (1), we can define
\[
	v(t,x):= -\log \mathbb P_x(\zeta \leq t),
	%\quad t > 0, x\in \mathbb R^d.
\quad t > 0, x\in E.
\]
	According to \eqref{eq: def of vtf} and monotonicity, we have
\[
	v(t,x)
	= \lim_{\theta \to \infty} V_t(\theta \mathbf 1_E)(x),
	\quad t>0, x\in E,
\]
	and
\begin{equation}
\label{eq: v and extinction}
	e^{-\langle v(t,\cdot), \mu \rangle}
	= \mathbb P_\mu(\zeta \leq t),
	%\quad t>0, \mu \in \mathcal M(\mathbb R^d).
\quad t>0, \mu \in \mathcal M(E).
\end{equation}


\begin{lem}\label{lem:extinc}
	Suppose that Assumptions \ref{asp: 1}, \ref{asp:IU} and \ref{asp: 3} hold. Then,
\begin{enumerate}
\item
	there exists $t_0>0$ such that
\[
	\langle v(t,\cdot),\mu\rangle <\infty, \quad t>t_0, \,
%\mu \in \mathcal M(\mathbb R^d).
\mu \in \mathcal M(E).
\]
\item
%	for each $\mu \in \mathcal M(\mathbb R^d)$,
	for each $\mu \in \mathcal M(E)$,
\[
	\lim_{t\rightarrow\infty}\langle v(t,\cdot),\mu\rangle=0.
\]
\item
	for each $s\geq 0$,
\begin{equation} \label{one point ratio limit}
	\lim_{t\to \infty} \sup_{x\in \mathbb R^d}\Big|\frac{v(t+s,x)}{\langle v(t,\cdot),\nu\rangle } - \phi(x)e^{\lambda s} \Big|
	=0.
\end{equation}
\end{enumerate}
\end{lem}
\begin{proof}
	We call $(V_t)_{t\geq 0}$ the cumulant semigroup of the superprocess $X$, because it satisfies the semigroup property in the sense that,
%for all $f\in \mathcal B_b^+(\mathbb R^d)$, 
for all $f\in \mathcal B_b^+(E)$,
$t,s \geq 0$ and $x\in E$, it holds that $V_tV_sf(x) = V_{t+s}f(x)$ (see \cite[Theorem 2.21]{Li2011Measure-valued}).
	We write
\[
	\psi_0(x,\lambda) = \psi(x,\lambda)+ \beta(x) \lambda,
	\quad x\in E, \lambda \geq 0.
\]
It is known, see \cite[Theorem 2.23]{Li2011Measure-valued} for example, that
%for each $f \in \mathcal B_b^+(\mathbb R^d)$, 
for each $f \in \mathcal B_b^+(E)$,
$(s,x)\mapsto V_sf(x)$ is the solution of the equation
\[
	V_sf(x) + \int_0^s P_{s-u}^\beta \Big(\psi_0\big(\cdot,V_u f(\cdot)\big)\Big)(x)du = P_s^\beta f(x),
	\quad t\geq 0, x\in E.
\]
	Taking $f = V_t(\theta \mathbf 1_E)$ and passing $\theta \to \infty$ in the above, we get
\begin{equation}
\label{eq: equation for vt}
	v(t+s,x) + \int_0^sP^\beta_{s-u}\Big(\psi_0\big(\cdot, v(t+u,\cdot)\big)\Big)(x)~du
	=P^\beta_s\big(v(t,\cdot)\big)(x),
	\quad t,s > 0, x\in E.
\end{equation}

%NEW
	Step 1.
%END NEW
	We will proof that there exists $t_0 >0$ such that for each $t\geq t_0, s> 0$ and $x \in E$, the both side of the above euqation are finite.
	In fact, according to Assumption \ref{asp: 3} (2) and \eqref{eq: v and extinction}, there exists a $t_1 > 0$ such that $\langle v_t, \nu\rangle < \infty $ for each $t>t_1$.
	According to Assumption \ref{asp:IU} (3) and \eqref{eq: equation for vt}, 
	%there exists $t_0>0$ such that for all $t \geq t_0,s>0$ and $x\in E$ we have
	there a family of positive $(c_s)_{s>0}$, 
	Let $(c_s)_{s>0}$ be the constants in Assumption \ref{asp:IU}, then for each $t \geq t_1,s>0$ and $x\in E$ we have
\begin{equation}
\label{eq:upp}
	v(t+s,x) \leq P^{\beta}_sv(t,x)\leq c_s\phi(x)\int_D\hat\phi(y)v(t,y)m(dy)\leq c_s\phi(x)\langle v(t,\cdot),\nu\rangle< \infty.
\end{equation}
%DEL
	%where $(c_s)_{s>0}$ are the constants appreared in Assumption \ref{asp:IU} (3) and only depend on the choice of $s$.
%END DEL
	This implies desired result.

%NEW
	This also implies the Assertion (1) of this lemma. In fact, from \eqref{eq:upp} and Assumption \ref{asp:IU} (2), we have for each $t>t_0$, $\sup_{x\in E}v(t,x) < \infty$.
%END NEW

%NEW
	Step 2.
	We will proof the Assertion (2) of this lemma.
%END NEW
	According to \eqref{IU}, 
	%the constants $(c_s)_{s>0}$ appeared in Assumption \ref{asp:IU} (3) can acturally be chosen such that $c_s \xrightarrow[s\to \infty]{} 0$.
	the constants $(c_s)_{s>0}$ appeared in Assumption \ref{asp:IU} (3) can be chosen such that $c_s \xrightarrow[s\to \infty]{} 0$.
%NEW
	Therefore, the constants $(c_s)_{s>0}$ appeared in \eqref{eq:upp} can also be chosen such that $c_s \xrightarrow[s\to \infty]{} 0$.
	Note that $\phi$ is bounded by Assumption \ref{asp:IU} (2).
%END NEW
	%In this case, 
	Therefore, taking $s\to \infty$ in \eqref{eq:upp}, we have
	%according to \eqref{eq: equation for vt}, there exists $t_0>0$ and constants $(c_s)_{s>0}$ with $c_s \xrightarrow[s\to \infty]{} 0$, such that for each $t \geq t_0, s>0$ and $\mu \in \mathcal M(E)$, we have
%\[
	%\langle v(t +s,\cdot),\mu\rangle
	%\leq  \langle P^\beta_sv(t,\cdot),\mu\rangle
	%\leq  c_s\langle \phi,\mu\rangle \langle v(t,\cdot), \nu \rangle
	%\xrightarrow[s\to \infty]{} 0.
	%<\infty.
%\]
\begin{equation}
\label{eq: boud of v_t convergence to 0}
	\sup_{x\in E}v(t,x)\xrightarrow[t\to \infty]{} 0,
\end{equation}
	as desired in this step.

%NEW
	Step 3.
	We will show that there exists $t_2>0$ such that 
\[
	\sup_{s: s>t_2} \frac{\langle \psi_0(\cdot, v(s,\cdot)),\nu\rangle}{ \langle v(s,\cdot), \nu\rangle} <\infty.
\]
	First note that $\mathbf P_\nu[X_t(\phi)] = e^{\lambda t}\nu(\phi)>0$. 
	Therefore, according to \eqref{eq: v and extinction}, we have $\langle v(t,\cdot),\nu\rangle = -\log \mathbb P_\nu\{X_t(\phi) = 0\}\in (0,\infty)$ for $t>t_1$. 
%END NEW
	For each $x\in E$, note that
\[
	\psi_0'(x,\lambda):=\frac{\partial}{\partial \lambda}\psi_0(x,\lambda)
	=2\alpha(x)^2\lambda+\int_0^{\infty}\left(1-e^{-r\lambda}\right)rn(x,dr), 
	\quad \lambda \geq 0,
\]
	is a nonnegative locally bounded increasing function of $\lambda$.
{\color{red}
$\lim_{\lambda\rightarrow 0+}\psi_0'(x,\lambda)=0$ for any $x\in E$.}  
	%And for any $\lambda>0$,
	%From \eqref{ext equ in}, we can see that there is some $t_0>0$, such that when $t>t_0$, $v(t,x)$ is bounded. 
	From \eqref{eq: boud of v_t convergence to 0}, we know that there is $t_3>0$, such that $\sup_{t>t_3,x\in E}v(t,x)<\infty$.
%NEW
	Define $L:= \sup_{t>t_3,x\in E}v(t,x)$. 
%END NEW	
	Then, for each $s>t_3$,
\begin{align}\label{upper bound}
	&\langle\psi_0(\cdot,v(s,\cdot)),\nu\rangle = \int_{x\in E} \int_{\lambda = 0}^{v(s,x)} \psi_0'(x,\lambda)d\lambda~\nu(dx)
	\\&\leq \int_{x\in E} \int_{\lambda = 0}^{v(s,x)} \psi_0'(x,L)d\lambda~\nu(dx)
	\leq \langle v(s,\cdot),\nu\rangle\sup_{x\in E} \psi_0'(x,L).
	%\frac{\langle\psi_0(v(s)),\nu\rangle}{\langle v(s),\nu\rangle}\leq \sup_{x\in E}\psi_0'(x,L)<\infty.
\end{align}
	Note that for each $\lambda>0$,
\[
	\sup_{x\in E}\psi_0'(x,\lambda)
	%\leq 2\|\alpha(x)^2\|_\infty\lambda+\|\int_0^{1}r^2n(x,dr)\|_\infty\lambda+2\|\int_1^{\infty}rn(x,dr)\|_\infty<\infty.
	\leq 2\|\alpha(\cdot)^2\|_\infty\lambda+\Big\|\int_0^{1}r^2n(\cdot,dr) \Big\|_\infty\lambda+2\Big\|\int_1^{\infty}rn(\cdot,dr)\Big\|_\infty
	<\infty.
\]
	So, we get the desired result in this step.

%ADDED
	Step 4. We will show that
\begin{equation}
\label{svfcondition}
	\lim_{u\to \infty} \frac{ \langle\psi_0(\cdot, v(u,\cdot)),\nu\rangle}{\langle v(u,\cdot),\nu\rangle}
	=0.
\end{equation}
%END ADDEd	
Integrating the both sides of \eqref{eq: equation for vt} with respect to measure $\nu$, we get
\begin{equation}\label{ext equ in}
	\langle v(t+s,\cdot), \nu\rangle + e^{\lambda s}\int_0^s e^{-\lambda u}\big\langle \psi_0\big(\cdot, v(t+u,\cdot)\big),\nu\big\rangle~du
	= e^{\lambda s}\langle v(t,\cdot),\nu \rangle,
	\quad t,s > 0.
\end{equation}  
%NEW
	Let $g(t)=e^{-\lambda t}\langle v(t,\cdot),\nu\rangle, t>t_1$.
	Then
%END NEW
	\eqref{ext equ in} can be written as 
\begin{align}
	%g(t+s)=g(t)-\int_t^{t+s}\dfrac{\langle\psi_0(v(u)),\nu\rangle}{\langle v(u),\nu\rangle}g(u)du, \quad s,t>0.
	g(t+s)
	&=g(t)-e^{-\lambda t}\int_0^{s} e^{-\lambda u} \big\langle\psi_0\big(\cdot, v(t+u,\cdot)\big),\nu\big\rangle du
	\\&=g(t)-\int_t^{t+s}\dfrac{\langle\psi_0(\cdot, v(u,\cdot)),\nu\rangle}{\langle v(u,\cdot),\nu\rangle}g(u)du,
	\quad t>t_1, s>0.
\end{align}
	Solving this elementary equation, we have that
\[
	g(t)
	=g(t_0)\exp\left\{-\int_{t_0}^t\dfrac{\langle\psi_0(\cdot, v(u,\cdot)),\nu\rangle}{\langle v(u,\cdot),\nu\rangle}du\right\},
	\quad t\geq t_1.
\]
%Now we prove limit \eqref{svfcondition}. 
  	From this we have, for each $u>t_1$,
\[
	\frac{\langle v(u,\cdot),\nu\rangle}{\langle v(u+1,\cdot),\nu\rangle}=\exp\Big\{-\lambda+\int_{u}^{u+1}\dfrac{\langle \psi_0(v(s,\cdot)),\nu\rangle}{\langle v(s,\cdot),\nu\rangle}ds\Big\}.
\]
	Noting the result in Step 3 and putting $t_4=\max\{t_1,t_2\}$, we have
\[
	C^*:= \sup_{u\geq t_4} \frac{\langle v(u,\cdot),\nu\rangle}{\langle v(u+1,\cdot),\nu\rangle} < \infty.
\]
	On the other hand, since
\[
v(t+1,x)\leq P^{\beta}_1v(t,x)\leq c_1\phi(x)\langle v(t),\nu\rangle.
\]
Therefore, for $u>t_0+1$, by mean value theorem,
\begin{eqnarray*}
\dfrac{\langle\psi_0(v(u)),\nu\rangle}{\langle v(u),\nu\rangle}&\leq& \dfrac{\langle\psi'_0(v(u))v(u),\nu\rangle}{\langle v(u),\nu\rangle}
\leq c_1\|\phi\|_\infty\dfrac{\langle v(u-1),\nu\rangle}{\langle v(u),\nu\rangle}\langle\psi_0'(v(u))\phi,\nu\rangle\\
 &\leq& c_1\|\phi\|_\infty C^{*}\langle\psi_0'(v(u))\phi,\nu\rangle.
\end{eqnarray*}
By dominated convergence theory, we have
\[
\lim_{u\rightarrow\infty}\langle\psi_0'(v(u))\phi,\nu\rangle=0.
\]
Then \eqref{svfcondition} follows.

If
\begin{equation}\label{svfcondition}
\lim_{u\rightarrow\infty}\dfrac{\langle\psi_0(v(u)),\nu\rangle}{\langle v(u),\nu\rangle}=0,
\end{equation}
then $\lim_{t\rightarrow\infty}g(t+s)/g(t)=1$ for any $s>0$.  Furthermore for any $s>0$
\[
\lim_{t\rightarrow\infty}\dfrac{\langle v(t+s),\nu\rangle}{\langle v(t),\nu\rangle}=e^{\lambda s}
\]
follows.
%%%%DELETE%%%%%%%%%%%%%%%%%%%%%%%%%%%%%%%%%%%%%%%%%%%%%%%%%%%%%%%%%%%%
%\begin{align}
	%\int_0^s e^{-\lambda u}\langle (\psi+\beta)(\cdot, v(t+u)),\hat\phi\rangle du
%	\int_0^s e^{-\lambda u}\big\langle \psi_0\big(\cdot, v(t+u,\cdot)\big),\nu\big\rangle~du
	%&\leq \int_0^s e^{-\lambda u}\langle (\psi+\beta)'(\cdot, v(t))v(t+u,\cdot),\hat\phi\rangle du
%	&\leq \int_0^s e^{-\lambda u}\big\langle\psi_0'\big(\cdot, v(t+u,\cdot)\big)v(t+u,\cdot),\nu\big\rangle~du%
%	\\&\leq Cs\langle (\psi_0+\beta)'(\cdot, v(t))\phi,\hat\phi\rangle \langle v(t,\cdot),\hat\phi\rangle .
%\end{align}
%	The last inequality comes from the inequality \eqref{eq:upp}.
%	{\bf (ZS: I can't see the last inequality. I discussed with RL, but I still can't see how to bound the integral while $u$ is near $0$. )}
%	Observe the equation \eqref{ext equ in}. On one hand,
%\[
%	\frac{\langle v(t+s,\cdot), \hat\phi\rangle }{e^{\lambda s}\langle v(t,\cdot),\hat\phi\rangle }\leq 1.
%\]
%	On the other hand,
%\[
%	\frac{\langle v(t+s,\cdot), \hat\phi\rangle }{e^{\lambda s}\langle v(t,\cdot),\hat\phi\rangle }\geq 1-Cs\langle (\psi+\beta)'(\cdot, %v(t))\phi,\hat\phi\rangle .
%\]
%	Because $\lim_{t\rightarrow\infty} v(t,x)=0$, and $\lim_{\lambda\rightarrow 0+}(\psi+\beta)'(x,\lambda)=0$, we have for any $s>0$,
%\[
%	\lim_{t\rightarrow\infty}\frac{\langle v(t+s,\cdot), \hat\phi\rangle }{\langle v(t,\cdot),\hat\phi\rangle }=e^{\lambda s}.
%\]
%%%%END%%%%%%%%%%%%%%%%%%%%%%%%%%%%%%%%%%%%%%%%%%%%%%%%%%%%%%%%%%%%%%%%%
	It follows from the second inequality of \eqref{eq:upp} that for any $\varepsilon>0$ there is $s_1>0$ such that when $s>s_1$, for any $x\in E$
\[
	v(t+s,x)\leq (1+\varepsilon)e^{\lambda s}\phi(x)\langle v(t,\cdot),\nu\rangle .
\]
	As a consequence,
\[
	\limsup_{t\rightarrow\infty}\frac{v(t+s,x)}{\langle v(t+s,\cdot),\nu\rangle }\leq (1+\varepsilon)\phi(x).
\]
	By the arbitrary of $\varepsilon>0$,
\begin{equation}\label{limsup}
	\limsup_{t\rightarrow\infty}\frac{v(t,x)}{\langle v(t,\cdot),\hat\phi\rangle }\leq \phi(x).
\end{equation}
{\color{red}
Rewrite \eqref{eq: equation for vt} as for $ t> 0, s>1, x\in E$,
\begin{eqnarray*}
&&v(t+s,x)=P^\beta_s\big(v(t,\cdot)\big)(x)-\int_0^sP^\beta_{s-u}\Big(\psi_0\big(\cdot, v(t+u,\cdot)\big)\Big)(x)~du\\
&=&P^\beta_s\big(v(t,\cdot)\big)(x)-\int_0^{s-1}P^\beta_{s-u}\Big(\psi_0\big(\cdot, v(t+u,\cdot)\big)\Big)(x)~du-\int_{s-1}^sP^\beta_{s-u}\Big(\psi_0\big(\cdot, v(t+u,\cdot)\big)\Big)(x)~du\\
&=&I^1_{t,s}-I_{t,s}^2-I_{t,s}^3.
\end{eqnarray*}
Applying \eqref{IU} again,
\begin{equation*}
I^1_{t,s}\geq (1-ce^{-\rho s})\phi(x)e^{\lambda s}\langle v(t,\cdot),\nu\rangle.
\end{equation*}
Then
\begin{equation}\label{lower 1}
\liminf_{t\to\infty}\dfrac{I^1_{t,s}}{\langle v(t+s,\cdot),\nu\rangle}\geq (1-ce^{-\rho s})\phi(x).
\end{equation}
And
\begin{eqnarray*}
I^2_{t,s}&\leq& (1+c)e^{\lambda s}\phi(x)\int_0^s e^{-\lambda u}\langle \psi_0(\cdot, v(t+u,\cdot)),\nu\rangle du\\
&=&(1+c)\phi(x)\left(e^{\lambda s}\langle v(t,\cdot),\nu\rangle-\langle v(t+s,\cdot),\nu\rangle\right).
\end{eqnarray*}
Therefore,
\begin{eqnarray}\label{lower 2}
\limsup_{t\to\infty}\dfrac{I^2_{t,s}}{\langle v(t+s,\cdot),\nu\rangle}&\leq& (1+c)\phi(x)\lim_{t\to\infty}\dfrac{e^{\lambda s}\langle v(t,\cdot),\nu\rangle-\langle v(t+s,\cdot),\nu\rangle}{\langle v(t+s,\cdot),\nu\rangle}=0.
\end{eqnarray}
By variable substitution in $I_{s,t}^3$,
\begin{eqnarray*}
I^3_{t,s}=\int_0^1 P^{\beta}_u\psi_0(\cdot, v(t+s-u,\cdot))(x)du.	
\end{eqnarray*}
When $t$ and $s$ are sufficiently large, it has been shown that there is a constant $C>0$ such that
\[
v(t+s-u,x)\leq C\langle v(t+s-u,\cdot),\nu\rangle\phi(x),\quad x\in E.
\]
\begin{eqnarray*}
I^3_{t,s}&\leq& \int_0^1P^{\beta}_u\psi_0(\cdot, C\langle v(t+s-u,\cdot),\nu\rangle\phi(\cdot))du\\
&\leq & C\int_0^1P_u^{\beta}\left(\phi\psi_0'(\cdot, C\langle v(t+s-1,\cdot),\nu\rangle\|\phi\|_{\infty})(x)\right)du\cdot\langle v(t+s-1,\cdot),\nu\rangle.
\end{eqnarray*}
Since the measure $\int_0^1 du p_u^{\beta}(x,y)\phi(y)m(dy)$ on $E$ is a finite measure.  By dominated convergence theory, it is obtained that
\begin{equation*}
\lim_{t\rightarrow\infty}\int_0^1P_u^{\beta}\left(\phi\psi_0'(\cdot, C\langle v(t+s-1),\nu\rangle\|\phi\|_{\infty})(x)\right)du=0, \quad \text{for all}\, x\in E.
\end{equation*}
Thus
\begin{eqnarray}\label{lower 3}
&&\limsup_{t\to\infty}\dfrac{I^3_{t,s}}{\langle v(t+s,\cdot),\nu\rangle}\\
&\leq&  C\limsup_{t\rightarrow\infty}\int_0^1P_u^{\beta}\left(\phi\psi_0'(\cdot, C\langle v(t+s-1,\cdot),\nu\rangle\|\phi\|_{\infty})(x)\right)du\cdot
\limsup_{t\rightarrow\infty}\dfrac{\langle v(t+s-1,\cdot),\nu\rangle}{\langle v(t+s,\cdot),\nu\rangle}\\
&=&0.
\end{eqnarray}
 Combining \eqref{lower 1}\eqref{lower 2}and \eqref{lower 3}, for any $s>1$, it follows that
\begin{eqnarray*}
	&&\liminf_{t\rightarrow\infty}\frac{v(t,x)}{\langle v(t,\cdot),\nu\rangle }
=\liminf_{t\rightarrow\infty}\frac{v(t+s,x)}{\langle v(t+s,\cdot),\nu\rangle }\\
&\geq& \liminf_{t\rightarrow\infty}\frac{I_{t,s}^1}{\langle v(t+s,\cdot),\nu\rangle }-\limsup_{t\to\infty}\dfrac{I^2_{t,s}}{\langle v(t+s,\cdot),\nu\rangle}-\limsup_{t\to\infty}\dfrac{I^3_{t,s}}{\langle v(t+s,\cdot),\nu\rangle}\\
	&\geq & (1-ce^{-\rho s})\phi(x).
\end{eqnarray*}
}
	Letting $s\to\infty$,
\begin{equation}\label{liminf}
	\liminf_{t\rightarrow\infty}\frac{v(t,x)}{\langle v(t,\cdot),\nu\rangle }\geq \phi(x).
\end{equation}
	Since the discussions for \eqref{limsup} and \eqref{liminf} come from the IU properties, so the convergence is uniform for $x$.  Therefore uniformly for $x\in E$,
\[
	\lim_{t\rightarrow\infty}\frac{v(t,x)}{\langle v(t,\cdot),\nu\rangle }=\phi(x).
\]
	Consequently uniformly for $x\in E$,
\[
	\lim_{t\rightarrow\infty}\frac{v(t+s,x)}{\langle v(t,\cdot),\nu\rangle }
	=\lim_{t\rightarrow\infty}\frac{v(t+s,x)}{\langle v(t+s,\cdot),\nu\rangle }\lim_{t\rightarrow\infty}\frac{\langle v(t+s,\cdot),\nu\rangle }{\langle v(t,\cdot),\nu\rangle }
	=\phi(x)e^{\lambda s}.
\]
\end{proof}
	In the arguments of the proof of the above lemma, we did not use the fact that $v(t,x)$ is related to the extinction probability.  We only use the existence of the solution \eqref{eq: diff equ} and it is integrable with respect to $\hat\phi(x)m(dx)$ after some time $t_1>0$.  If $f\in\mathcal B_b^+(E)$, then the problem \eqref{eq: diff equ} has unique bounded solution $V_tf(x)$.  Thus the limit result \eqref{one point ratio limit} holds for these solutions.  Meanwhile, the main point in the proof above is the uniform convergence of the expectation semigroup $P^{\beta}_t$ which is independent of the initial value.   Thus the limit result can be generalized as follows. We will give the result directly without proof.

\begin{cor}\label{general rate}
	For any $\mu\in\mathcal M(E)\backslash\{0\}$,
\begin{equation}\label{ext con}
	\lim_{t\rightarrow\infty}\frac{\langle v(t+s,\cdot),\mu\rangle }{\langle v(t,\cdot),\nu\rangle }=\langle \phi,\mu\rangle e^{\lambda s}.
\end{equation}
	For any $f\in\mathcal B_b^+(E)$, $V_tf(\cdot)$ is the solution to \eqref{eq: diff equ} with initial condition $f$. Then for any $s>0$,
\begin{equation}\label{ratio limits}
	\lim_{t\rightarrow\infty}\frac{\langle V_{t+s}f,\mu\rangle }{\langle V_{t}f,\nu\rangle }=\langle \phi,\mu\rangle e^{\lambda s}.
\end{equation}
\end{cor}

\begin{lem}\label{ratio limits 2}
	For any $f\in\mathcal B_b^+(E)$, let $V_tf(x)$ is the solution to \eqref{eq: diff equ}.  Then there is a nonnegative function $G(f)$ such that $G(v(t,\cdot))=t$ and such that for any $\mu\in\mathcal M(E)\backslash\{0\}$,
\begin{equation}\label{App of G}
	\lim_{t\rightarrow\infty}\frac{\langle V_tf,\mu\rangle }{\langle v(t,\cdot),\mu\rangle }=e^{\lambda G(f)}.
\end{equation}
\end{lem}

\begin{proof}
	Thanks to \eqref{ext con} and \eqref{ratio limits}, we only need to prove
\begin{equation}\label{spec ratio}
	\lim_{t\rightarrow\infty}\frac{\langle V_tf,\nu\rangle }{\langle v(t,\cdot),\nu\rangle }=e^{\lambda G(f)}.
\end{equation}
	Note that $\nu$ is an eigenfunction of {\color{red}$(\widehat P^\beta_t)$}, thus for any $t,s>0$,
\begin{equation}\label{ident: monotone}
 	e^{-\lambda(t+s)}\langle v(t+s,\cdot),\nu\rangle =e^{-\lambda t}\langle v(t,\cdot),\nu\rangle -\int_0^se^{-\lambda(t+u)}\langle (\psi+\beta)(v(t+u,\cdot)),\nu\rangle du.
\end{equation}
 	Therefore, $\langle v(t,\cdot),nu\rangle $ is a strictly decreasing continuous function of $t$.  Moreover it has limits
\[
	\lim_{t\rightarrow 0+}\langle v(t,\cdot),\nu\rangle =\infty,\qquad\quad \lim_{t\rightarrow\infty}\langle v(t,\cdot),\nu\rangle =0.
\]
	Therefore for any nonnegative Borel measurable function $f$ on $E$, there is a real number $G(f)$ such that
\begin{equation}\label{def of G}
	\langle v(G(f),\cdot),\nu\rangle =\langle f,\nu\rangle .
\end{equation}
	Specially, taking $f(\cdot)=v(t,\cdot)$ for some $t>0$, then $\langle v(G(v(t)),\cdot),\nu\rangle =\langle v(t,\cdot),\nu\rangle $.
	Since  $\langle v(t,\cdot),\nu\rangle $ is strictly decreasing, so $G(v(t,\cdot))=t$.\\
	Define the solution $V_tf(x)$ to equation \eqref{eq: diff equ} with initial condition $f$ and consider the function $v(t+G(f),x)$ on $E$. Then
\[
	\langle V_tf,\nu\rangle =e^{\lambda t}\langle f,\nu\rangle -\int_0^te^{\lambda(t-s)}\langle (\psi+\beta)(V_sf)),\nu\rangle ds.
\]
	And
\begin{eqnarray*}
	\langle v(t+G(f),\cdot),\nu\rangle &=&e^{\lambda t}\langle v(G(f),\cdot),\nu\rangle -\int_0^te^{\lambda(t-s)}\langle (\psi+\beta)(v(s+G(f),\cdot)),\nu\rangle ds\\
	&=&e^{\lambda t}\langle f,\nu\rangle -\int_0^te^{\lambda (t-s)}\langle (\psi+\beta)(v(s+G(f),\cdot)),\nu\rangle ds
\end{eqnarray*}
	Let $w(t,x)=V_tf(x)-v(t+G(f),x)$. Then
\[
	\langle w(t,\cdot),\nu\rangle =\int_0^te^{\lambda(t-s)}\langle (\psi+\beta)(v(s+G(f),\cdot))-(\psi+\beta)(V_sf,\cdot),\nu\rangle ds
\]
	Using mean value theorem, for any $x\in E$ and $s\in [0,t]$, there is $\xi_s(x)$ between $v(s+G(f),x)$ and $V_sf(x)$ such that
\[
	(\psi+\beta)(v(s+G(f),x))-(\psi+\beta)(V_sf(x))=-(\psi+\beta)'(x, \xi_s)w(s,x).
\]
	Since $v(s+G(f),x)$ and $V_sf(x)$ are both bounded on $[0,t]\times E$, writing $C'>0$ is their one upper bound,  and $\psi'(x, \lambda)$ is increasing with respect to $\lambda$ and bounded on $D$, so for any $t>0$,
\begin{eqnarray*}
	|\langle w(t,\cdot),\nu\rangle |=|\int_0^te^{\lambda(t-s)}\langle -(\psi+\beta)'(\cdot, \xi_s)w(s,\cdot),\nu\rangle ds|\leq \|\psi'(\cdot,C')\|_\infty\int_0^te^{\lambda(t-s)}|\langle w(s,\cdot),\nu\rangle |ds.
\end{eqnarray*}
	From Gronwell inequality, $\langle w(t,\cdot),\nu\rangle =0$.  Replacing $\langle V_tf,\nu\rangle $ in \eqref{spec ratio} by $\langle v(t+G(f),\cdot),\nu\rangle $, then
\[
	\lim_{t\rightarrow\infty}\frac{\langle V_tf,\nu\rangle }{\langle v(t,\cdot),\nu\rangle }=
	\lim_{t\rightarrow\infty}\frac{\langle v(t+G(f),\cdot),\nu\rangle }{\langle v(t,\cdot),\nu\rangle }=e^{\lambda G(f)},
\]
	from \eqref{ext con}.  As a result, for any $\mu\in\mathcal M(E)\backslash\{0\}$,
\[
	\lim_{t\rightarrow\infty}\frac{\langle V_tf,\mu\rangle }{\langle v(t,\cdot),\mu\rangle }
	=\lim_{t\rightarrow\infty}\frac{\langle V_tf,\mu\rangle }{\langle V_tf,\nu\rangle }
	\lim_{t\rightarrow\infty}\frac{\langle v(t+G(f),\cdot),\nu\rangle }{\langle v(t,\cdot),\mu\rangle }=e^{\lambda G(f)}.
\]
\end{proof}

\subsection{Proof of main theorems}
\begin{proof}[{\bf Proof of Theorem \ref{thm: distribution of zeta}}]
	For any finite measure $\mu\in \mathcal M(E)$,
\[
	\mathbb P_\mu(\zeta=\infty)=\lim_{t\rightarrow\infty}\mathbb P_\mu(\zeta>t)=1-e^{-\lim_{t\rightarrow\infty}\langle v(t,\cdot),\mu\rangle }=0,
\]
	follows from Lemma\ref{lem:extinc}.
	Meanwhile, thanks to \eqref{ext con}, for any $\mu,\tilde\mu\in \mathcal M(E)\backslash\{0\}$ and $s\in\mathbb R$,
\[
	\lim_{t\rightarrow\infty}\frac{\mathbb P_{\mu}(\zeta>t+s)}{\mathbb P_{\tilde\mu}(\zeta>t)}=\lim_{t\rightarrow\infty}\frac{1-e^{-\langle v(t+s,\cdot),\mu\rangle }}{1-e^{-\langle v(t,\cdot),\tilde\mu\rangle }}
	=\lim_{t\rightarrow\infty}\frac{\langle v(t+s,\cdot),\mu\rangle }{\langle v(t,\cdot),\tilde\mu\rangle }=\frac{\langle \phi,\mu\rangle }{\langle \phi,\tilde\mu\rangle }e^{\lambda s}.
\]

	We are left to prove the item $3$.  Integrating the both sides of \eqref{eq: equation for vt} with respect to $\mu\in\mathcal M(E)$, then multiplying the both sides with  $e^{-\lambda(s+t)}$, it follows that
\begin{equation}\label{ext equ int}
	e^{-\lambda(s+t)}\langle v(t+s,\cdot), \mu\rangle =e^{-\lambda t}\langle v(t,\cdot),\mu\rangle -e^{-\lambda t}\int_0^s e^{-\lambda u}\langle (\psi+\beta)(\cdot, v(t+u)),\mu\rangle du.
\end{equation}
	Thus $e^{-\lambda t}\langle v(t,\cdot), \mu\rangle$ is an decreasing function of $t$. Let
\begin{equation}\label{def of k}
 	\lim_{t\to\infty}e^{-\lambda t}\langle v(t,\cdot), \nu\rangle=k,
\end{equation}
 	then $k\in[0,\infty)$. Moreover, for any $x\in E$, \eqref{one point ratio limit} can yield
\[
	\lim_{t\rightarrow\infty}e^{-\lambda t}v(t,x)=\lim_{t\to\infty}e^{-\lambda t}\langle v(t,\cdot),\nu\rangle \lim_{t\to\infty}\dfrac{v(t,x)}{\langle v(t,\cdot),\nu\rangle }=k\phi(x).
\]
	Consequently,
\begin{eqnarray*}
	&&\lim_{t\rightarrow\infty}e^{-\lambda t}\mathbb P_{x}(\zeta>t)=\lim_{t\rightarrow\infty}e^{-\lambda t}\left(1-e^{\langle v(t,x)}\right)\\	
	&=&\lim_{t\rightarrow\infty}\dfrac{1-e^{ v(t,x)}}{v(t,x)}\lim_{t\to\infty}e^{-\lambda t}v(t,x)\\
	&=& k\phi(x).
\end{eqnarray*}

	Next, we investigate the value of $k$.  By the definition of the probability $\widetilde{\mathbb P}_\nu$ for $\nu(dx)=\hat\phi(x)m(dx)$, we obtain that for any $t\ge 0,$
\begin{align}\label{subcritical equality}
 	\langle\phi,\mu\rangle ^{-1}e^{-\lambda t}\mathbb P_\nu(\zeta>t)=\widetilde{\mathbb P}_\nu\left(\frac{1}{\langle\phi, X_{t}\rangle }\right)\hspace*{3.4 true cm}\\
 	=\mathbb Q_{\nu}\left[\frac{1}{\langle\phi, X_{t}\rangle +\sum_{\sigma\in(0, t]\bigcap\mathcal D^{\mathrm m}}\langle \phi, {\color{red}X_{t-\sigma}^{{\mathrm m},\sigma}\rangle +\sum_{\sigma\in (0, t]\bigcap \mathcal D^{\mathrm n}}\langle \phi, X_{t-\sigma}^{{\mathrm n}, \sigma}\rangle} }\right]
\end{align}

	When the initial value of $X$ is $\nu(dx)$, the initial distribution of the spine under $\mathbb Q_\nu$ is $\phi\cdot\nu$.
	From Lemma \ref{lem:reverse of the spine}, we have
\begin{eqnarray*}\label{duality}
    &&  \mathbb Q_{\nu}\left[\frac{1}{\langle\phi, X_{t}\rangle +\sum_{\sigma\in(0, t]\bigcap\mathcal D^{\mathrm m}}\langle \phi,{\color{red} X_{t-\sigma}^{{\mathrm m},\sigma}\rangle +\sum_{\sigma\in (0, t]\bigcap \mathcal D^{\mathrm n}}\langle \phi, X_{t-\sigma}^{{\mathrm n}, \sigma}\rangle} }\right]\\
    &=&\widehat{\mathbb Q}_{\nu}\left[\frac{1}{\langle\phi, X_{t}\rangle +\sum_{\sigma\in(0, t]\bigcap\mathcal D^{\mathrm m}}\langle \phi, \widehat X_{\sigma}^{{\mathrm m},\sigma}\rangle +\sum_{\sigma\in (0, t]\bigcap \mathcal D^{\mathrm n}}\langle \phi, \widehat X_{\sigma}^{{\mathrm n},\sigma}\rangle }\right]
\end{eqnarray*}
    Since $\lim_{t\to\infty}\langle\phi, X_{t}\rangle=0$ and $\sum_{\sigma\in(0, t]\bigcap\mathcal D^{\mathrm m}}\langle \phi, \widehat X_{\sigma}^{{\mathrm m},\sigma}\rangle +\sum_{\sigma\in (0, t]\bigcap \mathcal D^{\mathrm n}}\langle \phi, \widehat X_{\sigma}^{{\mathrm n},\sigma}\rangle $ is increasing with respect to $t$ having almost sure limit as $t\to\infty$ as well.  Therefore, by dominated convergence theory,
\begin{equation}\label{cons}
	k=\widehat{\mathbb Q}_{\nu}\left[\frac{1}{\sum_{\sigma\in\mathcal D^{\mathrm m}}\langle \phi, \widehat X_{\sigma}^{{\mathrm m},\sigma}\rangle +\sum_{\sigma\in \mathcal D^{\mathrm n}}\langle \phi, \widehat X_{\sigma}^{{\mathrm n},\sigma}\rangle }\right].
\end{equation}
	For the continuum immigration part,
\[
	\widehat{\mathbb Q}_{\nu}\left(\sum_{\sigma\in \mathcal D^{\mathrm n}}\langle \phi, \widehat X_{\sigma}^{{\mathrm n},\sigma}\rangle \right)=\int_0^\infty e^{\lambda s}\langle 2\alpha\phi, \phi\nu\rangle  ds=\frac{\langle 2\alpha\phi, \phi\nu\rangle}{-\lambda}<\infty.
\]
	Therefore, $\sum_{\sigma\in \mathcal D^{\mathrm n}}\langle \phi, \widehat X_{\sigma}^{{\mathrm n},\sigma}\rangle$ is finite almost surely.

	For the discrete immigration part, let $\mathcal G=\sigma(Y, {\color{red}( m_\sigma)_{\sigma\in\mathcal D^{\mathrm m}}})$.  When  $\int_E\hat{\phi}(y)l(y)m(dy)<\infty$, by Lemma \ref{import lemma},
\[
	\widehat{\mathbb Q}_{\nu}\left(\sum_{\sigma\in \mathcal D^{\mathrm m}}\langle \phi, \widehat X_{\sigma}^{{\mathrm m}}\rangle\Big|\mathcal G \right)
	=\sum_{\sigma<\infty}m_\sigma e^{\lambda \sigma}\phi(Y_{\sigma})<\infty,  \qquad\qquad \widehat{\mathbb Q}_{\nu}-{\mathrm a.s.}
\]
	Thus $k>0$.

	We claim that when $\int_D\hat\phi(y)l(y)m(dy)=\infty$,
\begin{equation}\label{infty}
	\sum_{\sigma\in \mathcal D^{\mathrm m}}\langle \phi, \widehat X_{\sigma}^{{\mathrm m},\sigma}\rangle =\infty,\quad\qquad  \widehat{\mathbb Q}_\nu-{\mathrm a.s.}
\end{equation}
	In the proof of \cite[Lemma $3.2$]{LiuRenSong2009Llog}.  It is shown that for any $N>0$,
\begin{equation}\label{inf}
	\int_0^\infty dt\int_{\phi(Y_t)^{-1}e^{Nt}}^\infty rn(Y_t,dr)
	=\infty,\quad \widehat{\mathbb Q}_\nu-{\mathrm a.s.}
\end{equation}
	Fix a path of $(Y_t)$.  And define stochastic
	time sequence
\[
	\tau_0:=\left\{t>0; m_t>\phi(Y_t)^{-1}e^{Nt}\right\},\,
	\tau_{i+1}:=\left\{t>\tau_i;\, m_t>\phi(Y_t)^{-1}e^{Nt}\right\},\, i=0,1,\cdots
\]
	Then $\tau_i<\infty$, $i=1,2,\ldots$ almost surely.
	If we can prove that $\sum_{i=0}^\infty I_{\left\{\langle\phi, \widehat X^{{\mathrm m},\tau_i}_{\tau_i}\rangle  >\varepsilon\right\}}=\infty,$ for some $\varepsilon>0,$ then our claim holds.  Similar to the argument for the proof of the second assertion of \cite[Lemma $2.2$]{LiuRenSong2009Llog}, we just need to prove
\[
	\widehat{\mathbb Q}_\nu
	\left[\left.\sum_{i=0}^\infty I_{\left\{\langle\phi, \widehat X^{{\mathrm m},\tau_i}_{\tau_i}\rangle  >\varepsilon\right\}} \right| Y\right]=\infty,\quad {\color{red}\widehat\Pi_{\phi\cdot\nu}}{\mathrm -a.s.}
\]
	Since given the spatial motion of the spine the immigration process is a Poisson point process, therefore,
\[
	\widehat{\mathbb Q}_\nu\left[\left.\sum_{i=0}^\infty I_{\left\{\langle\phi, \widehat X^{{\mathrm m},\tau_i}_{\tau_i}\rangle  >\varepsilon\right\}}\right| Y\right]=\int_0^\infty dt\int_{\phi(Y_t)^{-1}e^{Nt}}^\infty rn(Y_t, dr)\mathbb{P}_{r\delta_{Y_t}}\big(\langle\phi, X_t \rangle >\varepsilon\big).
\]
	If we can prove for any time $t>0$,
\begin{eqnarray}\label{last point}
	\inf_{r\geq \phi(x)^{-1}e^{Nt}, x\in E}\mathbb P_{r\delta_x}\big(\langle\phi, X_t
	\rangle >\varepsilon\big)>0,
\end{eqnarray}
	then from \eqref{inf}, \eqref{infty} is obtained.  By Chebyshev inequality,
\begin{eqnarray*}
	&&\mathbb P_{r\delta_x}\big(\langle\phi, X_t\rangle >\varepsilon\big)=\mathbb P_{r\delta_x}\left(e^{-\langle\phi, X_t\rangle }<e^{-\varepsilon}\right)\\
	&=&1-\mathbb P_{r\delta_x}\left(e^{-\langle\phi, X_t
	\rangle }\geq e^{-\varepsilon}\right)\geq 1-e^{\varepsilon }\mathbb P_{r\delta_x}e^{-\langle\phi, X_t\rangle }\\
	&=&1-e^{\varepsilon }e^{-rV_t\phi(x)},
\end{eqnarray*}
	where, $V_t\phi(x)$ is a classical solution to evolution equation
\begin{equation}\label{eq diff}
\begin{cases}
	\dfrac{\partial U}{\partial t}=AU-\beta(x)U-\psi(x, U),& x\in E, t>0;\\
	U(0,x)=\phi(x),& x\in E.\\
\end{cases}
\end{equation}
	Define a function on $[0,\infty)\times E,$ $V(t,x)=e^{-Nt}\phi(x).$  Then
\[
	(A-\beta)V(t,x)-\psi(x, V)-\frac{\partial V(t,x)}{\partial t}=(\lambda +N)V(t,x)-(\psi+\beta)(x,V).
\]
	It is obvious that $V$ is bounded on $[0,\infty)\times E$.  So we can choose $N$ large enough such that $(\lambda+N)V(t,x)-\psi(x,V)\geq 0$ on $[0,\infty)\times E.$
	Meanwhile, $V(0,x)=\phi(x)$.  Applying comparison theorem for semilinear equation, we obtain that $V_t\phi(x)\geq V(t,x)$.  So
\begin{eqnarray*}
	&&\inf_{r\geq \phi(x)^{-1}e^{Nt}, x\in E}\mathbb P_{r\delta_x}\big(\langle\phi, X_t \rangle >\varepsilon\big)\geq \inf_{r\geq \phi(x)^{-1}e^{Nt}, x\in D} 1-e^{\varepsilon}e^{-rV_t\phi(x)}\\
    &&\geq \inf_{r\geq \phi(x)^{-1}e^{Nt}, x\in E} 1-e^{\varepsilon}e^{-rV(t,x)}
     \geq 1-e^{\varepsilon-1}.
\end{eqnarray*}
	Choose $0<\varepsilon<1.$  The \eqref{last point} is obtained. And
\eqref{infty} follows. Thus $k=0$.
\end{proof}

\begin{rem}
 From the theorem we can see that $v(t,x)$  has the asymptotic behavior as $t\to\infty$
 \[
 v(t,x)\sim \phi(x)a(t)e^{\lambda t},
 \]
 where $\lim_{t\rightarrow\infty}a(t+s)/a(t)=1$ for all $s>0$.  When $\int_E\hat\phi(y)l(y)m(dy)<\infty$, $\lim_{t\rightarrow\infty}a(t)=k$, which is given in \eqref{def of k} or \eqref{cons}.  When $\int_E\hat\phi(y)l(y)m(dy)=\infty$, $\lim_{t\rightarrow\infty}a(t)=0$.
 \end{rem}
%%-------------------------------------------------------------------------------------------------------------------
%--------------------------------------------------------------------------------------------------------------------------


\begin{proof}[{\bf Proof of Theorem \ref{thm: qsd thm}}]
For any nonzero $\mu\in \mathcal M(E)$, and any $f\in\mathcal B_b^+(E)$,
\begin{eqnarray*}
\mathbb P_\mu\left(\left.1-\exp\{-\langle f, X_t\rangle \}\right|\zeta>t\right)= \frac{\mathbb P_\mu\left(1-\exp\{-\langle f, X_t\rangle \};\zeta>t\right)}{\mathbb P_\mu(\zeta>t)}=\frac{1-e^{-\langle V_tf,\mu\rangle }}{1-e^{-\langle v(t,\cdot),\mu}\rangle }.
\end{eqnarray*}
Because $\lim_{t\rightarrow\infty}\langle V_tf(\cdot),\mu\rangle =\lim_{t\rightarrow\infty}\langle v(t,\cdot),\mu\rangle =0$, due to lemma \ref{ratio limits 2},
\[
\lim_{t\rightarrow\infty}\mathbb P_\mu\left(\left.1-\exp\{-\langle f, X_t\rangle \}\right|\zeta>t\right)=\lim_{t\rightarrow\infty}\frac{\langle V_tf(\cdot),\mu\rangle }{\langle v(t,\cdot),\mu\rangle }=e^{\lambda G(f)}.
\]
Thus
\[
\lim_{t\rightarrow\infty}\mathbb P_\mu\left(\left.\exp\{-\langle f, X_t\rangle \}\right|\zeta>t\right)=1-e^{\lambda G(f)}.
\]
When $f=0$, $G(f)=\infty$. Thus $1-e^{\lambda G(f)}$ is a Laplace functional of probability on $\mathcal M(E)$. We denote by $\mathbf P^{\lambda}$ the corresponding probability, then
\[
\mathbf P^{\lambda}(e^{-\langle f,\omega\rangle })=1-e^{\lambda G(f)}.
\]
Let $M^{(\lambda)}\in (0,\infty)$ be the random variable satisfying for any $\theta>0$,
\[
\lim_{t\rightarrow\infty}\mathbb P_\mu\left(\left.\exp\{-\theta\langle \phi, X_t\rangle \}\right|\zeta>t\right)
=\mathbf P^{\lambda}(e^{-\theta M^{(\lambda)}})=1-e^{\lambda G(\theta\phi)}.
\]
Then for any $f\in \mathcal B_b^+(E)$,
\[
\mathbf P^{\lambda}(e^{-\langle f,M^{(\lambda)}\nu\rangle })
=\mathbf P^{\lambda}(e^{-\langle f,\nu\rangle M^{(\lambda)}})
=1-e^{\lambda G(\langle f,\nu\rangle \phi)}
\]
By the definition of the function $G$,
\[
\langle v(G(f),\cdot),\nu\rangle =\langle f,\nu\rangle =\langle f,\nu\rangle \langle \phi,\nu\rangle =\langle v(G(\langle f,\nu\rangle \phi),\cdot),\nu\rangle .
\]
Since $\langle v(t,\cdot),\nu\rangle $ is strictly decreasing with respect to $t$, so
\begin{equation}\label{iden:G}
G(f)=G(\langle f,\nu\rangle \phi).
\end{equation}
As a consequence, $\mathbf P^\lambda$ is concentrated on the set
$$
\mathcal S:=\{\mu\in\mathcal M(E); \mu(dx)=a\hat\phi(x)m(dx), a\geq 0\}.
$$
 Or we can say  $X_t|_{\zeta>t}$ converges to  $M^{(\lambda)}\hat\phi(x)m(dx)$ weakly under $\mathbb P_\mu$.


Therefore, the distribution of $M^{(\lambda)}\hat\phi(x)m(dx)$ under $\mathbf P^{\lambda }$ is the Yaglom distribution of $(X_t)$.  So it is the quasi-stationary distribution associated to $\lambda$.  The Laplace transform of $M^{(\lambda)}$ is given by
\begin{equation}\label{def of Y}
\mathbf P^{\lambda}(e^{-\theta M^{(\lambda)}})=1-e^{\lambda G(\theta\phi)}:=1-e^{\lambda B(\theta)},\qquad\quad\theta>0.
\end{equation}
According to \cite[Lemma $3.2$]{Lambert2007Quasi-stationary}, $1-e^{\gamma B(\theta)}$ for $\lambda\leq \gamma<0$ is also a Laplace transform of some probability measure on $(0,\infty)$.  Denote by $M^{(\gamma)}$ be the corresponding random variable.  Let $\mathbf P^{\gamma}$ be the distribution of $M^{(\gamma)}\hat\phi(x)m(dx)$.  We claim that $\mathbf P^{\gamma}$ is the quasi-stationary distribution of $X_t$ associated to the rate of mass decay $-\gamma$.  Since for any $f\in\mathcal B_b^+(E)$,
\[
\mathbf P^{\gamma}(e^{-\langle f,M^{(\gamma)}\nu\rangle })=\mathbf P^{\gamma}(e^{-\langle f,\nu\rangle M^{(\gamma)}})=1-e^{\gamma B(\langle f,\nu\rangle )}=1-e^{\gamma G(\langle f,\nu\rangle \phi)}=1-e^{\gamma G(f)}.
\]
Therefore,
\begin{eqnarray*}
\mathbf P^{\gamma}\mathbb P(\zeta>t)=1-\mathbf P^{\gamma}e^{-M^{(\gamma)}\langle v(t,\cdot),\nu\rangle }=1-(1-e^{\gamma G(v(t,\cdot))})=e^{\gamma t}.
\end{eqnarray*}
The last identity above comes from Lemma \ref{ratio limits 2}. Meanwhile, it follows from \eqref{iden:G} that
\[
\mathbf P^{\gamma}\mathbb P\left(1-e^{\langle f, X_t\rangle }\right)=1-\mathbf P^{\gamma}\left(e^{-\langle V_tf,\nu\rangle M^{(\gamma)}}\right)
=e^{\gamma G(V_tf)}=e^{\gamma G(\langle V_tf,\nu\rangle\phi)}.
\]
 We have shown that
\begin{equation}\label{ident: Vv}
\langle V_tf,\nu\rangle =\langle v(t+G(f),\cdot),\nu\rangle,
\end{equation}
and that
\begin{equation}\label{ident: Gv}
 G(\langle v(t+G(f),\cdot),\nu\rangle \phi)=G(v(t+G(f),\cdot))=t+G(f).
\end{equation}
Thus
\[
\mathbf P^{\gamma}\mathbb P\left(1-e^{\langle f, X_t\rangle }\right)=e^{\gamma(t+G(f))}.
\]
And furthermore,
\begin{eqnarray*}
\mathbf P^{\gamma}\mathbb P\left(e^{-\langle f, X_t\rangle }\big|\zeta>t\right)&=&1-\mathbf P^{\gamma}\mathbb P\left(1-e^{-\langle f, X_t\rangle }\big|\zeta>t\right)=1-e^{\gamma(t+G(f))}e^{-\gamma t}\\
&=&1-e^{\gamma G(f)}=\mathbf P^{\gamma}(e^{-\langle f,M^{(\gamma)}\nu\rangle }).
\end{eqnarray*}
Our claim is proved.  So far the proof of the existence of the QSD associated to $\gamma\in[\lambda,0)$ is complete.  Now let us discuss the uniqueness of the QSD.

  For $\gamma<0$, we assume there is a quasi-stationary distribution $\mathbf P$.  Then
\[
\mathbf P\mathbb P(\zeta>t)=1-\int_{\mathcal M(E)}e^{-\langle v(t,\cdot),\omega\rangle }\mathbf P(d\omega)=e^{\gamma t}.
\]
And for any nonnegative bounded Borel function $f$ on $E$,
\begin{eqnarray*}
\mathbf P\mathbb P\left(e^{-\langle f, X_t\rangle }\big|\zeta>t\right)&=&1-e^{-\gamma t}\mathbf P\mathbb P\left(1-e^{-\langle f, X_t\rangle }\right)\\
&=&1-e^{-\gamma t}\left(1-\int_{\mathcal M(E)}e^{-\langle V_tf,\omega\rangle }\mathbf P(d\omega)\right)\\
&=&\int_{\mathcal M(E)}e^{-\langle f,\omega\rangle }\mathbf P(d\omega).
\end{eqnarray*}
Recall that $\lim_{t\rightarrow\infty}\dfrac{\langle v(t+s,\cdot),\omega\rangle }{\langle v(t,\cdot),\nu\rangle }=\langle \phi,\omega\rangle e^{\lambda s} $ and that $\lim_{t\rightarrow\infty}\dfrac{\langle V_{t+s}f,\omega\rangle }{\langle V_tf,\nu\rangle }=\langle \phi,\omega\rangle e^{\lambda s}$ uniformly for all $\omega\in \mathcal M(E)\backslash\{0\}$.  For any $\varepsilon>0$, there is $T>0$, such that when $t>T$,
\[
(1-\varepsilon)\langle v(t,\cdot),\nu\rangle \langle \phi,\omega\rangle\leq \langle v(t,\cdot),\omega\rangle\leq (1+\varepsilon)\langle v(t,\cdot),\nu\rangle \langle \phi,\omega\rangle,\qquad \omega\in\mathcal M(E).
\]
Therefore, on one hand,
\begin{eqnarray*}
1&=&e^{-\gamma t}\left(1-\int_{\mathcal M(E)}e^{-\langle v(t,\cdot),\omega\rangle }\mathbf P(d\omega)\right)\\
&\leq &e^{-\gamma t}\left(1-\int_{\mathcal M(E)}e^{-\langle v(t,\cdot),\nu\rangle\langle\phi,\omega\rangle }\mathbf P(d\omega)\right)+e^{-\gamma t}\int_{\mathcal M(E)}\left(1-e^{-\varepsilon\langle v(t,\cdot),\nu\rangle\langle\phi,\omega\rangle  }\right)\mathbf P(d\omega)\\
&=&I_t+II_t.
\end{eqnarray*}
For the given $\varepsilon>0$, choose $s>0$, such that $e^{\lambda s}/2>\varepsilon$.  And there is $T_1>0$, such that $t>T_1$,
\[
\varepsilon\langle v(t,\cdot),\nu\rangle\langle\phi,\omega\rangle\leq \langle v(t+s,\cdot),\omega\rangle \quad\mbox{for all}\, \omega\in\mathcal M(E).
\]
So
\begin{eqnarray*}
0\leq II_t&\leq& e^{-\gamma t}\int_{\mathcal M(E)}\left(1-e^{-\varepsilon\langle v(t,\cdot),\nu\rangle\langle\phi,\omega\rangle  }\right)\mathbf P(d\omega)\\
&\leq& e^{-\gamma t}\int_{\mathcal M(E)}\left(1-e^{-\langle v(t+s,\cdot),\omega\rangle  }\right)\mathbf P(d\omega)\\
&=&e^{\gamma s}.
\end{eqnarray*}
We can see that when $\varepsilon\to0+$, $s$ can be chosen any large number.  Then
\[
\limsup_{t\to\infty}II_t\leq e^{\gamma s}.
\]
Let $s\to\infty$, $\lim_{t\to\infty}II_t=0$. Thus $\liminf_{t\to\infty}I_t\geq 1$.

On the other hand,
\begin{eqnarray*}
1&=&e^{-\gamma t}\left(1-\int_{\mathcal M(E)}e^{-\langle v(t,\cdot),\omega\rangle }\mathbf P(d\omega)\right)\\
&\geq &e^{-\gamma t}\left(1-\int_{\mathcal M(E)}e^{-\langle v(t,\cdot),\nu\rangle\langle\phi,\omega\rangle }\mathbf P(d\omega)\right)-e^{-\gamma t}\int_{\mathcal M(E)}\left(1-e^{-\varepsilon\langle v(t,\cdot),\nu\rangle\langle\phi,\omega\rangle  }\right)\mathbf P(d\omega)\\
&=&I_t-II_t.
\end{eqnarray*}
It follows that $\limsup_{t\to\infty}I_t\leq 1$. In conclusion,
\begin{equation}\label{limit1}
\lim_{t\rightarrow\infty}e^{-\gamma t}\left(1-\int_{\mathcal M(E)}e^{-\langle v(t,\cdot),\nu\rangle\langle\phi,\omega\rangle }\mathbf P(d\omega)\right)=1.
\end{equation}
Repeating the arguments for \eqref{limit1}, we can deduce the following limit
\begin{equation}\label{limit2}
\lim_{t\rightarrow\infty}e^{-\gamma t}\left(1-\int_{\mathcal M(E)}e^{-\langle V_tf,\nu\rangle \langle \phi,\omega\rangle }\mathbf P(d\omega)\right)=1-\int_{\mathcal M(E)}e^{-\langle f,\omega\rangle }\mathbf P(d\omega).
\end{equation}
Since $\langle V_tf,\nu\rangle =\langle v(t+G(f),\cdot),\nu\rangle $, combining \eqref{limit1} and \eqref{limit2}, we get
\begin{equation}\label{lap qsd}
\int_{\mathcal M(E)}e^{-\langle f,\omega\rangle }\mathbf P(d\omega)=1-e^{\gamma G(f)}.
\end{equation}
When  $\lambda\leq\gamma<0$, this Laplace functional is the same in form to that of the QSD $\mathbf P^{\gamma}$.  The uniqueness follows.  Now the assertions $(1)$ and $(2)$ are proved.

 If  $G(\theta\phi)$ is seen as a function of $\theta$ denoted by $B(\theta)$, then the definition \eqref{def of Y} of the
random variable $M^{(\lambda)}$ can yield
\[
0<\mathbf P^{\lambda}(M^{(\lambda)})=-\lambda\lim_{\theta\rightarrow 0+} e^{\lambda B(\theta)}B'(\theta).
\]
Observe that $\lim_{\theta\rightarrow 0+} B(\theta)=+\infty$, therefore $\lim_{\theta\rightarrow 0+}B'(\theta)=+\infty$. Taking derivative in \eqref{lap qsd},
and noting that $G(f)=G(\langle f,\nu\rangle\phi)$, for $r>0$,
\[
\int_{\mathcal M(E)}\langle f,\omega\rangle e^{-r\langle f,\omega\rangle }\mathbf P(d\omega)
=-\gamma \langle f,\nu\rangle  e^{\gamma B(r\langle f,\nu\rangle)}B'(r\langle f,\nu\rangle).
\]
Therefore,
\[
\int_{\mathcal M(E)}\langle f,\omega\rangle \mathbf P(d\omega)=-\gamma \langle f,\nu\rangle  \lim_{r\rightarrow 0+}e^{\gamma B(r)}B'(r)
\]
When $\gamma<\lambda$,
\[
\lim_{r\rightarrow 0+}e^{\gamma B(r)}B'(r)=0.
\]
Therefor if the QSD $\mathbf P^{\gamma}$ exists in the cases $\gamma<\lambda$, then
\[
\int_{\mathcal M(E)}\langle f,\omega\rangle \mathbf P^{\gamma}(d\omega)=0.
\]
So there is no QSD associated to the rate of mass decay $-\gamma$, for $\gamma<\lambda$.
\end{proof}

%%%-------------------------------------------------------------------------------------------------------------------------------------------------
{\color{blue}According to the definition for QSD in \cite{ChampagnatVillemonais2018Convergence}, there should be some probability measure $\mathbf Q^\gamma$ on $\mathcal M(E)$, $\gamma\in[\lambda, 0)$ such that
\[
 \mathbf Q^{\gamma}\mathbb P(X_t \in \cdot | \zeta > t) \xrightarrow[t\to \infty]{w} {\mathbf P^\gamma}(\cdot).
\]
From the proof above, we can find a class of $\mathbf Q^\gamma$, $\gamma\in (\lambda, 0)$, for $X$.
\begin{prop}
Let $Z^{(\alpha)}$ be a random variable whose Laplace exponent is proportional to $\Phi(\lambda)=\lambda^{\alpha}$ for $\alpha\in(0,1)$.  Let $\gamma=\alpha\lambda$ and $\mathbf Q^{\gamma}$ is the distribution of $Z^{(\alpha)}\hat\phi(x)m(dx)$.  Then
\[
 \mathbf Q^{\gamma}\mathbb P(X_t \in \cdot | \zeta > t) \xrightarrow[t\to \infty]{w} {\mathbf P^\gamma}(\cdot).
\]
\end{prop}
\begin{proof}
Since for any $f\in\mathcal B_b^+(E)$,
\[
\mathbf Q^{\gamma}\mathbb P\left(1-e^{\langle f, X_t\rangle }\right)=1-\mathbf Q^{\gamma}\left(e^{-\langle V_tf,\nu\rangle Z^{(\alpha)}}\right)
=1-e^{-c\langle V_tf,\nu\rangle^\alpha}=1-e^{-c\langle v(t+G(f),\cdot),\nu\rangle^\alpha},
\]
for some positive constant $c$. The last identity comes from \eqref{ident: Vv}.  Meanwhile,
\begin{eqnarray*}
\mathbf Q^{\gamma}\mathbb P(\zeta>t)=1-\mathbf Q^{\gamma}e^{-Z^{(\alpha)}\langle v(t,\cdot),\nu\rangle }=1-e^{-c\langle v(t,\cdot),\nu\rangle^\alpha}.
\end{eqnarray*}
Therefore, Thanks to corollary \ref{general rate},
\begin{eqnarray*}
&&\lim_{t\to\infty}\mathbf Q^{\gamma}\mathbb P\left(1-e^{\langle f, X_t\rangle }\big|\zeta>t\right)=\lim_{t\to\infty}\dfrac{1-e^{-c\langle v(t+G(f),\cdot),\nu\rangle^\alpha}}{1-e^{-c\langle v(t,\cdot),\nu\rangle^\alpha}}\\
&=&\lim_{t\to\infty}\dfrac{\langle v(t+G(f),\cdot),\nu\rangle^\alpha}{\langle v(t,\cdot),\nu\rangle^\alpha}= e^{\alpha\lambda G(f)}=1-(1-e^{\gamma G(f)})\\
&=&1-\mathbf P^{\gamma}(e^{-\langle f,\omega\rangle}).
\end{eqnarray*}
The proof is completed.
\end{proof}
}


%%--------------------------------------------------------------------------------------------------------------------------------------------------
%%----------------------------------------------------------------------------------------------------------------------------
\begin{proof}[Proof of Theorem \ref{thm: Qprocess}]
Assume $s>t$.  For any $A\in\mathscr F_t$, by the Markov property of $X$,
\[
\mathbb P_\mu(A|\zeta>s)=\dfrac{\mathbb P_\mu(A, \zeta>s)}{\mathbb P_\mu(\zeta>s)}=\dfrac{\mathbb P_\mu\big(\mathbb P_{X_t}(\zeta>s-t);A\big)}{\mathbb P_\mu(\zeta>s)},
\]
 Note that
\begin{eqnarray*}
\lim_{s\rightarrow\infty}\dfrac{\mathbb P_{X_t}(\zeta>s-t)}{\mathbb P_\mu(\zeta>s)}
&=&\lim_{s\rightarrow\infty}\dfrac{1-e^{-\langle v(s-t,\cdot),X_t\rangle }}{1-e^{-\langle v(s,\cdot),\mu\rangle }}
=\lim_{s\rightarrow\infty}\dfrac{\langle v(s-t,\cdot),X_t\rangle }{\langle v(s,\cdot),\mu\rangle }\\
&=& \dfrac{e^{-\lambda t}\langle \phi, X_t\rangle }{\langle \phi,\mu\rangle }=\dfrac{M_t}{\langle \phi,\mu\rangle }.
\end{eqnarray*}
The third identity follows from \eqref{ext con}.  From \eqref{eq:upp} we can get that there is a constant $\widetilde C>0$, such that for any $s>T>0$ and $x\in E$,
\[
v(s,x)\leq \widetilde C\phi(x)e^{\lambda T}\langle v(s-T,\cdot),\nu\rangle .
\]
Using the fact that $\lim_{x\rightarrow 0+}\dfrac{1-e^{-x}}{x}=1$, we choose $s$ sufficiently large such that
\[
1-e^{-\langle v(s,\cdot),\mu\rangle }>\frac{1}{2}\langle v(s,\cdot),\mu\rangle .
\]
Meanwhile since $1-e^{-x}\leq x$ for $x>0$, choosing $0<T_0<s-t$
\[
\dfrac{1-e^{-\langle v(s-t,\cdot),X_t\rangle }}{1-e^{-\langle v(s,\cdot),\mu\rangle }}
\leq \dfrac{2\langle v(s-t,\cdot),X_t\rangle }{\langle v(s,\cdot),\mu\rangle }\leq \dfrac{2\widetilde C\langle \phi,X_t\rangle e^{\lambda T_0}\langle v(s-t-T_0,\cdot),\nu\rangle }{\langle v(s,\cdot),\mu\rangle }.
\]
We already show in \eqref{ext con} that
\[
\lim_{s\rightarrow\infty}\dfrac{\langle v(s-t-T_0,\cdot),\nu\rangle }{\langle v(s,\cdot),\mu\rangle }
=e^{-\lambda(t+T_0)}\langle \phi,\mu\rangle ^{-1}.
\]
And $\langle \phi,X_t\rangle $ is integrable with respect to $\mathbb P_\mu$.  Thus by dominated convergence theorem,
\[
\lim_{s\rightarrow\infty}\mathbb P_\mu(A|\zeta>s)=\mathbb P_\mu\left(\frac{M_t}{\langle\phi,\mu\rangle };A\right)=\widetilde{\mathbb P}_\mu(A).
\]
\end{proof}
%%%%%%%%%%%%%%%%%%%%%%%%%%%%%%%%%%-----------------------------------------------------------------------------------------
\begin{proof}[Proof of Theorem \ref{thm: structure of Qprocess}]
It has been shown in section $2.3$, under $\widetilde{\mathbb P}_\mu$,  $X_t$ has a spine representation
(see Lemma \ref{spine structure}) for any $t>0$.  For any $f\in\mathcal B_b^+(E)$,
\[
\widetilde {\mathbb P}_{\mu}\left(e^{-\langle f, X_t\rangle }\right)=\mathbb Q_{\mu}\left(e^{-\langle f, X_t\rangle+\langle f, Z^{{\mathrm m},[0,t)}_t+Z^{{\mathrm n},[0,t)}_t\rangle }\right).
\]
When $\mu(dx)=\nu(dx)=\hat\phi(x)m(dx)$,
\[
\mathbb Q_{\nu}\left(e^{-\langle f, X_t\rangle+\langle f, Z^{{\mathrm m},[0,t)}_t+Z^{{\mathrm n},[0,t)}_t\rangle }\right)=\widehat{\mathbb Q}_{\nu}\left(e^{-\langle f, X_t\rangle-\langle f, \widehat Z^{{\mathrm m}}_t+\widehat Z^{{\mathrm n}}_t\rangle }\right).
\]
Since $\lim_{t\to\infty}\langle\phi, X_{t}\rangle=0$ in probability and $\sum_{\sigma\in(0, t]\bigcap\mathcal D^{\mathrm m}}\langle \phi, \widehat X_{\sigma}^{{\mathrm m},\sigma}\rangle +\sum_{\tau\in (0, t]\bigcap \mathcal D^{\mathrm n}}\langle \phi, \widehat X_{\tau}^{{\mathrm n},\tau}\rangle $ is increasing with respect to $t$ having almost sure limit as $t\to\infty$ as well.  Therefore, by dominated convergence theory,
\[
\lim_{t\to\infty}\widetilde {\mathbb P}_{\nu}\left(e^{-\langle f, X_t\rangle }\right)=\widehat{\mathbb Q}_{\nu}\left(e^{-\langle f,\sum_{\sigma\in\mathcal D^{\mathrm m}}\widehat X^{{\mathrm m},\sigma}_\sigma+\sum_{\tau\in\mathcal D^{\mathrm n}}\widehat X^{{\mathrm n},\tau}_\tau\rangle }\right)
\]
Denote the random measure $\sum_{\sigma\in\mathcal D^{\mathrm m}}\widehat X^{{\mathrm m},\sigma}_\sigma+\sum_{\tau\in\mathcal D^{\mathrm n}}\widehat X^{{\mathrm n},\tau}_\tau$ by $X_\infty$.
Then
\[
\lim_{t\to\infty}\widetilde {\mathbb P}_{\nu}\left(e^{-\langle f, X_t\rangle }\right)=\widehat{\mathbb Q}_\nu \left(e^{-\langle f,X_\infty\rangle}\right).
\]
Compare $X_\infty$ with the random measure in the definition \eqref{cons} of the constant $k$.  Then
$$
k=\widehat{\mathbb Q}_\nu\left(\dfrac{1}{\langle\phi, X_\infty\rangle}\right).
$$
According to the discussion for $k$, we know that $\langle\phi, X_\infty\rangle<\infty$ if and only if $\int_E\hat\phi(x)l(x)m(dx)<\infty$.


Now let us consider the cases for general initial value $\mu\in\mathcal M(E)$.  For $f\in\mathcal B_b^+(E)$, define function
\begin{equation}\label{def: H}
H(x,t):={\mathbb Q}_x\left(e^{-\langle f, Z^{\mathrm n, [0,t)}_{t} + Z^{\mathrm m, [0,t)}_{t}\rangle }\right).
\end{equation}
Then
\begin{eqnarray*}
&&\mathbb Q_\nu\left(\exp\Big\{-\langle f, X_t\rangle-\langle f, Z^{{\mathrm m},[0,t)}_t+Z^{{\mathrm n},[0,t)}_t\rangle \Big\}\right)\\
&=&\mathbb P_\nu\left(e^{-\langle f, X_t\rangle}\right)\int_E\phi(y)\hat\phi(y)H(y,t)m(dy).
\end{eqnarray*}
%%%%%%%%%%%%%%%%%%%%%%%%%%%%%%%%%%%%%%%%%%%%%%%%%%%%%%%%%%%%%%%%%%%%%%%%%%%%%%%%%%%%%%%%%%%%%%%%%%%%%%%%%%%%%%%%%%%%%%%%%%%%%%%%%%%%%%%%%%%%%%%%%%%%%%%%%%%%%%%%%%%  
Define the filtration generated by the spine process that $\mathcal{H}_t=\sigma\big(Y_s; s\leq t\big)$, $t\geq 0$.  Then for $T,t>0$,
\begin{equation}\label{subcritical upper bound}
 \begin{aligned}
 &H(x,t+T)\\
 =&\mathbb Q_{x}\mathbb Q_{x}\Big[\exp\Big\{-\sum_{\sigma\in (0, t+T]\bigcap \mathcal D^{\mathrm m}}\langle f, X_{t+T-\sigma}^{{\mathrm m},\sigma}\rangle -\sum_{\tau\in (0, t+T]\bigcap \mathcal D^{\mathrm n}}\langle f, X_{t+T-\tau}^{{\mathrm n}, \tau}\rangle \Big\}\Big| \mathcal H_t\Big]\\
 \leq&\widetilde\Pi_x\mathbb Q_{x}\Big[\exp\Big\{-\sum_{\sigma\in (t, t+T]\bigcap \mathcal D^{\mathrm m}}\langle f, X_{t+T-\sigma}^{{\mathrm m},\sigma}\rangle -\sum_{\tau\in (t, t+T]\bigcap \mathcal D^{\mathrm n}}\langle f, X_{t+T-\tau}^{{\mathrm n}, \tau}\rangle \Big\}\Big| \mathcal H_t\Big]\\
 =&
  \widetilde\Pi_x\mathbb Q_{Y_t}\Big[\exp\Big\{-\sum_{\sigma\in (0, T]\bigcap \mathcal D^{\mathrm m}}\langle \phi, X_{T-\sigma}^{{\mathrm m},\sigma}\rangle -\sum_{\tau\in (0, T]\bigcap \mathcal D^{\mathrm n}}\langle \phi, X_{T-\tau}^{{\mathrm n}, \tau}\rangle \Big\}\Big]\\
 =&\widetilde\Pi_x\left[ H(Y_t, T)\right].
 \end{aligned}
 \end{equation}
 From \eqref{IU}, there is some constants $c,\nu>0$ such that when $t>1$,
\[
 H(x,t+T)\leq \widetilde\Pi_x\left[ H(Y_t, T)\right]\leq (1+ce^{-\nu t})\int_E\phi(y)\hat\phi(y)H(y,T)m(dy)<\infty.
 \]
Since $H(x,t)\leq 1$.  Set $\overline \eta(x)$ to be the supremum limit of $H(x,t)$ as $t\to \infty$.
Fix time $T$ and let $t\to \infty$ in inequality \eqref{subcritical upper bound}. We can imply that
\begin{equation}\label{sub super}
\overline\eta(x)\leq \int_E\phi(y)\hat \phi(y)H(y,T)m(dy).
\end{equation}
   Using Fatou's lemma for supremum limit
in \eqref{sub super}, for any $x\in E$,
\begin{equation}\label{sup inequality}
\overline\eta(x)\leq \limsup_{T\rightarrow\infty}\int_E\phi(y)\hat \phi(y)H(y,T)m(dy)\leq \int_E\phi(y)\hat\phi(y)\overline{\eta}(y)m(dy).
\end{equation}
 Since $\overline{\eta}(\cdot)\leq 1$, $\overline\eta(\cdot)$ is a constant function by \eqref{sup inequality}.
 Denote $\overline\eta(\cdot)$ by $q(f)$.  If $q(f)\equiv 0,$ then
 \begin{equation}\label{limit}
 \lim_{t\rightarrow\infty}H(x,t)=q(f),\qquad \mbox{for all}\,\, x\in E.
 \end{equation}
  So in the
following, we assume $q(f)>0$.
 For any $\varepsilon_1>0$, let
$$
\mu_1(T)=\int_{\{x\in
E;H(x,T)>(1+\varepsilon_1)q(f)\}}\phi(x)\hat\phi(x)m(dx).
$$
Then $\lim_{T\rightarrow\infty}\mu_1(T)=0.$  For any $\varepsilon_2>0$, let
$$
\mu_2(T)=\int_{\{x\in
E;H(x,T)<(1-\varepsilon_2)q(f)\}}\phi(x)\hat\phi(x)m(dx).
$$
 Then we can deduce from \eqref{sup inequality} that
\begin{eqnarray}\label{sublimitinprob}
q(f)&\leq&
(1-\varepsilon_2)q(f)\mu_2(T)+\mu_1(T)+(1+\varepsilon_1)q(f)(1-\mu_1(T)-\mu_2(T))\\
&\le
&(1+\varepsilon_1)q(f)-(\varepsilon_1+\varepsilon_2)\mu_2(T)+C\mu_1(T),
\end{eqnarray}
where $C$ is some positive finite constant.  Hence
\begin{eqnarray*}\label{sublimitinequl}
q(f)&\leq&
\liminf_{T\rightarrow\infty}(1+\varepsilon_1)q(f)-(\varepsilon_1+\varepsilon_2)\mu_2(T)+C\mu_1(T)\\
&=&(1+\varepsilon_1)q(f)-(\varepsilon_1+\varepsilon_2)q(f)\limsup_{T\rightarrow\infty}\mu_1(T).
\end{eqnarray*}
Since $\varepsilon_1$ is an arbitrary positive constant.
\[
q(f)\leq q(f)-\varepsilon_1 q(f)\limsup_{T\rightarrow\infty}\mu_1(T).
\]
This is impossible unless $\limsup_{T\rightarrow\infty}\mu_1(T)=0.$
Therefore, $H(\cdot,T)$ converges to $q(f)$ in probability under probability $\phi(x)\hat{\phi}(x)m(dx)$ as $T\to\infty$.


Meanwhile, from the definition
\eqref{def: H} of function $H$, we can get the following inequality
\begin{equation}\label{subsub}
\begin{aligned}
     H(x,t+T)\geq& \mathbb Q_{x}\prod_{\sigma\leq t}I_{\{ X_{t+T-\sigma}^{{\mathrm m},\sigma}=0\}}\prod_{\tau\leq t}I_{\{ X_{t+T-\tau}^{{\mathrm n},\tau}=0\}}\\
&\cdot\mathbb Q_{Y_t}\Big[\exp\Big\{-\sum_{\sigma\in (0, T]\bigcap \mathcal D^{\mathrm m}}\langle \phi, X_{T-\sigma}^{{\mathrm m},\sigma}\rangle -\sum_{\tau\in (0, T]\bigcap \mathcal D^{\mathrm n}}\langle \phi, X_{T-\tau}^{{\mathrm n},\tau}\rangle \Big\}\Big]\\
=& \mathbb Q_{x}\left[\prod_{\sigma\leq t}I_{\{ X_{t+T-\sigma}^{{\mathrm m},\sigma}=0\}}\prod_{\tau\leq t}I_{\{ X_{t+T-\tau}^{{\mathrm n},\tau}=0\}}H(Y_t, T)\right].
\end{aligned}
\end{equation}
Consider the following probability,
\begin{eqnarray*}
\mathbb Q_{x}\left(\prod_{\sigma\leq t}I_{\{ X_{t+T-\sigma}^{{\mathrm m},\sigma}=0\}}=1\right)
=\widetilde\Pi_x\exp\left\{-\int_0^tds\int_0^\infty r(1-\mathbb P_{r\delta_{Y_s}}(\zeta<T+t-s))n(Y_s,dr)\right\}.
\end{eqnarray*}
Since in the case of $\lambda<0,$ the $(Y,\psi)$-superprocess starting from any finite measure is extinct in
finite time.  So by dominated convergence theorem,
\begin{equation}\label{1infty limit}
\lim_{T\rightarrow\infty}\int_0^tds\int_1^\infty r(1-\mathbb P_{r\delta_{Y_s}}(\zeta<T+t-s))n(Y_s,dr)=0,
\end{equation}
$\widetilde\Pi_x$ almost surely.   Note that
\[
1-\mathbb P_{r\delta_{Y_s}}(\zeta<T+t-s)\leq 1-(1-\mathbb P_{Y_s}(\zeta>T))^r.
\]
And recall that there are $T_0>0$ and $\eta>0$ such that when $T>T_0$, for any $x\in E$,
\[
\mathbb P_x(\zeta>T)\leq \eta \phi(x)e^{\lambda T}.
\]
Since when $x\rightarrow 0+$, $1-(1-x)^r\sim rx$ for any $r>0$, we assume $T_0$ is sufficiently large such that
$\eta \phi(x)e^{\lambda T}$ is small enough so that $1-(1-\mathbb P_{Y_s}(\zeta>T))^r\leq 2r\eta \phi(Y_s)e^{\lambda T}$.
Therefore,
\[
\int_0^tds\int_0^1 r(1-\mathbb P_{r\delta_{ Y_s}}(\zeta<T+t-s))n(Y_s,dr)\leq 2\eta e^{\lambda T}\int_0^t\phi(Y_s)ds\int_0^1 r^2 n(Y_s,dr).
\]
As a result,
\begin{equation}\label{01limit}
\lim_{T\rightarrow\infty}\int_0^tds\int_0^1 r(1-\mathbb P_{r\delta_{Y_s}}(\zeta<T+t-s))n(Y_s,dr)=0,
\end{equation}
$\widetilde\Pi_x$ almost surely.  Combining \eqref{1infty limit} and \eqref{01limit}, we get
\[
\lim_{T\rightarrow\infty}\mathbb Q_{x}\left(\prod_{\sigma\leq t}I_{\{ X_{t+T-\sigma}^{{\mathrm m},\sigma}=0\}}=1\right)=1.
\]
Meanwhile,
\begin{eqnarray*}
\mathbb Q_x\left(\prod_{\sigma\leq t}I_{\{ X_{t+T-\sigma}^{{\mathrm n},\sigma}=0\}}=1\right)
&=&\widetilde\Pi_x\exp\left\{-\int_0^t2\alpha(Y_s)\mathbb N_{Y_s}(\zeta<T+t-s)ds\right\}\\
&=&\widetilde\Pi_x\exp\left\{-\int_0^t2\alpha(Y_s)v(T+t-s,Y_s)ds\right\}.
\end{eqnarray*}
We also get
\[
\lim_{T\rightarrow\infty}\mathbb Q_x\left(\prod_{\tau\leq t}I_{\{ X_{t+T-\tau}^{{\mathrm n},\tau}=0\}}=1\right)=1,
\]
since $\lim_{T\rightarrow\infty} v(T+t-s,x)=0$ for any $x$, and it is bounded when time $T$ is sufficiently large.  By the inequality \eqref{IU}, for any $\varepsilon>0$ and $t>1$, there are $c>0$ and $\nu>0$, such that for any $x\in E$,
\begin{eqnarray*}
&&\limsup_{T\rightarrow\infty}\widetilde\Pi_x\left(|H(Y_t, T)-q(f)|>\varepsilon\right)\\
&\leq& \limsup_{T\rightarrow\infty}(1+ce^{-\nu t})\int_E\phi(y)\hat\phi(y)m(dy)I_{\{|H(y, T)-q(f)|>\varepsilon\}}=0.
\end{eqnarray*}
Then from the inequality \eqref{subsub}, we have for any $x\in E$,
\begin{eqnarray*}
\liminf_{T\rightarrow\infty}H(x, t+T)&\geq&  \liminf_{T\rightarrow\infty} \mathbb Q_x\left[\prod_{\sigma\leq t}I_{\{ X_{t+T-\sigma}^{{\mathrm m},\sigma}=0\}}\prod_{\tau\leq t}I_{\{ X_{t+T-\tau}^{{\mathrm n},\tau}=0\}}H(Y_t, T)\right]\\
&\geq& q(f)=\limsup_{t\rightarrow\infty}H(x, t).
\end{eqnarray*}
 Therefore  $\lim_{t\rightarrow\infty}H(x, t)=q(f)$ for any $x\in E$.
 %%%%%%%%%%%%%%%%%%%%%%%%%%%%%%%%%%%%%%%%%%%%%%%%%%%%%%%%%%%%%%%%%%%%%%%%%%%%%%%%%%%%%%%%%%%%%%%%%%%%%%%%%%%%%%%%%%%%%%%%%%%%%%%%%%%%%%%%%%%%%%%%%%%%%%%%%%%%%%%%%%%
Due to $0\leq H(x,t)\leq 1$,
\begin{equation*}
q(f)=\lim_{t\rightarrow\infty}\int_E\phi(x)\hat\phi(x)H(x,t)m(dx)
=\lim_{t\rightarrow\infty}\widetilde{\mathbb P}_{\nu}\left(e^{-\langle f, X_t\rangle }\right)
=\widehat{\mathbb Q}_{\nu}\left(e^{-\langle f, X_{\infty}\rangle }\right).
\end{equation*}
Therefore for any $\mu\in\mathcal M(E)\backslash\{0\}$, and $f\in\mathcal B_b^+(E)$,
\begin{eqnarray*}
\lim_{t\rightarrow\infty}\widetilde{\mathbb P}_\mu\left(e^{-\langle f, X_t\rangle}\right)&=&\lim_{t\rightarrow\infty}\mathbb P_\mu\left(e^{-\langle f, X_t\rangle}\right)
\lim_{t\to\infty}\dfrac{1}{\mu(\phi)}\int_E\phi(x)H(x, t)\mu(dx)=q(f)\\
&=&\widehat{\mathbb Q}_{\nu}\left(e^{-\langle f, X_{\infty}\rangle }\right).
\end{eqnarray*}
%%----------------------------------------------------------------------------------------------------------------------------
Note that
\[
\widehat{\mathbb Q}_{\nu}\big(\sum_{\tau\in \mathcal D^{\mathrm n}}\langle f, \widehat X_{\tau}^{{\mathrm n},\tau} \rangle \big)=\int_0^\infty2\langle \alpha P^{\beta}_sf,\phi\nu\rangle ds
\leq 2\|\alpha\phi\|_\infty\dfrac{\langle f,\nu\rangle }{-\lambda}<\infty,
\]
and that
\begin{eqnarray*}
&&\widehat{\mathbb Q}_{\nu}\Big(\sum_{\sigma\in [1,\infty)\bigcap\mathcal D^{\mathrm m}}\langle f, \widehat X_{\sigma}^{{\mathrm m},\sigma} \rangle|\mathcal G \Big)
=\sum_{t\in [1,\infty)\bigcap\mathcal D^{\mathrm m}}m_tP^{\beta}_tf( Y_t)\\
&&\leq \sum_{t\in \mathcal D^{\mathrm m}}(1+ce^{-\rho t})m_te^{\lambda t}
\phi(Y_t)\int_E\hat\phi(y)f(y)m(dy)\\
&&=\int_E\hat\phi(y)f(y)m(dy) \sum_{t\in \mathcal D^{\mathrm m}}(1+ce^{-\rho t})m_te^{\lambda t}\phi(Y_t)<\infty,
\end{eqnarray*}
$\mathbb Q_{\nu}-{\mathrm a.s.}$ when $\int_E\hat\phi(x)l(x)m(dx)<\infty$ by Lemma \ref{import lemma}.
 Thus in this case, the limit measure $X_\infty\in \mathcal M(E)$.  Denote the distribution of $X_\infty$ by $\mathbf P$.  Then $\mathbf P$
 is the equilibrium probability of the $Q$-process.
 %%----------------------------------------------------------------------------------------------------------------------------


When $\int_E\hat\phi(x)l(x)m(dx)<\infty$, we analyze the Yaglom distribution first. Since the Yaglom distribution is independent of the initial value
$\mu$.  Without loss of generality, we suppose $\mu(dx)=\nu(dx)=\hat\phi(x)m(dx)$.  For $f\in\mathcal B_b^+(E)$, using the martingale change of probability $\mathbb P_\nu$, we obtain
\begin{eqnarray*}
&&\mathbb P_\nu\left(\exp\{-\langle f, X_t\rangle \};\zeta>t\right)\\
&=&\mathbb P_\nu\left(\dfrac{M_t}{M_t}\exp\{-\langle f, X_t\rangle \};\zeta>t\right)\\
&=&\widetilde{\mathbb P}_\nu\left(\dfrac{1}{M_t}\exp\{-\langle f, X_t\rangle \}\right)\\
&=&e^{\lambda t}\mathbb Q_{\nu}\left(\dfrac{\exp\Big\{-\langle f, X_t\rangle -\langle f,  Z^{\mathrm m, [0,t)}_t+ Z^{\mathrm n, [0,t)}_t\rangle\Big \}}{\langle\phi, X_t\rangle +\langle\phi,  Z^{\mathrm m, [0,t)}_t+ Z^{\mathrm n, [0,t)}_t\rangle }\right)\\
&=&e^{\lambda t}\widehat{\mathbb Q}_{\nu}\left(\dfrac{\exp\Big\{-\langle f, X_t\rangle -\langle f,  \widehat Z^{\mathrm m}_t+ \widehat Z^{\mathrm n}_t\rangle\Big \}}{\langle\phi, X_t\rangle +\langle\phi,  \widehat Z^{\mathrm m}_t+ \widehat Z^{\mathrm n}_t\rangle }
\right).
\end{eqnarray*}
Since $\lim_{t\rightarrow\infty}X_t=0$ in probability and $\widehat Z^{\mathrm m}_t+ \widehat Z^{\mathrm n}_t$ is increasing having almost sure limit
 $X_\infty\in\mathcal M(E)$, as $t\to\infty$. Therefore
\[
\lim_{t\rightarrow\infty}\widehat{\mathbb Q}_{\nu}\left(\dfrac{\exp\Big\{-\langle f, X_t\rangle -\langle f,  \widehat Z^{\mathrm m}_t+ \widehat Z^{\mathrm n}_t\rangle\Big \}}{\langle\phi, X_t\rangle +\langle\phi,  \widehat Z^{\mathrm m}_t+ \widehat Z^{\mathrm n}_t\rangle }
\right)=\mathbf P\left(\frac{1}{\langle\phi, X_\infty\rangle }\exp\{-\langle f, X_{\infty}\rangle \}\right).
\]
  Meanwhile we note that
\[
\lim_{t\rightarrow\infty}e^{-\lambda t}\mathbb P_\nu(\zeta>t)=k={\mathbf P} \left(\frac{1}{\langle\phi, X_\infty\rangle }\right).
\]
Thus
\begin{eqnarray*}
&&\lim_{t\rightarrow\infty}\mathbb P_\nu\left(\exp\{-\langle f, X_t\rangle \}\Big|\zeta>t\right)=\lim_{t\rightarrow\infty}\dfrac{\mathbb P_\nu\left(\exp\{-\langle f, X_t\rangle \};\zeta>t\right)}{\mathbb P_\mu(\zeta>t)}\\
&&=\dfrac{\lim_{t\rightarrow\infty}\widehat{\mathbb Q}_{\nu}\left(\dfrac{\exp\Big\{-\langle f, X_t\rangle -\langle f,  \widehat Z^{\mathrm m}_t+ \widehat Z^{\mathrm n}_t\rangle\Big \}}{\langle\phi, X_t\rangle +\langle\phi,  \widehat Z^{\mathrm m}_t+ \widehat Z^{\mathrm n}_t\rangle }
\right)}{\lim_{t\rightarrow\infty}e^{-\lambda t}\mathbb P_\mu(\zeta>t)}\\
&&=\dfrac{\mathbf P\left(\dfrac{1}{\langle\phi, X_\infty\rangle }\exp\{-\langle f, X_{\infty}\rangle \}\right)}{{\mathbf P}\left(\dfrac{1}{\langle\phi, X_\infty\rangle }\right)}.
\end{eqnarray*}
Therefore the Yaglom distribution can be written as
\[
\mathbf P^{\lambda}(\cdot)=\dfrac{1}{k}{\mathbf P}\left(\dfrac{1}{\langle\phi, X_\infty\rangle }; X_\infty\in\cdot\right).
\]
Since $\mathbf P^{\lambda}$ is supported on $\mathcal S$, so is $X_\infty$. Define the random variable $M=\langle\phi,X_\infty\rangle$.  Then
$X_\infty(dx)=M\hat\phi(x)m(dx)$.  Therefore,
\begin{equation}\label{ident: k}
\mathbf P^{\lambda}(M^{(\lambda)})=\dfrac{1}{k}{\mathbf P}\left(\dfrac{M}{M }\right)=\dfrac{1}{k}.
\end{equation}
And for any $\theta>0$,
\[
Ee^{-\theta M}={\mathbf P}\left(\dfrac{M}{M }e^{-\theta M}\right)=k\mathbf P^{\lambda}(M^{(\lambda)}e^{-\theta M^{(\lambda)}})=\dfrac{E(M^{(\lambda)}e^{-\theta M^{(\lambda)}})}{EM^{(\lambda)}}.
\]
%%------------------------------------------------------------------------------------------------------------------------------
%%----------------------------------------------------------------------------------------------------------------------------
This says the equilibrium probability of the $Q$ process is a size-biased distribution of the Yaglom probability with weight function
 $\dfrac{M^{(\lambda)}}{EM^{(\lambda)}}$.


 When $\int_E\hat\phi(x)l(x)m(dx)=\infty$, it is shown in theorem \ref{thm: distribution of zeta} that
 $\sum_{s\in\mathcal D^{\mathrm m}} \langle \phi,\widehat X^{{\mathrm m},s}_s\rangle =\infty$, $\widehat{\mathbb Q}_\nu$ almost surely. Thus
\[
\langle \phi, X_{\infty}\rangle =\infty,\qquad \mathbf P-{\mathrm a.s.}
\]
In this case, the $Q$ process does not have equilibrium probability. And $\langle \phi, X_t\rangle $ converges to $\infty$ as $t\to\infty$ in probability with respect to $\widetilde{\mathbb P}_\mu$ for any $\mu\in \mathcal M(E)\backslash\{0\}$.
\end{proof}
%%%%%%%%%%%%%%%%%%%%%%%%%%%%%----------------------------------------------------------------------------------------------------------------
\begin{proof}[Proof of Proposition \ref{exp prop}]
The results in this proposition are obtained from \eqref{ident: k} and the result of theorem \ref{thm: distribution of zeta}.
\end{proof}
%%--------------------------------------------------------------------------------------------------------------------------------------------
\begin{prop}\label{inf div}
Let $L_{\mu, t}$ be the distribution of $X^D_t$ under the probability $\mathbb P_\mu$.  When $\int_D\widetilde\phi(x)m(x)dx<\infty$, given the path $\widehat Y$ of the spine under $\mathbb P_{\widetilde\phi,\phi}$,  $X^{\infty,D}$ has a infinitely divisible distribution on $M_F(D)$ with Levy measure $\mathcal N(\widehat{Y}, dw)=\int_0^\infty ds\int_0^\infty r n(\widehat{Y}_s, dr)L_{r\delta_{\widehat{Y}_s}, s}(dw)$.  In other words, for any $f\in\mathcal B_b^+(D)$,
\[
\mathbb P_{\widetilde\phi,\phi}\left[e^{-\langle f,X^{\infty,D}\rangle }\big|\widehat{Y}\right]=\exp\left\{-\int_{M_F(D)}(1-e^{-\langle f,\omega\rangle })\mathcal N(\widehat{Y}, d\omega)\right\}.
\]
 \end{prop}
 \begin{proof}
 From the definition \eqref{laplace Q} of $X^{\infty,D}$ and \eqref{time rev}, the Laplace functional of $X^{\infty,D}$ is given by for any nonnegative Borel measurable function $f$ on $D$,
 \begin{eqnarray*}
 &&\mathbb P_{\widetilde\phi,\phi}\left[e^{-\langle f,X^{\infty,D}\rangle }\big|\widehat{Y}\right]=\mathbb P_{\widetilde\phi,\phi}\left[e^{-\sum_{s<\infty}\langle f,M_s\rangle }\big|\widehat{Y}\right]\\
& =&\lim_{t\rightarrow\infty}\exp\left\{\int_0^tds\int_0^\infty \psi'(\widehat Y_s, U^s(0,\widehat Y_s))n(\widehat Y, dr)\right\}\\
&=&\exp\left\{\int_0^\infty \psi'(\widehat Y_s, U^s(0,\widehat Y_s))ds\right\},\qquad \widehat\Pi_{\widetilde\phi\phi}^\phi{\mathrm -a.s.}
 \end{eqnarray*}
 Since $\psi'(x,\lambda)=\int_0^\infty r(1-e^{-\lambda r})n(x,dr)$,  so
 \begin{eqnarray*}
  \psi'(\widehat Y_s, U^s(0,\widehat Y_s))&=&\int_0^\infty r
  (1-e^{-rU^s(f)(0,\widehat Y_s)})n(\widehat Y_s,dr)\\
  &=&\int_0^\infty r\int_{M_F(D)}(1-e^{-\langle f,\omega\rangle })L_{r\delta_{\widehat{Y}_s}, s}(d\omega)n(\widehat Y_s,dr).
 \end{eqnarray*}
 Therefore,
\begin{eqnarray*}
&&\mathbb P_{\widetilde\phi,\phi}\left[e^{-\langle f,X^{\infty,D}\rangle }\big|\widehat{Y}\right]\\
&=&\exp\left\{-\int_{M_F(D)}(1-e^{-\langle f,\omega\rangle })\int_0^\infty ds\int_0^\infty rL_{r\delta_{\widehat{Y}_s}, s}(d\omega)n(\widehat Y_s,dr)ds\right\}\\
&=&\exp\left\{-\int_{M_F(D)}(1-e^{-\langle f,\omega\rangle })\mathcal N(\widehat{Y}, d\omega)\right\},\qquad \widehat\Pi_{\widetilde\phi\phi}^\phi{\mathrm -a.s.}
\end{eqnarray*}
 Then our conclusions follows from \cite[Theorem 3.4.1]{Dawson1992Infinitely} for instance.\qed
 \end{proof}
%----- Bibliographic references-------------
%\bibliographystyle{amsplain}
%\bibliography{/Users/zhenyao/Nutstore/bib/bib.bib}

\begin{thebibliography}{10}

\bibitem{AthreyaNey1972Branching}
K.~B. Athreya and P.~E. Ney, \emph{Branching processes}, Springer-Verlag, New
  York-Heidelberg, 1972.

\bibitem{ChampagnatRoelly2008Limit}
N.~Champagnat and S.~R{\oe}lly, \emph{Limit theorems for conditioned multitype
  {D}awson-{W}atanabe processes and {F}eller diffusions}, Electron. J. Probab.
  \textbf{13} (2008), no.~25, 777--810.

\bibitem{ChampagnatVillemonais2018Convergence}
N.~Champagnat and D.~Villemonais, \emph{Convergence of the {F}leming-{V}iot
  process toward the minimal quasi-stationary distribution}, arXiv:1810.06849,
  2018.

\bibitem{Dawson1992Infinitely}
D.~A. Dawson, \emph{Infinitely divisible random measures and superprocesses},
  Stochastic analysis and related topics ({S}ilivri, 1990), Progr. Probab.,
  vol.~31, Birkh\"{a}user Boston, Boston, MA, 1992, pp.~1--129.

\bibitem{Dynkin1993Superprocesses}
E.~B. Dynkin, \emph{Superprocesses and partial differential equations}, Ann.
  Probab. \textbf{21} (1993), no.~3, 1185--1262.

\bibitem{EnglanderKyprianou2004Local}
J.~Engl\"{a}nder and A.~E. Kyprianou, \emph{Local extinction versus local
  exponential growth for spatial branching processes}, Ann. Probab. \textbf{32}
  (2004), no.~1A, 78--99.

\bibitem{Evans1993Two}
S.~N. Evans, \emph{Two representations of a conditioned superprocess}, Proc.
  Roy. Soc. Edinburgh Sect. A \textbf{123} (1993), no.~5, 959--971.

\bibitem{Grey1974Asymptotic}
D.~R. Grey, \emph{Asymptotic behaviour of continuous time, continuous
  state-space branching processes}, J. Appl. Probability \textbf{11} (1974),
  669--677.

\bibitem{HeathcoteSenetaVere-Jones1967A-refinement}
C.~R. Heathcote, E.~Seneta, and D.~Vere-Jones, \emph{A refinement of two
  theorems in the theory of branching processes}, Teor. Verojatnost. i
  Primenen. \textbf{12} (1967), 341--346.

\bibitem{KimSong2008Intrinsic}
P.~Kim and R.~Song, \emph{Intrinsic ultracontractivity of non-symmetric
  diffusion semigroups in bounded domains}, Tohoku Math. J. (2) \textbf{60}
  (2008), no.~4, 527--547.

\bibitem{Lambert2001Arbres}
A.~Lambert, \emph{Arbres, excursions et processus de {L}\'{e}vy completement
  asym\'{e}triques}, Ph.D. thesis, Universit\'{e} Pierre et Marie Curie-Paris
  VI, 2001.

\bibitem{Lambert2003Coalescence}
\bysame, \emph{Coalescence times for the branching process}, Adv. in Appl.
  Probab. \textbf{35} (2003), no.~4, 1071--1089.

\bibitem{Lambert2007Quasi-stationary}
\bysame, \emph{Quasi-stationary distributions and the continuous-state
  branching process conditioned to be never extinct}, Electron. J. Probab.
  \textbf{12} (2007), no.~14, 420--446.

\bibitem{Li2000Asymptotic}
Z.~Li, \emph{Asymptotic behaviour of continuous time and state branching
  processes}, J. Austral. Math. Soc. Ser. A \textbf{68} (2000), no.~1, 68--84.

\bibitem{Li2011Measure-valued}
\bysame, \emph{Measure-valued branching {M}arkov processes}, Probability and
  its Applications (New York), Springer, Heidelberg, 2011.

\bibitem{LiuRenSong2009Llog}
R.~Liu, Y.-X. Ren, and R.~Song, \emph{{$L\log L$} criterion for a class of
  superdiffusions}, J. Appl. Probab. \textbf{46} (2009), no.~2, 479--496.

\bibitem{LyonsPemantlePeres1995Conceptual}
R.~Lyons, R.~Pemantle, and Y.~Peres, \emph{Conceptual proofs of {$L\log L$}
  criteria for mean behavior of branching processes}, Ann. Probab. \textbf{23}
  (1995), no.~3, 1125--1138.

\bibitem{MeleardVillemonais2012Quasi-stationary}
S.~M\'{e}l\'{e}ard and D.~Villemonais, \emph{Quasi-stationary distributions and
  population processes}, Probab. Surv. \textbf{9} (2012), 340--410.

\bibitem{Penisson2010Conditional}
S.~P\'{e}nisson, \emph{Conditional limit theorems for multitype branching
  processes and illustration in epidemiological risk analysis}, Ph.D. thesis,
  Universit\"{a}t Potsdam; Universit\'{e} Paris Sud-Paris XI, 2010.

\bibitem{RenSongSun2017Spine}
Y.-X. Ren, R.~Song, and Z.~Sun, \emph{Spine decompositions and limit theorems
  for a class of critical superprocesses}, arXiv:1711.09188, 2017.

\bibitem{RenSongZhang2015Limit}
Y.-X. Ren, R.~Song, and R.~Zhang, \emph{Limit theorems for some critical
  superprocesses}, Illinois J. Math. \textbf{59} (2015), no.~1, 235--276.

\bibitem{RenSongZhang2017Central}
\bysame, \emph{Central limit theorems for supercritical branching nonsymmetric
  {M}arkov processes}, Ann. Probab. \textbf{45} (2017), no.~1, 564--623.

\bibitem{RenSongZhang2018Williams}
\bysame, \emph{Williams decomposition for superprocesses}, Electron. J. Probab.
  \textbf{23} (2018), no.~23, 33 pp.

\bibitem{RoellyRouault1989Processus}
S.~Roelly and A.~Rouault, \emph{Processus de {D}awson-{W}atanabe
  conditionn\'{e} par le futur lointain}, C. R. Acad. Sci. Paris S\'{e}r. I
  Math. \textbf{309} (1989), no.~14, 867--872.

\bibitem{Schaefer1974Banach}
H.~H. Schaefer, \emph{Banach lattices and positive operators}, Springer-Verlag,
  New York-Heidelberg, 1974.

\end{thebibliography}

\begin{comment}
\begin{thebibliography} {10}

\bibitem{AthreyaNey1972Branching}
Athreya, K. B. and Ney, P. E.:
\emph{Branching processes.}
Die Grundlehren der mathematischen Wissenschaften, Band 196. Springer-Verlag, New York-Heidelberg, 1972. xi+287 pp.
\MR{0373040}

\bibitem{BigginsKyprianou2004Measure}
Biggins, J. D. and Kyprianou, A. E.:
\emph{Measure change in multitype branching.}
Adv. in Appl. Probab. \textbf{36} (2004), no. 2, 544--581.
\MR{2058149}

\bibitem{ChampagnatRoelly2008Limit}
Champagnat, N. and Roelly, S.:
\emph{Limit theorems for conditioned multitype Dawson-Watanabe processes and Feller diffusions.}
Electron. J. Probab. \textbf{13} (2008), no. 25, 777–810.
\MR{2399296}

\bibitem{ChampagnatVillemonais2018Convergence}
{\color{blue}Champagnat, N. and Villemonais, D.:
\emph{Convergence of the Fleming-Viot process toward
theminimal quasi-stationary distribution.}
https://arxiv.org/pdf/1810.06849.pdf}

\bibitem{ChenRenYang2017Skeleton}
Chen, Z.-Q., Ren, Y.-X. and Yang, T.:
\emph{Skeleton decomposition and law of large numbers for supercritical superprocesses.}
Acta Appl. Math. (2017), 1--61.
\ARXIV{1709.00847}

\bibitem{Dawson1992Infinitely}
Dawson, D. A.:
\emph{Infinitely divisible random measures and superprocesses.}Stochastic analysis and related topics (Silivri, 1990), 1--129,
Progr. Probab., 31, Birkh{\"a}user Boston, Boston, MA, 1992.
\MR{1203373}

\bibitem{DelmasHenard2013A-Williams}
Delmas, J.-F. and H\'enard, O.:
\emph{A Williams decomposition for spatially dependent super-processes. }
Electron. J. Probab. \textbf{18} (2013), no. 37, 43 pp.
\MR{3035765}

\bibitem{Dynkin1993Superprocesses}
Dynkin, E. B.:
\emph{Superprocesses and partial differential equations.}
Ann. Probab. \textbf{21} (1993), no. 3, 1185--1262.
\MR{1235414}

\bibitem{EnglanderKyprianou2004Local}
Engl\"ander, J. and Kyprianou, A. E.:
\emph{Local extinction versus local exponential growth for spatial branching processes.}
Ann. Probab. \textbf{32} (2004), no. 1A, 78--99.
\MR{2040776}

\bibitem{Evans1993Two}
Evans, S. N.:
\emph{Two representations of a conditioned superprocess.}
Proc. Roy. Soc. Edinburgh Sect. A \textbf{123} (1993), no. 5, 959--971.
\MR{1249698}

\bibitem{Grey1974Asymptotic}
Grey, D. R.:
\emph{Asymptotic behaviour of continuous time, continuous state-space branching processes.}
J. Appl. Probability \textbf{11} (1974), 669--677.
\MR{0408016}

\bibitem{HeathcoteSenetaVere-Jones1967A-refinement}
Heathcote, C. R., Seneta, E. and Vere-Jones, D.:
\emph{A refinement of two theorems in the theory of branching processes.} (Russian summary)
Teor. Verojatnost. i Primenen. \textbf{12} 1967 341--346.
\MR{0217889}

\bibitem{KimSong2008Intrinsic}
Kim, P. and Song, R.:
\emph{Intrinsic ultracontractivity of non-symmetric diffusion semigroups in bounded domains.}
Tohoku Math. J. (2) \textbf{60} (2008), no. 4, 527--547.
\MR{2487824}

\bibitem{KimSong2008Intrinsic2}
Kim, P. and Song, R.:
\emph{Intrinsic ultracontractivity of nonsymmetric diffusions with measure-valued drifts and potentials.}
Ann. Probab. \textbf{36} (2008), no. 5, 1904–1945.
\MR{2440927}

\bibitem{KimSong2009Intrinsic}
Kim, P. and Song, R.:
\emph{Intrinsic ultracontractivity for non-symmetric Lévy processes.}
Forum Math. \textbf{21} (2009), no. 1, 43–66.
\MR{2494884}

\bibitem{Lambert2001Arbres}
Lambert, A.:
\emph{Arbres, excursions et processus de L\'evy completement asym\'etriques.}
Diss. Université Pierre et Marie Curie-Paris VI, 2001.

\bibitem{Lambert2003Coalescence}
Lambert, A.:
\emph{Coalescence times for the branching process.}
Adv. in Appl. Probab. \textbf{35} (2003), no. 4, 1071--1089.
\MR{2014270}

\bibitem{Lambert2007Quasi-stationary}
Lambert, A.:
\emph{Quasi-stationary distributions and the continuous-state branching process conditioned to be never extinct.}
Electron. J. Probab. \textbf{12} (2007), no. 14, 420--446.
\MR{2299923}

\bibitem{Li2000Asymptotic}
Li, Z.-H.:
\emph{Asymptotic behaviour of continuous time and state branching processes.}
J. Austral. Math. Soc. Ser. A \textbf{68} (2000), no. 1, 68--84.
\MR{1727226}

\bibitem{Li2011Measure-valued}
Li, Z.:
\emph{Measure-valued branching Markov processes.}
Probability and its Applications (New York). Springer, Heidelberg, 2011. xii+350 pp. ISBN: 978-3-642-15003-6
\MR{2760602}

\bibitem{LiuRenSong2009Llog}
Liu, R.-L., Ren, Y.-X. and Song, R.:
\emph{{$L \log L$} criterion for a class of superdiffusions.}
J. Appl. Probab. \textbf{46} (2009), no. 2, 479–496.
\MR{2535827}

\bibitem{LyonsPemantlePeres1995Conceptual}
Lyons, R., Pemantle, R. and Peres, Y.:
\emph{Conceptual proofs of $L\log L$ criteria for mean behavior of branching processes.}
Ann. Probab. \textbf{23} (1995), no. 3, 1125--1138.
\MR{1349164}

\bibitem{MeleardVillemonais2012Quasi-stationary}
M\'el\'eard, S. and Villemonais, D.:
\emph{Quasi-stationary distributions and population processes.}
Probab. Surv. \textbf{9} (2012), 340–410.
\MR{2994898}

\bibitem{Nagasawa1964Time}
Nagasawa, M.:
\emph{Time reversions of Markov processes.}
Nagoya Math. J. \textbf{24} (1964), 177--204.
\MR{0169290}

\bibitem{RenSongSun2017Spine}
Ren, Y.-X., Song, R. and Sun, Z.:
\emph{Spine decompositions and limit theorems for a class of critical superprocesses.}
Preprint.
\ARXIV{1711.09188}

\bibitem{RenSongYang2016Spine}
Ren, Y.-X., Song, R. and Yang, T.:
\emph{Spine decomposition and {$ L\log L $} criterion for superprocesses with non-local branching mechanisms.}
Preprint.
\ARXIV{1609.02257}

\bibitem{RenSongZhang2015Limit}
Ren, Y.-X., Song, R. and Zhang, R.:
\emph{Limit theorems for some critical superprocesses.}
Illinois J. Math. \textbf{59} (2015), no. 1, 235–276.
\MR{3459635}

\bibitem{RenSongZhang2017Central}
Ren, Y.-X., Song, R. and Zhang, R.:
\emph{Central limit theorems for supercritical branching nonsymmetric Markov processes.}
Ann. Probab. \textbf{45} (2017), no. 1, 564–623.
\MR{3601657}

\bibitem{RenSongZhang2018Williams}
Ren, Y.-X., Song, R. and Zhang, R.:
\emph{Williams decomposition for superprocesses.}
Electron. J. Probab. \textbf{23} (2018), Paper No. 23, 33 pp.
\MR{3771760}

\bibitem{RoellyRouault1989Processus}
Roelly, S. and Rouault, A.:
\emph{Processus de Dawson-Watanabe conditionn\'e par le futur lointain.} (French. English summary) [A Dawson-Watanabe process conditioned by the remote future]
C. R. Acad. Sci. Paris Sér. I Math. \textbf{309} (1989), no. 14, 867--872.
\MR{1055211}

\bibitem{Penisson2010Conditional}
P\'enisson, S.:
\emph{Conditional limit theorems for multitype branching processes and illustration in epidemiological risk analysis.}Diss. Universit?t Potsdam, Université Paris Sud-Paris XI, 2010.

\bibitem{Schaefer1974Banach}
Schaefer, H. H.:
\emph{Banach lattices and positive operators.}
Die Grundlehren der mathematischen Wissenschaften, Band 215. Springer-Verlag, New York-Heidelberg, 1974.
\MR{0423039}

\bibitem{Yaglom1947Certain}
Yaglom, A. M.:
\emph{Certain limit theorems of the theory of branching random processes.} (Russian)
Doklady Akad. Nauk SSSR (N.S.) \textbf{56} (1947), 795--798.
\MR{0022045}

\end{thebibliography}
\end{comment}
\end{document}
