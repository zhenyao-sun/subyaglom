% * The preamble
% ! This TeX file uses 'amsart' for its document class. 
\documentclass[12pt,a4paper]{amsart}
\setlength{\textwidth}{\paperwidth}
\addtolength{\textwidth}{-2in}
\calclayout
% ! 'amsmath', 'amsthm', 'amsfonts' package is automatically loaded by 'amsart' class.
% ! Below is the configuration of 'amsmath' package.
\numberwithin{equation}{section}
\allowdisplaybreaks
% ! Below is the configuration of 'amsthm' package.
\theoremstyle{plain}
\newtheorem{thm}{Theorem}[section]
\newtheorem{lem}[thm]{Lemma}
\newtheorem{prop}[thm]{Proposition}
\newtheorem{cor}[thm]{Corollaray}
\newtheorem{conj}[thm]{Conjecture}
\theoremstyle{definition}
\newtheorem{defi}[thm]{Definition}
\newtheorem{exa}[thm]{Example}
\newtheorem{asp}{Assumption}
\newtheorem{iss}{Issue}
\newtheorem{rem}[thm]{Remark}
\newtheorem*{ack}{Acknowledgment}
% ! 'inputenc' package is used to specify the set of characters allowed to input into this TeX file.
% \usepackage[utf8]{inputenc}
% ! 'foutenc' package is used to specify the set of characters allowed to output into the generated pdf file.
% \usepackage[T1]{fontenc}
% ! 'amssymb', 'mathrsfs' and 'mathtools' packages provide additional math symbols from the AMS default symbol fonts. They are compatible with the AMS class and is recommended to be used.
\usepackage{amssymb}
\usepackage{mathtools}
\mathtoolsset{showonlyrefs}
\usepackage{mathrsfs}
% ! 'hyperref' package is used to handle cross-referencing in the generated pdf file.
\usepackage[backref]{hyperref}
% ! Other packages
\usepackage{xcolor}
\usepackage[inline]{showlabels}
\usepackage{comment}
\usepackage{enumitem}
% * Top matter
\begin{document}
\title {Subcritical Superprocesses }
\author
[R. Liu, Y.-X. Ren, R. Song and Z. Sun]
{
  Rongli Liu, Yan-Xia Ren, Renming Song and Zhenyao Sun}
\address{
  Yan-Xia Ren\\
  School of Mathematical Sciences\\
  Peking University\\
  Beijing, P. R. China, 100871}
\email{
  yxren@math.pku.edu.cn}
\thanks{
  The research of Yan-Xia Ren is supported in part by NSFC (Grant Nos. 11671017 and 11731009).}
\address{
  Rongli Liu\\
  Information about Rongli Liu}
\email{rlliu@bjtu.edu.cn }
\thanks{
  The research of Rongli Liu is supported in part by NSFC
  (Grant No. 11301261), and the Fundamental Research Funds for the Central Universities (Grant No.  2017RC007)}
\address{
  Zhenyao Sun\\
  School of Mathematical Sciences\\
  Peking University\\
  Beijing, P. R. China, 100871}
\email{
  zhenyao.sun@pku.edu.cn}
\begin{abstract}
\begin{comment}
	In this paper, we discussed the decay rate of the extinction probability of a class of superdiffusions.
  Two different conditional limits, the Yaglom probability and the Q process, are considered.
  We characterize the Yaglom distribution by some function $G(f)$, which satisfies $G(v^t(0,\cdot))=t$ for each $t>0$ and where $v^t(0,x)=-\log(\mathbb P_x(\zeta<t))$.
  It is proved that the Yaglom distribution is the minimal quasi-stationary distribution.
  As for the Q process, we show that under finite LlogL moment assumptions, the Q process has the equilibrium distribution.
  And it is a size-biased Yaglom distribution.
  Moreover, we point out that under some conditional distribution the equilibrium  distribution is an infinitely divisible distribution.
\end{comment}
{\bf TBD}
\end{abstract}
\maketitle
% * Document body
% ** Introduction
\section{Introduction}
% *** (Backgroud)
\begin{comment}
  \subsection{Backgroud}
  % Suppose $(Z_n, n\ge 1)$ is a Galton-Watson branching process with offspring distribution $\{p_n\}$. 
  Suppose that $(Z_n)_{n \in \mathbb Z_+}$ is a Galton-Watson process with offspring distribution $(p_n)_{n\in \mathbb Z_+}$. 
  % Let $m:=\sum^{\infty}_{n=1}np_n$ be the mean number of children per particle.
  Let $m:=\sum^{\infty}_{n=0}np_n$ be the mean number of children per particle.
  % It is well known that when $m<1$ the extinction probability $q:=\lim_{n\rightarrow\infty}P\left(Z_n=0\right)$ is equal to $1$.
  % That is to say the process $(Z_n)$ is extinct in finite time almost surely.
  % A natural question is what the right decay rate of the probability is.
  A natural question is what the right decay rate of those non-extinction probability $P(Z_n >0)$ is.
  % In 1967, Heathcot, Seneta and Vere-Jone \cite{HeathcoteSenetaVere-Jones1967Refinement} answered the question, giving an LlogL criteria.
  In 1967, Heathcot, Seneta and Vere-Jone \cite{HeathcoteSenetaVere-Jones1967Refinement} answered this question with an LlogL criteria.
  % Let $L$ stand for a random variable with distribution $\{p_n\}$.
  Let $L$ stands for a random variable with distribution $(p_n)_{n \in \mathbb Z_+}$.
  \begin{thm}[\cite{HeathcoteSenetaVere-Jones1967Refinement}]
    The sequence $\{ P(Z_n>0)/m^n\}$ is decreasing. 
    If $m<1$, then the following are equivalent:
    \begin{enumerate}
    \item $\lim_{n\rightarrow\infty}P(Z_n>0)/m^n>0$,
    \item $\sup E[Z_n|Z_n>0]<\infty$,
    \item $E\left[L\log^+ L\right]<\infty$.
    \end{enumerate}
  \end{thm}
  % In 1995, Lyons, Pemantle and Peres developed a martingale change of measure method in \cite{LyonsPemantlePeres1995Conceptual} to give a new proof for this $L\log L$ theorem.
  In 1995, Lyons, Pemantle and Peres developed a martingale change of measure method in \cite{LyonsPemantlePeres1995Conceptual} to give a new proof for this $L\log L$ theorem.

  In \cite{Lambert2001Arbres,Lambert2003Coalescence}, Lambert discusses the similar equivalencies for continuous time continuous state branching process $(Z_t)$.
  Its branching mechanism function $\psi$ is specified by the L\'evy-Khinchin formula
\[
	\psi(\lambda)=-\beta\lambda+\sigma^2\lambda^2+\int_0^\infty \left(e^{-\lambda r}-1+\lambda r\right)n(dr),
\]
where $n(dr)$ is a positive measure on $(0,\infty)$ such that $\int_0^\infty(r^2\wedge r) n(dr)<\infty$ and where
\begin{equation}\label{eq:_extinc_assump_for_continuous}
	\int^\infty\frac{1}{\psi(\lambda)}d\lambda   <   \infty.
\end{equation}
We denote by $\mathbb P_x$ the law of $(Z_t)$ with initial value $Z_0=x>0$.
Then for any $\lambda, t\geq 0$,
\[
	\mathbb P_x\left(\exp\{-\lambda Z_t\}\right)=\exp(-xu_t(\lambda)),
\]
where $u_t(\lambda)$ is the unique solution to the following equation
\begin{equation}
  \begin{cases}
    \dfrac{d u_t(\lambda)}{dt}=-\psi(u_t(\lambda));\\
    u_0(\lambda)=\lambda.
  \end{cases}
\end{equation}
Let $\zeta:=\inf\{t\geq 0; Z_t=0\}$ be the extinction time of $Z$. There is a
nonnegative function $\varphi(t)$ such that for any $x>0$,
\[
	\mathbb P_x(Z_t>0)=\mathbb P_x(\zeta>t)=1-\exp(-x\varphi(t)), \qquad t>0.
\]
When $\beta<0$ (c.f.\cite{Grey1974Asymptotic}),
\[
	\mathbb P_x(\zeta<\infty)=1.
\]
Thus the decay rate of $\mathbb P_x(\zeta>t)$ is determined by the decay rate of
$\varphi(t)$. It is shown (c.f.\cite{Lambert2007Quasistationary}) that for any
$\lambda>0$,
\[
	\lim_{t\rightarrow\infty}\frac{u_t(\lambda)}{\varphi(t)}=G(\lambda),
\]
where $G(\lambda)=\exp(\beta\int_{\lambda}^\infty1/\psi(u)du)$. Moreover, the
following LlogL criteria holds.
\begin{thm}
  \label{thm:equivalent_for_cbp}
	When $\beta<0$, the following items are equivalent
  \begin{itemize}
  \item[$(i).$] $G'(0+)<\infty$;
  \item[$(ii).$] There is a positive constant $c$ such that $\varphi(t)\sim
    c\exp(\beta t)$;
  \item[$(iii).$] $\int_1^\infty r\log r n(dr)<\infty$.
  \end{itemize}
  In this case, $c^{-1}=G'(0+)$.
\end{thm}
As far as the two type dynamics running over large amounts of time are
concerned, special attentions are given to two conditional limits. One is called
Yaglom distribution and another one is called $Q$ process. Here we take the
continuous state branching process $Z$ for example to give the definitions of
the two limits. We can find the corresponding investigations for GW branching
processes in \cite{AthreyaNey1972Branching}.

The Yaglom distribution $\nu_{\beta}$ is the limit probability that for any
$x>0$ and Borel set $A$,
\[
	\lim_{t\rightarrow\infty}\mathbb P_x(Z_t\in A\big|\zeta>t)=\nu_{\beta}(A).
\]
The conditional convergence can be found in \cite{Li2000Asymptotic} where it is
also generalized to conditioning of the type $\{\zeta>t+r\}$ for any finite
$r>0$ instead of $\{\zeta>t\}$. A quasi-stationary distribution (QSD) is a
subinvariant distribution $\nu$ on $(0,\infty)$ satisfying
\[
	\mathbb P_{\nu}(Z_t\in A|\zeta>t)=\nu(A).
\]
It is shown in \cite{Lambert2007Quasistationary} that the Yaglom distribution
$\nu_{\beta}$ is the minimal QSD of $Z$ in the following sense. For any
$\gamma\in[\beta,0)$, there is a unique QSD $\nu_{\gamma}$ associated to the
rate of mass decay $-\gamma$, and there is no QSD associated to $\gamma<\beta$.


Let $\mathcal F_t=\sigma(Z_s,s\leq t), t\geq 0$ be the natural
$\sigma-$filtration generated by $Z$. Then due to
\cite{RenSongZhang2018Williams}, another conditional limit $\widetilde{\mathbb
  P}_x$ is defined in the sense that for any $x,t>0$ and $A\in\mathcal F_t$,
\begin{align}
	\lim_{s\rightarrow\infty}\mathbb P_x(A\big|\zeta>s)=\widetilde{\mathbb P}_x(A).
\end{align}
It can be expressed as an $h-$transform of $\mathbb P_x$ with martingale
$M_t=Z_te^{-\beta t}$ that
\[
	\left.\dfrac{d\widetilde{\mathbb P}_x}{d\mathbb P_x}\right|_{\mathcal F_t}=\frac{M_t}{x}.
\]
Moreover, under $\widetilde{\mathbb P}_x$, the process $Z$ is a branching
process with immigration called $Q$-process. Lambert established the connection
between these two limit distributions in \cite{Lambert2007Quasistationary}. Let
$\Upsilon$ be a random variable whose distribution is the Yaglom distribution
$\nu_\beta$. If $\int_1^\infty r\log r n(dr)<\infty$, then under
$\widetilde{\mathbb P}_x$, $Z_t$ converges in distribution to a positive random
variable $Z_\infty$ as $t\to\infty$, which has the distribution of the
size-biased Yaglom distribution
\[
	\widetilde{\mathbb P}_x(Z_\infty\in dr)=\frac{r}{\mathbb E\Upsilon}\mathbb P(\Upsilon\in dr).
\]
The studies on the Q processes,the Yaglom distributions and the quasi-stationary
distributions for more models we refer to the survey
\cite{MeleardVillemonais2012Quasistationary}, the thesis
\cite{Penisson2010Conditional} and the references therein.

Superdiffusions are a class of branching processes with spatial motions. Similar
to branching processes we introduced above, $0$ is also their absorbing state.
In this paper, we will discuss a class of superdiffusions which will be extinct
in finite time. We establish an $L\log L$ criterion taking use of the spine
method and investigate the Yaglom distributions and $Q$ processes. Moreover, we
study the QSD for our model as well. To state our main results, we need to
introduce the setup we are going to work with first.
\end{comment}

% *** (Main results)
\begin{comment}
\subsection{Main results}
\label{sec:main-results}
Let $E$ be a locally compact separable metric space. 
Denote by $\mathcal B_b(E,\mathbb R)$ the collection of all bounded measurable functions on
$E$. 
Denote by $\mathcal B_b(E, [0,\infty))$ the collection of all non-negative bounded measurable functions on $E$. 
Denote by $\mathcal M(E)$ the space of all finite measures on $E$ equipped with the weak topology. 
Write $\langle f,\mu\rangle$ or $\mu(f)$ for the integral $\int_E f(x)\mu(dx)$ whenever it makes sense.

Let the \emph{spatial motion} $Y=\{(Y_t)_{t\geq 0};(\Pi_x)_{x\in E}\}$ be an $E$-valued Hunt process with its lifetime denoted by $\tau$ and its transition semigroup denoted by $(P_t)_{t\geq
  0}$. 
Let the \emph{branching mechanism} $\psi$ be a function on $E\times[0,\infty)$ given by
\[
	\psi(x,z)
	=-\beta(x)z + \alpha(x)^2 z^2+ \int_{(0,\infty)} (e^{-rz}-1+zr )n(x, dr),\qquad x\in E, z\geq0,
\]
where $\beta, \alpha\in \mathcal B_b(E,\mathbb R)$ and $n$ is a $\sigma$-finite kernel from $E$ to $(0,\infty)$ with
\[
	\sup_{x\in E}\int_0^\infty (r\wedge r^2)n(x,dr)
	<\infty.
\]

({\bf Sun: I think it is better to use $\pi$ for the Levy measure.})

In this paper, we consider a \emph{$(Y,\psi)$-superprocess} $X$ which is defined as an $\mathcal M(E)$-valued Hunt process $X=\{(X_t)_{t\geq 0}; (\mathbb
P_\mu)_{\mu \in \mathcal M(E)}\}$ satisfying that for any $\mu \in \mathcal M(E), f\in \mathcal B_b(E, [0,\infty))$,
\begin{equation}
  \label{eq:_def_of_vtf}
  \mathbb P_\mu [e^{-\langle f,X_t\rangle}] = e^{-\langle V_tf, \mu\rangle},
  \quad t\geq 0,
\end{equation}
where, for each $f\in\mathcal B_b(E, [0,\infty))$, function $(t,x) \mapsto V_tf(x)$ on $[0,\infty) \times E$ is the unique locally bounded positive solution to the
equation
\begin{equation}\label{eq:FKPP_in_definition}
  V_t f(x) +   \Pi_x\left[\int_0^{t\wedge \tau} \psi \big(\xi_s,V_{t-s} f(\xi_s)\big) ds\right]
	= P_t f(x),
	\quad x \in E,\,\, t \geq 0.
\end{equation}
Here, we say a function $(t,x)\mapsto u(t,x)$ is \emph{locally bounded} on $[0,\infty) \times E$ if for each $T\geq 0$ we have $\sup_{t\in [0,T],x\in E} |u(t,x)| < \infty$. 
See \cite[Theorem 5.11]{Li2011Measurevalued} for the existence of such processes.


To simplify the notation, we also write $\mathbb P_x$ for $\mathbb P_{\delta_x}$.
Define the \emph{Feynman-Kac semigroup} $(P^\beta_t)_{t\geq 0}$ such that for any $ f\in \mathcal B_b(E,\mathbb R)$,
\begin{align}
	P^\beta_tf(x)
	:= \Pi_x \big[e^{\int_0^{t} \beta(Y_r)dr} f(Y_t)\mathbf 1_{\{t<\tau\}}\big],
	\quad t\geq 0, x\in E.
\end{align}
It is known, see \cite[Proposition 2.27]{Li2011Measurevalued} for example, $(P^\beta_t)_{t\geq 0}$ is \emph{the mean semigroup} of the superprocess $X$, in the sense that for any $\mu \in \mathcal M(E)$, and $f \in \mathcal B_b(E,\mathbb R)$,
\begin{align} \label{eq:Yaglom_type_result_without_2rd} 
\mathbb P_\mu [\langle f,X_t\rangle] = \mu(P^\beta_t f), \quad t \geq 0.
\end{align}


For the spatial motion $Y$, we always assume the following:
\begin{asp}\label{asp:1}
  There exist a $\sigma$-finite measure $m$ with full support on $E$ and a family
  of strictly positive, bounded continuous functions $\{ p_t(\cdot,\cdot): t > 0
  \}$ on $E \times E$ such that
  \begin{align}
  	P_tf(x)
  	= \int_E p_t(x,y) f(y) m(dy),
  	&\quad t>0, x \in E,f \in \mathcal B_b(E,\mathbb R);
  	\\ \int_E p_t(x,y)m(dx)\leq 1, &\quad t>0,y\in E;
  	\\ \int_E \int_E p_t(x,y)^2 m(dx) m(dy)
  	<\infty,
  	&\quad t> 0;
  \end{align}
  and that the functions $x \mapsto \int_E p_t(x,y)^2 m(dy)$ and $y \mapsto \int_E
  p_t(x,y)^2 m(dx)$ are both continuous.
\end{asp}

Under Assumption \ref{asp:1}, it is proved in \cite{RenSongZhang2015Limit,RenSongZhang2017Central} that there exists a family
of strictly positive, bounded continuous functions $\{ p^\beta_t(\cdot,\cdot): t
> 0 \}$ on $E \times E$ such that
\begin{align}
	P^\beta_t f(x)
	= \int_E p_t^\beta (x,y) f(y) m(dy),
	\quad \quad t>0, x \in E,f \in \mathcal B_b(E,\mathbb R).
\end{align}
Its corresponding dual semigroup $(\widehat P^{\beta}_t)_{t \geq 0}$ is given by
\begin{align}
	\widehat P^{\beta}_0 = I;
	\quad \widehat P^{\beta}_t f(y)
	:= \int_E p^\beta_t (x,y) f(x) m(dx),
	\quad t>0, y\in E, f\in \mathcal B_b(E,\mathbb R).
\end{align}
It is shown in \cite{RenSongZhang2015Limit, RenSongZhang2017Central} that both $(P^\beta_t)_{t \geq 0}$ and $(\widehat P_t^{\beta})_{t\geq 0}$ are strongly continuous semigroups of compact operators on $L^2(E,m)$. 
Let $A$ and $\widehat
A$ be the generators of the semigroups $(P^\beta_t)_{t \geq 0}$ and $(\widehat
P^\beta_t)_{t \geq 0}$, respectively. 
Denote by $\sigma(A)$ and $\sigma(\widehat
A)$ the spectra of $A$ and $\widehat A$, respectively. 
According to \cite[Theorem V.6.6]{Schaefer1974Banach}, $\lambda := \sup \text{Re}(\sigma(A))
= \sup \text{Re}(\sigma(\widehat A))$ is a common eigenvalue of multiplicity $1$ for both $A$ and $\widehat A$. 
It is also proved in \cite{RenSongZhang2015Limit,RenSongZhang2017Central} that the eigenfunctions $\phi$ of $A$ and $\widehat\phi$ of $\widehat A$ associated with the common eigenvalue $\lambda$ can be chosen to be strictly positive and continuous everywhere on $E$. 
Normalize $\phi$ and $\widehat\phi$ by
\[	
	\int_E \phi(x)^2 m(dx) = \int_E \phi(x) \widehat \phi(x) m(dx) = 1
\]
so that they are unique.

Notice that, for each $t \geq 0$ and $\mu \in \mathcal M(E)$, we have $ \mathbb
P_\mu[X_t(\phi)] = \mu(P^\beta_t \phi) = e^{\lambda t} \mu(\phi). $ 
If $\lambda
> 0$, the mean of $X_t(\phi)$ will increase exponentially; 
if $\lambda < 0$, the mean of $X_t(\phi)$ will decrease exponentially; 
and if $\lambda = 0$, the mean of $X_t(\phi)$ will be a constant. 
Therefore, we say $X$ is \emph{supercritical, critical} or \emph{subcritical}, according to $\lambda > 0$, $\lambda = 0$ or
$\lambda < 0$, respectively.

Throughout this paper, we assume the following for the mean semigroup $(P_t^\beta)_{t\geq 0}$ in addition:
\begin{asp}~
\label{asp:IU}
\begin{enumerate}
\item The superprocess $X$ is subcritical, i.e. $\lambda < 0$.
\item The eigenfunctions $\phi$ and $\widehat\phi$ are bounded on $E$.
\item \label{subasp:IU} 
  The mean semigroup $(P_t^\beta)_{t\geq 0}$ is \emph{intrinsically ultracontractive}, that is, for each $t>0$, there is a constant $c_t >0$ such that for each $x,y\in E$, $p^\beta_t(x,y) \leq c_t \phi(x) \widehat\phi(y)$.
\end{enumerate}
\end{asp}


It follows from \cite[Proposition 2.5]{KimSong2008Intrinsic} that, under Assumption \ref{asp:IU}.\eqref{subasp:IU}, for each $t>0$, there exists $c'_t > 0$ such that 
\begin{align}
	p^\beta_t (x,y) 
  \geq c_t' \phi(x) \widehat \phi(y),
  \quad x,y \in E.
\end{align}
Define $\nu(dy):= \widehat \phi(y) m(dy)$. 
Then from above, $\nu$ is a finite measure on $E$.
Therefore we can consider the superprocess $X$ with initial configuration $\nu$.
  
For each measure $\mu$ on $E$, we write $\|\mu\|:= \left\langle \mathbf 1_E,\mu \right\rangle$.
% Let $\zeta=\inf\{t>0: \langle \mathbf 1_E,X_t\rangle=0\}$ be the extinction time of the superprocess $X$.
% Similar to \cite{RenSongSun2019Spine,RenSongZhang2018Williams}, we add the following assumption to assure the process will be extinct in finite time almost surely.
Similar to \cite{RenSongSun2019Spine,RenSongZhang2018Williams}, we add the following non-persistent assumption:
\begin{asp}~ 
  \label{asp:3}
  \begin{enumerate}
  \item \label{subsup:point_non_presistence}
    % $\mathbb P_{x}(\zeta < t)>0$ for each $x\in E$ and $t>0$.
    $\mathbb P_{x}(\|X_t\| = 0)>0$ for each $x\in E$ and $t>0$.
  \item \label{subasp:measure_non_presistence}
    % $\mathbb P_{\nu}(\zeta< t)>0$ for some $t>0$ with $\nu(dx):=\widehat\phi(x)m(dx)$.
    $\mathbb P_{\nu}(\|X_t\| = 0)>0$ for some $t>0$.
  \end{enumerate}
\end{asp}
 Let $\zeta=\inf\{t>0: \|X_t\| = 0\}$ be the extinction time of the superprocess $X$.
 The above assumption says that the process will extinct in finite time with a positive probability provided the initial configuration is $\nu$ or $\delta_x$ with $x\in E$.

Define a new kernel $n^\phi(x, dr)$ from $E$ to $(0,\infty)$ such that
\begin{equation} \label{eq:phi_change}
	\int_0^\infty f(r)n^\phi(x,dr)=\int_0^\infty f(r\phi(x))n(x, dr),
	\quad x\in E, f\in \mathcal B_b((0,\infty), \mathbb R).
\end{equation}
Our first result is about the asymptotic behavior of the extinction time $\zeta$:

\begin{thm}\label{thm:distribution_of_zeta}
	Suppose that the superprocess $X$ satisfies Assumptions \ref{asp:1},
  \ref{asp:IU}, and \ref{asp:3}. Then,
  \begin{enumerate}
  \item
    \label{subthm:extinct_almost_sure}
    for each $\mu \in \mathcal M(E)$, we have  $\mathbb P_\mu(\zeta<\infty)=1$;
  \item
    for each $\mu,\widetilde\mu\in \mathcal M(E)\setminus\{0\}$ and $s>0$, we have
    \[
      \lim_{t\rightarrow\infty}\dfrac{\mathbb P_{\mu}(\zeta>t+s)}{\mathbb P_{\widetilde\mu}(\zeta>t)}=\frac{\langle \phi,\mu\rangle }{\langle \phi,\widetilde\mu\rangle }e^{\lambda s};
    \]
  \item
  \label{subthm:LlogL}
    there exists a constant $k\in [0,\infty)$, such that for any $x\in E$,
    \begin{equation}\label{eq:decay_rate}
      \lim_{t\rightarrow\infty} e^{-\lambda t}\mathbb P_x(\zeta>t)=k\phi(x).
    \end{equation}
    Moreover, the constant $k>0$ if and only if $\int_E \widehat\phi(y)l(y)m(dy)<\infty$ where
    \begin{equation}\label{eq:m}
      l(y):=\int_1^\infty r\log r~n^\phi(y, dr),\quad y \in E.
    \end{equation}
  \end{enumerate}
\end{thm}

In particular, for any $x,y\in E$ and $s\geq 0$, the second result in the above theorem can be written as
\begin{equation}\label{eq:ratio_result}
 	\lim_{t\rightarrow\infty}\frac{\mathbb P_x(\zeta>t+s)}{\mathbb P_y(\zeta>t)}=\frac{\phi(x)}{\phi(y)}e^{\lambda s}.
\end{equation}
So we can see that the effect of the position of the initial mass on the decay of the mass is a ratio of $\phi(\cdot)$ generally.
For each probability ${\mathbf P}$ on $\mathcal M(E)$, we define
\[
	( {\mathbf P} \mathbb P)(\cdot) := \int_{\mathcal M(E)} \mathbb P_\mu(\cdot) {\mathbf P}(d\mu).
\]
Then $\{(X_t)_{t\geq 0}; ({\mathbf P}\mathbb P)\}$ can be considered as a $(Y,\psi)$-superprocess with a random initial value $X_0$ whose distribution is ${\mathbf P}$.
We say a probability ${\mathbf P}$ on $\mathcal M(E)$ is a \emph{quise-stationary distribution (QSD)} of the superprocess $X$ if  for each $t\geq 0$,
\[
	({\mathbf P}\mathbb P)(X_t \in \cdot | \zeta > t) ={\mathbf P}(\cdot).
\]
According to the standard theory of QSD (see \cite{MeleardVillemonais2012Quasistationary}), if ${\mathbf P}$ is a QSD of $X$, then under $({\mathbf P}\mathbb P)$, the lifetime $\zeta$ has an exponential distribution with some constant $r > 0$, that is
\[
	( {\mathbf P}\mathbb P)(\zeta > t) = e^{-r t}.
\]
We refer to $r$ the \emph{rate of mass decay} associated to the QSD $\mathbf P$.

We say a probability ${\mathbf P}$ on $\mathcal M(E)$ is the \emph{Yaglom distribution} of the superprocess $X$ if for any $\mu\in \mathcal M(E)\setminus\{0\}$ we have
\[
	\mathbb P_\mu(X_t \in \cdot | \zeta > t) \xrightarrow[t\to \infty]{w} {\mathbf P}(\cdot).
\]
If the Yaglom distribution ${\mathbf P}$ of the superprocess $X$ exists, then it must be a QSD of $X$ (see \cite{MeleardVillemonais2012Quasistationary}).

Our second theorem is about the QSD and the Yaglom distribution of the superpocess $X$:
{\bf (The following theorem is not correct. But we want to get something similar.)}
\begin{thm}\label{thm:qsd_thm}
  Suppose that the superprocess $X$ satisfies Assumptions \ref{asp:1}, \ref{asp:IU}, and \ref{asp:3}.
  Then,
  \begin{enumerate}
  \item \label{thm:qsd_thm_1}
    for each $\gamma\in[\lambda,0)$, there is a unique probability measure ${\mathbf P}^{\gamma}$ on $\mathcal M(E)$ such that $ {\mathbf P}^\gamma$ is a $QSD$ of the superprocess $X$ with rate of mass decay $-\gamma$.
    Let $\nu(dx):=\widehat\phi(x) m(dx)$.
    Then ${\mathbf P}^\gamma$ is the distribution of the random measure $M^{(\gamma)}\nu(dx)$ where $M^{(\gamma)}$ is a non-negative random variable with Laplace transform
    \[
      E[e^{-\theta M^{(\gamma)}}]
      = 1 - e^{\gamma B(\theta)},
      \quad \theta \in (0,\infty).
    \]
    Here, map $B: \theta \mapsto B(\theta)$ with $\theta \in (0,\infty)$ is defined as the inverse of the map
    \[
      t
      \mapsto -\log \mathbb P_\nu(\zeta \leq t),
      \quad t\in (0,\infty).
    \]
  \item
    ${\mathbf P}^\lambda$ is the Yaglom distribution of $X$.
    In another word, for any  $f\in\mathcal B_b(E,[0,\infty))$ and any $\mu\in \mathcal M(E)\setminus\{0\}$,
    \[
      \lim_{t\rightarrow\infty}\mathbb P_{\mu}[e^{-\langle f,X_t\rangle} | \zeta>t]
      % = E [e^{- \langle f,\nu\rangle M^{(\lambda)}}],
      = E [e^{- \langle f,\nu\rangle M^{(\lambda)}}].
    \]
    In particular, for each $\mu \in \mathcal M(E)$, $\{\langle \phi, X_t\rangle; \mathbb P_{\mu}(\cdot| \zeta > t) \}$ converges to $M^{(\lambda)}$ in distribution.
  \item
    There is no QSD with rate of mass decay $-\gamma$ for each $\gamma\in(-\infty , \lambda)$.
  \end{enumerate}
\end{thm}

{\bf (The following proposition is not correct. But we still want something similar and we want to call it the LlogL criteria for the subcritical superprocesses.)}
\begin{prop}\label{eq:exp_prop}
	Suppose that the superprocess $X$ satisfies Assumptions \ref{asp:1},
  \ref{asp:IU}, and \ref{asp:3}.
	Let $M^{(\gamma)}$ be the random variables given by Theorem \ref{thm:qsd_thm} (1) where $\gamma \in [\lambda, 0)$.
  Then
  \begin{enumerate}
  \item
    $E[M^{(\gamma)}] = \infty$ for each $\gamma \in (\lambda, 0)$, and
  \item
    $E[M^{(\lambda)}] < \infty$ if and only if $\int_E \widehat\phi(y)l(y)m(dy)<\infty$.
  \end{enumerate}
  Moreover, if $\int_E \widehat\phi(y)l(y)m(dy)<\infty$ then the constant $k$ in \eqref{eq:decay_rate} is equal to $E[M^{(\lambda)}]^{-1}$.
\end{prop}
% Recall the LlogL criteria of Theorem \ref{thm:equivalent_for_cbp}, we can say the above equivalency in proposition \ref{eq:exp_prop} is the same to that between items $(i)$ and $(iii)$ there.

Define the process
\[
	M_t=e^{-\lambda t}  \langle \phi, X_t\rangle, \quad t\geq 0.
\]
It is well known that the process  $(M_t)_{t\geq 0}$ is a martingale with respect to the natural filtration $(\mathscr F_t)_{t\geq 0}$ of the superprocess $X$.
% For each $\mu \in \mathcal M(E)$, define probability $\widetilde{\mathbb P}_\mu$ as Doob's $h-$transform of $\mathbb P_\mu$
For each $\mu \in \mathcal M(E)$, let probability $\widetilde{\mathbb P}_\mu$ be Doob's $h-$transform of $\mathbb P_\mu$ such that
\begin{equation} \label{eq:martingale_transformation}
	\frac{d\widetilde{\mathbb P}_\mu|_{\mathscr F_t}}{d\mathbb P_\mu|_{\mathscr F_t}}
	=\frac{M_t}{\langle\phi,\mu\rangle },
	\quad t\geq 0.
\end{equation}
This kind of martingale measure transformation for branching processes and measure-valued processes have been widely studied.
We refer to the early papers \cite{EnglanderKyprianou2004Local,Evans1993Two,RoellyRouault1989Processus}, the thesis \cite{Penisson2010Conditional} and the references therein, and the recent papers \cite{ChampagnatRoelly2008Limit,RenSongSun2019Spine,RenSongZhang2018Williams}.
It is well known that the process $\{(X_t)_{t\geq 0}; \widetilde{\mathbb P}_{\mu}\}$ can be characterized by the so called spine decomposition theorem.
We will recall this decomposition in details for our model in section $2$.

Our third theorem says that $\{(X_t)_{t\geq 0}; \widetilde{\mathbb P}_{\mu}\}$ can be considered as the Q-processs of $X$, i.e. the process $\{(X_t)_{t\geq 0}; \mathbb P_{\mu}\}$ conditioned to be never extinct:
\begin{thm}\label{thm:Q_process}
	Under the assumptions \ref{asp:1},\ref{asp:IU} and \ref{asp:3}, for each  $\mu \in \mathcal M(E), t\geq 0$ and $A\in\mathscr F_t$, we have $\lim_{s\rightarrow\infty}\mathbb P_\mu(A |\zeta>s)=\widetilde{\mathbb P}_\mu(A). $
\end{thm}

It would be interesting to study the asymptotic behavior of this Q-process $\{(X_t)_{t\geq 0}; (\widetilde{\mathbb P}_\mu)_{\mu \in \mathcal M(E)}\}$.
Samilar to the definition of $\mathbf P\mathbb P$, for each probability $\mathbf
P$ on $\mathcal M(E)$, we define probability $\mathbf P\widetilde{\mathbb P}$.
Then $\{(X_t)_{t\geq 0}; (\mathbf P\widetilde{\mathbb P})\}$ can be considered as the Q-process with a random initial value $X_0$ whose distribution is $\mathbf P$.
We say a probability $\mathbf P$ on $\mathcal M(E)$ is an \emph{invariant probability} of the Q-process $\{(X_t)_{t\geq 0}; (\widetilde{\mathbb P}_\mu)_{\mu\in\mathcal M(E)}\}$ if
\[
	(\mathbf P\widetilde{\mathbb P})(X_t \in \cdot ) =\mathbf P(\cdot),	\quad t\geq 0.
\]

Our fourth result is the following:
{\bf (The following theorem is not entirely correct.)}
\begin{thm}\label{thm:structure_of_Qprocess}
	Under the assumptions \ref{asp:1},\ref{asp:IU} and \ref{asp:3}:
  \begin{enumerate}
  \item
    If $\int_E\widehat\phi(x)l(x)m(dx)<\infty$, then $\{(X_t)_{t\geq 0};(\widetilde{\mathbb P}_\mu)_{\mu\in\mathcal M(E)}\}$ has an invariant probability ${\mathbf P}$.
    For any $\mu\in\mathcal M(E)$, we have
\begin{align}
\label{eq:uniqueness_of_invariant_probability}
\widetilde{\mathbb P}_\mu(X_t \in \cdot ) \xrightarrow[t\to \infty]{w} {\mathbf P}(\cdot).
\end{align}
    The invariant probability $\mathbf P$ is the distribution of the random measure $M\widehat\phi(x)m(dx)$, where the non-negative random variable $M$ has Laplace transform
    \[
      E[e^{-\theta  M}] = \dfrac{E[M^{(\lambda)}e^{-\theta M^{(\lambda)}}]}{E[M^{(\lambda)}]},\quad \theta > 0.
    \]
  \item
    If $\int_E\widehat\phi(x)l(x)m(dx)<\infty$, then there is a random measure $K$ on $(0,\infty)$ such that
    \[
      E[e^{-\theta \widetilde M}] = E\Big[\exp\Big\{- \int_{(0,\infty)} (1-e^{-\theta z }) K(dz) \Big\}\Big].
    \]
  \item
    If $\int_E\widehat\phi(x)l(x)m(dx)=\infty$, then for each $\mu \in \mathcal M(E)$, we have $\lim_{t\rightarrow\infty}\langle \phi, X_t\rangle =\infty$ in probability with respect to $\widetilde{\mathbb P}_\mu$.
  \end{enumerate}
\end{thm}
{\bf (Why we only considered the existence of that equilibrium probability in the above theorem? Is it unique? Also, can I have this conjecture: Under $\mathbf P \widetilde{\mathbb P}$, $(X_t)_{t\geq 0}$ is a stationary (or even better, an ergodic) measure-valued process?)}

One important technique used to prove the above theorems is a ``spine-decomposition'' for the super-diffusion $X$ under a martingale change of measure.
This decomposition was used by Englander and Kyprianou in \cite{EnglanderKyprianou2004Local} to investigate the local extinction of super-diffusions, in which the branching mechanism is $\psi(x,z)-\beta(x)z=\alpha(x)^2z^2-\beta(x)z$.
This technique is usually used to investigate the properties of supercritical superdiffusions ($\lambda>0$).
Here we use it to analyze the subcritical case.
\end{comment}

% *** Main results
\subsection{Main results}
\label{sec:MR}
% **** Notations
Denote by $\mathcal B(E,B)$ the collection of all Borel measurable maps from a Borel space $E$ to a measurable space $B$.
Denote by $\mathcal B_b(E,B)$ the collection of all bounded elements in $\mathcal B(E,B)$.
Denote by $\mathcal M(E)$ the space of all Borel measures on a Borel space $E$. 
Denote by $\mathcal M_f(E)$ the space of all finite elements in $\mathcal M(E)$ equipped with the weak topology. 

% **** Definition of (S)uper(P)rocesses
\begin{defi}
\label{def:SP}
Let $E$ be a locally compact separable metric space. 
We say an $\mathcal M_f(E)$-valued Hunt process $X= \{(X_t)_{t\geq 0}; (\mathbb P_\mu)_{\mu\in \mathcal M_f(E)}\}$ is a \emph{$(Y,\psi)$-superprocess} if
\begin{enumerate}
\item
\label{def:SP:1}
  the \emph{spatial motion} $Y=\{(Y_t)_{t\geq 0};(\Pi_x)_{x\in E}\}$ is an $E$-valued Hunt process with its life time denoted as $\tau$;
\item
  \label{def:SP:2}
  the \emph{branching mechanism} $\psi$ is a map from $E\times[0,\infty)$ to $[0,\infty)$ given by
  \[
    \psi(x,z)
    =-\beta(x)z + \sigma(x)^2 z^2+ \int_{(0,\infty)} (e^{-rz}-1+zr ) \pi(x, dr),\qquad x\in E, z\geq0,
  \]
  where $\beta, \sigma\in \mathcal B_b(E,\mathbb R)$ and $\pi$ is a kernel from $E$ to $(0,\infty)$ such that
  \[
    \sup_{x\in E}\int_0^\infty (r\wedge r^2)\pi(x,dr)
    <\infty.
  \]
\item
  \label{def:SP:3}
  for each $f\in \mathcal B_b(E,[0,\infty))$, it holds that
  \begin{align}
    \mathbb P_\mu [e^{- X_t(f)}] = e^{-\mu(V_tf)},
    \quad \mu \in \mathcal M_f(E),t\geq 0,
  \end{align}
  where the map $(t,x) \mapsto V_tf(x)$ on $[0,\infty) \times E$ is the unique locally bounded positive solution to the equation
  \begin{align}
    V_t f(x) +   \Pi_x\Big[\int_0^{t\wedge \tau} \psi \big(\xi_s,V_{t-s} f(\xi_s)\big) ds\Big]
    = \Pi_x[f(\xi_t) \mathbf 1_{t < \tau}],
    \quad x \in E, t \geq 0.
  \end{align}
  Here, we say a map $(t,x) \mapsto f(t,x)$ on $[0,\infty)\times E$ is \emph{locally bounded} if for each $t_0 \geq 0$, $\sup_{t\in [0,t_0], x\in E} |f(t,x)| < \infty$.
\end{enumerate}
\end{defi}

% **** Theorem (Y)glom
\begin{thm}[Under the condition of Lemma \ref{lem:K}]
  \label{thm:Y}
	Let $X$ be the superprocess given by Definition \ref{def:SP}. 
  We say a probability ${\mathbf P}$ on $\mathcal M_f(E)$ is the \emph{Yaglom distribution} of the superprocess $X$ if for each $\mu\in \mathcal M_f(E)\setminus\{0\}$ we have
  \begin{align}
    \mathbb P_\mu(X_t \in \cdot | \|X_t\|>0) 
    \xrightarrow[t\to \infty]{law} {\mathbf P}(\cdot).
  \end{align}
  Then it holds that the Yaglom distribution $\mathbf P$ of $X$ exists.
\end{thm}
% ** Preliminary
\section{Preliminary}
% *** Preliminary results
\subsection{Superprocesses}
% ***** Lemma (C)umulant semigroup
  \begin{lem}[Fact]
    \label{lem:C}
    Let $X$ be the superprocess given in Definition \ref{def:SP}.
    There exists a unique family of operators $(\overline V_t)_{t \geq 0}$ from $\mathcal B(E, [0,\infty])$ to $\mathcal B(E, [0,\infty])$ satisfying the followings
    \begin{enumerate}[label=(\alph*)]
    \item
      for each $t\geq 0$ and $f \in \mathcal B_b(E, [0,\infty))$, we have $\overline V_tf = V_tf$.
    \item
      for each $t\geq 0$ and each $f_n \uparrow f$ pointwisely in $\mathcal B(E, [0,\infty])$, we have $\overline V_tf_n \uparrow \overline V_tf$ pointwisely.
    \end{enumerate}
    Moreover, $(\overline V_t)_{t\geq 0}$ satisfies that
    \begin{enumerate}
    \item
      for each $t\geq 0$ and $f\leq g$ in $\mathcal B(E,[0,\infty])$, we have $\overline V_tf \leq \overline V_tg$;
    \item
      for each $t, s\geq 0$, we have $\overline V_{t+s} = \overline V_t \overline V_s$;
    \end{enumerate}
    With some abuse of the notations, we still write $V_t = \overline V_t$ for each $t\geq 0$, and call $(V_t)_{t\geq 0}$ the \emph{cumulant semigroup} of the superprocess $X$.
  \end{lem}

% ***** Lemma: the (v)_t
  \begin{lem}[Fact]
    \label{lem:v}
    Define $v_t = V_t(\infty)$ for each $t\geq 0$, then it holds that 
    \begin{align}
      \mathbb P_\mu (\|X_t\| = 0) 
      = e^{- \mu(v_t)}
      , \quad \mu \in \mathcal M_f(E), t\geq 0.
    \end{align}      
  \end{lem}
  
% ***** Lemma (Mean)
  \begin{lem}[Fact]
    \label{lem:M}
    Let $X$ be a superprocess given by Definition \ref{def:SP}. 
    It holds that
    \begin{align} 
      \mathbb P_\mu [X_t(f)] = \mu(P^\beta_t f), 
      \quad t \geq 0, f\in \mathcal B_b(E, \mathbb R),
    \end{align} 
    where \emph{the mean semigroup} $(P_t^\beta)_{t\geq 0}$ is a family of operators on $\mathcal B_b(E, \mathbb R)$ given by
    \begin{align}
      P^\beta_tf(x)
      := \Pi_x \big[e^{\int_0^{t} \beta(Y_r)dr} f(Y_t)\mathbf 1_{\{t<\tau\}}\big],
      \quad f \in \mathcal B_b(E, \mathbb R),t\geq 0, x\in E.
    \end{align}
  \end{lem}
% ***** Lemma (E)quations of superprocesses 
\begin{comment}
  \begin{lem}[Fact]
    \label{lem:E}
    Let $X$ be the superprocess given in Definition \ref{def:SP}. 
    Define a function $\psi_0$ on $E\times [0,\infty)$ such that $ \psi_0(x,z)  := \psi(x,z)+ \beta(x) z$ for each  $x\in E$ and  $z \geq 0$.
    Using the monotonicity, extend $\psi$ as a $[0,\infty]$-valued function on $E \times [0,\infty]$ by setting that $\psi(x,\infty) = \lim_{z\uparrow \infty} \psi(x,z)$.
    Define operator $\Psi_0$ on $\mathcal B(E,[0,\infty])$ by $ \Psi_0 g(x):=\psi_0(x, g(x))$.
    Then we have
    \begin{align}
      V_sf+ \int_0^s P_{s-u}^\beta \Psi_0V_u f du = P_s^\beta f,
      \quad s\geq 0, f\in \mathcal B(E, [0,\infty]).
    \end{align}
  \end{lem}
\end{comment}
% ***** Lemma (E)igenvalue
  \begin{lem}[Need more conditions]
    \label{lem:S}
    Let $X$ be the superprocess given by Definition \ref{def:SP}. 
    Let $(P^\beta_t)_{t\geq 0}$ be the mean semigroup of $X$.
    Then there exist a number $\lambda \in \mathbb R$, a function $\phi \in \mathcal B(E, (0,\infty))$ and a measure $\nu \in \mathcal M_f(E)$ satisfying that
    \begin{align}
      & P_t^\beta \phi(x) 
        = e^{\lambda t} \phi(x),
        \quad x\in E, t\geq 0;
      \\ & \int_{x\in E} \nu(dx) P_t^\beta(x,\cdot) 
           = e^{\lambda t} \nu(\cdot),
           \quad t\geq 0;
      \\ & \nu(\phi) 
           = 1.   
    \end{align}
  \end{lem}
% ***** Proof of Lemma lem:S
  \begin{proof}
    {\bf TBD, obviously true if we make the IU assumption.}
  \end{proof}

% ***** The (K)ey Lemma
  \begin{lem}[Under the condition of Lemma \ref{lem:S} and maybe more conditions]
    \label{lem:K}
    Suppose that 
    \begin{enumerate}[label=(\alph*)]
    \item
      \label{lem:K:a}
      $\lambda < 0$;
    \item
      \label{lem:K:b}
      $P_t^\beta f(x) 
      = e^{\lambda t} \phi(x) \nu(f) \big( 1+ o(\limsup_{t\to \infty} \sup_{x\in E, f\in L_1^+(\nu)})\big)$;
    \item
      \label{lem:K:c}
      for each $t\geq 0$, $ P_t^\beta f(x) = \phi(x) \nu(f) O(\sup_{x\in E, f\in L_1^+(\nu)})$.
    \end{enumerate}
    Then for each $f\in \mathcal B(E, (0,\infty])$ the followings hold:
    \begin{enumerate}
    \item 
      \label{lem:K:1} 
      $V_tf = o(\limsup_{t\to \infty} \sup_{x\in E})$;
    \item 
      \label{lem:K:2}
      for each $s\geq 0$, we have $\nu(V_{t+s}f) = e^{\lambda s} \nu(V_tf) (1+o(\limsup_{t\to \infty}))$.
    \item 
      \label{lem:K:3}
      $ (\phi^{-1} \cdot V_tf)(x) = \nu(V_tf) \big( 1+ o(\limsup_{t\to \infty} \sup_{x\in E}) \big)$
    \end{enumerate}
  \end{lem}

% ***** Proof of Lemma lem:K
  \begin{proof}
    {\bf TBD, almost proved in the previous version under the IU assumption.}
  \end{proof}
% ***** Lemma (Y)aglom: definition of (Ga)mma
  \begin{lem}[Under the condition of Lemma \ref{lem:K}]
    \label{lem:Y:Ga}
    $\mathbf P_\mu(\|X_t\| > 0)>0$ for each $t\geq 0$ and $\mu \in \mathcal M_f(E)\setminus \{0\}$.
  \end{lem}
% ***** Proof of Lemma lem:Y:Ga  
  \begin{proof}
    For each $t\geq 0$, since $\mathbb P_\nu[X_t(\phi)] = \nu(P_t^\beta \phi) = e^{\lambda t} > 0$, we have the desired result for $\mathbb P_\nu$.
    {\bf (We still need to show this for all $\mathbb P_\mu$.)}
  \end{proof}

% *** Concave functions
\subsection{Concave functions}
% ***** Lemma: (C)onvexity of the Laplace exponents
\begin{lem}
  \label{lem:C}
  Suppose that $\{Z; P\}$ is a non-degenerate $[0,\infty]$-valued random variable. 
  Write $L(u):= - \log P[e^{- u Z}]$ with $u \in (0,\infty)$, then $L$ is a concave function.
\end{lem} 
% ***** Proof of Lemma C
\begin{proof}
	It can be verified that
  \begin{align}
    \frac{\partial L(u)}{\partial u} = P[e^{-u Z}]^{-1} P[Ze^{- u Z}]
    , \quad u > 0.
  \end{align}
  It can also be verified that
  \begin{align}
    & \frac{\partial^2 L(u)}{\partial u^2}
    = P[e^{-uZ}]^{-2}\Big( \frac{\partial P[Ze^{-uZ}]}{\partial u} \cdot P[e^{-uZ}] - P[Ze^{-uZ}] \cdot \frac{\partial P[e^{-uZ}]}{\partial u}\Big) \\
    & = - \frac{P[Z^2 e^{-uZ}]}{P[e^{-uZ}]} + \frac{P[Ze^{-uZ}]^2}{P[e^{-uZ}]^2} \\
    & = Q^{(u)}[Z]^2 - Q^{(u)}[Z^2] 
      \leq 0
      , \quad u \in (0,\infty).
  \end{align}
  Here, for each $u$, the probability measure $Q^{(u)}$ is given by $dQ^{(u)}:= \frac{e^{-uZ}}{P[e^{-uZ}]} dP$; and in the last step, we used Jason's inequality.
\end{proof}
% ***** Lemma that (Q) is concave
\begin{lem}
	\label{lem:Q}
  Suppose that $g(u)$ is a concave function on $u\in (0,\infty)$, then so is $q(u):= 1- e^{-g(u)}$.
\end{lem}
\begin{proof}
For each $u,v \in (0,\infty)$ and $r \in [0,1]$, writing $\bar r = 1-r$, we have
\begin{align}
	&q(ru+\bar r v) 
  = 1 - e^{- g(ru + \bar r v)}
  \geq 1 - e^{- \big( r g(u) + \bar r g(v)\big)} \\
  & \geq r(1- e^{- g(u)}) + \bar r (1 - e^{- g(v)})
    = rq(u) + \bar r q(v).
    \qedhere
\end{align}	
\end{proof}
% ** Proof of the Main results
\section{Proof of the main results.}
% *** Proof of Theorem (Y)aglom
\subsection{Proof of Theorem \ref{thm:Y}}
% **** Proof of Theorem (Y)aglom by admitting Lemma 
\begin{proof}[Proof of Theorem \ref{thm:Y} by admitting Lemma \ref{lem:Y:Gt} and \ref{lem:Y:Gs}]

Thanks to Lemma \ref{lem:Y:Ga}, we can define a family of $[0,\infty]$-valued functionals $(\Gamma_t)_{t\geq 0}$ on $\mathcal B(E, (0,\infty])$ such that
\begin{align}
  e^{-\Gamma_t f}:= \mathbf P_{\nu}\big[e^{- X_t(f)}\big| \|X_t\| > 0\big], 
  \quad f\in \mathcal B(E,(0,\infty]), t \geq 0.
\end{align}
% ***** Lemma (Y)aglom: definition of (Gt) 
\begin{lem}[Under the condition of Lemma \ref{lem:K}]
  \label{lem:Y:Gt}
  For each increasing time sequence $\mathbf t:= (t_n)_{n\in \mathbb N}$ with $ t_n \uparrow \infty$, we have that $G^{\mathbf t}(\infty \mathbf 1_E) = \infty$ and that 
  \begin{align}
    1 - e^{- G^{\mathbf t}V_sf} 
    = e^{s\lambda} \big(1- e^{- G^{\mathbf t} f}\big),
    \quad s \geq 0, f \in \mathcal B(E, (0,\infty])
  \end{align}
  where $G^{\mathbf t}$ is a $[0,\infty]$-valued monotone concave functional on $\mathcal B(E,(0,\infty])$ given by
  \begin{align}
    G^{\mathbf t}f 
    := \liminf_{n\to \infty} \Gamma_{(t_n)}f,
    \quad f\in \mathcal B(E, (0,\infty]).
  \end{align}
  Here, we say a functional $A$ defined on $\mathcal B(E,(0,\infty])$ is monotone if $f\leq g$ in $\mathcal B(E,(0,\infty])$ implies that $Af\leq Ag$; say $A$ is concave if for each $f\in \mathcal B(E,(0,\infty])$ map $u\mapsto A(uf)$ is concave in $u \in (0, \infty)$.  
\end{lem}
% ***** Lemma (Y)aglom: are G^t are (G)-(s)tar
\begin{lem}[Under the condition of Lemma \ref{lem:K} and \ref{lem:Y:Gs:Q}]
  \label{lem:Y:Gs}
  Suppose that $G^*$ is a $[0,\infty]$-valued monotone concave functional on $\mathcal B(E,(0,\infty])$ satisfying that $G^*(\infty \mathbf 1_E) = \infty$ and that 
  \begin{align}
    1 - e^{- G^* V_sf} = e^{s\lambda} \big(1- e^{- G^* f}\big),
    \quad s \geq 0, f \in \mathcal B(E, (0,\infty]).
  \end{align}
  Then for each time sequence $\mathbf t:= (t_n)_{n\in \mathbb N}$ with $ t_n \uparrow \infty$, we have $G^* = G^{\mathbf t}$.
\end{lem}
% ***** .
From the above two lemmas, using a sub-sub-sequence argument, we have that 
\begin{align}
  \label{eq:Y:1}
	Gf
  := \lim_{t\to \infty} \Gamma_tf, \quad f\in \mathcal B(E,(0,\infty])
\end{align}
is well-defined, and that $G$ is the unique $[0,\infty]$-valued monotone concave functional on $\mathcal B(E,(0,\infty])$ satisfying that $G(\infty \mathbf 1_E) = \infty$ and that
\begin{align}
  1 - e^{- G V_sf} = e^{s\lambda} \big(1- e^{- G f}\big),
  \quad s \geq 0, f \in \mathcal B(E, (0,\infty]).
\end{align}
Taking $f = \infty$ in the above equation, we have that $1 - e^{- Gv_s} = e^{\lambda s}$ for each $s\geq 0$.

We now claim that for each sequence of functions $(g_{n})_{n\in \mathbb N}$ in $\mathcal B(E, (0,\infty])$ such that $\lim_{n \to \infty} g_n = 0$ bounded pointwisely, we have $\lim_{n \to \infty} G g_n = 0$.
In fact, fixing such sequence $(g_{n})_{n\in \mathbb N}$, from Lemma \ref{lem:K}\ref{lem:K:c}, there exists a constant $C > 0$ such that 
\begin{align}
	V_1 g_n(x) \leq P^\beta_1 g_n(x) \leq C \phi(x) \nu(g_n),
  \quad n \in \mathbb N, x\in E.
\end{align}
Fixing this $C>0$, from the bounded convergence theorem, we have $\lim_{n\to \infty}C \nu(g_n) =0$.
{\bf (Here, we should be able to show that map $t \mapsto \nu(v_t)$ is a strictly decreasing 1-1 map from $(0,\infty)$ to $(0,\infty)$.)}
Denote by $R:(0,\infty) \to (0,\infty)$ as the inverse of the map $t \mapsto \nu(v_t)$, then it is a strictly decreasing $1-1$ map from $(0,\infty)$ to $(0,\infty)$. 
Define $t_n := R(2C\nu(g_n))> 0$.
Then we have that $\lim_{n\to \infty} t_n = \infty$ and that 
\begin{align}
	V_1 g_n(x) \leq \frac{1}{2} \phi(x) \nu(v_{(t_n)}), 
\quad n \in \mathbb N, x\in E.
\end{align}
Now, taking $f = \infty$ and $t = t_n$ in Lemma \ref{lem:K}\eqref{lem:K:3} we get that
\begin{align}
 (\phi^{-1} \cdot v_{(t_n)})(x) 
  = \nu(v_{(t_n)}) \big( 1+ o(\limsup_{n\to \infty} \sup_{x\in E}) \big).
\end{align}
In particular, we have that, for $n$ large enough,
\begin{align}
	\nu(v_{t_n}) \leq 2 \phi(x)^{-1} v_{(t_n)}(x), \quad x\in E.
\end{align}
Therefore, for $n$ large enough, we have that $V_1g_n \leq v_{(t_n)}$.
This implies that
\begin{align}
	& 1 - e^{- Gg_n}
  = e^{- \lambda} (1- e^{- GV_1g_n})
  \leq e^{- \lambda} (1- e^{- G v_{(t_n)}}) \\
  & = e^{- \lambda} e^{\lambda t_n},
  \quad \text{$n$ large enough.}
\end{align} 
Recall that $\lambda < 0$, we get the desired claim.

Now from \cite[Proposition 1.19]{Li2011MeasureValued}, \eqref{eq:Y:1} and the above claim, we have that there exists a unique probability $\mathbf P$ on $\mathcal M_f(E)$ such that 
\begin{align}
  e^{-Gf}
  = \int_{\mathcal M_f(E)} e^{- \mu(f)} \mathbf P(d\mu)
  , \quad f\in C_b(E, (0,\infty)),
\end{align}
and that
\begin{align}
	\mathbb P_{\nu}\big(X_t \in \cdot \big| \|X_t\|>0 \big) 
  \xrightarrow[t\to \infty]{law} \mathbf P(\cdot).
\end{align}
({\bf We still need to show for each $\mathbb P_\mu$ the Yaglom law exists.})
\end{proof}
% **** Proof of Lemma Y:Gt
\begin{proof}[Proof of Lemma \ref{lem:Y:Gt}]
  From $\Gamma_t(\infty \mathbf 1_E) = \infty$ for each $t\geq 0$, we have that $G^{\mathbf t}(\infty \mathbf 1_E) = \infty$.
  Notice that
  \begin{align}
    \label{eq:Y:Gt:1}
    & 1 - e^{- \Gamma_t f} 
    = \mathbf P_\nu \big[ 1 - e^{-X_t(f)} \big| \|X_t\|> 0\big] \\
    & = \frac{ \mathbf P_\nu [ 1 - e^{- X_t(f)}]}{ \mathbf P_\nu (\|X_t\| > 0)}
    = \frac{ 1 - e^{- \nu(V_tf)} }{ 1 - e^{- \nu(v_t)}},
    \quad t \geq 0, f \in \mathcal B(E,[0,\infty]).
  \end{align}
  Therefore,
  \begin{align}
    & 1 - e^{- \Gamma_t V_s f}
      = \frac{ 1 - e^{- \nu(V_{t+s} f)} }{ 1 - e^{- \nu(v_t)}}
     = \frac{ 1 - e^{- \nu(V_{t+s} f)} }{ 1 - e^{- \nu(V_tf)}} \frac{ 1 - e^{ - \nu(V_tf)}}{ 1 - e^{- \nu(v_t)}} \\
    & = \frac{ 1 - e^{- \nu(V_{t+s} f)} }{ 1 - e^{- \nu(V_tf)}} ( 1 - e^{- \Gamma_t f})
      , \quad t\geq 0, s \geq 0, f\in \mathcal B(E, (0,\infty]).
  \end{align}
{\bf (Here, we need to show that $\nu(V_tf) > 0$ for each $t\geq 0$ and $f\in \mathcal B(E,(0,\infty])$.)}
  Thus,
  \begin{align}
    & 1 - e^{- G^{\mathbf t} V_s f}
      = \liminf_{n\to \infty} ( 1 - e^{- \Gamma_{(t_n)} V_s f})
      = \liminf_{n\to \infty} \Big(  \frac{ 1 - e^{- \nu(V_{t_n+s}f)}}{ 1 - e^{- \nu(V_{(t_n)}f)}} (1 - e^{- \Gamma_{(t_n)} f}) \Big) \\
    & = \Big( \lim_{t \to \infty}   \frac{ 1 - e^{- \nu(V_{t+s}f)}}{ 1 - e^{- \nu(V_{t}f)}} \Big) \cdot \liminf_{n\to \infty} (1 - e^{- \Gamma_{(t_n)} f} ) \\
    &  = e^{s\lambda} (1 - e^{- G^{\mathbf t}f})
      , \quad s\geq 0, f\in \mathcal B(E,(0,\infty]).
  \end{align}
  Here in the last step, we used Lemma \ref{lem:K}.

  We still need to verify that $G^{\mathbf t}$ is monotone and concave.
  Note that, for each $t\geq 0$, from Lemma \ref{lem:C}, $\Gamma_t$ is monotone and concave. 
  Therefore, for each $f \leq g$ in $\mathcal B(E,(0,\infty])$, we have
\begin{align}
	G^{\mathbf t} f = \liminf_{n\to \infty} \Gamma_{(t_n)} f
  \leq \liminf_{n\to \infty} \Gamma_{(t_n)} g
  = G^{\mathbf t} g.
\end{align}
This says that $G^{\mathbf t}$ is monotone. 
Similarly, for each $u,v \in (0,\infty)$, $f\in \mathcal B(E,(0,\infty])$ and $r \in [0,1]$, writing $\bar r = 1 - r$, we have that
\begin{align}
	& G^{\mathbf t}\big((ru + \bar rv)f\big)
  = \liminf_{n \to \infty} \Gamma_{(t_n)}\big((ru + \bar rv)f\big)\\
  & \geq \liminf_{n\to \infty} \big(r\Gamma_{(t_n)} (uf) + \bar r\Gamma_{(t_n)}(vf)\big) \\
 & \geq r \big(\liminf_{n\to \infty} \Gamma_{(t_n)} (uf)\big) + \bar r \big(\liminf_{n\to \infty} \Gamma_{(t_n)}(vf) \big) \\
  & = r G^{\mathbf t} (uf) + \bar r G^{\mathbf t}(vf).
\qedhere
\end{align} 
\end{proof}
% **** Proof of Lem:Y:Gs
\begin{proof}[Proof of Lemma \ref{lem:Y:Gs} by admitting Lemma \ref{lem:Y:Gs:Q}]
% ***** Lemma Lem:Y:Gs:Q
\begin{lem}
  \label{lem:Y:Gs:Q}
Define a family of $[0,\infty]$-valued functionals $(Q_t)_{t\geq 0}$ on $\mathcal B(E,(0,\infty])$ by
\begin{align}
	Q_tg 
  := e^{- \lambda t}\big( 1 - e^{-G^*(gv_t)} \big).
\end{align}
Then it holds that
\begin{align}
	\lim_{t\to \infty} Q_t(u \mathbf 1_E) 
  = u,
  \quad u \in [0,1].
\end{align}
\end{lem}

% ***** .
Fix an arbitrary time sequence $\mathbf t=(t_n)_{n\in \mathbb N}$ and an arbitrary function $f\in \mathcal B_b(E)$.
We only have to proof that $G^* f = G^{\mathbf t}f$.
From the definition of $G^{\mathbf t}f$, we can chose a subsequence $\mathbf t'=(t'_n)_{n \in \mathbb N}$ of $\mathbf t$ such that $t_n'\uparrow \infty$ and that  $ 1 - e^{- G^{\mathbf t}f} =  \big( 1 - e^{-\Gamma_{( t_n')} f} \big) \cdot \big(1+ o(\limsup_{n \to \infty}  )\big). $
Therefore, from \eqref{eq:Y:Gt:1} and  Lemma \ref{lem:K} we have
\begin{align}
  & 1 - e^{- G^{\mathbf t}f}
  = \frac{1 - e^{- \nu( V_{(t_n')}f)}}{1- e^{- \nu(v_{(t_n')})}}  \big(1+o(\limsup_{n\to \infty})\big) \\
  & = \frac{\nu (V_{(t_n')} f)}{\nu(v_{(t_n')})}\big(1+o(\limsup_{n\to \infty})\big) 
  =  \frac{V_{(t_n')}f(x)}{v_{(t_n')}(x)} \big( 1 + o(\limsup_{n \to \infty} \sup_{x\in E})\big).
\end{align}
Fix an arbitrary $\epsilon > 0$. 
From above, there exists $n$ large enough such that
\begin{align}
  (1-\epsilon) \big(1 - e^{- G^{\mathbf t}f} \big)
  \leq \frac{V_{(t_n')}f(x)}{v_{(t'_n)}(x)}
  \leq (1+\epsilon) \big( 1 - e^{- G^{\mathbf t}f} \big),
  \quad x\in E.
\end{align}
This implies that for $n$ large enough 
\begin{align}
  \label{eq:Y:Gs:1}
  & Q_{(t'_n)}\big[ (1-\epsilon) (1-e^{-G^{\mathbf t}f})\mathbf 1_E \big]
    \leq Q_{(t'_n)}\Big( \frac{V_{(t'_n)}f}{v_{(t'_n)}} \Big) 
    \leq Q_{(t'_n)}\big[  (1+\epsilon) (1-e^{-G^{\mathbf t}f}) \mathbf 1_E \big].
\end{align}
Note from the definition of $(Q_t)_{t\geq 0}$ and $G^*$, we always have that
\begin{align}
	Q_t\Big ( \frac{V_tf}{v_t} \Big ) 
  = e^{- \lambda t}( 1 - e^{- G^*V_tf}  )
  = 1- e^{- G^* f}.
\end{align}	
Therefore, taking $n \to \infty$ in \eqref{eq:Y:Gs:1}, from Lemma \ref{lem:Y:Gs:Q} we get that
\begin{align}
	(1 - \epsilon) (1 - e^{- G^{\mathbf t}f})
  \leq 1 - e^{- G^* f} 
  \leq (1 + \epsilon) (1 - e^{- G^{\mathbf t} f}).
\end{align}
Taking $\epsilon \to 0$, we get the desired result.
\end{proof}
% **** Proof of Lemma lem:Y:Gs:Q
\begin{proof}[Proof of Lemma \ref{lem:Y:Gs:Q}]
% ***** .
Note that, for each $s,t\geq 0$ and $x\in E$, $u\mapsto V_s(uv_t)(x)$ is the laplace exponent of random variable $X_s(v_t)$ under probability $\mathbb P_{\delta_x}$.
({\bf It can be verified that this random variable is non-degenerate.})
Therefore, Lemma \ref{lem:C} says that, for each $s,t\geq 0$ and $x\in E$, $u\mapsto V_s(uv_t)(x)$ is concave on $[0,\infty)$. 
Therefore, for $u\in [0,1]$, writing $\bar u = 1- u$, we get that
\begin{align}
	V_s(uv_t)
  =V_s\big((u\cdot 1 + \bar u \cdot 0)v_t\big) 
  \geq uV_s(v_t) + \bar u V_s(0\cdot v_t) 
  = uv_{s+t},
  \quad u\in [0,1], s,t\geq 0.
\end{align} 
Using this, we have the following: 
\begin{align}
  & Q_{t+s}(u\mathbf 1_E) 
    = e^{- \lambda (t+s)} \big( 1-e^{-G^*(uv_{t+s})} \big) \\
  &  \leq e^{- \lambda(t+s)}\big( 1-e^{-G^*[V_s(uv_t)]} \big) \\
  & = e^{-\lambda t}\big( 1-e^{-G^*(uv_t)} \big)
    = Q_t(u\mathbf 1_E)
  , \quad t,s\geq 0, u \in [0,1].
\end{align}
Therefore, the following limits are well define:
\begin{align}
\label{eq:Y:Gs:Q:1}
	q(u)
  := \lim_{t\to \infty} Q_{t}(u \mathbf 1_E) \in [0,\infty),
  \quad u \in [0,1].
\end{align}
In particular, from $G^*(\infty) = \infty$ we have that
\begin{align}
	Q_t(\mathbf 1_E) = e^{- \lambda t} \big( 1-e^{-G^*v_t} \big)
  = e^{- \lambda t} e^{\lambda t}\big( 1-e^{-G^*\infty} \big)
  = 1,
\end{align}
which says that $q(1) = 1$.

We claim that $q$ is continuous on $[0,1]$.
In fact, from the monotonicity of $G^*$ we have that for each $t>0$, $u \mapsto Q_t(u \mathbf 1_E)$ is non-decreasing. 
Therefore, $u \mapsto q(u)$ is non-decreasing.
Also, notice that $G^*$ is a concave functional, so $u \mapsto G^*(u v_t)$ is concave for each $t\geq 0$. 
This and Lemma \ref{lem:Q} imply that, for each $t\geq 0$, $u \mapsto Q_t(u \mathbf 1_E)$ is concave.
Therefore, $u \mapsto q(u)$ is concave in $u\in [0,1]$.
Finally, since {\bf any non-decreasing and concave function must be continuous}, we get the desired claim.

Now, from Lemma \ref{lem:K}\eqref{lem:K:2} and \ref{lem:K}\eqref{lem:K:3} we have that
\begin{align}
	& e^{\lambda s}(\phi^{-1}v_t)(x) 
  = e^{\lambda s}\nu(v_{t})\big(1+o(\limsup_{t\to \infty}\sup_{x\in E})\big) \\
  & =\nu(v_{t+s}) \big(1+o(\limsup_{t\to \infty}\sup_{x\in E})\big) \\
  & = (\phi^{-1}v_{t+s})(x) (1+o(\limsup_{t\to \infty} \sup_{x\in E}))
    , \quad s\geq 0.
\end{align}
Therefore, for each $s\geq 0$ and $\epsilon>0$ there exists $t$ large enough such that for each $x\in E$ we have
\begin{align}
	1-\epsilon\leq \frac{e^{\lambda s}v_t(x)}{v_{t+s}(x)} 
  \leq 1+\epsilon,
\end{align}
which implies that
\begin{align}
	& Q_{t+s}[ (1-\epsilon)u\mathbf 1_E ]
   = e^{-\lambda(t+s)}( 1-e^{-G^*[(1-\epsilon)uv_{t+s}]} ) \\
  & \leq e^{-\lambda t} e^{-\lambda s}( 1- e^{-G^*(ue^{\lambda s}v_t)} )
    = e^{-\lambda s}Q_t(ue^{\lambda s} \mathbf 1_E)
    , \quad u \in [0,1].
\end{align}
and similarly that
\begin{align}
	e^{-\lambda s}Q_t[ue^{\lambda s}\mathbf 1_E] 
  \leq Q_{t+s}[(1+\epsilon)u\mathbf 1_E]
  , \quad u \in [0,1].
\end{align}
Now taking $t\to \infty$ in the above two inequality, we get from \eqref{eq:Y:Gs:Q:1} that, for each $s\geq 0, \epsilon > 0$ and $u \in [0,1]$:
 \begin{align}
   q((1-\epsilon)u)\leq e^{-\lambda s}q(u e^{\lambda s}) \leq q((1+\epsilon)u \wedge 1).
 \end{align}
Taking $\epsilon \to 0$, from the previous claim that $q$ is continuous, we get that 
\begin{align}
	q(u) 
  = e^{- \lambda s} q(u e^{\lambda s}), 
  \quad u \in [0,1], s \geq 0. 
\end{align} 
Therefore, recall that $q(1) = 1$, we have
\begin{align}
	q(u) 
  = u
  , \quad u \in [0,1].
\end{align}
\end{proof}
% * Bibliography
\bibliographystyle{amsplain}
\bibliography{../bib/bib.bib}
\end{document}
