% LiuRenSongSun2019Yaglom
% **Settings**
% **Pages Settings**
\documentclass[12pt,a4paper]{amsart}
\setlength{\textwidth}{\paperwidth}
\addtolength{\textwidth}{-2in}
\calclayout
% **Equations and Theorems Settings**
\numberwithin{equation}{section}
\allowdisplaybreaks
\theoremstyle{plain}
\newtheorem{thm}{Theorem}[section]
\newtheorem{lem}[thm]{Lemma}
\newtheorem{prop}[thm]{Proposition}
\newtheorem{cor}[thm]{Corollaray}
\newtheorem{fact}[thm]{Fact}
\theoremstyle{definition}
\newtheorem*{asp*}{Assumption}
\newtheorem*{ack*}{Acknowledgment}
\theoremstyle{remark}
\newtheorem{exa}[thm]{Example}
\newtheorem{claim}[thm]{Claim}
% **Other Settings**
% \usepackage[utf8]{inputenc}
% \usepackage[T1]{fontenc}
\usepackage{amssymb}
\usepackage{mathtools}
\mathtoolsset{showonlyrefs}
\usepackage{mathrsfs}
\usepackage{comment}
\usepackage{enumitem}
\everymath{\displaystyle}
\usepackage{hyperref}
\begin{document}
% **Document Information**
\title {Subcritical Superprocesses }
\author[R. Liu, Y.-X. Ren, R. Song and Z. Sun]{Rongli Liu, Yan-Xia Ren, Renming Song and Zhenyao Sun}
\address{Yan-Xia Ren\\ School of Mathematical Sciences\\ Peking University\\ Beijing, P. R. China, 100871}
\email{yxren@math.pku.edu.cn}
\thanks{The research of Yan-Xia Ren is supported in part by NSFC (Grant Nos. 11671017 and 11731009).}
\address{Rongli Liu\\ \textbf{Information about Rongli Liu}}
\email{rlliu@bjtu.edu.cn}
\thanks{The research of Rongli Liu is supported in part by NSFC (Grant No. 11301261), and the Fundamental Research Funds for the Central Universities (Grant No.  2017RC007)}
\address{Renming Song\\ Information about Renming Song}
\email{\textbf{email of RS}}
\address{Zhenyao Sun\\ School of Mathematical Sciences\\ Peking University\\ Beijing, P. R. China, 100871}
\email{zhenyao.sun@pku.edu.cn}
% **Abstract**
\begin{abstract}
TBD
\end{abstract}
\maketitle
% **Introduction**
\section{Introduction}
% **Background**
\subsection{Background} 
% **Settings**
\par
Let us give the definition of the superprocesses.
Let $E$ be a Lusin topological space. 
Let $E_\partial = E\cup\{\partial\} $ be another Lusin topological space by adding an extra isolatied point $\partial$ to $E$. 
Let $\xi:= \{(\xi)_{0\leq t < \zeta}; (\Pi_x)_{x\in E}\}$ be an $E$-valued (sub)Markov process with (sub)Markov transition kernels $(P_t)_{t\geq 0}$ and lifetime $\zeta$. 
Define an $E_\partial$-valued stochastic process $\tilde \xi = \{(\tilde \xi)_{t\geq 0} ; (\Pi_{x})_{x\in {E_\partial}}\} $ by setting that $\tilde \xi_t = \xi_t\cdot \mathbf 1_{0\leq t< \zeta} + \partial \cdot \mathbf 1_{t\geq \zeta}, t\geq 0$ and $\Pi_\partial (\zeta = 0) = 1$. 
It can be verified that $\tilde \xi$ is a $E_\partial$-valued Markov process. 
We say the (sub)Markov process $\xi$ is a Hunt process if $\tilde \xi$ is a Hunt process. 
From now on we always assume that $\xi$ is a Hunt process. 
Let $\psi$ be a function on $E \times [0,\infty)$ given by 
\begin{align} 
\psi(x,z) 
= \beta(x) z + \sigma(x)^2 z^2 + \int_0^\infty (e^{-zu} -1 + zu) \pi(x,du), 
\quad x\in E, z\geq 0 
\end{align} 
where $\beta, \sigma \in \mathcal B_b(E,\mathbb R)$ and that $(u \wedge u^2) \pi(x,du)$ is a bounded kernel from $E$ to $(0,\infty)$. 
Let $\mathcal M_f(E)$ denote the space of finite Borel measures on $E$ equipped with topology of weak convergence. 
The following result is well know, see \cite{Li2011MeasureValued} for example.
% **#{Fact:S!} ** 
\begin{fact} \label{Fact:S!} 
For each $f \in \mathcal B_b(E, [0,\infty))$, there is a unique locally bounded positive solution $(t,x)\mapsto V_tf(x)$ to equation
\begin{align} 
V_tf(x) + \int_0^t P_{s}V_{t-s}f(x)ds 
= P_tf(x), \quad t\geq 0, x\in E. 
\end{align}
Moreover, there exists an $\mathcal M_f(E)$-valued Hunt Process $X =\{(X_t)_{t\geq 0}; (\mathbb P_\mu)_{\mu \in \mathcal M_f(E)}\}$ satisfying that  
\begin{align} 
\mathbb P_\mu[e^{- X_t(f)}]  
= e^{- \mu(V_tf)}, \quad \mu \in \mathcal M_f(E), 
f \in \mathcal B_b(E,[0,\infty)). 
\end{align}
\end{fact}
% **Settings**
In this paper, we always denote by $X$ the process given by Fact \ref{Fact:S!} which is known as the $(\xi, \psi)$-superprocess.
% **Main Result**
\subsection{Main Result}
Let us introduce some notations and facts in order to give the precise formulation of the assumptions used in this paper.
% **Notations**
Let $(P_t^\beta)_{t\geq 0}$ be a semigroup of operators on $\mathcal B_b(E,\mathbb R)$ given by 
\begin{align} 
P_t^\beta f(x)
:= \Pi_x[e^{\int_0^t \beta(\xi_r)dr }f(\xi_t) \mathbf 1_{t < \zeta}], 
\quad f\in \mathcal B_b(E,\mathbb R), t\geq 0, x\in E.
\end{align}
The following fact shows how the mean behavior of the superprocess $X$ can be captured by this semigroup. 
It can be verified from \cite[Proposition 2.27]{Li2011MeasureValued} for example.
% **#{Fact:M!} ( (2.34), (2.36) and Proposition 2.27 of{Li2011MeasureValued} )** 
\begin{fact} \label{Fact:M!} 
It holds that
\begin{align}
\mathbb P_\mu[X_t(f)] 
= \mu (P_t^\beta f), 
\quad \mu \in \mathcal M_f(E), t\geq 0, f \in \mathcal B_b(E,\mathbb R).
\end{align}
\end{fact}
% **Settings**
\par
Let us now give the assumptions used in this paper. 
First, we always assume that there exists a number $\lambda \in \mathbb R$, a function $\phi \in \mathcal B_b(E,(0,\infty))$ and a probability measure $\nu$ with full support on $E$ such that for each $t\geq 0$, $P_t^\beta \phi = e^{\lambda t}\phi$, $\nu P_t^\beta = e^{\lambda t} \nu$ and $\nu(\phi) = 1$.
We also assume the followings.
% **Assumptions**
\begin{asp*}
\begin{enumerate}[label =(H\arabic*)]
%{Assumption:H1!} 
\item \label{Assumption:H1!}
$\lambda < 0$.
%{Assumption:H2!}
\item \label{Assumption:H2!} 
$
P_t^\beta f(x) 
= e^{\lambda t} \phi(x) \nu(f) \chi( \lim_{t\to \infty} \sup_{x\in E, f\in L_1^+(\nu)}).
$
%{Assumption:H3!} 
\item \label{Assumption:H3!}
$
P_t^\beta f(x) 
= \phi(x) \nu(f) O(\sup_{x\in E, f \in L_1^+(\nu)}),
\quad t>0.
$
%{Assumption:H4!} 
\item \label{Assumption:H4!}
$\mathbb P_\nu(\|X_t\| = 0)>0$ for large enough $t>0$.
\end{enumerate}
\end{asp*}
A simple consequence is the following.
% **#{Lemma:Nd!} **
\begin{lem} \label{Lemma:Nd!} 
It holds that $\mathbb P_\mu(\|X_t\| > 0) > 0$ provided $\mu \in \mathcal M_f(E)\setminus \{0\}$.
\end{lem}
% **Proof of{Lemma:Nd!} **
\begin{proof}
Simply note from Fact \ref{Fact:M!} that $X_t(\phi)$ has positive mean:
\begin{align}
\mathbb P_\mu(X_t(\phi)) 
= \mu(P_t^\beta \phi) 
=e^{\lambda t}\mu(\phi)>0.
\end{align}
This implies the desired result.
\end{proof}
Thanks to this lemma, we can now talk about the superprocess $X$ conditioned on survival up to a certain time $t$. 
The main result of this paper is the following.
% **#{Theorem:Y:H1:H2:H3:H4} **
\begin{thm} \label{Theorem:Y:H1:H2:H3:H4} 
The Yaglom limit of $X$ exists, i.e. there exists a probability distribution $\mathbf P$ on $\mathcal M_f(E)$ such that 
\begin{align}
 \mathbb P_\mu (X_t \in \cdot | \|X_t\|> 0 ) 
 \xrightarrow[t\to \infty]{d} \mathbf P(\cdot), 
 \quad \mu \in \mathcal M_f(E)\setminus \{0\}.
 \end{align}
\end {thm}
% **Outline of the proofs**
\subsection{Outline of the proofs}
% __Outline of the proof of Theorem{Theorem:Y:H1:H2:H3:H4} __
\subsubsection{Outline of the proof of Theorem \ref{Theorem:Y:H1:H2:H3:H4} }
\label{subsubsec:OY}
Using monotonicity, one can easily verify the following two facts.  
% **#{Fact:BV!} **
\begin{fact} \label{Fact:BV!} 
There exists a unique family of operators $(\overline V_t)_{t \geq 0}$ from $\mathcal B(E, [0,\infty])$ to $\mathcal B(E, [0,\infty])$ satisfying the followings: 
\begin{itemize} 
\item
$\overline V_tf = V_tf$ for each $t\geq 0$ and $f \in \mathcal B_b(E, [0,\infty))$; 
\item
$\overline V_tf_n \uparrow \overline V_t f$ pointwisely provided $t\geq 0$ and that $f_n \uparrow f$ pointwisely in $\mathcal B(E, [0,\infty])$. 
\end{itemize}
Moreover, $(\overline V_t)_{t\geq 0}$ satisfies that 
\begin{itemize}
\item
$\overline V_t f \leq \overline V_t g$ for each $t\geq 0$ and each $f\leq g$ in $\mathcal B(E,[0,\infty])$; and
\item 
$\overline V_{t+s} = \overline V_t \overline V_s$ for each $t, s\geq 0$. 
\end{itemize}
\end{fact}
With some abuse of the notations, we still write $V_t = \overline V_t$ for each $t\geq 0$, and call $(V_t)_{t\geq 0}$ the extended cumulant semigroup of the superprocess $X$.
% **#{Fact:sv1!} ** 
\begin{fact} \label{Fact:sv1!} 
Define $v_t = V_t(\infty\mathbf 1_E)$ for each $t\geq 0$, then it holds that 
\begin{align}
\mathbb P_\mu (\|X_t\| = 0) 
= e^{- \mu (v_t)}, 
\quad \mu \in \mathcal M_f(E), t\geq 0.
\end{align}
\end{fact}
The following Lemma will also be used later.
% **#{Lemma:sv2!}**
\begin{lem} \label{Lemma:sv2!} 
$\mu(v_t) > 0$ for each $\mu \in \mathcal M_f(E)\setminus\{0\}$ and $t \geq 0$.
\end{lem}
% **Proof of{Lemma:sv2!}**
\begin{proof} 
If $\nu(v_t) = 0$, then by Fact \ref{Fact:sv1!} we have $P_\mu(\|X_t \| = 0) = 1$.
This conflicts with Lemma \ref{Lemma:Nd!}.
\end{proof}
Therefore, in order to give the outline of the proof of Theorem \ref{Theorem:Y:H1:H2:H3:H4}, let us first admit the following four propositions.
% **#{Proposition:Vf1:H1:H2:H4} ** 
\begin{prop} \label{Proposition:Vf1:H1:H2:H4} 
Let $f\in \mathcal B(E, [0,\infty])$, then
\begin{align} 
 V_tf(x) = \phi(x) o(\lim_{t\to \infty} \sup_{x\in E}).
 \end{align}
In particular, we have $\mu(V_tf)= o(\lim_{t\to \infty})$ for each $\mu \in \mathcal M_f(E)$.
\end{prop} 
% **#{Proposition:Vf2:H1:H2:H3:H4}**
\begin{prop} \label{Proposition:Vf2:H1:H2:H3:H4} 
Let $f\in \mathcal B(E,[0,\infty])$, then 
\begin{align}
V_tf(x)
=\phi(x) \nu (V_tf) \chi(\lim_{t\to \infty} \sup_{x\in E}).
\end{align}
\end{prop}
% **Notation**
\par
Define a family of $[0,\infty]$-valued functionals $(\Gamma_t)_{t\geq 0}$ on $\mathcal B(E,[0,\infty])$ by 
\begin{align}
 e^{-\Gamma_t f} 
:= \mathbf P_{\nu}[e^{- X_t(f)}| \|X_t\| > 0], 
 \quad f\in \mathcal B(E,[0,\infty]), t \geq 0.
 \end{align}
We say a $[0,\infty]$-valued functional $A$ defined on $\mathcal B(E,[0,\infty])$ is monotone concave if
\begin{itemize}
\item
$A$ is a monotone functional, i.e. $f\leq g$ in $\mathcal B(E,[0,\infty])$ implies $Af \leq Ag$; and
\item
for each $f\in \mathcal B(E,[0,\infty])$ with $Af< \infty$, function $u \mapsto A(uf)$ is concave on $[0,1]$.
\end{itemize}
% **#{Proposition:G:H1:H2:H3:H4} **
\begin{prop} \label{Proposition:G:H1:H2:H3:H4} 
The limits $Gf:= \lim_{t\to \infty} \Gamma_t f$ exists in $[0,\infty]$ for each $f\in \mathcal B(E,[0,\infty])$. 
Moreover, $G$ is the unique $[0,\infty]$-valued monotone concave functional on $\mathcal B(E,[0,\infty])$ s.t. $G(\infty \mathbf 1_E) = \infty$; and
\begin{align} 
1 - e^{- GV_s f} 
= e^{s\lambda} (1 - e^{-Gf}), 
\quad s\geq 0, f\in \mathcal B(E,[0,\infty]).
\end{align}
\end{prop} 
% **#{Proposition:GD:H1:H2:H3:H4} **
\begin{prop} \label{Proposition:GD:H1:H2:H3:H4} 
For each $(g_n)_{n\in \mathbb N} \subset \mathcal B(E,[0,\infty])$ such that $\lim_{n\to \infty} g_n = 0$ bounded pointwisely, we have $\lim_{n\to \infty} G g_n = 0$.
\end{prop}
% **Proof of{Theorem:Y:H1:H2:H3:H4} by admitting{Proposition:Vf1:H1:H2:H4},{Proposition:Vf2:H1:H2:H3:H4}{Proposition:G:H1:H2:H3:H4} and{Proposition:GD:H1:H2:H3:H4}** 
\begin{proof}[ Proof of Theorem \ref{Theorem:Y:H1:H2:H3:H4} by admitting Proposition \ref{Proposition:Vf1:H1:H2:H4}, \ref{Proposition:Vf2:H1:H2:H3:H4}, \ref{Proposition:G:H1:H2:H3:H4} and \ref{Proposition:GD:H1:H2:H3:H4}]
From \cite[Proposition 1.19]{Li2011MeasureValued}, Proposition \ref{Proposition:G:H1:H2:H3:H4} and \ref{Proposition:GD:H1:H2:H3:H4}  we have that there exists a unique probability measure $\mathbf P$ on $\mathcal M_f(E)$ such that 
\begin{align}
 e^{-Gf} 
 = \int_{\mathcal M_f(E)} e^{- \mu(f)} \mathbf P(d\mu), 
 \quad f\in C_b (E, [0,\infty)),
\end{align}
and that
\begin{align}
 \mathbb P_{\nu}(X_t \in \cdot | \|X_t\|>0 ) 
 \xrightarrow[t\to \infty]{d} \mathbf P(\cdot).
 \end{align}
For each $\mu \in \mathcal M_f(E)\setminus \{0\}$ and $f \in \mathcal B(E,[0,\infty])$, we have by Proposition \ref{Proposition:Vf2:H1:H2:H3:H4} and \ref{Proposition:Vf1:H1:H2:H4} that
\begin{align}
\mu(V_tf)
 = \int_E \nu(V_tf) \phi(x) \chi(\lim_{t\to \infty}\sup_{x\in E}) \mu(dx) 
 = \nu(V_tf) \mu(\phi)\chi(\lim_{t\to \infty}).
\end{align}  
Therefore, for each $\mu \in \mathcal M_f(E)\setminus\{0\}$ and $f\in \mathcal B(E,[0,\infty])$, by Fact \ref{Fact:sv1!}, Lemma \ref{Lemma:sv2!}, and Proposition \ref{Proposition:Vf1:H1:H2:H4}, we have that
\begin{align}
 &\mathbb P_\mu [1 - e^{-X_t(f)}|\|X_t\|>0]
 = \frac{\mathbb P_\mu [ 1 - e^{- X_t(f)}]} {\mathbb P_\mu (\|X_t\| > 0) }
 = \frac{1 - e^{- \mu(V_tf)}} { \mathbb P_\mu(\|X_t\| > 0)} 
 \\&= \frac{1 - e^{- \mu(V_tf)}} {1 - e^{-\mu(v_t)}}
 = \frac{ \mu(V_t f) }{ \mu(v_t) }  \chi(\lim_{t\to \infty})
 = \frac{ \nu(V_tf) }{ \nu(v_t) } \chi(\lim_{t\to \infty}).
 \end{align}
Since above $\mu$ is arbitrary, using Proposition \ref{Proposition:G:H1:H2:H3:H4}, we can derive that for each $\mu \in \mathcal M_f(E)\setminus\{0\}$ and $f\in \mathcal B(E,[0,\infty])$,
\begin{align}
  &\mathbb P_\mu [1 - e^{-X_t(f)}|\|X_t\|>0]
 = \mathbb P_\nu [1 - e^{-X_t(f)}|\|X_t\|>0] \chi(\lim_{t\to \infty})
 \\&= (1 - e^{- \Gamma_t f}) \chi(\lim_{t\to \infty})
 = (1 - e^{-Gf}) \chi(\lim_{t\to \infty}).
 \end{align}
Now by \cite[Proposition 1.19]{Li2011MeasureValued}, we have for each $\mu \in \mathcal M_f(E)\setminus\{0\}$,
\begin{align}
 \mathbb P_\mu(X_t \in \cdot | \|X_t\|>0) 
 \xrightarrow[t\to \infty]{d} \mathbf P(\cdot)
 \end{align}
as desired.
\end{proof}
% **PROOF OF{Theorem:Y:H1:H2:H3:H4} **
\section{Proof of Theorem \ref{Theorem:Y:H1:H2:H3:H4}}
Thanks to Subsubsection \ref{subsubsec:OY}, in order to proof Theorem \ref{Theorem:Y:H1:H2:H3:H4}, we only have to verify Proposition \ref{Proposition:Vf1:H1:H2:H4}, \ref{Proposition:Vf2:H1:H2:H3:H4}, \ref{Proposition:G:H1:H2:H3:H4} and \ref{Proposition:GD:H1:H2:H3:H4}. 
% **Some basic results**
\subsection{Some basic results} 
Let us first give some basic results that will be used through out the paper.
The following fact can be verified directly from \cite[Theorem 2.23]{Li2011MeasureValued} and monotonicity.
% **#{Fact:P!}** 
\begin{fact} \label{Fact:P!}
Define operator $\Psi_0: \mathcal B(E, [0,\infty]) \to \mathcal B(E,[0,\infty])$ such that 
\begin{align}
 \Psi_0 f(x) 
 = \psi(x,f(x))+\beta(x)f(x), 
 \quad f\in \mathcal B_b(E,[0,\infty)), x\in E,
 \end{align}
and 
\begin{align}
 \Psi_0 f 
 = \lim_{n\to \infty} \Psi_0 (f\wedge n), \quad f\in \mathcal B(E,[0,\infty]).
 \end{align}
Then it holds that
\begin{align}
 V_s f + \int_0^s P_{s-u}^\beta \Psi_0 V_{u} f ~du
 = P_s^\beta f, 
 \quad f\in \mathcal B(E,[0,\infty]), s\geq 0.
 \end{align}
\end{fact}
% **#{Lemma:nV:H4} **
\begin{lem} \label{Lemma:nV:H4} 
As long as $t>0$ is large enough, then $\nu(V_tf) < \infty$ for each $f\in \mathcal B(E, [0,\infty])$.
\end{lem}
% **Proof of{Lemma:nV:H4} ** 
\begin{proof}
It can be verified from Fact \ref{Fact:BV!}, \ref{Fact:sv1!} and Assumption \ref{Assumption:H4!} that as long as $t>0$ is large enough, then
\begin{align}
 \nu(V_t f) \leq \nu(v_t)  
 = - \log \mathbb P_\nu (\|X_t\| = 0) 
 < \infty
 \end{align}
for each $f\in \mathcal B(E,[0,\infty])$.
\end{proof}
% **Proof of{Proposition:Vf1:H1:H2:H4} **
\subsection{Proof of Proposition \ref{Proposition:Vf1:H1:H2:H4}}
By Lemma \ref{Lemma:nV:H4}, there exists $t_0>0$ such that $V_tf \in L_1^+(\nu)$ for $t\geq t_0$. 
Fix this $t_0>0$, we can verify from Fact \ref{Fact:BV!}, \ref{Fact:P!}, Assumption \ref{Assumption:H2!} and \ref{Assumption:H1!} that
\begin{align}
 &V_{t+t_0}f (x) 
 = V_tV_{t_0}f(x)
 \leq P_t^\beta V_{t_0} f(x)
 \\&= e^{\lambda t}\phi(x) \nu( V_{t_0} f) \chi(\lim_{t\to \infty} \sup_{x\in E})
 = \phi(x) o(\lim_{t\to \infty}\sup_{x\in E}).
 \end{align}
The proof of Proposition \ref{Proposition:Vf1:H1:H2:H4} is completed.
% **Some intermediate results**
\subsection{Some intermediate results}
Let us now give some more crucial intermediate results that will be used later.
% **#{Lemma:nuP!}**
\begin{lem} \label{Lemma:nuP!} 
For each $t, s\geq 0$ and $f\in \mathcal B(E,[0,\infty])$,
\begin{align}
  e^{- \lambda (t+s)} \nu(V_{t+s}f) + \int_0^s e^{- \lambda (t+u)} \nu(\Psi_0 V_{t+u}f) du 
  = e^{- \lambda t} \nu(V_t f).
  \end{align}
\end{lem}
% **Proof of{Lemma:nuP!}**
\begin{proof}
Integrating the both sides of the equation in Fact \ref{Fact:P!} with respect to $\nu$, replacing $f$ with $V_t f$, we get the desired result.
\end{proof}
% **#{Lemma:nVn!} **
\begin{lem} \label{Lemma:nVn!} 
For each $t\geq 0$ and $f\in \mathcal B(E,[0,\infty])$, the followings hold:
\begin{itemize}
\item
$\nu(V_tf)=0 $ provided $\nu(f) = 0$; 
\item
$\nu(V_tf)>0$ provided $\nu(f)>0$.
\end{itemize}
\end{lem}
% **Proof of{Lemma:nVn!} **
\begin{proof}
Let $t\geq 0$ and $f\in \mathcal B(E,[0,\infty])$. 
Note by Fact \ref{Fact:M!}, 
\begin{align}
 \mathbb P_\nu[X_t(f)] 
 = \nu (P_t^\beta f)
 = e^{\lambda t}\nu (f).
 \end{align}
If $\nu(f) = 0$, then $\{X_t(f); \mathbb P_\nu\}$ is a.s. $0$, therefore 
\begin{align}
 \nu(V_t f) 
 = - \log \mathbb P_\nu[e^{-X_t(f)}]
 =0.
 \end{align}
If $\nu(f) > 0$, then $\{X_t(f); \mathbb P_\nu\}$ is a random variable with positive mean since
\begin{align}
\mathbb P_\nu[X_t(f)] = \nu(P_t^\beta f) = e^{\lambda t} \nu(f)>0.
\end{align}
Therefore, 
\begin{align}
 \nu(V_tf)
 = - \log \mathbb P_\nu[e^{-X_t(f)}]
 >0,
 \end{align}
as desired.
\end{proof}
% **#{Lemma:VfO:H4:H3} ** 
\begin{lem} \label{Lemma:VfO:H4:H3} 
For each $f\in \mathcal B(E,[0,\infty])$ it holds that 
\begin{align}
V_tf(x) 
= \phi(x) \nu(V_tf) O(\limsup_{t\to \infty} \sup_{x\in E}).
\end{align}
\end{lem}
% **Proof of{Lemma:VfO:H4:H3}**
\begin{proof}
Fix an $f\in \mathcal B(E,[0,\infty])$. 
According to Lemma \ref{Lemma:nV:H4} there exists a $t_0 > 0$ such that $\{V_tf:t\geq t_0\}\subset L_1^+(\nu)$. 
Therefore, it can be verified from Fact \ref{Fact:BV!}, \ref{Fact:P!} and Assumption \ref{Assumption:H3!} that
\begin{align}
 V_{t+1}f(x) 
  = V_1 V_t f
  \leq P_1^\beta V_t f(x)
  = \phi(x) \nu(V_tf) O(\sup_{t \geq t_0, x\in E}),
  \end{align}
as desired.
\end{proof}
% **Some Notations**
Define an operator $\Psi_0'$ on $\mathcal B(E,[0,\infty])$ such that
\begin{align}
 \Psi_0' f(x) 
 = \frac{\partial \psi_0}{ \partial z} (x, f(x)),
 \quad x\in E, f\in \mathcal B(E,[0,\infty))
 \end{align}
and that
\begin{align}
 \Psi_0' f 
 = \lim_{n\to \infty} \Psi_0'(f\wedge n),
 \quad x\in E, f\in \mathcal B(E,[0,\infty]).
 \end{align}
% **#{Lemma:PsV:H1:H2:H4}**
\begin{lem} \label{Lemma:PsV:H1:H2:H4} 
$\Psi_0'V_tf(x) = O(\limsup_{t\to \infty} \sup_{x\in E})$ for each $f\in \mathcal B(E, [0,\infty])$.
\end{lem}
% **Proof of{Lemma:PsV:H1:H2:H4}**
\begin{proof}
Note that 
\begin{align}
 \frac{\partial \psi_0 }{ \partial z} (x,z)
 = 2\alpha (x)^2 z + \int_0^\infty (1 - e^{- rz}) r \pi(x,dr), 
 \quad x\in E, z\geq 0.
  \end{align}
Therefore,
\begin{align}
  &\Psi_0' V_tf(x) 
  \leq 2\alpha (x)^2 V_t f(x) + V_t f(x) \int_0^1 r^2 \pi(x,dr) + 2\int_1^\infty r \pi(x,dr)
  \\&\leq O(\sup_{x\in E, t\geq 0})V_tf(x) + O(\sup_{x\in E, t\geq 0})
  = O(\limsup_{t\to \infty} \sup_{x\in E} ).
  \end{align}
In the last step, we used Proposition \ref{Proposition:Vf1:H1:H2:H4} and the assumption that $\phi$ is bounded.
\end{proof}
% **#{Lemma:nPPV:H1:H2:H4}**
\begin{lem} \label{Lemma:nPPV:H1:H2:H4} 
 $\nu(\Psi_0' V_t f) = o(\lim_{t\to \infty})$ for each $f\in \mathcal B(E,[0,\infty])$.
\end{lem}
% **Proof of{Lemma:nPPV:H1:H2:H4} **
\begin{proof}
Note that 
\begin{align}
 \frac{\partial \psi_0 }{ \partial z} (x,z)
 = 2\alpha (x)^2 z + \int_0^\infty (1 - e^{- rz}) r \pi(x,dr), 
 \quad x\in E, z\geq 0.
  \end{align}
It is elementary to see that for any fixed $x\in E$, $z\mapsto \frac{\partial \psi_0}{\partial z} (x,z)$ is non-negative, non-decreasing and continuous function on $[0,\infty)$. 
Fix an $f\in \mathcal B(E,[0,\infty])$. 
Note from Proposition \ref{Proposition:Vf1:H1:H2:H4} that 
\begin{align}
  V_tf(x) 
  = o(\lim_{t\to \infty}),
\quad x\in E.
  \end{align}
Therefore 
\begin{align}
 \Psi_0' V_tf(x) 
 =\frac{\partial \psi_0}{ \partial z}(x,V_tf(x)) 
 = o(\lim_{t\to \infty}),
\quad x\in E.
 \end{align}
On the other hand, we have by Lemma \ref{Lemma:PsV:H1:H2:H4} that
\begin{align}
  \Psi_0'V_t f(x) 
  = O(\limsup_{t\to \infty}\sup_{x\in E}).
  \end{align} 
Therefore, use bounded convergence theorem we have that 
\begin{align}
  \nu(\Psi_0' V_tf) 
  = o(\lim_{t\to \infty}),
  \end{align}
as desired.
\end{proof}
% **#{Proposition:nP:H1:H2:H3:H4}**
\begin{prop} \label{Proposition:nP:H1:H2:H3:H4} 
$\nu(\Psi_0 V_t f) = \nu(V_tf) o(\lim_{t\to \infty})$ for each $f\in \mathcal B(E,[0,\infty])$.
\end{prop}
% **Proof of{Proposition:nP:H1:H2:H3:H4} **
\begin{proof}[{Proof of Proposition \ref{Proposition:nP:H1:H2:H3:H4}}]
It is elementary analysis to see that 
\begin{align}
  \psi_0(x,z) 
  \leq z \frac{\partial \psi_0}{\partial z}(x,z),
  \quad x\in E, z\geq 0.
  \end{align}
Therefore for each $f\in \mathcal B(E,[0,\infty])$ we have by Lemma \ref{Lemma:nPPV:H1:H2:H4} and \ref{Lemma:VfO:H4:H3} that 
\begin{align}
  &\nu(\Psi_0 V_tf) 
\leq \nu((V_tf)\cdot (\Psi_0' V_tf))
\\&  \leq \nu(\Psi_0' V_tf) \sup_{x\in E} V_tf(x)
  = o(\lim_{t\to \infty}) \sup_{x\in E} V_tf(x)
  \\&= o(\lim_{t\to \infty}) O(\limsup_{t\to \infty}) \nu(V_tf)
  = \nu(V_tf) o(\lim_{t\to \infty}),
  \end{align}
as desired.
\end{proof}
% **#{Lemma:nVI:H4} **
\begin{lem} \label{Lemma:nVI:H4} 
As long as $t> 0$ is large enough then
\begin{align}
  \frac{\nu(V_{t+s} f)} {\nu(V_t f)} 
  = \exp\Big\{ \lambda s - \int_t^{t+s} \frac{\nu(\Psi_0 V_u f) }{\nu(V_u f)} du\Big\}
  \end{align}
for each $s\geq 0$, $f \in \mathcal B(E,[0,\infty])$ and $\nu(f)>0$.
\end{lem} 
% **Proof of{Lemma:nVI:H4} **
\begin{proof}
Acording to Lemma \ref{Lemma:nuP!}, for each $t, s\geq 0$ and $f\in \mathcal B(E,[0,\infty])$
\begin{align}
  e^{- \lambda (t+s)}\nu(V_{t+s}f) + \int_0^s e^{- \lambda (t+u)} \nu(\Psi_0 V_{t+u}f)du 
  = e^{- \lambda t} \nu(V_tf).
  \end{align}
According to Lemma \ref{Lemma:nV:H4}, there exists $t_0> 0$ such that the both sides of the above equation is finite for each $t\geq t_0$ and $f\in \mathcal B(E,[0,\infty])$. 
For the rest of the proof, we fix an $f \in \mathcal B(E,[0,\infty])$ satisfing $\nu(f)>0$. 
Then $H: t\mapsto e^{-\lambda t}\nu(V_tf)$ is absolutely continuous on $[t_0,\infty)$ and that
\begin{align}
  d H(t) 
  = - e^{- \lambda t} \nu(\Psi_0 V_t f) dt,
  \quad t\in [t_0,\infty).
  \end{align}
We can derive from this and Lemma \ref{Lemma:nVn!} that 
\begin{align}
  d \log H(t) 
  = - \frac{\nu(\Psi_0 V_tf )}{ \nu(V_tf)} dt,
  \quad t \in [t_0,\infty).
  \end{align}
Therefore, for each $t\geq t_0$ and $s\geq 0$ we have
\begin{align}
  \frac{\nu(V_t)}{ \nu(V_{t+s}f)}
  = e^{- \lambda s} \frac{H(t)}{H(t+s)} 
  = \exp\Big\{-\lambda s + \int_t^{t+s} \frac{\nu(\Psi_0 V_u f)}{ \nu(V_u f)} du\Big\},
  \end{align}
as desired.
\end{proof}
% **#{Proposition:nVR:H1:H2:H3:H4} **
\begin{prop} \label{Proposition:nVR:H1:H2:H3:H4} 
For each $f\in \mathcal B(E,[0,\infty])$, we have
\begin{align}
\nu(V_{t+s}f) = \nu(V_tf) e^{\lambda s \chi(\lim_{t\to \infty} \sup_{s\geq 0})}.
\end{align}
In particular, for each $f\in \mathcal B(E,[0,\infty])$ and $s\geq 0$, we have
\begin{align}
\nu(V_{t+s}f) 
= \nu(V_t f) e^{\lambda s} \chi(\lim_{t\to \infty}).
\end{align}
\end{prop}
% **Proof of{Proposition:nVR:H1:H2:H3:H4}**
\begin{proof}[{Proof of Proposition \ref{Proposition:nVR:H1:H2:H3:H4}}]
If $\nu(f) = 0$, then the desired result is trivial thanks to Lemma \ref{Lemma:nVn!}. 
Now, suppose that $\nu(f)>0$, then it can be verified from Lemma \ref{Lemma:nVI:H4} and Proposition \ref{Proposition:nP:H1:H2:H3:H4} that
\begin{align}
  &\frac{\nu(V_{t+s} f)} {\nu(V_t f)} 
  = \exp \Big\{ \lambda s - \int_t^{t+s} \frac{\nu(\Psi_0 V_u f)}{ \nu(V_u f)} du\Big\}
  \\&= \exp\Big\{\lambda s + \int_t^{t+s} o(\lim_{u\to \infty}\sup_{s\geq 0}) du\Big\}
  = \exp\Big\{\lambda s + so(\lim_{t\to \infty} \sup_{s\geq 0})\Big\},
  \end{align}
as desired.
\end{proof}
% **Proof of{Proposition:Vf2:H1:H2:H3:H4} **
\subsection{Proof of Proposition \ref{Proposition:Vf2:H1:H2:H3:H4}}
% __Outline of the Proof of{Proposition:Vf2:H1:H2:H3:H4} __
\subsubsection{Outline of the proof of Proposition \ref{Proposition:Vf2:H1:H2:H3:H4}}
We first present some Lemmas that will be used in the proof of Proposition \ref{Proposition:Vf2:H1:H2:H3:H4}.
% **#{Lemma:nullVf:H3:H4}**
\begin{lem} \label{Lemma:nullVf:H3:H4}
There exists $t_0>0$ such that $V_t f(x ) = 0$ for each $x\in E$, $t > t_0$ and $f \in \mathcal B(E,[0,\infty])$ with $\nu(f) = 0$.
\end{lem}
% **Proof of{Lemma:nullVf:H3:H4} ** 
\begin{proof}
According to Lemma \ref{Lemma:nV:H4}, there exists $t_0> 0$ such that $\{V_t f: t \geq t_0, f \in \mathcal B(E,[0,\infty])\} \subset L_1^+(\nu). $
Fix $t>t_0$ and $f \in \mathcal B(E,[0,\infty])$ with $\nu(f) = 0$. 
Then by Fact \ref{Fact:BV!}, \ref{Fact:P!}, Assumption \ref{Assumption:H3!} and Lemma \ref{Lemma:nVn!}, we have
\begin{align}V_tf(x) = V_{t-t_0} V_{t_0}f(x)
 \leq P_{t - t_0}^\beta V_{t_0} f(x)
 = \phi(x)\nu(V_{t_0} f) O(\sup_{x\in E, f\in \mathcal B(E,[0,\infty])})
 = 0,
\end{align}
as desired.
\end{proof}
% **#{Lemma:PV:H3:H4} **
\begin{lem} \label{Lemma:PV:H3:H4} 
There exists $t_0>0$ such that $P_s^\beta V_t f(x) < \infty$ for each $t \geq t_0, s> 0, x\in E$ and $f\in \mathcal B(E,[0,\infty])$.
\end{lem}
% **Proof of{Lemma:PV:H3:H4}** 
\begin{proof}
According to Lemma \ref{Lemma:nV:H4}, there exists $t_0> 0$ such that 
\begin{align}
 \{V_t f: t \geq t_0, f \in \mathcal B(E,[0,\infty])\} 
 \subset L_1^+(\nu).
 \end{align}
Therefore according to Assumption \ref{Assumption:H3!}, 
\begin{align}
 P_s^\beta V_tf(x)
 = \phi(x)\nu(V_tf) O(\sup_{x\in E, t\geq t_0, f\in \mathcal B(E,[0,\infty])}),
 \quad s>0.
 \end{align}
This implies the desired result in this Lemma.
\end{proof}
% **#{Lemma:PVf:H1:H2:H3:H4} ** 
\begin{lem} \label{Lemma:PVf:H1:H2:H3:H4} 
$P_s^\beta V_t f(x) = \phi(x) \nu(V_{t+s}f) \chi (\lim_{s\to \infty} \limsup_{t\to \infty} \sup_{x\in E})$ provided $f\in \mathcal B(E,[0,\infty])$.
\end{lem}
% **Proof of{Lemma:PVf:H1:H2:H3:H4} **
\begin{proof}
According to Lemma \ref{Lemma:nV:H4}, there exists $t_0 > 0$ such that $\{V_tf: t\geq t_0, f\in \mathcal B(E,[0,\infty])\} \subset L_1^+(\nu)$.
Therefore for each $f\in \mathcal B(E,[0,\infty])$ with $\nu(f)>0$, we have  by Assumption \ref{Assumption:H2!}, Lemma \ref{Lemma:nVn!} and by Proposition \ref{Proposition:nVR:H1:H2:H3:H4} that 
\begin{align}
&P_s^\beta V_t f(x) 
= e^{\lambda s} \phi(x) \nu(V_tf) \chi(\lim_{s\to \infty} \sup_{x\in E, t > t_0})
\\&=\phi(x) \nu(V_{t+s}f)  \frac{e^{\lambda s} \nu(V_tf)}{ \nu(V_{t+s}f)} \chi(\lim_{s\to \infty}\sup_{x\in E, t> t_0})
\\&= \phi(x) \nu(V_{t+s}f) \chi(\sup_{s\geq 0}\lim_{t\to \infty}\sup_{x\in E}) \chi(\lim_{s\to \infty} \sup_{x\in E, t> t_0}) 
\\&= \phi(x)\nu(V_{t+s}f)\chi(\lim_{s\to \infty} \limsup_{t\to \infty} \sup_{x\in E}).
\end{align}
And for each $f\in B(E,[0,\infty])$ with $\nu(f)= 0$, we have by Assumption \ref{Assumption:H2!} and Lemma \ref{Lemma:nVn!} that
\begin{align}
 P_s^\beta V_t f(x) 
 = e^{\lambda s} \phi(x) \nu(V_tf) \chi(\lim_{s\to \infty} \sup_{x\in E, t > t_0})
= 0
 = \phi(x)\nu(V_{t+s}f)\chi(\lim_{s\to \infty} \limsup_{t\to \infty} \sup_{x\in E}),
 \end{align} 
as desire.
\end{proof}
% **#{Lemma:IVf:H1:H2:H3:H4} **
\begin{lem} \label{Lemma:IVf:H1:H2:H3:H4} 
Define 
\begin{align} 
 I_{s,\epsilon} f 
 = \int_0^{s - \epsilon} P_{s - u}^\beta \Psi_0 V_u f ~du,
 \quad f\in \mathcal B(E,[0,\infty]), 0 < \epsilon < s < \infty,
 \end{align} 
then for each $f\in \mathcal B(E,[0,\infty])$ with $\nu(f)>0$, we have
\begin{align}
 I_{s,\epsilon}V_t f(x) 
 = \phi(x) \nu(V_{s+t} f) o(\lim_{t\to \infty} \sup_{x\in E}),
 \quad 0 < \epsilon < s< \infty.
 \end{align}
\end{lem}
% **Proof of{Lemma:IVf:H1:H2:H3:H4} **
\begin{proof}
Fix an $f\in \mathcal B(E,[0,\infty])$ with $\nu(f)>0$. 
Also fix $0<\epsilon < s< \infty$. 
According to Lemma \ref{Lemma:nV:H4} and Proposition \ref{Proposition:nP:H1:H2:H3:H4} we know that as long as $t>0$ is large enough, we have $\nu(\Psi_0 V_t f)< \infty$.
Thanks to this, we have that there exists $t_0>0$ such that for each $t\geq t_0$ and $u\geq 0$, we have
\begin{align}
\nu(P_u^\beta \Psi_0 V_t f) 
= e^{\lambda u}\nu(\Psi_0 V_t f)
< \infty.
\end{align} 
Now we can verify from Assumption \ref{Assumption:H3!} that
\begin{align}
 &I_{s,\epsilon} V_t f(x) 
 = \int_0^{s- \epsilon} P_{s-u}^\beta \Psi_0 V_{t+u} f (x)~du 
 = \int_0^{s- \epsilon} P_\epsilon^\beta (P_{s - \epsilon - u}^\beta \Psi_0 V_{t+u} f )(x) du 
 \\&= \int_0^{s - \epsilon} \phi(x) \nu(P_{s - \epsilon - u}^{\beta} \Psi_0 V_{t+u} f) O(\sup_{x\in E, t\geq t_0, u\in [0,s-\epsilon]}) ~du 
 \\&\leq \phi(x) O(\sup_{x\in E, t\geq t_0}) e^{(t+s-\epsilon)\lambda} \int_0^{s} e^{-\lambda (t+u)} \nu(\Psi_0 V_{t+u}f)~du
 \end{align}
Now by Lemma \ref{Lemma:nuP!}, \ref{Lemma:nVn!} and Proposition \ref{Proposition:nVR:H1:H2:H3:H4}, we have
\begin{align}
  &I_{s,\epsilon} V_t f(x) 
 \leq \phi(x) O(\sup_{x\in E, t\geq t_0}) e^{(t+s-\epsilon)\lambda} (e^{-\lambda t}\nu(V_tf)- e^{-\lambda(t+s)}\nu(V_{t+s}f)) 
 \\&= \phi(x) O(\sup_{x\in E, t\geq t_0})e^{-\epsilon \lambda} \nu(V_{t+s}f) \Big( \frac{e^{s \lambda }\nu(V_tf)}{\nu(V_{t+s}f)} - 1\Big)
 \\&= \phi(x) \nu(V_{t+s}f) O(\sup_{x\in E, t\geq t_0}) o(\lim_{t\to \infty}\sup_{x\in E})
 \\&= \phi(x)\nu(V_{t+s}f)o(\lim_{t\to \infty} \sup_{x\in E}),
 \end{align}
as desired.
\end{proof}
We also need the following proposition whose proof is postponed to Subsubsection \ref{subsubsec:PJVf}.
% **#{Proposition:JVf:H1:H2:H3:H4} **
\begin{prop} \label{Proposition:JVf:H1:H2:H3:H4} 
Define 
\begin{align}
 J_{s,\epsilon} f 
 = \int_{s-\epsilon}^s P_{s-u}^\beta \Psi_0 V_u f du,
 \quad f\in \mathcal B(E,[0,\infty]), 0< \epsilon < s< \infty.
 \end{align}
Then for each $f\in \mathcal B(E,[0,\infty])$ with $\nu(f)>0$,
\begin{align}
 J_{s,\epsilon} V_tf(x) 
 = \phi(x) \nu(V_{t+s}f) o(\lim_{\epsilon \to 0}\limsup_{t+s\to \infty} \sup_{x\in E}).
 \end{align}
\end{prop}
% **Proof of{Proposition:Vf2:H1:H2:H3:H4} by admitting{Proposition:JVf:H1:H2:H3:H4} **
\begin{proof}[{Proof of Proposition \ref{Proposition:Vf2:H1:H2:H3:H4} by admitting Proposition \ref{Proposition:JVf:H1:H2:H3:H4}}]
Suppose that $\nu(f) = 0$, then according to Lemma \ref{Lemma:nVn!} and \ref{Lemma:nullVf:H3:H4} we have for $t$ large enough, $\nu(V_tf) =0= V_tf(x), x\in E$. 
So, in this case, the desired result is true. 
For the rest of the proof, assume that $\nu(f)>0$. 
By Fact \ref{Fact:BV!}, \ref{Fact:P!} and Lemma \ref{Lemma:PV:H3:H4} there exists $t_0> 0$ such that for each $t\geq t_0, s> 0$, and $x\in E$, the following holds:
\begin{align}
 &V_{t+s}f (x) 
 = V_t V_s f(x)
 \\&= P_s^\beta V_t f(x) - \int_0^s P^\beta_{s-u}\Psi_0 V_uV_t f(x) du 
 \\&= P_s^\beta V_t f(x) - I_{s,\epsilon} V_tf(x) - J_{s,\epsilon} V_t f(x).
 \end{align}
Then, by Lemma \ref{Lemma:PVf:H1:H2:H3:H4}, \ref{Lemma:IVf:H1:H2:H3:H4} and Proposition \ref{Proposition:JVf:H1:H2:H3:H4}, we have that
\begin{align}
 &V_{t+s}f (x) 
 \\&= \phi(x)\nu(V_{t+s}f) \Big( \chi(\sup_{\epsilon > 0}\lim_{s\to \infty} \limsup_{t\to \infty} \sup_{x\in E}) - o(\sup_{0<\epsilon < s<\infty} \lim_{t\to \infty}\sup_{x\in E}) - o(\lim_{\epsilon \to 0} \limsup_{t+s\to \infty} \sup_{x\in E})\Big)
 \\&= \phi(x) \nu(V_{t+s}f) \chi(\lim_{\epsilon\to 0} \limsup_{s\to \infty} \limsup_{t\to \infty} \sup_{x\in E}).
 \end{align}
Therefore, we must have $V_{t}f (x) = \phi(x) \nu (V_{t}f) \chi(\lim_{t\to \infty} \sup_{x\in E})$.
\end{proof}
% __Proof of{Proposition:JVf:H1:H2:H3:H4} __
\subsubsection{Proof of Proposition \ref{Proposition:JVf:H1:H2:H3:H4}}
\label{subsubsec:PJVf}
% **#{Claim:PPV:H1:H2:H3:H4}**
\begin{claim} \label{Claim:PPV:H1:H2:H3:H4} 
For $f \in \mathcal B(E,[0,\infty])$ with $\nu(f)>0$, we have 
\begin{align}
 P_u^\beta \Psi_0 V_{t-u} f(x) 
 = \phi(x)\nu(V_tf) O(\sup_{\epsilon > 0} \limsup_{t\to \infty} \sup_{u \in (0,\epsilon), x\in E}).
 \end{align}
\end{claim}
% **Proof of{Proposition:JVf:H1:H2:H3:H4} by admitting{Claim:PPV:H1:H2:H3:H4}**
\begin{proof}[{Proof of Proposition \ref{Proposition:JVf:H1:H2:H3:H4} by admitting Claim \ref{Claim:PPV:H1:H2:H3:H4}}]
Fix an $f\in \mathcal B(E,[0,\infty])$ with $\nu(f)>0$.
Then it can be verified by Claim \ref{Claim:PPV:H1:H2:H3:H4} that
\begin{align}
 &J_{s,\epsilon}V_tf(x) 
 = \int_{s-\epsilon}^s P_{s-u}^\beta \Psi_0 V_{t+u} f(x)du 
 \\&= \int_0^\epsilon P_u^\beta \Psi_0 V_{t+s - u}f(x) du
 = \int_0^\epsilon \phi(x) \nu(V_{t+s}f) O(\sup_{\epsilon > 0} \limsup_{t+s\to \infty} \sup_{u \in (0,\epsilon), x\in E})~du
 \\&= \epsilon \phi(x)\nu(V_{t+s}f) O(\sup_{\epsilon > 0} \limsup_{t+s \to \infty} \sup_{x\in E})
 = \phi(x)\nu(V_{t+s}f)o(\lim_{\epsilon \to 0} \limsup_{t+s \to \infty} \sup_{x\in E}),
 \end{align}
as desired.
\end{proof}
Now, in order to proof Proposition \ref{Proposition:JVf:H1:H2:H3:H4}, we only have to proof Claim \ref{Claim:PPV:H1:H2:H3:H4}.
% **#{Fact:TO!}**
\begin{fact} \label{Fact:TO!} 
Suppose that $h(t,u) = O(\limsup_{t\to \infty} \sup_{u\geq 0})$, then 
\begin{align}
 h(t-u, u)
 = O(\sup_{\epsilon > 0} \limsup_{t\to \infty} \sup_{u \in (0,\epsilon)}).
 \end{align}
Suppose that $h(t,u) = o(\lim_{t\to \infty} \sup_{u \geq 0})$, then 
\begin{align}
 u\cdot h(t-u,u) 
 = o(\sup_{\epsilon > 0} \lim_{t\to \infty} \sup_{u \in (0,\epsilon)}).
 \end{align}
\end{fact}
% **#{Claim:PuPVt:H1:H2:H3:H4}**
\begin{claim} \label{Claim:PuPVt:H1:H2:H3:H4} 
For each $f\in \mathcal B(E,[0,\infty])$ with $\nu(f)>0$, we have 
\begin{align}
 P_u^\beta \Psi_0 V_{t} f(x) 
 = \phi(x) \nu(V_{t+u}f) e^{u\lambda o(\lim_{t\to \infty} \sup_{x\in E, u\geq 0})} O(\limsup_{t\to \infty}\sup_{x\in E, u\geq 0}).
 \end{align}
\end{claim}
% **Proof of{Claim:PPV:H1:H2:H3:H4} by admitting{Claim:PuPVt:H1:H2:H3:H4} **
\begin{proof}[{Proof of Claim \ref{Claim:PPV:H1:H2:H3:H4} by admitting Claim \ref{Claim:PuPVt:H1:H2:H3:H4}}]
Fix $f\in \mathcal B(E,[0,\infty])$ with $\nu(f)>0$. 
According to Claim \ref{Claim:PuPVt:H1:H2:H3:H4} and Fact \ref{Fact:TO!} we have that
\begin{align}
 & P_u^\beta \Psi_0 V_{t-u} f(x) 
 = \phi(x) \nu(V_{t}f) e^{\lambda o(\sup_{\epsilon > 0}\lim_{t\to \infty} \sup_{u\in (0,\epsilon), x\in E})} O(\sup_{\epsilon > 0} \limsup_{t\to \infty}\sup_{u\in (0,\epsilon), x\in E})
 \\&= \phi(x)\nu(V_tf) O(\sup_{\epsilon > 0} \limsup_{t\to \infty} \sup_{u \in (0,\epsilon), x\in E}),
 \end{align}
as desired.
\end{proof}
Now we only have to proof Claim \ref{Claim:PuPVt:H1:H2:H3:H4}.
% **#{Lemma:PVtV:H1:H2:H4}**
\begin{lem} \label{Lemma:PVtV:H1:H2:H4} 
For $f\in \mathcal B(E,[0,\infty])$,
\begin{align}
 \Psi_0 V_t f(x) 
 = V_tf(x) O(\limsup_{t\to \infty}\sup_{x\in E}).
 \end{align}
\end{lem}
% **Proof of{Lemma:PVtV:H1:H2:H4}**
\begin{proof}
It is elementary analysis to see that 
\begin{align}
 \psi_0(x,z) 
 \leq z \frac{\partial \psi_0}{\partial z} (x,z),
 \quad x\in E, z\geq 0.
 \end{align}
Therefore by Lemma \ref{Lemma:PsV:H1:H2:H4} we have
\begin{align}
 \Psi_0 V_t f (x) 
 \leq V_tf(x)\cdot \Psi_0' V_t f(x) 
 = V_tf(x) O(\limsup_{t\to \infty}\sup_{x\in E} ),
 \end{align}
as desired.
\end{proof}
% **Proof of{Claim:PuPVt:H1:H2:H3:H4}**
\begin{proof}[Proof of Claim \ref{Claim:PuPVt:H1:H2:H3:H4}]
Fix an $f\in \mathcal B(E,[0,\infty])$ with $\nu(f)>0$. 
From Lemma \ref{Lemma:nVn!} we know that $\nu(V_tf)>0$. 
Also note that $\phi$ strictly positive. 
Now by Lemma \ref{Lemma:PVtV:H1:H2:H4}  we have
\begin{align}
 &P_u^\beta \Psi_0 V_{t} f(x) 
 = \int_{E} \Psi_0V_tf(y) P_u^\beta (x,dy)
 \\&=\int_{E} V_tf(y) O(\limsup_{t\to \infty} \sup_{x,y\in E, u\geq 0}) P_u^\beta (x,dy)
 = O(\limsup_{t\to \infty}\sup_{x\in E,u\geq 0}) P_u^\beta V_t f(x)
 \\&= O(\limsup_{t\to \infty}\sup_{x\in E,u\geq 0}) (P_u^\beta \phi(x)) \cdot \nu(V_{t}f) \cdot\sup_{y\in E} \frac{V_tf(y)}{\nu(V_{t}f)\phi(y)}.
 \end{align}
From Lemma \ref{Lemma:VfO:H4:H3} and \ref{Proposition:nVR:H1:H2:H3:H4} we get
\begin{align}
 &P_u^\beta \Psi_0 V_{t} f(x)
 = e^{u\lambda} \phi(x) \nu(V_tf)O(\limsup_{t\to \infty} \sup_{x\in E, u \geq 0})
 \\&= e^{u\lambda }\phi(x)\nu(V_{t+u}f) e^{-u \lambda \chi(\lim_{t\to \infty} \sup_{x\in E, u\geq 0})} O(\limsup_{t\to \infty} \sup_{x\in E, u\geq 0})
 \\&= \phi(x) \nu(V_{t+u}f) e^{u\lambda o(\lim_{t\to \infty} \sup_{x\in E, u\geq 0})} O(\limsup_{t\to \infty}\sup_{x\in E, u\geq 0}),
 \end{align}
as desired.
\end{proof}
% **Proof of{Proposition:G:H1:H2:H3:H4}**
\subsection{Proof of Proposition \ref{Proposition:G:H1:H2:H3:H4}}
% __Outline of the Proof of{Proposition:G:H1:H2:H3:H4} __
\subsubsection{Outline of the proof of Proposition \ref{Proposition:G:H1:H2:H3:H4}}
Let us first give a quick lemma.
% **Notations**
For each unbounded increasing positive sequence $\mathbf t = (t_n)_{n\in \mathbb N}$, define $G^\mathbf t f = \liminf_{n\to \infty} \Gamma_{(t_n)} f$. 
% **#{Lemma:Gta!} **
\begin{lem} \label{Lemma:Gta!} 
For each unbounded increasing positive sequence $\mathbf t = (t_n)_{n\in \mathbb N}$, $G^\mathbf t$ is a $[0,\infty]$-valued monotone concave functional on $\mathcal B(E,[0,\infty])$ with $G^\mathbf t(\infty \mathbf 1_E) = \infty$.
\end{lem}
% **Proof of Lemma{Lemma:Gta!} ** 
\begin{proof}
Fix an unbounded increasing positive sequence $\mathbf t = (t_n)_{n\in \mathbb N}$, and observe that since $(\Gamma_t)_{t\geq 0}$ are $[0,\infty]$-valued, so is $G^{\mathbf t}$. 
Also, from $\Gamma_t(\infty \mathbf 1_E) = \infty$ for each $t\geq 0$ we have that $G^{\mathbf t}(\infty \mathbf 1_E) = \infty$. 
Let us now verify that $G^\mathbf t$ is monotone concave. 
In fact, for each $f \leq g$ in $\mathcal B(E,[0,\infty])$, we have
\begin{align} 
 G^{\mathbf t} f 
 = \liminf_{n\to \infty} \Gamma_{(t_n)} f
   \leq \liminf_{n\to \infty} \Gamma_{(t_n)} g
  = G^{\mathbf t} g.
   \end{align}
On the other hand, using Lemma \ref{Lemma:CP!}, we have for each $t\geq 0$, $f\in \mathcal B(E,[0,\infty])$, $u,v \in [0,\infty)$, $r\in [0,1]$ and $\bar r = 1 - r$, it holds that
\begin{align}
 \Gamma_t((ru+\bar r v)f) 
  \geq r \Gamma_t (uf) + \bar r \Gamma_t (vf).
 \end{align}
Therefore, for each $f\in \mathcal B(E,[0,\infty])$, $u,v \in [0,\infty)$, $r \in [0,1]$ and $\bar r = 1 - r$, we have
\begin{align}
 & G^{\mathbf t}((ru + \bar rv)f)
 = \liminf_{n \to \infty} \Gamma_{(t_n)}((ru + \bar rv)f)
 \\&\geq \liminf_{n\to \infty} (r\Gamma_{(t_n)} (uf) + \bar r\Gamma_{(t_n)}(vf)) 
 \geq r (\liminf_{n\to \infty} \Gamma_{(t_n)} (uf)) + \bar r (\liminf_{n\to \infty} \Gamma_{(t_n)}(vf) )
 \\&= r G^{\mathbf t} (uf) + \bar r G^{\mathbf t}(vf),
 \end{align}
as desired.
\end{proof}
In order to give the outline of the proof of Proposition \ref{Proposition:G:H1:H2:H3:H4}, let us first admit the following two propositions.
% **#{Proposition:Gtb:H1:H2:H3:H4} **
\begin{prop} \label{Proposition:Gtb:H1:H2:H3:H4} 
For each unbounded increasing positive sequence $\mathbf t = (t_n)_{n\in \mathbb N}$, $G^\mathbf t$ satisfies that 
\begin{align}
 1 - e^{-G^\mathbf t V_s f} 
 = e^{s\lambda} (1-e^{- G^\mathbf t f}), 
 \quad s\geq 0, f\in \mathcal B(E,[0,\infty]).
 \end{align}
\end{prop}
% **#{Proposition:G*:H1:H2:H3:H4} **
\begin{prop} \label{Proposition:G*:H1:H2:H3:H4} 
If $G^*$ is a $[0,\infty]$-valued monotone concave functional on $\mathcal B(E,[0,\infty])$ satisfying that $G^*(\infty \mathbf 1_E) = \infty$ and that
\begin{align}
 1 - e^{-G^* V_s f} 
 = e^{s\lambda} (1 - e^{- G^* f}),
 \quad s\geq 0, f\in \mathcal B(E,[0,\infty]),
 \end{align}
then $G^* = G^\mathbf t$ for each unbounded increasing positive sequence $\mathbf t = (t_n)_{n\in \mathbb N}$.
\end{prop}
% **Proof of{Proposition:G:H1:H2:H3:H4} by admitting{Lemma:Gta!},{Proposition:Gtb:H1:H2:H3:H4} and{Proposition:G*:H1:H2:H3:H4} **
\begin{proof}[Proof of Proposition \ref{Proposition:G:H1:H2:H3:H4} by admitting Proposition \ref{Lemma:Gta!}, \ref{Proposition:Gtb:H1:H2:H3:H4} and \ref{Proposition:G*:H1:H2:H3:H4}]
Obvious, using a sub-sequence type argument.
\end{proof}
% __A basic lemma__
\subsubsection{A basic lemma}
Thanks to the argument of previous subsubsection, we now only have to proof Proposition \ref{Proposition:Gtb:H1:H2:H3:H4} and \ref{Proposition:G:H1:H2:H3:H4}.
It turns out that the following simple Lemma plays a crucial role in both of their proofs.
% **#{Lemma:Gfnv!}**
\begin{lem} \label{Lemma:Gfnv!} 
It holds that
\begin{align}
 1 - e^{- \Gamma_t f} 
  = \frac{ 1 - e^{- \nu(V_tf)} }{ 1 - e^{- \nu(v_t)}},
  \quad t \geq 0, f\in \mathcal B(E,[0,\infty]).
 \end{align}
\end{lem}
% **Proof of{Lemma:Gfnv!} **
\begin{proof}
It can be verified directly from Lemma \ref{Lemma:sv2!} that
\begin{align}
 & 1 - e^{- \Gamma_t f} 
   = \mathbf P_\nu [ 1 - e^{-X_t(f)} | \|X_t\|> 0] \\
  & = \frac{ \mathbf P_\nu [ 1 - e^{- X_t(f)}]}{ \mathbf P_\nu (\|X_t\| > 0)}
  = \frac{ 1 - e^{- \nu(V_tf)} }{ 1 - e^{- \nu(v_t)}},
  \quad t \geq 0.
 \end{align}
\end{proof}
% __Proof of{Proposition:Gtb:H1:H2:H3:H4} __
\subsubsection{Proof of Proposition \ref{Proposition:Gtb:H1:H2:H3:H4}}
Fix the unbounded increasing positive sequence $\mathbf t = (t_n)_{n\in \mathbb N}$ and $f\in \mathcal B(E,[0,\infty])$. 

Let us first assume $\nu(f)=0$. 
In this case, by Lemma \ref{Lemma:nVn!}, we have for each $t\geq 0$, $\nu(V_tf) = 0$. 
According to Lemma \ref{Lemma:Gfnv!}, we know that $\Gamma_t f=0,t\geq 0$. 
This implies that $G^{\mathbf t}f = 0$. 
In other word, $\nu(f) = 0$ implies that $G^\mathbf tf = 0$.
So, for the similar reason, we also have $G^{\mathbf t}V_t f = 0, t\geq 0$. 
Now the desired result is clearly valid in this case.

For the rest of the proof, let us assume that $\nu(f) > 0$. 
In this case, according to Lemma \ref{Lemma:nVn!}, $\nu(V_tf)>0$ for each $t\geq 0$. 
Therefore, using Lemma \ref{Lemma:Gfnv!} we can write the following
\begin{align}
 & 1 - e^{- \Gamma_t V_s f}
 = \frac{ 1 - e^{- \nu(V_{t+s} f)} }{ 1 - e^{- \nu(v_t)}}
 = \frac{ 1 - e^{- \nu(V_{t+s} f)} }{ 1 - e^{- \nu(V_tf)}} \frac{ 1 - e^{ - \nu(V_tf)}}{ 1 - e^{- \nu(v_t)}} 
 \\ & = \frac{ 1 - e^{- \nu(V_{t+s} f)} }{ 1 - e^{- \nu(V_tf)}} ( 1 - e^{- \Gamma_t f}), 
 \quad t\geq 0, s \geq 0.
 \end{align}
Thus,
\begin{align}
 & 1 - e^{- G^{\mathbf t} V_s f}
 = \liminf_{n\to \infty} ( 1 - e^{- \Gamma_{(t_n)} V_s f})
 = \liminf_{n\to \infty} \Big( \frac{ 1 - e^{- \nu(V_{t_n+s}f)}}{ 1 - e^{- \nu(V_{(t_n)}f)}} (1 - e^{- \Gamma_{(t_n)} f}) \Big) 
 \\& = \Big( \lim_{t \to \infty} \frac{ 1 - e^{- \nu(V_{t+s}f)}}{ 1 - e^{- \nu(V_{t}f)}} \Big) \cdot \liminf_{n\to \infty} (1 - e^{- \Gamma_{(t_n)} f} )
 = e^{s\lambda} (1 - e^{- G^{\mathbf t}f}), \quad s\geq 0.
 \end{align}
Here, in the last step, we used Proposition \ref{Proposition:Vf1:H1:H2:H4} and \ref{Proposition:nVR:H1:H2:H3:H4}.
% __Proof of{Proposition:G*:H1:H2:H3:H4} __
\subsubsection{Proof of Proposition \ref{Proposition:G*:H1:H2:H3:H4}} 
% **Settings**
In this subsubsection, in order to proof Proposition \ref{Proposition:G*:H1:H2:H3:H4}, we always assume that $G^*$ is a $[0,\infty]$-valued monotone concave functional on $\mathcal B(E,[0,\infty])$ satisfying that $G^*(\infty \mathbf 1_E) = \infty$ and that 
\begin{align}
  1 - e^{- G^* V_sf} 
  = e^{s\lambda} (1- e^{- G^* f}),
  \quad s \geq 0, f \in \mathcal B(E, [0,\infty]).
  \end{align}
Define $(Q_t)_{t\geq 0}$ as a family of $[0,\infty)$-valued functional on $\mathcal B(E,[0,\infty])$ given by
\begin{align}
 Q_tg 
:= e^{- \lambda t}( 1 - e^{-G^*(gv_t)} ).
 \end{align}
% **#{Lemma:vp:H1:H2:H3:H4}**
\begin{lem} \label{Lemma:vp:H1:H2:H3:H4} 
$v_t$ is a $(0,\infty)$-valued funciton provided $t$ is large enough. 
\end{lem}
% **Proof of{Lemma:vp:H1:H2:H3:H4}**
\begin{proof}
Note from Lemma \ref{Lemma:sv2!}, $\nu(v_t)>0$. 
Therefore, from Proposition \ref{Proposition:Vf2:H1:H2:H3:H4} we get the desired result.
\end{proof}
% **#{Claim:GQ:H1:H2:H3:H4}**
\begin{claim} \label{Claim:GQ:H1:H2:H3:H4} 
 $\lim_{t\to \infty} Q_t(u \mathbf 1_E) = u$ for each $u\in [0,1]$.
\end{claim}
% **Proof of{Proposition:G*:H1:H2:H3:H4} by admitting{Claim:GQ:H1:H2:H3:H4}**
\begin{proof}[Proof of Proposition \ref{Proposition:G*:H1:H2:H3:H4} by admitting Claim \ref{Claim:GQ:H1:H2:H3:H4}]
Fix an unbounded increasing positive sequence $\mathbf t=(t_n)_{n\in \mathbb N}$ and a function $f\in \mathcal B(E,[0,\infty])$, we only have to proof that $G^* f = G^{\mathbf t}f.$ 
From the definition of $G^{\mathbf t} f$, we can chose a subsequence $\mathbf t'=(t'_n)_{n \in \mathbb N}$ of $\mathbf t$ such that $ 1 - e^{- G^{\mathbf t}f} =  ( 1 - e^{-\Gamma_{( t_n')} f} ) \chi(
\lim_{t\to \infty}). $ 
Therefore, from Lemma \ref{Lemma:Gfnv!}, Proposition \ref{Proposition:nVR:H1:H2:H3:H4} and \ref{Proposition:Vf2:H1:H2:H3:H4} we have
\begin{align}
   & 1 - e^{- G^{\mathbf t}f}
   = \frac{1 - e^{- \nu( V_{(t_n')}f)}}{1- e^{- \nu(v_{(t_n')})}}  \chi(\lim_{n\to \infty}) 
 \\& = \frac{\nu (V_{(t_n')} f)}{\nu(v_{(t_n')})} \chi(\lim_{n\to \infty}) 
   =  \frac{V_{(t_n')}f(x)}{v_{(t_n')}(x)}  \chi(\lim_{n \to \infty} \sup_{x\in E}).
 \end{align}
Note that $V_tf \leq v_t$ for each $t\geq 0$. 
Fix an arbitrary $\epsilon > 0$. 
From above, there exists $n$ large enough such that
\begin{align}
   (1-\epsilon) (1 - e^{- G^{\mathbf t}f} )
   \leq \frac{V_{(t_n')}f(x)}{v_{(t'_n)}(x)}
   \leq ((1+\epsilon) ( 1 - e^{- G^{\mathbf t}f} )) \wedge 1,
   \quad x\in E.
 \end{align}
Since $G^*$ is a monotone functional, we know that for each $t\geq 0$, $Q_t$ is also a monotone functional.
This imples that for $n$ large enough 
\begin{align}
 & Q_{(t'_n)}[ (1-\epsilon) (1-e^{-G^{\mathbf t}f})\mathbf 1_E ]
   \leq Q_{(t'_n)}\Big( \frac{V_{(t'_n)}f}{v_{(t'_n)}} \Big) 
   \\ &  \leq Q_{(t'_n)}[ ( (1+\epsilon) (1-e^{-G^{\mathbf t}f}) \wedge 1) \mathbf 1_E ].
 \end{align}
Note from Lemma \ref{Lemma:vp:H1:H2:H3:H4}, $v_t$ is a $(0,\infty)$-valued funciton provided $t$ is large enough. 
Now from the definition of $(Q_t)_{t\geq 0}$ and $G^*$, we always have for large enough $t>0$ that
\begin{align}
 Q_t \Big( \frac{V_tf}{v_t}  \Big) 
   = e^{- \lambda t}( 1 - e^{- G^*V_tf}  )
   = 1- e^{- G^* f}.
 \end{align}
Therefore, taking $n \to \infty$ in the previous inequality, from Claim \ref{Claim:GQ:H1:H2:H3:H4}  we get that
\begin{align}
 (1 - \epsilon) (1 - e^{- G^{\mathbf t}f})
   \leq 1 - e^{- G^* f} 
   \leq ((1 + \epsilon) (1 - e^{- G^{\mathbf t} f}))\wedge 1.
 \end{align}
Taking $\epsilon \to 0$, we get the desired result.
\end{proof}
Now, in order to proof Proposition \ref{Proposition:G*:H1:H2:H3:H4}, we only have to proof Claim \ref{Claim:GQ:H1:H2:H3:H4}.
First, we need a simple lemma.
% **#{Lemma:QM!}**
\begin{lem} \label{Lemma:QM!} 
For each $u \in [0,1]$, $Q_t(u\mathbf 1_E)$ is non-decreasing in $t\in (0,\infty)$. 
In particular, we can define the $[0,\infty]$-valued function $q(u):= \lim_{t\to \infty} Q_t(u\mathbf 1_E), u\in [0,1]$.
\end{lem}
% **Proof of{Lemma:QM!}**
\begin{proof} 
Note that $\mathbb P_{\delta_x}[e^{- X_s(uv_t)}] = e^{-V_s(uv_t)},x\in E, s,t>0, u \geq 0$.
Lemma \ref{Lemma:CP!} says that, for each $s,t > 0$ and $x\in E$, $u\mapsto V_s(uv_t)(x) $ is a $[0,\infty]$-valued concave function on $[0,\infty)$. 
Therefore, for $u\in [0,1]$, writing $\bar u = 1- u$, we get that
\begin{align}
 V_s(uv_t)
   =V_s((u\cdot 1 + \bar u \cdot 0)v_t) 
   \geq uV_s(v_t) + \bar u V_s(0\cdot v_t) 
   = uv_{s+t},
   \quad s,t > 0.
 \end{align} 
Using this, we have the following: 
\begin{align}
  & Q_{t+s}(u\mathbf 1_E) 
 = e^{- \lambda (t+s)} ( 1-e^{-G^*(uv_{t+s})} ) 
 \leq e^{- \lambda(t+s)}( 1-e^{-G^*[V_s(uv_t)]} ) \\
 & = e^{-\lambda t}( 1-e^{-G^*(uv_t)} )
     = Q_t(u\mathbf 1_E),
 \quad s,t > 0, u \in [0,1],
 \end{align}
as desired.
\end{proof}
% **#{Lemma:qC!}**
\begin{lem} \label{Lemma:qC!} 
Function $q$ is non-decreasing and concave on $[0,1]$; and $q(1) = 1$.
In particular, thanks to Lemma \ref{Lemma:CR!}, $q$ is a continuous $[0,1]$-valued function on $(0,1]$.
\end{lem}
% **Proof of{Lemma:qC!} **
\begin{proof} 
From $G^*(\infty \mathbf 1_E) = \infty$ and $V_t(\infty \mathbf 1_E) = v_t$ we have that
\begin{align}
Q_t(\mathbf 1_E) 
= e^{- \lambda t} ( 1-e^{-G^*v_t} ) 
= e^{- \lambda t} e^{\lambda t}( 1-e^{-G^*(\infty\mathbf 1_E)} )
= 1,
\quad t\geq 0.\end{align}
Therefore $q(1) = 1$ thanks to Lemma \ref{Lemma:QM!}.
The above argument also says that $G^*v_t < \infty$ for each $t>0$. 
Now from the condition that $G^*$ is monotone concave, we have that for each $t>0$, map $u \mapsto G^*(uv_t)$ is a non-decreasing and concave $[0,\infty)$-valued function on $[0,1]$.
From Lemma \ref{Lemma:CE!} we can verify that, for each $t> 0$, $u \mapsto Q_t(u \mathbf 1_E)$ is a $[0,\infty)$-valued, non-decreasing and concave function on $[0,1]$.
Finally, using Lemma \ref{Lemma:QM!} and \ref{Lemma:CL!}, we get the desired result. 
\end{proof}
% **Proof of{Claim:GQ:H1:H2:H3:H4}**
\begin{proof}[Proof of Claim \ref{Claim:GQ:H1:H2:H3:H4}]
From Proposition \ref{Proposition:Vf2:H1:H2:H3:H4} and \ref{Proposition:nVR:H1:H2:H3:H4} we have that
\begin{align}
& e^{\lambda s}(\phi^{-1}v_t)(x) 
= e^{\lambda s}\nu(v_{t})\chi(\lim_{t\to \infty}\sup_{x\in E}) 
\\&=\nu(v_{t+s}) \chi(\lim_{t\to \infty}\sup_{x\in E}) 
= (\phi^{-1}v_{t+s})(x) \chi(\lim_{t\to \infty} \sup_{x\in E}), 
\quad s\geq 0. \end{align}
Note from Lemma \ref{Lemma:vp:H1:H2:H3:H4}, $v_t$ is a $(0,\infty)$-valued funciton provided $t$ is large enough. 
Thus, for each $s\geq 0$ and $\epsilon >0$ there exists $t>0$ large enough such that
\begin{align} 
1-\epsilon
\leq \frac{e^{\lambda s}v_t(x)}{v_{t+s}(x)} 
\leq 1+\epsilon,
\quad x\in E. 
\end{align}
From this we can verify that for each $s\geq 0$ and $\epsilon >0$ there exists $t>0$ large enough such that for each $u \in [0,\infty)$,
\begin{align} 
& Q_{t+s}[ (1-\epsilon)u\mathbf 1_E ] 
= e^{-\lambda(t+s)}( 1-e^{-G^*[(1-\epsilon)uv_{t+s}]} ) 
\\ & \leq e^{-\lambda t} e^{-\lambda s}( 1- e^{-G^*(ue^{\lambda s}v_t)} )
= e^{-\lambda s}Q_t(ue^{\lambda s} \mathbf 1_E), 
\end{align}
and similarly that
\begin{align} e^{-\lambda s}Q_t[ue^{\lambda s}\mathbf 1_E]  \leq Q_{t+s}[(1+\epsilon)u\mathbf 1_E]. \end{align}
Now taking $t\to \infty$ in the above two inequality, we get from Leamma \ref{Lemma:QM!} that, for each $s\geq 0$ and $u\in (0,1), \epsilon > 0$ with $0 < (1 - \epsilon) u < (1+\epsilon)u < 1$, it holds that
\begin{align} 
q((1-\epsilon)u)
\leq e^{-\lambda s}q(u e^{\lambda s}) 
\leq q((1+\epsilon)u). 
\end{align}
According to Lemma \ref{Lemma:qC!}, $q$ is continuous on $(0,1]$ and that $q(1)= 1$.
So taking first $\epsilon \to 0$ and then $u \uparrow 1$ above, we get that 
\begin{align} 
q(1) 
=1
= e^{- \lambda s} q(e^{\lambda s}), 
\quad s \geq 0. 
\end{align} 
In other word, $q(u) = u$ on $(0,1]$.
Finally noticing that $q$ is non-negative and non-decreasing on $[0,1]$, we also have $q(0) = 0$.
\end{proof}
% **Proof of{Proposition:GD:H1:H2:H3:H4} **
\subsection{Proof of Proposition \ref{Proposition:GD:H1:H2:H3:H4}}
% **#{Lemma:OtO:H1:H2:H4} **
\begin{lem} \label{Lemma:OtO:H1:H2:H4} 
Map $t \mapsto \nu(v_t)$ is a strictly decreasing 1-1 map from $(0,\infty)$ to $(0,\infty)$.
\end{lem}
% **Proof of{Lemma:OtO:H1:H2:H4}**
\begin{proof} 
From Lemma \ref{Lemma:nV:H4} and \ref{Lemma:nuP!}, we know that $ t\mapsto e^{-\lambda t}\nu(v_t)$ is a non-increasing $[0,\infty)$-valued continuous function on $(0,\infty)$. 
From Assumption \ref{Assumption:H1!} that $\lambda < 0$ we know that $t\mapsto \nu(v_t)$ is a strictly decreasing $[0,\infty)$-valued continuous function on $(0,\infty)$.
Now, to get the desired result, we only need to show $\lim_{t\downarrow 0}\nu(v_t) = \infty$ and $\lim_{t\uparrow \infty}\nu(v_t) = 0$.
$\lim_{t\to \infty} \nu(v_t) = 0$ is valid thanks to Proposition \ref{Proposition:Vf1:H1:H2:H4}.
$\lim_{t\downarrow 0} \nu(v_t) = \infty$ is valid thanks to Fact \ref{Fact:sv1!} and the monotone convergence theorem:
$e^{-\nu(v_t)} = \mathbb P_\nu (\|X_t\| = 0) \xrightarrow[t\to \infty]{} 0$.
\end{proof}
% **Proof of{Proposition:GD:H1:H2:H3:H4}**
\begin{proof}[Proof of Proposition \ref{Proposition:GD:H1:H2:H3:H4}]
Fix a sequence $(g_{n})_{n\in \mathbb N}$ in $\mathcal B(E,[0,\infty])$ bounded pointwisely. 
From Fact \ref{Fact:P!} and Assumption \ref{Assumption:H3!}, there exists a constant $C > 0$ such that 
\begin{align}
V_1 g_n(x) \leq P^\beta_1 g_n(x) \leq C \phi(x) \nu(g_n),
\quad n \in \mathbb N, x\in E.
\end{align}
From the bounded convergence theorem, we have $\lim_{n\to \infty}\nu(g_n) =0$.
Denote by $R:(0,\infty) \to (0,\infty)$ the inverse of the map $t \mapsto \nu(v_t)$ on $(0,\infty)$, then according to Lemma \ref{Lemma:OtO:H1:H2:H4}, $R$ is a strictly decreasing 1-1 map from $(0,\infty)$ to $(0,\infty)$. 
Define $t_n:= R(2C\nu(g_n))> 0$.
Then we have that $\lim_{n\to \infty} t_n = \infty$ and that 
\begin{align}
V_1 g_n(x) \leq \frac{1}{2} \phi(x) \nu(v_{(t_n)}), 
\quad n \in \mathbb N, x\in E.
\end{align}
From Proposition \ref{Proposition:Vf2:H1:H2:H3:H4} we have that
\begin{align}
(\phi^{-1} \cdot v_{(t_n)})(x) 
= \nu(v_{(t_n)}) \chi(\lim_{n\to \infty} \sup_{x\in E}).
\end{align}
In particular, we have that, for $n$ large enough,
\begin{align}
\nu(v_{t_n}) \leq 2 \phi(x)^{-1} v_{(t_n)}(x), 
\quad x\in E.
\end{align}
Therefore, for $n$ large enough, we have that $V_1g_n \leq v_{(t_n)}$.
Taking $f = \infty$ in Proposition \ref{Proposition:G:H1:H2:H3:H4}, we have that $1 - e^{- Gv_s} = e^{\lambda s}$ for each $s\geq 0$.
This and Proposition \ref{Proposition:G:H1:H2:H3:H4} again imply that
\begin{align}
& 1 - e^{- Gg_n}
= e^{- \lambda} (1- e^{- GV_1g_n})
\leq e^{- \lambda} (1- e^{- G v_{(t_n)}}) 
\\& = e^{- \lambda} e^{\lambda t_n},
\quad \text{$n$ large enough.}
\end{align} 
Taking $n\to \infty$, we get the desired result.
\end{proof}
% **APPENDIX**
\appendix
\section{}
% **Extended values**
\subsection{Extended values}
In this paper, we often work with the extended non-negative real number system $[0,\infty]$ which consists of the non-negative real line $[0,\infty)$ and an extra point $\infty$. 
We consider $[0,\infty]$ as the one point compactification of $[0,\infty)$; and therefore, it is a compact Hausdorff space.
Preserving the original order in $[0,\infty)$, define $x < \infty$ for each $x\in [0,\infty)$.
We also make the following conventions that 
\begin{itemize}
\item
$x + \infty = \infty$ for each $x\in [0,\infty]$; 
\item
$x \cdot \infty = \infty$ for each $x\in (0,\infty]$; and
\item
$0 \cdot \infty = 0$; $\frac{1}{\infty} = 0$; $\frac{1}{0} = \infty$.
\end{itemize}
% **Big O Notations**
\subsection{Big O Notations}
Let $S$ be an arbitrary set. 
Let $B$ be a $[0,\infty]$-valued functional on the space of non-negative functions of $S$. 
Then for each $[-\infty, \infty]$-valued or $\mathbb C$-valued function $f$ on $S$, we define the following:
\begin{itemize}
\item
$f \in o(B)$ iff $B |f| = 0$; 
\item
$f \in O(B)$ iff $B|f| < \infty$; 
\item
$f \in \chi(B)$ iff $f - 1 \in o(B)$;
\item
$f\in \Theta (B)$ iff $f \in O(g)$ and $f^{-1} \in O(g)$.
\end{itemize}
With some abuse of notations, we sometimes write $f = o(B)$ meaning that $f\in o(B)$; and $o(B) = O(B)$ meaning that $o(B) \subset O(B)$. 
For example, take $S = \mathbb N $ and the functional $B$ as $\limsup_{n\to \infty}$, then for non-zero sequence $(b_n)_{n\in \mathbb N}$:
\begin{itemize}
\item
$a_n = b_n o(\limsup_{n\to \infty})$ means that $\lim_{n\to \infty}\frac{a_n}{b_n} =0$.
\item
$a_n = b_nO(\limsup_{n\to \infty})$ means that there exists $c>0$ and $N > 0$ such that for each $n\geq N$, $|a_n/b_n| \leq c$.
\item
$a_n = b_n \chi(\limsup_{n\to \infty})$ means that $\lim_{n\to \infty} a_n/b_n = 1$.
\item
$a_n = b_n \Theta (\limsup_{n\to \infty})$ for some positive $(b_n)_{n\geq 0}$ means that there exists $c_0,c_1>0$, and $N > 0$ such that for each $n \geq N$ we have $c_0 b_n \leq a_n \leq c_1b_n$.
\end{itemize}
% **Cancave functionals**
\subsection{Cancave functionals}
% **Notations**
We say a $\mathbb R$-valued (or $[0,\infty]$-valued) function $f$ on a convex subset $C$ of $\mathbb R$ is concave iff
\begin{align}
   f(rx+\bar r y) 
 \geq r f(x) + \bar r f(y),
 \quad x,y \in C, r \in [0,1], \bar r = 1 - r. 
 \end{align}
The following fact is due to Corollary 6.3.3. of \cite{Dudley2002Real}.
% **#{Fact:CC!} **
\begin{fact} \label{Fact:CC!} 
If $f$ is a $\mathbb R$-valued concave function on a convex subset $C$ of $\mathbb R$, then $f$ is continuous on $C^o$. 
Furthermore, both left and right derivatives of $f$ exist and are finite on $C^o$.
\end{fact}
% **#{Lemma:CR!}**
\begin{lem} \label{Lemma:CR!} 
If $f$ is a non-decreasing $\mathbb R$-valued concave function on $(a,b]$ where $a<b$ in $\mathbb R$, then $f$ is continuous on $(a,b]$.
\end{lem}
% **Proof of{Lemma:CR!}**
\begin{proof}
From Fact \ref{Fact:CC!} we know that $ f $ is continuous on $(a,b)$. 
From the fact that $ f $ is non-decreasing, we have $f(b-) = \lim_{x \to b} f(x) \leq f(b)$. 
So what is left is to proof that $f(b-) \geq f(b)$. 

From the concave property of the function $f$ on $(a,b]$ we know that, for a fixed $c \in (a,b)$,
\begin{align}
 f(rc + \bar r b) 
 \geq r f(c) + \bar r f(b),
 \quad r\in [0,1], \bar r = 1 - r.
 \end{align}
Taking $r\to 0$, we get that $f(b-)\geq f(b)$.
\end{proof}
% **#{Lemma:CL!}**
\begin{lem} \label{Lemma:CL!} 
Suppose that $(f_n)_{n \in \mathbb N}$ is a sequence of $[0,\infty)$-valued concave functions on a convex subset $C$ of $\mathbb R$, then so is $f:= \liminf_{n\to \infty} f_n.$
\end{lem}
% **Proof{Lemma:CL!}**
\begin{proof}
Since $(f_n)_{n \in \mathbb N}$ are non-negative, we know $f$ is also a non-negative function. 
For each $x,y\in C, r\in [0,1]$ and $\bar r = 1 - r$, from the fact that
\begin{align}
   f_n(rx+\bar r y) 
 \geq r f_n(x) + \bar r f_n(y),
  \quad n \in \mathbb N,
 \end{align}
we can verify that
\begin{align}
 & f(rx+ \bar r y) 
 \geq \liminf_{n\to \infty} (r f_n(x) + \bar r f_n(y))
 \\&\geq r (\liminf_{n\to \infty} f_n(x)) + \bar r (\liminf_{n\to \infty} f_n(y)) 
 = rf(x) + \bar r f(y),
 \end{align}
as desired.
\end{proof}
% **#{Lemma:CP!}**
\begin{lem} \label{Lemma:CP!} 
Suppose that $\{Z; P\}$ is a $[0,\infty]$-valued random variable. 
Define $L(u):= - \log P[e^{- u Z}]$ with $u \in [0,\infty)$, then $L$ is a $[0,\infty]$-valued concave function on $[0,\infty)$.
\end{lem}
% **Proof of{Lemma:CP!} **
\begin{proof}
If $P(Z < \infty) = 0$, then $L(u) = 0\cdot \mathbf 1_{u=0} + \infty \cdot \mathbf 1_{u>0}$. 
Therefore $L$ is concave. 
So for the rest of this proof, we assume that $P(Z < \infty) > 0$. 
Note now we have $L$ is a $[0,\infty)$-valued function on $[0,\infty)$. 
It can be verified that
\begin{align}
 \frac{\partial L(u)}{\partial u} 
 = P[e^{-u Z}]^{-1} P[Ze^{- u Z}]
, \quad u > 0.
 \end{align}
It can also be verified that
\begin{align}
 & \frac{\partial^2 L(u)}{\partial u^2}
 = P[e^{-uZ}]^{-2}\Big( \frac{\partial P[Ze^{-uZ}]}{\partial u} \cdot P[e^{-uZ}] - P[Ze^{-uZ}] \cdot \frac{\partial P[e^{-uZ}]}{\partial u}\Big) 
 \\& = - \frac{P[Z^2 e^{-uZ}]}{P[e^{-uZ}]} + \frac{P[Ze^{-uZ}]^2}{P[e^{-uZ}]^2} 
 = Q^{(u)}[Z]^2 - Q^{(u)}[Z^2] 
 \leq 0, 
 \quad u \in (0,\infty).
   \end{align}
Here, for each $u$, the probability measure $Q^{(u)}$ is given by 
\begin{align}
 dQ^{(u)}
:= \frac{e^{-uZ}}{P[e^{-uZ}]} dP
 \end{align} 
and in the last step, we used Jason's inequality. 
All these says that $L$ is concave on $(0,\infty)$. 
Finally, noticing from monotone convergence, $L$ is continuous on $0$, we complete the proof.
\end{proof}
% **#{Lemma:CE!}**
\begin{lem} \label{Lemma:CE!} 
Suppose that $g$ is a concave function on some convex subset $C$ of $\mathbb R$, then so is $q:= 1- e^{-g}.$
\end{lem}
% **Proof of{Lemma:CE!}**
\begin{proof}
For each $u,v \in C$, $r \in [0,1]$ and $\bar r = 1-r$, we have
\begin{align}
 &q(ru+\bar r v) 
 = 1 - e^{- g(ru + \bar r v)}
 \geq 1 - e^{- ( r g(u) + \bar r g(v))} 
 \\& \geq r(1- e^{- g(u)}) + \bar r (1 - e^{- g(v)})
 = rq(u) + \bar r q(v),
 \end{align}
as desired.
\end{proof}
% **REFERENCE**
\begin{thebibliography}{99}
%{Li2011MeasureValued} 
\bibitem{Li2011MeasureValued}
Li, Z.:
\emph{Measure-valued branching Markov processes.}
Probability and its Applications (New York). Springer, Heidelberg, 2011. xii+350 pp.
%{Dudley2002Real}
\bibitem{Dudley2002Real} 
Dudley, R. M.:
\emph{Real analysis and probability.}
Revised reprint of the 1989 original. Cambridge Studies in Advanced Mathematics, 74. Cambridge University Press, Cambridge, 2002. x+555 pp.
\end{thebibliography}
\end{document}