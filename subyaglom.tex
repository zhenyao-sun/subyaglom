% * The preamble
% ! This TeX file uses 'amsart' for its document class. 
\documentclass[12pt,a4paper]{amsart}
\setlength{\textwidth}{\paperwidth}
\addtolength{\textwidth}{-2in}
\calclayout
% ! 'amsmath', 'amsthm', 'amsfonts' package is automatically loaded by 'amsart' class.
% ! Below is the configuration of 'amsmath' package.
\numberwithin{equation}{section}
\allowdisplaybreaks
% ! Below is the configuration of 'amsthm' package.
\theoremstyle{plain}
\newtheorem{thm}{Theorem}[section]
\newtheorem{lem}[thm]{Lemma}
\newtheorem{prop}[thm]{Proposition}
\newtheorem{cor}[thm]{Corollaray}
\newtheorem{conj}[thm]{Conjecture}
\theoremstyle{definition}
\newtheorem{defi}[thm]{Definition}
\newtheorem{exa}[thm]{Example}
\newtheorem{asp}{Assumption}
\newtheorem{iss}{Issue}
\newtheorem{rem}[thm]{Remark}
\newtheorem*{ack}{Acknowledgment}
% ! 'inputenc' package is used to specify the set of characters allowed to input into this TeX file.
% \usepackage[utf8]{inputenc}
% ! 'foutenc' package is used to specify the set of characters allowed to output into the generated pdf file.
% \usepackage[T1]{fontenc}
% ! 'amssymb', 'mathrsfs' and 'mathtools' packages provide additional math symbols from the AMS default symbol fonts. They are compatible with the AMS class and is recommended to be used.
\usepackage{amssymb}
\usepackage{mathtools}
\mathtoolsset{showonlyrefs}
\usepackage{mathrsfs}
% ! 'hyperref' package is used to handle cross-referencing in the generated pdf file.
\usepackage[backref]{hyperref}
% ! Other packages
\usepackage{xcolor}
\usepackage[inline]{showlabels}
\usepackage{comment}
\usepackage{enumitem}
% * Top matter
\begin{document}
\title {Subcritical Superprocesses }
\author
[R. Liu, Y.-X. Ren, R. Song and Z. Sun]
{Rongli Liu, Yan-Xia Ren, Renming Song and Zhenyao Sun}
\address{
  Yan-Xia Ren\\
  School of Mathematical Sciences\\
  Peking University\\
  Beijing, P. R. China, 100871}
\email{
  yxren@math.pku.edu.cn}
\thanks{
  The research of Yan-Xia Ren is supported in part by NSFC (Grant Nos. 11671017 and 11731009).}
\address{
  Rongli Liu\\
  Information about Rongli Liu}
\email{rlliu@bjtu.edu.cn}
\thanks{
  The research of Rongli Liu is supported in part by NSFC
  (Grant No. 11301261), and the Fundamental Research Funds for the Central Universities (Grant No.  2017RC007)}
\address{
  Renming Song\\
  Information about Renming Song}
\email{email of RS}
\address{
  Zhenyao Sun\\
  School of Mathematical Sciences\\
  Peking University\\
  Beijing, P. R. China, 100871}
\email{zhenyao.sun@pku.edu.cn}
\begin{abstract}
{\bf TBD}
\end{abstract}
\maketitle
% * Document body
% ** (I)ntroduction
\section{Introduction}
% \label{sec:I}
% *** Main (R)esults
\subsection{Main results}
\label{sec:IR}
% **** Notations
Denote by $\mathcal B(E,B)$ the collection of all Borel measurable maps from a Borel space $E$ to a measurable space $B$.
Denote by $\mathcal B_b(E,B)$ the collection of all bounded elements in $\mathcal B(E,B)$.
Denote by $\mathcal M(E)$ the space of all Borel measures on a Borel space $E$. 
Denote by $\mathcal M_f(E)$ the space of all finite elements in $\mathcal M(E)$ equipped with the weak topology. 

% **** Definition of (S)uperrocesses
\begin{defi}
  \label{def:IRS}
Let $E$ be a locally compact separable metric space. 
We say an $\mathcal M_f(E)$-valued Hunt process $X= \{(X_t)_{t\geq 0}; (\mathbb P_\mu)_{\mu\in \mathcal M_f(E)}\}$ is a \emph{$(Y,\psi)$-superprocess} if
\begin{enumerate}
\item
\label{def:IRS1}
  the \emph{spatial motion} $Y=\{(Y_t)_{t\geq 0};(\Pi_x)_{x\in E}\}$ is an $E$-valued Hunt process with its life time denoted as $\tau$;
\item
  \label{def:IRS2}
  the \emph{branching mechanism} $\psi$ is a map from $E\times[0,\infty)$ to $[0,\infty)$ given by
  \[
    \psi(x,z)
    =-\beta(x)z + \sigma(x)^2 z^2+ \int_{(0,\infty)} (e^{-rz}-1+zr ) \pi(x, dr),\qquad x\in E, z\geq0,
  \]
  where $\beta, \sigma\in \mathcal B_b(E,\mathbb R)$ and $\pi$ is a kernel from $E$ to $(0,\infty)$ such that
  \[
    \sup_{x\in E}\int_0^\infty (r\wedge r^2)\pi(x,dr)
    <\infty.
  \]
\item
  \label{def:IRS3}
  for each $f\in \mathcal B_b(E,[0,\infty))$, it holds that
  \begin{align}
    \mathbb P_\mu [e^{- X_t(f)}] = e^{-\mu(V_tf)},
    \quad \mu \in \mathcal M_f(E),t\geq 0,
  \end{align}
  where the map $(t,x) \mapsto V_tf(x)$ on $[0,\infty) \times E$ is the unique locally bounded positive solution to the equation
  \begin{align}
    V_t f(x) +   \Pi_x[\int_0^{t\wedge \tau} \psi (\xi_s,V_{t-s} f(\xi_s)) ds]
    = \Pi_x[f(\xi_t) \mathbf 1_{t < \tau}],
    \quad x \in E, t \geq 0.
  \end{align}
  Here, we say a map $(t,x) \mapsto f(t,x)$ on $[0,\infty)\times E$ is \emph{locally bounded} if for each $t_0 \geq 0$, $\sup_{t\in [0,t_0], x\in E} |f(t,x)| < \infty$.
\end{enumerate}
\end{defi}

% **** Theorem (Y)glom
\begin{thm}
  \label{thm:IRY}
	Let $X$ be the superprocess given by Definition \ref{def:IRS}. 
  We say a probability ${\mathbf P}$ on $\mathcal M_f(E)$ is the \emph{Yaglom distribution} of the superprocess $X$ if for each $\mu\in \mathcal M_f(E)\setminus\{0\}$ we have
  \begin{align}
    \mathbb P_\mu(X_t \in \cdot | \|X_t\|>0) 
    \xrightarrow[t\to \infty]{d} {\mathbf P}(\cdot).
  \end{align}
  Then it holds that the Yaglom distribution $\mathbf P$ of $X$ exists.
\end{thm}
% ** (P)reliminary
\section{Preliminary}
% *** (S)uperprocesses
\subsection{Superprocesses}
% ***** (C)umulant semigroup
  \begin{lem}[$\Delta$]
    \label{lem:ACP}
    Let $X$ be the superprocess given in Definition \ref{def:IRS}.
    There exists a unique family of operators $(\overline V_t)_{t \geq 0}$ from $\mathcal B(E, [0,\infty])$ to $\mathcal B(E, [0,\infty])$ satisfying the followings
    \begin{enumerate}[label=(\alph*)]
    \item
      for each $t\geq 0$ and $f \in \mathcal B_b(E, [0,\infty))$, we have $\overline V_tf = V_tf$.
    \item
      for each $t\geq 0$ and each $f_n \uparrow f$ pointwisely in $\mathcal B(E, [0,\infty])$, we have $\overline V_tf_n \uparrow \overline V_tf$ pointwisely.
    \end{enumerate}
    Moreover, $(\overline V_t)_{t\geq 0}$ satisfies that
    \begin{enumerate}
    \item
      for each $t\geq 0$ and $f\leq g$ in $\mathcal B(E,[0,\infty])$, we have $\overline V_tf \leq \overline V_tg$;
    \item
      for each $t, s\geq 0$, we have $\overline V_{t+s} = \overline V_t \overline V_s$;
    \end{enumerate}
    With some abuse of the notations, we still write $V_t = \overline V_t$ for each $t\geq 0$, and call $(V_t)_{t\geq 0}$ the \emph{cumulant semigroup} of the superprocess $X$.
  \end{lem}

% ***** (E)xtinction exponent 
  \begin{lem}[$\Delta$]
    \label{lem:PSE}
    Define $v_t = V_t(\infty)$ for each $t\geq 0$, then it holds that 
    \begin{align}
      \mathbb P_\mu (\|X_t\| = 0) 
      = e^{- \mu(v_t)}
      , \quad \mu \in \mathcal M_f(E), t\geq 0.
    \end{align}      
  \end{lem}
  
% ***** (M)ean semigroup
  \begin{lem}[$\Delta$]
    \label{lem:PSM}
    Let $X$ be a superprocess given by Definition \ref{def:IRS}. 
    It holds that
    \begin{align} 
      \mathbb P_\mu [X_t(f)] = \mu(P^\beta_t f), 
      \quad t \geq 0, f\in \mathcal B_b(E, \mathbb R),
    \end{align} 
    where \emph{the mean semigroup} $(P_t^\beta)_{t\geq 0}$ is a family of operators on $\mathcal B_b(E, \mathbb R)$ given by
    \begin{align}
      P^\beta_tf(x)
      := \Pi_x [e^{\int_0^{t} \beta(Y_r)dr} f(Y_t)\mathbf 1_{\{t<\tau\}}],
      \quad f \in \mathcal B_b(E, \mathbb R),t\geq 0, x\in E.
    \end{align}
  \end{lem}

% ***** % Equations of superprocesses 
\begin{comment}
  \begin{lem}[Fact]
    \label{lem:E}
    Let $X$ be the superprocess given in Definition \ref{def:IRS}. 
    Define a function $\psi_0$ on $E\times [0,\infty)$ such that $ \psi_0(x,z)  := \psi(x,z)+ \beta(x) z$ for each  $x\in E$ and  $z \geq 0$.
    Using the monotonicity, extend $\psi$ as a $[0,\infty]$-valued function on $E \times [0,\infty]$ by setting that $\psi(x,\infty) = \lim_{z\uparrow \infty} \psi(x,z)$.
    Define operator $\Psi_0$ on $\mathcal B(E,[0,\infty])$ by $ \Psi_0 g(x):=\psi_0(x, g(x))$.
    Then we have
    \begin{align}
      V_sf+ \int_0^s P_{s-u}^\beta \Psi_0V_u f du = P_s^\beta f,
      \quad s\geq 0, f\in \mathcal B(E, [0,\infty]).
    \end{align}
  \end{lem}
\end{comment}

% ***** (S)pectural porperty
  \begin{lem}[$\Delta$]
    \label{lem:PSS}
    Let $X$ be the superprocess given by Definition \ref{def:IRS}. 
    Let $(P^\beta_t)_{t\geq 0}$ be the mean semigroup of $X$.
    Then there exist a number $\lambda \in \mathbb R$, a function $\phi \in \mathcal B_b(E, (0,\infty))$ and a measure $\nu \in \mathcal M_f(E)$ with full support on $E$ such that
    \begin{align}
      & P_t^\beta \phi(x) 
        = e^{\lambda t} \phi(x),
        \quad x\in E, t\geq 0;
      \\ & \int_{x\in E} \nu(dx) P_t^\beta(x,\cdot) 
           = e^{\lambda t} \nu(\cdot),
           \quad t\geq 0;
      \\ & \nu(\phi) 
           = 1.   
    \end{align}
  \end{lem}
  
% ***** (K)ey Lemma
  \begin{lem}[$\Box$]
    \label{lem:PSK}
    Suppose that 
    \begin{enumerate}[label=(\alph*)]
    \item
      \label{lem:PSKa}
      $\lambda < 0$;
    \item
      \label{lem:PSKb}
      $P_t^\beta f(x) 
      = e^{\lambda t} \phi(x) \nu(f) ( 1+ o(\limsup_{t\to \infty} \sup_{x\in E, f\in L_1^+(\nu)}))$;
    \item
      \label{lem:PSKc}
      for each $t\geq 0$, $ P_t^\beta f(x) = \phi(x) \nu(f) O(\sup_{x\in E, f\in L_1^+(\nu)})$.
    \end{enumerate}
    Then for each $f\in \mathcal B(E, [0,\infty])$ the followings hold:
    \begin{enumerate}
    \item 
      \label{lem:PSK1} 
      $V_tf = o(\limsup_{t\to \infty} \sup_{x\in E})$;
    \item 
      \label{lem:PSK2}
      for each $s\geq 0$, we have $\nu(V_{t+s}f) = e^{\lambda s} \nu(V_tf) (1+o(\limsup_{t\to \infty}))$.
    \item 
      \label{lem:PSK3}
      $ (\phi^{-1} \cdot V_tf)(x) = \nu(V_tf) ( 1+ o(\limsup_{t\to \infty} \sup_{x\in E}) )$
    \end{enumerate}
  \end{lem}

% ***** (N)on-presistency
  \begin{lem}[$\Delta$]
    \label{lem:PSN}
    $\mathbf P_\mu(\|X_t\| > 0)\in (0, 1)$ for each $t > 0$ and $\mu \in \mathcal M_f(E)\setminus \{0\}$.
  \end{lem}
  
% ***** Proof of lem:PSN  
  \begin{proof}
    For each $t\geq 0$ and $\mu \in \mathcal M_f(E)\setminus \{0\}$, since $\phi$ is strictly positive, we have $\mathbb P_\mu[X_t(\phi)] = \mu(P_t^\beta \phi) = e^{\lambda t} \mu(\phi) > 0$. 
    Then the desired result follows.  
    {\bf (We still need to show the non-presistence)}
  \end{proof}

% ** Proof of the (M)ain results
\section{Proof of the main results.}
% *** Proof of Theorem (Y)aglom
\subsection{Proof of Yaglom's theorem}
\label{sec:MY}
% **** Proof of Thm:IRY 
\begin{proof}[Proof of Theorem \ref{thm:IRY}]

% ***** \Gamma_t and G^t
Thanks to Lemma \ref{lem:PSN}, we can define a family of $[0,\infty]$-valued functionals $(\Gamma_t)_{t\geq 0}$ on $\mathcal B(E, [0,\infty])$ such that
\begin{align}
  e^{-\Gamma_t f}
  := \mathbf P_{\nu}[e^{- X_t(f)}| \|X_t\| > 0], 
  \quad f\in \mathcal B(E,[0,\infty]), t \geq 0.
\end{align}
For each unbounded increasing time sequence $\mathbf t:= (t_n)_{n\in \mathbb N}$, define
\begin{align}
  G^{\mathbf t}f 
  := \liminf_{n\to \infty} \Gamma_{(t_n)}f,
  \quad f\in \mathcal B(E, (0,\infty]).
\end{align}

% ***** Step 1. 
\emph{Step 1.}
We show that for each unbounded increasing time sequence $\mathbf t:= (t_n)_{n\in \mathbb N}$, $G^{\mathbf t}$ is a $[0,\infty]$-valued monotone concave functional on $\mathcal B(E,[0,\infty])$ satisfying that $G^{\mathbf t}(\infty \mathbf 1_E) = \infty$ and that
\begin{align}
  \label{eq:MY1}
  1 - e^{- G^{\mathbf t}V_sf} 
  = e^{s\lambda} (1- e^{- G^{\mathbf t} f}),
  \quad s \geq 0, f \in \mathcal B(E, [0,\infty]).
\end{align}
Here, we say a $[0,\infty]$-valued functional $A$ defined on $\mathcal B(E,[0,\infty])$ is \emph{monotone concave} if 
\begin{enumerate}
\item
  $f\leq g$ in $\mathcal B(E,[0,\infty])$ implies $Af\leq Ag$;
\item
  for each $f\in \mathcal B(E,[0,\infty])$ with $Af<\infty$, function $u\mapsto A(uf)$ is concave on $[0, 1]$.  
\end{enumerate}

In this step, we fix an unbounded increasing time sequence $\mathbf t = (t_n)_{n\in \mathbb N}$, and observe that since $(\Gamma_t)_{t\geq 0}$ are $[0,\infty]$-valued, so is $G^{\mathbf t}$. 
Also, from $\Gamma_t(\infty \mathbf 1_E) = \infty$ for each $t\geq 0$ we have that $G^{\mathbf t}(\infty \mathbf 1_E) = \infty$.

We now claim that for each $t\geq 0$, $f\in \mathcal B(E,[0,1])$, $u,v \in [0,\infty)$, $r\in [0,1]$ and $\bar r = 1 - r$, it holds that
\begin{align}
  \label{eq:MY1.5}
\Gamma_t((ru+\bar r v)f) 
 \geq r \Gamma_t (uf) + \bar r \Gamma_t (vf).
\end{align}
To see this, we fix $t\geq 0$, $f\in \mathcal B(E,[0,1])$ and observe from Lemma \ref{lem:ACP} that the desired \eqref{eq:MY1.5} is trivial provided $\Gamma_t(f) = - \log \mathbb P_\nu[e^{-X_t(f)}|\|X_t\|>0]< \infty$. 
On the other hand, if $\Gamma_t(f) = - \log \mathbb P_\nu[e^{-X_t(f)}|\|X_t\|>0] = \infty$ then $X_t(f) = \infty$ almost surely with respect to probability $\mathbb P_\mu(\cdot | \|X_t\|>0)$. 
Therefore in this case, $\Gamma_t(0\cdot f) = 0$ and $\Gamma_t(uf) = \infty$ for each $u > 0$. 
The desired result \eqref{eq:MY1.5} also follows.

Using the above claim, we can verify that $G^\mathbf t$ is monotone concave.
In fact, for each $f \leq g$ in $\mathcal B(E,[0,\infty])$, we have
\(	
G^{\mathbf t} f = \liminf_{n\to \infty} \Gamma_{(t_n)} f
  \leq \liminf_{n\to \infty} \Gamma_{(t_n)} g
  = G^{\mathbf t} g.
  \)
On the other hand, for each $u,v \in [0,\infty)$, $f\in \mathcal B(E,[0,\infty])$, $r \in [0,1]$ and $\bar r = 1 - r$, we have
\begin{align}
	& G^{\mathbf t}((ru + \bar rv)f)
   = \liminf_{n \to \infty} \Gamma_{(t_n)}((ru + \bar rv)f)
   \geq \liminf_{n\to \infty} (r\Gamma_{(t_n)} (uf) + \bar r\Gamma_{(t_n)}(vf)) \\
  & \geq r (\liminf_{n\to \infty} \Gamma_{(t_n)} (uf)) + \bar r (\liminf_{n\to \infty} \Gamma_{(t_n)}(vf) ) 
    = r G^{\mathbf t} (uf) + \bar r G^{\mathbf t}(vf).
\end{align}

  We still need to verify that \eqref{eq:MY1} is valid.
  In fact, if $f\equiv 0$ then it can be verified that the both sides of \eqref{eq:MY1} is $0$.
  In the rest of this step, we fix an $f\in \mathcal B(E,[0,\infty])\setminus \{0\}$.
  Note that for each $t\geq 0$, $\{X_t(f);\mathbb P_\nu\}$ is a non-degenerate non-negative random variable since $\mathbb P_\nu[X_t(f)] = \nu(P_t^\beta f) = e^{\lambda t}\nu(f)>0$. 
  Therefore from $e^{-\nu(V_tf)} = \mathbb P_\nu[e^{-X_t(f)}] < 1$ we know that $\nu(V_tf) \not\equiv 0$ for each $t\geq 0$.  
  Notice that
  \begin{align}
    \label{eq:MY2}
    & 1 - e^{- \Gamma_t f} 
    = \mathbf P_\nu [ 1 - e^{-X_t(f)} | \|X_t\|> 0] \\
    & = \frac{ \mathbf P_\nu [ 1 - e^{- X_t(f)}]}{ \mathbf P_\nu (\|X_t\| > 0)}
    = \frac{ 1 - e^{- \nu(V_tf)} }{ 1 - e^{- \nu(v_t)}},
    \quad t \geq 0.
  \end{align}
  Therefore,
  \begin{align}
    & 1 - e^{- \Gamma_t V_s f}
      = \frac{ 1 - e^{- \nu(V_{t+s} f)} }{ 1 - e^{- \nu(v_t)}}
     = \frac{ 1 - e^{- \nu(V_{t+s} f)} }{ 1 - e^{- \nu(V_tf)}} \frac{ 1 - e^{ - \nu(V_tf)}}{ 1 - e^{- \nu(v_t)}} \\
    & = \frac{ 1 - e^{- \nu(V_{t+s} f)} }{ 1 - e^{- \nu(V_tf)}} ( 1 - e^{- \Gamma_t f})
      , \quad t\geq 0, s \geq 0.
  \end{align}
  Thus,
  \begin{align}
    & 1 - e^{- G^{\mathbf t} V_s f}
      = \liminf_{n\to \infty} ( 1 - e^{- \Gamma_{(t_n)} V_s f})
      = \liminf_{n\to \infty} \Big(  \frac{ 1 - e^{- \nu(V_{t_n+s}f)}}{ 1 - e^{- \nu(V_{(t_n)}f)}} (1 - e^{- \Gamma_{(t_n)} f}) \Big) \\
    & = \Big( \lim_{t \to \infty}   \frac{ 1 - e^{- \nu(V_{t+s}f)}}{ 1 - e^{- \nu(V_{t}f)}} \Big) \cdot \liminf_{n\to \infty} (1 - e^{- \Gamma_{(t_n)} f} ) 
      = e^{s\lambda} (1 - e^{- G^{\mathbf t}f}), \quad s\geq 0.
  \end{align}
  Here in the last step, we used Lemma \ref{lem:PSK}.

% ***** Step 2  
  \emph{Step 2.}
  We show that if $G^*$ is a $[0,\infty]$-valued monotone concave functional on $\mathcal B(E,[0,\infty])$ satisfying that $G^*(\infty \mathbf 1_E) = \infty$ and that 
  \begin{align}
    1 - e^{- G^* V_sf} = e^{s\lambda} (1- e^{- G^* f}),
    \quad s \geq 0, f \in \mathcal B(E, [0,\infty]),
  \end{align}
  then
  \begin{align}
    \lim_{t\to \infty} Q_t(u \mathbf 1_E) 
    = u,
    \quad u \in [0,1],
  \end{align}
  where $(Q_t)_{t\geq 0}$ is a family of $[0,\infty)$-valued functional on $\mathcal B(E,[0,\infty])$ given by
  \begin{align}
    Q_tg 
    := e^{- \lambda t}( 1 - e^{-G^*(gv_t)} ).
  \end{align}

\emph{Substep 2.1.} We show that for each $u \in [0,1]$, $Q_t(u \mathbf 1_E)$ is non-decreasing in $t\in (0,\infty)$; and therefore we can define $\lim_{t\to \infty} Q_t(u \mathbf 1_E)=:q(u)\in [0,\infty)$.
In fact, for each $s,t> 0$ and $x\in E$, since $\mathbb P_{\delta_x} [e^{-X_s (v_t)}] = e^{- (V_s v_t) (x)} = e^{-v_{t+s}(x)} = \mathbb P_{\delta_x}(\|X_{t+s}\| = 0) \in (0, 1)$ we have that $\{X_s(v_t); \mathbb P_{\delta_x}\}$ is a non-degenerate random variable.
Therefore, Lemma \ref{lem:ACP} says that, for each $s,t > 0$ and $x\in E$, $u\mapsto V_s(uv_t)(x)$ is concave on $[0,\infty)$. 
Therefore, for $u\in [0,1]$, writing $\bar u = 1- u$, we get that
\begin{align}
	V_s(uv_t)
  =V_s((u\cdot 1 + \bar u \cdot 0)v_t) 
  \geq uV_s(v_t) + \bar u V_s(0\cdot v_t) 
  = uv_{s+t},
  \quad s,t > 0.
\end{align} 
Using this, we have the following: 
\begin{align}
  & Q_{t+s}(u\mathbf 1_E) 
    = e^{- \lambda (t+s)} ( 1-e^{-G^*(uv_{t+s})} ) 
  \leq e^{- \lambda(t+s)}( 1-e^{-G^*[V_s(uv_t)]} ) \\
  & = e^{-\lambda t}( 1-e^{-G^*(uv_t)} )
    = Q_t(u\mathbf 1_E)
  , \quad s,t > 0, u \in [0,1],
\end{align}
as required in this substep.
In particular, 

\emph{Substep 2.2.}
We show that $q$ is non-decreasing and concave on $[0,1]$.
In fact, from $G^*(\infty \mathbf 1_E) = \infty$ and $V_t(\infty \mathbf 1_E) = v_t$ we have that
\begin{align}
	Q_t(\mathbf 1_E) = e^{- \lambda t} ( 1-e^{-G^*v_t} )
  = e^{- \lambda t} e^{\lambda t}( 1-e^{-G^*(\infty\mathbf 1_E)} )
  = 1,
  \quad t\geq 0.
\end{align}
This says that $q(1) = 1$, and that $G^*v_t < \infty$ for each $t>0$. 
Therefore, from the condition that $G^*$ is monotone concave, we have that for each $t>0$, map $u \mapsto G^*(uv_t)$ is non-decreasing and concave on $[0,1]$.
From Lemma \ref{lem:ACE} we can verify that, for each $t> 0$, $u \mapsto Q_t(u \mathbf 1_E)$ is non-negative, non-decreasing and concave on $[0,1]$.
Finally, using Lemma \ref{lem:ACL} and Substep 2.1, we get the desired result in this substep. 

\emph{Substep 2.3.}
We show that $q(u) = u$ for each $u \in [0,1]$.
From Lemma \ref{lem:PSK}\eqref{lem:PSK2} and \ref{lem:PSK}\eqref{lem:PSK3} we have that
\begin{align}
	& e^{\lambda s}(\phi^{-1}v_t)(x) 
  = e^{\lambda s}\nu(v_{t})(1+o(\limsup_{t\to \infty}\sup_{x\in E})) 
   =\nu(v_{t+s}) (1+o(\limsup_{t\to \infty}\sup_{x\in E})) \\
  & = (\phi^{-1}v_{t+s})(x) (1+o(\limsup_{t\to \infty} \sup_{x\in E}))
    , \quad s\geq 0.
\end{align}
Therefore, for each $s\geq 0$ and $\epsilon >0$ there exists $t>0$ large enough such that
\begin{align}
	1-\epsilon\leq \frac{e^{\lambda s}v_t(x)}{v_{t+s}(x)} 
  \leq 1+\epsilon,
  \quad x\in E.
\end{align}
From this we can verify that for each $s\geq 0$ and $\epsilon >0$ there exists $t>0$ large enough such that
\begin{align}
	& Q_{t+s}[ (1-\epsilon)u\mathbf 1_E ]
   = e^{-\lambda(t+s)}( 1-e^{-G^*[(1-\epsilon)uv_{t+s}]} ) \\
  & \leq e^{-\lambda t} e^{-\lambda s}( 1- e^{-G^*(ue^{\lambda s}v_t)} )
    = e^{-\lambda s}Q_t(ue^{\lambda s} \mathbf 1_E),
    \quad u \in [0,\infty),
\end{align}
and similarly that
\begin{align}
	e^{-\lambda s}Q_t[ue^{\lambda s}\mathbf 1_E] 
  \leq Q_{t+s}[(1+\epsilon)u\mathbf 1_E].
\end{align}
Now taking $t\to \infty$ in the above two inequality, we get from substep 2.1 that, for each $s\geq 0, u\in (0,1), \epsilon > 0$ with $0 < (1 - \epsilon) u < (1+\epsilon)u < 1$, we have
 \begin{align}
   q((1-\epsilon)u)\leq e^{-\lambda s}q(u e^{\lambda s}) \leq q((1+\epsilon)u).
 \end{align}
 From substep 2.2 and Lemma \ref{lem:ACR} we have that $q$ is continuous on $(0,1]$. 
Taking $\epsilon \to 0$ and then $u \uparrow 1$, we get that 
\begin{align}
	1
  =q(1) 
  = e^{- \lambda s} q(e^{\lambda s}), 
  \quad s \geq 0. 
\end{align} 
This says that $q(u) = u$ on $(0,1]$.
Finally note that $q$ is non-negative and non-decreasing, we also have $q(0) = 0$.

% ***** Step 3
\emph{Step 3.} 
  We show that if $G^*$ is a $[0,\infty]$-valued monotone concave functional on $\mathcal B(E,[0,\infty])$ satisfying that $G^*(\infty \mathbf 1_E) = \infty$ and that 
  \begin{align}
    1 - e^{- G^* V_sf} = e^{s\lambda} (1- e^{- G^* f}),
    \quad s \geq 0, f \in \mathcal B(E, [0,\infty]),
  \end{align}
  then for each unbounded increasing time sequence $\mathbf t:= (t_n)_{n\in \mathbb N}$, we have $G^* = G^{\mathbf t}$.
  In fact, fixing an arbitrary unbounded increasing time sequence $\mathbf t=(t_n)_{n\in \mathbb N}$ and an arbitrary function $f\in \mathcal B(E,[0,\infty])$, we only have to proof that $G^* f = G^{\mathbf t}f$.
From the definition of $G^{\mathbf t}f$, we can chose a subsequence $\mathbf t'=(t'_n)_{n \in \mathbb N}$ of $\mathbf t$ such that $ 1 - e^{- G^{\mathbf t}f} =  ( 1 - e^{-\Gamma_{( t_n')} f} ) \cdot (1+ o(\limsup_{n\to \infty})). $
Therefore, from \eqref{eq:MY2} and  Lemma \ref{lem:PSK} we have
\begin{align}
  & 1 - e^{- G^{\mathbf t}f}
  = \frac{1 - e^{- \nu( V_{(t_n')}f)}}{1- e^{- \nu(v_{(t_n')})}}  (1+o(\limsup_{n\to \infty})) \\
  & = \frac{\nu (V_{(t_n')} f)}{\nu(v_{(t_n')})}(1+o(\limsup_{n\to \infty})) 
  =  \frac{V_{(t_n')}f(x)}{v_{(t_n')}(x)} ( 1 + o(\limsup_{n \to \infty} \sup_{x\in E})).
\end{align}
Fix an arbitrary $\epsilon > 0$. 
From above, there exists $n$ large enough such that
\begin{align}
  (1-\epsilon) (1 - e^{- G^{\mathbf t}f} )
  \leq \frac{V_{(t_n')}f(x)}{v_{(t'_n)}(x)}
  \leq (1+\epsilon) ( 1 - e^{- G^{\mathbf t}f} ) \wedge 1,
  \quad x\in E.
\end{align}
This implies that for $n$ large enough 
\begin{align}
  \label{eq:Y:Gs:1}
  & Q_{(t'_n)}[ (1-\epsilon) (1-e^{-G^{\mathbf t}f})\mathbf 1_E ]
    \leq Q_{(t'_n)}\Big( \frac{V_{(t'_n)}f}{v_{(t'_n)}} \Big) 
  \\ &  \leq Q_{(t'_n)}[ ( (1+\epsilon) (1-e^{-G^{\mathbf t}f}) \wedge 1) \mathbf 1_E ],
\end{align}
where $(Q_t)_{t\geq 0}$ is given by Step 2.
Note from the definition of $(Q_t)_{t\geq 0}$ and $G^*$, we always have that
\begin{align}
	Q_t \Big( \frac{V_tf}{v_t}  \Big) 
  = e^{- \lambda t}( 1 - e^{- G^*V_tf}  )
  = 1- e^{- G^* f}.
\end{align}	
Therefore, taking $n \to \infty$ in \eqref{eq:Y:Gs:1}, from Step 2 we get that
\begin{align}
	(1 - \epsilon) (1 - e^{- G^{\mathbf t}f})
  \leq 1 - e^{- G^* f} 
  \leq (1 + \epsilon) (1 - e^{- G^{\mathbf t} f})\wedge 1.
\end{align}
Taking $\epsilon \to 0$, we get the desired result.

% **** Step 4.
\emph{Step 4.}
From the step 1 and 3, using a sub-sub-sequence argument, we have that 
\begin{align}
  \label{eq:Y:1}
	Gf
  := \lim_{t\to \infty} \Gamma_tf, \quad f\in \mathcal B(E,[0,\infty])
\end{align}
is well-defined, and that $G$ is the unique $[0,\infty]$-valued monotone concave functional on $\mathcal B(E,[0,\infty])$ satisfying that $G(\infty \mathbf 1_E) = \infty$ and that
\begin{align}
  1 - e^{- G V_sf} = e^{s\lambda} (1- e^{- G f}),
  \quad s \geq 0, f \in \mathcal B(E, [0,\infty]).
\end{align}
In this step, we show that for each sequence of functions $(g_{n})_{n\in \mathbb N}$ in $\mathcal B(E, [0,\infty])$ such that $\lim_{n \to \infty} g_n = 0$ bounded pointwisely, we have $\lim_{n \to \infty} G g_n = 0$.
Taking $f = \infty$ in the above equation, we have that $1 - e^{- Gv_s} = e^{\lambda s}$ for each $s\geq 0$.

In fact, fixing such sequence $(g_{n})_{n\in \mathbb N}$, from Lemma \ref{lem:PSK}\ref{lem:PSKc}, there exists a constant $C > 0$ such that 
\begin{align}
	V_1 g_n(x) \leq P^\beta_1 g_n(x) \leq C \phi(x) \nu(g_n),
  \quad n \in \mathbb N, x\in E.
\end{align}
Fixing this $C>0$, from the bounded convergence theorem, we have $\lim_{n\to \infty}C \nu(g_n) =0$.
{\bf (Here, we should be able to show that map $t \mapsto \nu(v_t)$ is a strictly decreasing 1-1 map from $(0,\infty)$ to $(0,\infty)$.)}
Denote by $R:(0,\infty) \to (0,\infty)$ as the inverse of the map $t \mapsto \nu(v_t)$ on $(0,\infty)$, then $R$ is a strictly decreasing $1-1$ map from $(0,\infty)$ to $(0,\infty)$. 
Define $t_n := R(2C\nu(g_n))> 0$.
Then we have that $\lim_{n\to \infty} t_n = \infty$ and that 
\begin{align}
	V_1 g_n(x) \leq \frac{1}{2} \phi(x) \nu(v_{(t_n)}), 
\quad n \in \mathbb N, x\in E.
\end{align}
Now, taking $f = \infty$ and $t = t_n$ in Lemma \ref{lem:PSK}\eqref{lem:PSK3} we get that
\begin{align}
 (\phi^{-1} \cdot v_{(t_n)})(x) 
  = \nu(v_{(t_n)}) ( 1+ o(\limsup_{n\to \infty} \sup_{x\in E}) ).
\end{align}
In particular, we have that, for $n$ large enough,
\begin{align}
	\nu(v_{t_n}) \leq 2 \phi(x)^{-1} v_{(t_n)}(x), \quad x\in E.
\end{align}
Therefore, for $n$ large enough, we have that $V_1g_n \leq v_{(t_n)}$.
This implies that
\begin{align}
	& 1 - e^{- Gg_n}
  = e^{- \lambda} (1- e^{- GV_1g_n})
  \leq e^{- \lambda} (1- e^{- G v_{(t_n)}}) \\
  & = e^{- \lambda} e^{\lambda t_n},
  \quad \text{$n$ large enough.}
\end{align} 
Taking $n\to \infty$, we get the desired claim.

\emph{Final step.}
Now from \cite[Proposition 1.19]{Li2011MeasureValued} and step 4, we have that there exists a unique probability $\mathbf P$ on $\mathcal M_f(E)$ such that 
\begin{align}
  e^{-Gf}
  = \int_{\mathcal M_f(E)} e^{- \mu(f)} \mathbf P(d\mu)
  , \quad f\in C_b (E, [0,\infty)),
\end{align}
and that
\begin{align}
	\mathbb P_{\nu}(X_t \in \cdot | \|X_t\|>0 ) 
  \xrightarrow[t\to \infty]{law} \mathbf P(\cdot).
\end{align}
From Lemma \ref{lem:PSK3}.(\ref{lem:PSK1}) and (\ref{lem:PSK3}) we have that for each $\mu \in \mathcal M_f(E)\setminus \{0\}$ and $f \in \mathcal B(E,[0,\infty])$ that 
\begin{align}
	& \mu(V_tf) = \nu(V_tf) \mu\Big(\phi(1+o(\limsup_{t\to \infty}\sup_{x\in E})) \Big)
   = \nu(V_tf) \mu(\phi)(1+o(\limsup_{t\to \infty}))
  \\ & = o(\limsup_{t\to \infty}).
\end{align}
Therefore we have that for each $\mu \in \mathcal M_f(E)\setminus\{0\}$ and $f\in \mathcal B(E,[0,\infty])$, 
\begin{align}
  &\mathbb P_\mu[1 - e^{-X_t(f)}|\|X_t\|>0]
  = \frac{1 - e^{- \mu(V_tf)}}{1 - e^{-\mu(v_t)}}
  = \frac{\mu(V_tf)}{\mu(v_t)} (1+o(\limsup_{t\to \infty}))
  \\&= \frac{\nu(V_tf)}{\nu(v_t)}(1+o(\limsup_{t\to \infty})) 
  = \mathbb P_\nu [1 - e^{-X_t(f)}|\|X_t\|>0](1+o(\limsup_{t\to \infty}))
  \\ & = (1 - e^{-Gf}) (1+o(\limsup_{t\to \infty})).
\end{align}
Therefore we have that for each $\mu \in \mathcal M_f(E)\setminus\{0\}$,
\[
  \mathbb P_\mu(X_t \in \cdot | \|X_t\|>0)
  \xrightarrow[t\to \infty]{law} \mathbf P(\cdot).
  \qedhere
\]
\end{proof}

% **** Proof of Lemma lem:Y:Gs:Q
% ** (A)ppendix
\appendix
\section{Analytic facts}
% *** (C)oncave functions with extended values
\subsection{Concave functions and extended values}
% **** Extended values 
In this paper, we often work with the extended non-negative real number system $[0,\infty]$ which consists of the non-negative real line $[0,\infty)$ and an extra point $\infty$. 
We consider $[0,\infty]$ as the one point compactification of $[0,\infty)$; and therefore, it is a compact Hausdorff space.
Preserving the original order in $[0,\infty)$, define $x < \infty$ for each $x\in [0,\infty)$.
We also make the following conventions that 
\begin{enumerate}
\item
$x + \infty = \infty$ for each $x\in [0,\infty]$; 
\item
$x \cdot \infty = \infty$ for each $x\in (0,\infty]$;
\item
$0 \cdot \infty = 0$; $\frac{1}{\infty} = 0$; $\frac{1}{0} = \infty$.
\end{enumerate}

% **** Concave functions
We call $f$ a concave function on a convex subset $C$ of $\mathbb R$ iff $f$ is a $\mathbb R$-valued function on $C$ and that
\[
  f(rx+\bar r y) \geq r f(x) + \bar r f(y)
\]
for each $x,y \in C$, $r \in [0,1]$ and $\bar r = 1 - r$. 

Some known facts about the concave functions are listed below.

% **** Convex function is (C)ontinuous
\begin{lem}[{\cite[Corollary 6.3.3.]{Dudley2002Real}}]
  \label{lem:ACC}
	If $f$ is a concave function on a convex subset $C$ of $\mathbb R$, then $f$ is continuous on $C^o$.
 Furthermore, both left and right derivatives of $f$ exist and are finite on $C^o$.
\end{lem}

% **** non-decreasing convex function is continuous on the (R)ight boundary
\begin{lem}
  \label{lem:ACR}
	If $f$ is a non-decreasing concave function on $(a,b]$ where $a<b$ in $\mathbb R$, then $f$ is continuous on $(a,b]$.
\end{lem}

% **** Proof of Lem:ACR
\begin{proof}
From Lemma \ref{lem:ACC} we know that $f$ is continuous on $(a,b)$.
From the fact that $f$ is non-decreasing, we have $f(b-) = \lim_{x \to b} f(x) \leq f(b)$. 
So what is left is to proof that $f(b-) \geq f(b)$.
From the concave property of the function $f$ on $(a,b]$ we know that, for a fixed $c \in (a,b)$,
\[
f(rc + \bar r b) 
\geq r f(c) + \bar r f(b),
\quad r\in [0,1], \bar r = 1 - r.
\]
Taking $r\to 0$, we get that $f(b-)\geq f(b)$.
\end{proof}

% **** (L)iminf of convex functions is convex
\begin{lem}
	\label{lem:ACL}
  Suppose that $(f_n)_{n \in \mathbb N}$ is a sequence of non-negative concave functions on a convex subset $C$ of $\mathbb R$.
  Then so is $f:= \liminf_{n\to \infty} f_n$.
\end{lem}

% **** Proof of lem:ACL
\begin{proof}
  Since $(f_n)_{n \in \mathbb N}$ are non-negative, we know $f$ is also a non-negative function.
  From the fact that
\(
  f_n(rx+\bar r y) \geq r f_n(x) + \bar r f_n(y)
\)
for each $x,y\in C$, $r\in [0,1]$, $\bar r = 1 - r$ and $n \in \mathbb N$, we can verify that
\begin{align}
	& f(rx+ \bar r y) 
   \geq \liminf_{n\to \infty} (r f_n(x) + \bar r f_n(y))
   \\ & \geq r (\liminf_{n\to \infty} f_n(x)) + \bar r (\liminf_{n\to \infty} f_n(y)) 
    = rf(x) + \bar r f(y).
        \qedhere
\end{align}
\end{proof}

% **** Log-La(P)lac is concave
\begin{lem}
  \label{lem:ACP}
  Suppose that $\{Z; P\}$ is a $[0,\infty]$-valued random variable and that $P(Z < \infty)< \infty$. 
  Write $L(u):= - \log P[e^{- u Z}]$ with $u \in [0,\infty)$, then $L$ is a concave function on $[0,\infty)$.
\end{lem} 
% **** Proof of Lem:ACP
\begin{proof}
  Observe that since $P(Z < \infty)< \infty$, we have $L$ is a $[0,\infty)$-valued function on $[0,\infty)$.
	It can be verified that
  \begin{align}
    \frac{\partial L(u)}{\partial u} = P[e^{-u Z}]^{-1} P[Ze^{- u Z}]
    , \quad u > 0.
  \end{align}
  It can also be verified that
  \begin{align}
    & \frac{\partial^2 L(u)}{\partial u^2}
    = P[e^{-uZ}]^{-2}( \frac{\partial P[Ze^{-uZ}]}{\partial u} \cdot P[e^{-uZ}] - P[Ze^{-uZ}] \cdot \frac{\partial P[e^{-uZ}]}{\partial u}) \\
    & = - \frac{P[Z^2 e^{-uZ}]}{P[e^{-uZ}]} + \frac{P[Ze^{-uZ}]^2}{P[e^{-uZ}]^2} 
= Q^{(u)}[Z]^2 - Q^{(u)}[Z^2] 
      \leq 0
      , \quad u \in (0,\infty).
  \end{align}
  Here, for each $u$, the probability measure $Q^{(u)}$ is given by $dQ^{(u)}:= \frac{e^{-uZ}}{P[e^{-uZ}]} dP$; and in the last step, we used Jason's inequality.
  All these says that $L$ is concave on $(0,\infty)$.
  Finally, noticing that $L$ is continuous on $0$, we complete the proof.
\end{proof}
% **** (E)xponential linear transform 
\begin{lem}
	\label{lem:ACE}
  Suppose that $g$ is a concave function on some convex subset $C$ of $\mathbb R$, then so is $q:= 1- e^{-g}$.
\end{lem}

% **** Proof of lem:ACE
\begin{proof}
For each $u,v \in C$ and $r \in [0,1]$, writing $\bar r = 1-r$, we have
\begin{align}
	&q(ru+\bar r v) 
  = 1 - e^{- g(ru + \bar r v)}
  \geq 1 - e^{- ( r g(u) + \bar r g(v))} \\
  & \geq r(1- e^{- g(u)}) + \bar r (1 - e^{- g(v)})
    = rq(u) + \bar r q(v).
    \qedhere
\end{align}	
\end{proof}

% * Bibliography
\bibliographystyle{plain}
\bibliography{subyaglom.bib}
\end{document}
