%subyaglom8 Renming 2019-12-13
%subyaglom7 Renming 2019-12-10
%subyaglom6 Zhenyao 2019-12-07
%subyaglom5 Renming 2019-11-21
%subyaglom4 Yanxia 2019-11-16
%subyaglom3 Zhenyao 2019-11-13
%subyaglom2 Renming 2019-10-31
\documentclass[12pt,a4paper]{amsart}
\setlength{\textwidth}{\paperwidth}
\addtolength{\textwidth}{-2in}
\calclayout
\numberwithin{equation}{section}
\allowdisplaybreaks
\theoremstyle{plain}
\newtheorem{thm}{Theorem}[section]
\newtheorem{lem}[thm]{Lemma}
\newtheorem{prop}[thm]{Proposition}
\newtheorem{cor}[thm]{Corollaray}
\newtheorem{fact}[thm]{Fact}
\newtheorem{claim}[thm]{Claim}
\theoremstyle{definition}
\newtheorem*{ack*}{Acknowledgment}
\theoremstyle{remark}
\newtheorem{exa}[thm]{Example}
\usepackage{amssymb}
\usepackage{mathtools}
\mathtoolsset{showonlyrefs}
\usepackage{mathrsfs}
\usepackage{comment}
\everymath{\displaystyle}
\usepackage{hyperref}
\usepackage[inline]{showlabels}
\begin{document}
\title{Quasi-stationary distributions for subcritical superprocesses}
\author[R. Liu, Y.-X. Ren, R. Song, and Z. Sun]{Rongli Liu, Yan-Xia Ren, Renming Song, and Zhenyao Sun}
\address{Yan-Xia Ren\\ LMAM School of Mathematical Sciences \& Center for
Statistical Science\\ Peking University\\ Beijing 100871\\ P. R. China}
\email{yxren@math.pku.edu.cn}
\thanks{The research of Yan-Xia Ren is supported in part by NSFC (Grant Nos. 11671017 and 11731009)  and LMEQF.}
\address{Rongli Liu\\ Mathematics and Applied Mathematics\\ Beijing jiaotong University\\ Beijing 100044\\ P. R. China}
\email{rlliu@bjtu.edu.cn}
\thanks{The research of Rongli Liu is supported in part by NSFC (Grant No. 11301261), and the Fundamental Research Funds for the Central Universities (Grant No.  2017RC007)}
\address{Renming Song\\ Department of Mathematics\\ University of Illinois at Urbana-Champaign \\ Urbana \\ IL 61801\\ USA}
\email{rsong@illinois.edu}
\address{Zhenyao Sun\\ Faculty of Industrial Engineering and Management \\ Technion, Isreal Institute of Technology \\ Haifa 3200003\\ Isreal}
\email{zhenyao.sun@gmail.com}
\begin{abstract}
	Suppose that $X$ is a subcritical superprocess on a locally compact separable metric space. 
	Under some asymptotic conditions on the mean semigroup and some regularity conditions on the cumulant semigroup of $X$, we identify the Yaglom limit and all the quasi-stationary distributions of $X$.
\end{abstract}
\maketitle
\section{Introduction}
\subsection{Background}\label{sec:BGD}
	Suppose 
	%$(Z_n, n\ge 1)$ 
	$\{(Z_n)_{n\in \mathbb N}; P\}$
	is a Galton-Watson branching process with offspring distribution 
	%$\{p_n\}$. 
	$(p_n)_{n\in \mathbb N}$.
	Let $m:=\sum^{\infty}_{n=1}np_n$ be the mean number of children per particle. 
	It is well known that when $m<1$ the extinction probability 
	%$q:=\lim_{n\rightarrow\infty}P\left(Z_n=0\right)$ is equal to $1$. 
	\[
	P(Z_n = 0 \text{ for some $n \in \mathbb N$} | Z_0 = z_0) = 1, \quad z_0 \in \mathbb Z_+.
	\] 
	That is to say the process $Z$ is extinct in finite time almost surely. Let $\zeta:=\inf\{n\geq 0: Z_n=0\}$ be the extinction time of $Z$. 
	%The Yaglom distribution $\nu$ is the limit probability that for any integer 
	We say $\nu$ is the Yaglom limit of $Z$ if it is the limit probability on $\mathbb N$ that for any integer
	%$k>0$ and Borel set $A$,
	$z_0>0$ and Borel set $A$,
\[
	%\lim_{t\rightarrow\infty}\mathbb P_x\left(Z_n\in A\middle|\zeta>n \right)=\nu(A).
	\lim_{n\rightarrow\infty} P\left(Z_n\in A\middle|\zeta>n , Z_0 = z_0\right)=\nu(A).
\]
	%Yaglom \cite{Yaglom47} showed that when $m < 1$ and $Z_1$ has a finite second moment, the conditional distribution  of $Z_n$ given $Z_n>0$ converges to a proper probability distribution as $n\to\infty$. 
	Yaglom \cite{Yaglom47} showed that when $m < 1$ and the offspring distribution has a finite second moment, the conditional distribution  of $Z_n$ given $Z_n>0$ converges to a proper probability distribution as $n\to\infty$. 
	This was generalized to some cases without the second moment assumption by Heathcote, Seneta and Vere-Jones \cite{Heathcote}. See also Athreya and Ney \cite[pages 64-65]{AthreyaNey1972Branching}.

	Suppose $\{(Z_t)_{t \geq 0}; (P_x)_{x\geq 0}\}$  is a  continuous-state branching process taking values in $[0,\infty)$, where $0$ is an absorbing state.  Let $\zeta:=\inf\{t\geq 0: Z_t=0\}$ be the extinction time of $Z$.
	The Yaglom distribution $\nu$ is the limiting probability on $(0,\infty)$ such that for any $x>0$ and Borel set $A$,
\[
	\lim_{t\rightarrow\infty}\mathbb P_x(Z_t\in A\big|\zeta>t)=\nu(A).
\]
	The above conditional convergence can be found in \cite[Theorem 4.3]{Li00} where it is also generalized to conditioning of the type $\{\zeta>t+r\}$ for any finite $r>0$ instead of $\{\zeta>t\}$.

	Asmussen and Hering \cite{AH} studied limit behaviour of subcritical branching Markov processes, in which each particle lives for exponential time, then give birth to random number of particles, and particles move as independent Markov processes in between branching times and it is assumed that  the life times, reproduction of different individuals is independent.
	Asmussen and Hering proved that the Yaglom limit for subcritical  branching Markov processes exists under some conditions on the mean semigroup, see Theorem 1.6 of Chapter 4 in \cite{AH}.

	In this paper we will prove the Yaglom limt for some subcritical superprocesses exists. In the next subsection we introduce the model we are going to investigate. 
	
	\cite{HoppeSeneta1978Analytical} studied the Quasi-stationary distribution of branching processes.
	\cite{Hoppe1975Stationary} studied the regular variation property of the Yaglom limit for Multitype branching processes.
	\cite{Seneta1974Regularly} studied the regular variation property of the Yaglom limit of branching processes.
	\cite{JoffeSpitzer1967On} studied the yaglom theorem for multitype branching processes.
	\cite{Hering1977Subcritical} studied the Yaglom limit theorem for Branching diffusions.
	\cite{Lambert2007Quasi-stationary} studied the Yaglom Theorem and QSD for CSBP.
	
	Those papers considered superprocesses with conditioning on survival with different settings: \cite{Etheridge2003A-decomposition}, \cite{Evans1992The-entrance}, \cite{EvansPerkins1990Measure-valued},
	\cite{Serlet1996Occupation}, \cite{LiuRen2009Some}, \cite{RenSongZhang2015Limit}, \cite{RenSongSun2019Spine}, \cite{RenSongSun2018Limit} and \cite{ChampagnatRaelly2008Limit}.
	
	\cite{LyonsPemantlePeres1995Conceptual} gives a probabilistic proof to the existence of the Yaglom limit for branching processes.
	\cite{SenetaVere-Jones1968On} studied Yaglom type limit for Jirina processes.
	
	Recently, \cite{Labbe2013Quasi-stationary} considered the Quasi-stationary distribution of CSBP conditioned on non-explosive.
	\cite{Maillard2018The} considered the $\lambda$-invariant measure for the subcritical Galton-Watson trees which gives the new characterization for the Quasi-stationary distributions.

\subsection{Main result} \label{sec:super}

	We first recall some basics about superprocesses.
	Let $E$ be a locally compact separable metric space.
	%Suppose that $\partial$ is a separate point not contained in $E$.
	%We will use  $E_\partial$ to denote $E\cup\{\partial\} $.
	Let $E_\partial = E \cup \{\partial\}$ be the one point compactification of $E$.
	%We assume that  $\xi= \{(\xi_t)_{t\ge0}; (\Pi_x)_{x\in E}\}$ is a Hunt process on $E$ and $\zeta:=\inf\{t>0: \xi_t=\partial\}$ is the lifetime of $\xi$.
	Assume that $\xi = \{(\xi_t)_{t\ge0}; (\Pi_x)_{x\in E}\}$ is a $E_\partial$-valued Borel right process with $\partial$ as a absorbing state.
	Denote by $\zeta:=\inf\{t>0: \xi_t=\partial\}$ its life time.
	Let $\psi$ be a function on $E \times [0,\infty)$ given by
\begin{align}
	\psi(x,z)
	= -\beta(x) z + \sigma(x)^2 z^2 + \int_{(0,\infty)} (e^{-zu} -1 + zu) \pi(x,du),
	\quad x\in E, z\geq 0
\end{align}
	where $\beta, \sigma \in \mathcal B_b(E,\mathbb R)$ and $(u \wedge u^2) \pi(x,du)$ is a bounded kernel from $E$ to $(0,\infty)$.
	Let $\mathcal M_f(E)$ denote the space of all finite Borel measures on $E$ equipped with topology of weak convergence.
	%For any $f \in \mathcal B_b(E, [0,\infty))$, there is a unique locally bounded positive solution $(t,x)\mapsto V_tf(x)$ to the equation
	For any $f \in \mathcal B_b(E, [0,\infty))$, there is a unique locally bounded non-negative map on $(t,x)\mapsto V_tf(x)$ on $[0,\infty) \times E$ such that
\begin{equation} \label{eq:BGD.1}
	% V_tf(x) + \int_0^t P_{s} \psi(\cdot, V_{t-s}f(\cdot)) (x) ~ ds
	% = P_tf(x), \quad t\geq 0, x\in E.
	V_tf(x) + \Pi_x \left[ \int_0^{t\wedge \zeta} \psi \left(\xi_s, V_{t-s}f(\xi_s)\right) ds\right]
	= \Pi_x\left[ f(\xi_t) \mathbf 1_{t < \zeta}\right], \quad t\geq 0, x\in E.
\end{equation}
	Here, locally boundedness of $(t,x) \mapsto V_tf(x)$ means that $\sup_{0\leq t\leq T, x\in E} V_tf(x)< \infty$ for $T >0$.
	Moreover, there exists an $\mathcal M_f(E)$-valued Borel right process $X =\{(X_t)_{t\geq 0}; (\mathbb P_\mu)_{\mu \in \mathcal M_f(E)}\}$ such that
\begin{equation}
	\mathbb P_\mu[e^{- X_t(f)}]
	= e^{- \mu(V_tf)},
	\quad t\geq 0,~\mu \in \mathcal M_f(E), f \in \mathcal B_b(E,[0,\infty)).
\end{equation}
	We call $X$ a $(\xi, \psi)$-superprocess.
	See \cite{Li2011MeasureValued} for more details.

	Let us give some assumptions about the mean of the superprocess $X$. 
	Let $(P_t^\beta)_{t\geq 0}$ be the semigroup of operators on $\mathcal B_b(E,\mathbb R)$ given by
\begin{align}
	P_t^\beta f(x)
	:= \Pi_x\left[e^{\int_0^t \beta(\xi_r)dr }f(\xi_t) 1_{t < \zeta}\right],
	\quad f\in \mathcal B_b(E,\mathbb R), t\geq 0, x\in E.
\end{align}
	It is well-known (see \cite[Proposition 2.27]{Li2011MeasureValued}) that
\begin{equation} \label{Fact:M!}
	\mathbb P_\mu[X_t(f)] = \mu (P_t^\beta f),
	\quad \mu \in \mathcal M_f(E), t\geq 0, f \in \mathcal B_b(E,\mathbb R).
\end{equation}	
	In this paper, we will always assume that there exist a $\lambda<0$, a function $\phi \in \mathcal B_b(E,(0,\infty))$ and a probability measure $\nu$ with full support on $E$ such that for each $t\geq 0$, $P_t^\beta \phi = e^{\lambda t}\phi$, $\nu P_t^\beta = e^{\lambda t} \nu$ and $\nu(\phi) = 1$.
	The assumption of $\lambda<0$ says that the mean of $t\mapsto X_t(\phi)$ decay exponentially with rate $\lambda$, and in this case the superprocess $X$ is called subcritical.
	Denote by $L_1^+(\nu)$ the collection of 
	%non-negative functions on $E$ which are integrable with respect to the measure $\nu$.
	non-negative Borel functions on $E$ which are integrable with respect to the measure $\nu$.
	We will further assume the following two statements are true:
\begin{align}
\label{asp:H2!} \tag{H1}
&\begin{minipage}{0.8\textwidth}
	For all $t>0$, $x\in E$, and $f\in L_1^+(\nu)$, it holds that
\end{minipage}
\\
&\begin{minipage}{0.8\textwidth}
	\[ P_t^\beta f(x) = e^{\lambda t} \phi(x) \nu(f) (1+ C^{\eqref{asp:H2!}}_{t,x,f})\]
	for some real $C^{\eqref{asp:H2!}}_{t,x,f}$ with \[\sup_{x\in E, f\in L_1^+(\nu)} |C^{\eqref{asp:H2!}}_{t,x,f}| < \infty\] and \[\lim_{t\to \infty} \sup_{x\in E, f\in L_1^+(\nu)} |C^{\eqref{asp:H2!}}_{t,x,f}| = 0.\]
\end{minipage}
\end{align}
	and
\begin{equation}
\label{asp:H4!} \tag{H2}
\begin{minipage}{0.8\textwidth}
	There exists a $T^{\eqref{asp:H4!}}\geq 0$ such that $\mathbb P_\nu(\|X_t\| = 0)>0$ for all $t> T^{\eqref{asp:H4!}}$.
\end{minipage}
\end{equation}
	%Note that $L_1^+(\nu)$ in \eqref{asp:H2!} can be replaced by the collections of all non-negative functions $f$ with $\|f\|_{L_1(\nu)} = 1$.
	Note that $L_1^+(\nu)$ in \eqref{asp:H2!} can be replaced by the collections of all non-negative Borel functions $f$ with $\nu(f) = 1$.
	In fact, for any $f\in L_1^+(\nu)$ and $k \in (0,\infty)$, it is easy to see that $C^{\eqref{asp:H2!}}_{t,x,f} = C^{\eqref{asp:H2!}}_{t,x,kf}$.
	We  mention here that the constants in this paper might depend on the underlying process $\xi$ and the branching mechanism $\psi$. But since $\xi$ and $\psi$ are fixed, dependence on them will not be explicitly specified.

	Let us now give our first result. Since $\phi$ is strictly positive, we have
\begin{equation}
	\mathbb P_\mu[X_t(\phi)]
	\overset{\text{\eqref{Fact:M!}}}= \mu(P_t^\beta \phi)
	%=e^{\lambda t}\mu(\phi)>0.
	=e^{\lambda t}\mu(\phi)>0, \quad t\geq 0, \mu \in \mathcal M_f(E)\setminus\{0\}.
\end{equation}
	Thus,
\begin{equation}  \label{lem:Nd!}
		\mathbb P_\mu(\|X_t\| > 0) > 0,\quad t\geq 0,\mu \in \mathcal M_f(E)\setminus \{0\}.
\end{equation}
	Hence we can talk about the superprocess $X$ conditioned on survival up to time $t$.
\begin{thm} \label{Theorem:Y:H1:H2:H3:H4}
	Suppose that \eqref{asp:H2!} and \eqref{asp:H4!} hold. 
	Then there exists a probability measure $\mathbf Q_\lambda$ on $\mathcal M_f(E)$ such that
\begin{align}
 	\mathbb P_\mu \left(X_t \in \cdot \middle| \|X_t\| > 0 \right)
 	\xrightarrow[t\to \infty]{d} \mathbf Q_\lambda(\cdot),
 	\quad \mu \in \mathcal M_f(E)\setminus \{0\}.
\end{align}
\end{thm}

	Let us now introduce the concept of quasi-limiting distribution (QLD) and quasi-stationary distribution (QSD) for our superprocess $X$. 
	For any probability measure $\mathbf P$ on $\mathcal M_f(E)$, define $(\mathbf P\mathbb P)[\cdot] := \int_{\mathcal M_f(E)} \mathbb P_\mu[\cdot] \mathbf P(d\mu)$.
	We say a probability measure $\mathbf Q$ on $\mathcal M_f(E)$ is a QLD for $X$, if there exists a probability measure $\mathbf P$ on $\mathcal M_f(E)$ such that
	\[
	(\mathbf P\mathbb P)\left(X_t \in B \middle| \|X_t\|>0\right) \xrightarrow[t\to \infty]{} \mathbf Q(B), \quad B\in \mathcal B(\mathcal M_f(E)). 
	\]
	We say a probability measure $\mathbf Q$ on $\mathcal M_f(E)$ is a QSD of $X$, if 
	\[
	(\mathbf Q \mathbb P) \left( X_t \in B \middle | \|X_t\|>0 \right) = \mathbf Q(B), \quad t\geq 0, B \in \mathcal B(\mathcal M_f(E)).
	\]
	Standard theory says that the QLDs and the QSDs for a Markov process taking values in $[0,\infty)$ with $0$ as a absorbing state are equivalent, see \cite[Proposition 1]{MeleardVillemonais2012Quasi-stationary}. 
	Since $E$ is locally compact separable metric space, we know $\mathcal M_f(E)$ is a Polish space, see \cite[Lemma 4.3]{Kallenberg2017Random}.
	Thus $\mathcal M_f(E)$ is Borel isomorphism to $[0,\infty)$, see \cite[Theorem A.1.6]{Kallenberg2002Foundations}.
	And therefore we can apply results in \cite{MeleardVillemonais2012Quasi-stationary} to our $\mathcal M_f(E)$-valued superprocess $X$.
	In particular, according to \cite[Proposition 1]{MeleardVillemonais2012Quasi-stationary}, 
\begin{equation} \label{eq:S.1}
\begin{minipage}{0.9\textwidth}
		a probability measure $\mathbf Q$ on $\mathcal M_f(E)$ is a QLD for $X$ if and only if it is a QSD for $X$.
\end{minipage}
\end{equation}
	According to \cite[Proposition 2]{MeleardVillemonais2012Quasi-stationary}, 
\begin{equation} \label{eq:S.2}
\begin{minipage}{0.9\textwidth}
		if a probability measure $\mathbf Q$ on $\mathcal M_f(E)$ is a QSD of $X$, then there exists an $r\in (-\infty,0)$ such that $(\mathbf Q\mathbb P)(\|X_t\|>0) = e^{rt}$ for each $t\geq 0$.
		In this case, we call $r$ the mass decay rate of $\mathbf Q$.
\end{minipage}
\end{equation}
	
	Let us give two more technical assumptions before we give our next result. Denote by $C_b(E,[0,\infty))$ the collection of non-negative bounded continuous functions on $E$, then we assume that
\begin{equation}
\label{asp:HC} \tag{H3}
	\text{for each $t\geq 0$, $f \mapsto V_tf$ is a map from $C_b(E, [0,\infty))$ to $C_b(E,[0,\infty))$;}
\end{equation}  
and
\begin{equation}
\label{asp:HE} \tag{H4}
	\text{for each $t\geq 0$, $x \mapsto \mathbb P_{\delta_x}( \|X_t \|=0)$ is a continuous function on $E$.}
\end{equation} 
\begin{thm} \label{thm:QSD}
	Suppose that \eqref{asp:H2!}, \eqref{asp:H4!}, \eqref{asp:HC} and \eqref{asp:HE} hold.
	Then (1) for each $r \in [\lambda, 0)$, there exists a unique QSD $\mathbf Q_r$ of $X$ with mass decay rate $r$;
	(2) for each $r\in (-\infty, \lambda)$, there is no QSD for $X$ with mass decay rate $r$. 
\end{thm}

\subsection{Outline of the proof of Theorem \ref{Theorem:Y:H1:H2:H3:H4}}
	\label{subsec:OY}

	It is easy to see that the cumulate  semigroup $(V_t)_{t\geq 0}$ given by \eqref{eq:BGD.1} can be extended uniquely to a family of operators $(\overline V_t)_{t\geq 0}$
	on $\mathcal B(E,[0,\infty])$ such that for all $t\geq 0$, $f_n \uparrow f$ pointwisely in  $\mathcal B(E, [0,\infty])$ implies that $\overline V_tf_n \uparrow \overline V_tf$ pointwisely.
	Moreover, $(\overline V_t)_{t\geq 0}$ satisfies that
\begin{equation}
\begin{minipage}{0.8\textwidth}
	$\overline V_t f \leq \overline V_t g$ for all $t\geq 0$ and $f\leq g$ in $\mathcal B(E,[0,\infty])$;
\end{minipage}\label{Fact:BV!}
\end{equation}
\begin{equation}
\begin{minipage}{0.8\textwidth}
	$\overline V_{t+s} = \overline V_t \overline V_s$ for all $t, s\geq 0$;  and
\end{minipage} \label{eq:OY.0}
\end{equation}
\begin{equation}
\begin{minipage}{0.8\textwidth}
	$\mathbb P_\mu [e^{-X_t(f)}] = e^{- \mu(\overline V_tf)}$ for any $t\geq 0$, $\mu \in \mathcal M_f(E)$, and $f\in \mathcal B(E,[0,\infty])$.
\end{minipage} \label{eq:BGD.2}
\end{equation}
	With some abuse of notation, we still write $V_t = \overline V_t$ for $t\geq 0$, and call $(V_t)_{t\geq 0}$ the extended cumulant semigroup of the superprocess $X$.
	Define $v_t = V_t(\infty 1_E)$ for $t\geq 0$, then it holds that
\begin{equation} \label{eq:OY.1}
	\mathbb P_\mu (\|X_t\| = 0)
	= e^{- \mu (v_t)},
	\quad \mu \in \mathcal M_f(E), t\geq 0.
\end{equation}
	From this, we can verify
\begin{equation}\label{lem:sv2!}
	\text{$\mu(v_t) > 0$ for all $\mu \in \mathcal M_f(E)\setminus\{0\}$ and $t \geq 0$.}
\end{equation}
	In fact, if $\mu(v_t) = 0$, then by \eqref{eq:OY.1} we have $\mathbb P_\mu(\|X_t \| = 0) = 1$, which contradicts \eqref{lem:Nd!}.

	In the  proof of Theorem \ref{Theorem:Y:H1:H2:H3:H4}, we will use the following four propositions whose proofs are postponed to Subsections \ref{sec:Vf1}, \ref{sec:Vf2}, \ref{sec:G} and \ref{sec:GD} respectively.

\begin{prop} \label{prop:Vf1::H1:H2::Y}
	For any $f\in \mathcal B(E, [0,\infty]),~t > T^{\eqref{asp:H4!}}$ and $x\in E$, we have $V_tf(x) = C^{\ref{prop:Vf1::H1:H2::Y}}_{t,x,f} \phi(x)$ for some non-negative $C_{t,x,f}^{\ref{prop:Vf1::H1:H2::Y}}$ with $\lim_{t\to \infty} \sup_{x\in E} C^{\ref{prop:Vf1::H1:H2::Y}}_{t,x,f} = 0$.
	In particular, we have $\lim_{t\to \infty} \mu(V_tf)= 0 $ for all $\mu \in \mathcal M_f(E)$.
\end{prop}

\begin{prop} \label{prop:Vf2::H1:H2:H3:H4::Y}
	For any $f\in \mathcal B(E,[0,\infty]),~t>T^{\eqref{asp:H4!}}$ and $x\in E$, we have $V_tf(x) = \phi(x) \nu (V_tf) (1+C^{\ref{prop:Vf2::H1:H2:H3:H4::Y}}_{t,x,f}) $ for some real $C^{\ref{prop:Vf2::H1:H2:H3:H4::Y}}_{t,x,f}$ with $\lim_{t\to \infty} \sup_{x\in E} |C^{\ref{prop:Vf2::H1:H2:H3:H4::Y}}_{t,x,f}| = 0$.
\end{prop}

	For a probability measure $\mathbf P$ on $\mathcal M_f(E)$, we say $L$ is the log-Laplace functional of $\mathbf P$, if $Lf := - \log \int_{\mathcal M_f(E)}  e^{-\mu(f)} \mathbf P(d\mu)$ for $f\in \mathcal B(E,[0,\infty])$.
	We say $L$ is the log-Laplace functional of some finite random measure $Y$ if it is the log-Laplace functional of the distribution of $Y$.
	For each $t\geq 0$, denote by $\Gamma_t$ the log-Laplace transform of random measure $X_t$ under probability $\mathbf P_\nu(\cdot | \|X_t\|>0)$.

	We say a $[0,\infty]$-valued functional $A$ defined on $\mathcal B(E,[0,\infty])$ is monotone concave if
\begin{itemize}
\item
	$A$ is a monotone functional, i.e., $f\leq g$ in $\mathcal B(E,[0,\infty])$ implies $Af \leq Ag$; and
\item
	for any $f\in \mathcal B(E,[0,\infty])$ with $Af< \infty$, the function $u \mapsto A(uf)$ is concave on $[0,1]$.
\end{itemize}

\begin{prop} \label{prop:G::H1:H2:H3:H4::Y}
	The limit $Gf:= \lim_{t\to \infty} \Gamma_t f$ exists in $[0,\infty]$ for each $f\in \mathcal B(E,[0,\infty])$.
	Moreover, $G$ is the unique $[0,\infty]$-valued monotone concave functional on $\mathcal B(E,[0,\infty])$ such that $G(\infty  1_E) = \infty$ and that
\begin{align} \label{eq:G.0}
	1 - e^{- GV_s f}
	= e^{s\lambda} (1 - e^{-Gf}),
	\quad s\geq 0, f\in \mathcal B(E,[0,\infty]).
\end{align}
\end{prop}

\begin{prop} \label{prop::GD:H1:H2:H3:H4::Y}
	For any $(g_n)_{n\in \mathbb N} \subset \mathcal B(E,[0,\infty])$ such that $\lim_{n\to \infty} g_n = 0$ bounded pointwisely, we have $\lim_{n\to \infty} G g_n = 0$.
\end{prop}

\begin{proof}[Proof of Theorem \ref{Theorem:Y:H1:H2:H3:H4}]
	It follows from Lemma \ref{fact:WC}, Propositions \ref{prop:G::H1:H2:H3:H4::Y} and \ref{prop::GD:H1:H2:H3:H4::Y} that there exists a unique probability measure $\mathbf Q_\lambda$ on $\mathcal M_f(E)$ such that
\begin{align}\label{eq:Y.0}
 	\mathbb P_{\nu}(X_t \in \cdot | \|X_t\|>0 )
 	\xrightarrow[t\to \infty]{d} \mathbf Q_\lambda(\cdot).
\end{align}
	It follows from Proposition \ref{prop:Vf2::H1:H2:H3:H4::Y} that, for each $f \in \mathcal B(E,[0,\infty])$, there exists $T^{\ref{prop:Vf2::H1:H2:H3:H4::Y}}_f>0$ such that $\sup_{x\in E} |C^{\ref{prop:Vf2::H1:H2:H3:H4::Y}}_{t,x,f}|< \infty$ for all $t>T^{\ref{prop:Vf2::H1:H2:H3:H4::Y}}_f$.
	Thus for all $f \in \mathcal B(E,[0,\infty])$, $t>T^{\ref{prop:Vf2::H1:H2:H3:H4::Y}}_f$ and $\mu \in \mathcal M_f\setminus\{0\}$, we have
\begin{align}
	\mu(V_tf) 
	&= \int_E  \phi(x) \nu (V_tf) (1+C^{\ref{prop:Vf2::H1:H2:H3:H4::Y}}_{t,x,f})\mu(dx)
	\\ \label{eq:Y.1} & = \nu(V_tf) \mu(\phi)(1+ C^{\eqref{eq:Y.1}}_{\mu,t,f})
\end{align}
	for some real $C^{\eqref{eq:Y.1}}_{\mu,t,f}$ with $\lim_{t\to \infty} C^{\eqref{eq:Y.1}}_{\mu,t,f} = 0$.
	Therefore, for all $f\in \mathcal B(E,[0,\infty])$, $t > T_f^{\ref{prop:Vf2::H1:H2:H3:H4::Y}}$ and $\mu \in \mathcal M_f \setminus\{0\}$,
\begin{align}
	&\mathbb P_\mu \left[1 - e^{-X_t(f)} \middle|\|X_t\|>0\right]
	= \frac{\mathbb P_\mu [ 1 - e^{- X_t(f)}]} {\mathbb P_\mu (\|X_t\| > 0) }
	\overset{\text{\eqref{eq:BGD.2}}}= \frac{1 - e^{- \mu(V_tf)}} { \mathbb P_\mu(\|X_t\| > 0)}
	\\&\overset{\eqref{eq:OY.1}}= \frac{1 - e^{- \mu(V_tf)}} {1 - e^{-\mu(v_t)}}
	\\\label{eq:Y.1.5}& = \frac{ \mu(V_t f) }{ \mu(v_t) } (1+C^{\eqref{eq:Y.1.5}}_{\mu,t,f})
	\end{align}
	for some real $C^{\eqref{eq:Y.1.5}}_{\mu,t,f}$ with $\lim_{t\to \infty} |C^{\eqref{eq:Y.1.5}}_{\mu,t,f}| = 0$. 
	Here in the last equality we used \eqref{lem:sv2!}, Proposition \ref{prop:Vf1::H1:H2::Y} and the fact that $(1-e^{-x})/x \xrightarrow[x\to 0]{}1$. 
	Thus, for each $\mu \in \mathcal M_f(E)\setminus \{0\}$ and $f\in C_b(E,[0,\infty))$, we have
\begin{align}
&\mathbb P_\mu \left[1 - e^{-X_t(f)} \middle|\|X_t\|>0\right]
	 \overset{\text{\eqref{eq:Y.1} \& \eqref{eq:Y.1.5}}}= \frac{ \nu(V_tf) }{ \nu(v_t) } \frac{1+C^{\eqref{eq:Y.1}}_{\mu,t,f}}{1+C^{\eqref{eq:Y.1}}_{\mu, t,\infty \mathbf 1_E}}(1+ C^{\eqref{eq:Y.1.5}}_{\mu,t,f})
	\\& \overset{\text{\eqref{eq:Y.1.5}}}= \mathbb P_\nu \left[1 - e^{-X_t(f)} \middle| \|X_t\|>0\right](1+C^{\eqref{eq:Y.1.5}}_{\nu, t,f})^{-1}  \frac{1+C^{\eqref{eq:Y.1}}_{\mu,t,f}}{1+C^{\eqref{eq:Y.1}}_{\mu,  t,\infty \mathbf 1_E}}(1+ C^{\eqref{eq:Y.1.5}}_{\mu,t,f})
	\\&\xrightarrow[t\to \infty]{} \int_{\mathcal M_f(E)}(1-e^{-w(f)}) \mathbf Q_\lambda(dw),
\end{align}
	where in the last line above, we used \eqref{eq:Y.0}. 
	Therefore, according to \cite[Theorem 1.18]{Li2011MeasureValued},
	\[\mathbb P_\mu\left(X_t \in \cdot \middle| \|X_t\|>0\right) \xrightarrow[t\to \infty]{d} \mathbf Q_\lambda(\cdot). \qedhere\]
\end{proof}

\subsection{Outline of the proof of Theorem \ref{thm:QSD}}

	In this subsection, assuming that \eqref{asp:H2!}, \eqref{asp:H4!}, \eqref{asp:HC} and \eqref{asp:HE} hold, we give the proof of Theorem \ref{thm:QSD} using the following three Propositions \ref{prop:CQ}, \ref{prop:UC} and \ref{prop:EQ} whose proofs are postponed to Subsection \ref{sec:CQ}, \ref{sec:UC} and \ref{sec:EQ}, respectively.

\begin{prop} \label{prop:EQ}
	(1) The Yaglom limit $\mathbf Q_\lambda$ given by Theorem \ref{Theorem:Y:H1:H2:H3:H4} is a QSD of $X$ with mass decay rate of $\lambda$; and
	(2) for any $r \in (\lambda , \infty)$, there exists a probability measure $\mathbf Q_r$  on $\mathcal M_f(E)$ which is a QSD of $X$ with mass decay rate of $r$.
\end{prop}


\begin{prop} \label{prop:CQ}
	Suppose that $r \in (-\infty, 0)$ and that probability measure $\mathbf Q^*_{r}$ on $\mathcal M_f(E)$ is a QSD of the superprocess $X$ with mass decay rate of $r \in (-\infty, 0)$. Denote by $G^*_r$ the log-Laplace functional of $\mathbf Q^*_r$.
	Then we have that (1) $r \geq \lambda$; and (2) $G^*_r$ is a monotone concave functional on $\mathcal B(E,[0,\infty])$ with $G^*_r(\infty \mathbf 1_E) = \infty$ and that
	\[
	1 - e^{-G^*_r V_s f} = e^{r\lambda }(1- e^{- G^*_r f}), \quad s\geq 0, f\in \mathcal B(E,[0,\infty]).
	\] 
\end{prop}

\begin{prop} \label{prop:UC}
	Let $G$ be the unique functional on $\mathcal B(E,[0,\infty])$ given by Proposition \ref{prop:G::H1:H2:H3:H4::Y}.
	Let $r \in [\lambda, \infty)$. 
	If $G^*_r$ is a monotone concave functional on $\mathcal B(E,[0,\infty])$ such that
	\[
	1 - e^{-G^*_r V_s f} = e^{r\lambda }(1- e^{- G^*_r f}), \quad s\geq 0, f\in \mathcal B(E,[0,\infty]).
	\]
	Then $1 - e^{-G^*_rf} = (1 - e^{- G f})^{r/\lambda}$ for any $f\in \mathcal B(E,[0,\infty])$.  
\end{prop}

\begin{proof}[Proof of Theorem \ref{thm:QSD}]
	The non-existence of QSD for $X$ with mass decay rate $r < \lambda$ is due to Proposition \ref{prop:CQ} (1).
	The existence of QSD for $X$ with mass decay rate $r \in [\lambda,0)$ is due to Proposition \ref{prop:EQ}.
	The uniqueness of QSD for $X$ with mass decay rate $r\in [\lambda , 0)$ is due to Propositions \ref{prop:CQ}, \ref{prop:UC} and \cite[Theorem 1.17]{Li2011MeasureValued}.
\end{proof}

\section{Proofs of Propositions \ref{prop:Vf1::H1:H2::Y}--\ref{prop::GD:H1:H2:H3:H4::Y}}
\subsection{Proof of Proposition \ref{prop:Vf1::H1:H2::Y}} \label{sec:Vf1}
	Define a function
\[
	\psi_0(x,z) = \psi(x,z)+ \beta(x) z, \quad x\in E, z\in [0,\infty),
\]	
	and an operator $\Psi_0: \mathcal B(E, [0,\infty]) \to \mathcal B(E,[0,\infty])$ by 
\begin{align}
	\Psi_0 f(x)
	= \lim_{n\to \infty} \psi_0(x,f(x) \wedge n),
	\quad f\in \mathcal B(E,[0,\infty], x\in E.
\end{align}
	Then it can be verified from \cite[Theorem 2.23]{Li2011MeasureValued} and monotonicity that
\begin{align}\label{eq:Vf1.1}
	V_s f + \int_0^s P_{s-u}^\beta \Psi_0 V_{u} f ~du
	= P_s^\beta f,
	\quad f\in \mathcal B(E,[0,\infty]), s\geq 0.
\end{align}

	The following fact will be used repeatedly:
\begin{equation} \label{lem:nV::H2::Vf1}
	\{V_tf:t> T^{\eqref{asp:H4!}}, f\in \mathcal B(E, [0,\infty])\}\subset L_1^+(\nu).
\end{equation}
	To see this, note from \eqref{Fact:BV!}, \eqref{eq:OY.1} and \eqref{asp:H4!} that for  all $t> T^{\eqref{asp:H4!}}$ and $f\in \mathcal B(E,[0,\infty])$, we have $\nu(V_t f) \leq \nu(v_t)   = - \log \mathbb P_\nu (\|X_t\| = 0)  < \infty. $

\begin{proof}[{Proof of Proposition \ref{prop:Vf1::H1:H2::Y}}]
	Note that for all $s>0$ and $\epsilon>0$,
\begin{align}
	& V_{s+\epsilon +T^{\eqref{asp:H4!}}} f (x)
	\overset{\eqref{eq:OY.0}}= V_s V_{T^{\eqref{asp:H4!}}+\epsilon} f(x)
	\\&\leq P_s^\beta V_{T^{\eqref{asp:H4!}} + \epsilon} f(x),\quad\text{by \eqref{eq:Vf1.1}},
 	\\&= e^{\lambda s}\phi(x) \nu( V_{T^{\eqref{asp:H4!}} +\epsilon} f)  (1+ C^{\eqref{asp:H2!}}_{s,x,V_{T^{\eqref{asp:H4!}} + \epsilon} f}),
\label{eq:Vf1.2}
\end{align}
	where in the last equality above  we used \eqref{asp:H2!} and \eqref{lem:nV::H2::Vf1}.
	From this and the fact that $\lambda < 0$, we get the desired result.
\end{proof}

\subsection{Proof of Proposition \ref{prop:Vf2::H1:H2:H3:H4::Y}} \label{sec:Vf2}

	Another fact that will be used repeatedly is the following.
\begin{equation}
\begin{minipage}{0.9\textwidth}
	For any $f\in \mathcal B(E,[0,\infty])$, $\nu(f) = 0$ implies $\nu(V_tf)=0$ for all $t\ge 0$; and $\nu(f)>0$ implies $\nu(V_tf)>0$ for all $t\ge 0$.
\end{minipage}\label{lem:nVn!}
\end{equation}
	To see this, note by \eqref{Fact:M!} that $ \mathbb P_\nu[X_t(f)] = \nu (P_t^\beta f) = e^{\lambda t}\nu (f). $
	If $\nu(f) = 0$, then $X_t(f)=0, \mathbb P_\nu$-a.s., therefore $\nu(V_t f) = - \log \mathbb P_\nu[e^{-X_t(f)}] =0. $
	If $\nu(f) > 0$, then under $\mathbb P_\nu$, $X_t(f)$ is a random variable with positive mean.
	Therefore, $ \nu(V_tf) = - \log \mathbb P_\nu[e^{-X_t(f)}] >0$.

	Combining \eqref{lem:nVn!} with \eqref{eq:Vf1.2} we get that
\begin{equation}
\begin{minipage}{0.9\textwidth}
	for all $t>T^{\eqref{asp:H4!}},~x\in E$ and $f \in \mathcal B(E,[0,\infty])$ with $\nu(f) = 0$, we have $V_t f(x ) = 0$.
\end{minipage} \label{lem:nullVf::H1:H2!}
\end{equation}
Note  from \eqref{asp:H2!} and \eqref{lem:nV::H2::Vf1} that for all $s>0, t> T^{\eqref{asp:H4!}}, x\in E$ and $f\in \mathcal B(E,[0,\infty])$, we have
\begin{equation}
	P_s^\beta V_tf(x)  =e^{\lambda s} \phi(x)\nu(V_tf) (1+C^{\eqref{asp:H2!}}_{s,x,V_tf}) <\infty.
\label{lem:PV:H1:H2!}
\end{equation}

	In the proof of Proposition \ref{prop:Vf2::H1:H2:H3:H4::Y} we will use the following three lemmas whose proofs are postponed later.

\begin{lem} \label{prop:PVf:H1:H2:H3:H4}
	For all $s> 0,~t> T^{\eqref{asp:H4!}},~ x\in E$ and $f\in \mathcal B(E,[0,\infty])$, we have $P_s^\beta V_t f(x) = \phi(x) \nu(V_{t+s}f) (1+C^{\ref{prop:PVf:H1:H2:H3:H4}}_{s,t,x,f})$ for some real $C^{\ref{prop:PVf:H1:H2:H3:H4}}_{s,t,x,f}$ with $\lim_{s\to \infty} \varlimsup_{t\to \infty} \sup_{x\in E} |C^{\ref{prop:PVf:H1:H2:H3:H4}}_{s,t,x,f}| = 0$.
\end{lem}

\begin{lem} \label{prop:IVf:H1:H2:H3:H4}
	Define
\begin{align}
	I_{s,\epsilon} f
 	= \int_0^{s - \epsilon} P_{s - u}^\beta \Psi_0 V_u f ~du,
 	\quad f\in \mathcal B(E,[0,\infty]),0 < \epsilon < s < \infty.
\end{align}
	Then, for all $t> T^{\eqref{asp:H4!}},~0<\epsilon<s< \infty,~x\in E$ and $f\in \mathcal B(E,[0,\infty])$ with $\nu(f)>0$, we have $I_{s,\epsilon}V_t f(x) = \phi(x) \nu(V_{s+t} f) C^{\ref{prop:IVf:H1:H2:H3:H4}}_{t,\epsilon, s, x,f}$ for some non-negative $C^{\ref{prop:IVf:H1:H2:H3:H4}}_{t,\epsilon, s, x,f}$ with \[\lim_{t\to \infty} \sup_{x\in E} C^{\ref{prop:IVf:H1:H2:H3:H4}}_{t,\epsilon, s, x,f} = 0.\]
\end{lem}

\begin{lem} \label{prop:JVf:H1:H2:H3:H4}
	Define
\begin{align}
	J_{s,\epsilon} f
 	= \int_{s-\epsilon}^s P_{s-u}^\beta \Psi_0 V_u f ~du,
 	\quad f\in \mathcal B(E,[0,\infty]), 0< \epsilon < s< \infty.
\end{align}
	Then for all $t>T^{\eqref{asp:H4!}},~0<\epsilon<s< \infty,~x\in E$ and $f\in \mathcal B(E,[0,\infty])$ with $\nu(f)>0$, we have $ J_{s,\epsilon} V_tf(x) = \phi(x) \nu(V_{t+s}f) C^{\ref{prop:JVf:H1:H2:H3:H4}}_{t,\epsilon,s,x,f}$ for some non-negative $C^{\ref{prop:JVf:H1:H2:H3:H4}}_{t,\epsilon,s,x,f}$ with \[\lim_{\epsilon \to 0}\varlimsup_{t+s\to \infty} \sup_{x\in E} C^{\ref{prop:JVf:H1:H2:H3:H4}}_{t,\epsilon,s,x,f} =0.\]
\end{lem}

\begin{proof}[{Proof of Proposition \ref{prop:Vf2::H1:H2:H3:H4::Y}}]
Thanks to \eqref{lem:nVn!} and \eqref{lem:nullVf::H1:H2!}, we only need to consider the case that $\nu(f)>0$.
	In this case,
by Lemmas \ref{prop:PVf:H1:H2:H3:H4}, \ref{prop:IVf:H1:H2:H3:H4} and \ref{prop:JVf:H1:H2:H3:H4}.
we have for any $s>0$,
\begin{align}
	& V_{t+s}f (x)
	\overset{\eqref{eq:OY.0}}= V_s V_t f(x)
	\\&= P_s^\beta V_t f(x) - \int_0^s P^\beta_{s-u}\Psi_0 V_uV_t f(x) du, \quad\text{by \eqref{eq:Vf1.1} and \eqref{lem:PV:H1:H2!}},
	\\&= P_s^\beta V_t f(x) - I_{s,\epsilon} V_tf(x) - J_{s,\epsilon} V_t f(x), \quad \epsilon\in (0,s),
	\\\label{eq:Vf2.1} &=\phi(x)\nu(V_{t+s}f) \Big( 1+ C^{\ref{prop:PVf:H1:H2:H3:H4}}_{s,t,x,f} - C^{\ref{prop:IVf:H1:H2:H3:H4}}_{t,\epsilon, s, x,f}- C^{\ref{prop:JVf:H1:H2:H3:H4}}_{t,\epsilon,s,x,f}\Big).
\end{align}
	On the other hand, we have
\begin{align}\label{eq:Vf2.2}
	V_{t}f (x)
	= \phi(x) \nu(V_{t}f) (1+ C^{\ref{prop:Vf2::H1:H2:H3:H4::Y}}_{t,x,f})
	\quad \text{for some real $C^{\ref{prop:Vf2::H1:H2:H3:H4::Y}}_{t,x,f}$}.
\end{align}
	Combining \eqref{eq:Vf2.1} and \eqref{eq:Vf2.2}, we have for all $s>0$ and $\epsilon \in (0,s)$,
\[
	C^{\ref{prop:Vf2::H1:H2:H3:H4::Y}}_{t+s,x,f} = C^{\ref{prop:PVf:H1:H2:H3:H4}}_{s,t,x,f} - C^{\ref{prop:IVf:H1:H2:H3:H4}}_{t,\epsilon, s, x,f}- C^{\ref{prop:JVf:H1:H2:H3:H4}}_{t,\epsilon,s,x,f}.
\]
	 Using this and the fact that
\[
	\lim_{\epsilon \to 0}\varlimsup_{s\to \infty}\varlimsup_{t\to \infty}\sup_{x\in E}|C^{\ref{prop:PVf:H1:H2:H3:H4}}_{s,t,x,f} - C^{\ref{prop:IVf:H1:H2:H3:H4}}_{t,\epsilon, s, x,f}- C^{\ref{prop:JVf:H1:H2:H3:H4}}_{t,\epsilon,s,x,f}|=0,
\]
	it is easy to check that $\varlimsup_{s\to \infty}\varlimsup_{t\to \infty}\sup_{x\in E}|C^{\ref{prop:Vf2::H1:H2:H3:H4::Y}}_{t+s,x,f}|=0$.
	This implies
\[
	\lim_{t\to \infty} \sup_{x\in E}|C^{\ref{prop:Vf2::H1:H2:H3:H4::Y}}_{t,x,f}|=0.
	\qedhere
\]
\end{proof}

	Now we prove the three lemmas above.

\begin{proof}[Proof of Lemma \ref{prop:PVf:H1:H2:H3:H4}]
	
	Integrating  both sides of \eqref{eq:Vf1.1} with respect to $\nu$ and replacing $f$ by $V_t f$, we get that for all $t, s\geq 0$ and $f\in \mathcal B(E,[0,\infty])$,
\begin{align}\label{eq:nuP.1}
	e^{- \lambda (t+s)} \nu(V_{t+s}f) + \int_0^s e^{- \lambda (t+u)} \nu(\Psi_0 V_{t+u}f) du
	= e^{- \lambda t} \nu(V_t f).
\end{align}
	As a consequence of \eqref{eq:nuP.1}, we can get that for all $t>T^{\eqref{asp:H4!}}$, $s\geq 0$ and $f \in \mathcal B(E,[0,\infty])$ with $\nu(f)>0$,
\begin{equation}\label{Claim:nVI:H4}
	\frac{\nu(V_{t+s} f)} {\nu(V_t f)}
	= \exp\Big\{ \lambda s - \int_t^{t+s} \frac{\nu(\Psi_0 V_u f) }{\nu(V_u f)} ~du\Big\}.
\end{equation}
	In fact, first observe from \eqref{lem:nV::H2::Vf1} and \eqref{lem:nVn!} that both sides of \eqref{eq:nuP.1} are finite and positive if $t> T^{\eqref{asp:H4!}}$ and $\nu(f)>0$.
	Therefore the function $H: u\mapsto e^{-\lambda u}\nu(V_u f)$ is absolutely continuous on $(T^{\eqref{asp:H4!}},\infty)$ and
\begin{align}
	d H(u)
	= - e^{- \lambda u} \nu(\Psi_0 V_u f) du,
	\quad u\in (T^{\eqref{asp:H4!}},\infty),
\end{align}
	which implies that
\begin{align}
	d \log H(u)
	= - \frac{\nu(\Psi_0 V_u f )}{ \nu(V_u f)} du,
	\quad u \in (T^{\eqref{asp:H4!}},\infty).
\end{align}
	Now an elementary integration argument gives \eqref{Claim:nVI:H4}.
	
	Define an operator $\Psi_0'$ on $\mathcal B(E,[0,\infty])$ by
	\begin{align}
	\Psi_0' f(x)
	= \lim_{n\to \infty}\frac{\partial \psi_0}{ \partial z} (x, n\wedge f(x)),
	\quad x\in E, f\in \mathcal B(E,[0,\infty]).
	\end{align}
	We first claim that for all $t> T^{\eqref{asp:H4!}}, x\in E$ and $f\in \mathcal B(E,[0,\infty])$,
\begin{equation}
\begin{minipage}[c]{0.9\textwidth}
	$\Psi_0'V_tf(x) = C^{\eqref{lem:PsV:H1:H2:H4}}_{t,x,f}$ for some non-negative $C^{\eqref{lem:PsV:H1:H2:H4}}_{t,x,f}$ with $\varlimsup_{t\to \infty} \sup_{x\in E} C^{\eqref{lem:PsV:H1:H2:H4}}_{t,x,f} <\infty$.
\end{minipage} \label{lem:PsV:H1:H2:H4}
\end{equation}
	In fact, since
\begin{align}\label{e:derofpsi0-2}
	\frac{\partial \psi_0 }{ \partial z} (x,z)
	= 2\sigma (x)^2 z + \int_0^\infty (1 - e^{- rz}) r \pi(x,dr),
	\quad x\in E, z\geq 0,
\end{align}
	we have,
\begin{align}
	&\Psi_0' V_tf(x)
	\leq 2\sigma (x)^2 V_t f(x) + V_t f(x) \int_0^1 r^2 \pi(x,dr) + \int_1^\infty r \pi(x,dr)
	\\&\overset{\text{Proposition \ref{prop:Vf1::H1:H2::Y}}}= C^{\ref{prop:Vf1::H1:H2::Y}}_{t,x,f} \phi(x) \Big(2\sigma (x)^2 +\int_0^1 r^2 \pi(x,dr) \Big)+ \int_1^\infty r \pi(x,dr).
\end{align}
	Since $\phi$, $\sigma$ are bounded,  and $(r\wedge r^2)\pi(x,du)$ is a bounded kernel, \eqref{lem:PsV:H1:H2:H4} follows easily.
	
	We next claim that for all $t > T^{\eqref{asp:H4!}}$ and $f\in \mathcal B(E,[0,\infty])$,
\begin{equation}
\begin{minipage}{0.8\textwidth}
	$\nu(\Psi_0' V_t f) = C^{\eqref{Claim:nPPV:H1:H2:H4}}_{t,f}$ for some non-negative $C^{\eqref{Claim:nPPV:H1:H2:H4}}_{t,f}$ with $\lim_{t\to \infty} C^{\eqref{Claim:nPPV:H1:H2:H4}}_{t,f} = 0$.
\end{minipage} \label{Claim:nPPV:H1:H2:H4}
\end{equation}
	In fact, it follows from \eqref{e:derofpsi0-2} that, for any fixed $x\in E$, $z\mapsto \frac{\partial \psi_0}{\partial z} (x,z)$ is a non-negative, non-decreasing and continuous function on $[0,\infty)$ with $\frac{\partial \psi_0}{\partial z} (\cdot,0) \equiv 0$.
	Therefore for any $x\in E$, we have
\[
	\lim_{t\to \infty} \Psi_0' V_tf(x) =\lim_{t\to \infty} \frac{\partial \psi_0}{ \partial z}(x,V_tf(x)) \overset{\text{Proposition \ref{prop:Vf1::H1:H2::Y}}}{=} 0.
\]
	Using this, \eqref{lem:PsV:H1:H2:H4} and the bounded convergence theorem, we easily get  $\lim_{t\to \infty}\nu(\Psi_0' V_tf)  = 0. $

	Here is another claim that will be used below:
\begin{equation}
\begin{minipage}{0.9\textwidth}
	For all $t> T^{\eqref{asp:H4!}}, x\in E$ and $f\in \mathcal B(E,[0,\infty])$, it holds that $V_t f(x) = \phi(x) \nu(V_tf) C^{\eqref{Claim:VfO:H4:H3}}_{t,x,f}$ for some non-negative $C^{\eqref{Claim:VfO:H4:H3}}_{t,x,f}$ with $\varlimsup_{t\to \infty} \sup_{x\in E} C^{\eqref{Claim:VfO:H4:H3}}_{t,x,f} <\infty$.
\end{minipage} \label{Claim:VfO:H4:H3}
\end{equation}
	To see this, first note that \eqref{Claim:VfO:H4:H3} is trivial when $\nu(f) = 0$ thanks to \eqref{lem:nVn!} and \eqref{lem:nullVf::H1:H2!}. 
	Therefore, we only need to consider the case that $\nu(f)>0$.
	In this case, it follows from the elementary fact
\begin{align}\label{e:derofpsi0}
	\psi_0(x,z)
	\leq z \frac{\partial \psi_0}{\partial z}(x,z),
	\quad x\in E, z\geq 0
\end{align}
	that
\begin{align}
	&\nu(\Psi_0 V_tf)
	\leq \nu((V_tf)\cdot (\Psi_0' V_tf)) \leq \nu(V_tf) \sup_{y\in E} \Psi_0' V_tf(y)
	\overset{\eqref{lem:PsV:H1:H2:H4}}= \nu(V_tf) \sup_{y\in E} C^{\eqref{lem:PsV:H1:H2:H4}}_{t,y,f}.
\end{align}
	From \eqref{lem:nV::H2::Vf1}  we get that $\nu(V_tf) <\infty$.
	Thus for $t> T^{\eqref{asp:H4!}}$ and $f\in \mathcal B(E,[0,\infty])$,
\begin{equation}
\begin{minipage}{0.9\textwidth}
	$\nu(\Psi_0 V_t f)  = \nu(V_tf) C^{\eqref{eq:VfO.1}}_{t,f}$	for some non-negative $C^{\eqref{eq:VfO.1}}_{t,f} $ with $\varlimsup_{t\to \infty} C^{\eqref{eq:VfO.1}}_{t,f}  < \infty$.
\end{minipage}\label{eq:VfO.1}
\end{equation}
	Therefore, for any $s\geq 0$,
\begin{align}
	&  \frac{\nu(V_{t+s} f)} {\nu(V_t f)} \overset{\eqref{Claim:nVI:H4}}= \exp\Big\{ \lambda s - \int_t^{t+s} \frac{\nu(\Psi_0 V_u f) }{\nu(V_u f)} du\Big\}
	\\&\label{eq:VfO.2} \overset{\text{\eqref{eq:VfO.1}}}= \exp\Big\{ \lambda s - \int_t^{t+s} C^{\eqref{eq:VfO.1}}_{u,f} ~du\Big\}.
\end{align}
	Now note that
\begin{align}
	&V_{t}f(x) \overset{\eqref{Fact:BV!}}= V_{\epsilon} V_{t-\epsilon} f , \quad 0<\epsilon < t- T^{\eqref{asp:H4!}} ,
	\\&\leq P_\epsilon^\beta V_{t-\epsilon} f(x),\quad\text{by \eqref{eq:Vf1.1}},
	\\& \overset{\text{\eqref{asp:H2!}}}= \phi(x) \nu(V_{t-\epsilon}f) e^{\lambda  \epsilon} (1+C_{\epsilon,x, V_{t-\epsilon} f}^{\eqref{asp:H2!}} )
	\\& \label{eq:VfO.3}\overset{\text{\eqref{eq:VfO.2}}}= \phi(x)\nu(V_{t}f)\exp\Big\{ \int_{t-\epsilon}^t C^{\eqref{eq:VfO.1}}_{u,f} ~du\Big\} (1+C_{\epsilon,x, V_{t-\epsilon} f}^{\eqref{asp:H2!}} ).
\end{align}
	According to \eqref{lem:nV::H2::Vf1}  and \eqref{asp:H2!} we have
\begin{equation}	
	\varlimsup_{t\to \infty}\sup_{x\in E} |C_{\epsilon,x, V_{t-\epsilon} f}^{\eqref{asp:H2!}}| < \infty, \quad \epsilon > 0.
\end{equation}
	From this, $\eqref{eq:VfO.3}$, and the fact that $\varlimsup_{u\to \infty} C^{\eqref{eq:VfO.1}}_{u,f}  < \infty$, we can get \eqref{Claim:VfO:H4:H3} as claimed.
	
	We now use \eqref{lem:PsV:H1:H2:H4}, \eqref{Claim:nPPV:H1:H2:H4} and \eqref{Claim:VfO:H4:H3} to give the asymptotic ratio of $\nu(\Psi_0V_tf)$ and $\nu(V_tf)$.
	Note that we already obtained some result for this ratio in \eqref{eq:VfO.1}.
	We claim that the following stronger assertion is valid:
\begin{equation}\label{Claim:nP:H1:H2:H3:H4}
	\lim_{t\to \infty}C^{\eqref{eq:VfO.1}}_{t,f} = 0, \quad f\in \mathcal B(E,[0,\infty]).
\end{equation}
	To see this, we observe that
\begin{align}
	&\nu(\Psi_0 V_tf)
	\leq \nu((V_tf)\cdot (\Psi_0' V_tf)) ,\quad\text{by \eqref{e:derofpsi0}},
	\\&  \leq  \nu(\Psi_0' V_tf) \sup_{x\in E}V_tf(x)
	\\& =   C^{\eqref{Claim:nPPV:H1:H2:H4}}_{t,f} \nu(V_tf) \sup_{x\in E} (\phi(x) C^{\eqref{Claim:VfO:H4:H3}}_{t,x,f}),\quad\text{by \eqref{Claim:nPPV:H1:H2:H4} and \eqref{Claim:VfO:H4:H3}}.
\end{align}
	Since $\phi$ is bounded, \eqref{Claim:nP:H1:H2:H3:H4} follows.
	
	Using \eqref{Claim:nP:H1:H2:H3:H4}, we can get the following asymptotic ratio of $\nu(V_{t+s}f)$ and $\nu(V_tf)$:
\begin{equation} \label{Claim:nVR:H1:H2:H3:H4}
\begin{minipage}{0.9\textwidth}
	For all $t> T^{\eqref{asp:H4!}},~s \geq 0$ and $f\in \mathcal B(E,[0,\infty])$, we have $\nu(V_{t+s}f) = \nu(V_tf) \exp\{\lambda s (1+C^{\eqref{Claim:nVR:H1:H2:H3:H4}}_{t,s,f}) \}$ for some real $C^{\eqref{Claim:nVR:H1:H2:H3:H4}}_{t,s,f}$ with $\lim_{t\to \infty} \sup_{s\geq  0} |C^{\eqref{Claim:nVR:H1:H2:H3:H4}}_{t,s,f}| = 0$.
	In particular, for all $f\in \mathcal B(E,[0,\infty])$ with $\nu(f)>0$ and $s\geq 0$, we have $\lim_{t\to \infty} \frac{\nu(V_{t+s}f)}{\nu(V_tf)} = e^{\lambda s}$.
\end{minipage}
\end{equation}
	To see this, thanks to \eqref{lem:nVn!},
	we only need to consider the case $\nu(f)>0$. In this case,  it holds  that
\begin{align}
	&\frac{\nu(V_{t+s} f)} {\nu(V_t f)}
	\overset{\eqref{eq:VfO.2}}= \exp\Big\{\lambda s- \int_t^{t+s} C^{\eqref{eq:VfO.1}}_{u,f} ~du\Big\}
	=: \exp\{\lambda s (1+C^{\eqref{Claim:nVR:H1:H2:H3:H4}}_{t,s,f}) \}.
\end{align}
	Noticing that $C^{\eqref{Claim:nVR:H1:H2:H3:H4}}_{t,s,f} = -\frac{1}{\lambda s}\int_t^{t+s} C^{\eqref{eq:VfO.1}}_{u,f} ~du$ and by \eqref{Claim:nP:H1:H2:H3:H4} that $\lim_{u\to \infty}C^{\eqref{eq:VfO.1}}_{u,f} = 0$, so we have $\lim_{t\to \infty} \sup_{s> 0} |C^{\eqref{Claim:nVR:H1:H2:H3:H4}}_{t,s,f}| = 0. $
	
	We are now  ready to prove the conclusion of Lemm \ref{prop:PVf:H1:H2:H3:H4}.
	Again we only need to consider the case $\nu(f)>0$ thanks to \eqref{lem:nVn!} and  \eqref{lem:nullVf::H1:H2!}.
	In this case, by \eqref{lem:nV::H2::Vf1} and \eqref{lem:nVn!}, we have $0<\nu(V_{t}f)<\infty$.
	Therefore, we have
\begin{align}
	& P_s^\beta V_t f(x)
	\overset{\text{\eqref{asp:H2!}}}= e^{\lambda s} \phi(x) \nu(V_tf) (1+C^{\eqref{asp:H2!}}_{s,x,V_tf})
	\\&\overset{\eqref{Claim:nVR:H1:H2:H3:H4}}= \phi(x) \nu(V_{t+s}f) \exp\{-\lambda s C^{\eqref{Claim:nVR:H1:H2:H3:H4}}_{t,s,f}\} (1+C^{\eqref{asp:H2!}}_{s,x,V_tf}).
\end{align}
	From \eqref{asp:H2!} and \eqref{lem:nV::H2::Vf1}, we know that $\lim_{s\to \infty} \sup_{x\in E, t> T^{\eqref{asp:H4!}}} |C^{\eqref{asp:H2!}}_{s,x,V_tf}| = 0$.
	From \eqref{Claim:nVR:H1:H2:H3:H4}, we know that $\sup_{s\geq 0} \lim_{t\to \infty} |sC^{\eqref{Claim:nVR:H1:H2:H3:H4}}_{t,s,f}| = 0$.
	Therefore, we have \[\lim_{s\to \infty}\varlimsup_{t\to \infty}\sup_{x\in E}|\exp\{-\lambda s C^{\eqref{Claim:nVR:H1:H2:H3:H4}}_{t,s,f}\} (1+C^{\eqref{asp:H2!}}_{s,x,V_tf})-1| = 0. \qedhere\]
\end{proof}

\begin{proof}[Proof of Lemma \ref{prop:IVf:H1:H2:H3:H4}]
	For all $u\geq 0$, we have
\begin{align} \label{eq:IVf.25}
	&\nu(P_u^\beta \Psi_0 V_t f) = e^{\lambda u}\nu(\Psi_0 V_t f)
	\\&\overset{\eqref{eq:VfO.1}}=e^{\lambda u}\nu(V_tf) C^{\eqref{eq:VfO.1}}_{t,f}
	< \infty, \label{eq:IVf.5}
\end{align}
	where the last inequality follows from \eqref{lem:nV::H2::Vf1}.
	Therefore, we have
\begin{align}
 	& I_{s,\epsilon} V_t f(x)
 	= \int_0^{s- \epsilon} P_{s-u}^\beta \Psi_0 V_{t+u} f (x) du
 	= \int_0^{s- \epsilon} P_\epsilon^\beta (P_{s - \epsilon - u}^\beta \Psi_0 V_{t+u} f )(x) du
 	\\&= \int_0^{s - \epsilon} e^{\lambda \epsilon} \phi(x) \nu(P_{s - \epsilon - u}^{\beta} \Psi_0 V_{t+u} f)  \Big(1+C^{\eqref{asp:H2!}}_{\epsilon ,x , P_{s - \epsilon - u}^{\beta} \Psi_0 V_{t+u} f}\Big) du ,\quad\text{by \eqref{asp:H2!} and \eqref{eq:IVf.5}}
	\\&\overset{\text{\eqref{eq:IVf.25}}}= e^{(t+s)\lambda} \int_0^{s - \epsilon} \phi(x) e^{-\lambda (t+u)}\nu(\Psi_0 V_{t+u} f)  \Big(1+C^{\eqref{asp:H2!}}_{\epsilon ,x , P_{s - \epsilon - u}^{\beta} \Psi_0 V_{t+u} f}\Big) du
	\\&\leq \phi(x) \Big(1+\sup_{g\in L_1^+(\nu)}|C^{\eqref{asp:H2!}}_{\epsilon ,x , g}|\Big) e^{(t+s)\lambda} \int_0^{s} e^{-\lambda (t+u)} \nu(\Psi_0 V_{t+u}f)du,\quad\text{by \eqref{eq:IVf.5}}
 	\\&\overset{\eqref{eq:nuP.1}}= \phi(x) \Big(1+\sup_{g\in L_1^+(\nu)}|C^{\eqref{asp:H2!}}_{\epsilon ,x , g}|\Big)  e^{(t+s)\lambda} \Big(e^{-\lambda t}\nu(V_tf)- e^{-\lambda(t+s)}\nu(V_{t+s}f)\Big)
 	\\&= \phi(x) \Big(1+\sup_{g\in L_1^+(\nu)}|C^{\eqref{asp:H2!}}_{\epsilon ,x , g}|\Big) \nu(V_{t+s}f) \Big( \frac{e^{s \lambda }\nu(V_tf)}{\nu(V_{t+s}f)} - 1\Big),\quad \text{by \eqref{lem:nV::H2::Vf1} and \eqref{lem:nVn!}}
 	\\&\overset{\eqref{Claim:nVR:H1:H2:H3:H4}}= \phi(x) \Big(1+\sup_{g\in L_1^+(\nu)}|C^{\eqref{asp:H2!}}_{\epsilon ,x , g}|\Big) \nu(V_{t+s}f) ( \exp\{- \lambda s C^{\eqref{Claim:nVR:H1:H2:H3:H4}}_{t,s,f}\} - 1).
\end{align}
	It is easy to check that
\begin{equation}
	\lim_{t\to \infty}\sup_{x\in E}\Big|\Big(1+\sup_{g\in L_1^+(\nu)}|C^{\eqref{asp:H2!}}_{\epsilon ,x , g}|\Big)( \exp\{- \lambda s C^{\eqref{Claim:nVR:H1:H2:H3:H4}}_{t,s,f}\} - 1) \Big| = 0.
\end{equation}
	The desired result then follows.
\end{proof}

\begin{proof}[Proof of Lemma \ref{prop:JVf:H1:H2:H3:H4}]
	It follows from \eqref{e:derofpsi0} that for all $t> T^{\eqref{asp:H4!}}, x\in E$ and $f\in \mathcal B(E,[0,\infty])$,
\begin{equation}
	\Psi_0 V_t f (x)
	\leq V_tf(x)\cdot \Psi_0' V_t f(x) \overset{\eqref{lem:PsV:H1:H2:H4}}= V_tf(x) C^{\eqref{lem:PsV:H1:H2:H4}}_{t,x,f}.
\end{equation}
Since $\varlimsup_{t\to \infty} \sup_{x\in E} C^{\eqref{lem:PsV:H1:H2:H4}}_{t,x,f} <\infty$, we have
\begin{equation} \label{lem:PVtV:H1:H2:H4}
	\Psi_0 V_t f(x) = V_tf(x) C^{\eqref{lem:PVtV:H1:H2:H4}}_{t,x,f}
\end{equation}
	for some non-negative $C^{\eqref{lem:PVtV:H1:H2:H4}}_{t,x,f}$ with $\varlimsup_{t\to \infty} \sup_{x\in E}C^{\eqref{lem:PVtV:H1:H2:H4}}_{t,x,f} < \infty$.

	From this we can get that for all $u\geq 0$, $t>T^{\eqref{asp:H4!}}$, $x\in E$ and $f\in \mathcal B(E,[0,\infty])$,
\begin{equation} \label{Claim:PuPVt:H1:H2:H3:H4}
	P_u^\beta \Psi_0 V_{t} f(x) = \phi(x)\nu(V_{t+u}f) \exp\{-\lambda u C^{\eqref{Claim:nVR:H1:H2:H3:H4}}_{t,u,f} \} C^{\eqref{Claim:PuPVt:H1:H2:H3:H4}}_{t,u,x,f}
\end{equation}
	for some non-negative $C^{\eqref{Claim:PuPVt:H1:H2:H3:H4}}_{t,u,x,f}$ with $\varlimsup_{t\to \infty} \sup_{u\geq 0, x\in E} C^{\eqref{Claim:PuPVt:H1:H2:H3:H4}}_{t,u,x,f} < \infty$.
	To see this, we note that
	\begin{align}
	&P_u^\beta \Psi_0 V_{t} f(x)
	= \int_{E} \Psi_0V_tf(y) P_u^\beta (x,dy)
	\overset{\eqref{lem:PVtV:H1:H2:H4}}=\int_{E} V_tf(y)C^{\eqref{lem:PVtV:H1:H2:H4}}_{t,y,f} P_u^\beta (x,dy)
	\\&\overset{\eqref{Claim:VfO:H4:H3}}=\int_{E} \phi(y)\nu(V_tf)C^{\eqref{Claim:VfO:H4:H3}}_{t,y,f}C^{\eqref{lem:PVtV:H1:H2:H4}}_{t,y,f} P_u^\beta (x,dy)
	\\&\overset{\eqref{Claim:nVR:H1:H2:H3:H4}}=\int_{E} \phi(y)\nu(V_{t+u}f) \exp\{-\lambda u (1+C^{\eqref{Claim:nVR:H1:H2:H3:H4}}_{t,u,f}) \} C^{\eqref{Claim:VfO:H4:H3}}_{t,y,f}C^{\eqref{lem:PVtV:H1:H2:H4}}_{t,y,f} P_u^\beta (x,dy)
	\\& \leq \nu(V_{t+u}f) \exp\{-\lambda u (1+C^{\eqref{Claim:nVR:H1:H2:H3:H4}}_{t,u,f}) \} \Big(\sup_{z\in E} C^{\eqref{Claim:VfO:H4:H3}}_{t,z,f}C^{\eqref{lem:PVtV:H1:H2:H4}}_{t,z,f}\Big) \int_{E} \phi(y) P_u^\beta (x,dy)
	\\& = \nu(V_{t+u}f) \exp\{-\lambda u (1+C^{\eqref{Claim:nVR:H1:H2:H3:H4}}_{t,u,f}) \} \Big(\sup_{z\in E} C^{\eqref{Claim:VfO:H4:H3}}_{t,z,f}C^{\eqref{lem:PVtV:H1:H2:H4}}_{t,z,f}\Big) e^{\lambda u}\phi(x).
	\end{align}
	Then \eqref{Claim:PuPVt:H1:H2:H3:H4} follows from the fact that $\varlimsup_{t\to \infty} \Big(\sup_{z\in E} C^{\eqref{Claim:VfO:H4:H3}}_{t,z,f}C^{\eqref{lem:PVtV:H1:H2:H4}}_{t,z,f}\Big) < \infty$.
	
	Note that \eqref{Claim:PuPVt:H1:H2:H3:H4} gives the asymptotic behavior of $P_u^\beta \Psi_0 V_t f(x)$.
	We want to reformulated it into the asymptotic behavior  of $P_u^\beta \Psi_0 V_{t-u} f(x)$.	
	To do this, we use the following elementary facts: for any real function $h$ on $[0,\infty)^2$,
	\begin{align}\label{Fact:TO!}
	\varlimsup_{t\to \infty} \sup_{u\geq 0} |h(t,u)| < \infty & \implies \sup_{\epsilon > 0} \varlimsup_{t\to \infty} \sup_{u \in (0,\epsilon)} |h(t-u,u)| < \infty;
	\\ 	\lim_{t\to \infty} \sup_{u \geq 0} |h(t,u)| = 0 & \implies \sup_{\epsilon > 0} \lim_{t\to \infty} \sup_{u \in (0,\epsilon)} u\cdot |h(t-u,u)| = 0.
	\end{align}
	Observe that for all $u>0$, $t> T^{\eqref{asp:H4!}} + u$ and $f \in \mathcal B(E,[0,\infty])$,
	\begin{align}
	& P_u^\beta \Psi_0 V_{t-u} f(x)
	\overset{\eqref{Claim:PuPVt:H1:H2:H3:H4}}= \phi(x) \nu(V_{t}f) \exp\{-\lambda u C^{\eqref{Claim:nVR:H1:H2:H3:H4}}_{t-u,u,f} \} C^{\eqref{Claim:PuPVt:H1:H2:H3:H4}}_{t-u,u,x,f}.
	\end{align}
	From \eqref{Fact:TO!}, we know that
	\[
	\sup_{\epsilon > 0}\varlimsup_{t\to \infty} \sup_{u\in (0,\epsilon), x\in E} C^{\eqref{Claim:PuPVt:H1:H2:H3:H4}}_{t-u,u,x,f} < \infty
	\]
	and that
	\[
	\sup_{\epsilon > 0}\lim_{t\to \infty} \sup_{u\in (0,\epsilon)} uC^{\eqref{Claim:nVR:H1:H2:H3:H4}}_{t-u,u,f} =0.
	\]
	Thus,
\begin{equation}
	\label{Claim:PPV:H1:H2:H3:H4}
	P_u^\beta \Psi_0 V_{t-u} f(x) = \phi(x)\nu(V_tf) C^{\eqref{Claim:PPV:H1:H2:H3:H4}}_{t,u,f,x}
\end{equation}
	for some non-negative $C^{\eqref{Claim:PPV:H1:H2:H3:H4}}_{t, u,f,x}$ with $\sup_{\epsilon > 0} \varlimsup_{t\to \infty} \sup_{u \in (0,\epsilon), x\in E} C^{\eqref{Claim:PPV:H1:H2:H3:H4}}_{t,u,f,x} < \infty$.

	Finally, we note that
	\begin{align}
	&J_{s,\epsilon}V_tf(x) = \int_{s-\epsilon}^s P_{s-u}^\beta \Psi_0 V_{t+u} f(x)du
	= \int_0^\epsilon P_u^\beta \Psi_0 V_{t+s - u}f(x) du
	\\& \overset{\eqref{Claim:PPV:H1:H2:H3:H4}}= \int_0^\epsilon \phi(x) \nu(V_{t+s}f) C^{\eqref{Claim:PPV:H1:H2:H3:H4}}_{t+s,u,f,x}~du
	\leq \epsilon \phi(x)\nu(V_{t+s}f) \sup_{u\in (0,\epsilon)} C^{\eqref{Claim:PPV:H1:H2:H3:H4}}_{t+s,u,f,x}.
	\end{align}
	It is elementary to see that $\lim_{\epsilon \to 0}\varlimsup_{t+s \to \infty}\sup_{x\in E}\Big(\epsilon \sup_{u\in (0,\epsilon)} C^{\eqref{Claim:PPV:H1:H2:H3:H4}}_{t+s,u,f,x}\Big) = 0$.
\end{proof}


\subsection{Proof of Proposition \ref{prop:G::H1:H2:H3:H4::Y}}\label{sec:G}
%moved afterward
%\begin{proof}[Proof of Proposition \ref{prop:G::H1:H2:H3:H4::Y}]
%end move
	For any unbounded increasing positive sequence $\mathbf t = (t_n)_{n\in \mathbb N}$, define $G^\mathbf t f = \varliminf_{n\to \infty} \Gamma_{(t_n)} f$.
	Since $(\Gamma_t)_{t\geq 0}$ are $[0,\infty]$-valued functionals, so is $G^{\mathbf t}$.
	Also, from $\Gamma_t(\infty  \mathbf 1_E) = \infty$ for all $t\geq 0$ we have that $G^{\mathbf t}(\infty  \mathbf 1_E) = \infty$.
	We claim that $G^\mathbf t$ is monotone concave.
	In fact, for each $f \leq g$ in $\mathcal B(E,[0,\infty])$, we have
	\begin{equation}
	G^{\mathbf t} f
	= \varliminf_{n\to \infty} \Gamma_{(t_n)} f
	\leq \varliminf_{n\to \infty} \Gamma_{(t_n)} g
	= G^{\mathbf t} g.
	\end{equation}
	On the other hand, using Lemma \ref{Fact:CP!}, we have for all $t\geq 0$, $f\in \mathcal B(E,[0,\infty])$, $u,v \in [0,\infty)$, $r\in [0,1]$, it holds that
	\begin{align}
	\Gamma_t((ru+(1-r) v)f)
	\geq r \Gamma_t (uf) + (1-r) \Gamma_t (vf).
	\end{align}
	Therefore, for all $f\in \mathcal B(E,[0,\infty])$, $u,v \in [0,\infty)$, $r \in [0,1]$, we have
	\begin{align}
	& G^{\mathbf t}((ru + (1-r)v)f)
	= \varliminf_{n \to \infty} \Gamma_{(t_n)}((ru + (1-r)v)f)
	\\&\geq \varliminf_{n\to \infty} (r\Gamma_{(t_n)} (uf) + (1-r)\Gamma_{(t_n)}(vf))
	\\&\geq r (\varliminf_{n\to \infty} \Gamma_{(t_n)} (uf)) + (1-r) (\varliminf_{n\to \infty} \Gamma_{(t_n)}(vf) )
	\\&= r G^{\mathbf t} (uf) + (1-r) G^{\mathbf t}(vf).
	\end{align}

	To prove Proposition \ref{prop:G::H1:H2:H3:H4::Y}, we first prove two lemmas.

\begin{lem} \label{prop:Gtb:H1:H2:H3:H4}
	For any unbounded increasing positive sequence $\mathbf t = (t_n)_{n\in \mathbb N}$, $G^\mathbf t$ satisfies that
	\begin{align}
	1 - e^{-G^\mathbf t V_s f}
	= e^{s\lambda} (1-e^{- G^\mathbf t f}),
	\quad s\geq 0, f\in \mathcal B(E,[0,\infty]).
	\end{align}
\end{lem}
\begin{proof}
	First note for any $t>0$ and $f\in \mathcal B(E,[0,\infty])$, it holds that
\begin{equation}\label{lem:Gfnv!}
	1 - e^{- \Gamma_t f}
	= \frac{ \mathbf P_\nu [ 1 - e^{- X_t(f)}]}{ \mathbf P_\nu (\|X_t\| > 0)}
	= \frac{ 1 - e^{- \nu(V_tf)} }{ 1 - e^{- \nu(v_t)}}.
\end{equation}
	Fix the function $f\in \mathcal B(E,[0,\infty])$.
Thanks to \eqref{lem:nVn!} and \eqref{lem:Gfnv!}, we only need to consider the case $\nu(f) > 0$.
	In this case, by \eqref{lem:nVn!}, we have $\nu(V_tf)>0$ for each $t\geq 0$.
	Therefore, for any $s,t\geq 0$,
	\begin{align}
	& 1 - e^{- \Gamma_t V_s f}
	\overset{\eqref{lem:Gfnv!}}= \frac{ 1 - e^{- \nu(V_{t+s} f)} }{ 1 - e^{- \nu(v_t)}}
	= \frac{ 1 - e^{- \nu(V_{t+s} f)} }{ 1 - e^{- \nu(V_tf)}} \frac{ 1 - e^{ - \nu(V_tf)}}{ 1 - e^{- \nu(v_t)}}
	\\ &  \label{eq:Gtb.5}\overset{\eqref{lem:Gfnv!}}= \frac{ 1 - e^{- \nu(V_{t+s} f)} }{ 1 - e^{- \nu(V_tf)}} ( 1 - e^{- \Gamma_t f}).
	\end{align}
	Thus, for any $s\geq 0$,
	\begin{align}
	& 1 - e^{- G^{\mathbf t} V_s f}
	= \varliminf_{n\to \infty} ( 1 - e^{- \Gamma_{(t_n)} V_s f})
	\overset{\text{\eqref{eq:Gtb.5}}}= \varliminf_{n\to \infty} \Big( \frac{ 1 - e^{- \nu(V_{t_n+s}f)}}{ 1 - e^{- \nu(V_{(t_n)}f)}} (1 - e^{- \Gamma_{(t_n)} f}) \Big)
	\\& = \Big( \lim_{t \to \infty} \frac{ 1 - e^{- \nu(V_{t+s}f)}}{ 1 - e^{- \nu(V_{t}f)}} \Big) \cdot \varliminf_{n\to \infty} (1 - e^{- \Gamma_{(t_n)} f} )
	= e^{s\lambda} (1 - e^{- G^{\mathbf t}f}),
	\end{align}
	where the last equality follows from Proposition \ref{prop:Vf1::H1:H2::Y}, \eqref{Claim:nVR:H1:H2:H3:H4}, and the fact that
	$
	(1-e^{-x}) /x \xrightarrow[x\to 0]{} 1.
	\qedhere
	$
\end{proof}

\begin{lem} \label{prop:G*:H1:H2:H3:H4}
	Suppose that $r \in [\lambda,0)$.
	If $G^*_r$ is a $[0,\infty]$-valued monotone concave functional on $\mathcal B(E,[0,\infty])$ such that $G^*_r(\infty \mathbf 1_E) = \infty$ and that
	\begin{align}
	1 - e^{-G^*_r V_s f}
	= e^{s r} (1 - e^{- G_r^* f}),
	\quad s\geq 0, f\in \mathcal B(E,[0,\infty]),
	\end{align}
	then for any unbounded increasing positive sequence $\mathbf t = (t_n)_{n\in \mathbb N}$,
\begin{equation}
	1 - e^{-G_r^* f} = (1 - e^{- G^\mathbf t f})^{r/\lambda}, \quad f \in \mathcal B(E,[0,\infty]).
\end{equation}
\end{lem}

\begin{proof}%[Proof of Lemma \ref{prop:G*:H1:H2:H3:H4}]
	Let $(Q_t)_{t\geq 0}$ be a family of $[0,\infty)$-valued functionals on $\mathcal B(E,[0,\infty])$ given by
\[
	Q_tg
	:= e^{- r t}( 1 - e^{-G_r^*(gv_t)} ).
\]
	Note that, by \eqref{lem:sv2!}, $v_t(x)>0$ for all $x\in E$.
	It follows  from Proposition \ref{prop:Vf1::H1:H2::Y} that $v_t(x)<\infty$ for all $x\in E$ and all $t> T^{\eqref{asp:H4!}}$.
	Thus $v_t(\cdot)$ is a $(0,\infty)$-valued function for all $t> T^{\eqref{asp:H4!}}$.

	We claim that for any $u \in [0,1]$, $Q_t(u \mathbf 1_E)$ is non-increasing in $t\in (0,\infty)$.
	In particular, we can define the $[0,\infty]$-valued function $q(u):= \lim_{t\to \infty} Q_t(u \mathbf 1_E), u\in [0,1]$.
	In fact, note that $\mathbb P_{\delta_x}[e^{- X_s(uv_t)}] = e^{-V_s(uv_t)},x\in E, s,t>0, u \geq 0$.
	Lemma \ref{Fact:CP!} says that, for all $s,t > 0$ and $x\in E$, $u\mapsto V_s(uv_t)(x) $ is a $[0,\infty]$-valued concave function on $[0,\infty)$.
	Therefore, for $u\in [0,1]$, we have
\begin{align}
	V_s(uv_t)
	=V_s((u+ (1-u))v_t)
	\geq uV_s(v_t) + (1-u) V_s(0\cdot v_t)
	= uv_{s+t},
	\quad s,t > 0.
\end{align}
	Using this, we get
\begin{align}
	& Q_{t+s}(u \mathbf 1_E)
	= e^{- r (t+s)} ( 1-e^{-G^*(uv_{t+s})} )
	\leq e^{- r(t+s)}( 1-e^{-G^*[V_s(uv_t)]} ) \\
	& = e^{-r t}( 1-e^{-G^*(uv_t)} )
	= Q_t(u \mathbf 1_E),
	\quad s,t > 0, u \in [0,1].
\end{align}

	We want to show that $q(u)= u^{r/\lambda}, u\in [0,1]$.
	In order to do this, we first show that
\begin{equation} \label{lem:qC!}
\begin{minipage}{0.9\textwidth}
	the function $q$ is non-decreasing and concave on $[0,1]$ with $q(1) = 1$.
	In particular, thanks to Lemma \ref{Fact:CR!}, $q$ is a continuous function on $(0,1]$.
\end{minipage}
\end{equation}
	In fact, from $G^*(\infty  \mathbf 1_E) = \infty$ and $V_t(\infty  \mathbf 1_E) = v_t$, we get
	\[
	Q_t( \mathbf 1_E)
	= e^{- r t} ( 1-e^{-G^*v_t} )
	= e^{- r t} e^{r t}( 1-e^{-G^*(\infty\mathbf 1_E)} )
	= 1,
	\quad t\geq 0.
	\]
	Therefore $q(1) = 1$.
	The above argument also says that $G^*v_t < \infty$ for each $t>0$.
	Now from the condition that $G^*$ is monotone concave, we have that for all $t>0$, the map $u \mapsto G^*(uv_t)$ is a non-decreasing and concave $[0,\infty)$-valued function on $[0,1]$.
	From Lemma \ref{Fact:CE!} we get that, for each $t> 0$, $u \mapsto Q_t(u  \mathbf 1_E)$ is a $[0,\infty)$-valued, non-decreasing and concave function on $[0,1]$.
	Since the limit of concave functions is concave, we get \eqref{lem:qC!} by letting $t\to \infty$.

	We now show that
\begin{equation} \label{Claim:GQ:H1:H2:H3:H4}
	q(u) = u^{r/\lambda},\quad u\in [0,1].
\end{equation}
	To see this, note that for all $s\geq 0$, $t>T^{\eqref{asp:H4!}}$ and $x\in E$, we have that
	\begin{align}
	& e^{\lambda s}(\phi^{-1}v_t)(x)
	\overset{\text{Proposition \ref{prop:Vf2::H1:H2:H3:H4::Y}}}= e^{\lambda s}\nu(v_{t})(1+ C^{\ref{prop:Vf2::H1:H2:H3:H4::Y}}_{t,x,\infty  \mathbf 1_E})
	\\&\overset{\eqref{Claim:nVR:H1:H2:H3:H4}}=\nu(v_{t+s}) \exp\{-\lambda sC^{\eqref{Claim:nVR:H1:H2:H3:H4}}_{t,s,\infty \mathbf 1_E}\} (1+ C^{\ref{prop:Vf2::H1:H2:H3:H4::Y}}_{t,x,\infty  \mathbf 1_E})
	\\&\overset{\text{Proposition \ref{prop:Vf2::H1:H2:H3:H4::Y}}}= (\phi^{-1}v_{t+s})(x) (1+ C^{\ref{prop:Vf2::H1:H2:H3:H4::Y}}_{t+s,x,\infty \mathbf 1_E})^{-1} \exp\{-\lambda sC^{\eqref{Claim:nVR:H1:H2:H3:H4}}_{t,s,\infty  \mathbf 1_E}\} (1+ C^{\ref{prop:Vf2::H1:H2:H3:H4::Y}}_{t,x,\infty  \mathbf 1_E})
	\\& \label{eq:GQ.5}= (\phi^{-1}v_{t+s})(x) (1+C_{s,t,x}^{\eqref{eq:GQ.5}}),
	\end{align}
	for some real $C_{s,t,x}^{\eqref{eq:GQ.5}}$ with $\lim_{t\to \infty}\sup_{x\in E} |C_{s,t,x}^{\eqref{eq:GQ.5}}| =0$.
	Thus, we know that for all $s\geq 0$ and $\epsilon >0$ there exists $T^{\eqref{eq:GQ.6}}_{s,\epsilon}>0$ such that
	\begin{equation} \label{eq:GQ.6}
	1-\epsilon
	\leq \frac{e^{\lambda s}v_t(x)}{v_{t+s}(x)}
	\leq 1+\epsilon,
	\quad x\in E, t> T^{\eqref{eq:GQ.6}}_{s,\epsilon}.
	\end{equation}
	From this we get that for all $s\geq 0, \epsilon > 0$, $t\ge T^{\eqref{eq:GQ.6}}_{s,\epsilon}$, and $u\geq 0$,
	\begin{align}
	& Q_{t+s}[ (1-\epsilon)u \mathbf 1_E ]
	= e^{-r (t+s)}( 1-e^{-G^*[(1-\epsilon)uv_{t+s}]} )
	\\ & \leq e^{-r t} e^{-r s}( 1- e^{-G^*(ue^{\lambda s}v_t)} ),\quad\text{by \eqref{eq:GQ.6}},
	\\\label{eq:GQ.7}&= e^{-r s}Q_t(ue^{\lambda s}  \mathbf 1_E)
	\leq e^{-r(t+s)}( 1-e^{-G^*[(1+\epsilon)uv_{t+s}]} ),\quad\text{by \eqref{eq:GQ.6}},
	\\\label{eq:GQ.8}&= Q_{t+s}[ (1+\epsilon)u \mathbf 1_E ].
	\end{align}
	Letting $t\to \infty$ in the display above, we get that for all $s\geq 0$, $\epsilon > 0$ and $u$ satisfying $0 < (1 - \epsilon) u < (1+\epsilon)u < 1$, it holds that
\begin{align} \label{eq:GQ.9}
	& q((1-\epsilon)u)
	\leq e^{-r s}q(u e^{\lambda s})
	\leq q((1+\epsilon)u).
\end{align}
	Using \eqref{lem:qC!}, letting $\epsilon \to 0$ and then $u \uparrow 1$ in \eqref{eq:GQ.9}, we get that
\[
	q(1)
	=1
	= e^{- r s} q(e^{\lambda s}),
	\quad s \geq 0.
\]
	In other word, $q(u) = u^{r/\lambda}$ for $u\in (0,1]$.
	Finally noticing that $q$ is non-negative and non-decreasing on $[0,1]$, we also have $q(0) = 0$.

	We are now ready to finish the proof of Lemma \ref{prop:G*:H1:H2:H3:H4}.
	Fix an unbounded increasing positive sequence $\mathbf t=(t_n)_{n\in \mathbb N}$ and a function $f\in \mathcal B(E,[0,\infty])$, we only need to  prove that $1-G_r^* f =(1- G^{\mathbf t}f)^{r/\lambda}.$
	From the definition of $G^{\mathbf t} f$, we can choose a subsequence $\mathbf t'=(t'_n)_{n \in \mathbb N}$ of $\mathbf t$ such that for each $n\in \mathbb N$, we have $t'_n > T^{\eqref{asp:H4!}}$ and
\begin{equation} \label{eq:vp.5}
	1 - e^{- G^{\mathbf t}f}
	=  ( 1 - e^{-\Gamma_{( t_n')} f} ) (1+C^{\eqref{eq:vp.5}}_n)
\end{equation}
	for some real $C^{\eqref{eq:vp.5}}_n$ with $\lim_{n\to \infty} |C^{\eqref{eq:vp.5}}_n| =0$.
	Therefore, we have for any $n \in \mathbb N$,
\begin{align}
	& 1 - e^{- G^{\mathbf t}f}
	= \frac{1 - e^{- \nu( V_{(t_n')}f)}}{1- e^{- \nu(v_{(t_n')})}}  (1+C^{\eqref{eq:vp.5}}_n),\quad\text{by \eqref{eq:vp.5} and \eqref{lem:Gfnv!}},
	\\& \label{eq:vp.6}= \frac{\nu (V_{(t_n')} f)}{\nu(v_{(t_n')})}  (1+C^{\eqref{eq:vp.6}}_n)
\end{align}
	for some real $C^{\eqref{eq:vp.6}}_n$ with $\lim_{n\to \infty} |C^{\eqref{eq:vp.6}}_n| =0$, by Proposition \ref{prop:Vf1::H1:H2::Y} and the fact that $(1- e^{-x})/x \xrightarrow[x\to 0]{}1$.
	Thus
\begin{equation}
	1 - e^{- G^{\mathbf t}f}
	\overset{\text{ Proposition \ref{prop:Vf2::H1:H2:H3:H4::Y}}}=  \frac{V_{(t_n')}f(x)}{v_{(t_n')}(x)} \frac{1+C^{\ref{prop:Vf2::H1:H2:H3:H4::Y}}_{t_n',x,\infty \mathbf 1_E}}{1+C^{\ref{prop:Vf2::H1:H2:H3:H4::Y}}_{t_n',x,f}} (1+C^{\eqref{eq:vp.6}}_n).
\end{equation}
	It is elementary to see that $\lim_{n\to \infty} \sup_{x\in E} \Big|\frac{1+C^{\ref{prop:Vf2::H1:H2:H3:H4::Y}}_{t_n',x,\infty \mathbf 1_E}}{1+C^{\ref{prop:Vf2::H1:H2:H3:H4::Y}}_{t_n',x,f}} (1+C^{\eqref{eq:vp.6}}_n) -1 \Big| = 0$.
	Note from \eqref{Fact:BV!}, $V_tf \leq v_t$ for each $t\geq 0$.
	Therefore, for any $\epsilon>0$, there exists $N_\epsilon>0$ such that for any $n>N_\epsilon$,
\[
	(1-\epsilon) (1 - e^{- G^{\mathbf t}f} )
	\leq \frac{V_{(t_n')}f(x)}{v_{(t'_n)}(x)}
	\leq ((1+\epsilon) ( 1 - e^{- G^{\mathbf t}f} )) \wedge 1,
	\quad x\in E.
\]
	Since $G^*$ is a monotone functional, we know that for each $t\geq 0$, $Q_t$ is also a monotone functional.
	This implies that  for any $\epsilon>0$ and any $n>N_\epsilon$,
\begin{equation} \label{eq:vp.7}
	Q_{(t'_n)}[ (1-\epsilon) (1-e^{-G^{\mathbf t}f}) \mathbf 1_E ]
	\leq Q_{(t'_n)}\Big( \frac{V_{(t'_n)}f}{v_{(t'_n)}} \Big)
	\leq Q_{(t'_n)}[ ( (1+\epsilon) (1-e^{-G^{\mathbf t}f}) \wedge 1)  \mathbf 1_E ].
\end{equation}
	Note from the definition of $(Q_t)_{t\geq 0}$ and $G^*$, we always have for $t>T^{\eqref{asp:H4!}}$ that
\[
	Q_t \Big( \frac{V_tf}{v_t}  \Big)
	= e^{- r t}( 1 - e^{- G_r^*V_tf}  )
	= 1- e^{- G_r^* f}.
\]
	Therefore, taking $n \to \infty$ in \eqref{eq:vp.7}, and using \eqref{Claim:GQ:H1:H2:H3:H4}  we get that
\[
	\left((1 - \epsilon) (1 - e^{- G^{\mathbf t}f})\right)^{r/\lambda}
	\leq 1 - e^{- G_r^* f}
	\leq \left((1 + \epsilon) (1 - e^{- G^{\mathbf t} f})\wedge 1 \right)^{r/\lambda}.
\]
	Taking $\epsilon \to 0$, we get the desired result.
\end{proof}

\begin{proof}[Proof of Proposition \ref{prop:G::H1:H2:H3:H4::Y}]
	Combining  Lemmas \ref{prop:Gtb:H1:H2:H3:H4} and \ref{prop:G*:H1:H2:H3:H4} with a sub-sub-sequence type argument, we can easily get the conclusion of Proposition \ref{prop:G::H1:H2:H3:H4::Y}. 
\end{proof}

\subsection{Proof of Proposition \ref{prop::GD:H1:H2:H3:H4::Y}}
\begin{proof}[Proof of Proposition \ref{prop::GD:H1:H2:H3:H4::Y}] \label{sec:GD}
	From \eqref{eq:Vf1.1} and \eqref{asp:H2!}, we have
\begin{align}\label{eq:GD.1}
	&V_1 g_n(x)
	\leq P^\beta_1 g_n(x)
	\leq C^{\eqref{eq:GD.1}} \phi(x) \nu(g_n),
	\quad n \in \mathbb N, x\in E,
\end{align}
	where $C^{\eqref{eq:GD.1}}:= \sup_{x\in E, f\in L_+^1(\nu)}e^{\lambda }(1+|C^{\eqref{asp:H2!}}_{1,x,f}|)$.
	By the bounded convergence theorem, we have
\begin{equation} \label{eq:GD.11}
	\lim_{n\to \infty} \nu(g_n) =0.
\end{equation}
	On the other hand, from \eqref{eq:nuP.1}, we know that $ t\mapsto e^{-\lambda t}\nu(v_t)$ is a non-increasing $(0,\infty)$-valued continuous function on $(T^{\eqref{asp:H4!}},\infty)$.
	Since $\lambda <0$, we have
\begin{equation} \label{eq:GD.12}
\begin{minipage}{0.9\textwidth}
	$ t\mapsto \nu(v_t)$ is a strictly decreasing $(0,\infty)$-valued continuous function on $(T^{\eqref{asp:H4!}},\infty)$.
\end{minipage}
\end{equation}
	By Proposition \ref{prop:Vf1::H1:H2::Y}, we have
\begin{equation} \label{eq:GD.13}
	\lim_{t\to \infty}\nu(v_t) =0.
\end{equation}
	Using \eqref{eq:GD.11}, \eqref{eq:GD.12} and \eqref{eq:GD.13} we can see that there exist $n_0>0$ and a sequence $\{t_n: n>n_0\}$ of positive numbers such that
\begin{equation} \label{eq:GD.14}
	\lim_{n\to \infty} t_n = \infty
\end{equation}
	and that, for any $n>n_0$,
\begin{align} \label{eq:GD.2}& 2C^{\eqref{eq:GD.1}} \nu(g_n) \leq \nu(v_{t_n}). \end{align}
	It follows from Proposition \ref{prop:Vf2::H1:H2:H3:H4::Y} that there exists $n_1 > n_0$ such that for all $n>n_1$ and $x\in E$,
\begin{equation} \label{eq:GD.25}
	\nu(v_{t_n})\leq 2\phi(x)^{-1} v_{t_n}(x).
\end{equation}
	Now, for any $n>n_1$ and $x\in E$, we have
\begin{align}
	& V_1g_n(x) \leq C^{\eqref{eq:GD.1}} \phi(x)\nu(g_n),\quad\text{by \eqref{eq:GD.1}},
	\\& \leq \frac{1}{2}\phi(x)\nu(v_{t_n}),\quad\text{by \eqref{eq:GD.2}},
	\\\label{eq:GD.26} & \leq v_{t_n}(x),\quad\text{by \eqref{eq:GD.25}}.
\end{align}
	Therefore, for any $n>n_1$,
\begin{align}
	& 1 - e^{- Gg_n}
	\overset{\text{\eqref{eq:G.0}}}= e^{- \lambda} (1- e^{- GV_1g_n})
	\\&\leq e^{- \lambda} (1- e^{- G v_{(t_n)}}) 
	= e^{- \lambda} e^{\lambda t_n},
\end{align}
	where in the inequality above we used \eqref{eq:GD.26} and the monotonicity of $G$ (Proposition \ref{prop:G::H1:H2:H3:H4::Y}), and in the last equality, we used Proposition \ref{prop:G::H1:H2:H3:H4::Y} with $f = \infty$.
	Taking $n\to \infty$ in the display above, noticing \eqref{eq:GD.14} and the fact that $\lambda < 0$, we get the desired result.
\end{proof}

\section{Proof of Proposition \ref{prop:CQ}--\ref{prop:EQ}}
In this section, we assume that \eqref{asp:H2!}--\eqref{asp:HE} hold.

\subsection{Proof of Proposition \ref{prop:EQ}} \label{sec:EQ}
\begin{proof}[Proof of Proposition \ref{prop:EQ} (1)]
	From Proposition \ref{prop:Vf1::H1:H2::Y}, we have that 
	\begin{equation} \label{eq:EQ.1}
	\text{ there exists $T^{\eqref{eq:EQ.1}}>0$ such that for each $t>T^{\eqref{eq:EQ.1}}$, $\sup_{x\in E}v_t(x) < \infty$.}
	\end{equation}	
	From \eqref{asp:HE} and \eqref{eq:OY.1} we know that for $t > T^{\eqref{eq:EQ.1}}$, $v_t \in C_b(E,[0,\infty))$.  
	Denote by $G$ the functional given by Proposition \ref{prop:G::H1:H2:H3:H4::Y}; and by $G_\lambda$ the log-Laplace functional of the Yaglom limit $\mathbf Q_\lambda$. 
	Now, we can verify that,
	\begin{equation} \label{eq:EQ.2}
	G_\lambda f \overset{\text{Theorem \ref{Theorem:Y:H1:H2:H3:H4}}}= \lim_{t\to \infty} \Gamma_t f = Gf, \quad f\in C_b(E,[0,\infty]).
	\end{equation}
	In particular, we have for $t> T^{\eqref{eq:EQ.1}}$,
	\begin{align}
	&(\mathbf Q_\lambda \mathbb P) (\|X_t\|>0) 
	\overset{\text{\eqref{eq:OY.1}}}= \int_{\mathcal M_f(E)}(1-e^{-\mu(v_t)})\mathbf Q_\lambda (d\mu)
	\overset{\eqref{eq:EQ.2}}= 1 - e^{-G v_t}
	\\&\overset{\text{Proposition \ref{prop:G::H1:H2:H3:H4::Y}}}= e^{\lambda t}.  \label{eq:EQ.3}
	\end{align}
	Therefore, we have that for each $f\in C_b(E,[0,\infty])$ and $t > T^{\eqref{eq:EQ.1}}$,
	\begin{align}
	&(\mathbf Q_\lambda \mathbb P)[1 - e^{-X_t(f)} | \|X_t\| > 0] 
	\overset{\text{\eqref{eq:EQ.3}}}= e^{- \lambda t} (\mathbf Q_\lambda \mathbb P)[1- e^{-X_t(f)}] 
	\\& \overset{\eqref{eq:BGD.2}}= e^{- \lambda t} \int_{\mathcal M_f(E)} (1- e^{-\mu(V_tf)} ) \mathbf Q_\lambda (d\mu)
	= e^{-\lambda t} (1 - e^{- G_\lambda V_tf})
	\\& \overset{\text{\eqref{asp:HC} \& \eqref{eq:EQ.2}}}= e^{-\lambda t} (1 - e^{- G V_tf})
	\overset{\text{Proposition \ref{prop:G::H1:H2:H3:H4::Y}}}= 1 - e^{-Gf}
	\\&\overset{\eqref{eq:EQ.2}}= 1- e^{-G_\lambda f}
	=\int_{\mathcal M_f(E)} (1 - e^{-\mu(f)}) \mathbf Q_\lambda (d\mu).
	\end{align}
	According to \cite[Theorem 1.17]{Li2011MeasureValued}, we have that
	\[
	(\mathbf Q_\lambda \mathbb P)(\cdot |\|X_t\| > 0) = \mathbf Q_{\lambda}(\cdot), \quad t > T^{\eqref{eq:EQ.1}}.
	\]
	This says that $\mathbf Q_\lambda$ is a QLD of $X$, and therfore by \eqref{eq:S.1}, it is a QSD of $X$.
	From \eqref{eq:EQ.3} and \eqref{eq:S.2}, its mass decay rate can only be $\lambda$.
\end{proof}

\begin{proof}[Proof of Proposition \ref{prop:EQ} (2)]
	Denote by $\gamma = r / \lambda$. 
	Since $r \in (\lambda, 0)$, we have $\gamma \in (0,1)$.
	
	We first claim that there exists a $\mathbb N$-valued random variable $\{Z;P\}$ with probability generating function $P[s^Z] = 1 - (1- s)^{\gamma}, s\in [0,1]$. To see this, we set
	\[
	P(Z = n) = \frac{\gamma(1-\gamma ) \cdot (n-1-\gamma  )}{n!}, \quad n \in \mathbb N.
	\]
	Using Newton's binomial theorem (see \cite[Exercise 8.22]{Rudin1976Principles})
	\[
	1 - (1 - s)^\gamma = \sum_{n = 1}^\infty \frac{\gamma (1-\gamma)\cdot (n-1-\gamma )}{n!} s^n, \quad s\in [0,1],
	\] 
	we know such random variable exists.
	
	Now let $\{(Y_n)_{n \in \mathbb N}; P\}$ be $\mathcal M_f(E)$-valued i.i.d. sequence with law of the Yaglom limit $\mathbf Q_\lambda$. 
	Let $Z$ and $(Y_n)_{n\in \mathbb N}$ be independent of each other.  
	Define the probability $\mathbf Q_r$ on $\mathcal M_f(E)$ as the law of the finite random measure $\sum_{n=1}^Z Y_n$.
	
	In the rest of this proof, we will argue that $\mathbf Q_r$ is a QSD of $X$ with mass decay rate of $r$.
	To do this, denoting by $G_r$ the log-Laplace functional of $\mathbf Q_r$, we calculate that
	\begin{align}  
	&e^{- G_r f} 
	= P[ e^{-\sum_{n=1}^Z Y_n(f)} ] 	
	= P\left[P\left[ \prod_{n=1}^Z e^{-Y_n(f)} \middle | \sigma(Z)\right]\right]
	= P \left[ e^{-Z \cdot G_\lambda f}\right] 
	\\&= 1 - (1 - e^{-G_\lambda f})^\gamma, \quad f\in \mathcal B(E,[0,\infty]).  \label{eq:EQ.4}
	\end{align}
	Therefore, for each $t> 0$ and $f\in \mathcal B(E,[0,\infty])$, we have
	\begin{align}
	&(\mathbf Q_r \mathbb P)\left[ 1 - e^{-X_t (f)} \middle|\|X_t\|>0 \right] 
	= (\mathbf Q_r \mathbb P)(\|X_t\| >0)^{-1} \cdot (\mathbf Q_r \mathbb P) [1 - e^{- X_t(f)}]
	\\&\overset{\eqref{eq:BGD.2} \& \eqref{eq:OY.1}}= (1 - e^{-G_r v_t})^{-1}  (1 - e^{-G_r V_tf})
	\overset{\text{\eqref{eq:EQ.4}}}= (1 - e^{- G_\lambda v_t})^{-\gamma}(1 - e^{-G_\lambda V_tf})^{\gamma}
	\\&= (\mathbf Q_\lambda \mathbb P)\left[1 - e^{- X_t(f)}\middle| \|X_t\|>0\right]^{\gamma}
	\\& \overset{\text{Proposition \ref{prop:EQ} (1)}}= (1 - e^{- G_\lambda f})^{\gamma}
	\overset{\text{\eqref{eq:EQ.4}}}= 1 - e^{- G_r f}.
	\end{align} 
	This proofs that $\mathbf Q_r$ is a QSD. 
	To see its mass decay rate is $r$, we calculate that for each $t>0$, \begin{align}
	&(\mathbf Q_r \mathbb P)(\|X_t\|>0) = 1 - e^{- G_r v_t} \overset{\text{\eqref{eq:EQ.4}}}= (1 - e^{- G_\lambda v_t})^\gamma 
	\\&= (\mathbf Q_\lambda \mathbb P) (\|X_t > 0\|)^\gamma \overset{\text{Proposition \ref{prop:EQ} (1)}}= e^{ r t}. \qedhere\end{align} 
\end{proof}

\subsection{Proof of Proposition \ref{prop:CQ}} \label{sec:CQ}

\begin{proof}[Proof of Proposition \ref{prop:CQ} (1)]
	First observe that for any $t\geq 0$,
	\begin{equation} \label{eq:CQ.1}
	e^{rt} = (\mathbf Q_r^*\mathbb P)(\|X_t\|>0) 
	\overset{\eqref{eq:OY.1}}= \int_{\mathcal M_f(E)}  ( 1 - e^{-\mu(v_t)} ) \mathbf Q_r^*(d\mu)
	= 1 - e^{- G_r^*(v_t)}.
	\end{equation}
	According to Lemma \ref{Fact:CP!}, for any $t>0$, we know that $u\mapsto G_r^*(uv_t)$ is a $[0,\infty]$-valued concave function on $[0,\infty)$.
	According to Lemma \ref{Fact:CE!}, for any $t>0$, we know that $u \mapsto 1 - e^{- G_r^*(uv_t)}$ is a $[0,1]$-valued concave function on $[0, \infty)$.
	In particular, we have for any $t>0$ and $u \geq 0$ that
	\begin{equation} \label{eq:CQ.2}
		1 - e^{- G_r^*(uv_t)} \geq u(1 - e^{- G_r^*(1\cdot v_t)}) + (1-u) (1 - e^{- G_r^*(0 \cdot v_t)})  = u(1 - e^{- G_r^*(v_t)}).
	\end{equation}
	Now for any $s>0, \epsilon > 0$ and $t > T^{\eqref{eq:GQ.6}}_{s, \epsilon}$ we have
	\begin{align}
	& e^{rs} \overset{\text{\eqref{eq:CQ.1}}}= \frac{1 - e^{-G_r^*(v_{t+s})}}{1 - e^{-G_r^*(v_{t})}} \geq \frac{1 - e^{-G_r^*( \frac{e^{\lambda s}}{1+\epsilon}v_t )}}{1 - e^{-G_r^*(v_{t})}} ,\quad\text{by \eqref{eq:GQ.6}},
	\\&\geq \frac{e^{\lambda s}}{1+\epsilon},\quad\text{by \eqref{eq:CQ.2}}.
	\end{align}
	Letting $\epsilon \to 0$, we get the desired result.
\end{proof}

\begin{proof}[Proof of Proposition \ref{prop:CQ} (2)]
	According to Lemma \ref{Fact:CP!}, we know that $G_r^*$ is a monotone concave functional.
	Knowing that $\mathbf Q^*_r$ is a QSD for $X$ with mass decay rate $r$, it can be verified that for each $f\in \mathcal B(E,[0,\infty])$ and $t\geq 0$,
	\begin{align}
	&1 - e^{-G_r^*f} = \int_{\mathcal M_f(E)} (1-e^{-\mu(f)})\mathbf Q_r^*(d\mu)
	= (\mathbf Q_r^* \mathbb P) [1 - e^{-X_t(f)} |\|X_t\|>0]
	\\&= e^{-rt}(\mathbf Q_r^*\mathbb P)[1 - e^{- X_t(f)}]
	\overset{\eqref{eq:BGD.2}}= e^{-rt} \int_{\mathcal M_f(E)} (1 - e^{-\mu(V_tf)}) \mathbf Q_r^*(d\mu)
	\\&= e^{-rt} (1 - e^{-G_r^*V_tf}).
	\qedhere
	\end{align}
	
\end{proof}

\subsection{Proof of Proposition \ref{prop:UC}} \label{sec:UC}
\begin{proof}[Proof of Proposition \ref{prop:UC}]
	This is now obvious from Lemma \ref{prop:G*:H1:H2:H3:H4} and the fact that $Gf = \lim_{t\to \infty} \Gamma_t f$ for $f\in \mathcal B(E,[0,\infty])$, which is due to Theorem \ref{prop:G::H1:H2:H3:H4::Y}.
\end{proof}

\appendix\section{}
\subsection{Extended values} \label{sec:EV}
	In this paper, we often work with the extended non-negative real number system $[0,\infty]$ which consists of the non-negative real line $[0,\infty)$ and an extra point $\infty$.
	We consider $[0,\infty]$ as the one point compactification of $[0,\infty)$; and therefore, it is a compact Hausdorff space.
	We also make the following conventions that
\begin{itemize}
\item
	$x + \infty = \infty$ for each $x\in [0,\infty]$;
\item
	$x \cdot \infty = \infty$ for each $x\in (0,\infty]$;
\item
	$\frac{1}{\infty} = 0$; $\frac{1}{0} = \infty$; $e^{-\infty} =0$; $-\log 0 = \infty$.
\end{itemize}
	Note that $ \infty \cdot 0$ has no meaning, but we use the convention that $\infty \cdot 0 = 0$ when we are dealing with indicator functions. 
	For example, we may write expression like
\begin{equation}
	h(x)
	= g(x) \cdot  1_{x\in A} + \infty \cdot  1_{x \in E\setminus A}, \quad x\in E,
\end{equation}
	as a shorthand of
\begin{equation}
	x =
\begin{cases}
	g(x) & \text{if $x\in A$},
	\\ \infty & \text{if $x\in E\setminus A$}.
\end{cases}
\end{equation}

\subsection{Cancave functionals}
	We say an $\mathbb R$-valued (or $[0,\infty]$-valued) function $f$ on a convex subset $D$ of $\mathbb R$ is concave iff
\[
   	f(rx+(1-r) y)
 	\geq r f(x) + (1-r) f(y),
 	\quad x,y \in D, r \in [0,1].
\]
	The following lemmas about concave functions are elementary, we refer our readers to \cite[Chapter 6]{Dudley2002Real} for more details.


\begin{lem} \label{Fact:CR!}
	If $f$ is a non-decreasing $\mathbb R$-valued concave function on $(a,b]$ where $a<b$ in $\mathbb R$, then $f$ is continuous on $(a,b]$.
\end{lem}


\begin{lem} \label{Fact:CP!}
	Suppose that $\{Z; P\}$ is a $[0,\infty]$-valued random variable.
	Define $L(u):= - \log P[e^{- u Z}]$ with $u \in [0,\infty)$, then $L$ is a $[0,\infty]$-valued concave function on $[0,\infty)$.
\end{lem}

\begin{lem} \label{Fact:CE!}
	Suppose that $g$ is a concave function on some convex subset $D$ of $\mathbb R$, then so is $q:= 1- e^{-g}.$
\end{lem}

\subsection{Continuity theorem for the Laplace functional of random measures}
	In this subsection, we discuss the continuity theorem for for the Laplace functional of finite random measures on locally compact separable metric space.	
	The following result is not new, but for the sake of completeness we included its explicit form here.
	Let $E$ be a locally compact separable metric space. 
	Denote by $\mathcal M_f(E)$ the collection of all the finite Borel measures on $E$ equipped with the topology of weak convergence.
	According to \cite[Lemma 4.5]{Kallenberg2017Random}, $\mathcal M_f(E)$ is a Polish space.
\begin{lem} \label{fact:WC}
	Let $(\mathbf P_n)_{n\in \mathbb N}$ be a sequence of probability measures on $\mathcal M_f(E)$. 
	Suppose that (1) for each $f \in C_b(E,[0,\infty))$, $\int_{\mu \in \mathcal M_f(E)} e^{-\mu(f)} \mathbf P_n(d\mu)$ convergence to a $[0,1]$-valued number $R_f$ when $n\to \infty$; and (2) for each $f \in C_b(E,[0,\infty))$, we have $R_{u f} \xrightarrow[u \to 0]{} 0$.
	Then $(\mathbf P_n)_{n \in \mathbb N}$ convergence in distribution to a probability measure $\mathbf Q$ on $\mathcal M_f(E)$.
\end{lem}
\begin{proof}
	Notice that, for any $f\in C_b(E,[0,\infty])$ and $n\in \mathbb N$, $u \mapsto \int_{\mathcal M_f(E)} e^{- u\mu(f)} \mathbf P_n(d\mu)$ is the Laplace transform of the non-negative random variable $\{\mu(f); \mathbf P_n\}$. 
	According to the classical continuity theorem for the Laplace transform of non-negative random variables, see \cite[Theorem 5.22]{Kallenberg2002Foundations}, the conditions (1) and (2) now imply that, for each $f \in C_b(E,[0,\infty))$, $\{ \mu(f); \mathbf P_n\}$ convergence in distribution to a non-negative random variable when $n\to \infty$.
	Then, according to \cite[Corollary 4.14 and Theorem 4.19]{Kallenberg2017Random} there exists a probability measure $\mathbf Q$ on $\mathcal M_f(E)$ such that $\mathbf P_n$ convergence to $\mathbf Q$ in distribution when $n\to \infty$.
\end{proof}

\subsection{Intrinsic ultracontractivity: an example}
	In this subsection, we will give a example of a superprocess $X$ which satisfies \eqref{asp:H2!}.
	Let $E$ be a locally compact separable metric space.
	Let $\xi:= \{(\xi)_{0\leq t < \zeta}; (\Pi_x)_{x\in E}\}$ be an $E$-valued general Hunt process with general transition kernels $(P_t)_{t\geq 0}$ and lifetime $\zeta$.
	Let $\psi$ be a function on $E \times [0,\infty)$ given by
\begin{align}
	\psi(x,z)
	=- \beta(x) z + \sigma(x)^2 z^2 + \int_0^\infty (e^{-zu} -1 + zu) \pi(x,du),
	\quad x\in E, z\geq 0
\end{align}
	where $\beta, \sigma \in \mathcal B_b(E,\mathbb R)$ and $(u \wedge u^2) \pi(x,du)$ is a bounded kernel from $E$ to $(0,\infty)$.
	Let $X$ is a $(\xi, \psi)$-superprocess.

	We assume  that there exist an $\sigma$-finite measure $m$ with full support on $E$ and a family of strictly positive, bounded continuous finctions $\{p_t(\cdot,\cdot): t>0\}$ on $E\times E$ such that
\begin{align}
	\Pi_x[f(\xi_t) 1_{t< \zeta}] = \int_E p_t(x,y) f(y)m(dy), & \quad t>0, x\in E, f\in \mathcal B_b(E,\mathbb R);
	\\ \int_E p_t(x,y) m(dx) \leq 1, &\quad t>0, y\in E;
	\\ \int_E \int_E p_t(x,y)^2 m(dx)m(dy) < \infty, &\quad t>0;
\end{align}
	and the functions $x \mapsto \int_E p_t(x,y)^2m(dy)$ and $y\mapsto \int_E p_t(x,y)^2m(dx)$ are both continuous.

	Choose an arbitrary $ \mathfrak b\in \mathcal B_b(E,\mathbb R)$.
	Denote by $(P_t^\mathfrak b)_{t\geq 0}$ a semigroup of operators on $\mathcal B_b(E,\mathbb R)$ given by
\begin{equation}
	P_t^\mathfrak b f(x)
	:= \Pi_x[e^{\int_0^t \mathfrak b(\xi_s)ds} f(\xi_t) 1_{t< \zeta}],
	\quad f\in \mathcal B_b(E, \mathbb R), t\geq 0, x\in E.
\end{equation}
	Let us write $\langle f,g \rangle_m:= \int_E f(x)g(x) m(dx)$ for  the inner product of the Helbert space $L^2(E,m)$.
	Then it is proved in \cite{RenSongZhang2015Limit} and \cite{RenSongZhang2017Central} that there exists a family of strictly positive, bounded continuous functions $\{p_t^\mathfrak b: t> 0\}$ on $E\times E$ such that
\begin{equation} \label{eq:IU.0}
	e^{-\|\mathfrak b\|_\infty t} p_t(x,y)
	\leq p_t^\mathfrak b(x,y) \leq e^{\|\mathfrak b\|_\infty t}p_t(x,y),
	\quad t>0, x,y\in E
\end{equation}
	and that
\begin{equation}
	P_t^\mathfrak b f(x)
	= \int_E p_t^\mathfrak b(x,y) f(y) m(dy),
	\quad t>0, x\in E.
\end{equation}
	Define the dual semigroup $(\widehat {P^{\mathfrak b}_t} )_{t\geq 0}$ by
\begin{equation}
	\widehat {P_0^{\mathfrak b}}
	= I;
	\quad \widehat {P_t^{\mathfrak b}} f(x)
	:= \int_E p_t^\mathfrak b(y,x) f(y) m(dy),
	\quad t>0,x\in E, f\in \mathcal B_b(E,\mathbb R).
\end{equation}
	It is proved in \cite{RenSongZhang2015Limit} and \cite{RenSongZhang2017Central} that both $(P_t^\mathfrak b)_{t\geq 0}$ and $(\widehat {P_t^\mathfrak b})_{t\geq 0}$ are strongly continuous semigroups of compact operators on $L^2(E,m)$.	
	Let $L^\mathfrak b$ and $\widehat {L^\mathfrak b}$ be the generators of the semigroups of compact operators on $(P_t^\mathfrak b)_{t\geq 0}$ and $(\widehat {P_t^\mathfrak b})_{t\geq 0}$, respectively.
	Denote by $\sigma(L^\mathfrak b)$ and $\sigma(\widehat{L^\mathfrak b})$ the spectra of $L^\mathfrak b$ and $\widehat {L^{\mathfrak b}}$, respectively.
	According to Theorem 29 of \cite{Schaefer1974Banach}, $\lambda_\mathfrak b:= \sup \Re(\sigma(L^\mathfrak b)) = \sup \Re(\sigma( \widehat{L^\mathfrak b})) $ is a common eigenvalue of multiplicity $1$ for both $L^\mathfrak b$ and $\widehat {L^{\mathfrak b}}$.
	By the argument in \cite{RenSongZhang2015Limit} and \cite{RenSongZhang2017Central}, the eigenfunctions $h_\mathfrak b$ of $L^\mathfrak b$ and $\widehat h_\mathfrak b$ of $\widehat{L^\mathfrak b}$ associated with the eigenvalue $\lambda_\mathfrak b$ can be chosen to be strictly positive and continuous everywhere on $E$.
	Setting $\langle h_\mathfrak b,h_\mathfrak b\rangle_m = \langle h_\mathfrak b, \widehat h_\mathfrak b\rangle_m = 1$ so that $h_\mathfrak b$ and $\widehat h_\mathfrak b$ are uniquely determined pointwisely.

	For the rest of this subsection, we assume further that $h_0:= h_\mathfrak b|_{\mathfrak{b} \equiv 0}$ is bounded, and the semigroup $(P_t)_{t\geq 0}$ is intrinsically ultracontractive in the following sense: for all $t>0$ and $x, y \in E$, it holds that $p_t(x,y) = c_{t,x,y} h_0(x) \widehat h_0(y)$ for some non-negative $c_{t,x,y}$ with $\sup_{x,y \in E} c_{t,x,y}< \infty$.
	Here, $\widehat h_0 := \widehat h_\mathfrak b|_{\mathfrak{b}\equiv 0}$.
	Then, it is proved in \cite{RenSongZhang2015Limit} and \cite{RenSongZhang2017Central} that, for arbitrary $\mathfrak b \in \mathcal B_b(E,\mathbb R)$, $h_\mathfrak b$ is also bounded; and $(P_t^\mathfrak b)_{t\geq 0}$ is also intrinsically ultracontractive, in the sense that for any $t> 0$ and $x,y \in E$ we have
\begin{equation} \label{eq:IU.1}
	p^\mathfrak b_t(x,y)
	= h_\mathfrak b(x) \widehat h_\mathfrak b (y) C^{\eqref{eq:IU.1}}_{\mathfrak b,t,x,y}
\end{equation}
	for some positive $C^{\eqref{eq:IU.1}}_{\mathfrak b,t,x,y}$ with $\sup_{x,y \in E} C^{\eqref{eq:IU.1}}_{\mathfrak b,t,x,y}< \infty$.
	From this, it is proved in \cite{KimSong2008Intrinsic} that,
\begin{equation} \label{eq:IU.11}
	\sup_{x,y \in E} (C^{\eqref{eq:IU.1}}_{\mathfrak b,t,x,y})^{-1}
	< \infty,
	\quad t>0;
\end{equation}
	and that for any $t>0, x,y \in E$,
\begin{equation}\label{eq:IU.2}
	C^{\eqref{eq:IU.1}}_{\mathfrak b,t,x,y}
	= e^{t\lambda_\mathfrak{b}} (1+ C^{\eqref{eq:IU.2}}_{\mathfrak b,t,x,y})
\end{equation}
	for some real $C^{\eqref{eq:IU.2}}_{\mathfrak b,t,x,y}$ with $\lim_{t\to \infty} \sup_{x,y \in E} C^{\eqref{eq:IU.2}}_{\mathfrak b,t,x,y} =0$.
	Therefore,
\begin{align}
	& m(\widehat h_{\mathfrak b}) \overset{\text{\eqref{eq:IU.1}}}= \int_{E} p_t^\mathfrak{b}(x,y)h_\mathfrak{b}(x)^{-1} (C^{\eqref{eq:IU.1}}_{\mathfrak b,t,x,y})^{-1} m(dy), \quad x\in E,
	\\&\leq  h_\mathfrak{b}(x)^{-1} \Big(\sup_{z\in E}(C^{\eqref{eq:IU.1}}_{\mathfrak b,t,x,z})^{-1}\Big)  \int_{E} p_t^\mathfrak{b}(x,y)m(dy)
	\\& < \infty,\quad\text{by \eqref{eq:IU.0} and \eqref{eq:IU.11} and the fact that $h_\mathfrak{b}$ is strictly positive}.
\end{align}
	This allows us to define a probability measure $\nu_\mathfrak b (dx):= m(\widehat h_{\mathfrak b})^{-1} \widehat h_\mathfrak b (x)m(dx), x\in E$, and eigenfunction $\phi_\mathfrak{b}(x) := m(\widehat h_{\mathfrak b}) h_\mathfrak b(x), x\in E$.

	Finally we assume that $\lambda := \lambda_\beta < 0$. We now show that $X$ satisfies \eqref{asp:H2!} with $\phi:=\phi_\beta$ and $\nu:= \nu_\beta$.
	From their definitions, we see that the function $\phi \in \mathcal B_b(E,(0,\infty))$, and that the probability measure $\nu$ has full support on $E$.
	Further, it's easy to see that for each $t\geq 0$, $P_t^\beta \phi = e^{\lambda t}\phi$ and $\nu(\phi) = 1$.
	We also have that
\begin{align}
	&(\nu P_t^\beta)(dy) = \int_{x\in E}p_{t}^\beta(x,y)m(dy) \nu(dx)
	\\&= \int_{x\in E}p_{t}^\beta(x,y)m(dy) m(\widehat h_\beta)^{-1}\widehat h_\beta(x)m(dx)
	\\&=  m(\widehat h_\beta)^{-1}  \Big(\int_{x\in E} p_t^\beta(x,y) \widehat h_\beta(x) m(dx) \Big) m(dy)
	\\& = m(\widehat h_\beta)^{-1} e^{\lambda t}\widehat h_\beta(y) m(dy) = e^{\lambda t}\nu(dy), \quad y \in E, t>0.
\end{align}
	Therefore $\nu P_t^\beta = e^{\lambda t}\nu, t\geq 0$. Now for each $t>0, x \in E$ and $f\in L_1^+(\nu)$, we have
\begin{align}
	&P_t^\beta f(x) = \int_{E} p^\beta_t(x,y) f(y)m(dy)
	\overset{\text{\eqref{eq:IU.1}}}= \int_{E} h_\beta (x) \widehat h_\beta (y) C^{\eqref{eq:IU.1}}_{\beta,t,x,y} f(y) m(dy)
	\\&= \int_{E} \phi (x)  C^{\eqref{eq:IU.1}}_{\beta,t,x,y} f(y) \nu(dy)
	=: e^{\lambda t} \phi(x) \nu(f) (1+ C^{\eqref{asp:H2!}}_{t,x,f}).
\end{align}
	Finally, from \eqref{eq:IU.1} and \eqref{eq:IU.2}, it is elementary to verify that $C^{\eqref{asp:H2!}}_{t,x,f}$ satisfies \eqref{asp:H2!}.


\begin{thebibliography}{99}

	\bibitem{AH}
	Asmussens, S. and Hering, H. :\emph{Branching Processes}. Birkhauser, Boston, 1983.

	\bibitem{AthreyaNey1972Branching}
	Athreya, K. B. and Ney, P. E.:
	\emph{Branching processes.}
	Die Grundlehren der mathematischen Wissenschaften, Band 196. Springer-Verlag, New York-Heidelberg, 1972. xi+287 pp.
	\MR{0373040}
	
	\bibitem{ChampagnatRaelly2008Limit}
	Champagnat, N., and Raelly, S.:
	\emph{Limit theorems for conditioned multitype Dawson-Watanabe processes and Feller diffusions.}
	Electron. J. Probab. 13 (2008), no. 25, 777–810.

	\bibitem{Dudley2002Real}
	Dudley, R. M.:
	\emph{Real analysis and probability.}
	Revised reprint of the 1989 original. Cambridge Studies in Advanced Mathematics, 74. Cambridge University Press, Cambridge, 2002. x+555 pp.
	
	\bibitem{Etheridge2003A-decomposition}
	Etheridge, A. M., and Williams, D. R. E.:
	\emph{A decomposition of the $(1+\beta)$-superprocess conditioned on survival.}
	Proc. Roy. Soc. Edinburgh Sect. A 133 (2003), no. 4, 829–847.
	
	\bibitem{Evans1992The-entrance}
	Evans, S.:
	\emph{The entrance space of a measure-valued Markov branching process conditioned on nonextinction.}
	Canad. Math. Bull. 35 (1992), no. 1, 70–74.
	
	\bibitem{EvansPerkins1990Measure-valued}
	Evans, S.; Perkins, E.:
	\emph{Measure-valued Markov branching processes conditioned on nonextinction.}
	Israel J. Math. 71 (1990), no. 3, 329–337.

	\bibitem{Heathcote}
	Heathcote, R.,  Seneta, E.  and Vere-Jones, D.:  
	\emph{ A refinement of two theorems in the theory of branching processes}. 
	Theory Probab. Appl. 12 (1982), 297-301.
	
	\bibitem{Hering1977Subcritical}
	Hering, H.:
	\emph{Subcritical branching diffusions.}
	Compositio Math. 34 (1977), no. 3, 289–306.
	
	\bibitem{Hoppe1975Stationary}
	Hoppe, F.:
	\emph{Stationary measures for multitype branching processes.}
	J. Appl. Probability 12 (1975), 219–227.
	
	\bibitem{HoppeSeneta1978Analytical}
	Hoppe, F., and Seneta, E.:
	\emph{Analytical methods for discrete branching processes.} 
	Branching processes (Conf., Saint Hippolyte, Que., 1976), pp. 219–261, Adv. Probab. Related Topics, 5, Dekker, New York, 1978.
	
	\bibitem{JoffeSpitzer1967On}
	Joffe, A., and Spitzer, F.:
	\emph{On multitype branching processes with $\rho \leq 1$.}
	J. Math. Anal. Appl. 19 (1967), 409–430.
	
	\bibitem{Kallenberg2002Foundations}
	Kallenberg, O.:
	\emph{Foundations of modern probability.}
	Second edition. Probability and its Applications (New York). Springer-Verlag, New York, 2002. xx+638 pp. ISBN: 0-387-95313-2
	
	\bibitem{Kallenberg2017Random}
	Kallenberg, O.:
	\emph{Random measures, theory and applications.}
	Probability Theory and Stochastic Modelling, 77. Springer, Cham, 2017. xiii+694 pp. ISBN: 978-3-319-41596-3; 978-3-319-41598-7
	
	\bibitem{KimSong2008Intrinsic}
	Kim, P., Song, R.:
	\emph{Intrinsic ultracontractivity of non-symmetric diffusion semigroups in bounded domains.}
	Tohoku Math. J. (2) 60 (2008), no. 4, 527-547.
	
	\bibitem{Labbe2013Quasi-stationary}
	Labb\'e, C.:
	\emph{Quasi-stationary distributions associated with explosive CSBP.}
	Electron. Commun. Probab. 18 (2013), no. 57, 13 pp.
	
	\bibitem{Lambert2007Quasi-stationary}
	Lambert, A.:
	\emph{Quasi-stationary distributions and the continuous-state branching process conditioned to be never extinct.}
	Electron. J. Probab. 12 (2007), no. 14, 420–446.

	\bibitem{Li00} Li, Z.:
	\emph{Asymptotic behavior of continuous time and state branching processes}. J. Aus. Math. Soc. Series A 68(2000), 68C84. MR1727226

	\bibitem{Li2011MeasureValued}
	Li, Z.:
	\emph{Measure-valued branching Markov processes.}
	Probability and its Applications (New York). Springer, Heidelberg, 2011. xii+350 pp.
	
	\bibitem{LiuRen2009Some}
	Liu, R., and Ren, Y.-X.: 
	\emph{Some properties of superprocesses conditioned on non-extinction.} 
	Science in China Series A: Mathematics 52.4 (2009): 771-784.
	
	\bibitem{LyonsPemantlePeres1995Conceptual}
	Lyons, R., Pemantle, R., and Peres, Y.:
	\emph{Conceptual proofs of LlogL criteria for mean behavior of branching processes.}
	Ann. Probab. 23 (1995), no. 3, 1125–1138.
	
	\bibitem{Maillard2018The}
	Maillard, P.:
	\emph{The $\lambda$-invariant measures of subcritical Bienaymé-Galton-Watson processes.}
	Bernoulli 24 (2018), no. 1, 297–315.

	\bibitem{MeleardVillemonais2012Quasi-stationary}
	M\'el\'eard, S. and Villemonais, D.:
	\emph{Quasi-stationary distributions and population processes.}
	Probab. Surv. \textbf{9} (2012), 340C410.
	\MR{2994898}
	
	\bibitem{Overbeck1993Conditioned}
	Overbeck, L.:
	\emph{Conditioned super-Brownian motion.} 
	Probab. Theory Related Fields 96 (1993), no. 4, 545–570.
	
	\bibitem{RenSongSun2019Spine}
	Ren, Y.-X., Song, R., and Sun Z.: 
	\emph{Spine decompositions and limit theorems for a class of critical superprocesses.} 
	Acta Applicandae Mathematicae (2019): 1-41.
	
	\bibitem{RenSongSun2018Limit}
	Ren, Y.-X., Song, R., and Sun Z.: 
	\emph{Limit theorems for a class of critical superprocesses with stable branching.} 
	arXiv:1807.02837 (2018).
	

	\bibitem{RenSongZhang2015Limit}
	Ren, Y.-X., Song, R., and Zhang, R.:
	\emph{Limit theorems for some critical superprocesses.}
	Illinois J. Math. 59 (2015), no. 1, 235-276.

	\bibitem{RenSongZhang2017Central}
	Ren, Y.-X., Song, R., and Zhang, R.:
	\emph{Central limit theorems for supercritical branching nonsymmetric Markov processes.}
	Ann. Probab. 45 (2017), no. 1, 564-623.
	
	\bibitem{Rudin1976Principles}
	Rudin, W.:
	\emph{Principles of mathematical analysis.}
	Third edition. International Series in Pure and Applied Mathematics. McGraw-Hill Book Co., New York-Auckland-Düsseldorf, 1976.

	\bibitem{Schaefer1974Banach}
	Schaefer, H. H.:
	\emph{Banach lattices and positive operators.}
	Die Grundlehren der mathematischen Wissenschaften, Band 215. Springer-Verlag, New York-Heidelberg, 1974.
	
	\bibitem{Seneta1974Regularly}
	Seneta, E.:
	\emph{Regularly varying functions in the theory of simple branching processes.}
	Advances in Appl. Probability 6 (1974), 408–420.
	
	\bibitem{SenetaVere-Jones1968On}
	Seneta, E., and Vere-Jones, D.:
	\emph{On the asymptotic behaviour of subcritical branching processes with continuous state space.}
	Z. Wahrscheinlichkeitstheorie und Verw. Gebiete 10 (1968), 212–225.
	
	\bibitem{Serlet1996Occupation}
	Serlet, L:
	\emph{The occupation measure of super-Brownian motion conditioned to nonextinction.} 
	J. Theoret. Probab. 9 (1996), no. 3, 561–578.

	\bibitem{Yaglom47}Yaglom, A. M.: \emph{Certain limit theorems of the theory of branching processes}. Dokl. Acad.
	Nauk. SSSR 56 (1947), 795--798.

\end{thebibliography}
\end{document}
