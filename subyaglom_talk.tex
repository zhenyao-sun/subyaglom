% * Preamble
\documentclass[xcolor=dvipsnames]{beamer}
%\usepackage[UTF8,scheme=plain]{ctex}
\usepackage{hyperref}
\usepackage{lmodern}
\usepackage{mathrsfs}
\usepackage{graphicx}
\usepackage[export]{adjustbox}
\usepackage{comment}
\usepackage{mathtools}
\mathtoolsset{showonlyrefs}
\usetheme{Madrid}
\usecolortheme{seahorse}
\usecolortheme{rose}
\usefonttheme{serif}
\usefonttheme{structurebold}
\setbeamerfont{title}{shape=\itshape,family=\rmfamily}
\setbeamercolor{title}{fg=red!80!black,bg=red!20!white}
\everymath{\displaystyle}
% * Top matter
\title[QSDs for superprocesses]{Quasi-stationary distributions for subcritical superprocesses}
\author[Zhenyao Sun]{ 
  {\bf \Large Zhenyao Sun  }
}
\institute[]{
	Based on a joint work with {\bf Rongli Liu}, {\bf Yan-Xia Ren} and {\bf Renming Song}}
\date[]{
	Technion, Israel Institute of Technology
\\ February, 2020}
\begin{document}

\begin{frame}
  \titlepage
\end{frame}

% * Presentation
% ** Model
\section{Model}
% *** Settings
\subsection{Settings}
\begin{frame}{Definition of the superprocess}
	Let us first give the definition of the superprocess.
\begin{itemize}
\item
	Let the {\color{red} underlying space} $E$ be a Polish space. 
\item
	Let the {\color{red} spatial motion} $\{(\xi_t)_{0\leq t< \zeta}; (\Pi_x)_{x\in E}\}$ be an $E$-valued Borel right process with a random lifetime $\zeta \in (0,\infty]$. 
\item
	Let the {\color{red} branching mechanism} $\psi$ be a real function on $E\times [0,\infty)$ such that for any $x\in E$ and $z\geq 0$,
  \[\psi(x, z)=
  - \beta(x) z + \sigma(x)^2 z^2 + \int_{(0,\infty)} (e^{-zu} - 1 + zu) \pi(x,du)
  \]
	where $\beta$ and $\sigma$ are bounded measurable functions on $E$ and $(u\wedge u^2)\pi(x,du)$ is a bounded kernel from $E$ to $(0,\infty)$.

\item
	Let $\mathcal M_f(E)$ be the space of all finite Borel measures on $E$ equipped with the topology of weak convergence.
\end{itemize}
\end{frame}

\begin{frame}{Definition of the superprocess}
\begin{itemize}
\item
  Denote by $\mu(f)$, $\langle f, \mu\rangle$ or $\langle \mu, f\rangle$ the integration of a function $f$ and a measure $\mu$ whenever the integral is well-defined.
\item
	Say a function $f$ on $\mathbb R_+\times E$ is {\color{red} locally bounded} if 
$
    \sup_{s\in [0,t],x\in E} |f(s,x)|
    <\infty,
    \quad t\in \mathbb R_+.
$
\item 
	Let the {\color{red} cumulant semigroup} $(V_t)_{t\geq 0}$ be given by the following. 
	For any bounded non-negative function $f$ on $E$, there exists a unique locally bounded non-negative Borel function $(t,x) \mapsto V_tf(x)$ on $[0,\infty) \times E$ such that for any $t\geq 0$ and $x\in E$,
$
	V_tf(x) + \Pi_x \Big[ \int_0^{t\wedge \zeta} \psi(\xi_s, V_{t-s}f (\xi_s)) ds\Big] = \Pi_x[f(\xi_t) \mathbf 1_{t< \zeta}].
$
\end{itemize}
\end{frame}

% *** Superprocesses
\subsection{Superprocesses}
\begin{frame}{Definition of the superprocess}
\begin{itemize}
	\item 
	Let the {\color{red} $(\xi,\psi)$-superprocess} $\{(X_t)_{t\geq 0}; (\mathbb P_\mu)_{\mu\in \mathcal M_f(E)}\}$ be given as an $\mathcal M_f(E)$-valued Markov process such that for each bounded non-negative measurable function $f$ on $E$, $t\geq 0$ and finite measure $\mu$ on $E$,
$
	\label{eq: def of V_t}
	\mathbb{P}_{\mu}[e^{-\langle f,X_t \rangle}]
	= e^{-\langle V_tf, \mu \rangle}.
$
	\item
	$(\xi, \psi)$-superprocess exists (Watanabe (1968), Ikeda, Nagasawa and Watanabe (1968, 1969), Dawson (1975, 1977)).
\end{itemize}
\end{frame}

\begin{frame}{Definition of Yaglom limit, QLDs and QSDs}
	Let us now introduce the concept of Yaglom limit, QLDs and QSDs for the superprocess.
\begin{itemize}
\item 
	Denote $\mathbf 0$ the {\color{red} null measure} on $E$. Write $\mathcal M_f^o(E) = \mathcal M_f(E)\setminus \{\mathbf 0\}$. Any probability measure $\mathbf P$ on $\mathcal M_f^o(E)$ will also be understood as its unique extension on $\mathcal M_f(E)$ with $\mathbf P(\{\mathbf 0\}) = 0$.
\item 
	We say a probability measure $\mathbf Q$ on $\mathcal M_f^o(E)$ is the {\color{red} Yaglom limit} of the superprocess $X$ if for any $\mu \in \mathcal M_f^o(E)$, 
$
	\mathbb P_\mu(X_t \in \cdot | \|X_t\| > 0) \xrightarrow[t\to \infty]{d} \mathbf Q(\cdot).
$ 
\end{itemize}
\end{frame}

\begin{frame}{Definition of Yaglom limit and QSDs}
\begin{itemize}
\item 
	For any probability measure $\mathbf P$ on $\mathcal M_f(E)$, define $(\mathbf P \mathbb P)[\cdot]: = \int_{\mathcal M_f(E)} \mathbb P_\mu[\cdot] \mathbf P(d\mu)$.
\item 
	We say a probability measure $\mathbf Q$ on $\mathcal M_f^o(E)$ is a {\color{red} quasi-limit distribution (QLD)} of $X$, if there exists a probability measure $\mathbf P$ on $\mathcal M_f^o(E)$ such that
$
	(\mathbf P \mathbb P)(X_t \in B|\|X_t\|>0) \xrightarrow[t\to \infty]{} \mathbf Q(B), \quad B\in \mathcal B(\mathcal M_f^o(E)).
$		
\item 
We say a probability measure $\mathbf Q$ on $\mathcal M_f^o(E)$ is a {\color{red} quasi-stationary distribution (QSD)} of $X$, if 
$
(\mathbf Q \mathbb P)(X_t \in B|\|X_t\|>0) = \mathbf Q(B), \quad t\geq 0, B\in \mathcal B(\mathcal M_f^o(E)).
$		
\end{itemize}
\end{frame}

\begin{frame}{Motivation}
\begin{block}{Motivation}
	We want to investigate those sets: $\{\text{Yaglom limit of X}\}, \{\text{QLDs of X}\}$, and  $\{\text{QSDs of X}\}$.
\end{block}
Here are some basic facts (M\'el\'eard and Villemonais (2012)):
\begin{itemize}
\item 
	$\# \{\text{Yaglom limit of X}\} \leq 1$.
\item
	$ \{\text{Yaglom limit of X}\} \subset \{\text{QLDs of X}\} = \{\text{QSDs of X}\}$.
\item
	For any $\mathbf Q \in \{\text{QSDs of X}\}$, there exists an $r \in (-\infty, 0)$ such that $(\mathbf Q \mathbb \mathbb P)(X_t \neq \mathbf 0) = e^{rt}$. We say $r$ is the {\color{red} decay rate} of $\mathbf Q$.
\end{itemize}
\end{frame}

\begin{frame}{Criticality of superprocesses}
	Let us first discuss the criticality of the superprocesses.
\begin{itemize}
\item 
	{\color{red} The mean semigroup $(P_t^\beta)_{t\geq 0}$ of $X$} is given by 
$
	P_t^\beta f(x) : = \Pi_x\Big[e^{\int_0^t \beta(\xi_r) dr} f(\xi_t) \mathbf 1_{t< \zeta}\Big]
$
	where $f \in \mathcal B_b(E), t\geq 0$ and $x\in E$.
\item
	{\bf Assumption 0:} There exist a constant $\lambda < 0$, a bounded strictly positive Borel function $\phi$ on $E$, and a probability measure $\nu$ with full support on $E$ such that for each $t\geq 0$,
$
	P_t^\beta \phi = e^{\lambda t}\phi,
	\quad \nu P_t^\beta = e^{\lambda t}\nu,
	\quad \nu(\phi) = 1.
$
\item
	The assumption $\lambda < 0$ says that the mean of $(X_t(\phi))_{t\geq 0}$ decay exponentially with rate $-\lambda>0$, and in this case the superprocess $X$ is called {\color{red} subcritical}. 
\end{itemize}
\end{frame}

\begin{frame}{Intrinsic ultracontractive and non-persistent}
	We will also need the following two assumptions:
\begin{itemize}
\item
	Denote by $L_1^+(\nu)$ the collection of non-negative Borel functions on $E$ which are integrable w.r.t. $\nu$.
\item 
	{\bf Assumption 1:} For all $t>0, x\in E$ and $f\in L_1^+(\nu)$, it holds that $P_t^\beta f(x) = e^{\lambda t} \phi(x) \nu(f) (1+ C_{t,x,f})$ for some real $C_{t,x,f}$ with
	$\sup_{x\in E, f\in L_1^+(\nu)}|C_{t,x,f}|< \infty$ and
	$\lim_{t\to \infty}\sup_{x\in E, f\in L_1^+(\nu)}|C_{t,x,f}| =0$.
\item
	{\bf Assumption 2:} There exists $T\geq 0$ such that $\mathbb P_\nu(\|X_t\|=0)>0$ for all $t>T$. 
\end{itemize}
	One simple consequence of Assumption 1 is the following.
\begin{itemize}
	\item $\mathbb P_\mu(\|X_t\|>0)>0$ for each $t\geq 0$ and $\mu \in \mathcal M_f^o(E)$. 
	Hence we can condition the superprocess $X$ on survival up to a given time $t$.
\end{itemize}
\end{frame}

\begin{frame}{Main Results}
\begin{theorem}[Liu, Ren, Song and S. (2020)]
	If Assumptions 0-2 hold, then there exists a probability measure $\mathbf Q$ on $\mathcal M_f^o(E)$ such that for each $\mu \in \mathcal M_f^o(E)$,
	$\mathbb P_\mu (X_t \in \cdot | \|X_t\| > 0) \xrightarrow[t\to \infty]{d} \mathbf Q(\cdot)$. 
\end{theorem}
\begin{theorem}[Liu, Ren, Song and S. (2020)]
	Suppose that Assumptions 0-2 hold. Then (1) for each $r \in (-\infty, \lambda)$, there is no QSD for $X$ with decay rate $r$; and (2) for each $r\in [\lambda,0)$, there exists a unique QSD $\mathbf Q_r$ for $X$ with decay rate $r$.
	Moreover, for any $r \in [\lambda, 0)$ and non-negative Borel function $h$ on $E$,
$
	\Big(\int_{\mathcal M_f^o(E)} (1-e^{- w(h)}) \mathbf Q_r(dw)\Big)^{1/r}
	= \Big(\int_{\mathcal M_f^o(E)} (1-e^{- w(h)}) \mathbf Q (dw) \Big)^{1/\lambda}.
$
\end{theorem}
\end{frame}

\begin{frame}{Literature}
\begin{itemize}
	\item Yaglom limit for Galton-Watson processes are studied by Yaglom (1947), Heathcote, Seneta and Vere-Jones (1967), and Joff (1967).
	\item QSDs for Galton-Watson processes are studied by Hoppe and Seneta (1976).
	\item Yaglom limit for Multitype Galton-Watson processes are studied by Hoppe (1975), Hoppe and Seneta (1978), Joffe (1967).
	\item For Yaglom limit and QSDs for branching Markov processes, see Asmussen and Hering's book: \emph{branching processes} (1983) and the references therein. 
	\item Yaglom limit for continuous-state branching process (a degenerated superprocess) where studied by Li (2000) and Lambert (2007).
	\item QSDs for continuous-state branching processes are studied by Lambert (2007).
\end{itemize}
\end{frame}

\begin{frame}
  \[ \text{ \it \Large Thanks!}\]
\end{frame}

\end{document}
