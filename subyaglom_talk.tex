\documentclass[xcolor=dvipsnames]{beamer}
\usepackage{hyperref}
\usepackage{lmodern}
\usepackage{mathrsfs}
\usepackage{graphicx}
\usepackage[export]{adjustbox}
\usepackage{comment}
\usepackage{mathtools}
\mathtoolsset{showonlyrefs}
\usetheme{Madrid}
\usecolortheme{seahorse}
\usecolortheme{rose}
\usefonttheme{serif}
\usefonttheme{structurebold}
\setbeamerfont{title}{shape=\itshape,family=\rmfamily}
\setbeamercolor{title}{fg=red!80!black,bg=red!20!white}
\everymath{\displaystyle}
\title[QSDs for superprocesses]{Quasi-stationary distributions for subcritical superprocesses}
\author[Zhenyao Sun]{ 
  {\bf \Large Zhenyao Sun  }
}
\institute[]{
	Based on a joint work with {\bf Rongli Liu}, {\bf Yan-Xia Ren} and {\bf Renming Song}
\\ http://arxiv.org/abs/2001.06697v1}
\date[]{
	The Bernoulli-IMS One World Symposium.
\\ August, 2020}
\begin{document}

\begin{frame}
  \titlepage
\end{frame}

\section{Model}
\subsection{Settings}
\begin{frame}{Superprocess}
\begin{itemize}
\item
	Let $E$ be a Polish space. 
\item
	Let $\xi = \{(\xi_t)_{t\in [0,\zeta)}; (\Pi_x)_{x\in E}\}$ be a $E$-valued Borel right process with (sub)Markovian transition semigroup $(P_t)_{t\geq 0}$. 
\item
	Let $\psi$ be a function on $E\times \mathbb R_+$ such that
  \[
  \psi(x, z)=
  - \beta(x) z + \sigma(x)^2 z^2 + \int_{(0,\infty)} (e^{-zu} - 1 + zu) \pi(x,\mathrm du)
  \]
	where $\beta, \sigma \in \mathcal B_b(E)$ and $(u\wedge u^2)\pi(x,\mathrm du)$ is a bounded kernel from $E$ to $(0,\infty)$.

\item
	Let $\mathcal M_f(E)$ be the space of all finite Borel measures on $E$ equipped with the topology of weak convergence.
\item
Say a function $f$ on $\mathbb R_+\times E$ is locally bounded if 
\[
\sup_{s\in [0,t],x\in E} |f(s,x)|
<\infty,
\quad t\in \mathbb R_+.
\]
\end{itemize}
\end{frame}

\begin{frame}{Superprocess}
\begin{itemize}
\item
	For a measure $\mu$ and a function $f$, denote by $\mu(f)$ the integral of $f$ w.r.t. $\mu$ when it is well-defined.
\item 
	$\forall f\in \mathcal B_b^+(E)$, $\exists$ a unique locally bounded non-negative Borel function $(t,x) \mapsto V_tf(x)$ on $\mathbb R_+ \times E$ such that 
\[
	V_tf + \int_0^{t} P_s\psi\big(\cdot, V_{t-s}f (\cdot)\big)\mathrm ds = P_tf \quad \text{ on } E, t\geq 0.
\]
\item 
	$\exists$ an $\mathcal M_f(E)$-valued Borel right process $X=\{(X_t)_{t\geq 0}; (\mathbb P_\mu)_{\mu\in \mathcal M_f(E)}\}$ such that 
\[
	\mathbb{P}_{\mu}[e^{-X_t(f)}]
	= e^{-\mu(V_tf)}, 
	\quad f\in \mathcal B_b^+(E), t\geq 0, \mu\in \mathcal M_f(E).
\]
\item 
	We call this process the $(\xi, \psi)$-superprocess (Watanabe (1968), Ikeda, Nagasawa and Watanabe (1968, 1969), Dawson (1975, 1977)).
\end{itemize}
\end{frame}

\begin{frame}{Yaglom limit, QLDs and QSDs}
\begin{itemize}
\item 
	Denote $\mathbf 0$ the null measure on $E$. 
	Write $\mathcal M_f^o(E) = \mathcal M_f(E)\setminus \{\mathbf 0\}$. 
\item 
	For any probability measure $\mathbf P$ on $\mathcal M_f(E)$, define \[(\mathbf P \mathbb P)[\cdot]: = \int_{\mathcal M_f(E)} \mathbb P_\mu[\cdot] \mathbf P(\mathrm d\mu).\]
\end{itemize}
\end{frame}

\begin{frame}{Yaglom limit and QSDs}
\begin{itemize}
\item
	Suppose that $\mathbf Q$ is a probability measure on $\mathcal M_f(E)$ concentrated on $\mathcal M_f^o(E)$.
\item 
	Say $\mathbf Q$ is the {\color{red} Yaglom limit} of the superprocess $X$ if 
\[
	\mathbb P_\mu(X_t \in \cdot | X_t\neq \mathbf 0) \xrightarrow[t\to \infty]{\text{d}} \mathbf Q(\cdot), \quad \mu \in \mathcal M_f^o(E).
\]
\item 
	Say $\mathbf Q$ is a {\color{red} quasi-limit distribution (QLD)} of $X$, if $\exists$ a probability measure $\mathbf P$ on $\mathcal M_f^o(E)$ such that
\[
	(\mathbf P \mathbb P)(X_t \in B|X_t\neq \mathbf 0) \xrightarrow[t\to \infty]{} \mathbf Q(B), \quad B\in \mathcal B(\mathcal M_f^o(E)).
\]		
\item 
	Say $\mathbf Q$ is a {\color{red} quasi-stationary distribution (QSD)} of $X$, if 
\[
	(\mathbf Q \mathbb P)(X_t \in B|X_t\neq \mathbf 0) = \mathbf Q(B), \quad t\geq 0, B\in \mathcal B(\mathcal M_f^o(E)).
\]		
\end{itemize}
\end{frame}

\begin{frame}{Motivation}
\begin{block}{Motivation}
	We want to investigate those sets: $\{\text{Yaglom limit of X}\}, \{\text{QLDs of X}\}$, and  $\{\text{QSDs of X}\}$.
\end{block}
Here are some basic facts (M\'el\'eard and Villemonais (2012)):
\begin{itemize}
\item 
	$\# \{\text{Yaglom limit of X}\} \leq 1$.
\item
	$ \{\text{Yaglom limit of X}\} \subset \{\text{QLDs of X}\} = \{\text{QSDs of X}\}$.
\item
	For any $\mathbf Q \in \{\text{QSDs of X}\}$, there exists an $r \in (0,\infty)$ such that $(\mathbf Q \mathbb \mathbb P)(X_t \neq \mathbf 0) = e^{-rt}$. We say $r$ is the {\color{red} decay rate} of $\mathbf Q$.
\end{itemize}
\end{frame}

\begin{frame}{Criticality of superprocesses}
\begin{itemize}
\item 
	The mean semigroup $(P_t^\beta)_{t\geq 0}$ of $X$ is given by 
\[
	P_t^\beta f(x) 
	:= \Pi_x\Big[e^{\int_0^t \beta(\xi_r) \mathrm dr} f(\xi_t) \mathbf 1_{t< \zeta}\Big]
	= \mathbb P_{\delta_x}[X_t(f)],
\]
	where $f \in \mathcal B_b(E), t\geq 0$ and $x\in E$.
\item
	{\bf Assumption 0:} $\exists$ a constant $\lambda < 0$, a strictly positive $\phi\in \mathcal B_b(E)$, and a probability measure $\nu$ with full support on $E$ such that for each $t\geq 0$,
\[
	P_t^\beta \phi = e^{\lambda t}\phi,
	\quad \nu P_t^\beta = e^{\lambda t}\nu,
	\quad \nu(\phi) = 1.
\]
\item
	The assumption $\lambda < 0$ says that the mean of $(X_t(\phi))_{t\geq 0}$ decay exponentially with rate $-\lambda>0$. 
	In this case the superprocess $X$ is called {\color{red} subcritical}. 
\end{itemize}
\end{frame}

\begin{frame}{Intrinsic ultracontractive and non-persistent}
\begin{itemize}
\item
	Denote by $L_1^+(\nu)$ the collection of all function $f\in \mathcal B_b^+(E)$ which are integrable w.r.t. $\nu$.
\item 
	{\bf Assumption 1:} For all $t>0, x\in E$ and $f\in L_1^+(\nu)$, it holds that 
	\[
		P_t^\beta f(x) = e^{\lambda t} \phi(x) \nu(f) (1+ C_{t,x,f})
	\] 
	for some real $C_{t,x,f}$ with
	\[\sup_{x\in E, f\in L_1^+(\nu)}|C_{t,x,f}|< \infty\] and
	\[\lim_{t\to \infty}\sup_{x\in E, f\in L_1^+(\nu)}|C_{t,x,f}| =0.\]
\item
	{\bf Assumption 2:} There exists $T\geq 0$ such that $\mathbb P_\nu(X_t=\mathbf 0)>0$ for all $t>T$. 
\end{itemize}
\end{frame}

\begin{frame}{Main Results}
\begin{theorem}[Liu, Ren, Song and S. (2020)]
	If Assumptions 0-2 hold, then the Yaglom limit $\mathbf Q$ of $X$ exists. 
\end{theorem}
\begin{theorem}[Liu, Ren, Song and S. (2020)]
	Suppose that Assumptions 0-2 hold. Then 
\begin{itemize}
	\item 
	$\forall r > -\lambda$, there is no QSD of $X$ with decay rate $r$;
	\item
	$\forall r\in (0,-\lambda]$, $\exists$ a unique QSD $\mathbf Q_r$ of $X$ with decay rate $r$;
	\item
	$\mathbf Q$, the Yaglom limit $=$ $\mathbf Q_{-\lambda}$, the QSD with the highest decay rate.   
\end{itemize}
\end{theorem}
\end{frame}

\begin{frame}{Literature}
\begin{itemize}
\item 
	Yaglom limit for Galton-Watson processes were studied by Yaglom (1947), Heathcote, Seneta and Vere-Jones (1967), and Joff (1967).
\item 
	QSDs for Galton-Watson processes were studied by Hoppe and Seneta (1976).
\item 
	Yaglom limit for Multitype Galton-Watson processes are studied by Hoppe (1975), Hoppe and Seneta (1978), Joffe (1967).
\item 
	For Yaglom limit and QSDs for branching Markov processes, see Asmussen and Hering's book: \emph{branching processes} (1983) and the references therein. 
\item 
	Yaglom limit for continuous-state branching process (a degenerated superprocess) were studied by Li (2000) and Lambert (2007).
\item 
	QSDs for continuous-state branching processes were studied by Lambert (2007).
\end{itemize}
\end{frame}

\begin{frame}
  \[ \text{ \it \Large Thanks!}\]
\end{frame}

\end{document}
