\documentclass[12pt,a4paper]{amsart}
\setlength{\textwidth}{\paperwidth}\addtolength{\textwidth}{-2in}\calclayout
\usepackage{hyperref}
\usepackage{comment}
\usepackage{xcolor}

\title[Responses]{\large Responses to the referees' comments}
\begin{document}
\maketitle	
	We are grateful to the referees for the helpful comments.
	We have incorporated all the comments made by the referees.
	Here are the specific changes in response to the comments of the first referee.
\begin{itemize}
\item[1.]
	In the article, the authors use too many $C_{\dots}^{\dots}$ to denote different constants or functions.
	It is hard to distinguish these symbols, so the readers have to go back to their definitions frequently, which makes the paper not easy to read.
	And I think some of them are not necessary, for example, the $C$ in (2.16) and (2.18)...	
	{\it In the revised version, we
simplified the notation. $C^{(H1)}_{t,s,f}$ has been changed to $H_{t,s,f}$. Notation like $C^{(2.11)}_{t,x,f}$ has been changed to $C^j_{t, x, f}$.}
\\	
\item[2.]
	p1: Please note that the branching process $Z$ is nontrivial ($p_1<1$), otherwise, the extinction probability is $0$. 	
	{\it Noted as suggested.}
\\	
\item[3.]
	p1, line-1: ``(if a ...)" delete ().	
	{\it Changed as suggested.}
\\
\item[4.]
	p2, line 10: change ``$t\geq 0$.'' to ``$t\geq 0,$''.
	{\it Changed as suggested.}
\\
\item[5.]
	p3, condition (H2): give more explanation of condition (H2), or give some sufficient condition for (H2).
	{\it In the revised version, we explained this in Subsection 1.3.}
\\
\item[6.]
	p7, Proposition 1.8: $e^{r\lambda}$ should be $e^{rs}$, that is
\[
	1 - e^{- \mathcal L_{Q_r^*} V_sf} = e^{rs} (1 - e^{- \mathcal L_{Q_r^*}f}), \cdots
\]
	{\it Changed as suggested.}
\\
\item[7.]
	p7, Proposition 1.9: Idem.
	{\it Changed as suggested.}
\\
\item[8.]
	p8: in the last line of equation (2.3): $g\in \mathcal B(E,[0,\infty])$ should be $g\in L_1^+(\nu)$.	
	{\it Changed as suggested.}
\\	
\item[9.]
	p15, line 2: recall the definition of $\Gamma_t$.
	{\it Recalled as suggested.}
\\
\item[10.]
	p19: the formula below (2.46), change $v_{(t_n)}$ to $v_{t_n}$.
	{\it Changed as suggested.}
\\
\item[11.]
	p21, line7: the law of $Z$ should be
\[
	p(Z = n) = \frac{\gamma\prod_{k=1}^{n-1}(k-\gamma)}{n!}
	=\frac{\gamma(1-\gamma)\cdots (n-1-\gamma)}{n!},
	\quad n\in\mathbb Z_+.
\]
	{\it Changed as suggested.}
\\
\item[12.]
	p21: line9: Idem.
	{\it Changed as suggested.}
\\
\end{itemize}

	Here are the specific changes in response to the comments of the second referee.
\begin{itemize}
\item [1.]
	It is good to see that there exists examples that satisfy Condition (H1), but The authors wrote them at the end of the paper in appendix A.4. I recommend to write it before, in the main part of the paper. A good place to write it is before the proofs (Section 1.3).	
	Also it is good to say an example that satisfy condition (H2).
	Write a sentence after the conditions saying that you will give an example with the reference of the section.
	{\it In the revised version, we moved the content in appendix A.4 into Subsection 1.3. }
 {\it In Subsection 1.3 we also gave an example that satisfies condition (H2).}
	{\it We also added a new paragraph after (H1) and (H2) referring to Subsection 1.3.}
\\
\item[2.]
	After theorem 1.2, It would be nice to have a sentence explaining that in the rest of the paper you are going to give first the proof of Theorem 1.1 by using a series of Propositions. Later the proof of Theorem 1.2. with other propositions and finally you will give the proofs of all the propositions.
	{\it Explained as suggested.}
\\
\item[3.]
	Page 16 line 18. The equality is not true
\[
	V_s(uv_t) = V_s(u+(1-u)v_t) \dots
\]
	I think that should be $V_s(uv_t) = V_s(uv_t + (1-u)0).$	
   {\it This is a typo. We simply deleted ``$=V_s(u+(1-u)v_t)$''.}
\\
\end{itemize}

	Here are the specific changes in response to the comments of Editor.
\begin{itemize}
	\item
	I think some supporting examples for the TWO conditions (H1) and (H2) should be given.
	{\it Now examples are given in Subsection 1.3.}
\\
	\item
	The following comment should also be considered (Page 2, lines 3--4 from bottom): The specification
``bounded elements in $B(E,D)$"
does not seem natural since the meaning of a
``bounded set"
in the general metric space $D$ is not clear. In fact, if $r$ is a metric for the topology of $D$, then $r\land 1$ is a
``bounded"
metric equivalent to $r$.
	Under the metric $r\land 1$, all subsets of $D$ are
``bounded".	
	{\it We explained this notations with more details in the first paragraph of Subsection 1.2.}
\end{itemize}
\end{document}
