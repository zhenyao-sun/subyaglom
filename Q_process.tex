%Rongli Q_process 2020-02-11
%Zhenyao Q_process 2020-06-10
%Yanxia Q_process2 2020-06-22
%Rongli Q_process3 2020-07-01
%Yanxia Q_process4 2020-07-01
%Zhenyao Q_process5 2020-07-05
%Renming Q_process6 2020-07-06
%Zhenyao Q_process7 2020-07-19
%Yanxia  Q_process8
%Zhenyao Q_process9 2020-07-31
%Yanxia Q_process10 2020-08-03
%Renming Q-process11 2020-08-06
%Zhenyao Q-process12 2021-01-01
%Yanxia and Rongli Q-process13 2021-01-08
%Renming Q-process14 2021-01-10
\documentclass[12pt,a4paper]{amsart}
\setlength{\textwidth}{\paperwidth}
\addtolength{\textwidth}{-2in}
\calclayout
\numberwithin{equation}{section}
\allowdisplaybreaks
\theoremstyle{plain}
\newtheorem{thm}{Theorem}[section]
\newtheorem{asp:mean}{Assumption}[section]
\newtheorem{lem}[thm]{Lemma}
\newtheorem{prop}[thm]{Proposition}
\newtheorem{cor}[thm]{Corollaray}
\newtheorem{fact}[thm]{Fact}
\newtheorem{claim}[thm]{Claim}
\theoremstyle{definition}
\newtheorem*{ack*}{Acknowledgment}
\theoremstyle{remark}
\newtheorem{remark}{Remark}
\newtheorem{exa}[thm]{Example}
\usepackage{amssymb}
\usepackage{mathtools}
\mathtoolsset{showonlyrefs}
\usepackage{mathrsfs}
\usepackage{comment}
\usepackage[backref]{hyperref}
\usepackage[inline]{showlabels}
\usepackage{xcolor}
\usepackage{enumitem}

\begin{document}
\title {Subcritical Superprocesses Conditioned on Non-extinction}
\author[R. Liu]{Rongli Liu}
\address{Rongli Liu\\ Mathematics and Applied Mathematics\\ Beijing jiaotong University\\ Beijing 100044\\ P. R. China}
\email{rlliu@bjtu.edu.cn}
\thanks{The research of Rongli Liu is supported in part by NSFC (Grant No. 11301261), and the Fundamental Research Funds for the Central Universities (Grant No.  2017RC007)}
\author[Y.-X. Ren]{Yan-Xia Ren}
\address{Yan-Xia Ren\\ LMAM School of Mathematical Sciences \& Center for
Statistical Science\\ Peking University\\ Beijing 100871\\ P. R. China}
\email{yxren@math.pku.edu.cn}
\thanks{The research of Yan-Xia Ren is supported in part by NSFC (Grant Nos. 11671017 and 11731009)  and LMEQF.}
\author[R. Song]{Renming Song}
\address{Renming Song\\ Department of Mathematics\\ University of Illinois at Urbana-Champaign \\ Urbana \\ IL 61801\\ USA}
\email{rsong@illinois.edu}
\author[Z. Sun]{Zhenyao Sun}
\address{Zhenyao Sun\\ Faculty of Industrial Engineering and Management \\ Technion, Isreal Institute of Technology \\ Haifa 3200003\\ Isreal}
\email{zhenyao.sun@gmail.com}
\begin{abstract}
    TBD
\end{abstract}
\maketitle
\section{Introduction}
\subsection{Background}
{\bf RS: I think that we should use the setup of our previous paper: $E$ a Polish space and $\partial$ an isolated point not in $E$, and $\xi$ a Borel right process.
The reasons are: Killed BM in a ball is a Hunt process on the one-point compactification of the ball, but killed symmetric stable process in a ball is not a 
Hunt process on the one-point compactification of the ball. When $\partial$ an isolated point not in $E$, neither of the processes above are Hunt. So assuming Hunt
is too restrictive. For Proposition 1.3, we should just quote \cite[Theorem (62.19)]{Sharpe}. This way, the current Section 2 is no longer needed.}

The study of the extinction of populations is of a great interest in biology. Suppose that $(Z_n, n\ge 1)$ is a Galton-Watson process
%with offspring distribution $\{p_n\}$. 
with offspring distribution $\{p_m, m\ge 0\}$.
Let $m:=\sum^{\infty}_{n=1}np_n$ be the mean number of
children per particle. It is well known that when $m<1$ the extinction probability $q:=\lim_{n\rightarrow\infty}P\left(Z_n=0\right)$ is equal to
$1$. That is to say the process $(Z_n)$ becomes extinct in finite time almost surely. In this case, the non-extinction probability $P(Z_n>0)$ decays to $0$.  A natural question is  the right decay rate of the non-extinction probability. In 1967, Heathcot, Seneta and Vere-Jones \cite{HeathcoteSenetaVere-Jones1967A-refinement} answered the question, giving an $L\log L$ criteria. Let $L$ stand for a random variable with distribution 
%$\{p_n, n\ge 1\}$.
$\{p_n, n\ge 0\}$.

\begin{thm}[Heathcote, Seneta and Vere-Jones \cite{HeathcoteSenetaVere-Jones1967A-refinement}]
The sequence $\{ P(Z_n>0)/m^n\}$ is decreasing. If $m<1$, then the following are equivalent:
\begin{itemize}
\item[$(i)$] $\lim_{n\rightarrow\infty}P(Z_n>0)/m^n>0$,
\item[$(ii)$] $\sup E[Z_n|Z_n>0]<\infty$,
\item[$(iii)$] $E\left[L\log^+ L\right]<\infty$.
\end{itemize}
\end{thm}
In 1995, Lyons, Pemantle and Peres developed a martingale change of measure method in \cite{LyonsPemantlePeres1995Conceptual} to give a new proof for this $L\log L$ theorem.


In \cite{Lambert2001Arbres,Lambert2003Coalescence}, Lambert discussed  similar results for continuous  state  branching process (or CB-process for short) $\{Z_t, t\geq 0\}$. Suppose its branching mechanism $\psi$ is given by 
%the L\'evy-Khinchin formula:
\[
\psi(\lambda)=-\beta\lambda+\sigma^2\lambda^2+\int_0^\infty \left(e^{-\lambda r}-1+\lambda r\right)n(dr),
\]
where $n(dr)$ is a measure on $(0,\infty)$ such that $\int_0^\infty (r^2\wedge r) n(dr)<\infty$. Suppose also that the following Grey condition is satisfied:
\begin{equation}\label{extinc assump  for continuous}
\int^\infty\dfrac{1}{\psi(\lambda)}d\lambda<\infty.
\end{equation}
We denote by $\mathbb P_x$ the law of $\{Z_t,t\geq 0\}$ starting at $x>0$. Then for any $\lambda, t\geq 0$,
\[
\mathbb P_x\left(\exp\{-\lambda Z_t\}\right)=\exp(-xu_t(\lambda)),
\]
where $u_t(\lambda)$ is the unique solution to the following equation
\begin{equation}
\begin{cases}
\dfrac{d u_t(\lambda)}{dt}=-\psi(u_t(\lambda)),\\
u_0(\lambda)=\lambda.
\end{cases}
\end{equation}
Let $\zeta:=\inf\{t\geq 0; Z_t=0\}$ be the extinction time of $Z$.  There is
a nonnegative function $\varphi(t)$ such that for any $x>0$,
\[
\mathbb P_x(Z_t>0)=\mathbb P_x(\zeta>t)=1-\exp(-x\varphi(t)), \qquad t>0.
\]
When $\beta<0$, it holds that
 $
 \mathbb P_x(\zeta<\infty)=1
 $ (see, \cite{Grey1974Asymptotic}).
   The decay rate of $\mathbb P_x(\zeta>t)$ is determined by the decay rate of $\varphi(t)$.  It is shown (see, \cite{Lambert2007Quasi-stationary}) that for any $\lambda>0$,
\[
\lim_{t\rightarrow\infty}\dfrac{u_t(\lambda)}{\varphi(t)}=G(\lambda),
\]
where $G(\lambda)=\exp(\beta\int_{\lambda}^\infty1/\psi(u)du)$. Moreover, the following $L\log L$ criteria holds:

\begin{thm}\label{equivalent for cbp}
When $\beta<0$, the following items are equivalent
\begin{itemize}
\item[$(i).$] $G'(0+)<\infty$;
\item[$(ii).$] There is a positive constant $c$ such that $\varphi(t)\sim c\exp(\beta t)$;
\item[$(iii).$] $\int_1^\infty r\log r n(dr)<\infty$.
\end{itemize}
In this case, $c^{-1}=G'(0+)$.
 \end{thm}

%As far as the population dynamics which are extinct in finite time running over large amounts of time are concerned, 
In the case that the population becomes extinct in finite time, 
special attentions are given to 
%conditioning on non-extinction, 
asymptotic behavior conditioned on non-extinction, 
which can not only  lead to a stationary behavior
of the process, but also provides a lot of information about the evolution of the population before extinction.

Usually there are two ways to condition the process, and correspondingly there are two types of conditional limits. Here we state these two kinds of limits for  the continuous state branching process  $\{Z_t, t\geq 0\}$. One is called the Yaglom distribution defined by
\[
\nu(\cdot):=\lim_{t\to\infty} \mathbb P_x(Z_t\in\cdot|Z_{t}\neq 0),
\]
which is the minimal  quasi-stationary distribution of $\{Z_t, t\geq 0\}$. This kind of  conditional limit for continuous state branching processes  can be found in \cite{Li00}, where it is also generalized to conditioning of the type $\{Z_{t+r}\neq 0\}$ for any finite $r>0$ instead of $\{Z_{t}\neq 0\}$. Corresponding results for Galton-Watson processes can be found in \cite{AthreyaNey1972Branching}.

The other type of limit is the processes conditioned on being never extinct. More precisely,  let 
%$\mathcal F_t=\sigma(Z_s,s\leq t),  t\geq 0$ be the natural $\sigma$-filtration generated by 
 $\{\mathcal F_t=\sigma(Z_s,s\leq t): t\geq 0\}$ be the natural filtration of 
 $Z$. For any $x,t>0$ and $A\in\mathcal F_t$, define
\[
\lim_{s\rightarrow\infty}\mathbb P_x(A\big|\zeta>s)=\widetilde{\mathbb P}_x(A).
\]
The conditional limit $\widetilde{\mathbb P}_x$  can be expressed as an $h$-transform of $\mathbb P_x$ with martingale $M_t=Z_te^{-\beta t}$:
\[
\left.\dfrac{d\widetilde{\mathbb P}_x}{d\mathbb P_x}\right|_{\mathcal F_t}=\dfrac{M_t}{x}.
\]
Under $\widetilde{\mathbb P}_x$, the process $\{Z_t, t\geq 0\}$  can be seen as a branching model with immigrations along a immortal particle,
and  called the  $Q$-process. The study on this kind of limit can be traced back to Lamperti and Ney \cite{LamandNey} for Galton-Watson processes, also see  \cite{AthreyaNey1972Branching}.

The $Q$-process admits a stationary distribution.  In other words, the following limit probability distribution exists.
\[
\mathbb \mathbb{\mathbb{P}}(\cdot):=\lim_{t\to\infty}\widetilde{\mathbb{P}}_x(Z_t\in \cdot)=\lim_{t\to\infty}\lim_{\theta\to\infty}\mathbb{P}_x(Z_t\in\cdot|Z_{t+\theta}\neq 0).
\]
 Lambert \cite{Lambert2007Quasi-stationary} and Li \cite{Li00} established the connection between these two types of limit distributions for continuous state branching processes.  Let $\Upsilon$ be a random variable whose distribution is the Yaglom distribution $\nu$.  If $\int_1^\infty r\log r n(dr)<\infty$, then under $\widetilde{\mathbb P}_x$, $Z_t$ converges in distribution to a positive
random variable $Z_\infty$ as $t\to\infty$, which has the distribution of the size-biased Yaglom distribution
\[
\widetilde{\mathbb P}_x(Z_\infty\in dr)=\dfrac{r}{\mathbb E\Upsilon}\mathbb P(\Upsilon\in dr).
\]
 A quasi-stationary distribution (QSD) is a probability measure on $(0,\infty)$ satisfying
\[
\mathbb P_{\nu}(Z_t\in A|\zeta>t)=\nu(A).
\]
It is shown in \cite{Lambert2007Quasi-stationary} that the Yaglom distribution $\nu$ is the minimal QSD of $Z$.





  Similar results are investigated for continuous time and state multitype branching processes in \cite{Penisson2010Conditional} and \cite{Penisson2011Conditional}.  Asmussen and Hering \cite{AH} studied the limit behavior of subcritical branching %Markov processes, in which each particle lives for exponential time, then give birth to random number of particles, and particles move as independent Markov processes in between branching times and it is assume that  the life times, reproduction of different individuals are independent. 
  Markov processes.
% The studies on  the Yaglom distributions,  the quasi-stationary distributions, and the $Q$-processes  for more models  we refer to the thesis
For studies on  the Yaglom distributions,  the quasi-stationary distributions, and the $Q$-processes of other branching models, we refer the reader to
 \cite{SP}, \cite{SM} and the references therein.




In this paper, we are interested in the corresponding conditional limits for superprocesses. In our  previous paper \cite{LiuRenSongSun2020}, the Yaglom distribution and  quasi-stationary distributions for superprocesses are investigated.  \cite[Proposition $1$]{RoellyRouault1989Processus} and \cite{Evans1993Two} considered  superprocesses conditioned to  stay alive forever.  \cite{Evans1993Two} obtained a representation of such a process in terms of an immortal particle that moves around according to the underlying Markov process and throws off pieces of mass, which then proceed to evolve 
%in the same way that mass evolves for the unconditioned superprocess 
according to the law of the unconditioned superprocess
(see also \cite{EW} and \cite{EP} for the study of various aspects of conditioned superprocesses). Now this representation is developed as ``spine decomposition" for superprocesses, see .  %RS: which reference do you want to refer here?
We will introduce this spine decomposition, which is the main tool in this paper.

Other related literatures are   \cite{ChampagnatRoelly2008Limit}, \cite{Li20} and \cite{Sta}. \cite{ChampagnatRoelly2008Limit} studied the conditioned multitype  superprocesses, and  \cite{Li20} and \cite{Sta} gave some criterions for the existence of stationary distributions of superprocesses 
%without or with immigration.
with or without immigration.

\subsection{Main Result}\label{sec:M}
	 We first recall the definition of a superprocess.
	Let $E$ be a locally compact separable metric space, and $E_\partial := E \cup \{\partial\}$ be the one-point compactification of $E$.
	Denote by $\mathcal B(E, D)$ the collection of Borel maps  from $E$ to some metric space $D$.
	If $D$ is a subset of $\mathbb R$, we denote by $\mathcal B_b(E,D)$ the bounded measurable functions from $E$ to $D$.
	Let \emph{the underlying process}
	$\xi = \{(\xi_t)_{t\ge0}; (\Pi_x)_{x\in E_\partial}\}$
	 be an $E_\partial$-valued Hunt process with $\partial$ as an absorbing state.
	Denote by $\zeta:=\inf\{t>0: \xi_t=\partial\}$ the lifetime of $\xi$.
	Let \emph{the branching mechanism} $\psi$ be a function on $E \times \mathbb R_+$ given by
\begin{align}
	\psi(x,z)
	= -\beta(x) z + \sigma(x)^2 z^2 + \int_0^\infty (e^{-zu} -1 + zu) \pi(x,{\mathrm d}u),
	\quad x\in E, z\geq 0
\end{align}
	where $\beta, \sigma \in \mathcal B_b(E,\mathbb R)$ and
\begin{equation} \label{eq:M.01}
	\text{$(u \wedge u^2) \pi(x,{\mathrm d}u)$ is a bounded kernel from $E$ to $(0,\infty)$.}
\end{equation}
	Here and in the rest of this paper, for any $-\infty \leq a \leq b \leq \infty$, we write $\int_a^b$ for the integration on the open interval $(a,b)$.
	We also write $\int_b^a$ for $- \int_a^b$.
	Let $\mathcal M$ denote the space of all finite Borel measures on $E$ equipped with the topology of weak convergence.
		For any measure $\mu$ and function $f$, we use $\mu(f)$ to denote the integral of $f$ with respect to $\mu$ whenever the integral is well-defined.
		For any $f \in \mathcal B_b(E,\mathbb R_+)$, according to \cite[Proposition 2.20]{Li2011Measure-valued}, there is a unique locally bounded non-negative map $(t,x)\mapsto V_tf(x)$ on $\mathbb R_+\times E$ such that
\begin{equation} \label{eq:M.1}
	V_tf(x) + \Pi_x\Big[\int_0^{t\wedge \zeta} \psi\big(\xi_s, V_{t-s} f(\xi_s)\big) {\mathrm d}s\Big] = \Pi_x[f(\xi_t) \mathbf 1_{\{t< \zeta\}}], \quad t\geq 0, x\in E.
\end{equation}
	Here,  local boundedness of the map $(t,x) \mapsto V_tf(x)$ means that for any $T>0$,
\[
	\sup_{0\leq t\leq T, x\in E} V_tf(x) < \infty.
\]
	According to \cite[Proposition 2.21 and Theorem 5.12]{Li2011Measure-valued}, there exists
	an $\mathcal M$-valued
	Borel right process
		  $X =\{(X_t)_{t\geq 0}; (\mathrm P_\mu)_{\mu \in \mathcal M}\}$ such that
	\begin{equation} \label{eq:M.13}
	\mathrm P_\mu[e^{- X_t(f)}]
	= e^{- \mu(V_tf)},
	\quad \mu\in \mathcal M, t\geq 0, f \in \mathcal B_b(E,\mathbb R_+).
\end{equation}
	This process $X$ is known as a $(\xi, \psi)$-superprocess.
	
	Let us now give some basic assumptions on our superprocess.
	Since we are only concerned with distributional properties of the superprocess $X$, we will assume, without loss of generality, that $X$ is \emph{canonical},  i.e.,
	(1) $(X_t)_{t\geq 0}$ is the coordinate process of
		$\mathbb W$,
		the space of $\mathcal M$-valued right continuous paths on 
%		$\mathbb R_+$ {\color{red}having $\mathbf 0$ as a trap}; and
%RS better delete ``having $\mathbf 0$ as a trap'', in light of excursion measure
$\mathbb R_+$; and 
	(2) $\mathrm P_\mu(\mathrm dw)$ is a probability transition kernel from $\mathcal M$ to $\mathbb W$.
	We will use $(\mathscr F_t)_{t\geq 0}$ to denote the natural filtration on $\mathbb W$.
%	Define a Feynman-Kac semigroup of $\xi$ by
%\begin{align}\label{eq:M.15}
%	P_t^\beta f(x)
%	:= \Pi_x[e^{\int_0^t \beta(\xi_r) {\mathrm d}r }f(\xi_t) \mathbf 1_{{\color{red}\{t < \zeta\}}}],
%	\quad f\in \mathcal B_b(E,\mathbb R), t\geq 0, x\in E.
%\end{align}
	It is known (see \cite[Proposition 2.27]{Li2011Measure-valued}) that
\begin{equation} \label{eq:M.2}
	\mathrm P_\mu[X_t(f)] = \mu (P_t^\beta f),
	\quad t\geq 0, f \in \mathcal B_b(E,\mathbb R),
\end{equation}
where $(P_t^\beta)_{t\geq 0}$ is the Feynman-Kac semigroup of $\xi$ by
\begin{align}\label{eq:M.15}
	P_t^\beta f(x)
	:= \Pi_x[e^{\int_0^t \beta(\xi_r) {\mathrm d}r }f(\xi_t) \mathbf 1_{\{t < \zeta\}}],
	\quad f\in \mathcal B_b(E,\mathbb R), t\geq 0, x\in E.
\end{align}
	Thus $(P_t^\beta)_{t\geq 0}$ is called the \emph{mean semigroup}  of $X$.
%	For this mean semigroup, we will always assume that
We will always assume that
\begin{equation}\label{asp:H1} \tag{H1}
\begin{minipage}{0.9\textwidth}
	there exist a constant $\lambda < 0$, a function $\phi \in \mathcal B_b(E,(0,\infty))$ and a probability measure $\nu$ with full support on $E$ such that for each $t\geq 0$, $P_t^\beta \phi = e^{\lambda t} \phi$, $\nu P_t^\beta = e^{\lambda t} \nu$ and $\nu(\phi) =1$.
\end{minipage}
\end{equation}
	Denote by $L_1^+(\nu)$ the collection of non-negative Borel functions on $E$ which are integrable with respect to the measure $\nu$.
	Denote by $\mathbf 0$ the null measure on $E$.
	Write $\mathcal M^o = \mathcal M\setminus \{\mathbf 0\}$.
	We further assume that the following two conditions hold:
\begin{equation} \label{asp:H2} \tag{H2}
\begin{minipage}{0.9\textwidth}
	for all $t>0$, $x\in E$, and $f\in L_1^+(\nu)$, it holds that $P_t^\beta f(x) = e^{\lambda t} \phi(x) \nu(f) (1+ H_{t,x,f})$ for some $H_{t,x,f}\in \mathbb R$ with
\[
	\sup_{x\in E, f\in L_1^+(\nu)} |H_{t,x,f}|
	< \infty
	\text{ and }
	\lim_{t\to \infty} \sup_{x\in E, f\in L_1^+(\nu)} |H_{t,x,f}|
	= 0;
\]
\end{minipage}
\end{equation}
	and
\begin{equation} \label{asp:H3} \tag{H3}
	\mathrm P_\nu(X_t = \mathbf 0)>0, \quad t> 0.
\end{equation}

	To present our first result, we need to introduce a measure transformation.
	Under Assumption \eqref{asp:H1}, one can verify that, for any $\mu \in \mathcal M$, $(e^{-\lambda t}  X_t(\phi))_{t\geq 0}$ is a non-negative martingale under $\mathrm P_\mu$.
	In fact, for each $0\leq s\leq t< \infty$,
\begin{equation} \label{eq:M.25}
	\mathrm P_\mu[e^{-\lambda t}X_t(\phi)|\mathscr F_s]
	\overset{\text{Markov}} = e^{-\lambda t} \mathrm P_{X_s}[X_{t-s}(\phi)]
	\overset{\eqref{eq:M.2}}= e^{-\lambda t}X_s(P_{t-s}^\beta \phi)
	\overset{\eqref{asp:H1}}=e^{-\lambda s}X_s(\phi).
\end{equation}
	\begin{prop} \label{thm:T}
	Suppose that \eqref{asp:H1} holds. Then for each $\mu \in \mathcal M^o$, there exists a unique probability measure $\widetilde {\mathrm P_\mu}$ on $\mathbb W$ such that
\begin{equation} \label{eq:M.3}
	\frac{{\mathrm d}\widetilde{\mathrm P_\mu}|_{\mathscr F_t}}
	{{\mathrm d}\mathrm P_\mu|_{\mathscr F_t}}
	=\frac{e^{-\lambda t}X_t(\phi)}{\mu(\phi) },
	\quad t\geq 0.
\end{equation}
\end{prop}

\begin{remark}
	Denote by $E_\partial^*$ as a topological space with exactly the same elements as $E_\partial$, but with a different topology assuming that $\partial$ is an isolated point not contained in $E$.
	If one change the setting by assuming that $\xi$ is an $E_\partial^*$-valued Hunt process (instead of being an $E_\partial$-valued Hunt process), then the corresponding $(\xi,\psi)$-superprocess $X$ has an $\mathcal M$-valued Hunt version (See \cite[Theorem 5.12]{Li2011Measure-valued}).
	In this case, since $X$ has c\`adl\`ag sample paths, the existence of a measure $\widetilde {\mathrm P_\mu}$ on space of $\mathcal M$-valued c\`adl\`ag sample paths so that \eqref{eq:M.3} holds follows directly from the standard martingale change of measure theory (See \cite[Lemma 18.18]{Kallenberg2002Foundations}).
	However, this setting will exclude some interesting cases. For example, if $E$ is a standard ball in the Euclidean space, and $\xi$ is a Brownian motion killed (i.e. forced to take value $\partial$) upon leaving the ball, then $\xi$ is a $E_\partial$-valued Hunt process, but not a $E_\partial^*$-Hunt process.
\end{remark}


	
	This kind of martingale measure transformation for branching processes and measure-valued branching processes have been widely studied.
	We refer to the earlier papers {\color{red}\cite{EnglanderKyprianou2004Local,Evans1993Two,Penisson2010Conditional,RoellyRouault1989Processus}}
and the references therein.
	For recent developments, we refer to \cite{ChampagnatRoelly2008Limit,RenSongSun2020Spine,RenSongZhang2018Williams}.
	It is known that the process $\{(X_t)_{t\geq 0}; \widetilde{\mathrm P_\mu}\}$ can be characterized by the so called spine decomposition theorem.
	


	\begin{thm} \label{thm:Q}
	Suppose that \eqref{asp:H1}, \eqref{asp:H2} and \eqref{asp:H3} hold, then
	for any $\mu \in \mathcal M^o$,
\[
	\lim_{s \rightarrow \infty} \mathrm P_\mu(A |X_s\neq \mathbf 0)=
	\widetilde{\mathrm P_\mu}(A),
	\qquad 
	%\mu \in \mathcal M^o, 
	A\in \bigcup_{t\geq 0}\mathscr F_t.
\]
\end{thm}
The process  $\{(X_t)_{t\geq 0}; \widetilde{\mathrm P_\mu}\}$  is called the  Q-process of our superprocess $X$.

	Our second result is about the asymptotic behavior of the extinction probability.
	Define
\begin{equation}\label{eq:M.35}
	l(x)
	:= \int_{\frac{1}{\phi(x)}}^\infty u\phi(x) \ln \big(u\phi(x)\big) \pi(x, \mathrm du),
	\quad x\in E.
\end{equation}

\begin{thm} \label{thm:E}
	Suppose that \eqref{asp:H1}, \eqref{asp:H2} and \eqref{asp:H3} hold.	
	Then there exists $k\in [0,\infty)$ such that for any $\mu \in \mathcal M^o$,
\begin{equation}\label{eq:M.4}
	\lim_{t\rightarrow\infty} e^{-\lambda t}\mathrm P_\mu(X_t \neq \mathbf 0)
	=k\mu(\phi).
\end{equation}
	Moreover, $k>0$ if and only if $\nu(l)<\infty$.
\end{thm}

	Our third result is about the moment property of the Yaglom limit and the quasi-stationary distribution of $X$.
	For any probability measure $\mathbf P$ on $\mathcal M$, define
\[
	(\mathbf P\mathrm P)(\cdot)
	:= \int_{\mathcal M} \mathrm P_\mu(\cdot)\mathbf P({\mathrm d}\mu).
\]
	Any probability measure $\mathbf P$ on $\mathcal M^o$ will also be understood as its unique extension on $\mathcal M$ with  $\mathbf P(\{\mathbf 0\}) = 0$.
 We say a probability measure $\mathbf Q$ on $\mathcal M^o$ is a \emph{quasi-stationary distribution} (QSD) of $X$, if
\[
	(\mathbf Q \mathrm P) \left( X_t \in B \middle | X_t \neq \mathbf 0 \right)
	= \mathbf Q(B),
	\quad t\geq 0, B \in \mathscr B(\mathcal M).
\]
%	According to \cite[(1.5)]{LiuRenSongSun2020}, 
	According to \cite[(1.4)]{LiuRenSongSun2020}, 
	if a probability measure $\mathbf Q$ on $\mathcal M^o$ is a QSD of $X$, then there exists an $r\in (-\infty, 0)$ such that $(\mathbf Q\mathrm P)(X_t \neq \mathbf 0) = e^{rt}$ for all $t\geq 0$; and in this case, we call $r$ the \emph{mass decay rate} of $\mathbf Q$.
	It is proved in \cite[Theorem 1.2]{LiuRenSongSun2020} that, under Assumption \eqref{asp:H1}, \eqref{asp:H2} and \eqref{asp:H3},
	(1) for each $r\in [\lambda, 0)$, there exists a unique QSD $\mathbf Q_r$ of $X$ with mass decay rate $r$;
	and (2) for each $r\in (-\infty, \lambda)$, there is no QSD for $X$ with mass decay rate $r$.
	In particular, 
	%by \cite[Theorem 1.1, Proposition 1.7]{LiuRenSongSun2020}, 
	by \cite[Theorem 1.1, Proposition 2.5]{LiuRenSongSun2020}, 
	$\mathbf Q_\lambda$ is the \emph{Yaglom limit} of $X$, in the sense that
\begin{equation}
	\mathrm P_\mu(X_t \in \cdot | X_t \neq \mathbf 0)
	\xrightarrow[t\to \infty]{\text{weakly}} \mathbf Q_\lambda,
	\quad \mu \in \mathcal M^o.
\end{equation}

\begin{thm}\label{thm:L}
	Suppose that Assumptions \eqref{asp:H1}, \eqref{asp:H2} and \eqref{asp:H3} hold.
	Then (1)
	for any $r\in [\lambda, 0)$ and $\gamma \in (0, \frac{r}{\lambda})$, it holds that $\int_{{\mathcal M}_f(E)}\mu(\phi)^\gamma\mathbf Q_r({\mathrm d}\mu)<\infty$;
	(2) 
%	If $r=\lambda$ (which is equivalent to $\frac{r}{\lambda}=1$), 
	If $r=\lambda$,
	then $\int_{{\mathcal M}_f(E)}\mu(\phi)\mathbf Q_\lambda({\mathrm d}\mu) = k^{-1}$, where $k$ is  the constant in Theorem \ref{thm:E};
	(3) 
	If $r\in (\lambda, 0)$ %(which implies that $\frac{r}{\lambda}\in(0,1)$) 
	and $U^{(r)}$ is an $\mathcal M$-valued random variable with distribution $ \mathbf Q_r$, then $ \mathbf Q_r(\langle\phi, U^{(r)}\rangle^{r/\lambda})=\infty.$
\end{thm}

	Our fourth result characterizes the invariant distribution for the Q-process of $X$.
	A probability $\mathbf Q$ on $\mathcal M$ is called an \emph{invariant distribution} of the Q-process of $X$ if
\[
	(\mathbf Q\widetilde{\mathrm P})(X_t \in \cdot )
	=\mathbf Q(\cdot),	\quad t\geq 0.
\]

\begin{thm}\label{thm:I}
	Suppose that \eqref{asp:H1}, \eqref{asp:H2} and \eqref{asp:H3} hold.
\begin{enumerate}
\item
	If $\nu(l)<\infty$, then the Q-process of $X$ has an invariant distribution $\mathbf Q$ given by
\[
	\int_{{\color{red}{\mathcal M}}} e^{-\mu(f)}\mathbf Q(\mathrm d\mu)
	=\frac{\int_{{\color{red}{\mathcal M}}}\mu(\phi)e^{-\mu(f)}\mathbf Q_\lambda({\mathrm d}\mu)} {\int_{{\color{red}{\mathcal M}}}\mu(\phi)\mathbf Q_\lambda({\mathrm d}\mu)}, \quad f\in \mathcal B_b(E, \mathbb R_+).
\]	
Moreover, for each $\mu\in\mathcal M^o$, we have
\[
	\widetilde{\mathrm P_\mu}(X_t \in \cdot )
	\xrightarrow[t\to \infty]{weakly} {\mathbf Q}(\cdot).
\]
\item
	If $\nu(l) = \infty$, then for each $\mu \in \mathcal M^o$, we have
\begin{align}
	& \widetilde{\mathrm P_\mu}\big(X_t(\phi) > C\big)
	\xrightarrow[t\to \infty]{} 1, \quad C\geq 0.
\end{align}
\end{enumerate}
\end{thm}

%RS Did not make changes in this section since we are probably going to delete it. 
\section{Proof of Proposition \ref{thm:T}} \label{sec:T}
	Extend $\psi$ to a function $ \psi^\partial$ on $E_\partial \times \mathbb R_+$ by
\[
	\psi^\partial(x, z) = \begin{cases}
	\psi(x,z), &\quad x \in E, z\in \mathbb R_+,
	\\ 0, &\quad x=\partial, z\in \mathbb R_+.
	\end{cases}
\]
	For any $f \in \mathcal B_b(E_\partial,\mathbb R_+)$,  according to \cite[Proposition 2.20]{Li2011Measure-valued}, there is a unique locally bounded non-negative map $(t,x)\mapsto V^\partial_tf(x)$ on $\mathbb R_+\times E_\partial$ such that
\begin{equation}\label{eq:T.004}
	V_t^\partial f(x) + \Pi_x\Big[\int_0^t \psi^\partial\big(\xi_s, V^\partial_{t-s} f(\xi_s)\big) {\mathrm d}s\Big]
	= \Pi_x[f(\xi_t)],
	\quad t\in \mathbb R_+, x\in E_\partial.
\end{equation}
	Here, the local boundedness of the map $(t,x) \mapsto V_t^\partial f(x)$ means that  for any $T\geq 0$,
\[
	\sup_{0\leq t\leq T, x\in E_\partial} V^\partial_tf(x) < \infty.
\]
	Let $\mathcal M_\partial$ denote the space of all finite Borel measures on $E_\partial$ equipped with the topology of weak convergence.
	According to \cite[Proposition 2.21 and Theorem 5.11]{Li2011Measure-valued}, there exists an $\mathcal M_\partial$-valued Hunt process $X^\partial =\{(X^\partial_t)_{t\geq 0}; (\mathrm P^\partial_\mu)_{\mu \in \mathcal M_\partial}\}$ such that
\begin{equation}\label{eq:T.005}
	\mathrm P^\partial_\mu[e^{- X^\partial_t(f)}]
	= e^{- \mu(V^\partial_tf)},
	\quad \mu\in \mathcal M_\partial, t\geq 0, f \in \mathcal B_b(E_\partial,\mathbb R_+).
\end{equation}
	This process $X^\partial$ is known as a $(\xi, \psi^\partial)$-superprocess.

	We can and will assume that $X^\partial$ is \emph{canonical}  i.e.
	(1) $(X^\partial_t)_{t\geq 0}$ is the coordinate process of $\mathbb D(\mathcal M_\partial)$, the space of $\mathcal M_\partial$-valued c\`adl\`ag paths on $\mathbb R_+$; and
	(2) $\mathrm P^\partial_\mu(\mathrm dw)$ is a probability transition kernel from $\mathcal M_\partial$ to $\mathbb D(\mathcal M_\partial)$.
	We will use $(\mathscr F^\partial_t)_{t\geq 0}$ to denote the natural filtration on $\mathbb D(\mathcal M_\partial)$.
	Define a Feynman-Kac semigroup of $\xi$ by
\begin{align} \label{eq:T.01}
	P_t^{\beta,\partial} f(x)
	:= \Pi_x\Big[\exp\Big\{\int_0^t \beta(\xi_r) \mathbf 1_{{\color{red}\{\xi_r\in E\}}} {\mathrm d}r \Big\} f(\xi_t) \Big],
	\quad f\in \mathcal B_b(E_\partial,\mathbb R), t\geq 0, x\in E_\partial.
\end{align}
	It is known (see \cite[Proposition 2.27]{Li2011Measure-valued}) that
\begin{equation} \label{eq:T.02}
	\mathrm P_\mu^\partial [X^\partial_t(f)] = \mu (P_t^{\beta,\partial} f),
	\quad t\geq 0, f \in \mathcal B_b(E_\partial,\mathbb R), \mu \in \mathcal M_\partial.
\end{equation}
	Note that for each $t\geq 0$ and $x\in E_\partial$,
\begin{align}
	& P_t^{\beta,\partial} (\phi \mathbf 1_E) (x)
	\overset{\eqref{eq:T.01}}= \Pi_x\Big[\exp\Big\{\int_0^t \beta(\xi_r) \mathbf 1_{{\color{red}\{\xi_r\in E\}}} {\mathrm d}r \Big\} \phi(\xi_t) \mathbf 1_{{\color{red}\{\xi_t \in E\}}} \Big]
	\\\label{eq:T.03}&\overset{\eqref{eq:M.15}} = \mathbf 1_{{\color{red}\{x\in E\}}}P_t^\beta \phi (x)
	\overset{\eqref{asp:H1}} = e^{\lambda t} \phi (x)\mathbf 1_{{\color{red}\{x\in E\}}}.
\end{align}
	Define a map $\Gamma: \mu \mapsto \Gamma \mu$ from $\mathcal M_\partial$ to $\mathcal M$ by taking
\[
	(\Gamma \mu)(B) = \mu(B), \quad B\in \mathscr B(E).
\]
	
\begin{lem} \label{thm:T.1}
	For each $\mu \in \mathcal M_\partial$,
$
	\{(\Gamma X^\partial_t)_{t\geq 0}; \mathrm P_\mu^\partial\} \overset{\text{f.d.d.}}= \{(X_t)_{t\geq 0}; \mathrm P_{\Gamma\mu}\}.
$
\end{lem}


\begin{proof}
\emph{Step 1.}
	We note that for any $t\geq 0$, $x\in E_\partial$ and $f\in \mathcal B_b(E,\mathbb R_+)$, it holds that $V_t^\partial f^\partial (x) = V_tf(x) \mathbf 1_{{\color{red}\{x\in E\}}}$ where for each $y\in E_\partial$, $f^\partial(y) := f(y) \mathbf 1_{{\color{red}\{y\in E\}}}$.
	This is elementary to verify using \eqref{eq:T.004}.

\emph{Step 2.}
	We note that for any $t\geq s>0$ and $f\in \mathcal B_b(E,\mathbb R_+)$ we have
\[
	\mathrm P_\mu^\partial [e^{- (\Gamma X_t^\partial)(f)} | \mathscr F_s^\partial]
	= e^{- (\Gamma X_s^\partial)(V_{t-s}f)}.
\]
	In fact, for any $t\geq s>0$ and $f\in \mathcal B_b(E,\mathbb R_+)$,
\begin{align}
	&\mathrm P_\mu^\partial [e^{- (\Gamma X_t^\partial)(f)} | \mathscr F_s^\partial]
	\overset{\text{Markov}}= \mathrm P_{X_s^\partial}^\partial [e^{- (\Gamma X_{t-s}^\partial)(f)} ]
	= \mathrm P_{X_s^\partial}^\partial [e^{- X_{t-s}^\partial(f^\partial)} ]
	\\ &\overset{\eqref{eq:T.005}}= e^{- X_s^\partial(V_{t-s}^\partial f^\partial)}
	\overset{\text{Step 1}} = e^{- (\Gamma X_s^\partial)(V_{t-s}f)}.
\end{align}
	
\emph{Step 3.}
	We verify that for any $n\in \mathbb N$, $0\leq t_1 < \dots <t_n<\infty$ and $\{f_k: k = 1,\dots,n\}\subset \mathcal B_b(E,\mathbb R_+)$ we have
\[
	\mathrm P_\mu^\partial \Big[\exp \Big\{- \sum_{k=1}^n(\Gamma X_{t_k}^\partial)(f_k)\Big\}\Big]
	= \mathrm P_{\Gamma \mu} \Big[\exp \Big\{- \sum_{k=1}^n X_{t_k}(f_k)\Big\}\Big].
\]
	In fact, for any $t\geq 0$ and $f\in  \mathcal B_b(E,\mathbb R_+)$, we have
\[
	\mathrm P_\mu^\partial [e^{- (\Gamma X_{t}^\partial)(f)}]
	\overset{\text{Step 2}}= e^{- (\Gamma \mu)(V_{t}f)}
	\overset{\eqref{eq:M.13}}= \mathrm P_{\Gamma \mu} [e^{- ( X_{t})(f)}].
\]
	This gives the desired result when $n=1$.
	Now suppose the desired result is true when $n=m$ for some $m\in \mathbb N$.
	Then for any $0\leq t_1 < \dots <t_{m+1}<\infty$ and $\{f_k: k = 1,\dots, m+1\}\subset \mathcal B_b(E,\mathbb R_+)$ we have
\begin{align}
	&\mathrm P_\mu^\partial \Big[\exp \Big\{- \sum_{k=1}^{m+1}(\Gamma X_{t_k}^\partial)(f_k)\Big\} \Big]
	\\&= \mathrm P_\mu^\partial \Big[\exp \Big\{- \sum_{k=1}^{m}(\Gamma X_{t_k}^\partial)(f_k)\Big\} \mathrm P_\mu^\partial[e^{-(\Gamma X^\partial_{t_{m+1}}) (f_{m+1})}|\mathscr F^\partial_{t_m}]\Big]
	\\&\overset{\text{Step 2}} = \mathrm P_\mu^\partial \Big[\exp \Big\{- \sum_{k=1}^{m}(\Gamma X_{t_k}^\partial)(f_k)\Big\} e^{- (\Gamma X_{t_m}^\partial)(V_{t_{m+1}-t_m}f_{m+1})}\Big]
	\\&= \mathrm P_{\Gamma \mu} \Big[\exp \Big\{- \sum_{k=1}^{m}( X_{t_k})(f_k)\Big\}e^{-X_{t_m}(V_{t_{m+1}-t_m}f_{m+1})}\Big]
	\\&\overset{\text{Markov, }\eqref{eq:M.13}}= \mathrm P_{\Gamma \mu} \Big[\exp \Big\{- \sum_{k=1}^{m}( X_{t_k})(f_k)\Big\}\mathrm P_{\Gamma \mu}[e^{-X_{t_{m+1}}(f_{m+1})}|\mathscr F_{t_m}]\Big].
	\\& = \mathrm P_{\Gamma \mu} \Big[\exp \Big\{- \sum_{k=1}^{m+1}( X_{t_k})(f_k)\Big\}\Big].
\end{align}
	The desired result in this step then follows by induction.
	
\emph{Final Step.} The desired result of this Lemma now follows from Step 3 and \cite[Theorem 1.17]{Li2011Measure-valued}.
\end{proof}


\begin{lem} \label{thm:T.2}
	For each $\mu \in \mathcal M_\partial$, $\mathrm P_\mu^\partial$-almost surely, $(\Gamma X^\partial_t)_{t\geq 0}$ is an $\mathcal M$-valued right continuous process.
\end{lem}
\begin{proof}
	Since $X^\partial$ is a Hunt process, we have $\mathrm P_\mu^\partial(\Omega_0) = 1$ where
\[
	\Omega_0 := \{t\mapsto X_t^\partial \text{ is an $\mathcal M_\partial$-valued right continuous process on $[0,\infty)$}\}.
\]
	According to \cite[Theorem A.20]{Li2011Measure-valued}, we know that $\mathbf 1_E$ is finely continuous relative to $\xi$.
	According to \cite[Proposition 5.9]{Li2011Measure-valued}, we know that
$
	\mathrm P_\mu(\Omega_1) = 1
$
	where
\[
	\Omega_1 := \{t \mapsto X_t^\partial(\mathbf 1_E) \text{ is an $\mathbb R$-valued right continuous process on $[0,\infty)$}\}.
\]
	To finish the proof, we only have to show
\begin{equation} \label{eq:T.2}
	\Omega_0\cap \Omega_1
	\subset \{t\mapsto \Gamma X_t^\partial \text{ is an $\mathcal M$-valued right continuous process on $[0,\infty)$}\}.
\end{equation}
	In fact it is clear that in $\Omega_0$ for each $t\geq 0$ and compactly supported continuous function $f$ on $E$,
\begin{equation}\label{eq:T.21}
	(\Gamma X^\partial_t)(f) = X^\partial_t(f\mathbf 1_E) = \lim_{r\downarrow t} X^\partial_t(f\mathbf 1_E) = \lim_{r\downarrow t} (\Gamma X^\partial_t)(f).
\end{equation}
	In $\Omega_1$, we have for each $t\geq 0$,
\begin{align} \label{eq:T.22}
	&(\Gamma X^\partial_t)(\mathbf 1_E)
	= X^\partial_t(\mathbf 1_E)
	= \lim_{r\downarrow t} X^\partial_r (\mathbf 1_{E})
	= \lim_{r\downarrow t} (\Gamma X^\partial_r)(\mathbf 1_E).
\end{align}
	The desired result \eqref{eq:T.2} then follows from \eqref{eq:T.21}, \eqref{eq:T.22} and Lemma \ref{thm:A.1}.
\end{proof}

\begin{lem} \label{thm:T.3}
	For each $\mu \in \mathcal M_\partial$ with $\mu(\phi\mathbf 1_E)>0$, there exists
	a unique measure $\widetilde {\mathrm P_{\mu}^\partial}$ on $\mathbb D(\mathcal M_\partial)$ such that
\[
	\frac{\mathrm d \widetilde {\mathrm P_\mu^\partial} |_{\mathscr F_t^\partial}}{\mathrm d \mathrm P^\partial_\mu|_{\mathscr F_t^\partial}} = \frac{e^{-\lambda t}(\Gamma X^\partial_t)(\phi)}{\mu(\phi\mathbf 1_E)}, \quad t\geq 0.
\]
\end{lem}
\begin{proof}
	The uniqueness is due to Carath\'eodory's extension theorem and the fact that $\cup_{t\geq 0}\mathscr F_t^\partial$ is a ring generating the $\sigma$-field of $\mathbb W$.
	For the existence, thanks to \cite[Lemma 18.18]{Kallenberg2002Foundations}, we only have to verify that
\[
	\Big(\frac{e^{-\lambda t} (\Gamma X_t^\partial)(\phi)}{\mu(\phi \mathbf 1_E)}\Big)_{t\geq 0} \text{is a non-negative $\mathscr F_t^\partial$-martingale with mean $1$ under $\mathrm P_\mu^\partial$.}
\]
	In fact, for each $0\leq s\leq t<\infty$, we have
\begin{align}
	& \mathrm P_\mu^\partial  \Big[\frac{e^{-\lambda t} (\Gamma X_t^\partial)(\phi)}{\mu(\phi \mathbf 1_E)}\Big| \mathscr F^\partial_s\Big]
	= \frac{e^{-\lambda t}}{\mu(\phi \mathbf 1_E)} \mathrm P_\mu^\partial  [X_t^\partial(\phi \mathbf 1_E) | \mathscr F^\partial_s]
	\overset{\text{Markov}}= \frac{e^{-\lambda t}}{\mu(\phi \mathbf 1_E)} \mathrm P^\partial_{X_s^\partial}[X_{t-s}^\partial (\phi \mathbf 1_E)]
	\\ & \overset{\eqref{eq:T.02}}= \frac{e^{-\lambda t}}{\mu(\phi \mathbf 1_E)} X_s^\partial \big(P^{\beta, \partial}_{t-s} (\phi\mathbf 1_E)\big)
	\overset{\eqref{eq:T.03}}= \frac{e^{-\lambda s}}{\mu(\phi \mathbf 1_E)} X_s^\partial (\phi\mathbf 1_E)
	= \frac{e^{-\lambda s} (\Gamma X_s^\partial)(\phi)}{\mu(\phi \mathbf 1_E)}.
	\qedhere
\end{align}
\end{proof}

\begin{proof}[Proof of Proposition \ref{thm:T}]
	The uniqueness is due to Carath\'eodory's extension theorem and the fact that $\cup_{t\geq 0}\mathscr F_t$ is a ring generating the $\sigma$-field of $\mathbb W$.
	For the existence, first note that there exists a unique $\mu^\partial \in \mathcal M_\partial$ such that $\Gamma \mu^\partial = \mu$ and $\mu^{\partial}(\{\partial\}) = 0$.
	According to Lemma \ref{thm:T.2} and \ref{thm:T.3} we know that $\widetilde {\mathrm P_{\mu^\partial}^\partial}$-almost surely, $(\Gamma X^\partial_t)_{t\geq 0}$ has $\mathcal M$-valued right continuous sample path.
	Therefore, there exists a probability measure $\widetilde{\mathrm P_\mu}$ on $\mathbb W$ such that
$
	\{(\Gamma X^\partial_t)_{t\geq 0}; \widetilde {\mathrm P^\partial_{\mu^\partial}}\} \overset{\text{d}}\sim \widetilde {\mathrm P_\mu}.
$
	Now for any $n\in \mathbb N, t\geq 0$, $(t_i)_{i=1}^n \subset [0,t]$ and $(A_i)_{i=1}^n\subset \mathscr B(\mathcal M)$, we have
\begin{align}
	&\widetilde {\mathrm P_\mu}(\forall i\in \mathbb N\cap [1,n]: X_{t_i}\in A_i)
	= \widetilde {\mathrm P_{\mu^\partial}^\partial} (\forall i\in \mathbb N\cap [1,n]: \Gamma X^\partial_{t_i}\in A_i)
	\\ &\overset{\text{Lemma \ref{thm:T.3}}}= \mathrm P_{\mu^\partial}^\partial \Big[\frac{e^{-\lambda t}(\Gamma X^\partial_t)(\phi)}{\mu^\partial(\phi\mathbf 1_E)};\forall i\in \mathbb N\cap [1,n]: \Gamma X_{t_i}^{\partial}\in A_i\Big]
	\\ & \overset{\text{Lemma \ref{thm:T.1}}}= \mathrm P_\mu \Big[\frac{e^{-\lambda t}X_t(\phi)}{\mu(\phi)};\forall i\in \mathbb N\cap [1,n]: X_{t_i}\in A_i\Big].
\end{align}
	For a given $t\geq 0$, one can easily verify that
\begin{equation}
	\Big\{B \in \mathscr F_t: \widetilde {\mathrm P_\mu}(B) = \mathrm P_\mu \Big[\frac{e^{-\lambda t}X_t(\phi)}{\mu(\phi)}; B\Big]\Big\}
\end{equation}
	is a $\lambda$-system and that
\begin{equation}
	\Big\{ \bigcap_{i=1}^n\{X_{t_i}\in A_i\}: n\in \mathbb N, (t_i)_{i=1}^n \subset [0,t], (A_i)_{i=1}^n\subset \mathscr B(\mathcal M) \Big\}
\end{equation}
	is a $\pi$-system.
	From this and the $\pi$-$\lambda$ theorem, we can verify that for each $t\geq 0$ and $B \in \mathscr F_t$, we have
\[
	\widetilde {\mathrm P_\mu}(B)
	= \mathrm P_\mu\Big[\frac{e^{-\lambda t}X_t(\phi)}{\mu(\phi)}; B\Big].
	\qedhere
\]
\end{proof}


\section{Proof of Theorem \ref{thm:Q}} \label{sec:Q}
	
	Let us first recall some known results from \cite{LiuRenSongSun2020}.
	It is easy to see that the operators $(V_t)_{t\geq 0}$
	given by \eqref{eq:M.1} can be extended uniquely to a family of operators $(\overline V_t)_{t\geq 0}$ on $\mathcal B(E,[0,\infty])$ such that for any $t\geq 0$, $f_n \uparrow f$ pointwisely in  $\mathcal B(E, [0,\infty])$ implies that $\overline V_tf_n \uparrow \overline V_tf$ pointwisely.
	With some abuse of notation, we still write $V_t = \overline V_t$ for $t\geq 0$, and call $(V_t)_{t\geq 0}$ \emph{the extended cumulant semigroup} of the superprocess $X$.
	Define $v_t = V_t(\infty  \mathbf 1_E)$ for $t\geq 0$, then it holds that
	\begin{equation} \label{eq:Q.04}
	\mathrm P_\mu (X_t = \mathbf 0)
	= e^{- \mu (v_t)},
	\quad \mu \in \mathcal M, t\geq 0.
	\end{equation}
%	According to \cite[(1.10)]{LiuRenSongSun2020}) we have
By \cite[(2.5)]{LiuRenSongSun2020}) we have
\begin{equation}\label{eq:Q.05}
	\mu(v_t) > 0, \quad \mu \in \mathcal M^o, t \geq 0.
\end{equation}
%	According to \cite[Proposition 1.3]{LiuRenSongSun2020}, for each $\mu\in \mathcal M$ we have
It follows from \cite[Proposition 2.1]{LiuRenSongSun2020} that for any $\mu\in \mathcal M$ we have
\begin{equation} \label{eq:Q.055}
	\mu(v_t) \xrightarrow[t\to \infty]{} 0.
\end{equation}
%	According to \cite[(1.17)]{LiuRenSongSun2020}, for each 
According to \cite[(2.12)]{LiuRenSongSun2020}, for any
$t>0$ and $\mu \in \mathcal M^o$ we have
\begin{equation} \label{eq:Q.056}
	\text{$\mu(v_t) = \nu(v_t)\mu(\phi) (1+C_{\mu,t}^{\eqref{eq:Q.056}})$ for some $C_{\mu,t}^{\eqref{eq:Q.056}}\in \mathbb R$ with $\lim_{t\to \infty}|C_{\mu,t}^{\eqref{eq:Q.056}}| = 0$.}
\end{equation}
%	According to \cite[(2.14)]{LiuRenSongSun2020}, for each 
By \cite[(3.14)]{LiuRenSongSun2020}, for any 
$t>0$ and $x\in E$ we have
\begin{equation}  \label{eq:Q.0565}
	\text{$v_t(x) = \phi(x) \nu(v_t) C_{t,x}^{\eqref{eq:Q.0565}}$ for some $C_{t,x}^{\eqref{eq:Q.0565}}\geq 0$ with $\varlimsup_{t\to \infty}\sup_{x\in E}C_{t,x}^{\eqref{eq:Q.0565}}<\infty$.}
\end{equation}
%	According to \cite[(2.20)]{LiuRenSongSun2020}, for each 
It follows from 	\cite[(3.20)]{LiuRenSongSun2020} that for any
	$t\geq 0$, we have
\begin{equation}\label{eq:Q.057}
	\lim_{s\to \infty} \frac{\nu(v_s)}{\nu(v_{s-t})} = e^{\lambda t}.
\end{equation}


\begin{proof}[Proof of Theorem \ref{thm:Q}]
	Fix  $\mu\in \mathcal M^o$, $t\ge 0$ and $A\in\mathscr F_t$.
	
\emph{Step 1.} We show that
	\begin{equation} \label{eq:Q.1}
		\frac{\mathrm P_{X_t}(X_{s-t} \neq \mathbf 0)}{\mathrm P_\mu(X_{s} \neq \mathbf 0)}
		\xrightarrow[s\to \infty]{} \frac{e^{-\lambda t}X_t(\phi)}{\mu(\phi) },
		\quad \mathrm P_\mu\text{-a.s.}
\end{equation}
	Notice that for any $\bar \mu\in \mathcal M$,
	\begin{align}
		&\lim_{s\rightarrow\infty}\dfrac{ \bar \mu(v_{s-t}) }{ \mu(v_s) }
		\overset{\eqref{eq:Q.056}}=\lim_{s\to \infty} \frac{\nu(v_{s-t})\bar \mu (\phi) (1+C_{\bar \mu,s-t}^{\eqref{eq:Q.056}})}{\nu(v_s)\mu(\phi)(1+C_{\mu,s}^{\eqref{eq:Q.056}})}
		\overset{\eqref{eq:Q.056}}= \frac{\bar \mu(\phi)}{\mu(\phi)}\lim_{s\to \infty} \frac{\nu(v_{s-t}) }{\nu(v_s)}
		\\\label{eq:Q.25}&\overset{\eqref{eq:Q.057}}= \frac{e^{-\lambda t} \bar \mu(\phi)}{\mu(\phi)}.
	\end{align}
	Thus we have that, $\mathrm P_\mu$-almost surely,
	\begin{equation}
		\lim_{s\to\infty}\frac{\mathrm P_{X_t}(X_{s-t}\neq \mathbf 0)}{\mathrm P_\mu(X_s\neq \mathbf 0)}
		\overset{\eqref{eq:Q.04}}=\lim_{s\to\infty}\frac{1-e^{- X_t(v_{s-t}) }}{1-e^{- \mu(v_s) }}
		\overset{\eqref{eq:Q.05},\eqref{eq:Q.055}}=\lim_{s\rightarrow\infty}\dfrac{ X_t(v_{s-t}) }{ \mu(v_s) }
		\overset{\eqref{eq:Q.25}}= \frac{e^{-\lambda t} X_t(\phi)}{\mu(\phi)}.
	\end{equation}

\emph{
	Step 2. We show that there exist (deterministic) $c,s_0>0$ such that for any $s\geq s_0$,
	\begin{equation} \label{eq:Q.2}
		\frac{\mathrm P_{X_t}(X_{s-t} \neq \mathbf 0)}{\mathrm P_\mu(X_{s} \neq \mathbf 0)}
		\leq cX_t(\phi), \quad \mathrm P_\mu\text{-a.s.}
	\end{equation}
}
	Notice that, $\mathrm P_\mu$-almost surely,
	\begin{align}
		& \frac{\mathrm P_{X_t}(X_{s-t}\neq \mathbf 0)}{\mathrm P_\mu(X_s\neq \mathbf 0)}
		\overset{\eqref{eq:Q.04}}= \frac{1-e^{- X_t(v_{s-t}) }}{1-e^{- \mu(v_s) }}
		\leq \frac{X_t(v_{s-t}) }{1-e^{- \mu(v_s) }}
		\\\label{eq:Q.3}&\overset{\eqref{eq:Q.05},\eqref{eq:Q.055}}= \frac{X_t(v_{s-t}) }{ \mu(v_s)} C_{{\color{red}\mu} ,s}^{\eqref{eq:Q.3}},
		\\& \quad \text{for some deterministic $C_{{\color{red}\mu} ,s}^{\eqref{eq:Q.3}}>0$ with $\lim_{s\to \infty} C_{{\color{red}\mu},s}^{\eqref{eq:Q.3}} = 1$,}
		\\\label{eq:Q.4}&\overset{\eqref{eq:Q.0565}}\leq \frac{X_t(\phi) \nu(v_{s-t}) \sup_{x\in E} C_{s-t,x}^{\eqref{eq:Q.0565}}}{ \mu(v_s)} C_{{\color{red}\mu},s}^{\eqref{eq:Q.3}}.
	\end{align}
%	Then notice that
Since
	\begin{align}
		&  \varlimsup_{s\to \infty}\frac{\nu(v_{s-t}) \sup_{x\in E} C_{s-t,x}^{\eqref{eq:Q.0565}}}{ \mu(v_s)} C_{{\color{red}\mu} ,s}^{\eqref{eq:Q.3}}
		\overset{\eqref{eq:Q.25}} = \frac{e^{-\lambda t} \nu(\phi)}{\mu(\phi)} \varlimsup_{s\to \infty}C_{{\color{red}\mu} ,s}^{\eqref{eq:Q.3}} \sup_{x\in E}C_{s-t,x}^{\eqref{eq:Q.0565}}
		\\\label{eq:Q.5}&\overset{\eqref{eq:Q.3}} = \frac{e^{-\lambda t} \nu(\phi)}{\mu(\phi)} \varlimsup_{s\to \infty} \sup_{x\in E}C_{s-t,x}^{\eqref{eq:Q.0565}}
		\overset{\eqref{eq:Q.0565}}< \infty.
	\end{align}
	The desired result in this step follows from \eqref{eq:Q.4} and \eqref{eq:Q.5}.
	
	\emph{Final Step.} Now due to Steps 1 and 2, we can use the dominated convergence theorem to get
\begin{align}
&  \mathrm P_\mu(A|X_s\neq \mathbf 0)
	=\frac{\mathrm P_\mu(A, X_s \neq \mathbf 0)}{\mathrm P_\mu(X_s\neq \mathbf 0)}
	\overset{\text{Markov}}=\frac{\mathrm P_\mu\big(\mathrm P_{X_t}(X_{s-t} \neq \mathbf 0);A\big)}{\mathrm P_\mu(X_s\neq \mathbf 0)}.
	\\&\xrightarrow[s\to \infty]{\text{DCT}} \mathrm P_\mu\Big[\frac{e^{-\lambda t}X_t(\phi)}{\mu(\phi) }; A\Big]
	\overset{\eqref{eq:M.3}} = \widetilde{\mathrm P_\mu}[A].
	\qedhere
\end{align}
\end{proof}

\section{Proof of Theorem \ref{thm:E}} \label{sec:E}
	In order to prove Theorem \ref{thm:E}, we will need 
the spine decomposition theorem for superprocess $X$ in a special case, see Lemma \ref{thm:E.2} below.	
	To present this result, we first recall the \emph{Kuznetsov  measures} of the superprocess $X$.
	According to \cite[Proposition 1.3]{LiuRenSongSun2020} we have $v_t(x)<\infty$ for any $t>0, x\in E$.
	Therefore
\begin{align}
	\mathrm P_{\delta_x}(X_t = \mathbf 0) \overset{\eqref{eq:Q.04}}
	= e^{-v_t(x)} > 0, \quad t>0, x\in E.
\end{align}
	Now according to \cite[Section 8.4]{Li2011Measure-valued}, there is a unique family of $\sigma$-finite measures
	$(\mathrm N_x)_{x\in E}$
	on $\mathbb W$ such that
	(1) $\mathrm N_x(w_0 \neq \mathbf 0) = 0$ for any $x\in E$;\\
{\color{red}\bf RL: I only find Kuznetsov measure should satisfy: $w_t\to 0$, and $w_t/w_t(1)\to \delta_x$ as $t\to 0+$, $\mathrm N_x$ a.e.? And only defined on the space of paths on $(0,\infty)$?\\
YX: Since $w_t\to 0$, we may extend to the path space on  $(0,\infty)$.\\
RS: Better delete with 0 as a trap.}\\
	(2) $\mathrm N_x (\forall t > 0, w_t =\mathbf 0) =0$ for any $x\in E$;
	and (3) for any $\mu \in \mathcal M$, if $\mathcal N$ is a Poisson random measure on $\mathbb W$ with intensity
	\[
	(\mu\mathrm N)(\mathrm dw):=
	\int_{x\in E} \mathrm N_x(\mathrm dw)\mu(\mathrm dx), \quad w\in \mathbb W,
	\]
	then
	\begin{equation}
	\{(X_t)_{t> 0};\mathrm P_\mu\}
	\overset{\text{f.d.d.}}= \Big(\int_{\mathbb W} w_t\mathcal N(\mathrm dw)\Big)_{t> 0}.
	\end{equation}
	This family of measures $(\mathrm N_x)_{x\in E}$ is known as the Kuznetsov measures of $X$.
	It can be verified using Campbell's formula for Poisson random measures that for any $\mu\in \mathcal M$ and $t>0$,
\begin{equation} \label{eq:E.13}
	\mu(P_t^\beta f)
	\overset{\eqref{eq:M.2}}= \mathrm P_\mu[X_t(f)]
	=(\mu \mathrm N) [w_t(f)],
	\quad f\in \mathcal B_b(E, \mathbb R_+)
\end{equation}
	and
\begin{equation} \label{eq:E.14}
	(\mu\mathrm N) (w_t\neq \mathbf 0)
	= - \log \mathrm P_{\mu}(X_t = \mathbf 0).
\end{equation}
	It follows from \eqref{asp:H1} that for any $\mu\in \mathcal M^o$ there exists a probability measure $\widetilde {\mu \mathrm N}^{(t)}$ on $\mathbb W$ such that
	\begin{align}\label{eq:E.15}
	& \frac{\widetilde {\mu \mathrm N}^{(t)}(\mathrm dw) }{(\mu \mathrm N)(\mathrm dw)}
	= \frac{e^{-\lambda t}w_t(\phi)}{\mu(\phi)}, \quad w\in \mathbb W.
	\end{align}

	{\bf RS: The following paragraph will need to be changed if we get rid of the Hunt assumption.}
	Another ingredient for the spine decomposition theorem is the so-called spine process whose distribution will be given now.
	Recall that the spatial motion of the superprocess $\{(\xi_t)_{t\geq 0}; (\Pi_x)_{x\in E_\partial}\}$ is an $E_\partial$-valued Hunt process.
	We will assume, without loss of generality, that $\xi$ is canonical,  i.e., (1) $(\xi_t)_{t\geq 0}$ is the coordinate process of
	$\mathbb D(E_\partial)$,
	the space of $E_\partial$-valued c\`adl\`ag paths on $\mathbb R_+$; and
	(2) $\Pi_x(\mathrm d\xi)$ is a probability transition kernel from $E_\partial$ to $\mathbb D(E_\partial)$.
	Let  $(\mathscr F_t^{\xi})_{t\geq 0}$ be the natural filtration of process $(\xi_t)_{t\geq 0}$.
	For any probability measure $\mu$ on $E$, define
\begin{align}
	& (\mu \Pi)[\cdot] :=
	\int_{x\in E} \Pi_x[\cdot]\mu(\mathrm dx)
\end{align}
and let the probability $\widetilde {\mu \Pi}$ on $\mathbb D(E_\partial)$ be given by a martingale change of measure (\cite[Lemma 18.18]{Kallenberg2002Foundations}) so that
\begin{align}
	&  \frac{\mathrm d (\widetilde{\mu \Pi})|_{\mathscr F^\xi_t}}{\mathrm d (\mu \Pi)|_{\mathscr F^\xi_t}}
	=\frac{e^{\int_0^t \beta(\xi_s)ds}\phi(\xi_t) \mathbf 1_{\{t<\zeta\}}}{e^{\lambda t}\mu(\phi)},
	\quad t\geq 0.
\end{align}
We also define for any probability measure $\mu$ on $E$ that
\begin{equation}
	\widetilde \mu(\mathrm dx)
	:= \frac{\phi(x)}{\mu(\phi)} \mu(\mathrm dx),
	\quad x\in E.
\end{equation}
Note that
\begin{equation} \label{eq:E.162}
	\widetilde \nu(\mathrm dx)
	\overset{\eqref{asp:H1}}= \phi(x) \nu(\mathrm dx),
	\quad x\in E.
\end{equation}

\begin{prop} \label{thm:E.15}
	For any probability measure $\mu$ on $E$, $\{(\xi_t)_{t\geq 0}; \widetilde{\mu \Pi}\}$ is an $E$-valued Markov process with initial distribution $\widetilde \mu$ and transition kernel $(S_t)_{t\geq 0}$ given by
\begin{equation} \label{eq:E.17}
	S_t f(x)
	:= \frac{1}{\phi(x)e^{\lambda t}}P_t^\beta (\phi f) (x),
	\quad t\geq 0, f\in \mathcal B_b(E,\mathbb R^+),x\in E.
\end{equation}
	 Moreover, $\widetilde \nu$ is an ergodic measure of $(S_t)_{t\geq 0}$.
\end{prop}

\begin{remark}
	For the definition of ergodic measure of a given Markov semigroup, we refer our reader to \cite[Section 3.2]{DaPratoZabczyk1996Ergodicity}.
\end{remark}



	Let us now construct random elements
  	$\big\{X, \xi, N, (s_k, y_k,w^{(k)})_{k\in \mathbb N}; \mathrm Q\big\}$ so that:
\begin{enumerate}[label=(Q\arabic*),ref=Q\arabic*]
\item \label{asp:Q1}
	$\{X;\mathrm Q\}$ is a $\mathbb W$-valued random element with distribution $\mathrm P_\nu$;
{\\\color{red} YX: I suggest put the above term to the end of the item, and say that $X$ is independent of $(\xi, N, (s_k, y_k,w^{(k)})_{k\in \mathbb N}$. Then we do not need to condition on $X$.\\
RS: Agree. We should also assume that $N$ and $M$ are independent given $\xi$,
and that $\{(w^{(k)})_{k\in \mathbb N}\}$ is also independent of $N$. We should also make the formulation of the this spine decomposition and the spine decomposition in Section 5 as parallel as possible.}
\item \label{asp:Q2}
		$\{\xi = (\xi_t)_{t\in \mathbb R}; \mathrm Q(\cdot | X)\}$
%	 	 is a two-sided stationary process with
is a two-sided right continuous stationary Markov process on $E$ with
		 \[\{(\xi_t)_{t \geq 0}; \mathrm Q(\cdot | X)\} \overset{\text{d}} = \{(\xi_t)_{t \geq 0}; \widetilde{\nu\Pi}\};\]
\item \label{asp:Q3}
	$\{N; \mathrm Q(\cdot |X,\xi)\}$ is a Poisson random measure on $\mathbb R\times \mathbb W$ with intensity
\[
	2 \sigma(\xi_s)^2 {\mathrm d}s \cdot \mathrm N_{\xi_s}({\mathrm d}w),
	\quad (s,w)\in \mathbb R\times \mathbb W;
\]
\item \label{asp:Q4}
	$(s_k, y_k)_{k\in \mathbb N}$ is a sequence of $\mathbb R \times \mathbb R_+$-valued random elements
	such that, under $\mathrm Q(\cdot | X, \xi, N)$,
\[
	M(\mathrm ds,\mathrm dy)
	:= \sum_{k\in \mathbb N} \delta_{(s_k, y_k)}(\mathrm ds,\mathrm dy), \quad (s,y)\in \mathbb R \times \mathbb R_+;
\]	
is a Poisson random measure on $\mathbb R \times \mathbb R_+$ with intensity
\[
	\mathrm ds \cdot y \pi(\xi_s, \mathrm dy), \quad (s,y)\in \mathbb R \times \mathbb R_+;
\]
\item \label{asp:Q5}
	$\{(w^{(k)})_{k\in \mathbb N}; \mathrm Q(\cdot|\mathscr G)\}$ is a sequence of independent $\mathbb W$-valued random elements such that
\[
	\{w^{(k)}; \mathrm Q(\cdot| \mathscr G)\}
	\overset{\text{d}}\sim \mathrm P_{y_k\delta_{\xi_{s_k}}},
	\quad k\in \mathbb N,
\]
	where
\[
	\mathscr G
	:= \sigma(X, \xi, N, (s_k, y_k)_{k\in \mathbb N}).
\]
\end{enumerate}
	For the existence of Poisson random measures with given intensity, see \cite[Theorem 2.4]{Kyprianou2014Fluctuations} for example.
	
	For $-\infty \leq a < b \leq t<\infty$, define
\begin{equation} \label{eq:E.4}
	Z_t^{(a,b]}
	:= \int_{(a,b]\times \mathbb W} w_{t-s} N(\mathrm ds,\mathrm dw) + \sum_{k\in \mathbb N} w^{(k)}_{t-s_k} \mathbf 1_{\{s_k \in (a,b]\}}.
\end{equation}
	From \eqref{asp:Q4}, \eqref{asp:Q5} and \cite[Theorem 3.2(i)]{Kallenberg2017Random} we know that
\begin{equation}\label{eq:E.45}
\begin{minipage}{0.9\textwidth}
	under $\mathrm Q(\cdot | X, \xi, N)$, the random measure
\[
	R(\mathrm ds,\mathrm dw)
	:=\sum_{k\in \mathbb N} \delta_{(s_k, w^{(k)})}(\mathrm ds,\mathrm dw),
	\quad (s,w)\in \mathbb R\times \mathbb W
\]
	is a Poisson random measure on $\mathbb R \times \mathbb W$ with intensity
	\[
	\mathrm ds \cdot \int_{y\in (0,\infty)} y \pi(\xi_s, \mathrm dy)\cdot \mathrm P_{y\delta_{\xi_s}}(\mathrm dw).
	\]
\end{minipage}
\end{equation}
	Using this and \cite[Theorem 1.5 \& Corollary 1.6]{RenSongSun2020Spine}, we have the following spine decomposition theorem:
\begin{lem} \label{thm:E.2}
	It holds that
	\[
	\{(X_t + Z_t^{(0,t]})_{t\geq 0}; \mathrm Q\} \overset{\text{f.d.d.}}\sim \{(X_t)_{t\geq 0}; \widetilde{\mathrm P_\nu}\},
	\]
	and for each $T\geq 0$,
\[
	\{(Z_t^{(0,t]})_{t\in [0,T]}; \mathrm Q\} \overset{\text{f.d.d.}} \sim \{ (X_t)_{t\in [0,T]}; \widetilde{{\color{red}\nu}\mathrm N}^{(T)}\}.
\]
\end{lem}
	We will also need the following two propositions whose proofs are postponed later.
\begin{prop}\label{thm:E.4}	
	(1) If $\nu(l)<\infty$,
	then for any $\epsilon>0$,
\[
	\sum_{k\in \mathbb N} \mathbf 1_{\{s_k \leq 0\}} y_k e^{\epsilon s_k} \phi(\xi_{s_k})
	< \infty,
	\quad \mathrm Q\text{-a.s.}
\]
	(2) If  $ \nu(l)=\infty$,
	then for any $\epsilon>0$ and any $s_0\geq 0$,
   	\begin{equation}
	\int^{-s_0}_{-\infty} {\mathrm d}s
	\int_{\frac{e^{-\epsilon s}}{\phi(\xi_s)}}^\infty y\pi(\xi_s,{\mathrm d}y)
     	=\infty,
	\quad {\mathrm Q}\text{-a.s.}
	\end{equation}
\end{prop}
\begin{prop} \label{thm:E.5}
 	There exist $s_0, \epsilon, \theta>0$ and $\delta > 0$  such that for all $x\in E, s>s_0$ and $y\geq \frac{e^{\epsilon s}}{ \phi(x)}$, it holds that $\mathrm P_{y \delta_{x}}\big(w_{s}(\phi)>\theta\big) > \delta$.
\end{prop}

\begin{proof}[Proof of Theorem \ref{thm:E}]
\emph{Step 1.} We note that for any $-\infty < a < b \leq t<\infty$ and $s\in \mathbb R$,
\[
	\{Z_t^{(a,b]}; \mathrm Q\}
	\overset{\text{d}}= \{Z_{t+s}^{(a+s,b+s]}; \mathrm Q\}.
\]	
	This is due to the facts that $\{(\xi_t)_{t\in \mathbb R};\mathrm Q\}$ is a stationary Markov process (Proposition \ref{thm:E.15}) and that the random measures $N$ and $M$ are defined in a time-homogeneous way.
	We omit the details of the verification.
	
\emph{Step 2.} We show that there exists $k\in [0,\infty)$ such that \eqref{eq:M.4} holds when $\mu = \nu$.
	Note for any $t\geq 0$,
\begin{align}
	&e^{-\lambda t}\mathrm P_\nu(X_t\neq \mathbf 0)
	\overset{\eqref{eq:M.3}, \eqref{asp:H1}}= \widetilde{\mathrm P_\nu}[X_t(\phi)^{-1}]
	\overset{\text{Lemma \ref{thm:E.2}}}=\mathrm Q\big[ \big(X_t(\phi) +Z^{(0,t]}_t(\phi)\big)^{-1}\big]
	\\\label{eq:E.5} &\overset{\text{Step 1}}=\mathrm Q\big[ \big(X_t(\phi) +Z^{(-t,0]}_0(\phi)\big)^{-1}\big].
\end{align}
	We claim that
\begin{equation}\label{eq:E.51}
	X_t(\phi) \xrightarrow[t\to \infty]{} 0, \quad \mathrm Q\text{-a.s.}
\end{equation}
	In fact, by \eqref{eq:M.25}, we know that $\big\{\big(e^{-\lambda t} X_t(\phi)\big)_{t\geq 0}; \mathrm Q\big\}$ is a non-negative martingale.
	So by the martingale convergence theorem and the fact that $\lambda < 0$, we know \eqref{eq:E.51} holds.
	Note  by monotonicity we have that
\begin{equation} \label{eq:E.52}
	Z^{(-t,0]}_0(\phi)
	\xrightarrow[t\to \infty]{} Z^{(-\infty,0]}_0(\phi),
	\quad \mathrm Q\text{-a.s.}
\end{equation}
Note also that
\begin{equation}\label{eq:E.53}
	\big(X_t(\phi) + Z^{(-t,0]}_0(\phi)\big)^{-1}
	\leq Z^{(-1,0]}_0(\phi)^{-1},
	\quad t\geq 1,
\end{equation}
	and
\begin{align}
	&\mathrm Q [Z^{(-1,0]}_0(\phi)^{-1}]
	\overset{\text{Step 1}} = \mathrm Q [Z^{(0,1]}_1(\phi)^{-1}]
	\overset{\text{Lemma \ref{thm:E.2}}}=
	\widetilde {\mathrm N}_\nu^{(1)} [w_1(\phi)^{-1}]
	\\&\overset{\eqref{eq:E.15}}= e^{-\lambda}
	\mathrm N_\mu(w_1\neq \mathbf 0)
	\overset{\eqref{eq:E.14}} = -e^{-\lambda} \log \mathrm P_{\nu}(X_1 = \mathbf 0)
	\\& \label{eq:E.54} \overset{\eqref{asp:H1}}< \infty.
\end{align}
	Now by \eqref{eq:E.5}, \eqref{eq:E.51}, \eqref{eq:E.52}, \eqref{eq:E.53} and the dominated  convergence theorem, we obtain
	\begin{equation}\label{eq:E.55}
		 \lim_{t\to\infty} e^{-\lambda t}\mathrm P_{\nu}(X_t \neq \mathbf 0)
	 =\mathrm Q[Z^{(-\infty,0]}_0(\phi)^{-1}]
	 =:k<\infty.
	\end{equation}

\emph{Step 3.}
	Let $k$ be given by Step 2, we show that \eqref{eq:M.4} holds for all $\mu\in \mathcal M^o$.
	In fact, note that for any $\mu\in \mathcal M^o$,
\begin{align}
	\lim_{t\to \infty}\frac{\mathrm P_\mu(X_t \neq \mathbf 0)}{\mathrm P_\nu(X_t \neq \mathbf 0)}
	\overset{\eqref{eq:Q.04}} = \lim_{t\to \infty}\frac{1- e^{-\mu(v_t)}}{1- e^{-\nu(v_t)}}
	\overset{\eqref{eq:Q.05}, \eqref{eq:Q.055}} = \lim_{t\to \infty}\frac{\mu(v_t)}{\nu(v_t)}
	\overset{\eqref{eq:Q.25}, \eqref{asp:H1}} = \mu(\phi).
\end{align}
	The desired result in this step follows from this and Step 2.

\emph{Step 4.}
	We show that  $\nu(l)<\infty$ is a sufficient condition for $k>0$.
	Assume that $\nu(l)<\infty$.
	Notice that
\begin{align}
	&\mathrm Q\Big[\int_{(-\infty,0]\times \mathbb W}w_{-s}(\phi) N({\mathrm d}s, {\mathrm d}w) \Big]
	= \mathrm Q\bigg[\mathrm Q\Big[\int_{(-\infty,0]\times \mathbb W}w_{-s}(\phi) N({\mathrm d}s, {\mathrm d}w) \Big | X, \xi \Big]\bigg]
	\\& \overset{\text{\eqref{asp:Q3},Campbell}}= \mathrm Q\Big[\int_{(-\infty,0]\times \mathbb W}w_{-s}(\phi) 2 \sigma(\xi_s)^2 {\mathrm d}s \cdot \mathrm N_{\xi_s}({\mathrm d}w) \Big]
	\\& \overset{\eqref{eq:E.13}, \eqref{asp:H1}}=\mathrm Q \Big[\int^0_{-\infty} e^{-\lambda s} 2\sigma(\xi_s)^2 \phi(\xi_s) \mathrm ds\Big]
	\\&\label{eq:E.64}\overset{\text{\eqref{asp:Q2}, Proposition \ref{thm:E.15}}}= 2\int^0_{-\infty} e^{-\lambda s} \widetilde\nu(\sigma^2 \phi) \mathrm ds
	<\infty,
	\end{align}
	where in the last inequality, we used the fact that $\sigma$, $\phi$ are bounded and $\lambda < 0$.
	Recall that $\mathscr G := \sigma(X, \xi, N, (s_k, y_k)_{k\in \mathbb N})$. Note that we have $\mathrm Q$-almost surely,
\begin{align}
	& \mathrm Q\Big(\sum_{k\in \mathbb N}
	w^{(k)}_{-s_k}(\phi) \mathbf 1_{\{s_k \leq 0\}}\Big|\mathscr G \Big)
	\overset{\eqref{asp:Q5}}= \sum_{k\in \mathbb N} \mathbf 1_{\{s_k \leq 0\}} \mathrm P_{y_k\delta_{\xi_{s_k}}}[w^{(k)}_{-s_k}(\phi)]
	\\ &\overset{\eqref{eq:M.2}}= \sum_{k\in \mathbb N}
	\mathbf 1_{\{s_k \leq 0\}} y_k (P_{-s_k}^\beta \phi)(\xi_{s_k})
	\overset{\eqref{asp:H1}} = \sum_{k\in \mathbb N} \mathbf 1_{\{s_k \leq 0\}} y_k e^{-\lambda s_k} \phi(\xi_{s_k})
	\overset{\text{Proposition \ref{thm:E.4} (1)}} < \infty.
\end{align}
	From this, we get that
\begin{equation}
	\sum_{k\in \mathbb N} w^{(k)}_{-s_k}(\phi) \mathbf 1_{\{s_k \leq 0\}}< \infty, \quad \mathrm Q\text{-a.s.}
\end{equation}
	Now, we have
\begin{equation}
	Z_0^{(-\infty, 0]}(\phi) \overset{\eqref{eq:E.4}}= \int_{(-\infty,0]\times \mathbb W}w_{-s}(\phi) N({\mathrm d}s, {\mathrm d}w) + \sum_{k\in \mathbb N} w^{(k)}_{-s_k}(\phi) \mathbf 1_{s_k \leq 0}
	\overset{\eqref{eq:E.64}} < \infty.
\end{equation}
	Therefore
\begin{align}
	& k \overset{\eqref{eq:E.55}}= \mathrm Q[Z^{(-\infty,0]}_0(\phi)^{-1}] > 0.
\end{align}

\emph{Final Step.} We show that $\nu(l)<\infty$ is a necessary condition for $k>0$.
	We claim that if $\nu(l) = \infty$, then there exists an $\theta>0$ such that,
\begin{equation} \label{eq:E.7}
	\int_{\mathbb R\times \mathbb W}\mathbf 1_{\{s\leq 0, w_{-s}(\phi)>\theta\}} R(\mathrm ds,\mathrm dw)
	= \infty,
	\quad \mathrm Q\text{-a.s.}
\end{equation}
	Using this claim, we have
\begin{equation}
	Z_0^{(-\infty, 0]}(\phi)
	\overset{\eqref{eq:E.4}} \geq \int_{\mathbb R\times \mathbb W} \mathbf 1_{\{s\leq 0, w_{-s}(\phi)>\theta\}} w_{-s}(\phi) R(\mathrm ds,\mathrm dw)
	= \infty,
	\quad \mathrm Q\text{-a.s.}
\end{equation}
	Thus
\begin{align}
	& k \overset{\eqref{eq:E.55}}= \mathrm Q[Z^{(-\infty,0]}_0(\phi)^{-1}] = 0.
\end{align}
	Now we prove claim \eqref{eq:E.7}.
	Let $s_0, \epsilon, \theta, \delta > 0$ be given as in Proposition \ref{thm:E.5}.
	We have $\mathrm Q$-almost surely,
\begin{align}
	  & \int_{(s,w)\in \mathbb R\times \mathbb W} 
%	  (1 \wedge \mathbf 1_{{\color{red}\{s\leq 0, w_{-s}(\phi)> \theta\}}}) 
\mathbf 1_{\{s\leq 0, w_{-s}(\phi)> \theta\}}
	  \mathrm ds  \int_{y\in (0,\infty)} y \pi(\xi_s, \mathrm dy) \mathrm P_{y\delta_{\xi_s}}(\mathrm dw)
	\\&= \int_{-\infty}^0 \mathrm ds \int_0^\infty  y \pi(\xi_s, \mathrm dy) \mathrm P_{y \sigma_{\xi_s}}\big(w_{-s}(\phi)>\theta\big)
	\\&\overset{\text{Proposition \ref{thm:E.5}}}\geq \delta \int_{-\infty}^{-s_0} \mathrm ds
 \int_{\frac{e^{-\epsilon s}}{\phi(\xi_s)}}^\infty  y \pi(\xi_s, \mathrm dy)
  	\overset{\text{Proposition \ref{thm:E.4} (2)}}= \infty.
\end{align}
	Now from \eqref{eq:E.45} and \cite[Theorem 2.7(i)]{Kyprianou2014Fluctuations} we know that $\mathrm Q$-almost surely
\begin{equation}
	\mathrm Q \Big(\int_{\mathbb R\times \mathbb W}\mathbf 1_{\{s\leq 0, w_{-s}(\phi)>\theta\}} R(\mathrm ds,\mathrm dw)
	= \infty\Big| X, \xi, N\Big)  = 1.
\end{equation}
	This implies the desired claim.
\end{proof}

\begin{proof}[Proof of Proposition \ref{thm:E.15}]
	It is elementary to verify (see also \cite{KimSong2008Intrinsic}) that
\begin{itemize}
\item
	$(S_t)_{t\geq 0}$ is a Markov semigroup;
\item
	for any probability measure $\mu$ on $E$, $\{(\xi_t)_{t\geq 0}; \widetilde{\mu \Pi}\}$ is an $E$-valued Markov process with transition semigroup $(S_t)_{t\geq 0}$;
\item
	for any probability measure $\mu$ on $E$, $\{\xi_0; \widetilde{\mu \Pi}\} \overset{\text{d}}\sim \widetilde \mu$.
\end{itemize}	
	Also, using \eqref{asp:H1}, it can be verified that
\begin{itemize}
	\item
	$\widetilde \nu$ is 
%	{\color{red}the invariant probability measure} corresponding to 
the invariant distribution of the
	semigroup $(S_t)_{t\geq 0}$.
\end{itemize}
	Now we claim that for any $\varphi \in L^2(\widetilde \nu)$ satisfying $S_t \varphi = \varphi$ in $L^2(E,\widetilde \nu)$ for all $t\geq 0$, it holds that $\varphi$ is a constant $\widetilde \nu$-a.e.
	In fact for $\widetilde \nu$-almost every $x\in E$, we have
\begin{align}
	&\varphi(x)
	= S_t \varphi(x)
	\overset{\eqref{eq:E.17}}= \frac{1}{e^{\lambda t}\phi(x)}P_t^\beta (\phi \varphi^+) (x) - \frac{1}{e^{\lambda t}\phi(x)}P_t^\beta (\phi \varphi^-) (x)
	\\&\overset{\eqref{asp:H2}}{=}  \nu(\phi \varphi^+) (1+ H_{t,x,\phi \varphi^+}) - \nu(\phi \varphi^-) (1+ H_{t,x,\phi \varphi^-})
	\xrightarrow[t\to \infty]{} \nu( \phi \varphi),
\end{align}
	which implies the desired claim.
	Now from \cite[Theorem 3.2.4.]{DaPratoZabczyk1996Ergodicity}, we know that $\widetilde \nu$ is an ergodic measure of $(S_t)_{t\geq 0}$.
\end{proof}

\begin{proof}[Proof of Proposition \ref{thm:E.4}]
	(1) Suppose that $\nu(l)<\infty$.
Let $\delta\in (0,\epsilon)$ be arbitrary.
	We have
\begin{align}
	\sum_{k\in \mathbb N} \mathbf 1_{\{s_k \leq 0\}} y_k e^{\epsilon s_k} \phi(\xi_{s_k})
	\overset{\eqref{asp:Q4}}=\int_{\mathbb R\times (0,\infty)} \mathbf 1_{\{s\leq 0\}} e^{\epsilon s}y\phi(\xi_{s}) M(\mathrm ds,\mathrm dy)
	= \text{I}+\text{II},
\end{align}
	where
\begin{align}
	&\text{I}
	:=\int_{\mathbb R\times \mathbb W} \mathbf 1_{\{y\leq \frac{e^{-\delta s}}{\phi(\xi_{s})},s\leq 0\}}e^{\epsilon s}y\phi(\xi_{s}) M(\mathrm ds,\mathrm dw),
	\\ & \text{II}
	:= \int_{\mathbb R\times \mathbb W} \mathbf 1_{\{y> \frac{e^{-\delta s}}{\phi(\xi_{s})},s\leq 0\}}e^{\epsilon s}y\phi(\xi_{s}) M(\mathrm ds,\mathrm dw).
\end{align}

%\emph{Step 1.} We show that $\text{II}< \infty$, $\mathrm Q$-a.s.
We first show that $\text{II}< \infty$, $\mathrm Q$-a.s.
	Note that
\begin{align}
	&\mathrm Q\Big[\int_{-\infty}^0  \mathrm ds \int_{\frac{e^{-\delta s}}{\phi(\xi_{s})}}^\infty y \pi(\xi_s, \mathrm dy) \Big]
	\overset{\text{\eqref{asp:Q2}, Proposition \ref{thm:E.15}}}=\int_{-\infty}^0 \mathrm ds \int_E \widetilde \nu(\mathrm dx)\int_{\frac{e^{-\delta s}}{\phi(x)}}^\infty y\pi(x, \mathrm dy)
	\\&\overset{\eqref{eq:E.162}}=\int_{-\infty}^0 \mathrm ds \int_E \phi(x) \nu(\mathrm dx)\int_{\frac{e^{-\delta s}}{\phi(x)}}^\infty y\pi(x, \mathrm dy)
	\\&= \int_E\nu(\mathrm dx)\int_{\frac{1}{\phi(x)}}^\infty y\phi(x) \pi(x, \mathrm dy)\int_{-\frac{\ln (y\phi(x))}{\delta}}^{0}\mathrm ds
	\overset{\eqref{eq:M.35}}=\frac{\nu(l)}{\delta}
	<\infty.
\end{align}
	Therefore, we have $\mathrm Q$-almost surely
\begin{align}
&\int_{(s,y)\in \mathbb R\times (0,\infty)} \mathbf 1_{\{y> \frac{e^{-\delta s}}{\phi(\xi_{s})},s\leq 0\}} \Big(1\wedge (e^{\epsilon s}y\phi(\xi_{s}))\Big) \mathrm ds \cdot y \pi(\xi_s, \mathrm dy)
	\\&\leq   \int_{(s,y)\in \mathbb R\times (0,\infty)}  \mathbf 1_{\{y> \frac{e^{-\delta s}}{\phi(\xi_{s})},s\leq 0\}} \mathrm ds\cdot y \pi(\xi_s, \mathrm dy)
	< \infty.
\end{align}
	Now from \eqref{asp:Q4} and \cite[Theorem 2.7(i)]{Kyprianou2014Fluctuations} we have $\mathrm Q(\text{II} < \infty | X, \xi, N) = 1$.
	
%\emph{Step 2.}
%	We show that $\text{I} < \infty$, $\mathrm Q$-a.s.
We now show that $\text{I} < \infty$, $\mathrm Q$-a.s.
	In fact,
\begin{align}
	&\mathrm Q[\text{I}]
	= \mathrm Q\big[\mathrm Q[I|X,\xi,N]\big]
	\overset{\eqref{asp:Q4}}=\mathrm Q\Big[\int_{-\infty}^0 \mathrm ds \int_0^{\frac{e^{-\delta s}}{\phi(\xi_s)}} e^{\epsilon s}y\phi(\xi_s) y \pi(\xi_s, \mathrm dy)\Big]
	\\&=\mathrm Q \Big[ \int_{-\infty}^0 \mathrm ds \int_0^{1\wedge \frac{e^{-\delta s}}{\phi(\xi_s)}} e^{\epsilon s} y^2 \phi(\xi_s)\pi(\xi_s, \mathrm dy) + \int_{-\infty}^0 \mathrm ds \int_{1\wedge \frac{e^{-\delta s}}{\phi(\xi_s)}}^{\frac{e^{-\delta s}}{\phi(\xi_s)}} e^{\epsilon s}y\phi(\xi_s) y \pi(\xi_s, \mathrm dy)\Big]
	\\&\leq \mathrm Q \Big[ \int_{-\infty}^0 \mathrm ds \int_0^1 e^{\epsilon s} y^2 \phi(\xi_s)\pi(\xi_s, \mathrm dy) + \int_{-\infty}^0 \mathrm ds \int_{1}^{\infty} e^{(\epsilon-\delta) s} y \pi(\xi_s, \mathrm dy)\Big]
	\\&\begin{multlined}
		\overset{\text{\eqref{asp:Q2}, Proposition \ref{thm:E.15}}}\leq \|\phi\|_\infty\int_{-\infty}^0  e^{\epsilon s}\mathrm ds \int_E \widetilde \nu(\mathrm dx)\int_0^1y^2 \pi(x, \mathrm dy) + {}
		\\ \int_{-\infty}^0  e^{(\epsilon-\delta )s}\mathrm ds\int_E \widetilde \nu(\mathrm dx)\int_1^{\infty}y \pi(x, \mathrm dy)
	\end{multlined}
	\\&\overset{\eqref{eq:M.01}}<\infty.
\end{align}
	
	
	(2) Suppose that $\nu(l)=\infty$.
	Take
\begin{equation} \label{eq:E.934}
	K
	:= \max \{\|\phi\|_\infty,e^{\epsilon s_0}\}.
\end{equation}
	Define
\begin{equation}\label{eq:E.9341}
	\eta_T
	:= \int_{-T}^{0} \mathrm ds \int_{\frac{Ke^{-\epsilon s}}{\phi(\xi_s)}}^\infty y\pi(\xi_s,\mathrm dy),
	\quad T \in (0,\infty].
\end{equation}
	\emph{Step 1.} We show that $\mathrm Q[\eta_\infty]=\infty.$
	Applying Fubini's theorem, we get
\begin{align}
	&\mathrm Q[\eta_\infty]
	\overset{\text{\eqref{asp:Q2}, Proposition \ref{thm:E.15}}}= \int_{-\infty}^{0} \mathrm ds \int_E \widetilde\nu(\mathrm dx) \int_{\frac{Ke^{-\epsilon s}}{\phi(x)}}^\infty y\pi(x, \mathrm dy)
	\\ &\overset{\text{\eqref{eq:E.162}, Fubini}}= \int_E\phi(x)\nu(\mathrm dx)\int_{\frac{K}{\phi(x)}}^\infty y\pi(x, \mathrm dy)\int_{-\frac{1}{\epsilon}\ln(\frac{y\phi(x)}{K})}^0\mathrm ds
	\\&=\frac{1}{\epsilon}\int_E\phi(x)\nu(\mathrm dx)\int_{\frac{K}{\phi(x)}}^\infty\Big(\ln\big(y\phi(x)\big)-\ln K\Big)y\pi(x, \mathrm dy)
	\\\label{eq:E.935}&\geq \frac{1}{\epsilon}\int_E\nu(\mathrm dx) \int_{\frac{K}{\phi(x)}}^\infty y\phi(x)\ln \big(y\phi(x)\big)\pi(x, \mathrm dy)-\frac{A}{\epsilon},
\end{align}
	where
\begin{equation} \label{eq:E.936}
	A
	:= \ln K \cdot \sup_{x\in E} \int_1^\infty y \pi(x,\mathrm dy)
	\overset{\eqref{eq:M.01}}< \infty.
\end{equation}
	Since
\[
	\int_E\nu(\mathrm dx) \int_{\frac{1}{\phi(x)}}^\infty y\phi(x)\ln \big(y\phi(x)\big)\pi(x, \mathrm dy)
	\overset{\eqref{eq:M.35}}= \nu(l)
	=\infty
\]
	and
\begin{align}
	\int_E\nu(\mathrm dx) \int_{\frac{1}{\phi(x)}}^{\frac{K}{\phi(x)}} y\phi(x)\ln \big(y\phi(x)\big)\pi(x, \mathrm dy)
	\leq K \ln K \int_E \nu(\mathrm dx) \int_{\frac{1}{\|\phi\|_\infty}}^\infty \pi(x, \mathrm dy)
	\overset{\eqref{eq:M.01}}<\infty,
\end{align}
	we get that
\begin{equation}
	\int_E\nu(\mathrm dx) \int_{\frac{K}{\phi(x)}}^\infty y\phi(x)\ln \big(y\phi(x)\big)\pi(x, \mathrm dy)
	= \infty.
\end{equation}
	Now the desired result in this step follows from \eqref{eq:E.935}, \eqref{eq:E.936} and above.
	
\emph{Step 2.}
	We show that $\mathrm Q[\eta_T]<\infty$ for $T\in (0,\infty)$.
	Applying Fubini's theorem, we get
\begin{align}
	\label{eq:E.9375}&\mathrm Q[\eta_T]
	\overset{\text{\eqref{asp:Q2},Proposition \ref{thm:E.15}}}= \int_{-T}^{0} \mathrm ds \int_E \widetilde \nu(\mathrm dx) \int_{\frac{Ke^{-\epsilon s}}{\phi(x)}}^\infty y\pi(x, \mathrm dy)
	\\& \overset{\eqref{eq:E.162},\eqref{eq:E.934}}\leq \int_{-T}^{0} \mathrm ds \int_E \phi(x)\nu(\mathrm dx) \int_1^\infty y\pi(x, \mathrm dy)
	\\& \leq T \cdot \sup_{x\in E} \int_1^\infty y\pi(x, \mathrm dy)
	\overset{\eqref{eq:M.01}}< \infty.
\end{align}
	
\emph{Step 3.}
	We show that there exists $t_0>0$ such that for any $t>t_0$, $x\in E$, and  $f\in \mathcal B_b(E,\mathbb R_+)$, it holds that $S_tf(x) \leq 2\widetilde\nu( f).$
	In fact, let $H$ be as in \eqref{asp:H2}, then there exists a $t_0>0$ such that for any $t>t_0$, $x\in E$ and $f\in \mathcal B_b(E,\mathbb R_+)$, it holds that $|H_{t,x,\phi f}|\leq 1$.
	Now for any $t>t_0$, $x\in E$ and $f\in \mathcal B_b(E,\mathbb R_+)$ we have
\begin{align}
	&S_tf(x)
	\overset{\eqref{eq:E.17}}= \frac{1}{\phi(x)e^{\lambda t}}P_t^\beta (\phi f) (x)
	\overset{\eqref{asp:H2}}=  \nu(\phi f) (1+ H_{t,x,\phi f})
	\overset{\eqref{eq:E.162}}\leq 2\widetilde \nu(f).
\end{align}

\emph{Step 4.}
	Let $t_0$ be given in Step 3.
	We show that there exists a constant $C>0$ such that for all $T> t_0$, it holds that $\mathrm Q[\eta_T^2]\leq C\mathrm Q[\eta_T]^2$.
	First note that for any $T>t_0$,
\begin{align}
	&\mathrm Q[\eta_T^2]
	\overset{\eqref{eq:E.9341}}=\mathrm Q \Big[\Big(\int_{-T}^0\mathrm dt\int_{\frac{Ke^{-\epsilon t}}{\phi(\xi_t)}}^\infty r \pi(\xi_t, \mathrm dr) \Big)\cdot \Big(\int_{-T}^0\mathrm ds\int_{\frac{Ke^{-\epsilon s}}{\phi(\xi_s)}}^\infty u \pi(\xi_s, \mathrm du)\Big) \Big]
	\\&=2\mathrm Q\Big[ \int_{-T}^0\mathrm dt\int_{\frac{K e^{-\epsilon t}}{\phi(\xi_t)}}^\infty r \pi(\xi_t, \mathrm dr)\int_t^0\mathrm ds
	\int_{\frac{Ke^{-\epsilon s}}{\phi(\xi_s)}}^\infty u \pi(\xi_s, \mathrm du) \Big]
	=\text{III}+\text{IV}
\end{align}
	where
\begin{align}
	\text{III}
	:= 2\mathrm Q\Big[\int_{-T}^0\mathrm dt\int_{\frac{Ke^{-\epsilon t}}{\phi(\xi_t)}}^\infty r \pi(\xi_t, \mathrm dr)\int_t^{(t+t_0)\wedge 0}\mathrm ds\int_{\frac{Ke^{-\epsilon s}}{\phi(\xi_s)}}^\infty u \pi(\xi_s,\mathrm du) \Big]
\end{align}
	and
\begin{align}
	\text{IV}
	:= 2\mathrm Q\Big[\int_{-T}^0\mathrm dt \int_{\frac{Ke^{-\epsilon t}}{\phi(\xi_t)}}^\infty r \pi(\xi_t, \mathrm dr) \int_{(t+t_0)\wedge 0}^0\mathrm ds\int_{\frac{Ke^{-\epsilon s}}{\phi(\xi_s)}}^\infty u \pi(\xi_s, \mathrm du) \Big].
\end{align}
	
%\emph{Substep 4.1.}
%	We show that 
 We first show that
	there exists a constant $C'>0$ such that for all $T>t_0$, it holds that
$
	\text{III} \leq C' \mathrm Q[\eta_T]^2.
$
	In fact, note $t\mapsto \mathrm Q[\eta_t]$ is non-decreasing, we have for any $T > t_0$,
\begin{align}
	&\text{III}
	\overset{\eqref{eq:E.934}}\leq 2\mathrm Q\Big[\int_{-T}^0\mathrm dt\int_{\frac{Ke^{-\epsilon t}}{\phi(\xi_t)}}^\infty r \pi(\xi_t, \mathrm dr)\int_t^{t+t_0}\mathrm ds\int_{1}^\infty u \pi(\xi_s,\mathrm du) \Big]
	\\&\leq 2 t_0 \Big( \sup_{x\in E} \int_{1}^\infty u \pi(x,\mathrm du)\Big) \mathrm Q[\eta_T]
	\leq \frac{2 t_0}{\mathrm Q[\eta_{t_0}]} \Big( \sup_{x\in E} \int_{1}^\infty u \pi(x,\mathrm du)\Big) \mathrm Q[\eta_T]^2.
\end{align}
	Note that from \eqref{eq:M.01} and Step 2,
\[
	\frac{2 t_0}{\mathrm Q[\eta_{t_0}]} \Big( \sup_{x\in E} \int_{1}^\infty u \pi(x,\mathrm du)\Big) < \infty.
\]

%\emph{Substep 4.2.}
%	We show that 
We now show that
	for any $T>t_0$, $\text{IV} \leq 4 \mathrm Q[\eta_T]^2$.
	Let us fix $T>t_0$ and verify that
\begin{align}
	&\text{IV}
	= \mathrm Q\Big[ 2\int_{-T}^{-t_0} \mathrm dt  \mathrm \int_{\frac{Ke^{-\epsilon t}}{\phi(\xi_t)}}^\infty r \pi(\xi_t, \mathrm dr) \int_{t+t_0}^0\mathrm ds\int_{\frac{Ke^{-\epsilon s}}{\phi(\xi_s)}}^\infty u \pi(\xi_s, \mathrm du) \Big]
	\\&\overset{\text{Fubini}}= 2\int_{-T}^{-t_0}\mathrm Q\Big[\int_{\frac{Ke^{-\epsilon t}}{\phi(\xi_t)}}^\infty r \pi(\xi_t, \mathrm dr) \int_{t+t_0}^0\mathrm ds\int_{\frac{Ke^{-\epsilon s}}{\phi(\xi_s)}}^\infty u \pi(\xi_s, \mathrm du) \Big] \mathrm dt
	\\ & = 2\int_{-T}^{-t_0}\mathrm Q \bigg[ \mathrm Q\Big[\int_{\frac{Ke^{-\epsilon t}}{\phi(\xi_t)}}^\infty r \pi(\xi_t, \mathrm dr) \int_{t+t_0}^0\mathrm ds\int_{\frac{Ke^{-\epsilon s}}{\phi(\xi_s)}}^\infty u \pi(\xi_s, \mathrm du) \Big| \mathscr F_t^\xi \Big] \bigg]\mathrm dt
	\\ & \overset{\text{Fubini}} = 2\int_{-T}^{-t_0}\mathrm Q \bigg[ \int_{\frac{Ke^{-\epsilon t}}{\phi(\xi_t)}}^\infty r \pi(\xi_t, \mathrm dr) \int_{t+t_0}^0  \mathrm Q\Big[\int_{\frac{Ke^{-\epsilon s}}{\phi(\xi_s)}}^\infty u \pi(\xi_s, \mathrm du) \Big| \mathscr F_t^\xi \Big] \mathrm ds \bigg]\mathrm dt
	\\ & \overset{\text{\eqref{asp:Q2},Proposition \ref{thm:E.15}}} = 2\int_{-T}^{-t_0}\mathrm Q \bigg[ \int_{\frac{Ke^{-\epsilon t}}{\phi(\xi_t)}}^\infty r \pi(\xi_t, \mathrm dr) \int_{t+t_0}^0  \widetilde \Pi_{\xi_t}\Big[\int_{\frac{Ke^{-\epsilon s}}{\phi(\xi_{s-t})}}^\infty u \pi(\xi_{s-t}, \mathrm du) \Big] \mathrm ds \bigg]\mathrm dt
	\\&= 2\int_{-T}^{-t_0}\mathrm dt\int_E\widetilde \nu(\mathrm dy)\int_{\frac{Ke^{-\epsilon t}}{\phi(y)}}^\infty r \pi(y, \mathrm dr) \int_{t+t_0}^0 \widetilde\Pi_y\Big[\int_{\frac{Ke^{-\epsilon s}}{\phi(\xi_{s-t})}}^\infty u \pi(\xi_{s-t},\mathrm du)\Big] \mathrm ds
	\\\label{eq:E.938}&= 2\int_{-T}^{-t_0}\mathrm dt\int_E\widetilde \nu(\mathrm dy)\int_{\frac{Ke^{-\epsilon t}}{\phi(y)}}^\infty r \pi(y, \mathrm dr) \int_{t+t_0}^0 S_{s-t}h_s (y) \mathrm ds
\end{align}	
	where for each $s\leq 0$ and $y\in E$,
\begin{equation} \label{eq:E.939}
	h_s(y)
	:= \int_{\frac{Ke^{-\epsilon s}}{\phi(y)}}^\infty u \pi(y,\mathrm du)
	\overset{\eqref{eq:E.934}}\leq \sup_{x\in E}\int_{1}^\infty u \pi(x,\mathrm du)
	\overset{\eqref{eq:M.01}}<\infty.
\end{equation}
	Note for any $t\in (-T,-t_0)$ and $s\in (t+t_0,0)$, we have $s-t\geq t_0$, 
	%which with Step 2, 
	which, together with Step 3, 
	implies that for any $y\in E$, $S_{s-t}h_s(y) \leq 2 \widetilde\nu(h_s)$.
	Therefore,
\begin{align}
	&\text{IV}
	\overset{\eqref{eq:E.938}}\leq 4 \int_{-T}^{-t_0}\mathrm dt\int_E\widetilde \nu(\mathrm dy)\int_{\frac{Ke^{-\epsilon t}}{\phi(y)}}^\infty r \pi(y, \mathrm dr) \int_{t+t_0}^0 \widetilde \nu(h_s) \mathrm ds
	\\ &\overset{\eqref{eq:E.939}}\leq 4 \int_{-T}^0 \widetilde \nu(h_t)\mathrm dt \cdot \int_{-T}^0 \widetilde \nu(h_s)\mathrm ds
	\overset{\eqref{eq:E.9375}}= 4\mathrm Q[\eta_T]^2.
\end{align}

\emph{Step 5.}
	We show that $\mathrm Q(\eta_{\infty}=\infty) >0$.
	Note that for any $T\geq 0$,
\begin{align}
	&\sqrt{\mathrm Q[\eta_T^2] \mathrm Q\Big(\eta_T \geq \frac{1}{2}\mathrm Q[\eta_T]\Big)}
	\overset{\text{Cauchy-Schwartz}}\geq \mathrm Q[\eta_T\mathbf 1_{\eta_T\geq \frac{1}{2}\mathrm Q[\eta_T]}]
	\\&= \mathrm Q[\eta_T] - \mathrm Q[\eta_T\mathbf 1_{\eta_T< \frac{1}{2}\mathrm Q[\eta_T]} ]
	\geq \mathrm Q[\eta_T] - \mathrm Q\Big[\frac{1}{2}\mathrm Q[\eta_T]\mathbf 1_{\eta_T< \frac{1}{2}\mathrm Q[\eta_T]}\Big]
	\geq \frac{1}{2}\mathrm Q[\eta_T].
\end{align}
%	Let $t_0$ be given in Step 2.
	Let $t_0$ be as in Step 3.
	Then, there exists a $\widetilde C>0$ such that for any $T\geq t_0$,
\begin{align}\label{eq:E.94}
	\mathrm Q\Big(\eta_\infty \geq \frac{1}{2}\mathrm Q[\eta_T]\Big)
	\overset{\text{Monotone}}\geq \mathrm Q\Big(\eta_T \geq \frac{1}{2}\mathrm Q[\eta_T]\Big)
	\geq \frac{\mathrm Q[\eta_T]^2}{4\mathrm Q[\eta_T^2]}
	\overset{\text{Step 4}}\geq \widetilde C.
\end{align}
	By the monotone convergence theorem, we have
\[
	\mathrm Q[\eta_T]
	\xrightarrow[T\to \infty]{} \mathrm Q[\eta_\infty]
	\overset{\text{Step 2}}= \infty.
\]
	Now by the monotone convergence theorem again,
\begin{align}
	\mathrm Q(\eta_\infty =\infty)
	= \lim_{T\to \infty} \mathrm Q\Big(\eta_\infty \geq \frac{1}{2}\mathrm Q[\eta_T]\Big)
	\overset{\eqref{eq:E.94}}\geq \widetilde C.
\end{align}
	
\emph{Step 6.}
	From Proposition \ref{thm:E.15} we know that $\{(\xi_t)_{t\in \mathbb R}, \mathrm Q\}$ is an ergodic process.
	We will show that $\{\eta_\infty = \infty\}$ is an invariant event for this ergodic process $\{(\xi_t)_{t\in \mathbb R}; \mathrm Q\}$ in the sense that, for any $t \in \mathbb R$, $\mathrm Q(A_{0} \Delta A_{t}) = 0$ where
\[
	A_{t}
	:= \Big\{\int_{-\infty}^{0} \mathrm ds \int_{\frac{Ke^{-\epsilon s}}{\phi(\xi_{s+t})}}^\infty y\pi(\xi_{s+t},\mathrm dy) = \infty\Big\}, \quad t\in \mathbb R.
\]
	
%\emph{Substep 6.1.}
%	We show that 
(a) We first show that
	for $r\in \mathbb R$ and $t> 0$, $A_r \subset A_{r-t}$.
	In fact,  on the event $A_r$ we have
\begin{align}
	&\int_{-\infty}^0 \mathrm ds \int_{\frac{Ke^{-\epsilon s}}{\phi(\xi_{s+r-t})}}^\infty y\pi(\xi_{s+r-t},\mathrm dy)
	\\&\geq \int_{-\infty}^0 \mathrm ds \int_{\frac{K e^{-\epsilon (s-t)}}{\phi(\xi_{s+r-t})}}^\infty y\pi(\xi_{s+r-t},\mathrm dy)
	= \int_{-\infty}^{-t} \mathrm ds \int_{\frac{K e^{-\epsilon s}}{\phi(\xi_{s+r})}}^\infty y\pi(\xi_{s+r},\mathrm dy)
	\\&= \int_{-\infty}^0 \mathrm ds \int_{\frac{K e^{-\epsilon s}}{\phi(\xi_{s+r})}}^\infty y\pi(\xi_{s+r},\mathrm dy) - \int_{-t}^0 \mathrm ds \int_{\frac{K e^{-\epsilon s}}{\phi(\xi_{s+r})}}^\infty y\pi(\xi_{s+r},\mathrm dy)
	\\&\overset{\eqref{eq:E.934}}\geq \int_{-\infty}^0 \mathrm ds \int_{\frac{K e^{-\epsilon s}}{\phi(\xi_{s+r})}}^\infty y\pi(\xi_{s+r},\mathrm dy) - |t| \cdot \sup_{x\in E} \int_1^\infty y\pi(x,\mathrm dy)
	\overset{\eqref{eq:M.01}}= \infty.
\end{align}

%\emph{Substep 6.2.}
%	We show that 
(b) We show that
	for any $r\in \mathbb R$ and $t\geq 0$, $\mathrm Q$-almost surely,
\begin{align}
\label{eq:E.95}
	\int_{-\infty}^0 \mathrm ds \int_{\frac{Ke^{-\epsilon (s+t)}}{\phi(\xi_{s+r+t})}}^\frac{Ke^{-\epsilon s}}{\phi(\xi_{s+r+t})} y\pi(\xi_{s+r+t},\mathrm dy)
	< \infty.
\end{align}
	In fact,
\begin{align}
	&\mathrm Q\Big[ \int_{-\infty}^0 \mathrm ds \int_{\frac{Ke^{-\epsilon (s+t)}}{\phi(\xi_{s+r+t})}}^\frac{Ke^{-\epsilon s}}{\phi(\xi_{s+r+t})} y\pi(\xi_{s+r+t},\mathrm dy) \Big]
	\\&\overset{\text{\eqref{asp:Q2},Proposition \ref{thm:E.15}}} = \int_{-\infty}^0 \mathrm ds \int_E \widetilde \nu(\mathrm dx) \int^{\frac{Ke^{-\epsilon s}}{\phi(x)}}_\frac{Ke^{-\epsilon (s+t)}}{\phi(x)} y\pi(x,\mathrm dy)
	\\&= \int_E \widetilde \nu(\mathrm dx) \int^\infty_\frac{Ke^{-\epsilon t}}{\phi(x)} y\pi(x,\mathrm dy) \int_{-\frac{1}{\epsilon}\ln \frac{y\phi(x) e^{\epsilon t}}{K}}^{-\frac{1}{\epsilon}\ln\frac{y\phi(x)}{K}} \mathrm ds
	\\&= t\int_E \widetilde \nu(\mathrm dx) \int^\infty_\frac{Ke^{-\epsilon t}}{\phi(x)} y\pi(x,\mathrm dy)
	\overset{\eqref{eq:E.934}}\leq t \cdot \sup_{x\in E} \int^\infty_{\color{red}{e^{-\epsilon t}}} y\pi(x,\mathrm dy)
	\overset{\eqref{eq:M.01}}< \infty.
\end{align}

%\emph{Substep 6.3.}
%	We show that 
(c) We now show that
	for $r\in \mathbb R$ and $t> 0$, $A_r \cap \Omega_{r,t} \subset A_{r+t}$ where $\Omega_{r,t}$ is the event so that \eqref{eq:E.95} holds.
	In fact,  on event $A_r\cap \Omega_{r,t}$ we have
\begin{align}
	&\int_{-\infty}^0 \mathrm ds \int_{\frac{K e^{-\epsilon s}}{\phi(\xi_{s+r+t})}}^\infty y\pi(\xi_{s+r+t},\mathrm dy)
	\\&= \int_{-\infty}^0 \mathrm ds \int_{\frac{K e^{-\epsilon (s+t)}}{\phi(\xi_{s+r+t})}}^\infty y\pi(\xi_{s+r+t},\mathrm dy) - \int_{-\infty}^0 \mathrm ds \int_{\frac{Ke^{-\epsilon (s+t)}}{\phi(\xi_{s+r+t})}}^\frac{Ke^{-\epsilon s}}{\phi(\xi_{s+r+t})} y\pi(\xi_{s+r+t},\mathrm dy)
	\\&= \int_{-\infty}^t \mathrm ds \int_{\frac{Ke^{-\epsilon s}}{\phi(\xi_{s+r})}}^\infty y\pi(\xi_{s+r},\mathrm dy) -  \int_{-\infty}^0 \mathrm ds \int_{\frac{Ke^{-\epsilon (s+t)}}{\phi(\xi_{s+r+t})}}^\frac{Ke^{-\epsilon s}}{\phi(\xi_{s+r+t})} y\pi(\xi_{s+r+t},\mathrm dy)
	\\&\geq \int_{-\infty}^0 \mathrm ds \int_{\frac{Ke^{-\epsilon s}}{\phi(\xi_{s+r})}}^\infty y\pi(\xi_{s+r},\mathrm dy) -  \int_{-\infty}^0 \mathrm ds \int_{\frac{Ke^{-\epsilon (s+t)}}{\phi(\xi_{s+r+t})}}^\frac{Ke^{-\epsilon s}}{\phi(\xi_{s+r+t})} y\pi(\xi_{s+r+t},\mathrm dy)
	= \infty.
\end{align}

%\emph{Substep 6.4.}
(d) 
	For any $r<t$ in $\mathbb R$, 
	%from Substep 6.1, 
	from (a) above,
	we know that $\mathrm Q(A_t\setminus A_r) = 0$.
	%From Substeps 6.2 and 6.3, 
	From (b) and (c) above 
	we know that $\mathrm Q(A_r\setminus A_t) = 0$.
	Therefore, the desired result in Step 6 follows.

\emph{Final Step.}
%	From Step 5, Step 6 
From steps 5, 6
	and \cite[Theorem 1.2.4.(i)]{DaPratoZabczyk1996Ergodicity}, we get that
\begin{equation} \label{eq:E.96}
	\int_{-\infty}^{0} \mathrm ds \int_{\frac{K e^{-\epsilon s}}{\phi(\xi_{s})}}^\infty y \pi(\xi_{s},\mathrm dy) = \infty, \quad \mathrm Q\text{-a.s.}
\end{equation}
	From Proposition \ref{thm:E.15} and \eqref{asp:Q2}, we know that $\{(\xi_s)_{s\in \mathbb R}; \mathrm Q\}$ has the same distribution as $\{(\xi_{s-s_0})_{s\in \mathbb R}; \mathrm Q\}$.
	Therefore we have from \eqref{eq:E.96} that
\begin{equation}\label{eq:E.97}
	\int_{-\infty}^{0} \mathrm ds \int_{\frac{K e^{-\epsilon s}}{\phi(\xi_{s-s_0})}}^\infty y \pi(\xi_{s-s_0},\mathrm dy) = \infty, \quad \mathrm Q\text{-a.s.}
\end{equation}
	Therefore we have
\begin{align}
	&\int_{-\infty}^{-s_0} \mathrm ds \int_{\frac{e^{-\epsilon s}}{\phi(\xi_{s})}}^\infty y \pi(\xi_{s},\mathrm dy)
	= \int_{-\infty}^{0} \mathrm ds \int_{\frac{e^{-\epsilon s} e^{\epsilon s_0}}{\phi(\xi_{s-s_0})}}^\infty y \pi(\xi_{s-s_0},\mathrm dy)
	\\&\overset{\eqref{eq:E.934}}\geq \int_{-\infty}^{0} \mathrm ds \int_{\frac{K e^{-\epsilon s}}{\phi(\xi_{s-s_0})}}^\infty y \pi(\xi_{s-s_0},\mathrm dy)
	\overset{\eqref{eq:E.97}}= \infty.
	\qedhere
\end{align}
\end{proof}

\begin{proof}[Proof of Proposition \ref{thm:E.5}]
	From \cite[(3.20)]{LiuRenSongSun2020} we know that there exists $t_0,a,\epsilon>0$ such that for all $s\geq t_0$, we have $\nu(V_{s}\phi) \geq a \exp(-\epsilon s)$.
	According to \cite[Proposition 2.2]{LiuRenSongSun2020} we know that there exist $s_0'>0$, such that for all $s\geq s_0'$ and $x\in E$ we have $V_{s}\phi(x)\ge\frac{1}{2}\phi(x)\nu(V_{s}\phi)$.
	Now take $s_0:= t_0 \vee s_0'$, we have for all $s\geq s_0$ and $x\in E$, $V_{s}\phi(x)\geq {\color{red}\dfrac{a}{2}}\phi(x)e^{-\epsilon s}$.
	Let $\theta \in (0,a/2)$.
	We have for any $s>s_0$, $x\in E$ and $y\geq \frac{e^{\epsilon s}}{\phi(x)}$ that
\begin{align}
	&\mathrm P_{y\delta_x}\big(w_s(\phi)>\theta\big)
	=\mathrm P_{y\delta_x}\left(e^{-w_s(\phi)}<e^{-\theta}\right)
	\\&=1-\mathrm P_{y\delta_x}(e^{-w_s(\phi)}\geq e^{-\theta})
	\overset{\text{Chebyshev}}\geq 1-e^{\theta}\mathrm P_{y\delta_x}[e^{- w_s(\phi)}]
	\\&=1-e^{\theta}e^{-yV_{s}\phi(x)}
	\geq 1-e^{\theta}e^{-y \frac{a}{2}\phi(x)e^{-\epsilon s}}
	\geq 1 - e^{\theta - a/2}=: \delta >0.
	\qedhere
\end{align}
\end{proof}



\section{Proofs of Theorems \ref{thm:L} and \ref{thm:I}}



\subsection{Proof of Theorem \ref{thm:L}}

\begin{lem}\label{lem:ratio limit}
	(1) For any $s>0$ and $f\in\mathcal B(E,[0,\infty])$ with $\nu(f)>0$, 
	\begin{equation}\label{integ ratio limit}
		\lim_{t\to\infty}\dfrac{\langle V_{t+s}f, \nu\rangle}{\langle V_{t}f, \nu\rangle}=e^{\lambda s}.
	\end{equation}
	(2)
		For any $f\in\mathcal B(E,[0,\infty])$ with $\nu(f)>0$, there exist positive constants $a,N,T>0$ such that
	\begin{equation}\label{inequ:lower}
		\langle V_{t}f, \nu\rangle\geq ae^{-Nt},\quad \mbox{for any }\ t>T.
	\end{equation}
	(3) For each $s\geq 0$,
	\begin{equation} \label{one point ratio limit}
		\lim_{t\to \infty} \sup_{x\in E}\Big|\frac{v(t+s,x)}{\langle v(t,\cdot),\nu\rangle\phi(x) } - e^{\lambda s} \Big|=0.
	\end{equation}
\end{lem}



\begin{proof}
%	(1) is basically \cite[(2.20)]{LiuRenSongSun2020}. 
(1) is basically \cite[(3.20)]{LiuRenSongSun2020}. 
	(3) follows immediately from
%	\cite[(2.20) and Proposition 1.4]{LiuRenSongSun2020}. 
	\cite[(3.20) and Proposition 1.4]{LiuRenSongSun2020}. 
	Now we prove (2).
%	It follows from\cite[(2.4)]{LiuRenSongSun2020} 
	It follows from\cite[(3.4)]{LiuRenSongSun2020} 
	that, for any $f\in\mathcal B(E,[0,\infty])$ with $\nu(f)>0$, $\langle V_{t}f, \nu\rangle>0$ for all $t>0$. 
	%Now by \cite[(2.20)]{LiuRenSongSun2020},
	Now by \cite[(3.20)]{LiuRenSongSun2020},
	$$
	\langle V_{s+t}f, \nu\rangle=\langle V_{s}f, \nu\rangle\exp\{\lambda t(1+C_{s, t, f})\}
	$$
	for some $C_{s, t, f}$ with $\lim_{s\to\infty}\sup_{t\ge 0, f\in \mathcal B(E,[0,\infty])}|C_{s, t, f}|=0$. Take $s_0$ large enough so that $\sup_{t\ge 0, f\in \mathcal B(E,[0,\infty])}|C_{s_0, t, f}|\le \frac12$. The desired result follows immediately.
\end{proof}

%The following lemma is \cite[Proposition 1.4]{LiuRenSongSun2020}.
The following lemma is \cite[Proposition 2.2]{LiuRenSongSun2020}.
\begin{lem}\label{lem:extinc}
	Suppose that  \eqref{asp:H1}, \eqref{asp:H2} and \eqref{asp:H3} hold.
	\begin{enumerate}
		\item	
		For any  $ t>0$ and $ \mu \in \mathcal M_f(E)$, $\langle v(t,\cdot),\mu\rangle <\infty.$
		\item	For any $\mu \in \mathcal M_f(E)$,
		\[
		\lim_{t\rightarrow\infty}\langle v(t,\cdot),\mu\rangle=0.
		\]
	\end{enumerate}
\end{lem}

%We first state a result 
We now state another result
proved in \cite{LiuRenSongSun2020}. For $f\in \mathcal B(E, [0,\infty])$, Put
$$
\Gamma_t f:=-\log \mathrm P_{\nu}[e^{-X_t(f)}|X_t(1)>0].
$$
We say a $[0,\infty]$-valued functional $A$ defined on $\mathcal B(E,[0,\infty])$ is monotone concave if
	(1) $A$ is a monotone functional, i.e., $f\leq g$ in $\mathcal B(E,[0,\infty])$ implies $Af \leq Ag$; and
	(2) for any $f\in \mathcal B(E,[0,\infty])$ with $Af< \infty$, the function $u \mapsto A(uf)$ is concave on $[0,1]$.
%The following result is \cite[Proposition 1.5, Proposition 1.9]{LiuRenSongSun2020}.
The following result is \cite[Propositions 2.3 and 2.7]{LiuRenSongSun2020}.

\begin{lem} \label{prop:G}
(1) The limit $Gf:= \lim_{t\to \infty} \Gamma_t f$ exists in $[0,\infty]$ for each $f\in \mathcal B(E,[0,\infty])$.
	Moreover, $G$ is the unique $[0,\infty]$-valued monotone concave functional on $\mathcal B(E,[0,\infty])$ such that
	$G(\infty  \mathbf 1_E) = \infty$ and that
\begin{equation} \label{eq:G.0}
	1 - e^{- GV_s f}
	= e^{s\lambda} (1 - e^{-Gf}),
	\quad s\geq 0, f\in \mathcal B(E,[0,\infty]).
\end{equation}

(2) $G$ is the log-Laplace functional of $\mathbf Q_\lambda$, and for any $r\in[\lambda, 0)$,
%$L_\alpha(f):=1-\left(1-e^{-G(f)}\right)^\alpha, f\in\mathcal B(E,[0,\infty)),$
\[
L_\alpha(f):=1-\left(1-e^{-G(f)}\right)^\alpha,\quad  f\in\mathcal B(E,[0,\infty))
\]
is the Laplace functional of $\mathbf Q_r$, where $\alpha=r/\lambda\in(0,1]$.
\end{lem}


\begin{proof}[Proof of Theorem \ref{thm:L}]
(1) By \eqref{one point ratio limit} with $s=0$, we know that, for any $\varepsilon>0$, there is some $T_1>0$ such that when $t>T_1$,
\[
(1-\varepsilon)\langle v(t),\nu\rangle \phi(x)\leq v(t,x)\leq (1+\varepsilon)\langle v(t),\nu\rangle \phi(x),\qquad x\in E.
\]
By Lemma \ref{prop:G},  $1-e^{-G(f)}$ is non-decreasing  with respect to $f\in \mathcal B(E,[0,\infty])$, and $1-e^{-G(uf)}$ is a concave function of $u\in(0,1)$.
Fix an arbitrary $u\in (0, 1)$.
On one hand, we have for $t>T_1$,
\begin{eqnarray}\label{lower}
\dfrac{1-e^{-G(u\langle v(t),\nu\rangle \phi)}}{1-e^{-G(\langle v(t),\nu\rangle \phi)}}&\overset{\text{monotonicity}}\geq& \dfrac{\left(1-e^{-G(\frac{u}{1+\varepsilon}v(t))}\right)}{1-e^{-G(\frac{1}{1-\varepsilon} v(t))}}
\overset{\text{concavity}}\geq \dfrac{\dfrac{u}{1+\varepsilon}\left(1-e^{-G(v(t))}\right)}{1-e^{-G(\frac{1}{1-\varepsilon} v(t))}}
\\
&=&\dfrac{u}{1+\varepsilon}\left[e^{-\lambda t}\left(1-e^{-G(\frac{1}{1-\varepsilon} v(t))}\right)\right]^{-1}\nonumber\\
&\overset{\text{concavity}}\geq& \dfrac{u(1-\varepsilon)}{1+\varepsilon}\left[e^{-\lambda t}\left(1-e^{-G(v(t))}\right)\right]^{-1}=\dfrac{u(1-\varepsilon)}{1+\varepsilon},
\end{eqnarray}
where in the last inequality, we used \eqref{eq:G.0} with $f=\infty I_{E}.$
On the other hand,
for $\varepsilon>0$ small enough so that $u/(1-\varepsilon)<1$, we have
\begin{eqnarray}\label{upper}
\dfrac{1-e^{-G(u\langle v(t),\nu\rangle \phi)}}{1-e^{-G(\langle v(t),\nu\rangle \phi)}}&\overset{\text{monotonicity}}\leq &\dfrac{\left(1-e^{-G(\frac{u}{1-\epsilon}v(t))}\right)}{1-e^{-G(\frac{1}{1+\varepsilon} v(t))}}\overset{\text{concavity}}\leq \dfrac{\dfrac{u}{1-\varepsilon}\left(1-e^{-G(v(t))}\right)}{1-e^{-G(\frac{1}{1+\varepsilon} v(t))}}\\
&=&\dfrac{u}{1-\varepsilon}\left[e^{-\lambda t}\left(1-e^{-G(\frac{1}{1+\varepsilon} v(t))}\right)\right]^{-1}\\
 &\overset{\text{concavity}}\leq& \dfrac{u(1+\varepsilon)}{1-\varepsilon}\left[e^{-\lambda t}\left(1-e^{-G(v(t))}\right)\right]^{-1}
=\dfrac{u(1+\varepsilon)}{1-\varepsilon}.
\end{eqnarray}
Combining the two displays above, we get that for any $u\in (0, 1)$,
\[
\lim_{t\to\infty}\dfrac{1-e^{-G(u\langle v(t),\nu\rangle \phi)}}{1-e^{-G(\langle v(t),\nu\rangle \phi)}}
=u.
\]
Therefore, by Lemma \ref{lem:regu} in the Appendix, we have
\begin{equation}\label{eq regu}
1-e^{-G(u\phi)}\sim uL(u),\quad u\rightarrow 0+,
\end{equation}
where $L$ is slowly varying at $0$.

Note that for any $u>0$, we have
\begin{align}
\int_0^\infty e^{-us}{\mathbf Q}_\lambda(\{\mu: \mu(\phi)>s\})ds=&-\frac{1}{u}\int^\infty_0{\mathbf Q}_\lambda(\{\mu: \mu(\phi)>s\})d(e^{-us})\\
=&\frac{1}{u}{\mathbf Q}_\lambda(\{\mu: \mu(\phi)>0\})-\frac{1}{u}\int^\infty_0e^{-us}d({\mathbf Q}_\lambda(\{\mu: \mu(\phi)\le s\}))\\
=&\dfrac{1-e^{-G(u\phi)}}{u}.
\end{align}
It follows from  \eqref{eq regu} and Lemma \ref{lem: tau} that there is a function $L_1$, which is a slowly varying at infinity, such that
$$\int^x_0{\mathbf Q}_\lambda(\mu(\phi)>s)ds\sim L_1(x),\quad x\to\infty.$$
It then follows from Lemma \ref{lem:tail} that
$$s{\mathbf Q}_\lambda(\{\mu: \mu(\phi)>s\})=o(L_1(s))\qquad s\to\infty.$$
Therefore, for $r\in[\lambda, 0)$,
\begin{equation}
{\mathbf Q}_r(\{\mu: \mu(\phi)>s\})=o(s^{-\alpha}L_1(s)),\qquad s\to\infty,
\end{equation}
where $\alpha=\lambda/r.$
Thus for any $0<\gamma<\alpha$,
\[
{\mathbf Q}_r\left(\mu(\phi)^{\gamma}\right)=\int_{{\mathcal M}_f(E)}\mu(\phi)^\gamma\mathbf Q_r(d\mu)<\infty.
\]

(2)
For any $f\in\mathcal B_b(E, [0, \infty))$ and $t>0$,  using the definition of $\widetilde{\mathrm P}_\nu$, we have
\begin{eqnarray*}
&&\mathrm P_\nu\left(\exp\{-\langle f, X_t\rangle \}; X_t\neq \mathbf 0\right)
=\mathrm P_\nu\left(\dfrac{M_t(\phi)}{M_t(\phi)}\exp\{-\langle f, X_t\rangle \};X_t\neq \mathbf 0\right)\\
&&=\widetilde{\mathrm P}_\nu\left(\dfrac{1}{M_t(\phi)}\exp\{-\langle f, X_t\rangle \}\right)
=e^{\lambda t}\mathrm Q\left(\dfrac{\exp\Big\{-\langle f, X_t\rangle -\langle f,  Z_0^{(-t, 0]}\rangle\Big \}}{\langle\phi, X_t\rangle +\langle\phi,  Z_0^{(-t, 0]}\rangle }\right).
\end{eqnarray*}
Since ${\mathrm Q}( \exists t>0 \mbox{ such that }X_t=0)=1$,
using  \eqref{eq:E.53} and  \eqref{eq:E.54}, we have
%\[
\begin{equation}\label{e:new}
\lim_{t\to\infty}e^{-\lambda t}\mathrm P_\nu\left(\exp\{-\langle f, X_t\rangle \};X_t\neq \mathbf 0\right)={\mathrm Q}\left(\dfrac{\exp\left\{-\langle f,Z_0^{(-\infty, 0]}\rangle \right\}}{\langle \phi, Z_0^{(-\infty, 0]}\rangle}\right).
%\]
\end{equation}
%Note that for the continuum immigration part,
Note that
\[
\mathrm Q\left(\int_{(-\infty, 0]\times \mathbb W}w_{-s}(f)N(\mathrm ds, \mathrm dw)\right)=
\int^0_{-\infty}2\langle \sigma^2 P^{\beta}_{-s}f,\phi\nu\rangle \mathrm ds
\leq 2\|\sigma^2\phi\|_\infty\dfrac{\langle f,\nu\rangle }{-\lambda}<\infty.
\]
%Now we deal with the discrete immigration part.
Now we deal with $\sum_{k\in\mathbb N}w^{(k)}_{-s_k}(f)\mathbf 1_{s_k\in (-\infty, 0]}$.

%(i) 
If  $\int_E l(x) \nu(dx)<\infty$,
 by Proposition \ref{thm:E.4} (1) and \eqref{asp:H2} ,
\begin{eqnarray*}
&&\mathrm Q\left( \sum_{k\in\mathbb N}w^{(k)}_{-s_k}(f)\mathbf 1_{s_k\in (-\infty, 0]}\Big|\mathcal G\right)=\sum_{k\in\mathbb N}\mathbf 1_{s_k\in (-\infty, 0]}y_kP^{\beta}_{-s_k}f( \xi_{s_k})\\
&&\leq\sum_{k\in\mathbb N}\mathbf 1_{s_k\in (-\infty, 0]}(1+C^{(H2)}_{-s_k, \xi_{s_k}, f})
e^{-\lambda s_k}\phi(\xi_{s_k})\nu(f)\\
&&=\nu(f)\sum_{k\in\mathbb N}\mathbf 1_{s_k\in (-\infty, 0]}(1+H_{-s_k, \xi_{s_k}, f})
e^{-\lambda s_k}\phi(\xi_{s_k})<\infty, \quad \mathrm Q-\text{a.s.}
\end{eqnarray*}
 Thus in this case, the limit measure
$Z_0^{(-\infty, 0]}\in \mathcal M$.
 Denote the distribution of $Z_0^{(-\infty, 0]}$ under
  ${\mathrm Q}$ by $\mathbf Q$,
  which is also the limit distribution of $X_t$ under $\widetilde{\mathrm P}_\nu$ as $t\to\infty$.  
 % Then when $\int_E l(x)\nu(dx)<\infty$,
 Thus by \eqref{e:new},
\[
\lim_{t\rightarrow\infty}e^{-\lambda t}\mathrm P_\nu\left(\exp\{-\langle f, X_t\rangle \};
X_t\neq \mathbf 0\right)=
%\int_{{\mathcal M}_f(E)}\frac{1}{\mu(\phi)}e^{-\mu(f)}\mathbf Q(d\mu).
\int_{{\mathcal M}}\frac{1}{\mu(\phi)}e^{-\mu(f)}\mathbf Q(d\mu).
\]
In Theorem \ref{thm:E} we have shown that
\[
\lim_{t\rightarrow\infty}e^{-\lambda t}\mathrm P_\nu(X_t\neq \mathbf 0)=k
%=\int_{{\mathcal M}_f(E)}\frac{1}{\mu(\phi)}\mathbf Q(d\mu)<\infty.
=\int_{{\mathcal M}}\frac{1}{\mu(\phi)}\mathbf Q(d\mu)<\infty.
\]
Thus by the definition of the Yaglom distribution ${\mathbf Q}_\lambda $,
\begin{align*}
&\int_{\mathcal M}\exp\{-\langle f, \mu\rangle \}\mathbf Q_\lambda(\mathrm d\mu)
=\lim_{t\rightarrow\infty}\mathrm P_\nu\left(\exp\{-\langle f, X_t\rangle \}\Big|X_t\neq \mathbf 0\right)\\
&=\lim_{t\rightarrow\infty}\dfrac{\mathrm P_\nu\left(\exp\{-\langle f, X_t\rangle \};X_t\neq \mathbf 0\right)}{\mathrm P_\nu(X_t\neq \mathbf 0)}\\
&=\dfrac{\lim_{t\rightarrow\infty}e^{-\lambda t}\mathrm P_\nu\left(\exp\{-\langle f, X_t\rangle \};X_t\neq \mathbf 0\right)}{\lim_{t\rightarrow\infty}e^{-\lambda t}\mathrm P_\nu(X_t\neq \mathbf 0)}\\
&=\dfrac{\int_{{\mathcal M}_f(E)}\mu(\phi)^{-1}e^{-\mu(f)}\mathbf Q(d\mu)}{\int_{{\mathcal M}_f(E)}\mu(\phi)^{-1}\mathbf Q(d\mu)},
\end{align*}
which says that the Yaglom distribution ${\mathbf Q}_\lambda$ can be written as
\begin{equation}\label{rep: yaglom}
\mathbf Q_\lambda(\cdot)=\dfrac{1}{k}{\mathbf Q}\left(\dfrac{1}{\mu(\phi)}; \mu\in\cdot\right).
%\int_{\mathcal M}\mu(f)\mathbf Q_\lambda(\mathrm d\mu)=\dfrac{1}{k}\int_{\mathcal M}\frac{\mu(f)}{\mu(\phi)}\mathbf Q(\mathrm d\mu).
% YX: I changed back to the previous version, because they are not equavalent.
\end{equation}
Consequently
\begin{equation}\label{ident: k}
\mathbf Q_\lambda(\mu(\phi))=\dfrac{1}{k}{\mathbf Q}\left(\dfrac{\mu(\phi)}{\mu(\phi) }\right)=\dfrac{1}{k}<\infty.
%YX \int_{\mathcal M}\mu(\phi)\mathbf Q_\lambda(\mathrm d\mu)=1.
\end{equation}


(ii) If $\int_El(x)\nu(dx)=\infty$,
by Theorem \ref{thm:E},
$\lim_{t\to\infty}e^{-\lambda t}\mathrm P_\nu(X_t\neq \mathbf 0)=k=0$,
which implies that
 $\lim_{t\to\infty}e^{-\lambda t}\langle v(t),\nu\rangle=0$.  Since for any $s>0$,  $1-e^{-s}\leq s$, we have
$$G(\langle v(t),\nu\rangle\phi)\geq 1-e^{-G(\langle v(t),\nu\rangle\phi)}.$$
Also note that $\langle v(t),\nu\rangle
\int_{\mathcal M}\mu(\phi)\mathbf Q_\lambda(\mathrm d\mu)=G(\langle v(t),\nu\rangle\phi).$
Thus  for $t>T_0$,
\[
\int_{\mathcal M}\mu(\phi)\mathbf Q_\lambda(\mathrm d\mu)
\geq \dfrac{1-e^{-G(\langle v(t),\nu\rangle\phi)}}{\langle v(t),\nu\rangle}.
\]
 By \eqref{one point ratio limit} with $s=1$,
 there is some $T_0>0$ such that for $t>T_0$, $v(t+1,x)\leq 2\langle v(t),\nu\rangle\phi(x)$, $x\in E$.
 Therefore, as $t\to\infty$,
 \[
\int_{\mathcal M}\mu(\phi)\mathbf Q_\lambda(\mathrm d\mu)
\geq \dfrac{1-e^{-G(\frac{1}{2}v(t+1,x))}}{\langle v(t),\nu\rangle}\overset{\text{monotonicity}}\geq
\dfrac{1-e^{-G(v(t+1,x))}}{2\langle v(t),\nu\rangle}
=\dfrac{e^{\lambda(t+1)}}{2\langle v(t),\nu\rangle}\to\infty,
\]
which implies that
$\int_{\mathcal M}\mu(\phi)\mathbf Q_\lambda(\mathrm d\mu)=\infty$.

(3)
Recall that $U^{(r)}$ stands for an $\mathcal M$-valued random element with distribution $\mathbf Q_r$. Let $Y_r=\langle\phi, U^{(r)}\rangle$, then $Y_r\geq 0$.
Define $g(u)=\mathbf Q_\lambda\left(e^{-uY_\lambda}\right),\, u\geq 0$.
It follows from Lemma \ref{prop:G} (2) that
$$
\mathbf Q_r\left(e^{-uY_r}\right)=1-[1-g(u)]^\alpha,\, u\geq 0.
$$
Applying the identity
\[
x^\alpha=c\int_0^\infty\dfrac{1-e^{-xt}}{t^{\alpha+1}}dt,
\quad x>0,
\]
where $c=\left(\int_0^\infty\dfrac{1-e^{-t}}{t^{\alpha+1}}dt\right)^{-1}$, and $\mathbf Q_r(\{{\bf 0}\})=0$, we obtain
\begin{align*}
&\mathbf Q_r(\langle\phi, U^{(r)}\rangle^{\alpha})=\int_0^\infty x^\alpha \mathbf Q_r(Y_r\in dx)\\
=&c\int_0^\infty \left(\int_0^\infty\dfrac{1-e^{-xt}}{t^{\alpha+1}}dt \right)\mathbf Q_r(Y_r\in dx)\\
=&c\int_0^\infty\dfrac{1}{t^{\alpha+1}}dt\int_0^\infty\left(1-e^{-xt}\right)\mathbf Q_r(Y_r\in dx)\\
=&c\int_0^\infty\dfrac{1-\mathbf Q_r\left(e^{-tY_r}\right)}{t^{\alpha+1}}dt
=c\int_0^\infty\dfrac{(1-g(t))^\alpha}{t^{\alpha+1}}dt.
\end{align*}
Since $\dfrac{(1-g(t))^\alpha}{t^{\alpha+1}}\leq t^{-(\alpha+1)}$, $\mathbf Q_r(\langle\phi, U^{(r)}\rangle^{\alpha})<\infty$ if and only if $\int_0^1\dfrac{(1-g(t))^\alpha}{t^{\alpha+1}}dt<\infty$.
For any $M>0$, $1-g(t)\geq 1-\mathbf Q_\lambda\left(e^{-t(Y_\lambda\wedge M)}\right)\sim t \mathbf Q_\lambda(Y_\lambda\wedge M)$ as $t\to 0$. Therefore,
$\dfrac{(1-g(t))^\alpha}{t^{\alpha+1}}\gtrsim t^{-1}$ when $t$ is sufficiently small.  Since $\int_0^1 t^{-1}dt=\infty$,
$\int_0^1\dfrac{(1-g(t))^\alpha}{t^{\alpha+1}}dt=\infty$. The proof is complete.
\end{proof}

\subsection{Proof of Theorem \ref{thm:I}}
Before we give the proof  of Theorem \ref{thm:I}, we prove a lemma first.

\begin{lem}\label{l:new}
There exist $T_0$ and $\eta>0$ such that for all $T>T_0$ and $x\in E$,
$$
\mathrm P_x (X_T\neq\mathbf 0)\le \eta\phi(x)e^{-\lambda T}.
$$
\end{lem}
\begin{proof}
Note that
\begin{equation}\label{e:1}
P_x (X_T\neq\mathbf 0)=1-e^{-v_t(x)}\le v_t(x).
\end{equation}
By Lemma \ref{lem:ratio limit} (3), we have
$$
\lim_{t\to\infty}\sup_{x\in E}\Big|\frac{v_t(x)}{\nu(v_t)\phi(x)}-1\Big|=0.
$$
Thus there exists $t_0$ such that for all $t>t_0$ and $x\in E$,
\begin{equation}\label{e:2}
\frac{v_t(x)}{\nu(v_t)\phi(x)}\le 2.
\end{equation}
Combining \eqref{e:1} and \eqref{e:2} we get that  for all $t>t_0$ and $x\in E$,
\begin{equation}\label{e:3}
\mathrm P_x (X_t\neq\mathbf 0)\le 2 \nu(v_t)\phi(x).
\end{equation}
It follows from Theorem \ref{thm:E} there exist $t_1>0$ and $a>0$ such that
$1-e^{-\nu(v_t)}\le a e^{\lambda t}$ for all $t\ge t_1$. By Lemma \ref{lem:extinc},
we have
$$
\lim_{t\to\infty}\frac{1-e^{-\nu(v_t)}}{\nu(v_t)}=1.
$$
Thus exists $t_2\ge t_1$ such that for $t\ge t_2$,
$$
\frac{1-e^{-\nu(v_t)}}{\nu(v_t)}\ge 2.
$$
Hence for $t\ge t_2$ we have $\nu(v_t)\le\frac{a}2e^{\lambda t}$.
Combining this with \eqref{e:3} we immediately get the desired result.
\end{proof}

{\color{red} YX: The following spine decomposition is used in this subsection. But the only difference is the spine $\xi$. So, we can just replace (Q2) to the following\\
(Q2'): The \emph{spine process} $\{(\xi_t)_{t\geq 0}; \mathrm Q_\mu\}$ is process with
 \[\{(\xi_t)_{t \geq 0}; \mathrm Q_\mu\} \overset{\text{d}} = \{(\xi_t)_{t \geq 0}; \widetilde{\mu\Pi}\};\]
 }

In spirit of \cite{RenSongSun2020Spine}, for any $\mu\in \mathcal M^0$, let
$\{(\xi_t)_{t\geq 0}, \mathbf N^\xi; \mathrm Q_\mu\}$ be a spine decomposition of $\{(w_t)_{t\geq 0}; \widetilde {\mathrm N}_\mu\}$, i.e.
(i) $\{(\xi_t)_{t\geq 0}; \mathrm Q_\mu\} \overset{d}= \{(\xi_t)_{t\geq 0}; \widetilde \Pi_{\mu} \}$, and $\{(\xi_t)_{t\geq 0}; \mathrm Q_\mu\}$ is called the {\it spine process};
(ii) given the path of the spine
$\{(\xi_t)_{t\geq 0}; \mathrm Q_\mu\}$,
$\{\mathbf N^\xi; \mathrm Q_\mu\}$ is a Poisson random measure on $[0,\infty) \times \mathbb W$ with intensity
$$
2 \sigma(\xi_s)^2 \mathrm ds \cdot \mathrm N_{\xi_s}(\mathrm dw)+ \mathrm ds \cdot \int_{(0,\infty)} y \mathrm P_{y\delta_{\xi_s}}(X\in \mathrm dw) \pi(\xi_s, \mathrm dy).
$$
Then according to \cite{RenSongSun2020Spine}, we have that
\begin{align}
	& \{(w_t)_{t\geq 0}; \widetilde{\mathrm N}_\mu \} \overset{{f.d.d.}} = \Big\{ \Big( \int_{[0,t)\times \mathbb W} w_{t-s} \mathbf N^\xi(\mathrm ds, \mathrm dw) \Big)_{t\geq 0} ; \mathrm Q_\mu\Big\}.
\end{align}
Define\begin{align}
	& Z_t:= \int_{[0,t)\times \mathbb W} w_{t-s} \mathbf N^\xi(\mathrm ds, \mathrm dw), \quad t\geq 0.
\end{align}
According to \cite{RenSongSun2020Spine} and \cite{RenSongYang2016Spine},  we have
\begin{equation}\label{spine-decom1}
	\{(X_t)_{t\geq 0}; \widetilde {\mathrm P}_\mu\} \overset{f.d.d.} = \{ (X_t+ Z_t)_{t\geq 0}; \mathrm Q_\mu\},
\end{equation}
where  $\{(X_t)_{t\geq 0}; \mathrm Q_\mu\} \overset{d} =  \{(X_t)_{t\geq 0}; \mathrm P_\mu\}$,   and $\{(X_t)_{t\geq 0}; \mathrm Q_\mu\} $ and $\{(Z_t)_{t\geq 0}; \mathrm Q_\mu\}$ are independent.

More precisely,
we call $\{\mathbf N^\xi; \mathrm Q_\mu\}$ the immigration along the spine $(\xi_t)_{t\geq 0}$. We may decompose $\{\mathbf N^\xi; \mathrm Q_\mu\}$ as the sum of two parts:
$$
\mathbf N^\xi(\mathrm ds, \mathrm dw)=\mathrm n(\mathrm ds, \mathrm dw)+\mathrm m(\mathrm ds, \mathrm dw).
$$

We say $\{(\xi_t)_{t\geq 0}, (X^{\mathrm n, \sigma})_{\sigma\in \mathcal D^\mathrm n}, (X^{\mathrm m, \sigma})_{\sigma \in \mathcal D^\mathrm m}, (X_t)_{t\geq 0}; \mathrm Q_{\mu}\}$ is a \emph{spine representation} of $\{(X_t)_{t\geq 0}; \widetilde {\mathrm P}_\mu\}$ if the followings are true:
\begin{itemize}
	\item
	The \emph{spine process} $\{(\xi_t)_{t\geq 0}; \mathrm Q_\mu\}$ is a copy of $\{(Y_t)_{t\geq 0}; \widetilde \Pi_{\phi\cdot\mu}\}$.
	\item
	Given $\{(\xi_t)_{t\geq 0}; \mathrm Q_\mu\}$, \emph{the continuum immigration} $\{ (X^{\mathrm n,\sigma})_{\sigma \in \mathcal D^\mathrm n}; \mathrm Q_\mu(\cdot |\xi)\}$ is a $\mathbb W$-valued point process such that
	\[
	\mathrm n(\mathrm ds, \mathrm dw) := \sum_{\sigma\in \mathcal D^{\mathrm n}} \delta_{(\sigma, X^{\mathrm n,\sigma})}(\mathrm ds, \mathrm dw)
	\]
	is a Poisson random measure on $[0,T]\times \mathbb W$ with intensity
	\[
	\mathbf n(\mathrm ds, \mathrm dw):= 2 \sigma(\xi_s)^2 \mathrm ds \cdot \mathrm N_{\xi_s}(\mathrm dw).
	\]
	\item
	Given $\{(\xi_t)_{t\geq 0}; \mathrm Q_\mu\}$, \emph{the discrete immigration} $\{(X^{\mathrm m,\sigma})_{\sigma\in \mathcal D^{\mathrm m}}; \mathrm Q_\mu(\cdot |\xi)\}$ is a $\mathbb W$-valued point process such that
	\[
	\mathrm m(\mathrm ds, \mathrm dw) := \sum_{\sigma\in \mathcal D^{\mathrm n}} \delta_{(\sigma, X^{\mathrm n,\sigma})}(\mathrm ds, \mathrm dw)
	\]
	is a Poisson random measure on $[0,\infty ) \times \mathbb W$ with intensity
	\begin{align}
		\mathbf m(\mathrm ds, \mathrm dw):= \mathrm ds \cdot \int_{(0,\infty)} y \mathrm P_{y\delta_{\xi_s}}(X\in dw) \pi(\xi_s,dy);
	\end{align}
	\item
	Given $\{(\xi_t)_{t\geq 0}; \mathrm Q_\mu\}$, the continuum immigration $(X^{\mathrm n,\sigma})_{\sigma \in \mathcal D^n}$ and the discrete immigration $(X^{\mathrm m,\sigma})_{\sigma\in \mathcal D^{\mathrm m}}$ are independent of each other.
	\item
	$\{(X_t)_{t\geq 0}; \mathrm Q_\mu\}$ is a copy of the superprocess $\{(X_t)_{t\geq 0}; \mathrm P_\mu\}$, and is independent of the spine process $(\xi_t)_{t\geq 0}$, the continuum immigration $(X^{\mathrm n,\sigma})_{\sigma \in \mathcal D^\mathrm n}$ and the discrete immigration $(X^{\mathrm m,\sigma})_{\sigma\in \mathcal D^{\mathrm m}}$.
\end{itemize}

To simplify notations, for any $\mu \in \mathcal M^0_f(E)$ and	$t\geq 0$,  we define the following random measures:
\begin{align}
	Z^{\mathrm n}_t
	&:= \int_{(0, t]\times \mathbb W} w_{t-s} ~\mathrm n (\mathrm ds, \mathrm dw)
	= \sum_{\sigma \in \mathcal D^\mathrm n \cap (0, t]} X^{\mathrm n,\sigma}_{t-\sigma},
	\\ Z^{\mathrm m}_t
	&:= \int_{(0, t]\times \mathbb W} w_{t-s} ~\mathrm m (\mathrm ds, \mathrm dw)
	= \sum_{\sigma \in \mathcal D^\mathrm m \cap (0, t]} X^{\mathrm m,\sigma}_{t-\sigma}.
\end{align}
Then \begin{equation}\label{def-Zt}
	Z_t= Z^{\mathrm n}_{t} + Z^{\mathrm m}_{t},
\end{equation}
and \eqref{spine-decom1} can be written as
\begin{align}\label{spine-decom2}
	\{(X_t)_{t\geq 0}; \widetilde{\mathrm P}_\mu\}
	\overset{f.d.d.}{=}
	\{(X_t + Z^{\mathrm n}_{t} + Z^{\mathrm m}_{t} )_{t\geq 0}; \mathrm Q_\mu\}.
\end{align}
We call the above representation as spine decomposition of $\{(X_t)_{t\geq 0}; \widetilde{\mathrm P}_\mu\} $.

\begin{proof}[Proof of Theorem \ref{thm:I}]
According to the spine decomposition of $\{(X_t)_{t\geq 0}, \widetilde{\mathrm P_\mu}\}$  given by
\eqref{spine-decom2}, for any
$f\in\mathcal B_b(E,[0,\infty))$,
\[
\widetilde {\mathrm P}_{\mu}\left(e^{-\langle f, X_t\rangle }\right)=\mathrm Q_{\mu}\left(e^{-\langle f, X_t\rangle+\langle f, Z^{{\mathrm m},[0,t)}_t+Z^{{\mathrm n},[0,t)}_t\rangle }\right).
\]

(1) Suppose $\int_El(x)\nu(dx)<\infty$.
When $\mu=\nu$,
it has been shown in the proof of Theorem \ref{thm:L} that the distribution of
$X_t$ under $\widetilde {\mathrm P}_{\nu}$ converges weakly to $\mathbf Q$ as $t\to\infty$.
It was also shown there that $\mathbf Q$ is related to the Yaglom distribution $\mathbf Q_\lambda$ via \eqref{rep: yaglom}.
Note that $k=[\mathbf Q_\lambda(\mu(\phi))]^{-1}$.
\eqref{rep: yaglom} can be rewritten as
\begin{equation}\label{eq size bias}
{\mathbf Q}\left(\mu\in\cdot\right)=
\dfrac{\int_{\mathcal M}\mu(\phi)\mathbf 1_{\mu\in \cdot} \mathbf Q_\lambda(\mathrm d\mu)}{\int_{\mathcal M}\mu(\phi)\mathbf Q_\lambda(\mathrm d\mu)}.
\end{equation}
We now prove that for all $\mu\in\mathcal M^o$, the distribution of $X_t$ under $\widetilde {\mathrm P}_{\mu}$ converges weakly to $\mathbf Q$.  We do this in three steps.

{\bf Step 1}\quad For $f\in\mathcal B_b(E,[0,\infty))$, define
\begin{equation}\label{def: H}
H(x,t):={\mathrm Q}_x\left(e^{-\langle f, Z_{t}\rangle }\right)={\mathrm Q}_x\left(e^{-\langle f, Z^{\mathrm n}_{t} + Z^{\mathrm m}_{t}\rangle }\right),\quad x\in E.
\end{equation}
Set $\overline \eta(x):=\limsup_{t\to\infty}H(x,t), x\in E$.
In this step we prove that
$\overline\eta(\cdot)$ is $\nu$-almost surely a constant function.

Define  $\mathcal{H}_t=\sigma\big(\xi_s; s\leq t\big)$, $t\geq 0$, which is  the filtration generated by the spine process.  Then for $T,t>0$,
\begin{equation}\label{subcritical upper bound}
 \begin{aligned}
 &H(x,t+T)\\
 =&\mathrm Q_{x}\mathrm Q_{x}\Big[\exp\Big\{-\sum_{\sigma\in (0, t+T]\bigcap \mathcal D^{\mathrm m}}\langle f, X_{t+T-\sigma}^{{\mathrm m},\sigma}\rangle -\sum_{\tau\in (0, t+T]\bigcap \mathcal D^{\mathrm n}}\langle f, X_{t+T-\tau}^{{\mathrm n}, \tau}\rangle \Big\}\Big| \mathcal H_t\Big]\\
 \leq&\widetilde\Pi_x\mathrm Q_{x}\Big[\exp\Big\{-\sum_{\sigma\in (t, t+T]\bigcap \mathcal D^{\mathrm m}}\langle f, X_{t+T-\sigma}^{{\mathrm m},\sigma}\rangle -\sum_{\tau\in (t, t+T]\bigcap \mathcal D^{\mathrm n}}\langle f, X_{t+T-\tau}^{{\mathrm n}, \tau}\rangle \Big\}\Big| \mathcal H_t\Big]\\
 =&
   \widetilde\Pi_x\mathrm Q_{\xi_t}\Big[\exp\Big\{-\sum_{\sigma\in (0, T]\bigcap \mathcal D^{\mathrm m}}\langle f, X_{T-\sigma}^{{\mathrm m},\sigma}\rangle -\sum_{\tau\in (0, T]\bigcap \mathcal D^{\mathrm n}}\langle f, X_{T-\tau}^{{\mathrm n}, \tau}\rangle \Big\}\Big]\\
 =&\widetilde\Pi_x\left[ H(\xi_t, T)\right].
 \end{aligned}
 \end{equation}
By \eqref{asp:H2}, there are some constants $c,\rho>0$ such that when $t>1$,
\[
 H(x,t+T)\leq \widetilde\Pi_x\left[ H(\xi_t, T)\right]\leq
 (1+H_{t, x, H(\cdot, T)})\int_E\phi(y)H(y,T)\nu(\mathrm dy)<\infty.
 \]
For fixed  $T>0$, letting $t\to \infty$ in \eqref{subcritical upper bound}, we obtain that
\begin{equation}\label{sub super}
\overline\eta(x)\leq \int_E\phi(y)H(y,T)\nu(\mathrm dy).
\end{equation}
   Using Fatou's lemma, we get that, for any $x\in E$,
\begin{equation}\label{sup inequality}
\overline\eta(x)\leq
\limsup_{T\rightarrow\infty}\int_E\phi(y)H(y,T)\nu(\mathrm dy)
\leq \int_E\phi(y)\overline{\eta}(y)\nu(\mathrm dy).
\end{equation}
Using the fact that  $\overline{\eta}(\cdot)\leq 1$,
we see that $\overline\eta(\cdot)$ is $\nu$-almost surely a constant function.

{\bf Step 2}\quad
 Denote the $\nu$-\text{a.s.} value of the function $\overline\eta(\cdot)$ by $q(f)$.  In this step we prove that
 \begin{equation}\label{limit-H}
 \lim_{t\rightarrow\infty}H(x,t)=q(f),\qquad
 \text{ for } \nu-\text{a.e. }x\in E.
 \end{equation}
If $q(f)= 0,$ then the above is true obviously. So in the
following, we assume $q(f)>0$.

We first claim that  as $T\to\infty$, $H(\cdot,T)$ converges to $q(f)$ in probability
with respect to the probability measure $\phi(x)\nu(\mathrm dx)$.
For  any $\varepsilon_1\in (0, 1)$, let
$$
\mu_1(T)=\int_{\{x\in E;H(x,T)>(1+\varepsilon_1)q(f)\}}
\phi(x)\nu(\mathrm dx).
$$
Then $\limsup_{T\to\infty}H(x,T)=q(f)$ implies that $\lim_{T\rightarrow\infty}\mu_1(T)=0.$
For any $\varepsilon_2\in (0, 1)$, let
$$
\mu_2(T)=\int_{\{x\in E;H(x,T)<(1-\varepsilon_2)q(f)\}}
\phi(x)\nu(\mathrm dx).
$$
To prove the claim above, we only need to prove that  $\limsup_{T\rightarrow\infty}\mu_2(T)=0.$
It follows  from \eqref{sup inequality} that
\begin{eqnarray}\label{sublimitinprob}
q(f)&\leq&
(1-\varepsilon_2)q(f)\mu_2(T)+\mu_1(T)+(1+\varepsilon_1)q(f)(1-\mu_1(T)-\mu_2(T))\\
&\le
&(1+\varepsilon_1)q(f)-(\varepsilon_1+\varepsilon_2)q(f)\mu_2(T)+\mu_1(T).
\end{eqnarray}
 Hence
\begin{eqnarray*}\label{sublimitinequl}
q(f)&\leq&
\liminf_{T\rightarrow\infty}\left[(1+\varepsilon_1)q(f)-(\varepsilon_1+\varepsilon_2)\mu_2(T)
+\mu_1(T)\right]\\
&=&(1+\varepsilon_1)q(f)-(\varepsilon_1+\varepsilon_2)q(f)\limsup_{T\rightarrow\infty}\mu_2(T).
\end{eqnarray*}
Since $\varepsilon_1$ is an arbitrary positive constant.
\[
q(f)\leq q(f)-\varepsilon_2 q(f)\limsup_{T\rightarrow\infty}\mu_2(T).
\]
This is impossible unless $\limsup_{T\rightarrow\infty}\mu_2(T)=0.$
Thus the claim above is valid.


By the definition of $H$ given by \eqref{def: H},  we have
\begin{equation}\label{subsub}
\begin{aligned}
     H(x,t+T)\geq& \mathrm Q_{x}\prod_{\sigma\leq t}I_{\{ X_{t+T-\sigma}^{{\mathrm m},\sigma}=0\}}\prod_{\tau\leq t}I_{\{ X_{t+T-\tau}^{{\mathrm n},\tau}=0\}}\\
&\cdot\mathrm Q_{\xi_t}\Big[\exp\Big\{-\sum_{\sigma\in (0, T]\bigcap \mathcal D^{\mathrm m}}\langle f, X_{T-\sigma}^{{\mathrm m},\sigma}\rangle -\sum_{\tau\in (0, T]\bigcap \mathcal D^{\mathrm n}}\langle f, X_{T-\tau}^{{\mathrm n},\tau}\rangle \Big\}\Big]\\
=& \mathrm Q_{x}\left[\prod_{\sigma\leq t}I_{\{ X_{t+T-\sigma}^{{\mathrm m},\sigma}=0\}}\prod_{\tau\leq t}I_{\{ X_{t+T-\tau}^{{\mathrm n},\tau}=0\}}H(\xi_t, T)\right].
\end{aligned}
\end{equation}
Note that
\begin{eqnarray*}
\mathrm Q_{x}\left(\prod_{\sigma\leq t}I_{\{ X_{t+T-\sigma}^{{\mathrm m},\sigma}=0\}}=1\right)
=\widetilde\Pi_x\exp\left\{-\int_0^t\mathrm ds\int_0^\infty r(1-\mathrm P_{r\delta_{xi_s}}(X_{T+t-s}\neq\mathbf 0))\pi(\xi_s,\mathrm dr)\right\}.
\end{eqnarray*}
Lemma \ref{lem:extinc} (2) tells us the $(\xi,\psi)$-superprocess starting from any finite measure becomes extinct in finite time.  Therefore
 by the dominated convergence theorem,
\begin{equation}\label{1infty limit}
\lim_{T\rightarrow\infty}\int_0^t\mathrm ds\int_1^\infty r(1-\mathrm P_{r\delta_{\xi_s}}(X_{T+t-s}\neq\mathbf 0))\pi(\xi_s,\mathrm dr)=0,\quad \widetilde\Pi_x-\mbox{a.s.}
\end{equation}
  Note that
\[
1-\mathrm P_{r\delta_{\xi_s}}(X_{T+t-s}\neq\mathbf 0)\leq 1-(1-\mathrm P_{\xi_s}(X_T\neq\mathbf 0))^r.
\]
By Lemma \ref{l:new},
there are $T_0>0$ and $\eta>0$ such that when $T>T_0$,
\[
\mathrm P_x(X_T\neq \mathbf 0)\leq \eta \phi(x)e^{\lambda T}, \quad x\in E.
\]
Using elementary analysis one can easily check that
$$
1-(1-a)^r\le 2 ra, \quad a\le (0, \frac12), r\in (0, 1].
$$
By taking $T_0$  sufficiently large we have
$1-(1-\mathrm P_{\xi_s}(\zeta>T))^r\leq 2r\eta \phi(\xi_s)e^{\lambda T}$ for all  $T>T_0$ and $r\in(0,1]$.
Therefore,
\[
\int_0^tds\int_0^1 r(1-\mathrm P_{r\delta_{ \xi_s}}(X_{T+t-s}\neq \mathbf 0))\pi(\xi_s,dr)\leq 2\eta e^{\lambda T}\int_0^t\phi(\xi_s)ds\int_0^1 r^2 \pi(\xi_s,dr).
\]
Thus
\begin{equation}\label{01limit}
\lim_{T\rightarrow\infty}\int_0^t\mathrm ds\int_0^1 r(1-\mathrm P_{r\delta_{\xi_s}}(X_{T+t-s}\neq \mathbf 0))\pi(\xi_s,\mathrm dr)=0, \quad \widetilde\Pi_x-\mbox{a.s.}
\end{equation}
 Combining \eqref{1infty limit} and \eqref{01limit}, we get
\[
\lim_{T\rightarrow\infty}\mathrm Q_{x}\left(\prod_{\sigma\leq t}I_{\{ X_{t+T-\sigma}^{{\mathrm m},\sigma}=0\}}=1\right)=1.
\]
Similarly,
\begin{eqnarray*}
\mathrm Q_x\left(\prod_{\sigma\leq t}I_{\{ X_{t+T-\sigma}^{{\mathrm n},\sigma}=0\}}=1\right)
&=&\widetilde\Pi_x\exp\left\{-\int_0^t2\sigma(\xi_s)^2\mathrm N_{\xi_s}(X_{T+t-s}\neq \mathbf 0)\mathrm ds\right\}\\
&=&\widetilde\Pi_x\exp\left\{-\int_0^t2\sigma(\xi_s)^2v_{T+t-s}(\xi_s)\mathrm ds\right\}.
\end{eqnarray*}
We know that $v_{T+r}(x)$ is  bounded for $(r,x)\in (0,\infty)\times E$ when $T$ is large enough, and that  $\lim_{T\rightarrow\infty} v_{T+t-s}(x)=0$ for any $x$.  Thus by the bounded convergence theorem,
\[
\lim_{T\rightarrow\infty}\mathrm Q_x\left(\prod_{\tau\leq t}I_{\{ X_{t+T-\tau}^{{\mathrm n},\tau}=0\}}=1\right)=1,
\]
By \eqref{asp:H2},
for any $\varepsilon>0$ and $t>1$,
such that for any $x\in E$,
\begin{eqnarray*}
&&\limsup_{T\rightarrow\infty}\widetilde\Pi_x\left(|H(\xi_t, T)-q(f)|>\varepsilon\right)\\
&\leq& \limsup_{T\rightarrow\infty}(1+H_{t, x, H(\cdot, T)})\int_E\mathbf 1_{\{|H(y, T)-q(f)|>\varepsilon\}}\phi(y)\nu(\mathrm dy)=0.
\end{eqnarray*}
Then from the inequality \eqref{subsub}, we have for any $x\in E$,
\begin{eqnarray*}
\liminf_{T\rightarrow\infty}H(x, t+T)&\geq&  \liminf_{T\rightarrow\infty} \mathrm Q_x\left[\prod_{\sigma\leq t}I_{\{ X_{t+T-\sigma}^{{\mathrm m},\sigma}=0\}}\prod_{\tau\leq t}I_{\{ X_{t+T-\tau}^{{\mathrm n},\tau}=0\}}H(\xi_t, T)\right]\\
&\geq& q(f)=\limsup_{t\rightarrow\infty}H(x, t).
\end{eqnarray*}
 Therefore \eqref{limit-H} holds.

{\bf Step 3}\quad
Since $0\leq H(x,t)\leq 1$,
\begin{equation*}
q(f)
=\lim_{t\rightarrow\infty}\int_E\phi(x)H(x,t)\nu(\mathrm dx)
=\lim_{t\rightarrow\infty}\widetilde{\mathrm P}_{\nu}\left(e^{-\langle f, X_t\rangle }\right)
=\mathbf Q(e^{-\mu(f)}).
\end{equation*}
Therefore for any $\mu\in\mathcal M^o$, and $f\in\mathcal B_b(E,[0,\infty))$,
\begin{eqnarray*}
\lim_{t\rightarrow\infty}\widetilde{\mathrm P_\mu}\left(e^{-\langle f, X_t\rangle}\right)&=&\lim_{t\rightarrow\infty}\mathrm P_\mu\left(e^{-\langle f, X_t\rangle}\right)
\lim_{t\to\infty}\dfrac{1}{\mu(\phi)}\int_E\phi(x)H(x, t)\mu(\mathrm dx)\\
&=&q(f)=\mathbf Q(e^{-\mu(f)}).
\end{eqnarray*}
This says $\mathbf Q$, the distribution limit of $X_t$ under $\widetilde{\mathrm P}_{\nu}$, is the  limit distribution of  $X_t$ under $\widetilde{\mathrm P_\mu}$ for any $\mu\in{\mathcal M}^o$. Thus $\mathbf Q$ is the
equilibrium distribution of the $Q$-process, and is a size-biased distribution of the Yaglom probability $\mathbf Q_\lambda$ with weight function $\dfrac{\mu(\phi)}{\int_{{\mathcal M}}\mu(\phi)\mathbf Q_\lambda(\mathrm d\mu)}$.  The proof of $(1)$ is complete.



(2) When $\int_El(x)\nu(dx)=\infty$,
by Theorem \ref{thm:E}, for any $\mu\in \mathcal M^0$, we have
$$
\lim_{to\infty}e^{-\lambda t}\mathrm P_\mu(X_t\neq\mathbf 0)=0.
$$
By \eqref{eq:M.3}, we have
$$
e^{-\lambda t}\mathrm P_\mu(X_t\neq\mathbf 0)=\mu(\phi)\widetilde{P}_\mu[X_t(\phi)^{-1}].
$$
Combining the two displays above we get that $X_t(\phi)^{-1}$ converges to zero in probability
with respect to $\mathrm P_\mu$, which is equivalent to the desired result.
\end{proof}

\appendix
\section{}

\subsection{Weak convergence and vague convergence}
\begin{lem} \label{thm:A.1}
	Let $E$ be a Polish space.
	Let $\{\mu\}\cup\{\mu_n:n\in \mathrm N\}$ be a sequence of finite Borel measures on $E$.
	Suppose that $\mu_n$ converges to $\mu$ vaguely as $n\to \infty$, and that $\mu_n(E)$ converges to $\mu(E)$ as $n\to \infty$.
	Then $\mu_n$ converges to $\mu$ weakly.
\end{lem}
\begin{proof}
	Let $(\phi_j)_{j\in \mathrm N}$ be a sequence of function on $E$ such that
	(1) for each $j\in \mathrm N$, $\phi_j$ is compactly supported, $[0,1]$-valued, and continuous;
	(2) for each $x\in E$, $\phi_j(x)\xrightarrow[j\to \infty]{} 1$.
	Fix an arbitrary bounded continuous function $f$ on $E$.
	Then we can verify that for each $n$ and $j\in \mathrm N$,
 \begin{align}
 	&  |\mu_n(f) -\mu(f)|
 	= \big|\big[\mu_n(f\phi_j) + \mu_n\big(f(1-\phi_j)\big) \big]- \big[\mu(f\phi_j) + \mu\big(f(1-\phi_j)\big)\big]\big|
 	\\&\leq|\mu_n(f\phi_j) - \mu(f\phi_j)| + \big|\mu_n\big(f(1-\phi_j)\big)\big| + \big|\mu\big(f(1-\phi_j)\big)\big|
 	\\&\leq \big|\mu_n(f\phi_j) - \mu(f\phi_j)\big| + \|f\|_\infty \mu_n(1-\phi_j) + \|f\|_\infty\mu(1-\phi_j)
 	\\&\leq \big|\mu_n(f\phi_j) - \mu(f\phi_j)\big| + \|f\|_\infty [\mu_n(1-\phi_j)-\mu(1-\phi_j)] + 2\|f\|_\infty\mu\big(1-\phi_j\big)
 	\\&\leq |\mu_n(f\phi_j)-\mu(f\phi_j)| + \|f\|_\infty \big(|\mu_n(E)- \mu(E)| + |\mu_n(\phi_j)-\mu(\phi_j)|\big) + 2\|f\|_\infty \mu(1-\phi_j).
 \end{align}
 From this and the assumptions of the lemma, we have for each $j\in \mathrm N$,
 \begin{align}
 	\limsup_{n\to \infty} |\mu_n(f) - \mu(f)| \leq 2\|f\|_\infty \mu(1-\phi_j).
 \end{align}
 Finally, taking $j \to \infty$ above, using bounded convergence theorem, we have
 \[
 \lim_{n\to \infty} |\mu_n(f)-\mu(f)| = 0. \qedhere
 \]
\end{proof}

\subsection{Tauberian theorems}

In this subsection we collect some results on regularly varying functions and  Tauberian theorems, which are used in this paper.
The following lemma is from \cite[Appendix 13.6]{AH}.
\begin{lem}\label{lem:regu}
Let $f(x)$ be non-decreasing in $x\in (0,c)$. If
\[
\lim_{n\to\infty}\dfrac{f(\lambda\theta_n)}{f(\theta_n)}=\lambda^\alpha,\qquad \forall \lambda\in (0,1],
\]
for some $\alpha\in\mathbb R$ and some sequence $\{\theta_n\}$ of positive reals tending to $0$, as $n\to\infty$ in such a way that $\theta_n/\theta_{n+1}\leq c$ for $n\in\mathrm N$ and some $1<c<\infty$, then $f(x)$ is regularly varying with exponent $\alpha$.
\end{lem}


The following two lemmas are from \cite[Appendix 14]{AH}.
\begin{lem}\label{lem: tau}
Let $U(x)$ be a non-decreasing function on $[0,\infty)$ such that
\[
w(x)=\int_0^\infty e^{-xu} \mathrm dU(u)
\]
is finite for all $x>0$. If for some $\alpha\geq 0$, $w(x)\sim x^{-\alpha}L(1/x)$, $x\downarrow 0$, where $L$ is slowly varying at infinity, then
\[
U(x)\sim x^{\alpha}\dfrac{L(x)}{\Gamma(\alpha+1)},\qquad x\to\infty.
\]
 And if for some $\alpha\geq 0$, $w(x)\sim x^{-\alpha}L(x)$, $x\uparrow \infty$, then
\[
U(x)\sim x^{\alpha}\dfrac{L(1/x)}{\Gamma(\alpha+1)},\qquad x\downarrow 0.
\]
\end{lem}

\begin{lem}\label{lem:tail}
	For $x\in [\beta,\infty)$, let
\[
	U(x)
	= \int_\beta^xu(y)\mathrm dy,
\]
	where $u(y)$ is ultimately monotone.  If for some $\alpha\geq 0$, $U(x)=x^\alpha L(x)$,  where $L$ is slowly varying at infinity, then
\[
	\lim_{x\to\infty}\dfrac{xu(x)}{U(x)}
	= \alpha.
\]
\end{lem}





\begin{thebibliography}{99}
	
\bibitem{AH}Asmussens, S. and Hering, H. :\emph{Branching Processes}. Birkhauser, Boston, 1983.

\bibitem{AthreyaNey1972Branching}
Athreya, K. B. and Ney, P. E.:
\emph{Branching processes.}
Die Grundlehren der mathematischen Wissenschaften, Band 196. Springer-Verlag, New York-Heidelberg, 1972. xi+287 pp.
\MR{0373040}

\bibitem{BigginsKyprianou2004Measure}
Biggins, J. D. and Kyprianou, A. E.:
\emph{Measure change in multitype branching.}
Adv. in Appl. Probab. \textbf{36} (2004), no. 2, 544--581.
\MR{2058149}

\bibitem{ChampagnatRoelly2008Limit}
Champagnat, N. and Roelly, S.:
\emph{Limit theorems for conditioned multitype Dawson-Watanabe processes and Feller diffusions.}
Electron. J. Probab. \textbf{13} (2008), no. 25, 777C810.
\MR{2399296}

\bibitem{ChampagnatVillemonais2018Convergence}
Champagnat, N. and Villemonais, D.:
\emph{Convergence of the Fleming-Viot process toward
theminimal quasi-stationary distribution.}
https://urldefense.com/v3/__https://arxiv.org/pdf/1810.06849.pdf__;!!DZ3fjg!v9ktLEV6VKS8zoP9oP0VQ6ru7ocUg33iWCZqsEQDmR9BPy-Hf63327ONEcuGz1qx$ 

\bibitem{ChenRenYang2017Skeleton}
Chen, Z.-Q., Ren, Y.-X. and Yang, T.:
\emph{Skeleton decomposition and law of large numbers for supercritical superprocesses.}
Acta Appl. Math. 159(1)(2019) 225-285

\bibitem{DaPratoZabczyk1996Ergodicity}
	Da Prato, G. and Zabczyk, J.:
	\emph{Ergodicity for infinite dimensional systems.}
	Cambridge University Press, 1996

\bibitem{Dawson1992Infinitely}
Dawson, D. A.:
\emph{Infinitely divisible random measures and superprocesses.}Stochastic analysis and related topics (Silivri, 1990), 1--129,
Progr. Probab., 31, Birkh{\"a}user Boston, Boston, MA, 1992.
\MR{1203373}

\bibitem{DelmasHenard2013A-Williams}
Delmas, J.-F. and H\'enard, O.:
\emph{A Williams decomposition for spatially dependent super-processes. }
Electron. J. Probab. \textbf{18} (2013), no. 37, 43 pp.
\MR{3035765}

\bibitem{Dudley2002Real}
Dudley, R. M.:
\emph{Real analysis and probability.}
Revised reprint of the 1989 original. Cambridge Studies in Advanced Mathematics, 74. Cambridge University Press, Cambridge, 2002. x+555 pp.

\bibitem{Dynkin1993Superprocesses}
Dynkin, E. B.:
\emph{Superprocesses and partial differential equations.}
Ann. Probab. \textbf{21} (1993), no. 3, 1185--1262.
\MR{1235414}

\bibitem{EnglanderKyprianou2004Local}
Engl\"ander, J. and Kyprianou, A. E.:
\emph{Local extinction versus local exponential growth for spatial branching processes.}
Ann. Probab. \textbf{32} (2004), no. 1A, 78--99.
\MR{2040776}

\bibitem{Evans1993Two}
Evans, S. N.:
\emph{Two representations of a conditioned superprocess.}
Proc. Roy. Soc. Edinburgh Sect. A \textbf{123} (1993), no. 5, 959--971.
\MR{1249698}

%new added
\bibitem{EW} Etheridge, A. and Williams, D.R.E. A decomposition of the $(1+\beta)$-superprocess conditioned on survival. \emph{Proceed. of the Royal Soc. of Edinburgh} (2004)

\bibitem{EP} Evans, S.N. and Perkins, E. Measure-valued Markov Branching processes conditioned on non-extinction. \emph{Israel J. Math.}, 71, 329--337 (1990). MR1088825

%end new

\bibitem{Fitz}
Fitzsimmons, P. J.:
\emph{Construction and regularity of measure-valued Markov branching processes.}
Israel J. Math. \textbf{64} (1988), no. 3, 337¨C-361.

\bibitem{Grey1974Asymptotic}
Grey, D. R.:
\emph{Asymptotic behaviour of continuous time, continuous state-space branching processes.}
J. Appl. Probability \textbf{11} (1974), 669--677.
\MR{0408016}

\bibitem{Heathcote}
Heathcote, R.,  Seneta, E.  and Vere-Jones, D.:
\emph{ A refinement of two theorems in the theory of branching processes}.
Theory Probab. Appl. 12 (1982), 297-301.

\bibitem{HeathcoteSenetaVere-Jones1967A-refinement}
Heathcote, C. R., Seneta, E. and Vere-Jones, D.:
\emph{A refinement of two theorems in the theory of branching processes.} (Russian summary)
Teor. Verojatnost. i Primenen. \textbf{12} 1967 341--346.
\MR{0217889}



\bibitem{Joffe}Joffe, A. and Waughw, A. O.:  Exact distributions of kin numbers in a Galton-Watson
process. J. Appl. Probab. 19 (1982), 767-775.

\bibitem{Kallenberg2002Foundations}	
	Kallenberg, O.:
	\emph{Foundations of Modern probability.}
	Second Edition.	Springer-Verlag New York, 2002.
	
\bibitem{Kallenberg2017Random}
	Kallenberg, O.:
	\emph{Random measures, theory and applications.}
	Cham: Springer International Publishing, 2017.

\bibitem{KimSong2008Intrinsic}
Kim, P., Song, R.:
\emph{Intrinsic ultracontractivity of non-symmetric diffusion semigroups in bounded domains.}
Tohoku Math. J. (2) 60 (2008), no. 4, 527-547.
\MR{2487824}

\bibitem{KimSong2008Intrinsic2}
Kim, P. and Song, R.:
\emph{Intrinsic ultracontractivity of nonsymmetric diffusions with measure-valued drifts and potentials.}
Ann. Probab. \textbf{36} (2008), no. 5, 1904--1945.
\MR{2440927}

\bibitem{KimSong2009Intrinsic}
Kim, P. and Song, R.:
\emph{Intrinsic ultracontractivity for non-symmetric L\'evy processes.}
Forum Math. \textbf{21} (2009), no. 1, 43C66.
\MR{2494884}

\bibitem{Kyprianou2014Fluctuations}
	Kyprianou, A. E.:
	\emph{Fluctuations of L¨¦vy processes with applications: Introductory Lectures.}
	Springer Science \& Business Media, 2014.

\bibitem{Lambert2001Arbres}
Lambert, A.:
\emph{Arbres, excursions et processus de L\'evy completement asym\'etriques.}
Diss. Universit Pierre et Marie Curie-Paris VI, 2001.

\bibitem{Lambert2003Coalescence}
Lambert, A.:
\emph{Coalescence times for the branching process.}
Adv. in Appl. Probab. \textbf{35} (2003), no. 4, 1071--1089.
\MR{2014270}

\bibitem{Lambert2007Quasi-stationary}
Lambert, A.:
\emph{Quasi-stationary distributions and the continuous-state branching process conditioned to be never extinct.}
Electron. J. Probab. \textbf{12} (2007), no. 14, 420--446.
\MR{2299923}

%new added
\bibitem{LamandNey}Lamperti,J. and Ney, P. \emph{Conditioned branching process and their limiting diffusions.} Theory Probab. Appl. \textbf{13} (1968)
126--137.
%end new

\bibitem{Li00}
Li, Z.-H.:
\emph{Asymptotic behaviour of continuous time and state branching processes.}
J. Austral. Math. Soc. Ser. A \textbf{68} (2000), no. 1, 68--84.
\MR{1727226}

\bibitem{Li2011Measure-valued}
Li, Z.:
\emph{Measure-valued branching Markov processes.}
Probability and its Applications (New York). Springer, Heidelberg, 2011. xii+350 pp. ISBN: 978-3-642-15003-6
\MR{2760602}

%new added
\bibitem{Li20}
Li, Z.-H.:
\emph{Ergodicities and exponential ergodicities of Dawson-Watanabe type processes.} 
https://urldefense.com/v3/__https://arxiv.org/pdf/2002.09111.pdf__;!!DZ3fjg!v9ktLEV6VKS8zoP9oP0VQ6ru7ocUg33iWCZqsEQDmR9BPy-Hf63327ONETyoEWFY$ 
%end new

\bibitem{LiuRenSong2009Llog}
Liu, R.-L., Ren, Y.-X. and Song, R.:
\emph{{$L \log L$} criterion for a class of superdiffusions.}
J. Appl. Probab. \textbf{46} (2009), no. 2, 479--496.
\MR{2535827}

\bibitem{LiuRenSongSun2020}
Liu, R.-L., Ren, Y.-X., Song, R. and Sun, Z.
\emph{Quasi-stationary distributions for subcritical superprocesses.}
Stoch. Proc. Appl. \textbf{132} (2021), 108¨C-134.



\bibitem{LyonsPemantlePeres1995Conceptual}
Lyons, R., Pemantle, R. and Peres, Y.:
\emph{Conceptual proofs of $L\log L$ criteria for mean behavior of branching processes.}
Ann. Probab. \textbf{23} (1995), no. 3, 1125--1138.
\MR{1349164}

\bibitem{MeleardVillemonais2012Quasi-stationary}
M\'el\'eard, S. and Villemonais, D.:
\emph{Quasi-stationary distributions and population processes.}
Probab. Surv. \textbf{9} (2012), 340C410.
\MR{2994898}

\bibitem{Nagasawa1964Time}
Nagasawa, M.:
\emph{Time reversions of Markov processes.}
Nagoya Math. J. \textbf{24} (1964), 177--204.
\MR{0169290}

\bibitem{Penisson2010Conditional}
P\'enisson, S.:
\emph{Conditional limit theorems for multitype branching processes and illustration in epidemiological risk analysis.}Diss. Universitt Potsdam, Universit Paris Sud-Paris XI, 2010.

%new added
\bibitem{Penisson2011Conditional}P\'enisson, S.:
\emph{Continuous-time multitype branching processes conditioned on very late extinction}ESAIM: Probability and Statistics. \textbf{15}(2011) 417--442.
%end new

\bibitem{RenSongSun2020Spine}
	Ren, Y.-X., Song, R., and Sun, Z.:
	\emph{Spine decompositions and limit theorems for a class of critical superprocesses.}
	Acta Appl. Math. \textbf{165} (2020): 91--131.

\bibitem{RenSongZhang2015Limit}
Ren, Y.-X., Song, R., and Zhang, R.:
\emph{Limit theorems for some critical superprocesses.}
Illinois J. Math. 59 (2015), no. 1, 235-276.

\bibitem{RenSongZhang2017Central}
Ren, Y.-X., Song, R., and Zhang, R.:
\emph{Central limit theorems for supercritical branching nonsymmetric Markov processes.}
Ann. Probab. 45 (2017), no. 1, 564-623.

\bibitem{RenSongZhang2018Williams}
Ren, Y.-X., Song, R. and Zhang, R.:
\emph{Williams decomposition for superprocesses.}
Electron. J. Probab. \textbf{23} (2018), Paper No. 23, 33 pp.
\MR{3771760}

\bibitem{RenSongYang2016Spine}
Ren, Y.-X., Song, R. and Yang, T.:
\emph{Spine decomposition and {$ L\log L $} criterion for superprocesses with non-local branching mechanisms.}
Preprint.
ARXIV{1609.02257}

\bibitem{RoellyRouault1989Processus}
Roelly, S. and Rouault, A.:
\emph{Processus de Dawson-Watanabe conditionn\'e par le futur lointain.} (French. English summary) [A Dawson-Watanabe process conditioned by the remote future]
C. R. Acad. Sci. Paris Sr. I Math. \textbf{309} (1989), no. 14, 867--872.
\MR{1055211}

\bibitem{Schaefer1974Banach}
Schaefer, H. H.:
\emph{Banach lattices and positive operators.}
Die Grundlehren der mathematischen Wissenschaften, Band 215. Springer-Verlag, New York-Heidelberg, 1974. xi+376 pp. \MR{0423039}

\bibitem{SM}Sylvie M\'el\'eard, Denis Villemonais (2012): Quasi-stationary distributions and
population processes.Probability Surveys 9: 340--410.

\bibitem{SP} S. P\'enisson (2010): Conditional limit theorems for multitype branching
processes and illustration in epidemiological risk analysis. Mathematics [math]. Universit\:at Potsdam; Universit\'e Paris Sud - Paris
XI. English. <tel-00570458>

%new
\bibitem{Sharpe} M. Sharpe (1988): General theory of Markov processes. Academic Press, Boston, MA.
%end new

\bibitem{Sta}Stannat, W.: \emph{On transition semigroups of $(A,\Psi)$-superprocesses with immigration.} Ann. Probab. \textbf{31} (2003),1377--1412.




\bibitem{Yaglom1947}
Yaglom, A. M.:
\emph{Certain limit theorems of the theory of branching random processes.} (Russian)
Doklady Akad. Nauk SSSR (N.S.) \textbf{56} (1947), 795--798.
\MR{0022045}

\end{thebibliography}
\end{document}
