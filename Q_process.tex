%Rongli Q_process 2020-02-11
%Zhenyao Q_process 2020-06-10
%Yanxia Q_process2 2020-06-22
%Rongli Q_process3 2020-07-01
%Yanxia Q_process4 2020-07-01
%Zhenyao Q_process5 2020-07-05
%Renming Q_process6 2020-07-06
%Zhenyao Q_process7 2020-07-19
\documentclass[12pt,a4paper]{amsart}
\setlength{\textwidth}{\paperwidth}
\addtolength{\textwidth}{-2in}
\calclayout
\numberwithin{equation}{section}
\allowdisplaybreaks
\theoremstyle{plain}
\newtheorem{thm}{Theorem}[section]
\newtheorem{asp:mean}{Assumption}[section]
\newtheorem{lem}[thm]{Lemma}
\newtheorem{prop}[thm]{Proposition}
\newtheorem{cor}[thm]{Corollaray}
\newtheorem{fact}[thm]{Fact}
\newtheorem{remark}{Remark}
\newtheorem{claim}[thm]{Claim}
\theoremstyle{definition}
\newtheorem*{ack*}{Acknowledgment}
\theoremstyle{remark}
\newtheorem{exa}[thm]{Example}
\usepackage{amssymb}
\usepackage{mathtools}
\mathtoolsset{showonlyrefs}
\usepackage{mathrsfs}
\usepackage{comment}
%\usepackage{hyperref}
\usepackage[backref]{hyperref}
\usepackage[inline]{showlabels}
\usepackage{xcolor}
\newenvironment{proof*}[1][\proofname]{
	\renewcommand\qedsymbol{\rule{3mm}{3mm}}
	\begin{proof}[#1]}{\end{proof}}

\begin{document}
\title {Subcritical Superprocesses Conditioned on Non-extinction}
\author[R. Liu]{Rongli Liu}
\address{Rongli Liu\\ Mathematics and Applied Mathematics\\ Beijing jiaotong University\\ Beijing 100044\\ P. R. China}
\email{rlliu@bjtu.edu.cn}
\thanks{The research of Rongli Liu is supported in part by NSFC (Grant No. 11301261), and the Fundamental Research Funds for the Central Universities (Grant No.  2017RC007)}
\author[Y.-X. Ren]{Yan-Xia Ren}
\address{Yan-Xia Ren\\ LMAM School of Mathematical Sciences \& Center for
Statistical Science\\ Peking University\\ Beijing 100871\\ P. R. China}
\email{yxren@math.pku.edu.cn}
\thanks{The research of Yan-Xia Ren is supported in part by NSFC (Grant Nos. 11671017 and 11731009)  and LMEQF.}
\author[R. Song]{Renming Song}
\address{Renming Song\\ Department of Mathematics\\ University of Illinois at Urbana-Champaign \\ Urbana \\ IL 61801\\ USA}
\email{rsong@illinois.edu}
\author[Z. Sun]{Zhenyao Sun}
\address{Zhenyao Sun\\ Faculty of Industrial Engineering and Management \\ Technion, Isreal Institute of Technology \\ Haifa 3200003\\ Isreal}
\email{zhenyao.sun@gmail.com}
\begin{abstract}
\begin{comment}
Suppose that $E$ is a locally compact separable metric space, and  that  $X =\{(X_t)_{t\geq 0}; (\mathbb P_\mu)_{\mu \in \mathcal M_f(E)}\}$  is a subcritical superprocess, where $\mathcal M_f(E)$ is the space of all finite Borel measures on $E$. Under some conditions on the mean semigroup of $X$, we prove that the the $Q$ process of $X$ exists, and they have equilibrium probability if and only if the moment condition $\int_El(x)\nu({\mathrm d}x)<\infty$ is satisfied.
We also show that the equilibrium probability is a size-biased measure of the Yaglom distribution.
\end{comment}
    TBD
\end{abstract}
\maketitle
\section{Introduction}
\subsection{Background}
\begin{comment}
The study of the extinction of populations is of a great interest in biology. Conditioning on non-extinction can not only notably lead to a stationary behavior
of the process, but also provides a lot of information about the evolution of the population before extinction.  As far as the population dynamics which are extinct in finite time running over large amounts of time are concerned, special attentions are given to two conditional limits: one type of limits are the Yaglom distributions and the related quasi-stationary distributions, the other type of limits are the Q processes.   There are plenty of literatures investigating these properties of branching processes. See, for instance, Athreya and Ney \cite[pages 64-65]{AthreyaNey1972Branching},Grey\cite{Grey1974Asymptotic},Lyons, Pemantle and Peres \cite{LyonsPemantlePeres1995Conceptual}, Seneta and
Vere-Jones \cite{Heathcote},Heathcote,Joffe \cite{Joffe}, Yaglom \cite{Yaglom47} for Galton-Watson branching processes, Lambert\cite{Lambert2001Arbres, Lambert2003Coalescence},Li\cite[Theorem 4.3]{Li00} for continuous state and continuous time branching processes. Asmussen and Hering \cite{AH} studied limit behaviour of subcritical branching Markov processes, in which each particle  lives for exponential time, then give birth to random number of particles, and particles move as independent
Markov processes in between branching times and it is assume that  the life times, reproduction of different individuals is independent.

In the authors' paper \cite{LSYS}, the Yaglom distributions of the superprocesses are investigated.  In this paper, we are mainly studying the properties of the Q processes for superprocesses and discover their relationships with the Yaglom distributions..
\end{comment}
TBD 
\subsection{Main Result}\label{sec:M}
	We first recall the definition of a superprocess. 
%{	Let $E$ be a Polish space. 
%	Let $\partial$ be an isolated point not contained in $E$ and denote $E_\partial := E \cup \{\partial\}$. 
	Let $E$ be a locally compact separable metric space and $E_\partial := E \cup \{\partial\}$ the one-point compactification of $E$. \footnote{RS: For quasi-left continuity at the lifetime, it is better to take the one-point compactification.}\footnote{ZS: Let $E$ be a locally compact separable metric space. Let $(E_\partial, \mathcal T)$ be the one-point commodification of $E$ where $\mathcal T$ denotes all the corresponding open sets. Let $(E_\partial, \widetilde{\mathcal T})$ be a topology space by adding an isolated point $\partial$ to $E$ where $\widetilde{\mathcal T}$ denotes all the corresponding open sets. Then $\mathcal T \subsetneqq \widetilde {\mathcal T}$. (In fact $\{\partial\} \in \widetilde{\mathcal T}\setminus \mathcal T$.) Therefore a $(E_\partial,\mathcal T)$-valued Hunt process is not  necessarily a $(E_\partial,\widetilde{\mathcal T})$-valued Hunt process. \cite[Theorem 5.11]{Li2011Measure-valued} only proved that if $\xi$ is a $(E_\partial,\widetilde{\mathcal T})$-valued Hunt process, then its corresponding superprocess has Hunt version.}  %}

	Denote by $\mathcal B(E, D)$ the collection of Borel maps  from $E$ to some metric space $D$.
	Denote by $\mathcal B_b(E,D)$ the metrically bounded elements in $\mathcal B(E,D)$.  
	Let \emph{the underlying process} $\xi = \{(\xi_t)_{t\ge0}; (\Pi_x)_{x\in E}\}$ be an $E_\partial$-valued Hunt process with $\partial$ as an absorbing state. 
	Denote by $\zeta:=\inf\{t>0: \xi_t=\partial\}$ the lifetime of $\xi$. 
	Let \emph{the branching mechanism} $\psi$ be a function on $E \times \mathbb R_+$ given by 
\begin{align}
	\psi(x,z)
	= -\beta(x) z + \sigma(x)^2 z^2 + \int_{(0,\infty)} (e^{-zu} -1 + zu) \pi(x,{\mathrm d}u), 
	\quad x\in E, z\geq 0
\end{align}
	where $\beta, \sigma \in \mathcal B_b(E,\mathbb R)$ and $(u \wedge u^2) \pi(x,{\mathrm d}u)$ is a bounded kernel from $E$ to $(0,\infty)$.
	Let $\mathcal M_f(E)$ denote the space of all finite Borel measures on $E$ equipped with the topology of weak convergence.
	For any $\mu \in \mathcal M_f(E)$ and $f\in \mathcal B(E,\mathbb R)$, we use $\mu(f)$ to denote the integral of $f$ with respect to $\mu$ whenever the integral is well-defined. 
	For any $f \in \mathcal B_b(E,\mathbb R_+)$, there is a unique locally bounded non-negative map $(t,x)\mapsto V_tf(x)$ on $\mathbb R_+\times E$ such that 
\begin{equation} \label{eq:M.1}
	V_tf(x) + \Pi_x\Big[\int_0^{t\wedge \zeta} \psi\big(\xi_s, V_{t-s} f(\xi_s)\big) {\mathrm d}s\Big] = \Pi_x[f(\xi_t) \mathbf 1_{t< \zeta}], \quad t\geq 0, x\in E. 
\end{equation}
	Here, the local boundedness of the map $(t,x) \mapsto V_tf(x)$ means that  for any $T>0$, 
\[ 
	\sup_{0\leq t\leq T, x\in E} V_tf(x) < \infty.
\] 
	There exists an $\mathcal M_f(E)$-valued Hunt process $X =\{(X_t)_{t\geq 0}; (\mathbb P_\mu)_{\mu \in \mathcal M_f(E)}\}$ such that 
\begin{equation}
	\mathbb P_\mu[e^{- X_t(f)}]
	= e^{- \mu(V_tf)},
	\quad \mu\in \mathcal M_f(E), t\geq 0, f \in \mathcal B_b(E,\mathbb R_+). 
\end{equation}
	This process $X$ is known as a $(\xi, \psi)$-superprocess. 
	See \cite[Section 2.3 and Theorem 5.11]{Li2011Measure-valued} for more details.

	Let us now give some basic assumptions on our superprocess. 
	Since we are only concerned with distributional properties of the superprocess $X$, we will assume, without loss of generality, that $X$ is \emph{canonical},  i.e., 
	%{(1) $(X_t)_{t\geq 0}$ is the coordinate process of $\mathbb D$, the Skorokhod space of measure-valued c\`ad\`ag paths on $\mathbb R_+$; and 
	(1) $(X_t)_{t\geq 0}$ is the coordinate process of $\mathbb D$, the Skorokhod space of $M_f(E)$-valued c\`ad\`ag paths on $\mathbb R_+$; and %}
	(2) $\mathbb P_\mu(\mathrm dw)$ is a probability transition kernel from $\mathcal M_f(E)$ to $\mathbb D$. 
	We will use $(\mathscr F_t)_{t\geq 0}$ to denote the natural filtration on $\mathbb D$. 
	%{We will always assume that $m$ is a $\sigma$-finite Borel measure on $E$ with full support and that the spatial motion $\xi$ ha a transition density $p(t, x, y)$ with respect to $m$.  %}
	Define a Feynman-Kac semigroup of $\xi$ by
\begin{align}
	P_t^\beta f(x)
	:= \Pi_x[e^{\int_0^t \beta(\xi_r) {\mathrm d}r }f(\xi_t) \mathbf 1_{\{t < \zeta\}}], 
	\quad f\in \mathcal B_b(E,\mathbb R), t\geq 0, x\in E. 
\end{align}
	%{It is well known that  $(P_t^\beta)_{t\geq 0}$ has a transition density $p^\beta(t, x, y)$ with respect to $m$.%}
	It is known (see \cite[Proposition 2.27]{Li2011Measure-valued}) that 
\begin{equation} \label{eq:M.2}
	\mathbb P_\mu[X_t(f)] = \mu (P_t^\beta f),
	\quad t\geq 0, f \in \mathcal B_b(E,\mathbb R). 
\end{equation}
	Thus $(P_t^\beta)_{t\geq 0}$ is called the \emph{mean semigroup}  of $X$.
	For this mean semigroup, we will always assume that
\begin{equation}\label{asp:H1} \tag{H1}
\begin{minipage}{0.9\textwidth}
	there exist a constant $\lambda < 0$, a function $\phi \in \mathcal B_b(E,(0,\infty))$ and a probability measure $\nu$ with full support on $E$ such that for each $t\geq 0$, $P_t^\beta \phi = e^{\lambda t} \phi$, $\nu P_t^\beta = e^{\lambda t} \nu$ and $\nu(\phi) =1$. 
\end{minipage}
\end{equation}
	Denote by $L_1^+(\nu)$ the collection of non-negative Borel functions on $E$ which are integrable with respect to the measure $\nu$.
	Denote by $\mathbf 0$ the null measure on $E$.
	%{Write $\mathcal M_f^o(E) = \mathcal M_f(E)\setminus \{\mathbf 0\}$.%}
	We further assume that the following two conditions hold:
\begin{equation} \label{asp:H2} \tag{H2}
\begin{minipage}{0.9\textwidth}
	for all $t>0$, $x\in E$, and $f\in L_1^+(\nu)$, it holds that $P_t^\beta f(x) = e^{\lambda t} \phi(x) \nu(f) (1+ C^{\eqref{asp:H2}}_{t,x,f})$ for some $C^{\eqref{asp:H2}}_{t,x,f}\in \mathbb R$ with
\[
	\sup_{x\in E, f\in L_1^+(\nu)} |C^{\eqref{asp:H2}}_{t,x,f}| 
	< \infty
	\text{ and }
	\lim_{t\to \infty} \sup_{x\in E, f\in L_1^+(\nu)} |C^{\eqref{asp:H2}}_{t,x,f}| 
	= 0;
\]
\end{minipage}
\end{equation}
	and 
\begin{equation} \label{asp:H3} \tag{H3}
	\mathbb P_\nu(X_t = \mathbf 0)>0, \quad t> 0.
\end{equation}

	Our first result is about the Q-process of our superprocess $X$.
	Under Assumption \eqref{asp:H1}, one can verify that, for any $\mu \in \mathcal M_f(E)$, $(e^{-\lambda t}  X_t(\phi))_{t\geq 0}$ is a non-negative martingale under $\mathbb P_\mu$.
	In fact, for each $0\leq s\leq t< \infty$,
\begin{equation} \label{eq:M.25}
	\mathbb P_\mu[e^{-\lambda t}X_t(\phi)|\mathscr F_s]
	\overset{\text{Markov}} = e^{-\lambda t} \mathbb P_{X_s}[X_{t-s}(\phi)]
	\overset{\eqref{eq:M.2}}= e^{-\lambda t}X_s(P_{t-s}^\beta \phi)
	\overset{\eqref{asp:H1}}=e^{-\lambda s}X_s(\phi).
\end{equation}
	According to \cite[Lemma 18.18]{Kallenberg2002Foundations}, for each $\mu \in \mathcal M_f^o(E)$, there exists a unique probability measure $\widetilde {\mathbb P}_\mu$  on $\mathbb D$ such that
\begin{equation} \label{eq:M.3}
	\frac{{\mathrm d}\widetilde{\mathbb P}_\mu|_{\mathscr F_t}}{{\mathrm d}\mathbb P_\mu|_{\mathscr F_t}}
	=\frac{e^{-\lambda t}X_t(\phi)}{\mu(\phi) },
	\quad t\geq 0.
\end{equation}
	This kind of martingale measure transformation for branching processes and measure-valued branching processes have been widely studied.
	We refer to the earlier papers \cite{EnglanderKyprianou2004Local,Evans1993Two,RoellyRouault1989Processus,Penisson2010Conditional} and the references therein. 
	For recent developments, we referee to \cite{ChampagnatRoelly2008Limit,RenSongSun2020Spine,RenSongZhang2018Williams}.
	It is known that the process $\{(X_t)_{t\geq 0}; \widetilde{\mathbb P}_{\mu}\}$ can be characterized by the so called spine decomposition theorem.
	%{We will recall this decomposition  in Section \ref{spine-decom}.%}
\begin{thm} \label{thm:Q}
	Suppose that \eqref{asp:H1}, \eqref{asp:H2} and \eqref{asp:H3} hold, then
\[
	\lim_{s\rightarrow\infty}\mathbb P_\mu(A |X_s\neq \mathbf 0)=\widetilde{\mathbb P}_\mu(A), 
	\quad \mu \in \mathcal M_f^o(E), A\in \bigcup_{t\geq 0}\mathscr F_t.
\]
\end{thm}

	Our second result is about the asymptotic behavior of the extinction probability. 
	Define 
\begin{equation} 
	l(x)
	%{:= \int_1^\infty r\ln r\pi^\phi(x, {\mathrm d}r),\quad x \in E,
	:= \int_{(1,\infty)} r\ln r\pi^\phi(x, {\mathrm d}r),\quad x \in E,%}
\end{equation}
	where $\pi^\phi(x, \mathrm dr)$ is the kernel from $E$ to $(0,\infty)$ given by
\begin{equation} 
	%{
	%\int_0^\infty f(r)\pi^\phi(x,{\mathrm d}r)
	%=\int_0^\infty f\big(r\phi(x)\big)\pi(x, {\mathrm d}r),
	%\quad x\in E, f\in\mathcal B_b(\mathbb R_+,\mathbb R_+).
	\int_{(0,\infty)} f(r)\pi^\phi(x,{\mathrm d}r)
	=\int_{(0,\infty)} f\big(r\phi(x)\big)\pi(x, {\mathrm d}r),
	\quad x\in E, f\in\mathcal B_b((0,\infty),\mathbb R_+).
	%}
\end{equation}

\begin{thm} \label{thm:E}
	Suppose that \eqref{asp:H1}, \eqref{asp:H2} and \eqref{asp:H3} hold.	
	Then there exists $k\in [0,\infty)$ such that for any $\mu \in \mathcal M_f^o(E)$,
\begin{equation}
	\lim_{t\rightarrow\infty} e^{-\lambda t}\mathbb P_\mu(X_t \neq \mathbf 0)
	=k\mu(\phi). 
\end{equation}
	Moreover, $k>0$ if and only if $\nu(l)<\infty$.
\end{thm}

	Our third result is about the moment property of the Yaglom limit and the quasi-stationary distribution of $X$.
	For any probability measure $\mathbf P$ on $\mathcal M_f(E)$, define 
\[
	(\mathbf P\mathbb P)(\cdot) 
	:= \int_{\mathcal M_f(E)} \mathbb P_\mu(\cdot)\mathbf P({\mathrm d}\mu).
\] 
%{Let $\tau_0:=\inf\{t>0: X_t=\mathbf 0\}$ be the extinction time of $X$.%}
%{
	Write $\mathcal M_f^o(E) := \mathcal M_f(E)\setminus \{ \mathbf 0\}$. 
	Any probability measure $\mathbf P$ on $\mathcal M_f^o(E)$ will also be understood as its unique extension on $\mathcal M_f(E)$ with  $\mathbf P(\{\mathbf 0\}) = 0$. 
%}
	%{We say a probability measure $\mathbf Q$ on $\mathcal M_f(E)$ is a \emph{quasi-stationary distribution} (QSD) of $X$, if
	We say a probability measure $\mathbf Q$ on $\mathcal M^o_f(E)$ is a \emph{quasi-stationary distribution} (QSD) of $X$, if%}
\[
	%{(\mathbf Q \mathbb P) \left( X_t \in B \middle | \tau_0>t \right) 
	(\mathbf Q \mathbb P) \left( X_t \in B \middle | X_t \neq \mathbf 0 \right) 
	= \mathbf Q(B), 
	\quad t\geq 0, B \in \mathcal B(\mathcal M_f(E)).
\]
	According to \cite[(1.5)]{LiuRenSongSun2020}, if a probability measure $\mathbf Q$ on $\mathcal M_f^o(E)$ is a QSD of $X$, then there exists an $r\in (-\infty, 0)$ such that $(\mathbf Q\mathbb P)(X_t \neq \mathbf 0) = e^{rt}$ for all $t\geq 0$; and in this case, we call $r$ the \emph{mass decay rate} of $\mathbf Q$.
	It is proved in \cite[Theorem 1.2]{LiuRenSongSun2020} that, under Assumption \eqref{asp:H1}, \eqref{asp:H2} and \eqref{asp:H3}, 
	(1) for each $r\in [\lambda, 0)$, there exists a unique QSD $\mathbf Q_r$ of $X$ with mass decay rate $r$; 
	and (2) for each $r\in (-\infty, \lambda)$, there is no QSD for $X$ with mass decay rate $r$.
	In particular, by \cite[Theorem 1.1, Proposition 1.7]{LiuRenSongSun2020}, $\mathbf Q_\lambda$ is the \emph{Yaglom limit} of $X$, in the sense that
\begin{equation} 
	\mathbb P_\mu(X_t \in \cdot | X_t \neq \mathbf 0) 
	%{\xrightarrow[t\to \infty]{weakly} \mathbf Q_\lambda,
	\xrightarrow[t\to \infty]{\text{weakly}} \mathbf Q_\lambda,%}
	\quad \mu \in \mathcal M_f^o(E).   
\end{equation}

\begin{thm}\label{thm:L}
	Suppose that Assumptions \eqref{asp:H1}, \eqref{asp:H2} and \eqref{asp:H3} hold.
	Then (1) 
	for any $r\in [\lambda, 0)$ and $\gamma \in (0, \frac{r}{\lambda})$, it holds that $\int_{{\mathcal M}_f(E)}\mu(\phi)^\gamma\mathbf Q_r({\mathrm d}\mu)<\infty$; 
	(2) $\int_{{\mathcal M}_f(E)}\mu(\phi)\mathbf Q_\lambda({\mathrm d}\mu) = k^{-1}$ where 
	$k$ is  the constant in Theorem \ref{thm:E}. 
	\footnote{ZS: Maybe it also make sense to investigate the $r/\lambda$-moment of $\mathbf Q_r$ for other $r$.} 
\end{thm}

	Our fourth result characterizes the invariant distribution for the Q-process of $X$. %}
	For any probability measure $\mathbf P$ on $\mathcal M_f(E)$, define 
\[
	(\mathbf P\widetilde{\mathbb P})[\cdot] 
	:= \int_{\mathcal M_f(E)}\widetilde{\mathbb P}_\mu[\cdot] \mathbf P(\mathrm d\mu).
\]
	A probability $\mathbf Q$ on $\mathcal M_f(E)$ is called an \emph{invariant distribution} of the Q-process of $X$ if 
\[
	(\mathbf Q\widetilde{\mathbb P})(X_t \in \cdot ) 
	=\mathbf Q(\cdot),	\quad t\geq 0.
\]

\begin{thm}\label{thm:I}
	Suppose that \eqref{asp:H1}, \eqref{asp:H2} and \eqref{asp:H3} hold. 
\begin{enumerate}
\item
	If $\nu(l)<\infty$, then the Q-process of $X$ has an invariant distribution $\mathbf Q$ given by
\[
	\int_{{\mathcal M}_f(E)} e^{-\mu(f)}\mathbf Q(\mathrm d\mu)
	=\frac{\int_{{\mathcal M}_f(E)}\mu(\phi)e^{-\mu(f)}\mathbf Q_\lambda({\mathrm d}\mu)} {\int_{{\mathcal M}_f(E)}\mu(\phi)\mathbf Q_\lambda({\mathrm d}\mu)}, \quad f\in \mathcal B_b(E, \mathbb R_+).
\]	
Moreover, for each $\mu\in\mathcal M^o_f(E)$, we have
\[
	\widetilde{\mathbb P}_\mu(X_t \in \cdot ) \xrightarrow[t\to \infty]{weakly} {\mathbf Q}(\cdot).
\]
\item
	If $\nu(l) = \infty$, then for each $\mu \in \mathcal M^o_f(E)$, we have 
\begin{align} 
& \mathbb P_\mu\big(X_t(\phi) > C\big) \xrightarrow[t\to \infty]{} 1, \quad C\geq 0.
\end{align}
\end{enumerate}
\end{thm}

\section{Q-process} \label{sec:Q}
	%{
	Through out this section, we assume \eqref{asp:H1}, \eqref{asp:H2} and \eqref{asp:H3} hold. 
	Let us first recall some known results from \cite{LiuRenSongSun2020}.
	%}
%{
	It is easy to see that the operators $(V_t)_{t\geq 0}$
	given by \eqref{eq:M.1} can be extended uniquely to a family of operators $(\overline V_t)_{t\geq 0}$ on $\mathcal B(E,[0,\infty])$ such that for all $t\geq 0$, $f_n \uparrow f$ pointwisely in  $\mathcal B(E, [0,\infty])$ implies that $\overline V_tf_n \uparrow \overline V_tf$ pointwisely.
	With some abuse of notation, we still write $V_t = \overline V_t$ for $t\geq 0$, and call $(V_t)_{t\geq 0}$ \emph{the extended cumulant semigroup} of the superprocess $X$.
	Define $v_t = V_t(\infty  \mathbf 1_E)$ for $t\geq 0$, then it holds that
	\begin{equation} \label{eq:Q.04}
	\mathbb P_\mu (X_t = \mathbf 0)
	= e^{- \mu (v_t)},
	\quad \mu \in \mathcal M_f(E), t\geq 0.
	\end{equation}
	According to \cite[(1.10)]{LiuRenSongSun2020}) we have
\begin{equation}\label{eq:Q.05}
	\mu(v_t) > 0, \quad \mu \in \mathcal M_f^o(E), t \geq 0.
\end{equation}
	According to \cite[Proposition 1.3]{LiuRenSongSun2020}, for each $\mu\in \mathcal M_f(E)$ we have
\begin{equation} \label{eq:Q.055}
	\mu(v_t) \xrightarrow[t\to \infty]{} 0.
\end{equation} 
	According to \cite[(1.17)]{LiuRenSongSun2020}, for each $t>0$ and $\mu \in \mathcal M_f^o(E)$ we have 
\begin{equation} \label{eq:Q.056}
	\text{$\mu(v_t) = \nu(v_t)\mu(\phi) (1+C_{\mu,t}^{\eqref{eq:Q.056}})$ for some $C_{\mu,t}^{\eqref{eq:Q.056}}\in \mathbb R$ with $\lim_{t\to \infty}|C_{\mu,t}^{\eqref{eq:Q.056}}| = 0$.}
\end{equation}
	According to \cite[(2.14)]{LiuRenSongSun2020}, for each $t>0$ and $x\in E$ we have
\begin{equation}  \label{eq:Q.0565}
	\text{$v_t(x) = \phi(x) \nu(v_t) C_{t,x}^{\eqref{eq:Q.0565}}$ for some $C_{t,x}^{\eqref{eq:Q.0565}}\geq 0$ with $\varlimsup_{t\to \infty}\sup_{x\in E}C_{t,x}^{\eqref{eq:Q.0565}}<\infty$.}
\end{equation}
	According to \cite[(2.20)]{LiuRenSongSun2020}, for each $t\geq 0$, we have
\begin{equation}\label{eq:Q.057}
	\lim_{s\to \infty} \frac{\nu(v_s)}{\nu(v_{s-t})} = e^{\lambda t}.
\end{equation}
%}
%MOVED FROM BELOW
\begin{proof}[Proof of Theorem \ref{thm:Q}]
	%{Assume that $t>0$ and $A\in\mathscr F_t$ are fixed.%}
	Fix arbitrary $\mu\in \mathcal M^o_f(E)$, $t>0$ and $A\in\mathscr F_t$. 
	For $s>t$, 
	%{by the Markov property of $X$,%}
\begin{equation} \label{eq:Q.06}
	%{\mathbb P_\mu(A|\zeta>s)=\dfrac{\mathbb P_\mu(A, \zeta>s)}{\mathbb P_\mu(\zeta>s)}=\dfrac{\mathbb P_\mu\big(\mathbb P_{X_t}(\zeta>s-t);A\big)}{\mathbb P_\mu(\zeta>s)},
	\mathbb P_\mu(A|X_s\neq \mathbf 0)
	=\frac{\mathbb P_\mu(A, X_s \neq \mathbf 0)}{\mathbb P_\mu(X_s\neq \mathbf 0)}
	\overset{\text{Markov}}=\frac{\mathbb P_\mu\big(\mathbb P_{X_t}(X_{s-t} \neq \mathbf 0);A\big)}{\mathbb P_\mu(X_s\neq \mathbf 0)}.%}
\end{equation}
	%{
	We claim that 
\begin{equation} \label{eq:Q.1}
	\frac{\mathbb P_{X_t}(X_{s-t} \neq \mathbf 0)}{\mathbb P_\mu(X_{s} \neq \mathbf 0)}
	\xrightarrow[s\to \infty]{} \frac{e^{-\lambda t}X_t(\phi)}{\mu(\phi) }, 
	\quad \mathbb P_\mu\text{-a.s.}
\end{equation}
	and that there exist (deterministic) $c,s_0>0$ such that for any $s\geq s_0$,
\begin{equation} \label{eq:Q.2}
	\frac{\mathbb P_{X_t}(X_{s-t} \neq \mathbf 0)}{\mathbb P_\mu(X_{s} \neq \mathbf 0)} 
	\leq cX_t(\phi), \quad \mathbb P_\mu\text{-a.s.}
\end{equation}
	Now using \eqref{eq:Q.1}, \eqref{eq:Q.2} and dominated convergence theorem (DCT) we have
\begin{align} 
&  \mathbb P_\mu(A|X_s\neq \mathbf 0) 
	\overset{\eqref{eq:Q.06}}=  \frac{\mathbb P_\mu\big(\mathbb P_{X_t}(X_{s-t} \neq \mathbf 0);A\big)}{\mathbb P_\mu(X_s\neq \mathbf 0)}
	\\&\xrightarrow[s\to \infty]{\text{DCT}} \mathbb P_\mu\Big[\frac{e^{-\lambda t}X_t(\phi)}{\mu(\phi) }; A\Big]
	\overset{\eqref{eq:M.3}} = \widetilde{\mathbb P}_\mu[A]. 
\end{align}
	We still need to proof claims \eqref{eq:Q.1} and \eqref{eq:Q.2}.
	%}
%{
	%It follows from  Lemma \ref{lem:ratio limit} that as 
	%$s\to\infty$,
	%\[
	%\mathbb P_x(\zeta>s)\sim v(s,x)\sim \phi(x)a(s)e^{\lambda s},
	%\]
	%where $a(s)$ satisfies that for any $s>0$, $\lim_{r\rightarrow\infty}a(r+s)/a(r)=1$.
	%Thus we have
	%\begin{eqnarray*}
	%\lim_{s\rightarrow\infty}\dfrac{\mathbb P_{X_t}(\zeta>s-t)}{\mathbb P_\mu(\zeta>s)}
	%&=&\lim_{s\rightarrow\infty}\dfrac{1-e^{-\langle v(s-t,\cdot), X_t\rangle }}{1-e^{-\langle v(s,\cdot), \mu\rangle }}
	%=\lim_{s\rightarrow\infty}\dfrac{\langle v(s-t,\cdot), X_t\rangle }{\langle v(s,\cdot),\mu\rangle }\\
	%&=&\dfrac{e^{-\lambda t}\langle \phi, X_t\rangle }{\langle \phi, \mu\rangle }=\dfrac{M_t(\phi)}{\langle \phi, \mu\rangle }.
	%\end{eqnarray*}
\begin{proof*}[Proof of \eqref{eq:Q.1}]
	Firstly notice that for any $\widetilde \mu\in \mathcal M_f(E)$,
\begin{align}  
	&\lim_{s\rightarrow\infty}\dfrac{ \widetilde \mu(v_{s-t}) }{ \mu(v_s) } 
	\overset{\eqref{eq:Q.056}}=\lim_{s\to \infty} \frac{\nu(v_{s-t})\widetilde \mu (\phi) (1+C_{\widetilde \mu,s-t}^{\eqref{eq:Q.056}})}{\nu(v_s)\mu(\phi)(1+C_{\mu,s}^{\eqref{eq:Q.056}})}
	\overset{\eqref{eq:Q.056}}= \frac{\widetilde \mu(\phi)}{\mu(\phi)}\lim_{s\to \infty} \frac{\nu(v_{s-t}) }{\nu(v_s)}
	\\\label{eq:Q.25}&\overset{\eqref{eq:Q.057}}= \frac{e^{-\lambda t} \widetilde \mu(\phi)}{\mu(\phi)}. 
\end{align}
	Now we can verify that, $\mathbb P_\mu$-almost surely,
\begin{equation}
	\lim_{s\to\infty}\frac{\mathbb P_{X_t}(X_{s-t}\neq \mathbf 0)}{\mathbb P_\mu(X_s\neq \mathbf 0)}
	\overset{\eqref{eq:Q.04}}=\lim_{s\to\infty}\frac{1-e^{- X_t(v_{s-t}) }}{1-e^{- \mu(v_s) }}
	\overset{\eqref{eq:Q.05},\eqref{eq:Q.055}}=\lim_{s\rightarrow\infty}\dfrac{ X_t(v_{s-t}) }{ \mu(v_s) }
	\overset{\eqref{eq:Q.25}}= \frac{e^{-\lambda t} X_t(\phi)}{\mu(\phi)}.
	\qedhere
\end{equation}
\end{proof*}
%}
%{
	%We claim that for large $s$, $\dfrac{\mathbb P_{X_t}(\zeta>s-t)}{\mathbb P_\mu(\zeta>s)}$ can be dominated by a random variable which is integrable with respect to $\mathbb P_\mu$, then  by the dominated convergence theorem,
	%\[
	%\lim_{s\rightarrow\infty}\mathbb P_\mu(A|\zeta>s)=\mathbb P_\mu\left(\frac{M_t(\phi)}{\langle\phi,\mu\rangle };A\right)=\widetilde{\mathbb P}_\mu(A).
	%\]
	%Now we prove our claim above. From \eqref{upp} we see that there is a constant $\widetilde C>0$ such that for any $s>2T$ and $x\in E$,
	%\[
	%v(s,x)\leq \widetilde C\phi(x)e^{\lambda T}\langle v(s-T,\cdot),\nu\rangle .
	%\]
	%Using the fact that $\lim_{x\rightarrow 0+}(1-e^{-x})/x=1$, we can choose $s$ sufficiently large such that
	%\[
	%1-e^{-\langle v(s,\cdot),\mu\rangle }>\frac{1}{2}\langle v(s,\cdot),\mu\rangle .
	%\]
	%Since $1-e^{-x}\leq x$ for $x>0$, for  $s-t>2T$ we have
	%\[
	%\dfrac{1-e^{-\langle v(s-t,\cdot),X_t\rangle }}{1-e^{-\langle v(s,\cdot),\mu\rangle }}
	%\leq \dfrac{2\langle v(s-t,\cdot),X_t\rangle }{\langle v(s,\cdot),\mu\rangle }\leq \dfrac{2\widetilde C\langle \phi,X_t\rangle e^{\lambda T}\langle v(s-t-T,\cdot),\nu\rangle }{\langle v(s,\cdot),\mu\rangle }.
	%\]
	%By \eqref{one point ratio limit}, we have
	%\[
	%\lim_{s\rightarrow\infty}\dfrac{\langle v(s-t-T,\cdot),\nu\rangle }{\langle v(s,\cdot),\mu\rangle }
	%=e^{-\lambda(t+T)}\langle \phi,\mu\rangle ^{-1}.
	%\]
	%Noticing that $\langle \phi,X_t\rangle $ is integrable with respect to $\mathbb P_\mu$, our claim is true.
\begin{proof*}[Proof of \eqref{eq:Q.2}]
	Firstly notice that, $\mathbb P_\mu$-almost surely,
	\begin{align}
	& \frac{\mathbb P_{X_t}(X_{s-t}\neq \mathbf 0)}{\mathbb P_\mu(X_s\neq \mathbf 0)}
	\overset{\eqref{eq:Q.04}}= \frac{1-e^{- X_t(v_{s-t}) }}{1-e^{- \mu(v_s) }}
	\leq \frac{X_t(v_{s-t}) }{1-e^{- \mu(v_s) }}
	\\\label{eq:Q.3}&\overset{\eqref{eq:Q.05},\eqref{eq:Q.055}}= \frac{X_t(v_{s-t}) }{ \mu(v_s)} C_s^{\eqref{eq:Q.3}}, 
	\\& \quad \text{for some deterministic $C_s^{\eqref{eq:Q.3}}>0$ with $\lim_{s\to \infty} C_s^{\eqref{eq:Q.3}} = 1$,}
	\\\label{eq:Q.4}&\overset{\eqref{eq:Q.0565}}\leq \frac{X_t(\phi) \nu(v_{s-t}) \sup_{x\in E} C_{s-t,x}^{\eqref{eq:Q.0565}}}{ \mu(v_s)} C_s^{\eqref{eq:Q.3}}.
	\end{align}
	Then notice that
\begin{align} 
&  \varlimsup_{s\to \infty}\frac{\nu(v_{s-t}) \sup_{x\in E} C_{s-t,x}^{\eqref{eq:Q.0565}}}{ \mu(v_s)} C_s^{\eqref{eq:Q.3}}
	\overset{\eqref{eq:Q.25}} = \frac{e^{-\lambda t} \nu(\phi)}{\mu(\phi)} \varlimsup_{s\to \infty}C_s^{\eqref{eq:Q.3}} \sup_{x\in E}C_{s-t,x}^{\eqref{eq:Q.0565}} 
	\\\label{eq:Q.5}&\overset{\eqref{eq:Q.3}} = \frac{e^{-\lambda t} \nu(\phi)}{\mu(\phi)} \varlimsup_{s\to \infty} \sup_{x\in E}C_{s-t,x}^{\eqref{eq:Q.0565}} 
	\overset{\eqref{eq:Q.0565}}< \infty. 
\end{align}
	The desired result follows from \eqref{eq:Q.4} and \eqref{eq:Q.5}. 
\end{proof*}
\end{proof}
%END

%{
	\section{Extinction probability} \label{sec:E}
	In this section, we assume that \eqref{asp:H1}, \eqref{asp:H2} and \eqref{asp:H3} hold. 
	In order to proof Theorem \ref{thm:E}, we will need a special case for the spine decomposition theorem for superprocess $X$.
	The spine decomposition theorem in this specific case will be presented in Lemma \ref{thm:E.2} below.
	
	Let us first recall the \emph{Kuznetsov  measure} of the superprocess $X$. 
	According to \cite[Proposition 1.3]{LiuRenSongSun2020} we have 
\begin{equation} \label{eq:E.1}
	v_t(x)<\infty, \quad t>0, x\in E.
\end{equation}
	Therefore 
\begin{align} \label{eq:E.11}
	\mathbb P_{\delta_x}(X_t = \mathbf 0) \overset{\eqref{eq:Q.04}}= e^{-v_t(x)}\overset{\eqref{eq:E.1}} > 0, \quad t>0, x\in E.
\end{align}
	According to \cite[Section 8.4]{Li2011Measure-valued} and \eqref{eq:E.11}, there is a unique family of $\sigma$-finite measures $(\mathbb N_x)_{x\in E}$ on $\mathbb D$ such that 
	(1) $\mathbb N_x(w_0 \neq \mathbf 0) = 0$ for each $x\in E$;
	(2) $\mathbb N_x (\forall t > 0, w_t =\mathbf 0) =0$ for each $x\in E$; 
	and (3) for each $\mu \in \mathcal M_f(E)$, if $\mathcal N$ is a Poisson random measure on $\mathbb D$ with intensity
	\[
	(\mu\mathbb N)(\mathrm dw):= \int_{x\in E} \mathbb N_x(\mathrm dw)\mu(\mathrm dx), \quad w\in \mathbb D.
	\]
	then
	\begin{equation} \label{eq:E.12}
	\{(X_t)_{t> 0};\mathbb P_\mu\}
	\overset{\text{f.d.d.}}= \Big(\int_{\mathbb D} w_t\mathcal N(\mathrm dw)\Big)_{t> 0}.
	\end{equation}
	This family of measure $(\mathbb N_x)_{x\in E}$ is known as the Kuznetsov measures of $X$.
	It can be verified from Campbell's formula for Poisson random measure that for any $\mu\in \mathcal M_f(E)$ and $t>0$,
\begin{equation} \label{eq:E.13}
	\mu(P_t^\beta f) 
	\overset{\eqref{eq:M.2}}= \mathbb P_\mu[X_t(f)]
	\overset{\eqref{eq:E.12}}=(\mu \mathbb N) [w_t(f)], 
	\quad f\in \mathcal B_b(E, \mathbb R_+).
\end{equation}
	and 
\begin{equation} \label{eq:E.14}
	(\mu\mathbb N) (w_t\neq \mathbf 0) 
	\overset{\eqref{eq:E.12}} = - \log \mathbb P_{\mu}(X_t = \mathbf 0). 
\end{equation}

\begin{lem} \label{thm:E.1}
	For any probability measure $\mu$ on $E$, there exists a probability measure $\widetilde {\mu\mathbb N}$ on $\mathbb D$ satisfying that
	\begin{align} 
	& \frac{\widetilde {\mu \mathbb N}(\mathrm dw)|_{\mathscr F_t} }{\mu\mathbb N(\mathrm dw)|_{\mathscr F_t}}  = \frac{e^{-\lambda t}w_t(\phi)}{\mu(\phi)}, \quad w\in \mathbb D, t\geq 0.
	\end{align}
\end{lem}
\begin{proof}
	TBD
\end{proof}
	
	Recall that $\{(\xi_t)_{t\geq 0}; (\Pi_x)_{x\in E}\}$ is the spatial motion of the superprocess. 
	Let  $(\mathscr F_t^{\xi})_{t\geq 0}$ be the natural filtration of process $(\xi_t)_{t\geq 0}$.
	For any probability measure $\mu$ on $E$, define
\begin{align} 
& (\mu \Pi)[\cdot] := \int_{x\in E} \Pi_x[\cdot]\mu(\mathrm dx),
\end{align}
	and let the probability $\widetilde {\mu \Pi}$ be the Doob's $h$-transform of $\mu\Pi$ given by
\begin{align} 
	&  \frac{\mathrm d (\widetilde{\mu \Pi})|_{\mathscr F^\xi_t}}{\mathrm d (\mu \Pi)|_{\mathscr F^\xi_t}} 
	=\frac{e^{\int_0^t \beta(\xi_s)ds}\phi(\xi_t) \mathbf 1_{\{t<\zeta\}}}{e^{\lambda t}\mu(\phi)}, 
	\quad t\geq 0.
\end{align}

\begin{lem} \label{thm:E.15}
	$\{(\xi_t)_{t\geq 0}; \widetilde{\nu \Pi}\}$ is a stationary process with $\xi_0 \overset{d}\sim \widetilde \nu$ where $\widetilde \nu$ is a probability measure on $E$ given by \[\widetilde \nu(\mathrm dx) := \phi(x) \nu(\mathrm dx), \quad x\in E.\]
\end{lem}
\begin{proof}
	TBD
\end{proof}

\begin{lem}  \label{thm:E.16}
	There exist random elements $\big\{X, \xi, N, (s_k, y_k,w^{(k)})_{k\in \mathbb N}; \mathbb Q\big\}$ so that:
\begin{enumerate}
\item
	$X$ is an $\mathbb D$-valued random element with $\{X; \mathbb Q\} \overset{\text{d}} \sim \mathrm P_\nu$;
\item
	$\xi$ is a stationary c\`adl\`ag stochastic process with $\{(\xi_t)_{t \geq 0}; \mathbb Q(\cdot | X)\} \overset{\text{d}} \sim \widetilde{\nu \Pi}$;
\item
	$\big\{N; \mathbb Q\big(\cdot \big|X,\xi\big)\big\}$ is a Poisson random measure on $\mathbb R\times \mathbb D$ with intensity 
\[
	2 \sigma(\xi_s)^2 {\mathrm d}s \cdot \mathbb N_{\xi_s}({\mathrm d}w), 
	\quad (s,w)\in \mathbb R\times \mathbb D;
\]
\item
	$(s_k, y_k)_{k\in \mathbb N}$ is a list of $\mathbb R \times \mathbb R_+$-random elements satisfying that $\{M; \mathbb Q(\cdot | X, \xi, N)\}$ is a Poisson random measure on $\mathbb R \times \mathbb R_+$ with intensity
\[
	\mathrm ds \cdot y \pi(\xi_s, \mathrm dy), \quad (s,y)\in \mathbb R \times \mathbb R_+
\]
	where
\[
	M(\mathrm ds,\mathrm dy) 
	:= \sum_{k\in \mathbb N} \delta_{(s_k, y_k)}(\mathrm ds,\mathrm dy), \quad (s,y)\in \mathbb R \times \mathbb R_+.
\]
\item 
	$\{(w^{(k)})_{k\in \mathbb N}; \mathbb Q(\cdot|\mathscr G)\}$ is a list of independent $\mathbb D$-valued random elements satisfying that 
\[
	\{w^{(k)}; \mathbb Q(\cdot| \mathscr G)\} 
	\overset{\text{d}}\sim \mathbb P_{y_k\delta_{\xi_{(s_k)}}}, 
	\quad k\in \mathbb N,
\]
	where
\[
	\mathscr G 
	:= \sigma(X, \xi, N, (s_k, y_k)_{k\in \mathbb N}). 
\]
\end{enumerate}
\end{lem}
\begin{proof}
	TBD
\end{proof}
	For the rest of this section, let $\big\{X, \xi, N, (s_k, y_k,w^{(k)})_{k\in \mathbb N}; \mathbb Q\big\}$ be given by Lemma \ref{thm:E.16}.
\begin{lem} \label{thm:E.19}
	Random measure
\[
	\Big\{\sum_{k\in \mathbb N} \delta_{s_k, w^{(k)}}(\mathrm ds,\mathrm dw); \mathbb Q(\cdot | X, \xi, N)\Big\}
\]
	is a Poisson random measure on $\mathbb R \times \mathbb D$ with intensity
\[
	\mathrm ds \cdot \int_{y\in (0,\infty)} y \pi(\xi_s, \mathrm dy)\cdot \mathbb P_{y\delta_{\xi_s}}(\mathrm dw).
\]
\end{lem}
\begin{proof}
	TBD
\end{proof}
\begin{lem}\label{thm:E.2}
	It holds that 
\[
	\{(X_t + Z_t^{(0,t]})_{t\geq 0}; \mathbb Q\} \overset{\text{d}}\sim \widetilde{\mathbb P_\nu}
\]
	and
\[
	\{(Z_t^{(0,t]})_{t\geq 0}; \mathbb Q\} \overset{\text{d}}\sim \widetilde{\nu\mathbb N}
\]
	where, for each $-\infty \leq a < b \leq t<\infty$,
\begin{equation} \label{eq:E.4}
	Z_t^{(a,b]}:= \int_{(a,b]\times \mathbb D} w_{t-s} N(\mathrm ds,\mathrm dw) + \sum_{k\in \mathbb N} w^{(k)}_{t-s_k} \mathbf 1_{s_k \in (a,b]}.
\end{equation}
\end{lem}

\begin{lem}\label{thm:E.3}
	It holds that, for each $-\infty < a < b \leq t<\infty$ and $s\in \mathbb R$,
	\[
	\{Z_t^{(a,b]}; \mathbb Q\} \overset{\text{d}}= \{Z_{t+s}^{(a+s,b+s]}; \mathbb Q\}.
	\]
\end{lem}
\begin{proof}
	TBD
\end{proof}
%}
%{
\begin{lem}\label{thm:E.4}	
	(1) If $\nu(l)<\infty$,
	then for any $\epsilon>0$,
	\[
	\sum_{k\in \mathbb N} \mathbf 1_{s_k \in (-\infty,0]} y_k e^{\epsilon s_k} \phi(\xi_{s_k}) < \infty, \quad \mathbb Q\text{-a.s.}
	\]
	(2) If  $ \nu(l)=\infty$,
	then for any $\epsilon>0$ and any $s_0>0$,
	\begin{equation}
	\int^{s_0}_{-\infty} {\mathrm d}s
	\int_{\phi(\xi_s)^{-1}e^{-\epsilon s}}^\infty y\pi(\xi_s,{\mathrm d}y)
	=\infty, 
	\quad {\mathbb Q}\text{-a.s.}
	\end{equation}
\end{lem}
\begin{proof}
	TBD. See Appendix.
\end{proof}
%}
%{
	We will also need the following Lemma.
\begin{lem} \label{thm:E.5}
	For any $\epsilon > 0$, there exists an $s_0, \varepsilon>0$ and $\delta > 0$,  such that for each $x\in E, s>s_0$ and $y\geq \phi(x)^{-1}e^{-\epsilon s}$, it holds that
$
	\mathbb P_{y \delta_{x}}\big(w_{s}(\phi)>\varepsilon\big) > \delta.
$
\end{lem}
\begin{proof}
	By Chebyshev's inequality, for any $\varepsilon>0$,
	\begin{align}
	&\mathbb P_{r\delta_x}\big(\langle\phi, X_{-t}\rangle >\varepsilon\big)=\mathbb P_{r\delta_x}\left(e^{-\langle\phi, X_{-t}\rangle }<e^{-\varepsilon}\right)\\
	=&1-\mathbb P_{r\delta_x}\left(e^{-\langle\phi, X_{-t}
		\rangle }\geq e^{-\varepsilon}\right)\geq 1-e^{\varepsilon }\mathbb P_{r\delta_x}e^{-\langle\phi, X_{-t}\rangle }\\
	=&1-e^{\varepsilon }e^{-rV_{-t}\phi(x)},\label{Cheby}
	\end{align}
	According to Lemma \ref{lem:rate} and \eqref{inequ:lower}, 
	when $-t$ is sufficiently large, we have
	\[
	V_{-t}\phi(x)\ge\frac{1}{2}\phi(x)\nu(V_{-t}\phi)\geq \dfrac{a}{2}\phi(x)e^{Nt},\quad x\in E.
	\]
	Therefore, using \eqref{Cheby}, we obtain that, when $-t$ is sufficiently large,
	\begin{eqnarray*}
		&&\inf_{r\geq \phi(x)^{-1}e^{-Nt}, x\in E}\mathbb P_{r\delta_x}\big(\langle\phi, X_{-t} \rangle >\varepsilon\big)\geq \inf_{r\geq \phi(x)^{-1}e^{-Nt}, x\in E} \left(1-e^{\varepsilon}e^{-rV_{-t}\phi(x)}\right)\\ 
		&&     \geq 1-e^{\varepsilon-a/2}.
	\end{eqnarray*}
	Then we get \eqref{last point} by	choosing $\varepsilon\in(0, a/2)$ and $\delta=1-e^{\varepsilon-a/2}$.  Then
	\eqref{Z-infty} follows, and thus $k=0$. The proof is finished.
\end{proof}
%}
%MOVED FROM BELOW
\begin{proof}[Proof of Theorem \ref{thm:E}] ${}$
	%{According to Lemma \ref{lem:extinc}, we only need to prove our result when the initial measure is $\nu$.
\begin{proof*}[Proof of the existence of $k\in [0,\infty)$.]
	We only need to prove the existence of $k\in [0,\infty)$ for initial configuration $\nu$.
	In fact, for each $\mu\in \mathcal M_f^o(E)$,
\begin{align}
	\lim_{t\to \infty}\frac{\mathbb P_\mu(X_t \neq \mathbf 0)}{\mathbb P_\nu(X_t \neq \mathbf 0)}
	\overset{\eqref{eq:Q.04}} = \lim_{t\to \infty}\frac{1- e^{-\mu(v_t)}}{1- e^{-\nu(v_t)}}
	\overset{\eqref{eq:Q.05}, \eqref{eq:Q.055}} = \lim_{t\to \infty}\frac{\mu(v_t)}{\nu(v_t)}
	\overset{\eqref{eq:Q.25}, \eqref{asp:H1}} = \mu(\phi). 
\end{align}
%}

	%{
	%It is easy to see that for any $t\ge 0,$
	%\begin{equation}\label{eq:E.5}
	%e^{-\lambda t}\mathbb P_\nu(\zeta>t)
	%\overset{\eqref{eq:M.3}}= \nu(\phi)\widetilde{\mathbb P}_\nu\left(\frac{1}{\langle\phi, X_{t}\rangle }\right)
	%=\mathbb Q\left[\frac{1}{\langle\phi, X_{t}\rangle +\int^0_{-t}w_{-s}(\phi)\mathbf N^\xi({\mathrm d}s, {\mathrm d}w)}\right],
	%\end{equation}
	%where $\widetilde {\mathbb P}_\nu$ is defined in \eqref{eq:M.3}.
	It can be verified that for any $t\geq 0,$
	\begin{align}\label{eq:E.5}
	&e^{-\lambda t}\mathbb P_\nu(X_t\neq \mathbf 0)
	\overset{\eqref{eq:M.3}, \eqref{asp:H1}}= \widetilde{\mathbb P_\nu}[X_t(\phi)^{-1}]
	\overset{\text{Lemma \ref{thm:E.1}}}=\mathbb Q\big[ \big(X_t(\phi) +Z^{(0,t]}_t(\phi)\big)^{-1}\big]
	\\&\overset{\text{Lemma \ref{thm:E.3}}}=\mathbb Q\big[ \big(X_t(\phi) +Z^{(-t,0]}_0(\phi)\big)^{-1}\big].
	\end{align}
	%}	
	%{Note that $\lim_{t\to\infty}\langle\phi, X_{t}\rangle=0$, $\mathbb Q$-a.s., and
	%$$
	%\int^0_{-t}w_{-s}(\phi)\mathbf N^\xi({\mathrm d}s, {\mathrm d}w)\uparrow
	%\int^0_{-\infty}w_{-s}(\phi)\mathbf N^\xi({\mathrm d}s, {\mathrm d}w)
	%\quad \mbox{as } t\uparrow\infty.
	%$$
	
	Note that  
\begin{equation}\label{eq:E.51}
	X_t(\phi) \xrightarrow[t\to \infty]{} 0, \quad \mathbb Q\text{-a.s.}
\end{equation}
	In fact, by Lemma \ref{thm:E.16} (1) and \eqref{eq:M.25}, we know that $\big\{\big(e^{-\lambda t} X_t(\phi)\big)_{t\geq 0}; \mathbb Q\big\}$ is a non-negative martingale. 
	So according to martingale convergence theorem and the fact that $\lambda < 0$, we know \eqref{eq:E.51} holds.

	We observe by monotonicity that
\begin{equation} \label{eq:E.52}
	Z^{(-t,0]}_0(\phi)
	\xrightarrow[t\to \infty]{} Z^{(-\infty,0]}_0(\phi),
	\quad \mathbb Q\text{-a.s.}
\end{equation}
	%}
%{Also note that
%\begin{equation}\label{domi-Zt1}
	%\frac{1}{\langle\phi, X_t\rangle + \int^0_{-t}w_{-s}(\phi)\mathbf N^\xi({\mathrm d}s, {\mathrm d}w)}\leq \frac{1}{\int^0_{-1}w_{-s}(\phi)\mathbf N^\xi({\mathrm d}s, {\mathrm d}w)},\quad t\geq 1,
%\end{equation}
	%and
	%\begin{equation}\label{eq:E.53}
	%\mathbb Q\left(\frac{1}{\int^0_{-1}w_{-s}(\phi)\mathbf N^\xi({\mathrm d}s, {\mathrm d}w)}\right)= \widetilde {\mathbb N}_{\phi\cdot\nu} [w_1(\phi)^{-1}]
	%= \mathbb N_{\phi\cdot\nu} (w_1\neq \mathbf 0) < \infty.
%\end{equation}
	Also observe that
	\begin{equation}\label{eq:E.53}
	\big(X_t(\phi) + Z^{(-t,0]}_0(\phi)\big)^{-1}
	\leq Z^{(-1,0]}_0(\phi)^{-1},
	\quad t\geq 1,
	\end{equation}
	and
	\begin{align}
	&\mathbb Q [Z^{(-1,0]}_0(\phi)^{-1}]
	\overset{\text{Lemma \ref{thm:E.3}}} = \mathbb Q [Z^{(0,1]}_1(\phi)^{-1}]
	\overset{\text{Lemma \ref{thm:E.2}}}= \widetilde {\nu\mathbb N} [w_1(\phi)^{-1}]
	\\&\overset{\text{Lemma \ref{thm:E.1}}}= e^{-\lambda}(\nu\mathbb N) (w_1\neq \mathbf 0) 
	\overset{\eqref{eq:E.14}} = -e^{-\lambda} \log \mathbb P_{\nu}(X_1 = \mathbf 0) 
	\\& \label{eq:E.54} \overset{\eqref{asp:H1}}< \infty.
	\end{align}
%}
	Now by \eqref{eq:E.5}, \eqref{eq:E.51}, \eqref{eq:E.52}, \eqref{eq:E.53}, \eqref{eq:E.54} and the dominated  convergence theorem, we obtain
	\begin{equation}\label{eq:E.55}
	%{\lim_{t\to\infty} e^{-\lambda t}\mathbb P_{\nu}(\tau_0>t)=\mathbb Q\left(\frac{1}{\int^0_{-\infty}w_{-s}(\phi)\mathbf N^\xi({\mathrm d}s, {\mathrm d}w)}\right):=k<\infty.
	 \lim_{t\to\infty} e^{-\lambda t}\mathbb P_{\nu}(X_t \neq \mathbf 0)
	 =\mathbb Q[Z^{(-\infty,0]}_0(\phi)^{-1}]
	 =:k<\infty.
	 \qedhere
	%}
	\end{equation}
\end{proof*}
	\begin{proof*}[Proof of that $\nu(l)<\infty$ is sufficient for $k>0$.]
	%{Now we try to find an equivalent condition for $k>0$.%}
	%{ For the continuum immigration part, we have
	Let us now assume $\nu(l)<\infty$.
	Notice that
	%}
	\begin{align}
	%{\mathbb Q\left(Z^{\mathrm n}_\infty \right)=%}
	%{\mathbb Q\left(\int^0_{-\infty}w_{-s}(\phi)\mathbf N^\xi({\mathrm d}s, {\mathrm d}w) \right)
	&\mathbb Q\Big[\int_{(-\infty,0]\times \mathbb D}w_{-s}(\phi) N({\mathrm d}s, {\mathrm d}w) \Big]
	= \mathbb Q\bigg[\mathbb Q\Big[\int_{(-\infty,0]\times \mathbb D}w_{-s}(\phi) N({\mathrm d}s, {\mathrm d}w) \Big | X, \xi \Big]\bigg]%}
	\\& \overset{\text{Theorem \ref{thm:E.16} (3)}}= \mathbb Q\Big[\int_{(-\infty,0]\times \mathbb D}w_{-s}(\phi) 2 \sigma(\xi_s)^2 {\mathrm d}s \cdot \mathbb N_{\xi_s}({\mathrm d}w) \Big]%}
	%{\\&=\int^0_{-\infty} e^{-\lambda s}\nu(2\sigma^2\phi^2)  \mathrm ds=\frac{\nu(2\sigma^2\phi^2)}{-\lambda}<\infty.
	\\& \overset{\eqref{eq:E.13}, \eqref{asp:H1}}=\mathbb Q \Big[\int^0_{-\infty} e^{-\lambda s} 2\sigma(\xi_s)^2 \phi(\xi_s) \mathrm ds\Big]
	\\&\overset{\text{Theorem \ref{thm:E.16} (2)}}=\widetilde{\nu \Pi} \Big[\int^0_{-\infty} e^{-\lambda s} 2\sigma(\xi_s)^2 \phi(\xi_s) \mathrm ds\Big]
	\\&\label{eq:E.64}\overset{\text{Theorem \ref{thm:E.15}}}= 2\int^0_{-\infty} e^{-\lambda s} \widetilde\nu(\sigma^2 \phi) \mathrm ds
	<\infty,
	%}
	\end{align}
%{
	where in the last inequality, we used the fact that $\sigma$, $\phi$ are bounded and $\lambda < 0$.
%}

%{
	Note that 
\begin{equation} \label{eq:E.65}
\begin{minipage}{0.9\textwidth}
	if $\{Z; \mathrm P\}$ is a non-negative random variable such that there exists a $\sigma$-field $\mathscr H$ with $\mathrm P(\mathrm P[Z|\mathscr H] < \infty) = 1$, then $\mathrm P(Z < \infty) = 1$.
\end{minipage}
\end{equation}
	In fact, from
\[
	\infty \cdot \mathrm P(Z = \infty | \mathscr F)
	 = \mathrm P[Z ; Z = \infty | \mathscr F] 
	 \leq \mathrm P[Z| \mathscr F] < \infty, \quad \mathrm P\text{-a.s.}
\] 
	we know that 
\[
	\mathrm P(Z = \infty| \mathscr F) = 0, \quad \mathrm P\text{-a.s.}
\]
	and therefore,
\[
	\mathrm P(Z = \infty) = \mathrm P[\mathrm P(Z = \infty| \mathscr F)] = 0.
\]
%}

%{Therefore, $Z^{\mathrm n}_\infty$ is finite almost surely.%}
	
%{	For the discrete immigration part, let $\mathcal G=\sigma((\xi_t)_{t\in\mathbb R}, ( m_\sigma)_{\sigma\in\mathcal D^{\mathrm m}})$.
	Recall from Lemma \ref{thm:E.16} (5) that $\mathscr G := \sigma(X, \xi, N, (s_k, y_k)_{k\in \mathbb N})$.
	Therefore, $\mathbb Q$-almost surely, 
\begin{align} 
	& \mathbb Q\Big(\sum_{k\in \mathbb N} w^{(k)}_{-s_k}(\phi) \mathbf 1_{s_k \in (-\infty,0]}\Big|\mathscr G \Big)
	\overset{\text{Lemma \ref{thm:E.16} (5)}}= \sum_{k\in \mathbb N} \mathbf 1_{s_k \in (-\infty,0]} \mathbb P_{y_k\delta_{\xi_{s_k}}}[w^{(k)}_{-s_k}(\phi)] 
	\\ &\overset{\eqref{eq:M.2}}= \sum_{k\in \mathbb N} \mathbf 1_{s_k \in (-\infty,0]} y_k (P_{-s_k}^\beta \phi)(\xi_{s_k})
	\overset{\eqref{asp:H1}} = \sum_{k\in \mathbb N} \mathbf 1_{s_k \in (-\infty,0]} y_k e^{-\lambda s_k} \phi(\xi_{s_k})
	\\ & \overset{\text{Lemma \ref{thm:E.4} (1)}} < \infty.
\end{align}
	Therefore, by \eqref{eq:E.65} we know that
\begin{equation} \label{eq:E.67} 
	\sum_{k\in \mathbb N} w^{(k)}_{-s_k}(\phi) \mathbf 1_{s_k \in (-\infty,0]}< \infty, \quad \mathbb Q\text{-a.s.}
\end{equation}
	Now, we have 
\begin{equation} \label{eq:E.68}
	Z_0^{(-\infty, 0]}(\phi) \overset{\eqref{eq:E.4}}= \int_{(-\infty,0]\times \mathbb D}w_{-s}(\phi) N({\mathrm d}s, {\mathrm d}w) + \sum_{k\in \mathbb N} w^{(k)}_{-s_k}(\phi) \mathbf 1_{s_k \in (-\infty,0]}
	\overset{\eqref{eq:E.64}, \eqref{eq:E.67}} < \infty.
\end{equation}
	Therefore
\begin{align} 
& k \overset{\eqref{eq:E.55}}= \mathbb Q[Z^{(-\infty,0]}_0(\phi)^{-1}] \overset{\eqref{eq:E.68}}> 0. \qedhere
\end{align}
%}
%{When  $\nu(l)<\infty$,  by Lemma \ref{thm:E.4} (1),
	%\[
	%\mathbb Q\left(Z^{\mathrm m}_\infty\Big|\mathcal G \right)
	%=\sum_{-\infty<\sigma\le 0}m_\sigma e^{\lambda \sigma}\phi(\xi_{\sigma})<\infty,  \qquad\qquad \mathbb Q-{\mathrm a.s.}
	%\]
	%Thus in this case, $k>0$. %}
	\end{proof*}
\begin{proof*}[Proof of that $\nu(l)<\infty$ is necessary for $k>0$]
	%{
	%We claim that when
	%$\nu(l)=\infty$,
	%\begin{equation}\label{Z-infty}
	%\lim_{t\rightarrow\infty}\int^0_{-\infty}w_{-s}(\phi)\mathbf N^\xi({\mathrm d}s, {\mathrm d}w)
	%=\infty,\quad\qquad \mathbb Q-{\mathrm a.s.},
	%\end{equation} which implies that $k=0$ by \eqref{eq:E.55}.
	%Now we prove the above claim. By Lemma \ref{thm:E.4} (2), for any $N>0$,
	%\begin{equation}\label{inf}
	%{\int^0_{-\infty} dt\int_{\phi(\xi_t)^{-1}e^{-Nt}}^\infty r\pi(\xi_t, \mathrm dr) =\infty,\quad \widehat{\mathbb Q}_\nu-{\mathrm a.s.}
	%\int^0_{-\infty} \mathrm dt\int_{\phi(\xi_t)^{-1}e^{-Nt}}^\infty r\pi(\xi_t, \mathrm dr) =\infty,\quad \mathbb Q\text{-a.s.} %}
	%\end{equation}
	%}
	%{Fix a path of $(\xi_t)_{t\in \mathbb R}$ and define
	%\[
	%\tau_1:=\sup\left\{t<0; m_t>\phi(\xi_t)^{-1}e^{-Nt}\right\},\,
	%\tau_{i+1}:=\sup\left\{t<\tau_i;\, m_t>\phi(\xi_t)^{-1}e^{-Nt}\right\},\, i=1,2,\cdots
	%\]
	%}
	%{ Then $\tau_i<\infty$, $i=1,2,\ldots$ almost surely from Lemma \ref{thm:E.4}.%}
	%{If we can prove that $\sum_{i=1}^\infty I_{\left\{\langle\phi, X^{{\mathrm m},\tau_i}_{-\tau_i}\rangle  >\varepsilon\right\}}=\infty$ for some $\varepsilon>0$, then our claim holds.
	We claim that when $\nu(l) = \infty$, for any $\varepsilon>0$, $\mathbb Q(\cdot | X, \xi, N)$-almost surely, 
\begin{equation} \label{eq:E.7}
	\#\{k\in \mathbb N: s_k\leq 0, w_{-s_k}^{(k)}(\phi)> \varepsilon\} =\infty.
\end{equation} 
	Therefore
\begin{equation} \label{eq:E.71}
	Z_0^{(-\infty, 0]} \overset{\eqref{eq:E.4}}\geq \sum_{k\in \mathbb N} \mathbf 1_{s_k \in (-\infty, 0]}  w^{(k)}_{-s_k}(\phi) 
	\overset{\eqref{eq:E.7}}= \infty, \quad \mathbb Q(\cdot | X, \xi, N)\text{-a.s.}
\end{equation}
	Thus
\begin{align} 
& k \overset{\eqref{eq:E.55}}= \mathbb Q[Z^{(-\infty,0]}_0(\phi)^{-1}] 
=\mathbb Q\big[\mathbb Q[Z^{(-\infty,0]}_0(\phi)^{-1}| \xi, X, N]\big]  \overset{\eqref{eq:E.71}}= 0. \qedhere
\end{align}
\end{proof*}

\begin{proof*}[Proof of Claim \eqref{eq:E.7}]
	Fix an arbitrary $\epsilon>0$. 
	Suppose that $s_0, \varepsilon>0$ and $\delta > 0$ are given by Lemma \ref{thm:E.5}.
	Then we can verify that
\begin{align}
	&\mathbb Q\big(\#\{k\in \mathbb N:s_k\leq 0, w_{-s_k}^{(k)}(\phi)> \varepsilon\} \big| X, \xi, N\big)
	\\&\overset{\text{Lemma \ref{thm:E.19}}}= \int_{-\infty}^0 \mathrm ds \int_{(0,\infty)}  y \pi(\xi_s, \mathrm dy) \mathbb P_{y \sigma_{\xi_s}}\big(w_{-s}(\phi)>\varepsilon\big)
	\\&\overset{\text{Lemma \ref{thm:E.5}}}\geq  \delta \int_{-\infty}^{-s_0} \mathrm ds \int_{(\phi(\xi_s)^{-1}e^{-\epsilon s},\infty)}  y \pi(\xi_s, \mathrm dy) 
	\\ \label{eq:E.9} & \overset{\text{Lemma \ref{thm:E.4} (2)}}= \infty.
\end{align}
	The desired result then follows from \eqref{eq:E.9} and Lemma \ref{thm:E.19}.
\end{proof*}
	%We also claim that 
%\[
	%\sum_{k \in \mathbb N} \delta_{s_k, w^{(k)}} (\mathrm ds,\mathrm dw)
%\]
	%is a Poisson random measure with intensity 
%\[
	%\mathrm ds \cdot \int_{y\in (0,\infty)} y \pi (\xi_s, \mathrm dy)\mathbb P_{y \delta_{\xi_{s}}} (\mathrm dw).
%\]
%}

%{
	%Since for any $T>0$, 
	%conditional on $\sigma((\xi_t)_{t\in \mathbb R})$, $\sum_{\tau_i\leq T} I_{\left\{\langle\phi, X^{{\mathrm m},\tau_i}_{-\tau_i}\rangle  >\varepsilon\right\}}$ is a Poisson random variable with parameter
	%\begin{equation}
	%\mathbb Q
	%\left[\left.\sum_{\tau_i\geq -T} I_{\left\{\langle\phi, X^{{\mathrm m},\tau_i}_{-\tau_i}\rangle  >\varepsilon\right\}} \right| (\xi_t)_{t\in \mathbb R}\right]
	%=\int^0_{-T} \mathrm dt\int_{\phi(\xi_t)^{-1}e^{-Nt}}^\infty r\pi(\xi_t, \mathrm dr)\mathbb{P}_{r\delta_{\xi_t}}\big(\langle\phi, X_{-t} \rangle >\varepsilon\big),
	%\end{equation}
	%we only need to prove, for some $\varepsilon>0$,
	%\begin{equation}\label{infty-sum}
	%\int^0_{-\infty} \mathrm dt\int_{\phi(\xi_t)^{-1}e^{-Nt}}^\infty r\pi(\xi_t, \mathrm dr)\mathbb{P}_{r\delta_{\xi_t}}\big(\langle\phi, X_{-t} \rangle >\varepsilon\big)=\infty \quad \widetilde \Pi_{\phi\cdot\nu}{\mathrm -a.s.}
	%\end{equation}
	%If we can prove for some $\varepsilon>0$, there exist $t_0>0$ and $\delta>0$ such that 
	%when $t<-t_0$,
	%\begin{equation}\label{last point}
	%\inf_{r\geq \phi(x)^{-1}e^{-Nt}, x\in E}\mathbb P_{r\delta_x}\big(\langle\phi, X_{-t}
	%\rangle >\varepsilon\big)>\delta,
	%\end{equation}
	%then from \eqref{inf}, \eqref{infty-sum} is obtained.  By Chebyshev's inequality, for any $\varepsilon>0$,
	%\begin{align}
	%&\mathbb P_{r\delta_x}\big(\langle\phi, X_{-t}\rangle >\varepsilon\big)=\mathbb P_{r\delta_x}\left(e^{-\langle\phi, X_{-t}\rangle }<e^{-\varepsilon}\right)\\
	%=&1-\mathbb P_{r\delta_x}\left(e^{-\langle\phi, X_{-t}
	%	\rangle }\geq e^{-\varepsilon}\right)\geq 1-e^{\varepsilon }\mathbb P_{r\delta_x}e^{-\langle\phi, X_{-t}\rangle }\\
	%=&1-e^{\varepsilon }e^{-rV_{-t}\phi(x)},\label{Cheby}
	%\end{align}
	%According to Lemma \ref{lem:rate} and \eqref{inequ:lower}, 
	%when $-t$ is sufficiently large, we have
	%\[
	%V_{-t}\phi(x)\ge\frac{1}{2}\phi(x)\nu(V_{-t}\phi)\geq \dfrac{a}{2}\phi(x)e^{Nt},\quad x\in E.
	%\]
	%Therefore, using \eqref{Cheby}, we obtain that, when $-t$ is sufficiently large,
	%\begin{eqnarray*}
%		&&\inf_{r\geq \phi(x)^{-1}e^{-Nt}, x\in E}\mathbb P_{r\delta_x}\big(\langle\phi, X_{-t} \rangle >\varepsilon\big)\geq \inf_{r\geq \phi(x)^{-1}e^{-Nt}, x\in E} \left(1-e^{\varepsilon}e^{-rV_{-t}\phi(x)}\right)\\ 
%		&&     \geq 1-e^{\varepsilon-a/2}.
%	\end{eqnarray*}
%	Then we get \eqref{last point} by	choosing $\varepsilon\in(0, a/2)$ and $\delta=1-e^{\varepsilon-a/2}$.  Then
%	\eqref{Z-infty} follows, and thus $k=0$. The proof is finished.
%}
\end{proof}
%END MOVED FROM BLOW

\section{Preliminaries}
\begin{comment}
\subsection{Spine process and its time reverse}\label{spine-decom}

	Through out this subsection, we suppose Assumptions \eqref{asp:H1}, \eqref{asp:H2} and \eqref{asp:H3} hold. 
	Let $\{(\xi_t)_{t\geq 0}; (\Pi_x)_{x\in E}\}$ be the spatial motion, and let  $(\mathscr F_t^{\xi})_{t\geq 0}$ be the natural filtration of process $(\xi_t)_{t\geq 0}$.
For each $x\in E$, let the probability $\widetilde \Pi_{x}$ be the Doob $h$-transform of $\Pi_x$ defined by
\begin{align}
	\dfrac{{\mathrm d}\widetilde{\Pi}_x|_{\mathscr F^{\xi}_t}}{{\mathrm d}\Pi_x|_{\mathscr F^{\xi}_t}}= \frac{e^{\int_0^t \beta(\xi_s)ds}\phi(\xi_t) \mathbf 1_{\{t<\tau\}}}{e^{\lambda t}\phi(x)},
	\quad t\geq 0.
\end{align}
It can be verified (see \cite{KimSong2008Intrinsic} for example) that the process $\{(\xi_t)_{t\geq 0}; (\widetilde\Pi_x)_{x\in E}\}$ is a time homogeneous Markov process.  Its transition density with respect to the measure $m$ is given by
\begin{equation}
\label{eq: tilde p}
    \tilde p(t, x, y)
    :=\frac{\mbox{e}^{-\lambda t}}{\phi(x)}\ p^\beta(t, x, y)\phi(y),
    \quad x,y \in E,t>0.
\end{equation}
	It can also be verified that $\phi(y)\nu({\mathrm d}y)$ is the unique
invariant probability measure
of $\{(\xi_t)_{t\geq 0}; (\widetilde\Pi_x)_{x\in E}\}$.
\end{comment}


It follows from \cite[Theorem 2.7]{KimSong2008Intrinsic} that there exists $c, \rho > 0$ such that
\begin{equation}\label{IU}
 \sup_{x,y\in E}\Big|\frac{\tilde p(t,x,y)}{\phi(y) \widehat\phi(y)}- 1\Big|
 =\sup_{x,y\in E}\Big|\frac{e^{-\lambda t}p^\beta(t,x,y)}{\phi(x) \widehat\phi(y)}- 1\Big|
	\leq c\,e^{-\rho t},
	\quad t\geq 1. \footnote{ZS: I'm about to replace this with (H2)}
\end{equation}
\begin{comment}
	Let $\{(\widehat{Y}_t)_{t\geq 0}; (\widehat{\Pi}_x)_{x\in E}\}$ be  the dual of $\{({Y}_t)_{t\geq 0}; ({\Pi}_x)_{x\in E}\}$, which is $E$-valued Markov  process whose transition density with respect to measure $m$ is given by
\[
    \hat{p}(t,x,y)
    =e^{-\lambda t}p^\beta(t,y,x)\frac{{\widehat\phi}(y)}{{\widehat\phi}(x)}
    =\tilde p(t,y,x)\frac{\phi(y){\widehat\phi}(y)}{\phi(x){\widehat\phi}(x)},
    \quad x,y \in E,\,\, t> 0.
\]
\end{comment}
It is easy to check that $(\widehat Y_t)_{t\geq 0}$ has
unique  invariant probability distribution $\phi(x)\widehat\phi(x)m({\mathrm d}x)$,
and is exponentially ergodic in the sense that
\begin{equation}\label{IU'}
	\sup_{x,y\in E}\left|\frac{\hat{p}(t, x,y)}{\phi(y) \widehat\phi(y)}- 1\right|\le c\,\mbox{e}^{-\rho t}, \quad t\geq 1,
\end{equation} where $c, \rho > 0$ are the constant in \eqref{IU}.

\begin{comment}
For any $\mu \in \mathcal M_f(E)\backslash\{\mathbf{0}\}$, define
\[
	\Pi_{\mu}(\cdot)
	:= \mu(E)^{-1}\int_{E} \Pi_x(\cdot)\mu({\mathrm d}x)
\]
    and
\[
	\widetilde\Pi_{\mu}(\cdot):= \mu(E)^{-1} \int_E\widetilde\Pi_x(\cdot)\mu({\mathrm d}x).
\]
For any function $f \in \mathcal B_b(E,[0,\infty))$
and measure $\mu \in \mathcal M_f(E)$, define
\[
 (f\cdot \mu)({\mathrm d}x) := f(x)\mu({\mathrm d}x),
    \quad x\in E.
\]

Since $\phi\cdot\nu$  is the unique invariant probability measure
of $\{(\xi_t)_{t\geq 0}; (\widetilde\Pi_x)_{x\in E}\}$, $\{(\xi_t)_{t\geq 0}; \widetilde\Pi_{\phi\cdot\nu}\}$ is a stationary Markov process.

\begin{lem}
For each $\mu\in \mathcal M_f(E)\backslash\{{\bf 0}\}$,
we have
\[
	\dfrac{{\mathrm d}\widetilde \Pi_{\phi\cdot\mu}|_{\mathscr F_t^{\xi}}}{{\mathrm d}\Pi_{\mu}|_{\mathscr F_t^{\xi}}}
  	:= \frac{e^{\int_0^t \beta(\xi_s)ds}\phi(\xi_t) \mathbf 1_{\{t<\tau\}}}{\mu(E)^{-1}e^{\lambda t}\langle \phi,\mu\rangle},
  	\quad t\geq 0.
\]
\end{lem}
\begin{proof}
Fix an arbitrary time $t\geq 0$. Fix an arbitrary event $A \in \mathscr F_t^{\xi}$.
Then we have
\begin{align}
	&\widetilde{\Pi}_{\phi\cdot\mu}(A)
	= \langle\phi, \mu\rangle^{-1} \int_E \widetilde \Pi_x(A)\phi(x)\mu({\mathrm d}x)
	\\&=\langle\phi, \mu\rangle^{-1} \int_E  \Pi_x\Big[\frac{e^{\int_0^t \beta(\xi_s)ds}\phi(\xi_t) \mathbf 1_{\{t<\tau\}}}{e^{\lambda t}\phi(x)} \mathbf 1_A\Big]\phi(x)\mu({\mathrm d}x)
	\\&= \mu(E)^{-1}\int_E  \Pi_x\Big[\frac{e^{\int_0^t \beta(\xi_s)ds}\phi(\xi_t) \mathbf 1_{\{t<\tau\}}}{\mu(E)^{-1}e^{\lambda t}\langle \phi,\mu\rangle} \mathbf 1_A\Big]\mu({\mathrm d}x)
	\\&= \Pi_{\mu}\Big[\frac{e^{\int_0^t \beta(\xi_s)ds}\phi(\xi_t) \mathbf 1_{\{t<\tau\}}}{\mu(E)^{-1} e^{\lambda t}\langle \phi,\mu\rangle} \mathbf 1_A\Big].
	\qedhere
\end{align}
\end{proof}

As a consequence of the lemma above, we have
\[
{\mathrm d}\widetilde\Pi_{\phi\cdot\nu}|_{\mathscr F_t^{\xi}}=
\frac{e^{\int_0^t \beta(\xi_s)ds}\phi(\xi_t) \mathbf 1_{\{t<\tau\}}}{e^{\lambda t}}
{\mathrm d}\Pi_{\nu}|_{\mathscr F_t^{\xi}}, \quad t\ge 0.
\]
\end{comment}

\begin{comment}
\begin{lem}
\label{lem:reverse of the spine}
	Let $\nu(dx):=\widehat\phi(x)m({\mathrm d}x)$.
	For each $T > 0$, we have
\[
	\{(Y_{T-t})_{0\leq t\leq T}; \widetilde \Pi_{\phi \cdot \nu}\}
	\overset{f.d.d.} = \{(\widehat Y_{t})_{0\leq t\leq T}; \widehat \Pi_{\phi \cdot \nu}\}
\]
\end{lem}
The proof of the above lemma is given in the Appendix.
\end{comment}

\begin{comment}
\subsection{Kuznetsov measures}


Denote by $\mathbb D_e$ the space of $\mathcal M_f(E)$-valued c\`adl\`ag paths $w$ on $(0,\infty)$ with $\mathbf 0$ as a trap.
According to \cite[Section 8.4]{Li2011Measure-valued}, there is a unique family of $\sigma$-finite measures $(\mathbb N_x)_{x\in E}$ on $\mathbb D_e$ such that (1) 
$\mathbb N_x ( w_t =\mathbf 0, \forall t > 0) =0$ for each $x\in E$; 
and (2) for each $\mu \in \mathcal M(E)$, if $\mathcal N$ is a Poisson random measure on $\mathbb D_e$ with intensity
\[
\mathbb N_\mu({\mathrm d}w):= \int_E \mathbb N_x({\mathrm d}w)\mu({\mathrm d}x), \quad w\in \mathbb D_e,
\]
then
\[
\{(X_t)_{t> 0};\mathbb P_\mu\}
\overset{f.d.d.}{=} \left(\int_{\mathbb D_e} w_t\mathcal N({\mathrm d}w)\right)_{t> 0}.
\]
This family of measure $(\mathbb N_x)_{x\in E}$ is known as the \emph{Kuznetsov measures} of $X$.
According to \cite[Theorem 8.22]{Li2011Measure-valued}, for each $x\in E$, the Kuznetsov measure $\mathbb N_x$ can be, and will be,  understood as a measure on $\mathbb D$.
\end{comment}


\begin{comment}
\subsection{Spine decomposition}
Suppose that $X$ is the superprocess introduced in Section \ref{sec:M} which satisfies Assumptions \ref{asp:H1}, \ref{asp:H2} and \ref{asp:H3}. 
Recall that we always  assume that $X$ is canonical. Fix an arbitrary $\mu\in \mathcal M_f(E)\setminus\{\mathbf{\mathbf{0}}\}$.  Let  $\widetilde {\mathbb P}_\mu$ be the probability defined by  \eqref{eq:M.3},
and let $\widetilde {\mathbb N}_\mu$ be the unique probability measure on $\mathbb D$ such that for each $t\geq 0$,
\[
{\mathrm d} \widetilde {\mathbb N}_\mu|_{\mathscr F_t} = \frac{w_t(\phi)}{\mathbb N_\mu[w_t(\phi)]} {\mathrm d}\mathbb N_\mu|_{\mathscr F_t}.
\]



Let us first recall the spine decomposition theorem.
Recall that $\xi$ is the spatial motion of the superprocess $X$.
In the  spirit of \cite{RenSongSun2020Spine}, we let $\{(Y_t)_{t\geq 0}, \mathbf N^Y; \mathbb Q_\mu\}$
 be a spine decomposition of $\{(w_t)_{t\geq 0}; \widetilde {\mathbb N}_\mu\}$, i.e.,
(i) $\{(Y_t)_{t\geq 0}; \mathbb Q_\mu\} \overset{d}= \{(Y_t)_{t\geq 0}; \widetilde \Pi_{\mu} \}$, and $\{(Y_t)_{t\geq 0}; \mathbb Q_\mu\}$ is called the {\it spine process};
(ii) given the path of the spine
$\{(Y_t)_{t\geq 0}; \mathbb Q_\mu\}$,
$\{\mathbf N^Y; \mathbb Q_\mu\}$ is a Possion random measure on $[0,\infty) \times \mathbb D$ with intensity
\begin{align}\label{e:immigration}
 2 \sigma(Y_s)^2 {\mathrm d}s \cdot \mathbb N_{Y_s}({\mathrm d}w)+ {\mathrm d}s \cdot \int_{(0,\infty)} y \mathbb P_{y\delta_{Y_s}}(X\in {\mathrm d}w) \pi(Y_s,{\mathrm d}y).
\end{align}
Then according to \cite{RenSongSun2020Spine}, we have that
\begin{align}
 & \{(w_t)_{t\geq 0}; \widetilde{ \mathbb N}_\mu \} \overset{{f.d.d.}} = \Big\{ \Big( \int_{[0,t)\times \mathbb D} w_{t-s} \mathbf N^Y({\mathrm d}s,{\mathrm d}w) \Big)_{t\geq 0} ; \mathbb Q_\mu\Big\}.
 \end{align}
Define
\begin{align}
& Z_t:= \int_{[0,t)\times \mathbb D} w_{t-s} \mathbf N^Y({\mathrm d}s,{\mathrm d}w), \quad t\geq 0.
\end{align}
According to \cite{RenSongSun2020Spine} and \cite{RenSongYang2016Spine},  we have
\begin{equation}\label{spine-decom1}
\{(X_t)_{t\geq 0}; \widetilde {\mathbb P}_\mu\} \overset{f.d.d.} = \{ (X_t+ Z_t)_{t\geq 0}; \mathbb Q_\mu\},
\end{equation}
where  $\{(X_t)_{t\geq 0}; \mathbb Q_\mu\} \overset{d} =  \{(X_t)_{t\geq 0}; \mathbb P_\mu\}$,   and $\{(X_t)_{t\geq 0}; \mathbb Q_\mu\} $ and $\{(Z_t)_{t\geq 0}; \mathbb Q_\mu\}$ are independent.
We call $\{\mathbf N^Y; \mathbb Q_\mu\}$ the immigration along the spine $(Y_t)_{t\geq 0}$. 
\end{comment}

\begin{comment}
%RS I do no think that we need this, so I am commenting this out. Some version of it is moved later.
%We may decompose $\{\mathbf N^Y; \mathbb Q_\mu\}$ 
Using \eqref{e:immigration}, we may decompose $\{\mathbf N^Y; \mathbb Q_\mu\}$ 
as the sum of two parts:
    $$\mathbf N^Y(ds,dw)=\mathrm n({\mathrm d}s,{\mathrm d}w)+\mathrm m({\mathrm d}s,{\mathrm d}w).$$
More precisely,
        we say $\{(Y)_{t\geq 0}, (X^{\mathrm n, \sigma})_{\sigma\in \mathcal D^\mathrm n}, (X^{\mathrm m, \sigma})_{\sigma \in \mathcal D^\mathrm m}, (X_t)_{t\geq 0}; \mathbb Q_{\mu}\}$ is a \emph{spine representation} of $\{(X_t)_{t\geq 0}; \widetilde {\mathbb P}_\mu\}$ if the followings are true:
\begin{itemize}
\item
    The \emph{spine process} $\{(Y_t)_{t\geq 0}; \mathbb Q_\mu\}$ is a copy of $\{(Y_t)_{t\geq 0}; \widetilde \Pi_{\phi\cdot\mu}\}$.
\item
	Given $\{(Y_t)_{t\geq 0}; \mathbb Q_\mu\}$, \emph{the continuum immigration} $\{ (X^{\mathrm n,\sigma})_{\sigma \in \mathcal D^\mathrm n}; \mathbb Q_\mu(\cdot |Y)\}$ is a $\mathbb D$-valued point process such that
\[
	\mathrm n({\mathrm d}s,{\mathrm d}w) := \sum_{\sigma\in \mathcal D^{\mathrm n}} \delta_{(\sigma, X^{\mathrm n,\sigma})}({\mathrm d}s,{\mathrm d}w)
\]
 is a Poisson random measure on $[0,T]\times \mathbb D$ with intensity
\[
\mathbf n({\mathrm d}s,{\mathrm d}w):= 2 \sigma(Y_s)^2 {\mathrm d}s \cdot \mathbb N_{Y_s}({\mathrm d}w).
\]
\item
	Given $\{(Y_t)_{t\geq 0}; \mathbb Q_\mu\}$, \emph{the discrete immigration} $\{(X^{\mathrm m,\sigma})_{\sigma\in \mathcal D^{\mathrm m}}; \mathbb Q_\mu(\cdot |Y)\}$ is a $\mathbb D$-valued point process such that
\[
	\mathrm m({\mathrm d}s,{\mathrm d}w) := \sum_{\sigma\in \mathcal D^{\mathrm n}} \delta_{(\sigma, X^{\mathrm n,\sigma})}({\mathrm d}s,{\mathrm d}w)
\]
	is a Poisson random measure on $[0,\infty ) \times \mathbb D$ with intensity
\begin{align}
 \mathbf m({\mathrm d}s,{\mathrm d}w):= {\mathrm d}s \cdot \int_{(0,\infty)} y \mathbb P_{y\delta_{Y_s}}(X\in {\mathrm d}w) \pi(Y_s,{\mathrm d}y);
\end{align}
\item
	Given $\{(Y_t)_{t\geq 0}; \mathbb Q_\mu\}$, the continuum immigration $(X^{\mathrm n,\sigma})_{\sigma \in \mathcal D^n}$ and the discrete immigration $(X^{\mathrm m,\sigma})_{\sigma\in \mathcal D^{\mathrm m}}$ are independent of each other.
\item
	$\{(X_t)_{t\geq 0}; \mathbb Q_\mu\}$ is a copy of the superprocess $\{(X_t)_{t\geq 0}; \mathbb P_\mu\}$, and is independent of the spine process $(Y_t)_{t\geq 0}$, the continuum immigration $(X^{\mathrm n,\sigma})_{\sigma \in \mathcal D^\mathrm n}$ and the discrete immigration $(X^{\mathrm m,\sigma})_{\sigma\in \mathcal D^{\mathrm m}}$.
\end{itemize}


%		To simplify notations, for each $\mu \in \mathcal M_f(E)\setminus\mathbf{0}$,
%	$t\geq 0$ and each $B \in \mathscr B([0,t))$,  define the following random measures:
For any $\mu \in \mathcal M_f(E)\setminus\mathbf{0}$,
and	$t\geq 0$, define
\begin{align}
%	Z^{\mathrm n,B}_t
	Z^{\mathrm n}_t
%	&:= \int_{B\times \mathbb D} w_{t-s} ~\mathrm n ({\mathrm d}s,{\mathrm d}w)
%	= \sum_{\sigma \in \mathcal D^\mathrm n \cap B} X^{\mathrm n,\sigma}_{t-\sigma},
&:= \int_{[0, t)\times \mathbb D} w_{t-s} ~\mathrm n ({\mathrm d}s,{\mathrm d}w)
	= \sum_{\sigma \in \mathcal D^\mathrm n \cap [0, t)} X^{\mathrm n,\sigma}_{t-\sigma},	
%	\\ Z^{\mathrm m,B}_t
	\\ Z^{\mathrm m}_t
%	&:= \int_{B\times \mathbb D} w_{t-s} ~\mathrm m ({\mathrm d}s,{\mathrm d}w)
%	= \sum_{\sigma \in \mathcal D^\mathrm m \cap B} X^{\mathrm m,\sigma}_{t-\sigma}.
&:= \int_{[0, t)\times \mathbb D} w_{t-s} ~\mathrm m ({\mathrm d}s,{\mathrm d}w)
	= \sum_{\sigma \in \mathcal D^\mathrm m \cap [0, t)} X^{\mathrm m,\sigma}_{t-\sigma}.
  \end{align}
Then \begin{equation}\label{def-Zt}
%Z_t= Z^{\mathrm n, [0,t)}_{t} + Z^{\mathrm m, [0,t)}_{t},
Z_t= Z^{\mathrm n}_{t} + Z^{\mathrm m}_{t},
\end{equation}
 and \eqref{spine-decom1} can be written as
\begin{align}\label{spine-decom2}
	\{(X_t)_{t\geq 0}; \widetilde{\mathbb P}_\mu\}
	\overset{f.d.d.}{=}
%\{(X_t + Z^{\mathrm n, [0,t)}_{t} + Z^{\mathrm m, [0,t)}_{t} )_{t\geq 0}; \mathbb Q_\mu\}.
\{(X_t + Z^{\mathrm n}_{t} + Z^{\mathrm m}_{t} )_{t\geq 0}; \mathbb Q_\mu\}.
\end{align}
We call the above representation as a spine decomposition of $\{(X_t)_{t\geq 0}; \widetilde{\mathbb P}_\mu\} $.
\end{comment}

\begin{comment} 
Now we show that the process $\{(Z_t)_{t\ge0}, \mathbb Q_\nu\}$ is stochastically non-decreasing. For this, we construct processes $\{(\xi_t)_{t\in\mathbb R}, \mathbf N^\xi, (X_t)_{t\ge 0}; \mathbb Q\}$
as follows:
\begin{itemize}
\item $\{(\xi_t)_{t\in\mathbb R}; \mathbb Q\}$ is a two-sided stationary Markov process with
$\{(\xi_t)_{t\ge0}; \mathbb Q\}\overset{f.d.d.}{=}\{(\xi_t)_{t\ge0}; \widetilde \Pi_{\phi\cdot\nu}\}$;
\item Conditioned on $\{(\xi_t)_{t\in\mathbb R}; \mathbb Q\}$, $\{(\mathbf N^\xi; \mathbb Q\}$ is a Poisson random measure on $\mathbb R\times \mathbb D$ with intensity
\begin{align}\label{e:newimmigration}
2\sigma(\xi_s^2){\mathrm d}s\cdot\mathbb N_{\xi_s}({\mathrm d}w)+
{\mathrm d}s\cdot\int_{(0, \infty)}y\mathbb P_{y\delta_{\xi_s}}({\mathrm d}w)\pi(\xi_s, {\mathrm d}y), \quad s\in \mathbb R, w\in \mathbb D;
\end{align}
\item $\{(X_t)_{t\ge 0}; \mathbb Q\}$ is a copy of the superprocess $\{(X_t)_{t\ge 0}; \mathbb P_\nu\}$, and is independent of $\{(\xi_t)_{t\in\mathbb R}, \mathbf N^\xi\}$.
\end{itemize}
\end{comment}
It follows from the spine decomposition that
\begin{align}\label{e:newspine1}
\{(w_t)_{t\ge 0}; \widetilde{\mathbb N}_{\phi\cdot\nu}\}\overset{f.d.d.}{=}\left\{\left(\int^t_0w_{t-s}\mathbf N^\xi({\mathrm d}s, {\mathrm d}w)\right)_{t\ge 0}; \mathbb Q
\right\}
\end{align}
and
\begin{align}\label{e:newspine2}
\{(X_t)_{t\ge 0}; \widetilde {\mathbb P}_\nu\}\overset{f.d.d.}{=}\left\{\left(X_t+\int^t_0w_{t-s}\mathbf N^\xi({\mathrm d}s, {\mathrm d}w)\right)_{t\ge 0}; \mathbb Q
\right\}.
\end{align}
\begin{comment}
\begin{lem}\label{stoch-increas}
The process $\{(w_t(\phi))_{t\ge0}; \widetilde{\mathbb N}_{\phi\cdot\nu}\}$ is stochastically non-decreasing.
\end{lem}

\begin{proof} Note that for any $t\ge 0$,
\begin{align*}
\{w_t(\phi);  \widetilde{\mathbb N}_{\phi\cdot\nu}\}\overset{d.}{=}\left\{\int^t_0w_{t-s}\mathbf N^\xi({\mathrm d}s, {\mathrm d}w); \mathbb Q\right\}
\overset{d.}{=}\left\{\int^0_{-t}w_{-s}\mathbf N^\xi({\mathrm d}s, {\mathrm d}w); \mathbb Q\right\},
\end{align*}
where the second equality follows from the stationarity of $\{(\xi_t)_{t\in\mathbb R}; \mathbb Q\}$. It is obvious that
\[
t\mapsto \int^0_{-t}w_{-s}\mathbf N^\xi({\mathrm d}s, {\mathrm d}w)
\]
is non-decreasing.
\end{proof}
\end{comment}

\begin{comment}
\subsection{Reverse spine representation}
	Suppose that $X$ is the superprocess introduced in Section \ref{sec:M} which satisfies Assumptions \eqref{asp:H1}, \eqref{asp:H2} and \eqref{asp:H3}. 
 Recall that $\nu := \widehat \phi \cdot m \in \mathcal M_f(E)$,
    and $\widetilde {\mathbb P}_\nu$ is defined in  \eqref{eq:M.3}. 
  Let $Z_t$ be defined by \eqref{def-Zt}. In this subsection we aim to prove that  $\{(Z_t)_{t\ge 0}, {\mathbb Q}_\nu\}$ 
    is stochastically non-decreasing, and give another decomposition of $\{(X_t)_{t\geq 0}; \widetilde{\mathbb P}_\mu\} $.
   Now we define stochastic processes $$\{(Y)_{t\geq 0}, (X^{\mathrm n, \sigma})_{\sigma\in \mathcal D^\mathrm n}, (X^{\mathrm m, \sigma})_{\sigma \in \mathcal D^\mathrm m}, (X_t)_{t\geq 0}; \widehat {\mathbb Q}_{\nu}\}$$ in the following way:
    \begin{itemize}
\item
      $\{(Y_t)_{t\geq 0}; \widehat {\mathbb Q}_\nu\}$ is a copy of $\{(Y_t)_{t\geq 0}; \widehat \Pi_{\phi\cdot\nu}\}$.
\item
    Conditioned on $\{(Y_t)_{t\geq 0}; \widehat{\mathbb Q}_\nu\}$,
    \emph{the  continuum immigration}
    $\{ (X^{\mathrm n,\sigma})_{\sigma \in \mathcal D^\mathrm n}; \widehat{\mathbb Q}_\nu(\cdot |Y)\}$ is a $\mathbb D$-valued point process such that
\[
    \mathrm n({\mathrm d}s,{\mathrm d}w)
    = \sum_{\sigma\in \mathcal D^{\mathrm n}} \delta_{(\sigma, X^{\mathrm n,\sigma})}({\mathrm d}s,{\mathrm d}w)
\]
is a Poisson random measure on $[0,T]\times \mathbb D$ with density
\[
\mathbf n({\mathrm d}s,{\mathrm d}w)= 2\sigma(Y_s)^2 {\mathrm d}s \cdot \mathbb N_{Y_s}({\mathrm d}w).
\]
\item
    Conditioned on $\{(Y_t)_{t\geq 0}; \widehat{\mathbb Q}_\nu\}$,
       \emph{the discrete immigration}
    $\{(X^{\mathrm m,\sigma})_{\sigma\in \mathcal D^{\mathrm m}}; \widehat{\mathbb Q}_\nu(\cdot |Y)\}$ is a $\mathbb D$-valued point process such that
\[
    \mathrm m({\mathrm d}s,{\mathrm d}w)
    = \sum_{\sigma\in \mathcal D^{\mathrm m}} \delta_{(\sigma, X^{\mathrm m,\sigma})}({\mathrm d}s,{\mathrm d}w)
\]
	is a Poisson random measure on $[0,\infty ) \times \mathbb D$ with intensity
\begin{align}
 \mathbf m({\mathrm d}s,{\mathrm d}w)= {\mathrm d}s \cdot \int_{(0,\infty)} y \mathbb P_{y\delta_{Y_s}}(X\in {\mathrm d}w) \pi(Y_s,{\mathrm d}y);
\end{align}
\item
	Given $\{(Y_t)_{t\geq 0}; \widehat{\mathbb Q}_\nu\}$, the  continuum immigration $(X^{\mathrm n,\sigma})_{\sigma \in \mathcal D^n}$ and the discrete immigration $(X^{\mathrm m,\sigma})_{\sigma\in \mathcal D^{\mathrm m}}$ are independent of each other.
\item
	$\{(X_t)_{t\geq 0}; \widehat {\mathbb Q}_\nu\}$ is a copy of the superprocess $\{(X_t)_{t\geq 0}; \mathbb P_\mu\}$ which is independent of the spine process $(Y_t)_{t\geq 0}$, the continuum immigration $(X^{\mathrm n,\sigma})_{\sigma \in \mathcal D^n}$ and the  discrete immigration $(X^{\mathrm m,\sigma})_{\sigma\in \mathcal D^{\mathrm m}}$.
\end{itemize}

	For each $t\geq 0$, with respect to probability $\widehat{\mathbb Q}_\nu$, define the following random measures:
\[\begin{split}
	\widehat Z^{\mathrm n}_t
	&:= \int_{[0,t)\times \mathbb D} w_{s} ~\mathrm n ({\mathrm d}s,{\mathrm d}w)
	= \sum_{\sigma \in \mathcal D^\mathrm n \cap [0,t)} X^{\mathrm n,\sigma}_{\sigma},
	\\ \widehat Z^{\mathrm m}_t
	&:= \int_{[0,t)\times \mathbb D} w_{s} ~\mathrm m ({\mathrm d}s,{\mathrm d}w)
	= \sum_{\sigma \in \mathcal D^\mathrm m \cap [0,t)} X^{\mathrm m,\sigma}_{\sigma},
\\ \widehat Z_t&:=\widehat Z^{\mathrm n}_t+\widehat Z^{\mathrm m}_t.
\end{split}\]
\begin{lem}\label{stoch-increas} Let $\widehat Z_{t}$ be defined above.
	For each fixed $t\geq 0$,
\begin{equation}\label{reverse-spine-decom}
	\{X_t; \widetilde{\mathbb P}_\nu\}
	\overset{d}{=}\{X_t +  Z_{t}; {\mathbb Q}_\nu\}
	\overset{d}{=}\{X_t + \widehat Z_{t}; \widehat{\mathbb Q}_\nu\}.
\end{equation}
\end{lem}
\begin{proof}
	Fix an arbitrary time $t\geq 0$. By \eqref{spine-decom2}, we only have to prove that
\[
	\{Z^{\mathrm n,[0,t)}_{t} + Z^{\mathrm m,[0,t)}_{t}; \mathbb Q_\nu\}
	\overset{d}{=}
	\{\widehat Z^{\mathrm n}_{t} + \widehat Z^{\mathrm m}_{t}; \widehat{\mathbb Q}_\nu\}.
\]
	In fact, for each $f\in \mathcal B_b^+(E)$, from campbell's formula, we have
\begin{align}
	&-\log \mathbb Q_\nu \left [\left. e^{-\langle f, Z^{\mathrm n,[0,t)}_{t} + Z^{\mathrm m,[0,t)}_{t}\rangle}\right |(Y_t)_{t\geq 0}\right]
	\\&= \int_{[0,t)\times \mathbb D} \left(1-e^{- \langle f, w_{t-s}\rangle}\right)\left(\mathbf n({\mathrm d}s,{\mathrm d}w) + \mathbf m({\mathrm d}s,{\mathrm d}w)\right)
    \\&= \int_{[0,t)} \left(2\sigma(Y_s)^2 \cdot \mathbb N_{Y_s}\left(1-e^{-w_{t-s}(f)}\right) + \int_{(0,\infty)} y \mathbb P_{y\delta_{Y_s}}\left(1-e^{-X_{t-s}(f)}\right)\pi(Y_s,{\mathrm d}y)\right) {\mathrm d}s
    \\&= \int_{[0,t)} \left(2\sigma(Y_s)^2 \cdot (V_{t-s}f)(Y_s) + \int_{(0,\infty)} y \left(1-e^{-y\cdot(V_{t-s}f)(Y_s)}\right)\pi(Y_s,{\mathrm d}y)\right) {\mathrm d}s
	\\&= \int_{[0,t)} \psi_0'\left( Y_s, V_{t-s}f(Y_s)\right){\mathrm d}s
\end{align}
	and
\begin{align}
	&-\log \widehat{\mathbb Q}_\nu \left [\left. e^{-(\widehat Z^{\mathrm n}_{t} + \widehat Z^{\mathrm m}_{t})(f)}\right |(Y_t)_{t\geq 0}\right]
	\\&= \int_{[0,t)\times \mathbb D} \left(1-e^{- w_s(f)}\right)\left(\mathbf n({\mathrm d}s,{\mathrm d}w) + \mathbf m({\mathrm d}s,{\mathrm d}w)\right)
    \\&= \int_{[0,t)} \left(2\sigma(Y_s)^2 \cdot \mathbb N_{Y_s}\left(1-e^{-w_{s}(f)}\right) + \int_{(0,\infty)} y \mathbb P_{y\delta_{Y_s}}\left(1-e^{-X_{s}(f)}\right)\pi(Y_s,{\mathrm d}y)\right) {\mathrm d}s	
    \\&= \int_{[0,t)} \left(2\sigma(Y_s)^2 \cdot (V_{t}f)(Y_s) + \int_{(0,\infty)} y \left(1-e^{-y\cdot(V_{t}f)(Y_s)}\right)\pi(Y_s,{\mathrm d}y)\right) {\mathrm d}s
	\\&= \int_{[0,t)} \psi_0'\left( Y_s, V_{t}f(Y_s)\right){\mathrm d}s.
\end{align}
	Therefore, according to Lemma \ref{lem:reverse of the spine}, for each
$f\in \mathcal B_b(E,[0,\infty))$,
 we have
\begin{align}
  	&\mathbb Q_\nu  \big[e^{-(Z^{\mathrm n,[0,t)}_{t} + Z^{\mathrm m,[0,t)}_{t})(f)}\big]
  	= \widetilde \Pi_{\phi\cdot\nu} \big[e^{-\int_{[0,t)} \psi_0'( Y_s, V_{t-s}f(Y_s)){\mathrm d}s}\big]
  	\\&= \widetilde \Pi_{\phi\cdot\nu} \big[e^{-\int_{[0,t)} \psi_0'( Y_{t-s}, V_{s}f(Y_{t-s})){\mathrm d}s}\big]
  	= \widehat \Pi_{\phi\cdot\nu} \big[e^{-\int_{[0,t)} \psi_0'( Y_{s}, V_{s}f(Y_{s})){\mathrm d}s}\big]
  	\\&= \widehat{\mathbb Q}_\nu \big [e^{-(\widehat Z^{\mathrm n}_{t} + \widehat Z^{\mathrm m}_{t})(f)}\big].
  	\qedhere
\end{align}
\end{proof}


 Lemma \ref{stoch-increas} says that $\{(Z_t)_{t\geq 0}; \mathbb Q_\nu \}$ is stochastically non-decreasing. We constructed  a non-decreasing stochastic process $\{(\widehat Z_t)_{t\geq 0}; \mathbb Q_\nu \}$ such that for each fixed $t>0$,  $\{\widehat Z_t; \widehat{\mathbb Q}_\nu\} \overset{d} = \{Z_t; \mathbb Q_\nu\}$. Note that $\{(Y_t)_{t\geq 0}; \widehat {\mathbb Q}_\nu\}$ is a version   of $\{(Y_t)_{t\geq 0}; \widehat \Pi_{\phi\cdot\nu}\}$, which can be regarded as time reverse of  a copy of $\{(Y_t)_{t\geq 0}; \widetilde\Pi_{\phi\cdot\nu}\}$, see Lemma \ref{lem:reverse of the spine}. The decomposition \ref{reverse-spine-decom} will be called {\it reverse spine representation} of $\{X_t; \widetilde{\mathbb P}_\nu\}$.
\end{comment}

\begin{comment}
Using \eqref{e:newimmigration}, we may decompose $\{\mathbf N^\xi; \mathbb Q\}$ 
as the sum of two parts:
 $$
 \mathbf N^\xi(ds,dw)=\mathrm n({\mathrm d}s,{\mathrm d}w)+\mathrm m({\mathrm d}s,{\mathrm d}w).
 $$
 More precisely, if 
$\{(\xi_t)_{t\in\mathbb R}, (X^{\mathrm n, \sigma})_{\sigma\in \mathcal D^\mathrm n}, (X^{\mathrm m, \sigma})_{\sigma \in \mathcal D^\mathrm m}, (X_t)_{t\geq 0}; \mathbb Q\}$ satisfies
\begin{itemize}
\item
   $\{(\xi_t)_{t \in\mathbb R}; \mathbb Q_\mu\}$ s a two-sided stationary Markov process with
$$
\{(\xi_t)_{t\ge0}; \mathbb Q\}\overset{f.d.d.}{=}\{(\xi_t)_{t\ge0}; \widetilde \Pi_{\phi\cdot\nu}\};
$$
\item
	Conditioned on $\{(\xi_t)_{t\in\mathbb R}; \mathbb Q\}$, \emph{the continuum immigration} $\{ (X^{\mathrm n, \sigma})_{\sigma \in \mathcal D^\mathrm n}; \mathbb Q(\cdot |(\xi_t)_{t\in\mathbb R})\}$ is a $\mathbb D$-valued point process such that
\[
	\mathrm n({\mathrm d}s,{\mathrm d}w) := \sum_{\sigma\in \mathcal D^{\mathrm n}} \delta_{(\sigma, X^{\mathrm n, \sigma})}({\mathrm d}s,{\mathrm d}w)
\] 
is a Poisson random measure on $\mathbb R\times \mathbb D$ with intensity
\[
\mathbf n({\mathrm d}s,{\mathrm d}w):= 2 \sigma(\xi_s)^2 {\mathrm d}s \cdot \mathbb N_{Y_s}({\mathrm d}w);
\]
\item
	Conditioned on $\{(\xi_t)_{t\in\mathbb R}; \mathbb Q\}$, \emph{the discrete immigration} $\{(X^{\mathrm m, \sigma})_{\sigma\in \mathcal D^{\mathrm m}}; \mathbb Q(\cdot |(\xi_t)_{t\in\mathbb R})\}$ is a $\mathbb D$-valued point process such that
\[
	\mathrm m({\mathrm d}s,{\mathrm d}w) := \sum_{\sigma\in \mathcal D^{\mathrm n}} \delta_{(\sigma, X^{\mathrm m, \sigma})}({\mathrm d}s,{\mathrm d}w)
\]
	is a Poisson random measure on $\mathbb R \times \mathbb D$ with intensity
\begin{align}
 \mathbf m({\mathrm d}s,{\mathrm d}w):= {\mathrm d}s \cdot \int_{(0,\infty)} y \mathbb P_{y\delta_{\xi_s}}(X\in {\mathrm d}w) \pi(\xi_s,{\mathrm d}y);
\end{align}
\item
	Given $\{(\xi_t)_{t\in\mathbb R}; \mathbb Q\}$, , the continuum immigration $(X^{\mathrm n,\sigma})_{\sigma \in \mathcal D^n}$ and the discrete immigration $(X^{\mathrm m,\sigma})_{\sigma\in \mathcal D^{\mathrm m}}$ are independent of each other.
\item
	$\{(X_t)_{t\geq 0}; \mathbb Q\}$ is a copy of the superprocess $\{(X_t)_{t\geq 0}; \mathbb P_\mu\}$, and is independent of the spine process $(\xi_t)_{t\in \mathbb R}$, the continuum immigration $(X^{\mathrm n, \sigma})_{\sigma \in \mathcal D^\mathrm n}$ and the discrete immigration $(X^{\mathrm m, \sigma})_{\sigma\in \mathcal D^{\mathrm m}}$.
\end{itemize}
\end{comment}


For any $t\geq 0$, define
\begin{align}
	Z^{\mathrm n}_t
&:= \int_{[0, t)\times \mathbb D} w_{t-s} ~\mathrm n ({\mathrm d}s,{\mathrm d}w)
	= \sum_{\sigma \in \mathcal D^\mathrm n \cap [0, t)} X^{\mathrm n,\sigma}_{t-\sigma},	
	\\ Z^{\mathrm m}_t
&:= \int_{[0, t)\times \mathbb D} w_{t-s} ~\mathrm m ({\mathrm d}s,{\mathrm d}w)
	= \sum_{\sigma \in \mathcal D^\mathrm m \cap [0, t)} X^{\mathrm m,\sigma}_{t-\sigma}.
  \end{align}
Then \begin{equation}\label{def-Zt}
\int^t_0w_{t-s}\mathbf N^\xi({\mathrm d}s, {\mathrm d}w)= Z^{\mathrm n}_{t} + Z^{\mathrm m}_{t},
\end{equation}
 and \eqref{e:newspine2} can be written as
\begin{align}\label{spine-decom2}
	\{(X_t)_{t\geq 0}; \widetilde{\mathbb P}_\nu\}
	\overset{f.d.d.}{=}
\{(X_t + Z^{\mathrm n}_{t} + Z^{\mathrm m}_{t} )_{t\geq 0}; \mathbb Q\}.
\end{align}
We call the above representation a spine decomposition of $\{(X_t)_{t\geq 0}; \widetilde{\mathbb P}_\nu\} $.

\begin{comment}
The following lemma is similar to
 Lemma $3.2$ in \cite{LiuRenSong2009Llog} and the argument to prove it is similar. For readers convenience, we will give the detail proof in the Appendix.
\begin{lem}\label{thm:E.4}
	Suppose that
\[
\{(\xi_t)_{t\in \mathbb R}, (X^{\mathrm n, \sigma})_{\sigma\in \mathcal D^\mathrm n}, (X^{\mathrm m, \sigma})_{\sigma \in \mathcal D^\mathrm m}, (X_t)_{t\geq 0}; {\mathbb Q}\}
\]
	is a spine representation of $\{(X_t)_{t\geq 0}; \widetilde {\mathbb P}_\nu\}$.
	Let $(m_\sigma)_{\sigma\in \mathcal D^{\mathrm m}}$ be the $\mathbb R^+$-valued point process defined by
\[
	m_\sigma
	= X^{\mathrm m, \sigma}_0(\mathbf 1_E),
	\quad \sigma \in \mathcal D^{\mathrm m},
\]
	and define
\[
	\sigma_0=0,\quad \sigma_i=\sup\{s\in\mathcal D^{\mathrm m};\ s<\sigma_{i-1},\ 
	m_s\phi(\xi_s)>1\}, 
	\quad\eta_i=m_{\sigma_i},\quad i=1,2,\cdots.
\]

(1) If $\nu(l)<\infty$,
then for any $\varepsilon>0$,
\[
	\quad \mathbb Q\text{-a.s.}
\]

(2) If  $ \nu(l)=\infty$,
then for any $\epsilon>0$ and any $K>0$,
\begin{equation}\label{inteqinfty}
\int^0_{-\infty} {\mathrm d}t
\int_{K\phi(\xi_t)^{-1}e^{-\varepsilon t}}^\infty r\pi(\xi_t,{\mathrm d}r)
=\infty, 
\quad {\mathbb Q}\text{-a.s.}
\end{equation}
 and for any $\varepsilon>0$,
\[
	\limsup_{i\rightarrow\infty} e^{\varepsilon\sigma_i}\eta_i \phi(\xi_{\sigma_i})
	=\infty,
\quad {\mathbb Q}\text{-a.s.}
\footnote{I am not sure which result you are talk about, the display above is needed later.}
\footnote{YR: The following result is not needed, and can be deleted.}
\]
\end{lem}
\end{comment}

\section{Proofs of Main Results}
\begin{comment}
\subsection{Proof of Theorem \ref{thm:Q}}

First note that the operators $(V_t)_{t\geq 0}$
given by \eqref{eq:M.1} can be extended uniquely to  $\mathcal B(E,[0,\infty])$ such that
\begin{equation}
\begin{minipage}{0.8\textwidth}
	$\mathbb P_\mu [e^{-X_t(f)}] = e^{- \mu(V_tf)}$ for $t\geq 0$, $\mu \in \mathcal M_f(E)$, and $f\in \mathcal B(E,[0,\infty])$.
\end{minipage} \label{eq:BGD.2}
\end{equation}
Define $v(t,x) = V_t(\infty  \mathbf 1_E)(x)=\lim_{\theta \to \infty} V_t(\theta \mathbf 1_E)(x)$ for $t\geq 0, x\in E$, then it holds that
\begin{equation} \label{eq: v and extinction}
	\mathbb P_\mu(\tau_0 \leq t)=\mathbb P_\mu (\|X_t\| = 0)
	= e^{- \mu (v_t)},
	\quad \mu \in \mathcal M_f(E), t\geq 0.
\end{equation}

We now give some basic properties of $V_tf$. Put
\[
\psi_0(x,z)=\psi(x,z)+\beta(x)z,
\quad x\in E, z>0.
\]
For any $f\in\mathcal B(E,[0,\infty])$ and any $t\geq 0$, \eqref{eq:M.1} can be rewritten as
\begin{equation} \label{eq:BGD.3}
	V_{t+s}f(x) + \Big[\int_0^{s}P^{\beta}_{s-u} \psi_0(\cdot, V_{t+u} f(\cdot))(x) {\mathrm d}s\Big] = P^\beta_sV_tf(x) \quad s\geq 0, x\in E.
\end{equation}
Integrating both sides of \eqref{eq:BGD.3} with respect to $\nu$, we obtain
\begin{equation}\label{eq:BGD.4}
\langle V_{t+s}f, \nu\rangle + \Big[\int_0^{s}e^{\lambda(s-u)}\langle \psi_0(\cdot, V_{t+u} f(\cdot)),\nu\rangle {\mathrm d}s\Big] =e^{\lambda s}\langle V_tf,\nu\rangle.
\end{equation}
\end{comment}
The following lemma is \cite[Proposition 1.4]{LiuRenSongSun2020}.
\begin{lem}\label{lem:rate}
	Suppose that Assumptions \eqref{asp:H1}, \eqref{asp:H2} and \eqref{asp:H3} hold. 
Let $(V_t)_{t\ge0}$ be the extended operators defined above on  $\mathcal B(E,[0,\infty])$.
Then
\[
\lim_{t\to\infty}\sup_{x\in E, f\in \mathcal B(E,[0,\infty]) }\left|\dfrac{V_tf(x)}{\phi(x)\nu(V_tf)}-1\right|=0.
\]
\end{lem}
\begin{comment}

The results in the following two lemmas are actually proved in \cite{LiuRenSongSun2020}, but not stated as  separate results. For the reader's convenience, we state theses results as two lemmas and give outlines of their proofs.
\end{comment}

\begin{lem}\label{lem:extinc}
	Suppose that Assumptions \eqref{asp:H1}, \eqref{asp:H2} and \eqref{asp:H3} hold. 
\begin{enumerate}
\item	
	For any  $ t>0$ and $ \mu \in \mathcal M_f(E)$, $\langle v(t,\cdot),\mu\rangle <\infty.$ 
\item	For any $\mu \in \mathcal M_f(E)$,
\[
	\lim_{t\rightarrow\infty}\langle v(t,\cdot),\mu\rangle=0.
\]
\end{enumerate}
\end{lem}
\begin{comment}
\begin{proof} (1)
First note  that, by \eqref{eq:BGD.3},  $v(t,x)$ satisfies 
\begin{equation}
\label{eq: equation for vt}
	v(t+s,x) + \int_0^sP^\beta_{s-u}\Big(\psi_0\big(\cdot, v(t+u,\cdot)\big)\Big)(x)~{\mathrm d}u
	=P^\beta_s\big(v(t,\cdot)\big)(x),
	\quad t,s > 0, x\in E.
\end{equation}
According to \eqref{asp:H3}, we have $\langle v_t, \nu\rangle < \infty $ for $t>0$.
According to \eqref{asp:H2} and \eqref{eq: equation for vt},  for all $t >0, s>0$ and $x\in E$ we have
\begin{equation}
\label{upp}
v(t+s,x) \leq P^{\beta}_sv(t,x)
=e^{\lambda s}\phi(x)\langle v(t,\cdot),\nu\rangle(1+C^{(H2)}_{s, x, v(t, \cdot)}).
\end{equation}
Since 
\[
\sup_{x\in E, f\in L_1^+(\nu)} |C^{\eqref{asp:H2}}_{t,x,f}| 
	< \infty,
\]
it follows from \eqref{upp}\eqref{upp} that, for all $t >0, s>0$ and $\mu \in \mathcal M_f(E)$, we have
\begin{align}\label{e:l3.1}
\langle v(t +s,\cdot),\mu\rangle
	\leq  e^{\lambda s}\langle \phi,\mu\rangle(1+C^{(H2)}_{s, x, v(t, \cdot)})<\infty.
\end{align}
(2) 
Since $\lambda<0$ and 
\[
\lim_{t\to\infty}\sup_{x\in E, f\in L_1^+(\nu)} |C^{\eqref{asp:H2}}_{t,x,f}| 
	=0,
\]
the desired assertion follows immediately from \eqref{e:l3.1}.
\end{proof}
\end{comment}

\begin{lem}\label{lem:ratio limit}
(1) For any $f\in\mathcal B(E,[0,\infty])$ with $\nu(f)>0$, and any $s>0$,
\begin{equation}\label{integ ratio limit}
\lim_{t\to\infty}\dfrac{\langle V_{t+s}f, \nu\rangle}{\langle V_{t}f, \nu\rangle}=e^{\lambda s}.
\end{equation}
(2) There are constants $a,N,T>0$ such that
\begin{equation}\label{inequ:lower}
\langle V_{t}f, \nu\rangle\geq ae^{-Nt},\quad \mbox{for any }\ t>T.
\end{equation}
(3) For each $s\geq 0$,
\begin{equation} \label{one point ratio limit}
	\lim_{t\to \infty} \sup_{x\in E}\Big|\frac{v(t+s,x)}{\langle v(t,\cdot),\nu\rangle\phi(x) } - e^{\lambda s} \Big|=0.
\end{equation}
\end{lem}
\begin{comment}
\begin{proof}
(1)  Suppose $f\in\mathcal B(E,[0,\infty])$ with $\nu(f)>0$. 
For $t, s>0$, put
$$
H_t(s)=e^{-\lambda(s+t)}\nu(V_{t+s}f),\quad\mbox{ and }\quad 
L_t(s)=\dfrac{\langle \psi_0(\cdot, V_{t+s} f(\cdot)),\nu\rangle}{\langle V_{t+s} f,\nu\rangle}.
$$  
Then \eqref{eq:BGD.4} can be rewritten as
\[
H_t(s)=H_t(0)-\int_0^sL_t(u)H(u){\mathrm d}u,\quad s>0.
\]
Solving this ordinary differential equation (in $s$), we get
\begin{equation}\label{eq:ratio}
\dfrac{\langle V_{t+s}f, \nu\rangle }{e^{\lambda s}\langle V_{t}f, \nu\rangle }
=\dfrac{H_t(s)}{H_t(0)}
=\exp\left\{-\int_0^s L_t(u){\mathrm d}u \right\}.
\end{equation}
Note that
\[
\dfrac{\partial \psi_0(x,z)}{\partial z}(x,z)=2\sigma(x)^2z+\int_0^\infty \left(1-e^{-rz}\right)r\pi(x,{\mathrm d}r)
\]
is an increasing function in $z\in (0, \infty)$ and is bounded in $x\in E$.  It follows that, for all
$(x, z)\in E\times (0, \infty)$,
$\psi_0(x,z)\leq z\dfrac{\partial \psi_0(x,z)}{\partial z}$. Lemma \ref{lem:rate} implies that there is $T_0>0$ such that for $t>T_0$,
\begin{eqnarray*}
L_t(u)&=&\dfrac{\langle \psi_0(\cdot, V_{t+u} f(\cdot)),\nu\rangle}{\langle V_{t+u} f,\nu\rangle}\leq \dfrac{\sup_{x\in E}V_{t+u}f(x)}{\langle V_{t+u}f, \nu\rangle}\langle \dfrac{\partial \psi_0(x,z)}{\partial z}(\cdot,V_{t+u}f(\cdot)),\nu\rangle\\
&\leq& 2\|\phi\|_\infty \langle \dfrac{\partial \psi_0(x,z)}{\partial z}(\cdot,V_{t+u}f(\cdot)),\nu\rangle.
\end{eqnarray*}
By Lemma \ref{lem:extinc} (2),  $\lim_{t\to\infty} V_{t+u}f(x)=0$ for all $x\in E$. By the dominated convergence theorem, 
$$\lim_{t\to\infty}\int_0^s\frac{\partial \psi_0(x,z)}{\partial z}(\cdot,V_{t+u}f(\cdot)),\nu\rangle {\mathrm d}u=0.
$$ 
Therefore $\lim_{t\to\infty}\int_0^s L_t(u){\mathrm d}u=0$ and \eqref{integ ratio limit} follows from \eqref{eq:ratio}. 
 
 (2) Note that
\[
\dfrac{\partial \psi_0(x,z)}{\partial z}(x,z)\leq \left[2\sigma(x)^2+\int_0^1r^2\pi(x,{\mathrm d}r)\right]z+\int_1^\infty r\pi(x,{\mathrm d}r).
\]
By our assumption on $\psi$, there are $C_1, C_2>0$ such that $\dfrac{\partial \psi_0(x,z)}{\partial z}(x,z)\leq C_1z+C_2$.  According to Lemma \ref{lem:extinc}, there are  constants $T>0$ and  $C_3>0$ such that for  $t\geq T$, $V_{t+u}f\leq C_3$,  $\langle \dfrac{\partial \psi_0(x,z)}{\partial z}(\cdot,V_{t+u}f(\cdot)),\nu\rangle\leq C_1C_3+C_2$. Consequently,  there is $N>0$ such that  $L_t(u)\leq N$ for $u>0$ and $t\geq T$.  Thus by \eqref{eq:ratio},
\[
\langle V_{T+s}f, \nu\rangle\geq \langle V_{T}f, \nu\rangle e^{-Ns}=e^{NT}\langle V_{T}f, \nu\rangle e^{-N(s+T)},\quad s>0.
\]
Hence \eqref{inequ:lower} holds  for $a=e^{NT}\langle V_{T}f, \nu\rangle$.

 (3) Note that when $f\equiv\infty$, $V_tf(x)=v(t,x)$. \eqref{one point ratio limit} follows from \eqref{integ ratio limit} and Lemma \ref{lem:rate} with $f\equiv\infty$. 
\end{proof}


\begin{proof}[Proof of Theorem \ref{thm:Q}]
Assume that $t>0$ and $A\in\mathscr F_t$ are fixed. For $s>t$, by the Markov property of $X$,
\[
\mathbb P_\mu(A|\zeta>s)=\dfrac{\mathbb P_\mu(A, \zeta>s)}{\mathbb P_\mu(\zeta>s)}=\dfrac{\mathbb P_\mu\big(\mathbb P_{X_t}(\zeta>s-t);A\big)}{\mathbb P_\mu(\zeta>s)},
\]
It follows from  Lemma \ref{lem:ratio limit} that as 
$s\to\infty$,
 \[
 \mathbb P_x(\zeta>s)\sim v(s,x)\sim \phi(x)a(s)e^{\lambda s},
 \]
where $a(s)$ satisfies that for any $s>0$, $\lim_{r\rightarrow\infty}a(r+s)/a(r)=1$.
Thus we have
\begin{eqnarray*}
\lim_{s\rightarrow\infty}\dfrac{\mathbb P_{X_t}(\zeta>s-t)}{\mathbb P_\mu(\zeta>s)}
&=&\lim_{s\rightarrow\infty}\dfrac{1-e^{-\langle v(s-t,\cdot), X_t\rangle }}{1-e^{-\langle v(s,\cdot), \mu\rangle }}
=\lim_{s\rightarrow\infty}\dfrac{\langle v(s-t,\cdot), X_t\rangle }{\langle v(s,\cdot),\mu\rangle }\\
&=&\dfrac{e^{-\lambda t}\langle \phi, X_t\rangle }{\langle \phi, \mu\rangle }=\dfrac{M_t(\phi)}{\langle \phi, \mu\rangle }.
\end{eqnarray*}
We claim that for large $s$, $\dfrac{\mathbb P_{X_t}(\zeta>s-t)}{\mathbb P_\mu(\zeta>s)}$ can be dominated by a random variable which is integrable with respect to $\mathbb P_\mu$, then  by the dominated convergence theorem,
\[
\lim_{s\rightarrow\infty}\mathbb P_\mu(A|\zeta>s)=\mathbb P_\mu\left(\frac{M_t(\phi)}{\langle\phi,\mu\rangle };A\right)=\widetilde{\mathbb P}_\mu(A).
\]
Now we prove our claim above. From \eqref{upp} we see that there is a constant $\widetilde C>0$ such that for any $s>2T$ and $x\in E$,
\[
v(s,x)\leq \widetilde C\phi(x)e^{\lambda T}\langle v(s-T,\cdot),\nu\rangle .
\]
Using the fact that $\lim_{x\rightarrow 0+}(1-e^{-x})/x=1$, we can choose $s$ sufficiently large such that
\[
1-e^{-\langle v(s,\cdot),\mu\rangle }>\frac{1}{2}\langle v(s,\cdot),\mu\rangle .
\]
Since $1-e^{-x}\leq x$ for $x>0$, for  $s-t>2T$ we have
\[
\dfrac{1-e^{-\langle v(s-t,\cdot),X_t\rangle }}{1-e^{-\langle v(s,\cdot),\mu\rangle }}
\leq \dfrac{2\langle v(s-t,\cdot),X_t\rangle }{\langle v(s,\cdot),\mu\rangle }\leq \dfrac{2\widetilde C\langle \phi,X_t\rangle e^{\lambda T}\langle v(s-t-T,\cdot),\nu\rangle }{\langle v(s,\cdot),\mu\rangle }.
\]
By \eqref{one point ratio limit}, we have
\[
\lim_{s\rightarrow\infty}\dfrac{\langle v(s-t-T,\cdot),\nu\rangle }{\langle v(s,\cdot),\mu\rangle }
=e^{-\lambda(t+T)}\langle \phi,\mu\rangle ^{-1}.
\]
Noticing that $\langle \phi,X_t\rangle $ is integrable with respect to $\mathbb P_\mu$, our claim is true.
\end{proof}
\end{comment}

\begin{comment}
\subsection{Proof of Theorem \ref{thm:E}}
\begin{proof}
According to Lemma \ref{lem:extinc}, we only need to prove our result  
when the initial measure is $\nu$.
 It is easy to see that for any $t\ge 0,$
\begin{equation}%\label{eq:E.5}
 	e^{-\lambda t}\mathbb P_\nu(\zeta>t)\overset{\eqref{eq:M.3}}= \nu(\phi)\widetilde{\mathbb P}_\nu\left(\frac{1}{\langle\phi, X_{t}\rangle }\right)
=\mathbb Q\left[\frac{1}{\langle\phi, X_{t}\rangle +\int^0_{-t}w_{-s}(\phi)\mathbf N^\xi({\mathrm d}s, {\mathrm d}w)}\right],
\end{equation}
where $\widetilde {\mathbb P}_\nu$ is defined in \eqref{eq:M.3}.
		
Note that $\lim_{t\to\infty}\langle\phi, X_{t}\rangle=0$, $\mathbb Q$-a.s., and
$$
\int^0_{-t}w_{-s}(\phi)\mathbf N^\xi({\mathrm d}s, {\mathrm d}w)\uparrow
\int^0_{-\infty}w_{-s}(\phi)\mathbf N^\xi({\mathrm d}s, {\mathrm d}w)
\quad \mbox{as } t\uparrow\infty.
$$
\end{comment}
Also note that
 \begin{equation}\label{domi-Zt1}
\frac{1}{\langle\phi, X_t\rangle + \int^0_{-t}w_{-s}(\phi)\mathbf N^\xi({\mathrm d}s, {\mathrm d}w)}\leq \frac{1}{\int^0_{-1}w_{-s}(\phi)\mathbf N^\xi({\mathrm d}s, {\mathrm d}w)},\quad t\geq 1,
 \end{equation}
and
\begin{equation}%\label{eq:E.53}
\mathbb Q\left(\frac{1}{\int^0_{-1}w_{-s}(\phi)\mathbf N^\xi({\mathrm d}s, {\mathrm d}w)}\right)= \widetilde {\mathbb N}_{\phi\cdot\nu} [w_1(\phi)^{-1}]
= \mathbb N_{\phi\cdot\nu} (w_1\neq \mathbf 0) < \infty.
    \end{equation}
 \begin{comment}
By \eqref{eq:E.5}  and the dominated  convergence theorem, we obtain
\begin{equation}\label{eq:E.55}
\lim_{t\to\infty} e^{-\lambda t}\mathbb P_{\nu}(\tau_0>t)=\mathbb Q\left(\frac{1}{\int^0_{-\infty}w_{-s}(\phi)\mathbf N^\xi({\mathrm d}s, {\mathrm d}w)}\right):=k<\infty.
\end{equation}

Now we try to find an equivalent condition for $k>0$.
For the continuum immigration part, we have
\[
\mathbb Q\left(Z^{\mathrm n}_\infty \right)
=\mathbb Q\left(\int^0_{-\infty}w_{-s}(\phi)\mathbf N^\xi({\mathrm d}s, {\mathrm d}w) \right)
=\int^0_{-\infty} e^{-\lambda s}\nu(2\sigma^2\phi^2)  \mathrm ds=\frac{\nu(2\sigma^2\phi^2)}{-\lambda}<\infty.
\]
Therefore, 
$Z^{\mathrm n}_\infty$ is finite almost surely.

For the discrete immigration part, let 
$\mathcal G=\sigma((\xi_t)_{t\in\mathbb R}, ( m_\sigma)_{\sigma\in\mathcal D^{\mathrm m}})$.
When  $\nu(l)<\infty$,  by Lemma \ref{thm:E.4} (1),
\[
\mathbb Q\left(Z^{\mathrm m}_\infty\Big|\mathcal G \right)
=\sum_{-\infty<\sigma\le 0}m_\sigma e^{\lambda \sigma}\phi(\xi_{\sigma})<\infty,  \qquad\qquad \mathbb Q-{\mathrm a.s.}
\]
Thus in this case, $k>0$. We claim that when
$\nu(l)=\infty$,
\begin{equation}\label{Z-infty}
\lim_{t\rightarrow\infty}\int^0_{-\infty}w_{-s}(\phi)\mathbf N^\xi({\mathrm d}s, {\mathrm d}w)
=\infty,\quad\qquad \mathbb Q-{\mathrm a.s.},
\end{equation} which implies that $k=0$ by \eqref{eq:E.55}.
Now we prove the above claim. By Lemma \ref{thm:E.4} (2), for any $N>0$,
\begin{equation}\label{inf}
\int^0_{-\infty} dt\int_{\phi(\xi_t)^{-1}e^{-Nt}}^\infty r\pi(\xi_t, \mathrm dr)
=\infty,\quad \widehat{\mathbb Q}_\nu-{\mathrm a.s.}
\end{equation}
Fix a path of $(\xi_t)_{t\in \mathbb R}$ and define
\[
\tau_1:=\sup\left\{t<0; m_t>\phi(\xi_t)^{-1}e^{-Nt}\right\},\,
\tau_{i+1}:=\sup\left\{t<\tau_i;\, m_t>\phi(\xi_t)^{-1}e^{-Nt}\right\},\, i=1,2,\cdots
\]
Then $\tau_i<\infty$, $i=1,2,\ldots$ almost surely from Lemma \ref{thm:E.4}.
If we can prove that 
$\sum_{i=1}^\infty I_{\left\{\langle\phi, X^{{\mathrm m},\tau_i}_{-\tau_i}\rangle  >\varepsilon\right\}}=\infty$ for some $\varepsilon>0$, then our claim holds.
Since for any $T>0$, 
conditional on $\sigma((\xi_t)_{t\in \mathbb R})$, $\sum_{\tau_i\leq T} I_{\left\{\langle\phi, X^{{\mathrm m},\tau_i}_{-\tau_i}\rangle  >\varepsilon\right\}}$ is a Poisson random variable with parameter
\begin{equation}
\mathbb Q
\left[\left.\sum_{\tau_i\geq -T} I_{\left\{\langle\phi, X^{{\mathrm m},\tau_i}_{-\tau_i}\rangle  >\varepsilon\right\}} \right| (\xi_t)_{t\in \mathbb R}\right]=\int^0_{-T} \mathrm dt\int_{\phi(\xi_t)^{-1}e^{-Nt}}^\infty r\pi(\xi_t, \mathrm dr)\mathbb{P}_{r\delta_{\xi_t}}\big(\langle\phi, X_{-t} \rangle >\varepsilon\big),
\end{equation}
we only need to prove, for some $\varepsilon>0$,
\begin{equation}\label{infty-sum}
\int^0_{-\infty} \mathrm dt\int_{\phi(\xi_t)^{-1}e^{-Nt}}^\infty r\pi(\xi_t, \mathrm dr)\mathbb{P}_{r\delta_{\xi_t}}\big(\langle\phi, X_{-t} \rangle >\varepsilon\big)=\infty \quad \widetilde \Pi_{\phi\cdot\nu}{\mathrm -a.s.}
\end{equation}
If we can prove for some $\varepsilon>0$, there exist $t_0>0$ and $\delta>0$ such that 
when $t<-t_0$,
\begin{equation}\label{last point}
\inf_{r\geq \phi(x)^{-1}e^{-Nt}, x\in E}\mathbb P_{r\delta_x}\big(\langle\phi, X_{-t}
\rangle >\varepsilon\big)>\delta,
\end{equation}
then from \eqref{inf}, \eqref{infty-sum} is obtained.  By Chebyshev's inequality, for any $\varepsilon>0$,
\begin{align}
&\mathbb P_{r\delta_x}\big(\langle\phi, X_{-t}\rangle >\varepsilon\big)=\mathbb P_{r\delta_x}\left(e^{-\langle\phi, X_{-t}\rangle }<e^{-\varepsilon}\right)\\
=&1-\mathbb P_{r\delta_x}\left(e^{-\langle\phi, X_{-t}
\rangle }\geq e^{-\varepsilon}\right)\geq 1-e^{\varepsilon }\mathbb P_{r\delta_x}e^{-\langle\phi, X_{-t}\rangle }\\
	=&1-e^{\varepsilon }e^{-rV_{-t}\phi(x)},\label{Cheby}
\end{align}
According to Lemma \ref{lem:rate} and \eqref{inequ:lower}, 
when $-t$ is sufficiently large, we have
\[
V_{-t}\phi(x)\ge\frac{1}{2}\phi(x)\nu(V_{-t}\phi)\geq \dfrac{a}{2}\phi(x)e^{Nt},\quad x\in E.
\]
Therefore, using \eqref{Cheby}, we obtain that, when $-t$ is sufficiently large,
\begin{eqnarray*}
  &&\inf_{r\geq \phi(x)^{-1}e^{-Nt}, x\in E}\mathbb P_{r\delta_x}\big(\langle\phi, X_{-t} \rangle >\varepsilon\big)\geq \inf_{r\geq \phi(x)^{-1}e^{-Nt}, x\in E} \left(1-e^{\varepsilon}e^{-rV_{-t}\phi(x)}\right)\\ 
&&     \geq 1-e^{\varepsilon-a/2}.
\end{eqnarray*}
Then we get \eqref{last point} by	choosing $\varepsilon\in(0, a/2)$ and $\delta=1-e^{\varepsilon-a/2}$.  Then
\eqref{Z-infty} follows, and thus $k=0$. The proof is finished.
\end{proof}
\end{comment}

\subsection{Proof of Theorem \ref{thm:L}}
We first state a result proved in \cite{LiuRenSongSun2020}. For $f\in \mathcal B(E, [0,\infty])$, Put
$$
\Gamma_t f:=-\log \mathbb P_{\nu}[e^{-X_t(f)}|X_t(1)>0].
$$
We say a $[0,\infty]$-valued functional $A$ defined on $\mathcal B(E,[0,\infty])$ is monotone concave if
	(1) $A$ is a monotone functional, i.e., $f\leq g$ in $\mathcal B(E,[0,\infty])$ implies $Af \leq Ag$; and
	(2) for any $f\in \mathcal B(E,[0,\infty])$ with $Af< \infty$, the function $u \mapsto A(uf)$ is concave on $[0,1]$.
The following result is \cite[Proposition 1.5, Proposition 1.9]{LiuRenSongSun2020}.

\begin{lem} \label{prop:G}
(1) The limit $Gf:= \lim_{t\to \infty} \Gamma_t f$ exists in $[0,\infty]$ for each $f\in \mathcal B(E,[0,\infty])$.
	Moreover, $G$ is the unique $[0,\infty]$-valued monotone concave functional on $\mathcal B(E,[0,\infty])$ such that
	$G(\infty  \mathbf 1_E) = \infty$ and that
\begin{equation} \label{eq:G.0}
	1 - e^{- GV_s f}
	= e^{s\lambda} (1 - e^{-Gf}),
	\quad s\geq 0, f\in \mathcal B(E,[0,\infty]).
\end{equation}

(2) $G$ is the log-Laplace functional of $\mathbf Q_\lambda$, and for any $r\in[\lambda, 0)$,
$L_\alpha(f):=1-\left(1-e^{-G(f)}\right)^\alpha, f\in\mathcal B(E,[0,\infty)),$
is the Laplace functional of $\mathbf Q_r$, where $\alpha=r/\lambda\in(0,1]$.
\end{lem}


\begin{proof}[Proof of Theorem \ref{thm:L}]
(1) Using \eqref{one point ratio limit} with $s=0$, we know that, for any $\varepsilon>0$, there is some $T_1>0$ such that when $t>T_1$,
\[
(1-\varepsilon)\langle v(t),\nu\rangle \phi(x)\leq v(t,x)\leq (1+\varepsilon)\langle v(t),\nu\rangle \phi(x),\qquad x\in E.
\]
By Lemma \ref{prop:G},  $1-e^{-G(f)}$ is a non-decreasing function with respect to $f\in \mathcal B(E,[0,\infty])$, and $1-e^{-G(uf)}$ is a concave function with respect to $u\in(0,1)$.
On one hand, we obtain that for $t>T_1$,
\begin{eqnarray}\label{lower}
\dfrac{1-e^{-G(u\langle v(t),\nu\rangle \phi)}}{1-e^{-G(\langle v(t),\nu\rangle \phi)}}&\overset{\text{monotonicity}}\geq& \dfrac{\left(1-e^{-G(\frac{u}{1+\varepsilon}v(t))}\right)}{1-e^{-G(\frac{1}{1-\varepsilon} v(t))}}
\overset{\text{concavity}}\geq \dfrac{\dfrac{u}{1+\varepsilon}\left(1-e^{-G(v(t))}\right)}{1-e^{-G(\frac{1}{1-\varepsilon} v(t))}}
\\
&=&\dfrac{u}{1+\varepsilon}\left[e^{-\lambda t}\left(1-e^{-G(\frac{1}{1-\varepsilon} v(t))}\right)\right]^{-1}\nonumber\\
&\overset{\text{concavity}}\geq& \dfrac{u(1-\varepsilon)}{1+\varepsilon}\left[e^{-\lambda t}\left(1-e^{-G(v(t))}\right)\right]^{-1}=\dfrac{u(1-\varepsilon)}{1+\varepsilon},
\end{eqnarray}
where in the last inequality, we used \eqref{eq:G.0} with $f=\infty I_{E}.$
On the other hand, without lose of generality, we choose $\varepsilon>0$ such that $u/(1-\varepsilon)<1$, then
\begin{eqnarray}\label{upper}
\dfrac{1-e^{-G(u\langle v(t),\nu\rangle \phi)}}{1-e^{-G(\langle v(t),\nu\rangle \phi)}}&\overset{\text{monotonicity}}\leq &\dfrac{\left(1-e^{-G(\frac{u}{1-\epsilon}v(t))}\right)}{1-e^{-G(\frac{1}{1+\varepsilon} v(t))}}\overset{\text{concavity}}\leq \dfrac{\dfrac{u}{1-\varepsilon}\left(1-e^{-G(v(t))}\right)}{1-e^{-G(\frac{1}{1+\varepsilon} v(t))}}\\
&=&\dfrac{u}{1-\varepsilon}\left[e^{-\lambda t}\left(1-e^{-G(\frac{1}{1+\varepsilon} v(t))}\right)\right]^{-1}\\
 &\overset{\text{concavity}}\leq& \dfrac{u(1+\varepsilon)}{1-\varepsilon}\left[e^{-\lambda t}\left(1-e^{-G(v(t))}\right)\right]^{-1}
=\dfrac{u(1+\varepsilon)}{1-\varepsilon}.
\end{eqnarray}
Combining the two inequalities above, it follows that
\[
\lim_{t\to\infty}\dfrac{1-e^{-G(u\langle v(t),\nu\rangle \phi)}}{1-e^{-G(\langle v(t),\nu\rangle \phi)}}=u,\qquad u\in (0,1).
\]
And therefore, by Lemma \ref{lem:regu} in the Appendix, 
\begin{equation}\label{eq regu}
1-e^{-G(u\phi)}\sim uL(u),\quad u\rightarrow 0+,
\end{equation}
where $L$ is slowly varying at $0$.

Note that for any $u>0$, we have
\begin{align}
\int_0^\infty e^{-us}{\mathbf Q}_\lambda(\{\mu: \mu(\phi)>s\})ds=&-\frac{1}{u}\int^\infty_0{\mathbf Q}_\lambda(\{\mu: \mu(\phi)>s\})d(e^{-us})\\
=&\frac{1}{u}{\mathbf Q}_\lambda(\{\mu: \mu(\phi)>0\})-\frac{1}{u}\int^\infty_0e^{-us}d({\mathbf Q}_\lambda(\{\mu: \mu(\phi)\le s\}))\\
=&\dfrac{1-e^{-G(u\phi)}}{u}.
\end{align}
It follows from  \eqref{eq regu} and Lemma \ref{lem: tau} that there is a function $L_1$, which is a slowly varying at infinity, such that
$$\int^x_0{\mathbf Q}_\lambda(\mu(\phi)>s)ds\sim L_1(x),\quad x\to\infty.$$
Then by Lemma \ref{lem:tail},
$$s{\mathbf Q}_\lambda(\{\mu: \mu(\phi)>s\})=o(L_1(s))\qquad s\to\infty.$$
Therefore, for $r\in[\lambda, 0)$,
\begin{equation}
{\mathbf Q}_r(\{\mu: \mu(\phi)>s\})=o(s^{-\alpha}L_1(s)),\qquad s\to\infty,
\end{equation}
where $\alpha=\lambda/r.$
Thus for any $0<\gamma<\alpha$,
\[
{\mathbf Q}_r\left(\mu(\phi)^{\gamma}\right)=\int_{{\mathcal M}_f(E)}\mu(\phi)^\gamma\mathbf Q_r(d\mu)<\infty.
\]

(2) For any $f\in\mathcal B_b^+(E)$ and $t>0$, using the definition of $\widetilde{\mathbb P}_\nu$, we obtain
\begin{eqnarray*}
&&\mathbb P_\nu\left(\exp\{-\langle f, X_t\rangle \};\zeta>t\right)
=\mathbb P_\nu\left(\dfrac{M_t(\phi)}{M_t(\phi)}\exp\{-\langle f, X_t\rangle \};\zeta>t\right)\\
&&=\widetilde{\mathbb P}_\nu\left(\dfrac{1}{M_t(\phi)}\exp\{-\langle f, X_t\rangle \}\right)
=e^{\lambda t}\mathbb Q_{\nu}\left(\dfrac{\exp\Big\{-\langle f, X_t\rangle -\langle f,  Z_t\rangle\Big \}}{\langle\phi, X_t\rangle +\langle\phi,  Z_t\rangle }\right)\\
&&=e^{\lambda t}\widehat{\mathbb Q}_{\nu}\left(\dfrac{\exp\Big\{-\langle f, X_t\rangle -\langle f,  \widehat Z_t\rangle\Big \}}{\langle\phi, X_t\rangle +\langle\phi,  \widehat Z_t\rangle }
\right).
\end{eqnarray*}
Since $\lim_{t\rightarrow\infty}X_t=0$ in probability with respect to $\widehat{\mathbb Q}_{\nu}$,  using domination \eqref{domi-Zt1} and \eqref{eq:E.53}, we have
\[
\lim_{t\to\infty}e^{-\lambda t}\mathbb P_\nu\left(\exp\{-\langle f, X_t\rangle \};\zeta>t\right)=\widehat{\mathbb Q}_{\nu}\left(\dfrac{\exp\left\{-\langle f,\widehat Z_\infty\rangle \right\}}{\langle \phi,\widehat Z_\infty\rangle}\right),
\]
where $\widehat Z_\infty:=\sum_{\sigma\in\mathcal D^{\mathrm m}}\widehat X^{{\mathrm m},\sigma}_\sigma+\sum_{\sigma\in\mathcal D^{\mathrm n}}\widehat X^{{\mathrm n},\sigma}_\sigma$.  Note that for the continuum immigration part,
\[
\widehat{\mathbb Q}_{\nu}\big(\sum_{\tau\in \mathcal D^{\mathrm n}}\langle f, \widehat X_{\tau}^{{\mathrm n},\tau} \rangle \big)
=\int_0^\infty2\langle \sigma^2 P^{\beta}_sf,\phi\nu\rangle ds
\leq 2\|\sigma^2\phi\|_\infty\dfrac{\langle f,\nu\rangle }{-\lambda}<\infty.
\]

Now we discuss the discrete immigration part.


(i) If  $\int_E l(x) \nu(dx)<\infty$,
 by Lemma \ref{thm:E.4} (1),
\begin{eqnarray*}
&&\widehat{\mathbb Q}_{\nu}\Big(\sum_{\sigma\in [1,\infty)\bigcap\mathcal D^{\mathrm m}}\langle f, \widehat X_{\sigma}^{{\mathrm m},\sigma} \rangle\Big|\mathcal G \Big)
=\sum_{t\in [1,\infty)\bigcap\mathcal D^{\mathrm m}}m_tP^{\beta}_tf( Y_t)\\
&&\leq \sum_{t\in \mathcal D^{\mathrm m}}(1+ce^{-\rho t})m_te^{\lambda t}
\phi(Y_t)
\int_E\widehat\phi(y)f(y)m(dy)\\
&&=\langle f,\nu\rangle
\sum_{t\in \mathcal D^{\mathrm m}}(1+ce^{-\rho t})m_te^{\lambda t}\phi(Y_t)<\infty,\quad \widehat{\mathbb Q}_{\nu}-{\mathrm a.s.}
\end{eqnarray*}
 Thus in this case, the limit measure $\widehat Z_\infty\in \mathcal M_f(E)$.  Denote the distribution of $\widehat Z_\infty$ under
  $\widehat{\mathbb Q}_{\nu}$ by $\mathbf Q$, which is also the limit distribution of $X_t$ under $\widetilde{\mathbb P}_\nu$ as $t\to\infty$.  Then when
$\int_E l(x)\nu(dx)<\infty$,
\[
\lim_{t\rightarrow\infty}e^{-\lambda t}\mathbb P_\nu\left(\exp\{-\langle f, X_t\rangle \};\zeta>t\right)=
\int_{{\mathcal M}_f(E)}\frac{1}{\mu(\phi)}e^{-\mu(f)}\mathbf Q(d\mu).
\]
In Theorem \ref{thm:E} we have shown that
\[
\lim_{t\rightarrow\infty}e^{-\lambda t}\mathbb P_\nu(\zeta>t)=k
=\int_{{\mathcal M}_f(E)}\frac{1}{\mu(\phi)}\mathbf Q(d\mu)<\infty.
\]
Thus by the definition of the Yaglom distribution ${\mathbf Q}_\lambda $,
\begin{eqnarray*}
\mathbf Q_\lambda(\exp\{-\langle f, X_t\rangle \})&=&\lim_{t\rightarrow\infty}\mathbb P_\nu\left(\exp\{-\langle f, X_t\rangle \}\Big|\zeta>t\right)=\lim_{t\rightarrow\infty}\dfrac{\mathbb P_\nu\left(\exp\{-\langle f, X_t\rangle \};\zeta>t\right)}{\mathbb P_\nu(\zeta>t)}\\
&=&\dfrac{\lim_{t\rightarrow\infty}e^{-\lambda t}\mathbb P_\nu\left(\exp\{-\langle f, X_t\rangle \};\zeta>t\right)}{\lim_{t\rightarrow\infty}e^{-\lambda t}\mathbb P_\nu(\zeta>t)}\\
&=&\dfrac{\int_{{\mathcal M}_f(E)}\mu(\phi)^{-1}e^{-\mu(f)}\mathbf Q(d\mu)}{\int_{{\mathcal M}_f(E)}\mu(\phi)^{-1}\mathbf Q(d\mu)},
\end{eqnarray*}
which says that the Yaglom distribution ${\mathbf Q}_\lambda$ can be written as
\begin{equation}\label{rep: yaglom}
\mathbf Q_\lambda(\cdot)=\dfrac{1}{k}{\mathbf Q}\left(\dfrac{1}{\mu(\phi)}; \mu\in\cdot\right).
\end{equation}
Consequently
\begin{equation}\label{ident: k}
\mathbf Q_\lambda(\mu(\phi))=\dfrac{1}{k}{\mathbf Q}\left(\dfrac{\mu(\phi)}{\mu(\phi) }\right)=\dfrac{1}{k}<\infty.
\end{equation}


(ii) If $\int_El(x)\nu(dx)=\infty$,
by Theorem \ref{thm:E}, $\lim_{t\to\infty}e^{-\lambda t}\mathbb P_\nu(\zeta>t)=k=0$, which implies that
 $\lim_{t\to\infty}e^{-\lambda t}\langle v(t),\nu\rangle=0$.  Since for any $s>0$,  $1-e^{-s}\leq s$, we have
$$G(\langle v(t),\nu\rangle\phi)\geq 1-e^{-G(\langle v(t),\nu\rangle\phi)}.$$
Also note that $\langle v(t),\nu\rangle\mathbf Q_\lambda(\mu(\phi))=G(\langle v(t),\nu\rangle\phi).$
Then  for $t>T_0$,
\[
\mathbf Q_\lambda(\mu(\phi))\geq \dfrac{1-e^{-G(\langle v(t),\nu\rangle\phi)}}{\langle v(t),\nu\rangle}.
\]
 Using \eqref{one point ratio limit} with $s=1$,
 there is some $T_0>0$ such that for $t>T_0$, $v(t+1,x)\leq 2\langle v(t),\nu\rangle\phi(x)$, $x\in E$.
 Therefore
 \[
\mathbf Q_\lambda(\mu(\phi))\geq \dfrac{1-e^{-G(\frac{1}{2}v(t+1,x))}}{\langle v(t),\nu\rangle}\overset{\text{monotonicity}}\geq
\dfrac{1-e^{-G(v(t+1,x))}}{2\langle v(t),\nu\rangle}
=\dfrac{e^{\lambda(t+1)}}{2\langle v(t),\nu\rangle}\to\infty, \quad {\rm as}\, t\to\infty,
\]
which implies that $\mathbf Q_\lambda(\mu(\phi))=\infty$.
\end{proof}

\subsection{Proof of Theorem \ref{thm:I}}
\begin{proof}
According to the spine decomposition of $\{(X_t)_{t\geq 0}, \widetilde{\mathbb P}_\mu\}$  given by
\eqref{spine-decom2}, for any
$f\in\mathcal B_b(E,[0,\infty))$,
\[
\widetilde {\mathbb P}_{\mu}\left(e^{-\langle f, X_t\rangle }\right)=\mathbb Q_{\mu}\left(e^{-\langle f, X_t\rangle+\langle f, Z^{{\mathrm m},[0,t)}_t+Z^{{\mathrm n},[0,t)}_t\rangle }\right).
\]

(1) Suppose $\int_El(x)\nu(dx)<\infty$.
When $\mu(dx)=\nu(dx)=\hat\phi(x)m(dx)$, it has been shown in the proof of Theorem \ref{thm:L} that the distribution of
$X_t$ under $\widetilde {\mathbb P}_{\nu}$ converges weakly to $\mathbf Q$ as $t\to\infty$.
It is also shown in \eqref{rep: yaglom} that $\mathbf Q$ is related to the Yaglom distribution:
\begin{equation}\label{eq:2}
\mathbf Q_\lambda(\cdot)=\dfrac{1}{k}{\mathbf Q}\left(\dfrac{1}{\mu(\phi) }; \mu\in\cdot\right).
\end{equation}
Note that $k=[\mathbf Q_\lambda(\mu(\phi))]^{-1}$. \eqref{eq:2} can be rewritten as
\begin{equation}\label{eq size bias}
{\mathbf Q}\left(\mu\in\cdot\right)=\dfrac{\mathbf Q_\lambda(\mu(\phi); \mu\in \cdot)}{\mathbf Q_\lambda(\mu(\phi))}.
\end{equation}
We are left to prove for all $\mu\in\mathcal M_f(E)\backslash\{0\}$, the distribution of $X_t$ under $\widetilde {\mathbb P}_{\mu}$ converges weakly to $\mathbf Q$.  We use three steps to finish it.

{\bf Step 1}\quad For $f\in\mathcal B_b(E,[0,\infty))$, define
\begin{equation}\label{def: H}
H(x,t):={\mathbb Q}_x\left(e^{-\langle f, Z_{t}\rangle }\right)={\mathbb Q}_x\left(e^{-\langle f, Z^{\mathrm n, [0,t)}_{t} + Z^{\mathrm m, [0,t)}_{t}\rangle }\right),\quad x\in E.
\end{equation}
Set $\overline \eta(x):=\limsup_{t\to\infty}H(x,t), x\in E$.
In this step we prove that $\overline \eta(\cdot)$ is a constant function.

Define  $\mathcal{H}_t=\sigma\big(Y_s; s\leq t\big)$, $t\geq 0$, which is  the filtration generated by the spine process.  Then for $T,t>0$,
\begin{equation}\label{subcritical upper bound}
 \begin{aligned}
 &H(x,t+T)\\
 =&\mathbb Q_{x}\mathbb Q_{x}\Big[\exp\Big\{-\sum_{\sigma\in (0, t+T]\bigcap \mathcal D^{\mathrm m}}\langle f, X_{t+T-\sigma}^{{\mathrm m},\sigma}\rangle -\sum_{\tau\in (0, t+T]\bigcap \mathcal D^{\mathrm n}}\langle f, X_{t+T-\tau}^{{\mathrm n}, \tau}\rangle \Big\}\Big| \mathcal H_t\Big]\\
 \leq&\widetilde\Pi_x\mathbb Q_{x}\Big[\exp\Big\{-\sum_{\sigma\in (t, t+T]\bigcap \mathcal D^{\mathrm m}}\langle f, X_{t+T-\sigma}^{{\mathrm m},\sigma}\rangle -\sum_{\tau\in (t, t+T]\bigcap \mathcal D^{\mathrm n}}\langle f, X_{t+T-\tau}^{{\mathrm n}, \tau}\rangle \Big\}\Big| \mathcal H_t\Big]\\
 =&
   \widetilde\Pi_x\mathbb Q_{Y_t}\Big[\exp\Big\{-\sum_{\sigma\in (0, T]\bigcap \mathcal D^{\mathrm m}}\langle f, X_{T-\sigma}^{{\mathrm m},\sigma}\rangle -\sum_{\tau\in (0, T]\bigcap \mathcal D^{\mathrm n}}\langle f, X_{T-\tau}^{{\mathrm n}, \tau}\rangle \Big\}\Big]\\
 =&\widetilde\Pi_x\left[ H(Y_t, T)\right].
 \end{aligned}
 \end{equation}
 From \eqref{IU}, there are some constants $c,\rho>0$ such that when $t>1$,
\[
 H(x,t+T)\leq \widetilde\Pi_x\left[ H(Y_t, T)\right]
\leq (1+ce^{-\rho t})\int_E\phi(y)\widehat\phi(y)H(y,T)m(dy)<\infty.
 \]
For fixed  $T>0$, letting $t\to \infty$ in inequality \eqref{subcritical upper bound}, we obtain that
\begin{equation}\label{sub super}
\overline\eta(x)\leq \int_E\phi(y)\widehat \phi(y)H(y,T)m(dy).
\end{equation}
   Using Fatou's lemma, for any $x\in E$,
\begin{equation}\label{sup inequality}
\overline\eta(x)\leq
\limsup_{T\rightarrow\infty}\int_E\phi(y)\widehat \phi(y)H(y,T)m(dy)
\leq \int_E\phi(y)\widehat\phi(y)\overline{\eta}(y)m(dy).
\end{equation}
Then using the fact that  $\overline{\eta}(\cdot)\leq 1$, $\overline\eta(\cdot)$ is a constant function.

{\bf Step 2}\quad
 Denote $\overline\eta(\cdot)$ by $q(f)$.  In this step we prove that
 \begin{equation}\label{limit-H}
 \lim_{t\rightarrow\infty}H(x,t)=q(f),\qquad \mbox{for all}\,\, x\in E.
 \end{equation}
If $q(f)\equiv 0,$ then the above is true obviously. So in the
following, we assume $q(f)>0$.

We first claim that  as $T\to\infty$, $H(\cdot,T)$ converges to $q(f)$ in probability under probability
$\phi(x)\widehat{\phi}(x)m(dx)$. In fact,
 for any $\varepsilon_1>0$, let
$$
\mu_1(T)=\int_{\{x\in E;H(x,T)>(1+\varepsilon_1)q(f)\}}
\phi(x)\widehat\phi(x)m(dx).
$$
Then $\limsup_{T\to\infty}H(x,T)=q(f)$ implies that $\lim_{T\rightarrow\infty}\mu_1(T)=0.$  For any $\varepsilon_2>0$, let
$$
\mu_2(T)=\int_{\{x\in E;H(x,T)<(1-\varepsilon_2)q(f)\}}
\phi(x)\widehat\phi(x)m(dx).
$$
To prove the claim, we only need to prove that  $\limsup_{T\rightarrow\infty}\mu_2(T)=0.$
 We deduce from \eqref{sup inequality} that
\begin{eqnarray}\label{sublimitinprob}
q(f)&\leq&
(1-\varepsilon_2)q(f)\mu_2(T)+\mu_1(T)+(1+\varepsilon_1)q(f)(1-\mu_1(T)-\mu_2(T))\\
&\le
&(1+\varepsilon_1)q(f)-(\varepsilon_1+\varepsilon_2)q(f)\mu_2(T)+\mu_1(T).
\end{eqnarray}
 Hence
\begin{eqnarray*}\label{sublimitinequl}
q(f)&\leq&
\liminf_{T\rightarrow\infty}\left[(1+\varepsilon_1)q(f)-(\varepsilon_1+\varepsilon_2)\mu_2(T)+C\mu_1(T)\right]\\
&=&(1+\varepsilon_1)q(f)-(\varepsilon_1+\varepsilon_2)q(f)\limsup_{T\rightarrow\infty}\mu_2(T).
\end{eqnarray*}
Since $\varepsilon_1$ is an arbitrary positive constant.
\[
q(f)\leq q(f)-\varepsilon_2 q(f)\limsup_{T\rightarrow\infty}\mu_2(T).
\]
This is impossible unless $\limsup_{T\rightarrow\infty}\mu_2(T)=0.$



Meanwhile, from the definition of $H$ given by
\eqref{def: H}, we get the following domination:
\begin{equation}\label{subsub}
\begin{aligned}
     H(x,t+T)\geq& \mathbb Q_{x}\prod_{\sigma\leq t}I_{\{ X_{t+T-\sigma}^{{\mathrm m},\sigma}=0\}}\prod_{\tau\leq t}I_{\{ X_{t+T-\tau}^{{\mathrm n},\tau}=0\}}\\
&\cdot\mathbb Q_{Y_t}\Big[\exp\Big\{-\sum_{\sigma\in (0, T]\bigcap \mathcal D^{\mathrm m}}\langle f, X_{T-\sigma}^{{\mathrm m},\sigma}\rangle -\sum_{\tau\in (0, T]\bigcap \mathcal D^{\mathrm n}}\langle f, X_{T-\tau}^{{\mathrm n},\tau}\rangle \Big\}\Big]\\
=& \mathbb Q_{x}\left[\prod_{\sigma\leq t}I_{\{ X_{t+T-\sigma}^{{\mathrm m},\sigma}=0\}}\prod_{\tau\leq t}I_{\{ X_{t+T-\tau}^{{\mathrm n},\tau}=0\}}H(Y_t, T)\right].
\end{aligned}
\end{equation}
Note that
\begin{eqnarray*}
\mathbb Q_{x}\left(\prod_{\sigma\leq t}I_{\{ X_{t+T-\sigma}^{{\mathrm m},\sigma}=0\}}=1\right)
=\widetilde\Pi_x\exp\left\{-\int_0^tds\int_0^\infty r(1-\mathbb P_{r\delta_{Y_s}}(\zeta<T+t-s))\pi(Y_s,dr)\right\}.
\end{eqnarray*}
Lemma \ref{lem:extinc} (2) tells us the $(Y,\psi)$-superprocess starting from any finite measure is extinct in finite time.  Therefore
 by dominated convergence theorem,
\begin{equation}\label{1infty limit}
\lim_{T\rightarrow\infty}\int_0^tds\int_1^\infty r(1-\mathbb P_{r\delta_{Y_s}}(\zeta<T+t-s))\pi(Y_s,dr)=0,\quad \widetilde\Pi_x-\mbox{a.s.}
\end{equation}
  Note that
\[
1-\mathbb P_{r\delta_{Y_s}}(\zeta<T+t-s)\leq 1-(1-\mathbb P_{Y_s}(\zeta>T))^r.
\]
By Lemma \ref{lem:extinc}, there are $T_0>0$ and $\eta>0$ such that when $T>T_0$, for any $x\in E$,
\[
\mathbb P_x(\zeta>T)\leq \eta \phi(x)e^{\lambda T}.
\]
Since when $x\rightarrow 0+$, $1-(1-x)^r\sim rx$ for any $r>0$, we may take $T_0$  sufficiently large such that
$\eta \phi(x)e^{\lambda T}$ is small enough so that $1-(1-\mathbb P_{Y_s}(\zeta>T))^r\leq 2r\eta \phi(Y_s)e^{\lambda T}$ for all  $T>T_0$ and $r\in(0,1]$.
Therefore,
\[
\int_0^tds\int_0^1 r(1-\mathbb P_{r\delta_{ Y_s}}(\zeta<T+t-s))\pi(Y_s,dr)\leq 2\eta e^{\lambda T}\int_0^t\phi(Y_s)ds\int_0^1 r^2 \pi(Y_s,dr).
\]
Then by the dominated convergence theorem,
\begin{equation}\label{01limit}
\lim_{T\rightarrow\infty}\int_0^tds\int_0^1 r(1-\mathbb P_{r\delta_{Y_s}}(\zeta<T+t-s))\pi(Y_s,dr)=0, \quad \widetilde\Pi_x-\mbox{a.s.}
\end{equation}
 Combining \eqref{1infty limit} and \eqref{01limit}, we get
\[
\lim_{T\rightarrow\infty}\mathbb Q_{x}\left(\prod_{\sigma\leq t}I_{\{ X_{t+T-\sigma}^{{\mathrm m},\sigma}=0\}}=1\right)=1.
\]
Similarly,
\begin{eqnarray*}
\mathbb Q_x\left(\prod_{\sigma\leq t}I_{\{ X_{t+T-\sigma}^{{\mathrm n},\sigma}=0\}}=1\right)
&=&\widetilde\Pi_x\exp\left\{-\int_0^t2\sigma(Y_s)^2\mathbb N_{Y_s}(\zeta<T+t-s)ds\right\}\\
&=&\widetilde\Pi_x\exp\left\{-\int_0^t2\sigma(Y_s)^2v(T+t-s,Y_s)ds\right\}.
\end{eqnarray*}
Since $v(T+r, x)$ is  bounded for $(r,x)\in (0,\infty)\times E$ when $T$ is large enough, and $\lim_{T\rightarrow\infty} v(T+t-s,x)=0$ for any $x$,  by the bounded convergence theorem, we  have
\[
\lim_{T\rightarrow\infty}\mathbb Q_x\left(\prod_{\tau\leq t}I_{\{ X_{t+T-\tau}^{{\mathrm n},\tau}=0\}}=1\right)=1,
\]
By the inequality \eqref{IU}, for any $\varepsilon>0$ and $t>1$,
such that for any $x\in E$,
\begin{eqnarray*}
&&\limsup_{T\rightarrow\infty}\widetilde\Pi_x\left(|H(Y_t, T)-q(f)|>\varepsilon\right)\\
&\leq& \limsup_{T\rightarrow\infty}(1+ce^{-\rho t})\int_E\phi(y)\widehat\phi(y)m(dy)I_{\{|H(y, T)-q(f)|>\varepsilon\}}=0.
\end{eqnarray*}
Then from the inequality \eqref{subsub}, we have for any $x\in E$,
\begin{eqnarray*}
\liminf_{T\rightarrow\infty}H(x, t+T)&\geq&  \liminf_{T\rightarrow\infty} \mathbb Q_x\left[\prod_{\sigma\leq t}I_{\{ X_{t+T-\sigma}^{{\mathrm m},\sigma}=0\}}\prod_{\tau\leq t}I_{\{ X_{t+T-\tau}^{{\mathrm n},\tau}=0\}}H(Y_t, T)\right]\\
&\geq& q(f)=\limsup_{t\rightarrow\infty}H(x, t).
\end{eqnarray*}
 Therefore \eqref{limit-H} holds.

{\bf Step 3}\quad
Since $0\leq H(x,t)\leq 1$,
\begin{equation*}
q(f)
=\lim_{t\rightarrow\infty}\int_E\phi(x)\widehat\phi(x)H(x,t)m(dx)
=\lim_{t\rightarrow\infty}\widetilde{\mathbb P}_{\nu}\left(e^{-\langle f, X_t\rangle }\right)
=\mathbf Q(e^{-\mu(f)}).
\end{equation*}
Therefore for any $\mu\in\mathcal M_f(E)\setminus\{\mathbf{0}\}$, and $f\in\mathcal B_b(E,[0,\infty))$,
\begin{eqnarray*}
\lim_{t\rightarrow\infty}\widetilde{\mathbb P}_\mu\left(e^{-\langle f, X_t\rangle}\right)&=&\lim_{t\rightarrow\infty}\mathbb P_\mu\left(e^{-\langle f, X_t\rangle}\right)
\lim_{t\to\infty}\dfrac{1}{\mu(\phi)}\int_E\phi(x)H(x, t)\mu(dx)\\
&=&q(f)=\mathbf Q(e^{-\mu(f)}).
\end{eqnarray*}

This says $\mathbf Q$, the distribution limit of $X_t$ under $\widetilde{\mathbb P}_{\nu}$, is the  limit distribution of  $X_t$ under $\widetilde{\mathbb P}_{\mu}$ for any $\mu\in{\mathcal M}_f(E)\setminus\{\mathbf 0\}$. Thus $\mathbf Q$ is the
equilibrium distribution of the $Q$-process, and is a size-biased distribution of the Yaglom probability $\mathbf Q_\lambda$ with weight function $\dfrac{\mu(\phi)}{\mathbf Q_\lambda(\mu(\phi))}$.  The proof of $(1)$ is finished.



(2) When $\int_El(x)\nu(dx)=\infty$,
 it is shown in Theorem \ref{thm:E} that
 $\sum_{s\in\mathcal D^{\mathrm m}} \langle \phi,\widehat X^{{\mathrm m},s}_s\rangle =\infty$, $\widehat{\mathbb Q}_\nu$ almost surely. Thus
\[
\langle \phi, \widehat Z_{\infty}\rangle =\infty,\qquad \widehat{\mathbb Q}_\nu-{\mathrm a.s.}
\]
In this case, for any $\mu\in \mathcal M_f(E)\backslash\{0\}$, $\lim_{t\to\infty}\langle \phi, X_t\rangle =\infty$ in probability with respect to $\widetilde{\mathbb P}_\mu$. The  $Q$-process $\{(X_t)_{t\geq 0}; \widetilde{\mathbb P}_{\mu}\}$ does not have equilibrium distribution.
\end{proof}

\section{Appendix}

\subsection{Tauberian theorems}

In this subsection we collect some results on regularly varying functions and  Tauberian theorems, which are used in this paper.
The following lemma is from \cite[Appendix 13.6]{AH}.
\begin{lem}\label{lem:regu}
Let $f(x)$ be monotone increasing for $x\in (0,c)$. If
\[
\lim_{n\to\infty}\dfrac{f(\lambda\theta_n)}{f(\theta_n)}=\lambda^\alpha,\qquad \forall \lambda\in (0,1],
\]
for some $\alpha\in\mathbb R$ and some sequence $\{\theta_n\}$ of positive reals tending to $0$, as $n\to\infty$ in such a way that $\theta_n/\theta_{n+1}\leq c$ for $n\in\mathbb N$ and some $1<c<\infty$, then $f(x)$ is regularly varying with exponent $\alpha$.
\end{lem}


The following two lemmas are from \cite[Appendix 14]{AH}.
\begin{lem}\label{lem: tau}
Let $U(x)$ be a monotone non-decreasing function on $[0,\infty)$ such that
\[
w(x)=\int_0^\infty e^{-xu} dU(u)
\]
is finite for all $x>0$. If for some $\alpha\geq 0$, $w(x)\sim x^{-\alpha}L(1/x)$, $x\downarrow 0$, where $L$ is slowly varying at infinity, then
\[
U(x)\sim x^{\alpha}\dfrac{L(x)}{\Gamma(\alpha+1)},\qquad x\to\infty.
\]
 And if for some $\alpha\geq 0$, $w(x)\sim x^{-\alpha}L(x)$, $x\uparrow \infty$, then
\[
U(x)\sim x^{\alpha}\dfrac{L(1/x)}{\Gamma(\alpha+1)},\qquad x\downarrow 0.
\]
\end{lem}

\begin{lem}\label{lem:tail}
For $x\in [\beta,\infty)$, let
\[
U(x)=\int_\beta^xu(y)dy,
\]
where $u(y)$ is ultimately monotone.  If for some $\alpha\geq 0$, $U(x)=x^\alpha L(x)$,  where $L$ is slowly varying at infinity, then
\[
\lim_{x\to\infty}\dfrac{xu(x)}{U(x)}=\alpha.
\]
\end{lem}

\subsection{Proof of Lemma   \ref{thm:E.4}}

\begin{comment}
\begin{proof}[Proof of Lemma \ref{lem:reverse of the spine}]
	Since $E$ is a Polish space, we only need to prove that for any fixed $T>0$, $n \in \mathbb N$ and $0= t_1\leq \dots \leq t_n \leq T$,
	 The following identity holds
\[
	\widetilde \Pi_{\phi \cdot \nu}\{Y_{T-t_i}\in B_i,\forall i=1,\dots, n\}
	\overset{d}{=}\widehat \Pi_{\phi \cdot \nu}\{\widehat Y_{t_i}\in B_i,\forall i=1,\dots, n\},
\]
for arbitrary $B_i \in \mathscr B(E)$,  $i=1,\dots, n$.
	In fact, on one hand, we have
\begin{align}
	&\widetilde \Pi_{\phi \cdot \nu}\{Y_{T-t_i}\in B_i,\forall i=1,\dots, n\}
 \\&= \int_{y_n\in B_n} \phi(y_n)\widehat\phi(y_n) m(dy_n)\int_{y_{n-1}\in B_{n-1}} \tilde p_{t_n - t_{n-1}}(y_n,y_{n-1})m(dy_{n-1})
	\\& \qquad \dots \int_{y_1\in B_1} \tilde p_{t_2 - t_1}(y_2,y_1)m(dy_1)
	\\&= \int_{E^n} \Big(\prod_{i=1}^n \mathbf 1_{\{y_i\in B_i\}}\Big)\cdot\Big(\prod_{i=1}^{n-1} \tilde
 p_{t_{i+1}-t_i}(y_{i+1},y_i)\Big)\cdot\phi(y_n)\widehat\phi(y_n)\cdot \Big(\prod_{i=1}^nm(dy_i)\Big).
\end{align}
	On the other hand, we have
\begin{align}
	&\widehat \Pi_{\phi\cdot\nu}\{Y_{t_i}\in B_i, \forall i = 1,\dots, n\}
 \\&= \int_{y_1\in B_1} \phi(y_1)\widehat \phi(y_1)m(dy_1) \int_{y_2\in B_2} \hat p_{t_2-t_1}(y_1,y_2)m(dy_2)
	\\&\qquad \dots\int_{y_n\in B_n}\hat p_{t_n-t_{n-1}}(y_{n-1},y_n)m(dy_n)
	\\&= \int_{y_1\in  B_1} \phi(y_1)\widehat \phi(y_1)m(dy_1) \int_{y_2\in B_2} \tilde p_{t_2-t_1}(y_2,y_1)\frac{\phi(y_2)\widehat\phi(y_2)}{\phi(y_1)
     \widehat \phi(y_1)}m(dy_2)
 \\&\qquad \dots\int_{y_n\in B_n}\tilde p_{t_n-t_{n-1}}(y_n,y_{n-1})\frac{\phi(y_n)\widehat\phi(y_n)}{\phi(y_{n-1})\widehat\phi(y_{n-1})}m(dy_n)
\\&= \int_{E^n} \Big(\prod_{i=1}^n \mathbf 1_{\{y_i\in B_i\}}\Big)\cdot\Big(\prod_{i=1}^{n-1} \tilde p_{t_{i+1}-t_i}(y_{i+1},y_i)\Big)\cdot\phi(y_n)\widehat\phi(y_n)\cdot \Big(\prod_{i=1}^n m(dy_i)\Big).
\qedhere
\end{align}
\end{proof}
\end{comment}

\begin{proof}[Proof of Lemma \ref{thm:E.4}] The spine processes  $\{(Y_t)_{t\geq 0}; \widehat{\mathbb Q}_\nu\}$ and  $\{(Y_t)_{t\geq 0}; \widehat {\mathbb Q}_\nu\}$  are the same process. We only need to prove the result for one of $\widehat{\mathbb Q}_{\nu}$ and $\mathbb Q_{\nu}$. In the following we only prove the results for  $\widehat{\mathbb Q}_{\nu}$.



Since $\phi$ is bounded from above, $\sigma_i$ is strictly
increasing with respect to $i$.

(1) Suppose
$\int_E l(y)\nu(dy)<\infty$.
For any $N>0$, we have
\begin{equation}\label{sum}
\begin{array}{rl}
&\displaystyle\sum_{s\in \mathcal D^m}\mbox{e}^{-\varepsilon s}m_s\phi({Y}_s)\\
=&\displaystyle\sum_{s\in \mathcal D^m}
\mbox{e}^{-\epsilon s}m_s\phi(Y_s)
1_{\{\phi(Y_s)m_s\le\mbox{e}^{N
s}\}}+\sum_{s\in \mathcal
D^m}\mbox{e}^{-\varepsilon s}m_s\phi(Y_s)
1_{\{m_s\phi(Y_s)>\mbox{e}^{N
s}\}}\\
=&\displaystyle\sum_{s\in \mathcal
D^m}\mbox{e}^{-\varepsilon s}m_s\phi(Y_s) 1_{\{
\phi(Y_s)m_s\le \mbox{e}^{N s}
\}}+\sum_{i=1}^\infty\mbox{e}^{-\varepsilon\sigma_i}\eta_i
\phi(Y_{\sigma_i})
1_{\{\eta_i\phi(Y_{\sigma_i})>\mbox{e}^{N
\sigma_i}\}}\\=&I+II.\end{array}
\end{equation}
Note that
\begin{eqnarray*}
\sum_{i=1}^\infty \widehat{\mathbb Q}_{\nu}
\left(\eta_i\phi(Y_{\sigma_i}) > \mbox{e}^{N
\sigma_i}\right)&=&\sum_{i=1}^\infty
 \widehat{\mathbb Q}_{\nu}\left[\widehat{\mathbb Q}_{\nu}\left(\eta_i
\phi(Y_{\sigma_i})>\mbox{e}^{N
\sigma_i}\big|\sigma(Y)\right)\right]\\
&=&\widehat{\mathbb Q}_{\nu}\left[\widehat{\mathbb Q}_{\nu}\left(\sum_{i=1}^\infty
1_{\{\eta_i>\mbox{e}^{N \sigma_i}
\phi(Y_{\sigma_i})^{-1}\}}\Big|
\sigma(Y)\right)\right]\\
&=&\widehat{\mathbb Q}_{\nu}\left[\int_0^\infty
\left(\int^{\infty}_{\phi(Y_s)^{-1}\mbox{e}^{N s}}
r\pi(Y_s, dr)\right)ds\right]
\end{eqnarray*}
Recall that under $\widehat{\mathbb Q}_{\nu}$, $Y$ starts at the
invariant measure
$\phi(x)\nu(dx)=\phi(x)\widehat\phi(x)m(dx)$.
 So we have
\begin{eqnarray*}
\sum_{i=1}^\infty \widehat{\mathbb Q}_{\nu} \left(\eta_i\phi(
Y_{\sigma_i}) > \mbox{e}^{N\sigma_i}\right)
&=&\int_0^\infty ds
\int_E m(dy)\phi(y)\widehat{\phi}(y)\int^{\infty}_{\phi(y)^{-1} e^{N s}}r\pi(y, dr)\\
 &=& \int_E\phi(y)\widehat{\phi}(y)m(dy)\int_{\phi(y)^{-1}}^\infty r\pi(y, dr)\int^{\frac{\ln (r\phi(y))}{N}}_{0}ds\\
&=&N^{-1}\int_E l(y)\nu(dy) .
\end{eqnarray*}
By the assumption that
$\int_E l(y)\nu(dy)<\infty$ and
the Borel-Cantelli Lemma,  we get
\begin{equation}\label{io}
\widehat{\mathbb Q}_{\nu}\Big(\eta_i\phi(\widehat
Y_{\sigma_i})>\mbox{e}^{N \sigma_i} \mbox{ i. o.}\Big)=0
\end{equation}
for all $N>0$,  which implies that
\begin{equation}\label{big}
II<\infty.\quad \widehat{\mathbb Q}_{\nu}-\mbox{a.s.}
\end{equation}
Meanwhile for $\varepsilon<N$,
$$
\begin{array}{rl}
&\widehat{\mathbb Q}_{\nu}I=\displaystyle \widehat{\mathbb Q}_{\nu}\left[\sum_{s\in\mathcal
D^m} e^{-\varepsilon s}m_s\phi(Y_s) 1_{\{m_s\le
e^{N s} \phi(Y_s)^{-1}\}}\right]\\
&=\displaystyle \widehat{\mathbb Q}_{\nu}\int_0^\infty dt
e^{-\varepsilon t}\int_0^{\phi({Y}_t)^{-1}
e^{N t}}\phi(Y_t)r^2 \pi(Y_t, dr)\\
&\le \displaystyle\|\phi\|_{\infty}\widehat{\mathbb Q}_{\nu}\int_0^\infty
dt e^{-\varepsilon t}\int_0^1r^2 \pi(Y_t, dr)+
\widehat{\mathbb Q}_{\nu}\int_0^\infty dt
e^{-(\varepsilon-N)t}\int_1^{\infty}r \pi(Y_t, dr),
\end{array}
$$
where for the second term of the last inequality we used the fact
that $r\le\phi({Y}_t)^{-1} \mbox{e}^{N t}$ implies that
$r\phi({Y}_t)\le \mbox{e}^{N t}$. By the assumption that
$\sup_{x\in E}\int_0^\infty (r\wedge r^2) \pi(x, dr)<\infty,$ we have
$\widehat{\mathbb Q}_{\nu}I<\infty$, which implies that
\begin{equation}\label{small}
I<\infty,\quad \widehat{\mathbb Q}_{\nu}-\mbox{a.s.}
\end{equation}
Combining \eqref{sum}, \eqref{big} and \eqref{small}, we see that
$\sum_{s\in {{\mathcal D^m}}}\mbox{e}^{-\varepsilon s}m_s
\phi(Y_s)<\infty \quad\widehat{\mathbb Q}_{\nu}-\mbox{a.s.}$



(2) Suppose that $ \int_El(y)\nu(dy)=\infty$ and $\epsilon>0$. We only need to prove \eqref{inteqinfty} for large $K$.
Put $K_0:=1\vee(\max_{x\in E}\phi(x))$.
Then for $K\ge K_0$,
$
K\inf_{x\in E}\phi(x)^{-1}\geq 1.
$
We  need to prove that for $K\ge K_0$, 
\begin{equation}\label{inteqinfty}
\xi_\infty:=\int_0^\infty dt\int_{K\phi(Y_t)^{-1}e^{\varepsilon t}}^\infty r\pi(Y_t,dr)
=\infty,\quad \widehat{\mathbb Q}_{\nu}-{\rm a.s.} \footnote{ZS: The label here is multiple defined.}
\end{equation}
To prove \eqref{inteqinfty}, we first prove that
\begin{equation}\label{mean=infty}
\widehat{\mathbb Q}_{\nu}\left[\int_0^\infty dt\int_{K\phi(Y_t)^{-1}
e^{\varepsilon t}}^\infty r \pi(Y_t, dr)\right]=\infty.
\end{equation}
Applying Fubini's Theorem, we get
\begin{eqnarray*}
&&\widehat{\mathbb Q}_{\nu}\left[\int_0^\infty dt\int_{K\phi(Y_t)^{-1}e^{\varepsilon t}}^\infty r\pi(Y_t,dr)\right]\nonumber\\
&=& \int_E\phi(y)\widehat{\phi}(y)m(dy)\int_0^\infty dt\int_{K\phi(y)^{-1}e^{\varepsilon t}}^\infty r\pi(y, dr)\\
&=& \int_E\phi(y)\widehat{\phi}(y)m(dy)\int_{K\phi(y)^{-1}}^\infty r\pi(y,dr)\int_0^{\frac{1}{\varepsilon}\ln(\frac{r\phi(y)}{K})}dt\\
&=&\frac{1}{\varepsilon}\int_E\phi(y)\nu(dy)\int_{K\phi(y)^{-1}}^\infty(\ln[r\phi(y)]-\ln K)r\pi(y, dr)\\
&\ge&\frac{1}{\varepsilon}\int_E\phi(y)\nu(dy)\left[\int_{K\phi(y)^{-1}}^\infty r\ln[r\phi(y)]\pi(y, dr)-A\right]\\
&=&\frac{1}{\varepsilon}\int_E\nu(dy) \int_{K}^\infty r\ln r\pi^\phi(y, dr)-\frac{A}{\varepsilon}\int_E\phi(y)\nu(dy),
\end{eqnarray*}
for some positive constant $A$, where in the inequality we used the
facts that $K\phi(y)^{-1}>1$ for any
$y\in E$ and $\sup_{y\in E}\int^\infty_1r \pi(y, dr)<\infty$.
Since
$$
\int_E\nu(dy)\int_1^\infty r\ln r \pi^\phi(y, dr)=\infty,
$$
and
$$
\int_E\nu(dy)\int_{1}^{K} r\ln r \pi^\phi(y, dr)\leq K\log K\int_E \widehat{\phi}(y)\pi(y,[\|\phi\|_{\infty}^{-1},\infty))dy<\infty,
$$
we get that
$$
\int_E\nu(dy)\int_K^\infty r\ln r \pi^\phi(y, dr)=\infty,
$$
and therefore, \eqref{mean=infty} holds.


By \eqref{IU'}, there exists constant $c>0$ such that for
any $t>c$ and any
$f\in \mathcal B_b(E,[0,\infty))$,
\begin{equation}\label{domi-p}
 \frac{1}{2}\int_E\phi(y)\widehat{\phi}(y)f(y)m(dy)\leq \int_E\hat p(t, x, y)f(y)m(dy)\leq 2\int_E\phi(y)\widehat{\phi}(y)f(y)m(dy),\quad x\in E.
\end{equation}
For $T>c$, we define
$$
 \xi_T=\int_0^T dt\int_{K\phi(Y_t)^{-1}e^{\varepsilon t}}^\infty r\pi(Y_t, dr),
$$
and
$$
 A_T=\int_c^Tdt\int_E\widehat{\phi}(y)m(dy)\int_{K e^{\varepsilon t}}^\infty r\pi^\phi(y, dr).
$$
Since $\left\{\xi_\infty=\infty\right\}$ is an invariant event, by
the ergodic property of $Y$ under $\widehat{\mathbb Q}_{\nu}$, \eqref{inteqinfty} is
enough to prove
\begin{equation}\label{positive-prob}
\widehat{\mathbb Q}_{\nu}\left(\xi_\infty=\infty\right)>0.
\end{equation}
Note that
\begin{equation}\label{domi-mean}
 \widehat{\mathbb Q}_{\nu} \xi_T=\int_0^Tdt\int_E\nu(dy)\int_{K e^{\varepsilon t}}^\infty r\pi^\phi(y, dr)\ge A_T,
\end{equation}
and
\begin{equation}\label{mean-limit}
\begin{array}{rl}
\lim_{T\rightarrow\infty}\widehat{\mathbb Q}_{\nu}\xi_T\ge A_\infty
=&\displaystyle\int_c^\infty dt\int_E \nu(dy)\int_{K e^{\varepsilon t}}^\infty r \pi^\phi(y, dr)\\
=&\displaystyle\int_E\nu(dy)\int_{Ke^{\varepsilon c}}^\infty \left(\frac{1}{\varepsilon}(\ln r-\ln K)-c\right)r\pi^\phi(y, dr)\\
\ge&\displaystyle C\int_El(y)\nu(dy)=\infty,
\end{array}
\end{equation}
where $C$ is a positive constant.
By  the Cauchy-Schwartz inequality, we have
\begin{equation}\label{Durrett-domi}
\widehat{\mathbb Q}_{\nu}\left(\xi_T\geq \frac{1}{2}\widehat{\mathbb Q}_{\nu}
\xi_T\right)\geq \frac{(\widehat{\mathbb Q}_{\nu}
\xi_T)^2}{4\widehat{\mathbb Q}_{\nu}(\xi_T^2)}.
\end{equation}
If we can prove that there is a constant $\widehat C>0$ such that
for all $T>c,$
\begin{equation}\label{uniform lower bound}
\frac{(\widehat{\mathbb Q}_{\nu}
\xi_T)^2}{4\widehat{\mathbb Q}_{\nu}(\xi_T^2)}\geq \widehat C.
\end{equation}
Then by \eqref{Durrett-domi} we would get
$$
\widehat{\mathbb Q}_{\nu}\left(\xi_T\geq \frac{1}{2}\widehat{\mathbb Q}_{\nu}
\xi_T\right)\geq \widehat C,
$$
and therefore
\begin{eqnarray*}
&&\widehat{\mathbb Q}_{\nu}\left(\xi_\infty\geq
\frac{1}{2}\widehat{\mathbb Q}_{\nu} \xi_T\right)
\geq\widehat{\mathbb Q}_{\nu}\left(\xi_T\geq \frac{1}{2} \widehat{\mathbb Q}_{\nu}
\xi_T\right)\ge\widehat C>0.
\end{eqnarray*}
Since $\lim_{T\rightarrow\infty}\widehat{\mathbb Q}_{\nu} \xi_T=\infty$
(see \eqref{mean-limit}), the above inequality implies
\eqref{positive-prob}.  Now we only need to prove
 \eqref{uniform lower bound}. For this purpose we first estimate
$\widehat{\mathbb Q}_{\nu}(\xi_T^2)$:
\begin{eqnarray*}
\widehat{\mathbb Q}_{\nu}
\xi_T^2
&=&\widehat{\mathbb Q}_{\nu}\int_0^Tdt\int_{K\phi(Y_t)^{-1}e^{\varepsilon t}}^\infty r \pi(Y_t, dr)\int_0^Tds\int_{K\phi(Y_s)^{-1}e^{\varepsilon s}}^\infty u \pi(Y_s, du)\\
&=&2\widehat{\mathbb Q}_{\nu}\int_0^Tdt\int_{K\phi(Y_t)^{-1} e^{\varepsilon t}}^\infty r \pi(Y_t, dr)\int_t^Tds
\int_{K\phi(Y_s)^{-1}e^{\varepsilon s}}^\infty u \pi({Y}_s, du)\\
&=&2\widehat{\mathbb Q}_{\nu}\int_0^Tdt\int_{K\phi(Y_t)^{-1}e^{\varepsilon t}}^\infty r \pi(Y_t,dr)
\int_t^{(t+c)\wedge T}ds\int_{K\phi(Y_s)^{-1}e^{\varepsilon s}}^\infty u\, n({Y}_s,\ du)\\
&&+2\widehat{\mathbb Q}_{\nu}\int_0^{ T}dt\int_{K \phi(Y_t)^{-1}e^{\varepsilon t}}^\infty r \pi(Y_t, dr)
\int_{(t+c)\wedge T}^Tds\int_{K\phi(Y_s)^{-1}e^{\varepsilon s}}^\infty u \pi(Y_s, du)\\
&=&III+IV,
\end{eqnarray*}
where
$$
III=\displaystyle 2\widehat{\mathbb Q}_{\nu}\int_0^Tdt\int_{K\phi(Y_t)^{-1} e^{\varepsilon t}}^\infty r \pi(Y_t, dr)\int_t^{(t+c)\wedge T}ds\int_{K\phi(
Y_s)^{-1}e^{\varepsilon s}}^\infty u \pi(Y_s,du)
$$
and
\begin{eqnarray*}
IV& =& 2\widehat{\mathbb Q}_{\nu}\int_0^{T}dt \int_{K\phi(Y_t)^{-1}e^{\varepsilon t}}^\infty r \pi(Y_t, dr)
\int_{(t+c)\wedge T}^Tds\int_{K\phi(Y_s)^{-1}e^{\varepsilon s}}^\infty u \pi(Y_s, du)\\
&=& 2\int_0^{T}dt\int_E\phi(y)\widehat \phi(y)m(dy)\int_{K\phi(y)^{-1}e^{\varepsilon t}}^\infty r \pi(y, dr)\times\\
&& \ \ \ \times \int_{(t+c)\wedge T}^Tds\int_E\hat p(s-t,y,z)m(dz)\int_{K\phi(z)^{-1}e^{\varepsilon s}}^\infty u \pi(z,du).
\end{eqnarray*}
By our assumption on the kernel $n$ we have that $ \|\int_1^\infty r\pi(\cdot,dr)\|_{\infty}<\infty$. Since $K\inf_{x\in
E}\phi(x)^{-1}\ge 1$,  we have
$$
III\leq C_1\widehat{\mathbb Q}_{\nu}\xi_T,
$$
for some positive constant $ C_1$ which does not depend on $T$.
Using \eqref{domi-p} and the definition of $\pi^{\phi}$, we get that
$$
\begin{array}{rl}
&\displaystyle\int_{(t+c)\wedge T}^Tds\int_E\hat p(s-t,y,z)m(dz)\int_{K\phi(z)^{-1}e^{\varepsilon s}}^\infty u \pi(z,du)\\
\le&\displaystyle 2\int_{(t+c)\wedge T}^Tds\int_E\phi(z)\widehat\phi(z)m(dz)\int_{K\phi(z)^{-1}e^{\varepsilon s}}^\infty u \pi(z, du)\\
\le&\displaystyle 2\int_{c}^Tds\int_E\nu(dz)\int_{0}^\infty(\phi(z)u)I_{(\phi(z)u>Ke^{\varepsilon s})} \pi(z, du)\\
=&\displaystyle 2\int_{c}^Tds\int_E\nu(dz)\int_{Ke^{\varepsilon s}}^\infty r \pi^{\phi}(z, dr)=2A_T.
\end{array}
$$
Then using \eqref{domi-mean},  we have
$$
IV\leq 4 A_T\widehat{\mathbb Q}_{\nu}\xi_T\leq 4 (\widehat{\mathbb Q}_{\nu}
\xi_T)^2.
$$
Combining the estimates above on $III$ and $IV$, we get that there
exists a $C_2>0$ independent of $T$ such that for $T>c$,
$$
\widehat{\mathbb Q}_{\nu}(\xi_T^2)\le 4{\color{red}}
(\widehat{\mathbb Q}_{\nu}(\xi_T))^2+ C_1\widehat{\mathbb Q}_{\nu}(\xi_T)\le
C_2(\widehat{\mathbb Q}_{\nu}(\xi_T))^2.
$$
Then we have \eqref{uniform lower bound} with $\widehat C=1/C_2$, we finished the proof of
\eqref{inteqinfty}.

Now we prove that
$$
\limsup_{i\rightarrow\infty}e^{-\varepsilon\sigma_i}
\eta_i\phi(Y_{\sigma_i})=\infty,\quad
\widehat{\mathbb Q}_{\nu}-\mbox{a.s.}
$$
It suffices to prove that for any $K\geq K_0$,
\begin{equation}\label{>K}
\limsup_{i\rightarrow\infty}e^{-\varepsilon\sigma_i}
\eta_i\phi(Y_{\sigma_i})>K,\quad \widehat{\mathbb Q}_{\nu}-{\rm
a.s.}
\end{equation}
Note that for any $T\in (0,\infty)$, conditional on
$\sigma(Y)$, $\sharp\{i: \sigma_i\in(0, T];
\eta_i>K\phi(Y_{\sigma_i})^{-1}e^{\varepsilon\sigma_i}\}$
is a Poisson random variable with parameter $\int_0^Tdt
\int_{K\phi(Y_t)^{-1}e^{\varepsilon t}}^\infty r \pi(Y_t, dr)$ a.s. Since
$$
\widehat{\mathbb Q}_{\nu}\int_0^Tdt\int_{K\phi(
Y_t)^{-1}e^{\varepsilon t}}^\infty r\pi(Y_t,dr)
=\int^T_0dt\int_Em(dy)\phi(y)\widehat \phi(y)\int_{K\phi(y)^{-1}e^{\varepsilon t}}^\infty r\pi(y, dr)
<\infty,
$$
we have
$$
\int_0^Tdt\int_{K\phi (Y_t)^{-1}e^{\varepsilon t}}^\infty r\pi(Y_t, dr) <\infty,\quad \widehat{\mathbb Q}_{\nu}-{\rm a.s.}
$$
Consequently we have
\begin{equation}\label{number-finite}
\sharp\Big\{i: \sigma_i\in(0,T];
\eta_i>K\phi(Y_{\sigma_i})^{-1}e^{\varepsilon\sigma_i}\Big\}<\infty,\quad
\widehat{\mathbb Q}_{\nu}-{\rm a.s.}
\end{equation}
Now \eqref{>K} follows from \eqref{inteqinfty}.

\end{proof}




\begin{thebibliography}{99}
	
\bibitem{AH}Asmussens, S. and Hering, H. :\emph{Branching Processes}. Birkhauser, Boston, 1983.

\bibitem{AthreyaNey1972Branching}
Athreya, K. B. and Ney, P. E.:
\emph{Branching processes.}
Die Grundlehren der mathematischen Wissenschaften, Band 196. Springer-Verlag, New York-Heidelberg, 1972. xi+287 pp.
\MR{0373040}

\bibitem{BigginsKyprianou2004Measure}
Biggins, J. D. and Kyprianou, A. E.:
\emph{Measure change in multitype branching.}
Adv. in Appl. Probab. \textbf{36} (2004), no. 2, 544--581.
\MR{2058149}

\bibitem{ChampagnatRoelly2008Limit}
Champagnat, N. and Roelly, S.:
\emph{Limit theorems for conditioned multitype Dawson-Watanabe processes and Feller diffusions.}
Electron. J. Probab. \textbf{13} (2008), no. 25, 777C810.
\MR{2399296}

\bibitem{ChampagnatVillemonais2018Convergence}
Champagnat, N. and Villemonais, D.:
\emph{Convergence of the Fleming-Viot process toward
theminimal quasi-stationary distribution.}
https://arxiv.org/pdf/1810.06849.pdf

\bibitem{ChenRenYang2017Skeleton}
Chen, Z.-Q., Ren, Y.-X. and Yang, T.:
\emph{Skeleton decomposition and law of large numbers for supercritical superprocesses.}
Acta Appl. Math. 159(1)(2019) 225-285

\bibitem{Dawson1992Infinitely}
Dawson, D. A.:
\emph{Infinitely divisible random measures and superprocesses.}Stochastic analysis and related topics (Silivri, 1990), 1--129,
Progr. Probab., 31, Birkh{\"a}user Boston, Boston, MA, 1992.
\MR{1203373}

\bibitem{DelmasHenard2013A-Williams}
Delmas, J.-F. and H\'enard, O.:
\emph{A Williams decomposition for spatially dependent super-processes. }
Electron. J. Probab. \textbf{18} (2013), no. 37, 43 pp.
\MR{3035765}

\bibitem{Dudley2002Real}
Dudley, R. M.:
\emph{Real analysis and probability.}
Revised reprint of the 1989 original. Cambridge Studies in Advanced Mathematics, 74. Cambridge University Press, Cambridge, 2002. x+555 pp.

\bibitem{Dynkin1993Superprocesses}
Dynkin, E. B.:
\emph{Superprocesses and partial differential equations.}
Ann. Probab. \textbf{21} (1993), no. 3, 1185--1262.
\MR{1235414}

\bibitem{EnglanderKyprianou2004Local}
Engl\"ander, J. and Kyprianou, A. E.:
\emph{Local extinction versus local exponential growth for spatial branching processes.}
Ann. Probab. \textbf{32} (2004), no. 1A, 78--99.
\MR{2040776}

\bibitem{Evans1993Two}
Evans, S. N.:
\emph{Two representations of a conditioned superprocess.}
Proc. Roy. Soc. Edinburgh Sect. A \textbf{123} (1993), no. 5, 959--971.
\MR{1249698}

\bibitem{Grey1974Asymptotic}
Grey, D. R.:
\emph{Asymptotic behaviour of continuous time, continuous state-space branching processes.}
J. Appl. Probability \textbf{11} (1974), 669--677.
\MR{0408016}

\bibitem{Heathcote}
Heathcote, R.,  Seneta, E.  and Vere-Jones, D.:  
\emph{ A refinement of two theorems in the theory of branching processes}. 
Theory Probab. Appl. 12 (1982), 297-301.

\bibitem{HeathcoteSenetaVere-Jones1967A-refinement}
Heathcote, C. R., Seneta, E. and Vere-Jones, D.:
\emph{A refinement of two theorems in the theory of branching processes.} (Russian summary)
Teor. Verojatnost. i Primenen. \textbf{12} 1967 341--346.
\MR{0217889}

\bibitem{Joffe}Joffe, A. and Waughw, A. O.:  Exact distributions of kin numbers in a Galton-Watson
process. J. Appl. Probab. 19 (1982), 767-775.

\bibitem{Kallenberg2002Foundations}	
	Kallenberg, O.:
	\emph{Foundations of Modern probability.}
	Second Edition.	Springer-Verlag New York, 2002.

\bibitem{KimSong2008Intrinsic}
Kim, P., Song, R.:
\emph{Intrinsic ultracontractivity of non-symmetric diffusion semigroups in bounded domains.}
Tohoku Math. J. (2) 60 (2008), no. 4, 527-547.
\MR{2487824}

\bibitem{KimSong2008Intrinsic2}
Kim, P. and Song, R.:
\emph{Intrinsic ultracontractivity of nonsymmetric diffusions with measure-valued drifts and potentials.}
Ann. Probab. \textbf{36} (2008), no. 5, 1904--1945.
\MR{2440927}

\bibitem{KimSong2009Intrinsic}
Kim, P. and Song, R.:
\emph{Intrinsic ultracontractivity for non-symmetric L\'evy processes.}
Forum Math. \textbf{21} (2009), no. 1, 43C66.
\MR{2494884}

\bibitem{Lambert2001Arbres}
Lambert, A.:
\emph{Arbres, excursions et processus de L\'evy completement asym\'etriques.}
Diss. Universit Pierre et Marie Curie-Paris VI, 2001.

\bibitem{Lambert2003Coalescence}
Lambert, A.:
\emph{Coalescence times for the branching process.}
Adv. in Appl. Probab. \textbf{35} (2003), no. 4, 1071--1089.
\MR{2014270}

\bibitem{Lambert2007Quasi-stationary}
Lambert, A.:
\emph{Quasi-stationary distributions and the continuous-state branching process conditioned to be never extinct.}
Electron. J. Probab. \textbf{12} (2007), no. 14, 420--446.
\MR{2299923}

\bibitem{Li00}
Li, Z.-H.:
\emph{Asymptotic behaviour of continuous time and state branching processes.}
J. Austral. Math. Soc. Ser. A \textbf{68} (2000), no. 1, 68--84.
\MR{1727226}

\bibitem{Li2011Measure-valued}
Li, Z.:
\emph{Measure-valued branching Markov processes.}
Probability and its Applications (New York). Springer, Heidelberg, 2011. xii+350 pp. ISBN: 978-3-642-15003-6
\MR{2760602}

\bibitem{LiuRenSong2009Llog}
Liu, R.-L., Ren, Y.-X. and Song, R.:
\emph{{$L \log L$} criterion for a class of superdiffusions.}
J. Appl. Probab. \textbf{46} (2009), no. 2, 479--496.
\MR{2535827}

\bibitem{LiuRenSongSun2020}
Liu, R.-L., Ren, Y.-X., Song, R. and Sun, Z.
\emph{Quasi-stationary distributions for subcritical superprocesses.}
Preprint.
ARXIV{2001.06697}

\bibitem{LyonsPemantlePeres1995Conceptual}
Lyons, R., Pemantle, R. and Peres, Y.:
\emph{Conceptual proofs of $L\log L$ criteria for mean behavior of branching processes.}
Ann. Probab. \textbf{23} (1995), no. 3, 1125--1138.
\MR{1349164}

\bibitem{MeleardVillemonais2012Quasi-stationary}
M\'el\'eard, S. and Villemonais, D.:
\emph{Quasi-stationary distributions and population processes.}
Probab. Surv. \textbf{9} (2012), 340C410.
\MR{2994898}

\bibitem{Nagasawa1964Time}
Nagasawa, M.:
\emph{Time reversions of Markov processes.}
Nagoya Math. J. \textbf{24} (1964), 177--204.
\MR{0169290}

\bibitem{Penisson2010Conditional}
P\'enisson, S.:
\emph{Conditional limit theorems for multitype branching processes and illustration in epidemiological risk analysis.}Diss. Universitt Potsdam, Universit Paris Sud-Paris XI, 2010.

\bibitem{RenSongSun2020Spine} 
	Ren, Y.-X., Song, R., and Sun, Z.:
	\emph{Spine decompositions and limit theorems for a class of critical superprocesses.}
	Acta Appl. Math. \textbf{165} (2020): 91--131.

\bibitem{RenSongZhang2015Limit}
Ren, Y.-X., Song, R., and Zhang, R.:
\emph{Limit theorems for some critical superprocesses.}
Illinois J. Math. 59 (2015), no. 1, 235-276.

\bibitem{RenSongZhang2017Central}
Ren, Y.-X., Song, R., and Zhang, R.:
\emph{Central limit theorems for supercritical branching nonsymmetric Markov processes.}
Ann. Probab. 45 (2017), no. 1, 564-623.

\bibitem{RenSongZhang2018Williams}
Ren, Y.-X., Song, R. and Zhang, R.:
\emph{Williams decomposition for superprocesses.}
Electron. J. Probab. \textbf{23} (2018), Paper No. 23, 33 pp.
\MR{3771760}

\bibitem{RenSongYang2016Spine}
Ren, Y.-X., Song, R. and Yang, T.:
\emph{Spine decomposition and {$ L\log L $} criterion for superprocesses with non-local branching mechanisms.}
Preprint.
ARXIV{1609.02257}

\bibitem{RoellyRouault1989Processus}
Roelly, S. and Rouault, A.:
\emph{Processus de Dawson-Watanabe conditionn\'e par le futur lointain.} (French. English summary) [A Dawson-Watanabe process conditioned by the remote future]
C. R. Acad. Sci. Paris Sr. I Math. \textbf{309} (1989), no. 14, 867--872.
\MR{1055211}

\bibitem{Schaefer1974Banach}
Schaefer, H. H.:
\emph{Banach lattices and positive operators.}
Die Grundlehren der mathematischen Wissenschaften, Band 215. Springer-Verlag, New York-Heidelberg, 1974. xi+376 pp. \MR{0423039}

\bibitem{Yaglom1947}
Yaglom, A. M.:
\emph{Certain limit theorems of the theory of branching random processes.} (Russian)
Doklady Akad. Nauk SSSR (N.S.) \textbf{56} (1947), 795--798.
\MR{0022045}

\end{thebibliography}
\end{document}
