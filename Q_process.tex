%Rongli Q_process 2020-02-11
%Zhenyao Q_process 2020-06-10
%Yanxia Q_process2 2020-06-22
%Rongli Q_process3 2020-07-01
%Yanxia Q_process4 2020-07-01
%Zhenyao Q_process5 2020-07-05
%Renming Q_process6 2020-07-06
%Zhenyao Q_process7 2020-07-19
%Yanxia  Q_process8
%Zhenyao Q_process9 2020-07-31
%Yanxia Q_process10 2020-08-03
%Renming Q-process11 2020-08-06
\documentclass[12pt,a4paper]{amsart}
\setlength{\textwidth}{\paperwidth}
\addtolength{\textwidth}{-2in}
\calclayout
\numberwithin{equation}{section}
\allowdisplaybreaks
\theoremstyle{plain}
\newtheorem{thm}{Theorem}[section]
\newtheorem{asp:mean}{Assumption}[section]
\newtheorem{lem}[thm]{Lemma}
\newtheorem{prop}[thm]{Proposition}
\newtheorem{cor}[thm]{Corollaray}
\newtheorem{fact}[thm]{Fact}
\newtheorem{remark}{Remark}
\newtheorem{claim}[thm]{Claim}
\theoremstyle{definition}
\newtheorem*{ack*}{Acknowledgment}
\theoremstyle{remark}
\newtheorem{exa}[thm]{Example}
\usepackage{amssymb}
\usepackage{mathtools}
\mathtoolsset{showonlyrefs}
\usepackage{mathrsfs}
\usepackage{comment}
\usepackage[backref]{hyperref}
\usepackage[inline]{showlabels}
\usepackage{xcolor}

\begin{document}
\title {Subcritical Superprocesses Conditioned on Non-extinction}
\author[R. Liu]{Rongli Liu}
\address{Rongli Liu\\ Mathematics and Applied Mathematics\\ Beijing jiaotong University\\ Beijing 100044\\ P. R. China}
\email{rlliu@bjtu.edu.cn}
\thanks{The research of Rongli Liu is supported in part by NSFC (Grant No. 11301261), and the Fundamental Research Funds for the Central Universities (Grant No.  2017RC007)}
\author[Y.-X. Ren]{Yan-Xia Ren}
\address{Yan-Xia Ren\\ LMAM School of Mathematical Sciences \& Center for
Statistical Science\\ Peking University\\ Beijing 100871\\ P. R. China}
\email{yxren@math.pku.edu.cn}
\thanks{The research of Yan-Xia Ren is supported in part by NSFC (Grant Nos. 11671017 and 11731009)  and LMEQF.}
\author[R. Song]{Renming Song}
\address{Renming Song\\ Department of Mathematics\\ University of Illinois at Urbana-Champaign \\ Urbana \\ IL 61801\\ USA}
\email{rsong@illinois.edu}
\author[Z. Sun]{Zhenyao Sun}
\address{Zhenyao Sun\\ Faculty of Industrial Engineering and Management \\ Technion, Isreal Institute of Technology \\ Haifa 3200003\\ Isreal}
\email{zhenyao.sun@gmail.com}
\begin{abstract}
\begin{comment}
Suppose that $E$ is a locally compact separable metric space, and  that  $X =\{(X_t)_{t\geq 0}; (\mathrm P_\mu)_{\mu \in \mathcal M_f(E)}\}$  is a subcritical superprocess, where $\mathcal M_f(E)$ is the space of all finite Borel measures on $E$. Under some conditions on the mean semigroup of $X$, we prove that the the $Q$ process of $X$ exists, and they have equilibrium probability if and only if the moment condition $\int_El(x)\nu({\mathrm d}x)<\infty$ is satisfied.
We also show that the equilibrium probability is a size-biased measure of the Yaglom distribution.
\end{comment}
    TBD
\end{abstract}
\maketitle
\section{Introduction}
\subsection{Background}
\begin{comment}
The study of the extinction of populations is of a great interest in biology. Conditioning on non-extinction can not only notably lead to a stationary behavior
of the process, but also provides a lot of information about the evolution of the population before extinction.  As far as the population dynamics which are extinct in finite time running over large amounts of time are concerned, special attentions are given to two conditional limits: one type of limits are the Yaglom distributions and the related quasi-stationary distributions, the other type of limits are the Q processes.   There are plenty of literatures investigating these properties of branching processes. See, for instance, Athreya and Ney \cite[pages 64-65]{AthreyaNey1972Branching},Grey\cite{Grey1974Asymptotic},Lyons, Pemantle and Peres \cite{LyonsPemantlePeres1995Conceptual}, Seneta and
Vere-Jones \cite{Heathcote},Heathcote,Joffe \cite{Joffe}, Yaglom \cite{Yaglom47} for Galton-Watson branching processes, Lambert\cite{Lambert2001Arbres, Lambert2003Coalescence},Li\cite[Theorem 4.3]{Li00} for continuous state and continuous time branching processes. Asmussen and Hering \cite{AH} studied limit behaviour of subcritical branching Markov processes, in which each particle  lives for exponential time, then give birth to random number of particles, and particles move as independent
Markov processes in between branching times and it is assume that  the life times, reproduction of different individuals is independent.

In the authors' paper \cite{LSYS}, the Yaglom distributions of the superprocesses are investigated.  In this paper, we are mainly studying the properties of the Q processes for superprocesses and discover their relationships with the Yaglom distributions..
\end{comment}
	TBD
\subsection{Main Result}\label{sec:M}
	We first recall the definition of a superprocess.
	Let $E$ be a locally compact separable metric space, and $E_\partial := E \cup \{\partial\}$ be the one-point compactification of $E$. 
	Denote by $\mathcal B(E, D)$ the collection of Borel maps  from $E$ to some metric space $D$.
	Denote by $\mathcal B_b(E,D)$ the metrically bounded elements in $\mathcal B(E,D)$.
	Let \emph{the underlying process} 
	$\xi = \{(\xi_t)_{t\ge0}; (\Pi_x)_{x\in E_\partial}\}$
	 be an $E_\partial$-valued Hunt process with $\partial$ as an absorbing state.
	Denote by $\zeta:=\inf\{t>0: \xi_t=\partial\}$ the lifetime of $\xi$.
	Let \emph{the branching mechanism} $\psi$ be a function on $E \times \mathbb R_+$ given by
\begin{align}
	\psi(x,z)
	= -\beta(x) z + \sigma(x)^2 z^2 + \int_{(0,\infty)} (e^{-zu} -1 + zu) \pi(x,{\mathrm d}u),
	\quad x\in E, z\geq 0
\end{align}
	where $\beta, \sigma \in \mathcal B_b(E,\mathbb R)$ and $(u \wedge u^2) \pi(x,{\mathrm d}u)$ is a bounded kernel from $E$ to $(0,\infty)$.
	Let $\mathcal M_f(E)$ denote the space of all finite Borel measures on $E$ equipped with the topology of weak convergence.
		For any measure $\mu$ and function $f$, we use $\mu(f)$ to denote the integral of $f$ with respect to $\mu$ whenever the integral is well-defined.
		For any $f \in \mathcal B_b(E,\mathbb R_+)$, according to \cite[Proposition 2.20]{Li2011Measure-valued}, there is a unique locally bounded non-negative map $(t,x)\mapsto V_tf(x)$ on $\mathbb R_+\times E$ such that
\begin{equation} \label{eq:M.1}
	V_tf(x) + \Pi_x\Big[\int_0^{t\wedge \zeta} \psi\big(\xi_s, V_{t-s} f(\xi_s)\big) {\mathrm d}s\Big] = \Pi_x[f(\xi_t) \mathbf 1_{t< \zeta}], \quad t\geq 0, x\in E.
\end{equation}
	Here, the local boundedness of the map $(t,x) \mapsto V_tf(x)$ means that for any $T>0$,
\[
	\sup_{0\leq t\leq T, x\in E} V_tf(x) < \infty.
\]
	According to \cite[Proposition 2.21 and Theorem 5.12]{Li2011Measure-valued}, there exists 
	an $\mathcal M_f(E)$-valued 
	Borel right process 
		  $X =\{(X_t)_{t\geq 0}; (\mathrm P_\mu)_{\mu \in \mathcal M_f(E)}\}$ such that
	\begin{equation}
	\mathrm P_\mu[e^{- X_t(f)}]
	= e^{- \mu(V_tf)},
	\quad \mu\in \mathcal M_f(E), t\geq 0, f \in \mathcal B_b(E,\mathbb R_+).
\end{equation}
	This process $X$ is known as a $(\xi, \psi)$-superprocess.
	
	Let us now give some basic assumptions on our superprocess.
	Since we are only concerned with distributional properties of the superprocess $X$, we will assume, without loss of generality, that $X$ is \emph{canonical},  i.e.,
	(1) $(X_t)_{t\geq 0}$ is the coordinate process of
		$\mathbb W$,
		the space of $\mathcal M_f(E)$-valued right continuous paths on $\mathbb R_+$; and 
	(2) $\mathrm P_\mu(\mathrm dw)$ is a probability transition kernel from $\mathcal M_f(E)$ to $\mathbb W$.
	We will use $(\mathscr F_t)_{t\geq 0}$ to denote the natural filtration on $\mathbb W$.
	Define a Feynman-Kac semigroup of $\xi$ by
\begin{align}\label{eq:M.15}
	P_t^\beta f(x)
	:= \Pi_x[e^{\int_0^t \beta(\xi_r) {\mathrm d}r }f(\xi_t) \mathbf 1_{\{t < \zeta\}}],
	\quad f\in \mathcal B_b(E,\mathbb R), t\geq 0, x\in E.
\end{align}
	It is known (see \cite[Proposition 2.27]{Li2011Measure-valued}) that
\begin{equation} \label{eq:M.2}
	\mathrm P_\mu[X_t(f)] = \mu (P_t^\beta f),
	\quad t\geq 0, f \in \mathcal B_b(E,\mathbb R).
\end{equation}
	Thus $(P_t^\beta)_{t\geq 0}$ is called the \emph{mean semigroup}  of $X$.
	For this mean semigroup, we will always assume that
\begin{equation}\label{asp:H1} \tag{H1}
\begin{minipage}{0.9\textwidth}
	there exist a constant $\lambda < 0$, a function $\phi \in \mathcal B_b(E,(0,\infty))$ and a probability measure $\nu$ with full support on $E$ such that for each $t\geq 0$, $P_t^\beta \phi = e^{\lambda t} \phi$, $\nu P_t^\beta = e^{\lambda t} \nu$ and $\nu(\phi) =1$.
\end{minipage}
\end{equation}
	Denote by $L_1^+(\nu)$ the collection of non-negative Borel functions on $E$ which are integrable with respect to the measure $\nu$.
	Denote by $\mathbf 0$ the null measure on $E$.
	Write $\mathcal M_f^o(E) = \mathcal M_f(E)\setminus \{\mathbf 0\}$.
	We further assume that the following two conditions hold:
\begin{equation} \label{asp:H2} \tag{H2}
\begin{minipage}{0.9\textwidth}
	for all $t>0$, $x\in E$, and $f\in L_1^+(\nu)$, it holds that $P_t^\beta f(x) = e^{\lambda t} \phi(x) \nu(f) (1+ C^{\eqref{asp:H2}}_{t,x,f})$ for some $C^{\eqref{asp:H2}}_{t,x,f}\in \mathbb R$ with
\[
	\sup_{x\in E, f\in L_1^+(\nu)} |C^{\eqref{asp:H2}}_{t,x,f}|
	< \infty
	\text{ and }
	\lim_{t\to \infty} \sup_{x\in E, f\in L_1^+(\nu)} |C^{\eqref{asp:H2}}_{t,x,f}|
	= 0;
\]
\end{minipage}
\end{equation}
	and
\begin{equation} \label{asp:H3} \tag{H3}
	\mathrm P_\nu(X_t = \mathbf 0)>0, \quad t> 0.
\end{equation}

	Our first result is about the Q-process of our superprocess $X$.
	Under Assumption \eqref{asp:H1}, one can verify that, for any $\mu \in \mathcal M_f(E)$, $(e^{-\lambda t}  X_t(\phi))_{t\geq 0}$ is a non-negative martingale under $\mathrm P_\mu$.
	In fact, for each $0\leq s\leq t< \infty$,
\begin{equation} \label{eq:M.25}
	\mathrm P_\mu[e^{-\lambda t}X_t(\phi)|\mathscr F_s]
	\overset{\text{Markov}} = e^{-\lambda t} \mathrm P_{X_s}[X_{t-s}(\phi)]
	\overset{\eqref{eq:M.2}}= e^{-\lambda t}X_s(P_{t-s}^\beta \phi)
	\overset{\eqref{asp:H1}}=e^{-\lambda s}X_s(\phi).
\end{equation}
	\begin{lem} \label{thm:T}
	Suppose that \eqref{asp:H1} hold. Then for each $\mu \in \mathcal M_f^o(E)$, 
	%RS Do we need the uniqueness? If yes, we need to say a few words about the uniqueness in the proof.
	%ZS We only need to mentionthat the uniquness is obvious.
	%there exists a unique probability measure $\widetilde {\mathrm P_\mu}$  
	%ZS: there exists a  probability measure $\widetilde {\mathrm P}_\mu$  
	there exists a unique probability measure $\widetilde {\mathrm P_\mu}$
	on $\mathbb W$ such that
\begin{equation} \label{eq:M.3}
	\frac{{\mathrm d}\widetilde{\mathrm P}_\mu|_{\mathscr F_t}}
	{{\mathrm d}\mathrm P_\mu|_{\mathscr F_t}}
	=\frac{e^{-\lambda t}X_t(\phi)}{\mu(\phi) },
	\quad t\geq 0.
\end{equation}
\end{lem}

	(We mention in passing here tat $\widetilde {\mathrm P}_\mu$ is not equal to the measure defined by $\int_E\mu(\mathrm dx)\widetilde {\mathrm P}_x(\cdot)$.)
	This kind of martingale measure transformation for branching processes and measure-valued branching processes have been widely studied.
	We refer to the earlier papers \cite{EnglanderKyprianou2004Local,Evans1993Two,RoellyRouault1989Processus,Penisson2010Conditional} and the references therein.
	For recent developments, we refer to \cite{ChampagnatRoelly2008Limit,RenSongSun2020Spine,RenSongZhang2018Williams}.
	It is known that the process $\{(X_t)_{t\geq 0}; \widetilde{\mathrm P}_{\mu}\}$ can be characterized by the so called spine decomposition theorem.
	\begin{thm} \label{thm:Q}
	Suppose that \eqref{asp:H1}, \eqref{asp:H2} and \eqref{asp:H3} hold, then
\[
	\lim_{s \rightarrow \infty} \mathrm P_\mu(A |X_s\neq \mathbf 0)=
	\widetilde{\mathrm P}_\mu(A),
	\qquad \mu \in \mathcal M_f^o(E), A\in \bigcup_{t\geq 0}\mathscr F_t.
\]
\end{thm}

	Our second result is about the asymptotic behavior of the extinction probability.
	Define
\begin{equation}
	l(x)
	:= \int_{(1,\infty)} r\ln r\pi^\phi(x, {\mathrm d}r),\quad x \in E,
\end{equation}
	where $\pi^\phi(x, \mathrm dr)$ is the kernel from $E$ to $(0,\infty)$ given by
\begin{equation}
	\int_{(0,\infty)} f(r)\pi^\phi(x,{\mathrm d}r)
	=\int_{(0,\infty)} f\big(r\phi(x)\big)\pi(x, {\mathrm d}r),
	\quad x\in E, f\in\mathcal B_b((0,\infty),\mathbb R_+).
\end{equation}

\begin{thm} \label{thm:E}
	Suppose that \eqref{asp:H1}, \eqref{asp:H2} and \eqref{asp:H3} hold.	
	Then there exists $k\in [0,\infty)$ such that for any $\mu \in \mathcal M_f^o(E)$,
\begin{equation}
	\lim_{t\rightarrow\infty} e^{-\lambda t}\mathrm P_\mu(X_t \neq \mathbf 0)
	=k\mu(\phi).
\end{equation}
	Moreover, $k>0$ if and only if $\nu(l)<\infty$.
\end{thm}

	Our third result is about the moment property of the Yaglom limit and the quasi-stationary distribution of $X$.
	For any probability measure $\mathbf P$ on $\mathcal M_f(E)$, define
\[
	(\mathbf P\mathrm P)(\cdot)
	:= \int_{\mathcal M_f(E)} \mathrm P_\mu(\cdot)\mathbf P({\mathrm d}\mu).
\]
	Any probability measure $\mathbf P$ on $\mathcal M_f^o(E)$ will also be understood as its unique extension on $\mathcal M_f(E)$ with  $\mathbf P(\{\mathbf 0\}) = 0$.
 We say a probability measure $\mathbf Q$ on $\mathcal M^o_f(E)$ is a \emph{quasi-stationary distribution} (QSD) of $X$, if
\[
	(\mathbf Q \mathrm P) \left( X_t \in B \middle | X_t \neq \mathbf 0 \right)
	= \mathbf Q(B),
	\quad t\geq 0, B \in \mathcal B(\mathcal M_f(E)).
\]
	According to \cite[(1.5)]{LiuRenSongSun2020}, if a probability measure $\mathbf Q$ on $\mathcal M_f^o(E)$ is a QSD of $X$, then there exists an $r\in (-\infty, 0)$ such that $(\mathbf Q\mathrm P)(X_t \neq \mathbf 0) = e^{rt}$ for all $t\geq 0$; and in this case, we call $r$ the \emph{mass decay rate} of $\mathbf Q$.
	It is proved in \cite[Theorem 1.2]{LiuRenSongSun2020} that, under Assumption \eqref{asp:H1}, \eqref{asp:H2} and \eqref{asp:H3},
	(1) for each $r\in [\lambda, 0)$, there exists a unique QSD $\mathbf Q_r$ of $X$ with mass decay rate $r$;
	and (2) for each $r\in (-\infty, \lambda)$, there is no QSD for $X$ with mass decay rate $r$.
	In particular, by \cite[Theorem 1.1, Proposition 1.7]{LiuRenSongSun2020}, $\mathbf Q_\lambda$ is the \emph{Yaglom limit} of $X$, in the sense that
\begin{equation}
	\mathrm P_\mu(X_t \in \cdot | X_t \neq \mathbf 0)
	\xrightarrow[t\to \infty]{\text{weakly}} \mathbf Q_\lambda,
	\quad \mu \in \mathcal M_f^o(E).
\end{equation}

\begin{thm}\label{thm:L}
	Suppose that Assumptions \eqref{asp:H1}, \eqref{asp:H2} and \eqref{asp:H3} hold.
	Then (1)
	for any $r\in [\lambda, 0)$ and $\gamma \in (0, \frac{r}{\lambda})$, it holds that $\int_{{\mathcal M}_f(E)}\mu(\phi)^\gamma\mathbf Q_r({\mathrm d}\mu)<\infty$;
	(2) If $r=\lambda$ (which is equivalent to $\frac{r}{\lambda}=1$), then $\int_{{\mathcal M}_f(E)}\mu(\phi)\mathbf Q_\lambda({\mathrm d}\mu) = k^{-1}$ where $k$ is  the constant in Theorem \ref{thm:E};
	(3) If $r\in (\lambda, 0)$ 
	%(which is implies that $\frac{r}{\lambda}\in(0,1)$),  then $ \mathbf Q_r(\langle\phi, X^{(r)}\rangle^{r/\lambda})=\infty.$
	(which is implies that $\frac{r}{\lambda}\in(0,1)$) and $U^{(r)}$ is an $\mathcal M_f(E)$-valued random variable with distribution $ \mathbf Q_r$, then $ \mathbf Q_r(\langle\phi, U^{(r)}\rangle^{r/\lambda})=\infty.$
\end{thm}

	Our fourth result characterizes the invariant distribution for the Q-process of $X$.
	A probability $\mathbf Q$ on $\mathcal M_f(E)$ is called an \emph{invariant distribution} of the Q-process of $X$ if
\[
	(\mathbf Q\widetilde{\mathrm P})(X_t \in \cdot )
	=\mathbf Q(\cdot),	\quad t\geq 0.
\]

\begin{thm}\label{thm:I}
	Suppose that \eqref{asp:H1}, \eqref{asp:H2} and \eqref{asp:H3} hold.
\begin{enumerate}
\item
	If $\nu(l)<\infty$, then the Q-process of $X$ has an invariant distribution $\mathbf Q$ given by
\[
	\int_{{\mathcal M}_f(E)} e^{-\mu(f)}\mathbf Q(\mathrm d\mu)
	=\frac{\int_{{\mathcal M}_f(E)}\mu(\phi)e^{-\mu(f)}\mathbf Q_\lambda({\mathrm d}\mu)} {\int_{{\mathcal M}_f(E)}\mu(\phi)\mathbf Q_\lambda({\mathrm d}\mu)}, \quad f\in \mathcal B_b(E, \mathbb R_+).
\]	
Moreover, for each $\mu\in\mathcal M^o_f(E)$, we have
\[
	\widetilde{\mathrm P}_\mu(X_t \in \cdot ) 
	\xrightarrow[t\to \infty]{weakly} {\mathbf Q}(\cdot).
\]
\item
	If $\nu(l) = \infty$, then for each $\mu \in \mathcal M^o_f(E)$, we have
\begin{align}
	& \widetilde{\mathrm P}_\mu\big(X_t(\phi) > C\big) 
	\xrightarrow[t\to \infty]{} 1, \quad C\geq 0.
\end{align}
\end{enumerate}
\end{thm}

\section{Martingale transformation} \label{sec:T}
%RS We can think about simplifying (or deleting) this section by using Section 5 of \cite{Fitz}. The problem is that Fitzsimmons did not mention clearly what kind of cemetery point he was using. He only said that ``introduce a cemetry point as usual''.
%ZS Fitzsimmons can only mean adding isolated point as cemetry point since the space he been working with is not assumed to be locally bounded.
	In this section, we assume that \eqref{asp:H1} holds, and we will give the proof of Lemma \ref{thm:T}.
	Note that $\xi$ is a $E_\partial$-valued Hunt process.
	Extend $\psi$ as a function $ \psi^\partial$ on $E_\partial \times \mathbb R_+$ by
\[
	\psi^\partial(x, z) = \begin{cases}
	\psi(x,z), &\quad x \in E, z\in \mathbb R_+,
	\\ 0, &\quad x=\partial, z\in \mathbb R_+.
	\end{cases}
\]
	For any $f \in \mathcal B_b(E_\partial,\mathbb R_+)$,  according to \cite[Proposition 2.20]{Li2011Measure-valued}, there is a unique locally bounded non-negative map $(t,x)\mapsto V^\partial_tf(x)$ on $\mathbb R_+\times E_\partial$ such that
\begin{equation}
	V_t^\partial f(x) + \Pi_x\Big[\int_0^t \psi^\partial\big(\xi_s, V^\partial_{t-s} f(\xi_s)\big) {\mathrm d}s\Big]
	= \Pi_x[f(\xi_t)],
	\quad t\in \mathbb R_+, x\in E_\partial.
\end{equation}
	Here, the local boundedness of the map $(t,x) \mapsto V_t^\partial f(x)$ means that  for any $T\geq 0$,
\[
	\sup_{0\leq t\leq T, x\in E_\partial} V^\partial_tf(x) < \infty.
\]
	According to \cite[Proposition 2.21 and Theorem 5.11]{Li2011Measure-valued}, there exists an $\mathcal M_f(E_\partial)$-valued Hunt process $X^\partial =\{(X^\partial_t)_{t\geq 0}; (\mathrm P^\partial_\mu)_{\mu \in \mathcal M_f(E_\partial)}\}$ such that
\begin{equation}
	\mathrm P^\partial_\mu[e^{- X^\partial_t(f)}]
	= e^{- \mu(V^\partial_tf)},
	\quad \mu\in \mathcal M_f(E_\partial), t\geq 0, f \in \mathcal B_b(E_\partial,\mathbb R_+).
\end{equation}
	This process $X^\partial$ is known as a $(\xi, \psi^\partial)$-superprocess.

	We can and will assume that $X^\partial$ is \emph{canonical}  i.e.
	(1) $(X^\partial_t)_{t\geq 0}$ is the coordinate process of $\mathbb D^\partial$, the space of $\mathcal M_f(E_\partial)$-valued c\`adl\`ag paths on $\mathbb R_+$; and 
	(2) $\mathrm P^\partial_\mu(\mathrm dw)$ is a probability transition kernel from $\mathcal M_f(E_\partial)$ to $\mathbb D^\partial$.
	We will use $(\mathscr F^\partial_t)_{t\geq 0}$ to denote the natural filtration on $\mathbb D^\partial$.
	Define a Feynman-Kac semigroup of $\xi$ by
\begin{align} \label{eq:T.01}
	P_t^{\beta,\partial} f(x)
	:= \Pi_x\Big[\exp\Big\{\int_0^t \beta(\xi_r) \mathbf 1_{\xi_r\in E} {\mathrm d}r \Big\} f(\xi_t) \Big],
	\quad f\in \mathcal B_b(E_\partial,\mathbb R), t\geq 0, x\in E_\partial.
\end{align}
	It is known (see \cite[Proposition 2.27]{Li2011Measure-valued}) that
\begin{equation} \label{eq:T.02}
	\mathrm P_\mu^\partial [X^\partial_t(f)] = \mu (P_t^{\beta,\partial} f),
	\quad t\geq 0, f \in \mathcal B_b(E_\partial,\mathbb R), \mu \in \mathcal M_f(E_\partial).
\end{equation}
	Note that for each $t\geq 0$ and $x\in E_\partial$,
\begin{align}
	& P_t^{\beta,\partial} (\phi \mathbf 1_E) (x)
	\overset{\eqref{eq:T.01}}= \Pi_x\Big[\exp\Big\{\int_0^t \beta(\xi_r) \mathbf 1_{\xi_r\in E} {\mathrm d}r \Big\} \phi(\xi_t) \mathbf 1_{\xi_t \in E} \Big]
	\\\label{eq:T.03}&\overset{\eqref{eq:M.15}} = \mathbf 1_{x\in E}P_t^\beta \phi (x)
	\overset{\eqref{asp:H1}} = e^{\lambda t} \phi (x)\mathbf 1_{x\in E}.
\end{align}
	Define a map $\Gamma: \mu \mapsto \Gamma \mu$ from $\mathcal M_f(E_\partial)$ to $\mathcal M_f(E)$ by taking
\[
	(\Gamma \mu)(B) = \mu(B), \quad B\in \mathscr B(E).
\]
	
\begin{lem} \label{thm:T.1}
	For each $\mu \in \mathcal M_f(E_\partial)$,
$
	\{(\Gamma X^\partial_t)_{t\geq 0}; \mathrm P_\mu^\partial\} \overset{\text{f.d.d.}}= \{(X_t)_{t\geq 0}; \mathrm P_{\Gamma\mu}\}.
$
\end{lem}
\begin{proof}	
%ZS: The proof of this lemma will be changed accoding to Yanxia's segguestion.
	Fix an arbitrary $\mu\in \mathcal M_f(E_\partial)$, $T\geq 0$, a finite measure $\lambda$ on $[0,T]$, and a bounded non-negative Borel function $f:(s,x)\mapsto f_s(x)$ on $[0,T]\times E$.
	According to \cite[Theorem 5.15]{Li2011Measure-valued}, we know that
\begin{equation} \label{eq:T.05}
\mathrm P_{r,\Gamma\mu}\Big[e^{-\int_{[r,T]} X_{s-r}(f_s)\lambda(\mathrm ds)  } \Big] = e^{-(\Gamma\mu)(u_r)}, \quad 0\leq r\leq T,
\end{equation}
	where $u:(s,x)\mapsto u_s(x)$ is the unique bounded non-negative solution on $[0,T]\times E$ of
\begin{equation} \label{eq:T.1}
	u_r(x) + \int_r^t \Pi_{r, x}\big[\psi\big(\xi_{s-r}, u_s(\xi_{s-r})\big)\mathbf 1_{\xi_{s-r}\in E}\big] \mathrm ds
	= \int_{[r,t]} \Pi_{r, x} \big[f_s(\xi_{s-r}) \mathbf 1_{\xi_{s-r} \in E}\big] \lambda(\mathrm ds).
\end{equation}
	Extend $f$ and $u$ as bounded non-negative functions $f^\partial$ and $u^\partial$ on $[0,T]\times E_\partial$ by setting
\[
	f^\partial_s(x) = f_s(x) \mathbf 1_{x\in E}, \quad u^\partial_s(x) = u_s(x) \mathbf 1_{x\in E},\quad s\in [0,T], x\in E_\partial.
\]
	Now we can verify from \eqref{eq:T.1} that $u^\partial:(s,x)\mapsto u^\partial_s(x)$ is a bounded non-negative solution on $[0,T]\times E_\partial$ of
\begin{equation}
	u_r^\partial(x) + \int_r^t \Pi_{r, x}\big[\psi^\partial\big(\xi_{s-r}, u^\partial_s(\xi_{s-r})\big)\big] \mathrm ds
	= \int_{[r,t]} \Pi_{r, x} \big[f_s^\partial(\xi_{s-r})\big] \lambda(\mathrm ds).
\end{equation}
	From this and \cite[Theorem 5.15]{Li2011Measure-valued}, we have that
\begin{equation} \label{eq:T.15}
	\mathrm P_{r, \mu^\partial} 
	[e^{-\int_{[r,T]} X^\partial_{s-r}(f^\partial_s) \lambda(\mathrm ds)}]
	= e^{-\mu(u^\partial_r)},
	\quad 0\leq r\leq T.
\end{equation}
Thus
\begin{align}
	&\mathrm P_{\Gamma\mu}\Big[e^{-\int_{[0,T]} X_s(f_s)\lambda(\mathrm ds)  } \Big]
	\overset{\eqref{eq:T.05}}= e^{-(\Gamma \mu)(u_0)}
	= e^{-\mu(u_0^\partial)}
	\\&\overset{\eqref{eq:T.15}}= \mathrm P_\mu^\partial \Big[ e^{-\int_{[0,T]} X_s^\partial(f_s^\partial)\lambda(\mathrm ds)} \Big]
	= \mathrm P_\mu^\partial \Big[ e^{-\int_{[0,T]} (\Gamma X_s)(f_s)\lambda(\mathrm ds)} \Big].
\end{align}
	The desired result then follows from this and the fact that $\lambda$ and $f$ have been chosen arbitrarily.
\end{proof}

\begin{lem} \label{thm:T.2}
	For each $\mu \in \mathcal M_f(E_\partial)$, $\mathrm P_\mu^\partial$-almost surely, $(\Gamma X^\partial_t)_{t\geq 0}$ is an $\mathcal M_f(E)$-valued right continuous process.
\end{lem}
\begin{proof}
	Fix $\mu\in \mathcal M_f(E_\partial)$.
	Since $X^\partial$ is a Hunt process, we have $\mathrm P_\mu^\partial(\Omega_0) = 1$ where
\[
	\Omega_0 := \{t\mapsto X_t^\partial \text{ is an $\mathcal M_f(E_\partial)$-valued right continuous process on $[0,\infty)$}\}.
\]
	According to \cite[Theorem A.20]{Li2011Measure-valued}, we know that $\mathbf 1_E$ is finely continuous relative to $\xi$.
	According to \cite[Proposition 5.9]{Li2011Measure-valued}, we know that
$
	\mathrm P_\mu(\Omega_1) = 1
$
	where
\[
	\Omega_1 := \{t \mapsto X_t^\partial(\mathbf 1_E) \text{ is an $\mathbb R$-valued right continuous process on $[0,\infty)$}\}.
\]
	To finish the proof, we only have to show
\begin{equation} \label{eq:T.2}
	\Omega_0\cap \Omega_1
	\subset \{t\mapsto \Gamma X_t^\partial \text{ is an $\mathcal M_f(E)$-valued right continuous process on $[0,\infty)$}\}.
\end{equation}
	In fact it is clear that in $\Omega_0$ for each $t\geq 0$ and compactly supported continuous function $f$ on $E$,
\begin{equation}\label{eq:T.21}
	(\Gamma X^\partial_t)(f) = X^\partial_t(f\mathbf 1_E) = \lim_{r\downarrow t} X^\partial_t(f\mathbf 1_E) = \lim_{r\downarrow t} (\Gamma X^\partial_t)(f).
\end{equation}
	In $\Omega_1$, we have for each $t\geq 0$,
\begin{align} \label{eq:T.22}
	&(\Gamma X^\partial_t)(\mathbf 1_E)
	= X^\partial_t(\mathbf 1_E)
	= \lim_{r\downarrow t} X^\partial_r (\mathbf 1_{E})
	= \lim_{r\downarrow t} (\Gamma X^\partial_r)(\mathbf 1_E).
\end{align}
	The desired result \eqref{eq:T.2} then follows from \eqref{eq:T.21}, \eqref{eq:T.22} and Lemma \ref{thm:A.1}.
\end{proof}

\begin{lem} \label{thm:T.3}
	For each $\mu \in \mathcal M_f(E_\partial)$ with $\mu(\phi\mathbf 1_E)>0$, there exists 
%RS do we need uniqueness. If yes, we need to say a few words about uniqueness in the proof.
%ZS I changed it back. We can cimply mention that the uniqueness is obvious.
	a unique measure $\widetilde {\mathrm P_{\mu}^\partial}$ on $\mathbb D^\partial$ such that
%	a  measure $\widetilde {\mathrm P_{\mu}^\partial}$ on $\mathbb D^\partial$ such that
\[
	\frac{\mathrm d \widetilde {\mathrm P_\mu^\partial} |_{\mathscr F_t^\partial}}{\mathrm d \mathrm P^\partial_\mu|_{\mathscr F_t^\partial}} = \frac{e^{-\lambda t}(\Gamma X^\partial_t)(\phi)}{\mu(\phi\mathbf 1_E)}, \quad t\geq 0.
\]
\end{lem}
\begin{proof}
	Thanks to \cite[Lemma 18.18]{Kallenberg2002Foundations}, we only have to verify that
\[
	\Big(\frac{e^{-\lambda t} (\Gamma X_t^\partial)(\phi)}{\mu(\phi \mathbf 1_E)}\Big)_{t\geq 0} \text{is a non-negative $\mathscr F_t^\partial$-martingale with mean $1$ under $\mathrm P_\mu^\partial$.}
\]
	In fact, for each $0\leq s\leq t<\infty$, we have
\begin{align}
	& \mathrm P_\mu^\partial  \Big[\frac{e^{-\lambda t} (\Gamma X_t^\partial)(\phi)}{\mu(\phi \mathbf 1_E)}\Big| \mathscr F^\partial_s\Big]
	= \frac{e^{-\lambda t}}{\mu(\phi \mathbf 1_E)} \mathrm P_\mu^\partial  [X_t^\partial(\phi \mathbf 1_E) | \mathscr F^\partial_s]
	\overset{\text{Markov}}= \frac{e^{-\lambda t}}{\mu(\phi \mathbf 1_E)} \mathrm P^\partial_{X_s^\partial}[X_{t-s}^\partial (\phi \mathbf 1_E)]
	\\ & \overset{\eqref{eq:T.02}}= \frac{e^{-\lambda t}}{\mu(\phi \mathbf 1_E)} X_s^\partial \big(P^{\beta, \partial}_{t-s} (\phi\mathbf 1_E)\big)
	\overset{\eqref{eq:T.03}}= \frac{e^{-\lambda s}}{\mu(\phi \mathbf 1_E)} X_s^\partial (\phi\mathbf 1_E)
	= \frac{e^{-\lambda s} (\Gamma X_s^\partial)(\phi)}{\mu(\phi \mathbf 1_E)}.
	\qedhere
\end{align}
\end{proof}

\begin{proof}[Proof of Lemma \ref{thm:T}]
	Fix an arbitrary $\mu \in \mathcal M^o_f(E)$.
	There exists a unique $\mu^\partial \in \mathcal M_f(E_\partial)$ such that $\Gamma \mu^\partial = \mu$ and $\mu^{\partial}(\{\partial\}) = 0$.
	According to Lemma \ref{thm:T.2} and \ref{thm:T.3} we know that $\widetilde {\mathrm P_{\mu^\partial}^\partial}$-almost surely, $(\Gamma X^\partial_t)_{t\geq 0}$ has $\mathcal M_f(E)$-valued right continuous sample path.
	Therefore, there exists a probability measure $\widetilde{\mathrm P_\mu}$ on $\mathbb W$ such that
\[
	\{(\Gamma X^\partial_t)_{t\geq 0}; \widetilde {\mathrm P^\partial_{\mu^\partial}}\} \overset{\text{d}}\sim \widetilde {\mathrm P_\mu}
\]
	Now for any $n\in \mathrm N, t\geq 0$, any strictly increasing $(t_i)_{i=1}^n \subset [0,t]$, and any $(A_i)_{i=1}^n\subset \mathscr B(\mathcal M_f(E))$, we have
\begin{align}
	&\widetilde {\mathrm P_\mu}(\forall i\in \mathrm N\cap [1,n]: X_{t_i}\in A_i)
	= \widetilde {\mathrm P_{\mu^\partial}^\partial} (\forall i\in \mathrm N\cap [1,n]: \Gamma X^\partial_{t_i}\in A_i)
	\\ &\overset{\text{Lemma \ref{thm:T.3}}}= \mathrm P_{\mu^\partial}^\partial \Big[\frac{e^{-\lambda t}(\Gamma X^\partial_t)(\phi)}{\mu^\partial(\phi\mathbf 1_E)};\forall i\in \mathrm N\cap [1,n]: \Gamma X_{t_i}^{\partial}\in A_i\Big]
	\\ & \overset{\text{Lemma \ref{thm:T.1}}}= \mathrm P_\mu \Big[\frac{e^{-\lambda t}X_t(\phi)}{\mu(\phi)};\forall i\in \mathrm N\cap [1,n]: X_{t_i}\in A_i\Big].
\end{align}
	From this and the $\pi$-$\lambda$ theorem, we can verify that for each $t\geq 0$ and $B \in \mathscr F_t$, we have
\[
	\widetilde {\mathrm P_\mu}(B) = \mathrm P_\mu\Big[\frac{e^{-\lambda t}X_t(\phi)}{\mu(\phi)}; B\Big].
	\qedhere
\]
\end{proof}


\section{Proof of Theorem \ref{thm:Q}} \label{sec:Q}
	
	Throughout this section, we assume \eqref{asp:H1}, \eqref{asp:H2} and \eqref{asp:H3} hold.
	Let us first recall some known results from \cite{LiuRenSongSun2020}.
	It is easy to see that the operators $(V_t)_{t\geq 0}$
	given by \eqref{eq:M.1} can be extended uniquely to a family of operators $(\overline V_t)_{t\geq 0}$ on $\mathcal B(E,[0,\infty])$ such that for all $t\geq 0$, $f_n \uparrow f$ pointwisely in  $\mathcal B(E, [0,\infty])$ implies that $\overline V_tf_n \uparrow \overline V_tf$ pointwisely.
	With some abuse of notation, we still write $V_t = \overline V_t$ for $t\geq 0$, and call $(V_t)_{t\geq 0}$ \emph{the extended cumulant semigroup} of the superprocess $X$.
	Define $v_t = V_t(\infty  \mathbf 1_E)$ for $t\geq 0$, then it holds that
	\begin{equation} \label{eq:Q.04}
	\mathrm P_\mu (X_t = \mathbf 0)
	= e^{- \mu (v_t)},
	\quad \mu \in \mathcal M_f(E), t\geq 0.
	\end{equation}
	According to \cite[(1.10)]{LiuRenSongSun2020}) we have
\begin{equation}\label{eq:Q.05}
	\mu(v_t) > 0, \quad \mu \in \mathcal M_f^o(E), t \geq 0.
\end{equation}
	According to \cite[Proposition 1.3]{LiuRenSongSun2020}, for each $\mu\in \mathcal M_f(E)$ we have
\begin{equation} \label{eq:Q.055}
	\mu(v_t) \xrightarrow[t\to \infty]{} 0.
\end{equation}
	According to \cite[(1.17)]{LiuRenSongSun2020}, for each $t>0$ and $\mu \in \mathcal M_f^o(E)$ we have
\begin{equation} \label{eq:Q.056}
	\text{$\mu(v_t) = \nu(v_t)\mu(\phi) (1+C_{\mu,t}^{\eqref{eq:Q.056}})$ for some $C_{\mu,t}^{\eqref{eq:Q.056}}\in \mathbb R$ with $\lim_{t\to \infty}|C_{\mu,t}^{\eqref{eq:Q.056}}| = 0$.}
\end{equation}
	According to \cite[(2.14)]{LiuRenSongSun2020}, for each $t>0$ and $x\in E$ we have
\begin{equation}  \label{eq:Q.0565}
	\text{$v_t(x) = \phi(x) \nu(v_t) C_{t,x}^{\eqref{eq:Q.0565}}$ for some $C_{t,x}^{\eqref{eq:Q.0565}}\geq 0$ with $\varlimsup_{t\to \infty}\sup_{x\in E}C_{t,x}^{\eqref{eq:Q.0565}}<\infty$.}
\end{equation}
	According to \cite[(2.20)]{LiuRenSongSun2020}, for each $t\geq 0$, we have
\begin{equation}\label{eq:Q.057}
	\lim_{s\to \infty} \frac{\nu(v_s)}{\nu(v_{s-t})} = e^{\lambda t}.
\end{equation}

{\color{gray}
The following results will be used in later sections. The first result is basically
\cite[Proposition 1.3]{LiuRenSongSun2020}.
\begin{lem}\label{lem:extinc}
	Suppose that Assumptions \eqref{asp:H1}, \eqref{asp:H2} and \eqref{asp:H3} hold.
\begin{enumerate}
\item	
	For any  $ t>0$ and $ \mu \in \mathcal M_f(E)$, $\langle v_t(\cdot),\mu\rangle <\infty.$
\item	For any $\mu \in \mathcal M_f(E)$,
\[
	\lim_{t\rightarrow\infty}\langle v_t(\cdot),\mu\rangle=0.
\]
\end{enumerate}
\end{lem}

\begin{lem}\label{lem:ratio limit}
(1) For any $f\in\mathcal B(E,[0,\infty])$ with $\nu(f)>0$, and any $s>0$,
\begin{equation}\label{integ ratio limit}
\lim_{t\to\infty}\dfrac{\langle V_{t+s}f, \nu\rangle}{\langle V_{t}f, \nu\rangle}=e^{\lambda s}.
\end{equation}
(2) 
For any $f\in\mathcal B(E,[0,\infty])$ with $\nu(f)>0$, there exist positive constants $a,N,T>0$ such that
\begin{equation}\label{inequ:lower}
\langle V_{t}f, \nu\rangle\geq ae^{-Nt},\quad \mbox{for any }\ t>T.
\end{equation}
(3) For each $s\geq 0$,
\begin{equation} \label{one point ratio limit}
	\lim_{t\to \infty} \sup_{x\in E}\Big|\frac{v(t+s,x)}{\langle v(t,\cdot),\nu\rangle\phi(x) } - e^{\lambda s} \Big|=0.
\end{equation}
\end{lem}

\begin{proof}
(1) is basically \cite[(2.20)]{LiuRenSongSun2020}. (3) follows immediately from
\cite[(2.20) and Proposition 1.4]{LiuRenSongSun2020}. Now we prove (2).
It follows from\cite[(2.4)]{LiuRenSongSun2020} that for any $f\in\mathcal B(E,[0,\infty])$ with $\nu(f)>0$, $\langle V_{t}f, \nu\rangle>0$ for all $t>0$. Now by \cite[(2.20)]{LiuRenSongSun2020},
$$
\langle V_{s+t}f, \nu\rangle=\langle V_{s}f, \nu\rangle\exp\{\lambda t(1+C_{s, t, f})\}
$$
for some $C_{s, t, f}$ with $\lim_{s\to\infty}\sup_{t\ge 0, f\in \mathcal B(E,[0,\infty])}|C_{s, t, f}|=0$. Take $s_0$ large enough so that $\sup_{t\ge 0, f\in \mathcal B(E,[0,\infty])}|C_{s_0, t, f}|\le \frac12$. The desired result follows immediately.
\end{proof}
}

\begin{proof}[Proof of Theorem \ref{thm:Q}]
	Fix arbitrary $\mu\in \mathcal M^o_f(E)$, $t\ge 0$ and $A\in\mathscr F_t$.
	For $s>t$,
\begin{equation} \label{eq:Q.06}
	\mathrm P_\mu(A|X_s\neq \mathbf 0)
	=\frac{\mathrm P_\mu(A, X_s \neq \mathbf 0)}{\mathrm P_\mu(X_s\neq \mathbf 0)}
	\overset{\text{Markov}}=\frac{\mathrm P_\mu\big(\mathrm P_{X_t}(X_{s-t} \neq \mathbf 0);A\big)}{\mathrm P_\mu(X_s\neq \mathbf 0)}.
\end{equation}
		We claim that
\begin{equation} \label{eq:Q.1}
	\frac{\mathrm P_{X_t}(X_{s-t} \neq \mathbf 0)}{\mathrm P_\mu(X_{s} \neq \mathbf 0)}
	\xrightarrow[s\to \infty]{} \frac{e^{-\lambda t}X_t(\phi)}{\mu(\phi) },
	\quad \mathrm P_\mu\text{-a.s.}
\end{equation}
	and that there exist (deterministic) $c,s_0>0$ such that for any $s\geq s_0$,
\begin{equation} \label{eq:Q.2}
	\frac{\mathrm P_{X_t}(X_{s-t} \neq \mathbf 0)}{\mathrm P_\mu(X_{s} \neq \mathbf 0)}
	\leq cX_t(\phi), \quad \mathrm P_\mu\text{-a.s.}
\end{equation}
With these claims, we can use \eqref{eq:Q.1}, \eqref{eq:Q.2} and the dominated convergence theorem (DCT) to get
\begin{align}
&  \mathrm P_\mu(A|X_s\neq \mathbf 0)
	\overset{\eqref{eq:Q.06}}=  \frac{\mathrm P_\mu\big(\mathrm P_{X_t}(X_{s-t} \neq \mathbf 0);A\big)}{\mathrm P_\mu(X_s\neq \mathbf 0)}
	\\&\xrightarrow[s\to \infty]{\text{DCT}} \mathrm P_\mu\Big[\frac{e^{-\lambda t}X_t(\phi)}{\mu(\phi) }; A\Big]
	\overset{\eqref{eq:M.3}} = \widetilde{\mathrm P}_\mu[A].
\end{align}

    We now prove  claims \eqref{eq:Q.1} and \eqref{eq:Q.2}.
	We first prove \eqref{eq:Q.1}.
	Notice that for any $\bar \mu\in \mathcal M_f(E)$,
\begin{align}
	&\lim_{s\rightarrow\infty}\dfrac{ \bar \mu(v_{s-t}) }{ \mu(v_s) }
	\overset{\eqref{eq:Q.056}}=\lim_{s\to \infty} \frac{\nu(v_{s-t})\bar \mu (\phi) (1+C_{\bar \mu,s-t}^{\eqref{eq:Q.056}})}{\nu(v_s)\mu(\phi)(1+C_{\mu,s}^{\eqref{eq:Q.056}})}
	\overset{\eqref{eq:Q.056}}= \frac{\bar \mu(\phi)}{\mu(\phi)}\lim_{s\to \infty} \frac{\nu(v_{s-t}) }{\nu(v_s)}
	\\\label{eq:Q.25}&\overset{\eqref{eq:Q.057}}= \frac{e^{-\lambda t} \bar \mu(\phi)}{\mu(\phi)}.
\end{align}
	Thus we have that, $\mathrm P_\mu$-almost surely,
\begin{equation}
	\lim_{s\to\infty}\frac{\mathrm P_{X_t}(X_{s-t}\neq \mathbf 0)}{\mathrm P_\mu(X_s\neq \mathbf 0)}
	\overset{\eqref{eq:Q.04}}=\lim_{s\to\infty}\frac{1-e^{- X_t(v_{s-t}) }}{1-e^{- \mu(v_s) }}
	\overset{\eqref{eq:Q.05},\eqref{eq:Q.055}}=\lim_{s\rightarrow\infty}\dfrac{ X_t(v_{s-t}) }{ \mu(v_s) }
	\overset{\eqref{eq:Q.25}}= \frac{e^{-\lambda t} X_t(\phi)}{\mu(\phi)},
\end{equation}
	which implies \eqref{eq:Q.1} holds.

	Now we prove \eqref{eq:Q.2}.
	Notice that, $\mathrm P_\mu$-almost surely,
\begin{align}
	& \frac{\mathrm P_{X_t}(X_{s-t}\neq \mathbf 0)}{\mathrm P_\mu(X_s\neq \mathbf 0)}
	\overset{\eqref{eq:Q.04}}= \frac{1-e^{- X_t(v_{s-t}) }}{1-e^{- \mu(v_s) }}
	\leq \frac{X_t(v_{s-t}) }{1-e^{- \mu(v_s) }}
	\\\label{eq:Q.3}&\overset{\eqref{eq:Q.05},\eqref{eq:Q.055}}= \frac{X_t(v_{s-t}) }{ \mu(v_s)} C_s^{\eqref{eq:Q.3}},
	\\& \quad \text{for some deterministic $C_s^{\eqref{eq:Q.3}}>0$ with $\lim_{s\to \infty} C_s^{\eqref{eq:Q.3}} = 1$,}
	\\\label{eq:Q.4}&\overset{\eqref{eq:Q.0565}}\leq \frac{X_t(\phi) \nu(v_{s-t}) \sup_{x\in E} C_{s-t,x}^{\eqref{eq:Q.0565}}}{ \mu(v_s)} C_s^{\eqref{eq:Q.3}}.
\end{align}
	Then notice that
\begin{align}
&  \varlimsup_{s\to \infty}\frac{\nu(v_{s-t}) \sup_{x\in E} C_{s-t,x}^{\eqref{eq:Q.0565}}}{ \mu(v_s)} C_s^{\eqref{eq:Q.3}}
	\overset{\eqref{eq:Q.25}} = \frac{e^{-\lambda t} \nu(\phi)}{\mu(\phi)} \varlimsup_{s\to \infty}C_s^{\eqref{eq:Q.3}} \sup_{x\in E}C_{s-t,x}^{\eqref{eq:Q.0565}}
	\\\label{eq:Q.5}&\overset{\eqref{eq:Q.3}} = \frac{e^{-\lambda t} \nu(\phi)}{\mu(\phi)} \varlimsup_{s\to \infty} \sup_{x\in E}C_{s-t,x}^{\eqref{eq:Q.0565}}
	\overset{\eqref{eq:Q.0565}}< \infty.
\end{align}
	The desired result \eqref{eq:Q.2}  follows from \eqref{eq:Q.4} and \eqref{eq:Q.5}.
\end{proof}

	\section{Proof of Theorem \ref{thm:E}} \label{sec:E}
	In this section, we assume that \eqref{asp:H1}, \eqref{asp:H2} and \eqref{asp:H3} hold.
	In order to prove Theorem \ref{thm:E}, we will need a special case for the spine decomposition theorem for superprocess $X$.
	The spine decomposition theorem in this specific case will be presented in Lemma \ref{thm:E.2} below.
	
	We first recall the \emph{Kuznetsov  measure} of the superprocess $X$.
	According to \cite[Proposition 1.3]{LiuRenSongSun2020} we have
\begin{equation} \label{eq:E.1}
	v_t(x)<\infty, \quad t>0, x\in E.
\end{equation}
	Therefore
\begin{align} \label{eq:E.11}
	\mathrm P_{\delta_x}(X_t = \mathbf 0) \overset{\eqref{eq:Q.04}}= e^{-v_t(x)}\overset{\eqref{eq:E.1}} > 0, \quad t>0, x\in E.
\end{align}
	According to \cite[Section 8.4]{Li2011Measure-valued} and \eqref{eq:E.11}, there is a unique family of $\sigma$-finite measures
	$(\mathrm N_x)_{x\in E}$
	on $\mathbb W$ such that
	(1) $\mathrm N_x(w_0 \neq \mathbf 0) = 0$ for each $x\in E$;
	(2) $\mathrm N_x (\forall t > 0, w_t =\mathbf 0) =0$ for each $x\in E$;
	and (3) for each $\mu \in \mathcal M_f(E)$, if $\mathcal N$ is a Poisson random measure on $\mathbb W$ with intensity
	\[
%ZS: I changed this notation back
	(\mu\mathrm N)(\mathrm dw):= 
%	\mathrm N_\mu(\mathrm dw):= 
	\int_{x\in E} \mathrm N_x(\mathrm dw)\mu(\mathrm dx), \quad w\in \mathbb W,
	\]
	then
	\begin{equation} \label{eq:E.12}
	\{(X_t)_{t> 0};\mathrm P_\mu\}
	\overset{\text{f.d.d.}}= \Big(\int_{\mathbb W} w_t\mathcal N(\mathrm dw)\Big)_{t> 0}.
	\end{equation}
	This family of measure $(\mathrm N_x)_{x\in E}$ is known as the Kuznetsov measures of $X$.
	It can be verified using Campbell's formula for Poisson random measures that for any $\mu\in \mathcal M_f(E)$ and $t>0$,
\begin{equation} \label{eq:E.13}
	\mu(P_t^\beta f)
	\overset{\eqref{eq:M.2}}= \mathrm P_\mu[X_t(f)]
%ZS: I changed it back
	\overset{\eqref{eq:E.12}}=(\mu \mathrm N) [w_t(f)],
%	\overset{\eqref{eq:E.12}}=\mathrm N_\mu [w_t(f)],
	\quad f\in \mathcal B_b(E, \mathbb R_+)
\end{equation}
	and
\begin{equation} \label{eq:E.14}
%ZS I changed it back
	(\mu\mathrm N) (w_t\neq \mathbf 0)
% \mathrm N_\mu (w_t\neq \mathbf 0)
	\overset{\eqref{eq:E.12}} = - \log \mathrm P_{\mu}(X_t = \mathbf 0).
\end{equation}

%	It follows from \eqref{eq:M.25} and \eqref{eq:E.13} that for any $\mu\in \mathcal M^o_f(E)$ there exists a probability measure $\widetilde {\mathrm N}_\mu$ on $\mathbb W$ such that
	It follows from \eqref{eq:M.25} and \eqref{eq:E.13} that for any $\mu\in \mathcal M^o_f(E)$ there exists a probability measure $\widetilde {\mu \mathrm N}^{(t)}$ on $\mathbb W$ such that
	\begin{align}\label{eq:E.15}
%ZS I chnaged it back
	& \frac{\widetilde {\mu \mathrm N}^{(t)}(\mathrm dw) }{(\mu \mathrm N)(\mathrm dw)}
	= \frac{e^{-\lambda t}w_t(\phi)}{\mu(\phi)}, \quad w\in \mathbb W.
%	& \frac{\widetilde {\mathrm N}_\mu|_{\mathscr F_t}(\mathrm dw) }{\mathrm N_\mu|_{\mathscr F_t}(\mathrm dw)}
%	= \frac{e^{-\lambda t}w_t(\phi)}{\mu(\phi)}, \quad t\ge 0, w\in \mathbb W.
	\end{align}
%new
%ZS I deleted it.
%Again we caution that $\widetilde {\mathrm N}_\mu$ is not equal to the measure defined by 
%$\int_E\mu(\mathrm dx)\widetilde {\mathrm N}_x(\cdot)$.
%end new
	
	Recall that $\{(\xi_t)_{t\geq 0}; (\Pi_x)_{x\in E}\}$ is the spatial motion of the superprocess.
	Let  $(\mathscr F_t^{\xi})_{t\geq 0}$ be the natural filtration of process $(\xi_t)_{t\geq 0}$.
%ZS: I changed it back
	For any probability measure $\mu$ on $E$, define
\begin{align}
	& (\mu \Pi)[\cdot] := 
	\int_{x\in E} \Pi_x[\cdot]\mu(\mathrm dx)
\end{align}
and let the probability $\widetilde {\mu \Pi}$ be the Doob's $h$-transform of $\mu\Pi$ given by
\begin{align}
	&  \frac{\mathrm d (\widetilde{\mu \Pi})|_{\mathscr F^\xi_t}}{\mathrm d (\mu \Pi)|_{\mathscr F^\xi_t}}
	=\frac{e^{\int_0^t \beta(\xi_s)ds}\phi(\xi_t) \mathbf 1_{\{t<\zeta\}}}{e^{\lambda t}\mu(\phi)},
	\quad t\geq 0.
\end{align}
%For any $x\in E$, we let $\widetilde{\Pi}_x$  the Doob $h$-transform of $\Pi_x$ defined by 
%$$
%\frac{\mathrm d \widetilde{\Pi}_x|_{\mathscr F^\xi_t}}{\mathrm d \Pi_x|_{\mathscr F^\xi_t}}
%	=\frac{e^{\int_0^t \beta(\xi_s)ds}\phi(\xi_t) \mathbf 1_{\{t<\zeta\}}}{e^{\lambda t}\phi(x)},
%	\quad t\geq 0.
%$$
%For any $\mu\in \mathcal M_f(E)$, we use the usual notation
%$$
%\widetilde{\Pi}_\mu(\cdot)=\mu(E)^{-1}\int_E\widetilde{\Pi}_x(\cdot)\mu(\mathrm dx).
%$$
%
%ZS: let us use $\widetilde \nu$ instead of \bar \mu. 
%moved from below
We define
\[
%ZS \bar \nu(\mathrm dx) := \phi(x) \nu(\mathrm dx), \quad x\in E.
\widetilde \nu(\mathrm dx) := \phi(x) \nu(\mathrm dx), \quad x\in E.
\]
%end move

\begin{lem} \label{thm:E.15}
%ZS: I changed it back
	$\{(\xi_t)_{t\geq 0}; \widetilde{\nu \Pi}\}$ is a stationary process with $\xi_0 \overset{d}\sim \widetilde \nu$.
%	$\{(\xi_t)_{t\geq 0}; \widetilde{\Pi}_{\bar \nu}\}$ is a stationary Markov process. 
%	where $\bar \nu$ is a probability measure on $E$ given by 
%	\[\bar \nu(\mathrm dx) := \phi(x) \nu(\mathrm dx), \quad x\in E.\]
\end{lem}
\begin{proof}
%	TBD
For any $f\in \mathcal{B}_b(E, \mathbb R)$ and $t>0$, 
\begin{align*}
\widetilde{\Pi}_{\bar\nu}[f(\xi_t)]&=e^{-\lambda t}\int_E\nu(\mathrm dx)P^\beta_t(\phi f)(x)
=\int_E\nu(\mathrm dx)\phi(x)f(x).
\end{align*}
The proof is complete.
\end{proof}

{\color{gray}
%We need the spine decomposition with a general initial measure. So I an resurrecting the more general spine decomposition from an earlier version, with some modifications. We still need to unify the presentation with the special spine decomposition. I will leave this to Zhenyao
%reinstated from an earlier version with modifications
In spirit of \cite{RenSongSun2020Spine}, for any $\mu\in \mathcal M^0_f(E)$, let
$\{(\xi_t)_{t\geq 0}, \mathbf N^\xi; \mathrm Q_\mu\}$ be a spine decomposition of $\{(w_t)_{t\geq 0}; \widetilde {\mathrm N}_\mu\}$, i.e.
(i) $\{(\xi_t)_{t\geq 0}; \mathrm Q_\mu\} \overset{d}= \{(\xi_t)_{t\geq 0}; \widetilde \Pi_{\mu} \}$, and $\{(\xi_t)_{t\geq 0}; \mathrm Q_\mu\}$ is called the {\it spine process};
(ii) given the path of the spine
$\{(\xi_t)_{t\geq 0}; \mathrm Q_\mu\}$,
$\{\mathbf N^\xi; \mathrm Q_\mu\}$ is a Poisson random measure on $[0,\infty) \times \mathbb W$ with intensity
$$
 2 \sigma(\xi_s)^2 \mathrm ds \cdot \mathrm N_{\xi_s}(\mathrm dw)+ \mathrm ds \cdot \int_{(0,\infty)} y \mathrm P_{y\delta_{\xi_s}}(X\in \mathrm dw) \pi(\xi_s, \mathrm dy).
$$
Then according to \cite{RenSongSun2020Spine}, we have that
\begin{align}
 & \{(w_t)_{t\geq 0}; \widetilde{\mathrm N}_\mu \} \overset{{f.d.d.}} = \Big\{ \Big( \int_{[0,t)\times \mathbb W} w_{t-s} \mathbf N^\xi(\mathrm ds, \mathrm dw) \Big)_{t\geq 0} ; \mathrm Q_\mu\Big\}.
 \end{align}
Define\begin{align}
& Z_t:= \int_{[0,t)\times \mathbb W} w_{t-s} \mathbf N^\xi(\mathrm ds, \mathrm dw), \quad t\geq 0.
\end{align}
According to \cite{RenSongSun2020Spine} and \cite{RenSongYang2016Spine},  we have
    \begin{equation}\label{spine-decom1}
    \{(X_t)_{t\geq 0}; \widetilde {\mathrm P}_\mu\} \overset{f.d.d.} = \{ (X_t+ Z_t)_{t\geq 0}; \mathrm Q_\mu\},
    \end{equation}
    where  $\{(X_t)_{t\geq 0}; \mathrm Q_\mu\} \overset{d} =  \{(X_t)_{t\geq 0}; \mathrm P_\mu\}$,   and $\{(X_t)_{t\geq 0}; \mathrm Q_\mu\} $ and $\{(Z_t)_{t\geq 0}; \mathrm Q_\mu\}$ are independent.
    
        We call $\{\mathbf N^\xi; \mathrm Q_\mu\}$ the immigration along the spine $(\xi_t)_{t\geq 0}$. We may decompose $\{\mathbf N^\xi; \mathrm Q_\mu\}$ as the sum of two parts:
    $$
    \mathbf N^\xi(\mathrm ds, \mathrm dw)=\mathrm n(\mathrm ds, \mathrm dw)+\mathrm m(\mathrm ds, \mathrm dw).
    $$
More precisely,
        we say $\{(\xi_t)_{t\geq 0}, (X^{\mathrm n, \sigma})_{\sigma\in \mathcal D^\mathrm n}, (X^{\mathrm m, \sigma})_{\sigma \in \mathcal D^\mathrm m}, (X_t)_{t\geq 0}; \mathrm Q_{\mu}\}$ is a \emph{spine representation} of $\{(X_t)_{t\geq 0}; \widetilde {\mathrm P}_\mu\}$ if the followings are true:
\begin{itemize}
\item
    The \emph{spine process} $\{(\xi_t)_{t\geq 0}; \mathrm Q_\mu\}$ is a copy of $\{(Y_t)_{t\geq 0}; \widetilde \Pi_{\phi\cdot\mu}\}$.
\item
	Given $\{(\xi_t)_{t\geq 0}; \mathrm Q_\mu\}$, \emph{the continuum immigration} $\{ (X^{\mathrm n,\sigma})_{\sigma \in \mathcal D^\mathrm n}; \mathrm Q_\mu(\cdot |\xi)\}$ is a $\mathbb W$-valued point process such that
\[
	\mathrm n(\mathrm ds, \mathrm dw) := \sum_{\sigma\in \mathcal D^{\mathrm n}} \delta_{(\sigma, X^{\mathrm n,\sigma})}(\mathrm ds, \mathrm dw)
\]
 is a Poisson random measure on $[0,T]\times \mathbb W$ with intensity
\[
\mathbf n(\mathrm ds, \mathrm dw):= 2 \sigma(\xi_s)^2 \mathrm ds \cdot \mathrm N_{\xi_s}(\mathrm dw).
\]
\item
	Given $\{(\xi_t)_{t\geq 0}; \mathrm Q_\mu\}$, \emph{the discrete immigration} $\{(X^{\mathrm m,\sigma})_{\sigma\in \mathcal D^{\mathrm m}}; \mathrm Q_\mu(\cdot |\xi)\}$ is a $\mathbb W$-valued point process such that
\[
	\mathrm m(\mathrm ds, \mathrm dw) := \sum_{\sigma\in \mathcal D^{\mathrm n}} \delta_{(\sigma, X^{\mathrm n,\sigma})}(\mathrm ds, \mathrm dw)
\]
	is a Poisson random measure on $[0,\infty ) \times \mathbb W$ with intensity
\begin{align}
 \mathbf m(\mathrm ds, \mathrm dw):= \mathrm ds \cdot \int_{(0,\infty)} y \mathrm P_{y\delta_{\xi_s}}(X\in dw) \pi(\xi_s,dy);
\end{align}
\item
	Given $\{(\xi_t)_{t\geq 0}; \mathrm Q_\mu\}$, the continuum immigration $(X^{\mathrm n,\sigma})_{\sigma \in \mathcal D^n}$ and the discrete immigration $(X^{\mathrm m,\sigma})_{\sigma\in \mathcal D^{\mathrm m}}$ are independent of each other.
\item
	$\{(X_t)_{t\geq 0}; \mathrm Q_\mu\}$ is a copy of the superprocess $\{(X_t)_{t\geq 0}; \mathrm P_\mu\}$, and is independent of the spine process $(\xi_t)_{t\geq 0}$, the continuum immigration $(X^{\mathrm n,\sigma})_{\sigma \in \mathcal D^\mathrm n}$ and the discrete immigration $(X^{\mathrm m,\sigma})_{\sigma\in \mathcal D^{\mathrm m}}$.
\end{itemize}


		To simplify notations, for any $\mu \in \mathcal M^0_f(E)$ and	$t\geq 0$,  we define the following random measures:
\begin{align}
	Z^{\mathrm n}_t
	&:= \int_{(0, t]\times \mathbb W} w_{t-s} ~\mathrm n (\mathrm ds, \mathrm dw)
	= \sum_{\sigma \in \mathcal D^\mathrm n \cap (0, t]} X^{\mathrm n,\sigma}_{t-\sigma},
	\\ Z^{\mathrm m}_t
	&:= \int_{(0, t]\times \mathbb W} w_{t-s} ~\mathrm m (\mathrm ds, \mathrm dw)
	= \sum_{\sigma \in \mathcal D^\mathrm m \cap (0, t]} X^{\mathrm m,\sigma}_{t-\sigma}.
  \end{align}
Then \begin{equation}\label{def-Zt}
%Z_t= Z^{\mathrm n, [0,t)}_{t} + Z^{\mathrm m, [0,t)}_{t},
Z_t= Z^{\mathrm n}_{t} + Z^{\mathrm m}_{t},
\end{equation}
 and \eqref{spine-decom1} can be written as
\begin{align}\label{spine-decom2}
	\{(X_t)_{t\geq 0}; \widetilde{\mathrm P}_\mu\}
	\overset{f.d.d.}{=}
	\{(X_t + Z^{\mathrm n}_{t} + Z^{\mathrm m}_{t} )_{t\geq 0}; \mathrm Q_\mu\}.
\end{align}
We call the above representation as spine decomposition of $\{(X_t)_{t\geq 0}; \widetilde{\mathrm P}_\mu\} $.

We now give a spine decomposition in the case $\mu=\nu$ which can be used to show that
$(Z_t)_{t\ge 0}$ is stochastically increasing.
%end resintstement
}

\begin{lem}  \label{thm:E.16}
	There exist random elements
  	$\big\{X, \xi, N, (s_k, y_k,w^{(k)})_{k\in \mathbb N}; \mathrm Q\big\}$
	so that:
\begin{enumerate}
\item
	$X$ is a $\mathbb W$-valued random element with $\{X; \mathrm Q\} \overset{\text{d}} \sim \mathrm P_\nu$;
\item
%{
		$\xi = (\xi_t)_{t\in \mathbb R}$
	 %is a stationary c\`adl\`ag 
	 %stochastic process with $\{(\xi_t)_{t \geq 0}; \mathrm Q(\cdot | X)\} \overset{\text{d}} \sim \widetilde{\nu \Pi}$;
	 %Markov process with 
	 is a stationary process with 
	 %$$\{(\xi_t)_{t \geq 0}; \mathrm Q(\cdot | X)\} \overset{\text{d}} =\{(\xi_t)_{t \geq 0}; \widetilde{\Pi}_{\bar\nu}\};$$
	 $\{(\xi_t)_{t \geq 0}; \mathrm Q(\cdot | X)\} \overset{\text{d}} = \{(\xi_t)_{t \geq 0}; \widetilde{\nu\Pi}\};$
%}
\item
	$\big\{N; \mathrm Q\big(\cdot \big|X,\xi\big)\big\}$ is a Poisson random measure on $\mathbb R\times \mathbb W$ with intensity
\[
	2 \sigma(\xi_s)^2 {\mathrm d}s \cdot \mathrm N_{\xi_s}({\mathrm d}w),
	\quad (s,w)\in \mathbb R\times \mathbb W;
\]
\item
	$(s_k, y_k)_{k\in \mathbb N}$ is a list of $\mathbb R \times \mathbb R_+$-random elements %satisfying that $\{M; \mathrm Q(\cdot | X, \xi, N)\}$ is a Poisson random measure on $\mathbb R \times \mathbb R_+$ with intensity
	such that, under $\mathrm Q(\cdot | X, \xi, N)$, 
\[
	M(\mathrm ds,\mathrm dy)
	:= \sum_{k\in \mathbb N} \delta_{(s_k, y_k)}(\mathrm ds,\mathrm dy), \quad (s,y)\in \mathbb R \times \mathbb R_+.
\]	
is a Poisson random measure on $\mathbb R \times \mathbb R_+$ with intensity
\[
	\mathrm ds \cdot y \pi(\xi_s, \mathrm dy), \quad (s,y)\in \mathbb R \times \mathbb R_+.
\]
%	where
%\[
%	M(\mathrm ds,\mathrm dy)
%	:= \sum_{k\in \mathbb N} \delta_{(s_k, y_k)}(\mathrm ds,\mathrm dy), \quad (s,y)\in \mathbb R \times \mathbb R_+.
%\]
\item
	$\{(w^{(k)})_{k\in \mathbb N}; \mathrm Q(\cdot|\mathscr G)\}$ is a list of independent $\mathbb W$-valued random elements satisfying that
\[
	\{w^{(k)}; \mathrm Q(\cdot| \mathscr G)\}
	\overset{\text{d}}\sim \mathrm P_{y_k\delta_{\xi_{s_k}}},
	\quad k\in \mathbb N,
\]
	where
\[
	\mathscr G
	:= \sigma(X, \xi, N, (s_k, y_k)_{k\in \mathbb N}).
\]
\end{enumerate}
\end{lem}
\begin{proof}
%{
%This is just a matter of construction. We usually construct things so that, given the spine process $\xi$, the random measure $N$ and the random measure in the next lemma are independent;  $X$ is also independent of the spine process $\xi$,   the random measure $N$ and the random measure in the next lemma. I will leave this for Zhenyao to take care of.
	Suppose that $\Omega_1,\Omega_2,\Omega_3,\Omega_4,\Omega_5$ are canonical space for random elements $X$, $N$, $M$, 
%}
\end{proof}
	For the rest of this section, let $\big\{X, \xi, N, (s_k, y_k,w^{(k)})_{k\in \mathbb N}; \mathrm Q\big\}$ be given by Lemma \ref{thm:E.16}.
	%moved from below
	For $-\infty \leq a < b \leq t<\infty$, we define
\begin{equation} \label{eq:E.4}
	Z_t^{(a,b]}:= \int_{(a,b]\times \mathbb W} w_{t-s} N(\mathrm ds,\mathrm dw) + \sum_{k\in \mathbb N} w^{(k)}_{t-s_k} \mathbf 1_{s_k \in (a,b]}.
\end{equation}
	%end move
\begin{lem} \label{thm:E.19}
	The random measure
\[
	\Big\{\sum_{k\in \mathbb N} \delta_{s_k, w^{(k)}}(\mathrm ds,\mathrm dw); \mathrm Q(\cdot | X, \xi, N)\Big\}
\]
	is a Poisson random measure on $\mathbb R \times \mathbb W$ with intensity
\[
	\mathrm ds \cdot \int_{y\in (0,\infty)} y \pi(\xi_s, \mathrm dy)\cdot \mathrm P_{y\delta_{\xi_s}}(\mathrm dw).
\]
\end{lem}
\begin{proof}
	TBD
\end{proof}
\begin{lem}\label{thm:E.2}
	It holds that
\[
	\{(X_t + Z_t^{(0,t]})_{t\geq 0}; \mathrm Q\} \overset{\text{f.d.d.}}\sim \{(X_t)_{t\geq 0}; \widetilde{\mathrm P}_\nu\}.
\]
	and for each $T\geq 0$,
\[
	\{(Z_t^{(0,t]})_{t\in [0,T]}; \mathrm Q\} \overset{\text{f.d.d.}} \sim \big\{ (X_t)_{t\in [0,T]}; \widetilde{\mathrm N}_\mu\big\}.
\]
%	where, for each $-\infty \leq a < b \leq t<\infty$,
%\begin{equation} \label{eq:E.4}
%	Z_t^{(a,b]}:= \int_{(a,b]\times \mathbb W} w_{t-s} N(\mathrm ds,\mathrm dw) + \sum_{k\in \mathrm N} w^{(k)}_{t-s_k} \mathbf 1_{s_k \in (a,b]}.
%\end{equation}
\end{lem}

%new
\begin{proof}
	TBD
\end{proof}
%end new

\begin{lem}\label{thm:E.3}
	It holds that, for each $-\infty < a < b \leq t<\infty$ and $s\in \mathbb R$,
	\[
	\{Z_t^{(a,b]}; \mathrm Q\} \overset{\text{d}}= \{Z_{t+s}^{(a+s,b+s]}; \mathrm Q\}.
	\]
\end{lem}
\begin{proof}
	TBD
\end{proof}
\begin{lem}\label{thm:E.4}	
	(1) If $\nu(l)<\infty$,
	then for any $\epsilon>0$,
	\[
%	\sum_{k\in \mathrm N} 
	\sum_{k\in \mathbb N} 
	\mathbf 1_{s_k \in (-\infty,0]} y_k e^{\epsilon s_k} \phi(\xi_{s_k}) < \infty, \quad \mathrm Q\text{-a.s.}
	\]
	(2) If  $ \nu(l)=\infty$,
	then for any $\epsilon>0$ and any $s_0>0$,
   	\begin{equation}
	\int^{s_0}_{-\infty} {\mathrm d}s
	\int_{\phi(\xi_s)^{-1}e^{-\epsilon s}}^\infty y\pi(\xi_s,{\mathrm d}y)
     	=\infty,
	\quad {\mathrm Q}\text{-a.s.}
	\end{equation}
\end{lem}
\begin{proof}
%	TBD. See Appendix.
The proof is given in the Appendix.
\end{proof}
	We will also need the following lemma.
\begin{lem} \label{thm:E.5}
 There exist $s_0, \epsilon, \theta>0$ and $\delta > 0$  such that for all
    $x\in E, s>s_0$ and $r\geq \phi(x)^{-1}e^{-\epsilon s}$, it holds that
\begin{equation}\label{domi-below}
	\mathrm P_{r \delta_{x}}\big(w_{s}(\phi)>\theta\big) > \delta.
\end{equation}
\end{lem}
\begin{proof}
	By Chebyshev's inequality, for any $\theta>0$,
	\begin{align}
	&\mathrm P_{r\delta_x}\big(w_s(\phi)>\theta\big)=\mathrm P_{r\delta_x}\left(e^{-w_s(\phi)}<e^{-\theta}\right)\\
	=&1-\mathrm P_{r\delta_x}\left(e^{-w_s(\phi)}\geq e^{-\theta}\right)\geq 1-e^{\theta}\mathrm P_{r\delta_x}e^{- w_s(\phi)}\\
	=&1-e^{\theta}e^{-rV_{s}\phi(x)},\label{Cheby}
	\end{align}
%According to Lemma \ref{lem:rate} and Lemma \ref{lem:ratio limit} (2),
According to \cite[Proposition 1.3]{LiuRenSongSun2020} and Lemma \ref{lem:ratio limit} (2),
%{\bf YX: (Here we used results appeared later, so we need to move them forward.)}
there exist $a,N, s_0>0$ such that when $s>s_0$, we have
	\[
	V_{s}\phi(x)\ge\frac{1}{2}\phi(x)\nu(V_{s}\phi)\geq
 \dfrac{a}{2}\phi(x)e^{-Ns},\quad x\in E.
	\]
	Therefore, using \eqref{Cheby}, we obtain that, when $s>s_0$,
	\begin{eqnarray*}
		&&\inf_{r\geq \phi(x)^{-1}e^{Ns}, x\in E}\mathrm P_{r\delta_x}\big(w_s(\phi)>\theta\big)\geq \inf_{r\geq \phi(x)^{-1}e^{Ns}, x\in E} \left(1-e^{\theta}e^{-rV_s\phi(x)}\right)\\
		&&     \geq 1-e^{\theta-a/2}.
	\end{eqnarray*}
We get \eqref{domi-below} by
choosing $\theta\in(0, a/2)$ and $\delta=1-e^{\theta-a/2}$.
\end{proof}

\begin{proof}[Proof of Theorem \ref{thm:E}]
\emph{Step 1:} We prove the  existence of $k\in [0,\infty)$.
	We only need to prove the existence of $k\in [0,\infty)$ 
	%for initial configuration $\nu$.
	for $\mu=\nu$.
	In fact, for any $\mu\in \mathcal M_f^o(E)$,
\begin{align}
	\lim_{t\to \infty}\frac{\mathrm P_\mu(X_t \neq \mathbf 0)}{\mathrm P_\nu(X_t \neq \mathbf 0)}
	\overset{\eqref{eq:Q.04}} = \lim_{t\to \infty}\frac{1- e^{-\mu(v_t)}}{1- e^{-\nu(v_t)}}
	\overset{\eqref{eq:Q.05}, \eqref{eq:Q.055}} = \lim_{t\to \infty}\frac{\mu(v_t)}{\nu(v_t)}
	\overset{\eqref{eq:Q.25}, \eqref{asp:H1}} = \mu(\phi).
\end{align}
%	It can be verified that for any $t\geq 0,$
Now for any $t\geq 0$,
	\begin{align}\label{eq:E.5}
	&e^{-\lambda t}\mathrm P_\nu(X_t\neq \mathbf 0)
	\overset{\eqref{eq:M.3}, \eqref{asp:H1}}= \widetilde{\mathrm P_\nu}[X_t(\phi)^{-1}]
	\overset{\text{Lemma \ref{thm:E.2}}}=\mathrm Q\big[ \big(X_t(\phi) +Z^{(0,t]}_t(\phi)\big)^{-1}\big]
	\\&\overset{\text{Lemma \ref{thm:E.3}}}=\mathrm Q\big[ \big(X_t(\phi) +Z^{(-t,0]}_0(\phi)\big)^{-1}\big].
	\end{align}
	
%	Note that
We claim that
\begin{equation}\label{eq:E.51}
	X_t(\phi) \xrightarrow[t\to \infty]{} 0, \quad \mathrm Q\text{-a.s.}
\end{equation}
	In fact, by Lemma \ref{thm:E.16} (1) and \eqref{eq:M.25}, we know that $\big\{\big(e^{-\lambda t} X_t(\phi)\big)_{t\geq 0}; \mathrm Q\big\}$ is a non-negative martingale.
	So according to the martingale convergence theorem and the fact that $\lambda < 0$, we know \eqref{eq:E.51} holds.

%	We observe by monotonicity that
Note  by monotonicity we have that
\begin{equation} \label{eq:E.52}
	Z^{(-t,0]}_0(\phi)
	\xrightarrow[t\to \infty]{} Z^{(-\infty,0]}_0(\phi),
	\quad \mathrm Q\text{-a.s.}
\end{equation}
Note also that
	\begin{equation}\label{eq:E.53}
	\big(X_t(\phi) + Z^{(-t,0]}_0(\phi)\big)^{-1}
	\leq Z^{(-1,0]}_0(\phi)^{-1},
	\quad t\geq 1,
	\end{equation}
	and
	\begin{align}
	&\mathrm Q [Z^{(-1,0]}_0(\phi)^{-1}]
	\overset{\text{Lemma \ref{thm:E.3}}} = \mathrm Q [Z^{(0,1]}_1(\phi)^{-1}]
	 \overset{\text{Lemma \ref{thm:E.2}}}= 
	 %\widetilde {\nu\mathrm N}^{(1)} [w_1(\phi)^{-1}]
	 \widetilde {\mathrm N}_\nu [w_1(\phi)^{-1}]
		\\&\overset{\eqref{eq:E.15}}= e^{-\lambda}
		%(\nu\mathrm N) (w_1\neq \mathbf 0)
		\mathrm N_\mu(w_1\neq \mathbf 0)
	\overset{\eqref{eq:E.14}} = -e^{-\lambda} \log \mathrm P_{\nu}(X_1 = \mathbf 0)
	\\& \label{eq:E.54} \overset{\eqref{asp:H1}}< \infty.
	\end{align}
	Now by \eqref{eq:E.5}, \eqref{eq:E.51}, \eqref{eq:E.52}, \eqref{eq:E.53}, \eqref{eq:E.54} and the dominated  convergence theorem, we obtain
	\begin{equation}\label{eq:E.55}
		 \lim_{t\to\infty} e^{-\lambda t}\mathrm P_{\nu}(X_t \neq \mathbf 0)
	 =\mathrm Q[Z^{(-\infty,0]}_0(\phi)^{-1}]
	 =:k<\infty.
	\end{equation}

\emph{Step 2:} We prove that  $\nu(l)<\infty$ is sufficient for $k>0$.
	We assume that $\nu(l)<\infty$.
	Notice that
		\begin{align}
		&\mathrm Q\Big[\int_{(-\infty,0]\times \mathbb W}w_{-s}(\phi) N({\mathrm d}s, {\mathrm d}w) \Big]
	= \mathrm Q\bigg[\mathrm Q\Big[\int_{(-\infty,0]\times \mathbb W}w_{-s}(\phi) N({\mathrm d}s, {\mathrm d}w) \Big | X, \xi \Big]\bigg]
	\\& \overset{\text{Theorem \ref{thm:E.16} (3)}}= \mathrm Q\Big[\int_{(-\infty,0]\times \mathbb W}w_{-s}(\phi) 2 \sigma(\xi_s)^2 {\mathrm d}s \cdot \mathrm N_{\xi_s}({\mathrm d}w) \Big]
		\\& \overset{\eqref{eq:E.13}, \eqref{asp:H1}}=\mathrm Q \Big[\int^0_{-\infty} e^{-\lambda s} 2\sigma(\xi_s)^2 \phi(\xi_s) \mathrm ds\Big]
	\\&\overset{\text{Theorem \ref{thm:E.16} (2)}}=
%	\widetilde{\nu \Pi} 
\widetilde{\Pi}_{\bar \nu}
	\Big[\int^0_{-\infty} e^{-\lambda s} 2\sigma(\xi_s)^2 \phi(\xi_s) \mathrm ds\Big]
	\\&\label{eq:E.64}\overset{\text{Theorem \ref{thm:E.15}}}= 
	%2\int^0_{-\infty} e^{-\lambda s} \widetilde\nu(\sigma^2 \phi) \mathrm ds
	2\int^0_{-\infty} e^{-\lambda s} \bar\nu(\sigma^2 \phi) \mathrm ds
	<\infty,
	\end{align}
	where in the last inequality, we used the fact that $\sigma$, $\phi$ are bounded and $\lambda < 0$.

%RS this is elementary, I do not think we need to write out the details.
\begin{comment}
	Note that
\begin{equation} \label{eq:E.65}
\begin{minipage}{0.9\textwidth}
	if $\{Z; \mathrm P\}$ is a non-negative random variable such that there exists a $\sigma$-field $\mathscr H$ with $\mathrm P(\mathrm P[Z|\mathscr H] < \infty) = 1$, then $\mathrm P(Z < \infty) = 1$.
\end{minipage}
\end{equation}
	In fact, from
\[
	\infty \cdot \mathrm P(Z = \infty | \mathscr F)
	 = \mathrm P[Z ; Z = \infty | \mathscr F]
	 \leq \mathrm P[Z| \mathscr F] < \infty, \quad \mathrm P\text{-a.s.}
\]
	we know that
\[
	\mathrm P(Z = \infty| \mathscr F) = 0, \quad \mathrm P\text{-a.s.}
\]
	and therefore,
\[
	\mathrm P(Z = \infty) = \mathrm P[\mathrm P(Z = \infty| \mathscr F)] = 0.
\]
\end{comment}

%	Recall from Lemma \ref{thm:E.16} (5) that $\mathscr G := \sigma(X, \xi, N, (s_k, y_k)_{k\in \mathrm N})$.
%	Therefore, $\mathrm Q$-almost surely,
	Recall from Lemma \ref{thm:E.16} (5) that $\mathscr G = \sigma(X, \xi, N, (s_k, y_k)_{k\in \mathbb N})$. Note that we have $\mathrm Q$-almost surely,
\begin{align}
%	& \mathrm Q\Big(\sum_{k\in \mathrm N} 
	& \mathrm Q\Big(\sum_{k\in \mathbb N} 
	w^{(k)}_{-s_k}(\phi) \mathbf 1_{s_k \in (-\infty,0]}\Big|\mathscr G \Big)
	\overset{\text{Lemma \ref{thm:E.16} (5)}}= 
%	\sum_{k\in \mathrm N} 
	\sum_{k\in \mathbb N} 
	\mathbf 1_{s_k \in (-\infty,0]} \mathrm P_{y_k\delta_{\xi_{s_k}}}[w^{(k)}_{-s_k}(\phi)]
	\\ &\overset{\eqref{eq:M.2}}= 
%	\sum_{k\in \mathrm N} 
	\sum_{k\in \mathbb N} 
	\mathbf 1_{s_k \in (-\infty,0]} y_k (P_{-s_k}^\beta \phi)(\xi_{s_k})
	\overset{\eqref{asp:H1}} = 
%	\sum_{k\in \mathrm N} 
	\sum_{k\in \mathbb N} 
	\mathbf 1_{s_k \in (-\infty,0]} y_k e^{-\lambda s_k} \phi(\xi_{s_k})
	\\ & \overset{\text{Lemma \ref{thm:E.4} (1)}} < \infty.
\end{align}
%	Therefore, by \eqref{eq:E.65} we know that
From this, we immediately get that
\begin{equation} \label{eq:E.67}
%	\sum_{k\in \mathrm N} 
	\sum_{k\in \mathbb N} 	
	w^{(k)}_{-s_k}(\phi) \mathbf 1_{s_k \in (-\infty,0]}< \infty, \quad \mathrm Q\text{-a.s.}
\end{equation}
	Now, we have
\begin{equation} \label{eq:E.68}
	Z_0^{(-\infty, 0]}(\phi) \overset{\eqref{eq:E.4}}= \int_{(-\infty,0]\times \mathbb W}w_{-s}(\phi) N({\mathrm d}s, {\mathrm d}w) + 
%	\sum_{k\in \mathrm N} 	
	\sum_{k\in \mathbb N} 
	w^{(k)}_{-s_k}(\phi) \mathbf 1_{s_k \in (-\infty,0]}
	\overset{\eqref{eq:E.64}, \eqref{eq:E.67}} < \infty.
\end{equation}
	Therefore
\begin{align}
& k \overset{\eqref{eq:E.55}}= \mathrm Q[Z^{(-\infty,0]}_0(\phi)^{-1}] \overset{\eqref{eq:E.68}}> 0.
\end{align}

\emph{Step 3:} We prove that $\nu(l)<\infty$ is necessary for $k>0$.
%		We claim that when $\nu(l) = \infty$, for any $\varepsilon>0$, $\mathrm Q$-almost surely,
	We claim that if $\nu(l) = \infty$, then for any $\varepsilon>0$, 
\begin{equation} \label{eq:E.7}
	\#\{k\in \mathbb N: s_k\leq 0, w_{-s_k}^{(k)}(\phi)> \varepsilon\} =\infty,
	\quad \mathrm Q\text{-a.s.}
\end{equation}
%	Therefore
Using this claim, we have
\begin{equation} \label{eq:E.71}
	Z_0^{(-\infty, 0]} \overset{\eqref{eq:E.4}}\geq 
%	\sum_{k\in \mathrm N} 
	\sum_{k\in \mathbb N} 	
	\mathbf 1_{s_k \in (-\infty, 0]}  w^{(k)}_{-s_k}(\phi)
	\overset{\eqref{eq:E.7}}= \infty,
		\quad \mathrm Q\text{-a.s.}
	\end{equation}
	Thus
\begin{align}
	& k \overset{\eqref{eq:E.55}}= \mathrm Q[Z^{(-\infty,0]}_0(\phi)^{-1}]
		\overset{\eqref{eq:E.71}}= 0.
\end{align}

%		\emph{Step 4.} We prove claim \eqref{eq:E.7}.
Now we prove claim \eqref{eq:E.7}.
\begin{comment}
	Note that
	\begin{equation} \label{eq:E.72}
	\begin{minipage}{0.9\textwidth}
	if $\{Z; \mathrm P\}$ is a non-negative random variable such that there exists a $\sigma$-field $\mathscr H$ with $\mathrm P\big(\mathrm P[Z=\infty|\mathscr H] =1\big) = 1$, then $\mathrm P(Z = \infty) = 1$.
	\end{minipage}
	\end{equation}
	In fact,
	\[
	\mathrm P(Z = \infty)
	= \mathrm P\big(P(Z = \infty|\mathscr H)\big)
	= 1.
	\]
\end{comment}
%	
%  Now let  $s_0, \epsilon, \theta, \delta > 0$ be given as in Lemma \ref{thm:E.5}.
Let $s_0, \epsilon, \theta, \delta > 0$ be given as in Lemma \ref{thm:E.5}.
	We have $\mathrm Q$-almost surely,
\begin{align}
	  &\mathrm Q\big(\#\{k\in \mathbb N:s_k\leq 0, w_{-s_k}^{(k)}(\phi)> \theta\} \big| X, \xi, N\big) 
	\\\overset{\text{Lemma \ref{thm:E.19}}}= &\int_{-\infty}^0 \mathrm ds \int_{(0,\infty)}  y \pi(\xi_s, \mathrm dy) \mathrm P_{y \sigma_{\xi_s}}\big(w_{-s}(\phi)>\theta\big)
	\\\overset{\text{Lemma \ref{thm:E.5}}}\geq &\delta \int_{-\infty}^{-s_0} \mathrm ds
 \int_{(\phi(\xi_s)^{-1}e^{-\epsilon s},\infty)}  y \pi(\xi_s, \mathrm dy)
  	\\  \overset{\text{Lemma \ref{thm:E.4} (2)}}=& \infty.
\end{align}
	Now from Lemma \ref{thm:E.19} we know that $\mathrm Q$-almost surely
\begin{equation} \label{eq:E.9}
	\mathrm Q \big(\#\{k\in \mathbb N:s_k\leq 0, w_{-s_k}^{(k)}(\phi)> \theta\} = \infty\big| X, \xi, N\big)  = 1.
\end{equation}
%	The desired result then follows from \eqref{eq:E.9} and \eqref{eq:E.72}. 
Now the desired result follows immediately.
\end{proof}

\begin{comment}
\section{Preliminaries}
It follows from \cite[Theorem 2.7]{KimSong2008Intrinsic} that there exists $c, \rho > 0$ such that
\begin{equation}\label{IU}
 \sup_{x,y\in E}\Big|\frac{\tilde p(t,x,y)}{\phi(y) \widehat\phi(y)}- 1\Big|
 =\sup_{x,y\in E}\Big|\frac{e^{-\lambda t}p^\beta(t,x,y)}{\phi(x) \widehat\phi(y)}- 1\Big|
	\leq c\,e^{-\rho t},
	\quad t\geq 1. \footnote{ZS: I'm about to replace this with (H2)}
\end{equation}
It is easy to check that $(\widehat Y_t)_{t\geq 0}$ has
unique  invariant probability distribution $\phi(x)\widehat\phi(x)m({\mathrm d}x)$,
and is exponentially ergodic in the sense that
\begin{equation}\label{IU'}
	\sup_{x,y\in E}\left|\frac{\hat{p}(t, x,y)}{\phi(y) \widehat\phi(y)}- 1\right|\le c\,\mbox{e}^{-\rho t}, \quad t\geq 1,
\end{equation} where $c, \rho > 0$ are the constant in \eqref{IU}.


It follows from the spine decomposition that
\begin{align}\label{e:newspine1}
\{(w_t)_{t\ge 0}; \widetilde{\mathrm N}_{\phi\cdot\nu}\}\overset{f.d.d.}{=}\left\{\left(\int^t_0w_{t-s}\mathbf N^\xi({\mathrm d}s, {\mathrm d}w)\right)_{t\ge 0}; \mathrm Q
\right\}
\end{align}
and
\begin{align}\label{e:newspine2}
\{(X_t)_{t\ge 0}; \widetilde {\mathrm P}_\nu\}\overset{f.d.d.}{=}\left\{\left(X_t+\int^t_0w_{t-s}\mathbf N^\xi({\mathrm d}s, {\mathrm d}w)\right)_{t\ge 0}; \mathrm Q
\right\}.
\end{align}



For any $t\geq 0$, define
\begin{align}
	Z^{\mathrm n}_t
&:= \int_{[0, t)\times \mathbb W} w_{t-s} ~\mathrm n ({\mathrm d}s,{\mathrm d}w)
	= \sum_{\sigma \in \mathcal D^\mathrm n \cap [0, t)} X^{\mathrm n,\sigma}_{t-\sigma},	
	\\ Z^{\mathrm m}_t
&:= \int_{[0, t)\times \mathbb W} w_{t-s} ~\mathrm m ({\mathrm d}s,{\mathrm d}w)
	= \sum_{\sigma \in \mathcal D^\mathrm m \cap [0, t)} X^{\mathrm m,\sigma}_{t-\sigma}.
  \end{align}
Then \begin{equation}\label{def-Zt}
\int^t_0w_{t-s}\mathbf N^\xi({\mathrm d}s, {\mathrm d}w)= Z^{\mathrm n}_{t} + Z^{\mathrm m}_{t},
\end{equation}
 and \eqref{e:newspine2} can be written as
\begin{align}\label{spine-decom2}
	\{(X_t)_{t\geq 0}; \widetilde{\mathrm P}_\nu\}
	\overset{f.d.d.}{=}
\{(X_t + Z^{\mathrm n}_{t} + Z^{\mathrm m}_{t} )_{t\geq 0}; \mathrm Q\}.
\end{align}
We call the above representation a spine decomposition of $\{(X_t)_{t\geq 0}; \widetilde{\mathrm P}_\nu\} $.
\end{comment}


%\section{Proofs of Main Results}
\section{Proofs of Theorems \ref{thm:L} and \ref{thm:I}}

%RS The first result is deleted and the other two moved to Section 3.
\begin{comment}
The following lemma is \cite[Proposition 1.4]{LiuRenSongSun2020}.
\begin{lem}\label{lem:rate}
	Suppose that Assumptions \eqref{asp:H1}, \eqref{asp:H2} and \eqref{asp:H3} hold.
Let $(V_t)_{t\ge0}$ be the extended operators defined above on  $\mathcal B(E,[0,\infty])$.
Then
\[
\lim_{t\to\infty}\sup_{x\in E, f\in \mathcal B(E,[0,\infty]) }\left|\dfrac{V_tf(x)}{\phi(x)\nu(V_tf)}-1\right|=0.
\]
\end{lem}

\begin{lem}\label{lem:extinc}
	Suppose that Assumptions \eqref{asp:H1}, \eqref{asp:H2} and \eqref{asp:H3} hold.
\begin{enumerate}
\item	
	For any  $ t>0$ and $ \mu \in \mathcal M_f(E)$, $\langle v(t,\cdot),\mu\rangle <\infty.$
\item	For any $\mu \in \mathcal M_f(E)$,
\[
	\lim_{t\rightarrow\infty}\langle v(t,\cdot),\mu\rangle=0.
\]
\end{enumerate}
\end{lem}


\begin{lem}\label{lem:ratio limit}
(1) For any $f\in\mathcal B(E,[0,\infty])$ with $\nu(f)>0$, and any $s>0$,
\begin{equation}\label{integ ratio limit}
\lim_{t\to\infty}\dfrac{\langle V_{t+s}f, \nu\rangle}{\langle V_{t}f, \nu\rangle}=e^{\lambda s}.
\end{equation}
(2) There are constants $a,N,T>0$ such that
\begin{equation}\label{inequ:lower}
\langle V_{t}f, \nu\rangle\geq ae^{-Nt},\quad \mbox{for any }\ t>T.
\end{equation}
(3) For each $s\geq 0$,
\begin{equation} \label{one point ratio limit}
	\lim_{t\to \infty} \sup_{x\in E}\Big|\frac{v(t+s,x)}{\langle v(t,\cdot),\nu\rangle\phi(x) } - e^{\lambda s} \Big|=0.
\end{equation}
\end{lem}


Note that
 \begin{equation}\label{domi-Zt1}
\frac{1}{\langle\phi, X_t\rangle + \int^0_{-t}w_{-s}(\phi)\mathbf N^\xi({\mathrm d}s, {\mathrm d}w)}\leq \frac{1}{\int^0_{-1}w_{-s}(\phi)\mathbf N^\xi({\mathrm d}s, {\mathrm d}w)},\quad t\geq 1,
 \end{equation}
and
\begin{equation}
\mathrm Q\left(\frac{1}{\int^0_{-1}w_{-s}(\phi)\mathbf N^\xi({\mathrm d}s, {\mathrm d}w)}\right)= \widetilde {\mathrm N}_{\phi\cdot\nu} [w_1(\phi)^{-1}]
= \mathrm N_{\phi\cdot\nu} (w_1\neq \mathbf 0) < \infty.
    \end{equation}
\end{comment}

\subsection{Proof of Theorem \ref{thm:L}}
We first state a result proved in \cite{LiuRenSongSun2020}. For $f\in \mathcal B(E, [0,\infty])$, Put
$$
\Gamma_t f:=-\log \mathrm P_{\nu}[e^{-X_t(f)}|X_t(1)>0].
$$
We say a $[0,\infty]$-valued functional $A$ defined on $\mathcal B(E,[0,\infty])$ is monotone concave if
	(1) $A$ is a monotone functional, i.e., $f\leq g$ in $\mathcal B(E,[0,\infty])$ implies $Af \leq Ag$; and
	(2) for any $f\in \mathcal B(E,[0,\infty])$ with $Af< \infty$, the function $u \mapsto A(uf)$ is concave on $[0,1]$.
The following result is \cite[Proposition 1.5, Proposition 1.9]{LiuRenSongSun2020}.

\begin{lem} \label{prop:G}
(1) The limit $Gf:= \lim_{t\to \infty} \Gamma_t f$ exists in $[0,\infty]$ for each $f\in \mathcal B(E,[0,\infty])$.
	Moreover, $G$ is the unique $[0,\infty]$-valued monotone concave functional on $\mathcal B(E,[0,\infty])$ such that
	$G(\infty  \mathbf 1_E) = \infty$ and that
\begin{equation} \label{eq:G.0}
	1 - e^{- GV_s f}
	= e^{s\lambda} (1 - e^{-Gf}),
	\quad s\geq 0, f\in \mathcal B(E,[0,\infty]).
\end{equation}

(2) $G$ is the log-Laplace functional of $\mathbf Q_\lambda$, and for any $r\in[\lambda, 0)$,
$L_\alpha(f):=1-\left(1-e^{-G(f)}\right)^\alpha, f\in\mathcal B(E,[0,\infty)),$
is the Laplace functional of $\mathbf Q_r$, where $\alpha=r/\lambda\in(0,1]$.
\end{lem}


\begin{proof}[Proof of Theorem \ref{thm:L}]
(1) Using \eqref{one point ratio limit} with $s=0$, we know that, for any $\varepsilon>0$, there is some $T_1>0$ such that when $t>T_1$,
\[
(1-\varepsilon)\langle v(t),\nu\rangle \phi(x)\leq v(t,x)\leq (1+\varepsilon)\langle v(t),\nu\rangle \phi(x),\qquad x\in E.
\]
By Lemma \ref{prop:G},  $1-e^{-G(f)}$ is non-decreasing  with respect to $f\in \mathcal B(E,[0,\infty])$, and $1-e^{-G(uf)}$ is a concave function of $u\in(0,1)$.
%new
Fix an arbitrary $u\in (0, 1)$.
%end new
On one hand, we have for $t>T_1$,
\begin{eqnarray}\label{lower}
\dfrac{1-e^{-G(u\langle v(t),\nu\rangle \phi)}}{1-e^{-G(\langle v(t),\nu\rangle \phi)}}&\overset{\text{monotonicity}}\geq& \dfrac{\left(1-e^{-G(\frac{u}{1+\varepsilon}v(t))}\right)}{1-e^{-G(\frac{1}{1-\varepsilon} v(t))}}
\overset{\text{concavity}}\geq \dfrac{\dfrac{u}{1+\varepsilon}\left(1-e^{-G(v(t))}\right)}{1-e^{-G(\frac{1}{1-\varepsilon} v(t))}}
\\
&=&\dfrac{u}{1+\varepsilon}\left[e^{-\lambda t}\left(1-e^{-G(\frac{1}{1-\varepsilon} v(t))}\right)\right]^{-1}\nonumber\\
&\overset{\text{concavity}}\geq& \dfrac{u(1-\varepsilon)}{1+\varepsilon}\left[e^{-\lambda t}\left(1-e^{-G(v(t))}\right)\right]^{-1}=\dfrac{u(1-\varepsilon)}{1+\varepsilon},
\end{eqnarray}
where in the last inequality, we used \eqref{eq:G.0} with $f=\infty I_{E}.$
On the other hand, 
%without lose of generality, we choose $\varepsilon>0$ such that $u/(1-\varepsilon)<1$, then
for $\varepsilon>0$ small enough so that $u/(1-\varepsilon)<1$, we have
\begin{eqnarray}\label{upper}
\dfrac{1-e^{-G(u\langle v(t),\nu\rangle \phi)}}{1-e^{-G(\langle v(t),\nu\rangle \phi)}}&\overset{\text{monotonicity}}\leq &\dfrac{\left(1-e^{-G(\frac{u}{1-\epsilon}v(t))}\right)}{1-e^{-G(\frac{1}{1+\varepsilon} v(t))}}\overset{\text{concavity}}\leq \dfrac{\dfrac{u}{1-\varepsilon}\left(1-e^{-G(v(t))}\right)}{1-e^{-G(\frac{1}{1+\varepsilon} v(t))}}\\
&=&\dfrac{u}{1-\varepsilon}\left[e^{-\lambda t}\left(1-e^{-G(\frac{1}{1+\varepsilon} v(t))}\right)\right]^{-1}\\
 &\overset{\text{concavity}}\leq& \dfrac{u(1+\varepsilon)}{1-\varepsilon}\left[e^{-\lambda t}\left(1-e^{-G(v(t))}\right)\right]^{-1}
=\dfrac{u(1+\varepsilon)}{1-\varepsilon}.
\end{eqnarray}
%Combining the two inequalities above, it follows that
Combining the two displays above, we get that for any $u\in (0, 1)$,
\[
\lim_{t\to\infty}\dfrac{1-e^{-G(u\langle v(t),\nu\rangle \phi)}}{1-e^{-G(\langle v(t),\nu\rangle \phi)}}
%=u,\qquad u\in (0,1).
=u.
\]
%And therefore, by Lemma \ref{lem:regu} in the Appendix,
Therefore, by Lemma \ref{lem:regu} in the Appendix, w ehave
\begin{equation}\label{eq regu}
1-e^{-G(u\phi)}\sim uL(u),\quad u\rightarrow 0+,
\end{equation}
where $L$ is slowly varying at $0$.

Note that for any $u>0$, we have
\begin{align}
\int_0^\infty e^{-us}{\mathbf Q}_\lambda(\{\mu: \mu(\phi)>s\})ds=&-\frac{1}{u}\int^\infty_0{\mathbf Q}_\lambda(\{\mu: \mu(\phi)>s\})d(e^{-us})\\
=&\frac{1}{u}{\mathbf Q}_\lambda(\{\mu: \mu(\phi)>0\})-\frac{1}{u}\int^\infty_0e^{-us}d({\mathbf Q}_\lambda(\{\mu: \mu(\phi)\le s\}))\\
=&\dfrac{1-e^{-G(u\phi)}}{u}.
\end{align}
It follows from  \eqref{eq regu} and Lemma \ref{lem: tau} that there is a function $L_1$, which is a slowly varying at infinity, such that
$$\int^x_0{\mathbf Q}_\lambda(\mu(\phi)>s)ds\sim L_1(x),\quad x\to\infty.$$
%Then by Lemma \ref{lem:tail},
It then follows from Lemma \ref{lem:tail} that
$$s{\mathbf Q}_\lambda(\{\mu: \mu(\phi)>s\})=o(L_1(s))\qquad s\to\infty.$$
Therefore, for $r\in[\lambda, 0)$,
\begin{equation}
{\mathbf Q}_r(\{\mu: \mu(\phi)>s\})=o(s^{-\alpha}L_1(s)),\qquad s\to\infty,
\end{equation}
where $\alpha=\lambda/r.$
Thus for any $0<\gamma<\alpha$,
\[
{\mathbf Q}_r\left(\mu(\phi)^{\gamma}\right)=\int_{{\mathcal M}_f(E)}\mu(\phi)^\gamma\mathbf Q_r(d\mu)<\infty.
\]

(2) 
%For any $f\in\mathcal B_b^+(E)$ and $t>0$,  using the definition of $\widetilde{\mathrm P}_\nu$, we obtain
For any $f\in\mathcal B_b(E, [0, \infty))$ and $t>0$,  using the definition of $\widetilde{\mathrm P}_\nu$, we have
\begin{eqnarray*}
%&&\mathrm P_\nu\left(\exp\{-\langle f, X_t\rangle \};\zeta>t\right)
%=\mathrm P_\nu\left(\dfrac{M_t(\phi)}{M_t(\phi)}\exp\{-\langle f, X_t\rangle \};\zeta>t\right)\\
%&&=\widetilde{\mathrm P}_\nu\left(\dfrac{1}{M_t(\phi)}\exp\{-\langle f, X_t\rangle \}\right)
%=e^{\lambda t}\mathrm Q_{\nu}\left(\dfrac{\exp\Big\{-\langle f, X_t\rangle -\langle f,  Z_t\rangle\Big \}}{\langle\phi, X_t\rangle +\langle\phi,  Z_t\rangle }\right)\\
%&&=e^{\lambda t}\widehat{\mathrm Q}_{\nu}\left(\dfrac{\exp\Big\{-\langle f, X_t\rangle -\langle f,  \widehat Z_t\rangle\Big \}}{\langle\phi, X_t\rangle +\langle\phi,  \widehat Z_t\rangle }
%\right).
&&\mathrm P_\nu\left(\exp\{-\langle f, X_t\rangle \}; X_t\neq \mathbf 0\right)
=\mathrm P_\nu\left(\dfrac{M_t(\phi)}{M_t(\phi)}\exp\{-\langle f, X_t\rangle \};X_t\neq \mathbf 0\right)\\
&&=\widetilde{\mathrm P}_\nu\left(\dfrac{1}{M_t(\phi)}\exp\{-\langle f, X_t\rangle \}\right)
=e^{\lambda t}\mathrm Q\left(\dfrac{\exp\Big\{-\langle f, X_t\rangle -\langle f,  Z_0^{(-t, 0]}\rangle\Big \}}{\langle\phi, X_t\rangle +\langle\phi,  Z_0^{(-t, 0]}\rangle }\right).
\end{eqnarray*}
Since $\lim_{t\rightarrow\infty}X_t=0$ in probability with respect to 
%$\widehat{\mathrm Q}_{\nu}$,  
$\widehat{\mathrm Q}$,
%using domination \eqref{domi-Zt1} and \eqref{eq:E.53}, we have
Using the inequalities  \eqref{eq:E.53} and  \eqref{eq:E.54}, we have
%RS I think that we need to assume $f$ is bounded continuous, since we are using weak convergence of measures.
\[
%\lim_{t\to\infty}e^{-\lambda t}\mathrm P_\nu\left(\exp\{-\langle f, X_t\rangle \};\zeta>t\right)=\widehat{\mathrm Q}_{\nu}\left(\dfrac{\exp\left\{-\langle f,\widehat Z_\infty\rangle \right\}}{\langle \phi,\widehat Z_\infty\rangle}\right),
\lim_{t\to\infty}e^{-\lambda t}\mathrm P_\nu\left(\exp\{-\langle f, X_t\rangle \};X_t\neq \mathbf 0t\right)={\mathrm Q}\left(\dfrac{\exp\left\{-\langle f,Z_0^{(-\infty, 0]}\rangle \right\}}{\langle \phi,\widehat Z_0^{(-\infty, 0]}\rangle}\right),
\]
%where $\widehat Z_\infty:=\sum_{\sigma\in\mathcal D^{\mathrm m}}\widehat X^{{\mathrm m},\sigma}_\sigma+\sum_{\sigma\in\mathcal D^{\mathrm n}}\widehat X^{{\mathrm n},\sigma}_\sigma$.  
Note that for the continuum immigration part,
\[
%\widehat{\mathrm Q}_{\nu}\big(\sum_{\tau\in \mathcal D^{\mathrm n}}\langle f, \widehat %X_{\tau}^{{\mathrm n},\tau} \rangle \big)
%=\int_0^\infty2\langle \sigma^2 P^{\beta}_sf,\phi\nu\rangle ds
%\leq 2\|\sigma^2\phi\|_\infty\dfrac{\langle f,\nu\rangle }{-\lambda}<\infty.
\mathrm Q\left(\int_{(-\infty, 0]\times \mathbb W}w_{-s}(f)N(\mathrm ds, \mathrm dw)\right)=
\int^0_{-\infty}2\langle \sigma^2 P^{\beta}_{-s}f,\phi\nu\rangle \mathrm ds
\leq 2\|\sigma^2\phi\|_\infty\dfrac{\langle f,\nu\rangle }{-\lambda}<\infty.
\]

Now we deal with the discrete immigration part.


(i) If  $\int_E l(x) \nu(dx)<\infty$,
 by Lemma \ref{thm:E.4} (1) and (H2),
\begin{eqnarray*}
%&&\widehat{\mathrm Q}_{\nu}\Big(\sum_{\sigma\in [1,\infty)\bigcap\mathcal D^{\mathrm m}}\langle f, \widehat X_{\sigma}^{{\mathrm m},\sigma} \rangle\Big|\mathcal G \Big)
%=\sum_{t\in [1,\infty)\bigcap\mathcal D^{\mathrm m}}m_tP^{\beta}_tf( Y_t)\\
%&&\leq \sum_{t\in \mathcal D^{\mathrm m}}(1+ce^{-\rho t})m_te^{\lambda t}
%\phi(Y_t)
%\int_E\widehat\phi(y)f(y)m(dy)\\
%&&=\langle f,\nu\rangle
%\sum_{t\in \mathcal D^{\mathrm m}}(1+ce^{-\rho t})m_te^{\lambda t}\phi(Y_t)<\infty,\quad \widehat{\mathrm Q}_{\nu}-{\mathrm a.s.}
&&\mathrm Q\left( \sum_{k\in\mathbb N}w^{(k)}_{-s_k}(f)\mathbf 1_{s_k\in (-\infty, 0]}\Big|\mathcal G\right)=\sum_{k\in\mathbb N}\mathbf 1_{s_k\in (-\infty, 0]}y_kP^{\beta}_{-s_k}f( \xi_{s_k})\\
&&\leq\sum_{k\in\mathbb N}\mathbf 1_{s_k\in (-\infty, 0]}(1+C^{(H2)}_{-s_k, \xi_{s_k}, f})
e^{-\lambda s_k}\phi(\xi_{s_k})\nu(f)\\
&&=\nu(f)\sum_{k\in\mathbb N}\mathbf 1_{s_k\in (-\infty, 0]}(1+C^{\eqref{asp:H2}}_{-s_k, \xi_{s_k}, f})
e^{-\lambda s_k}\phi(\xi_{s_k})<\infty, \quad \mathrm Q-\text{a.s.}
\end{eqnarray*}
 Thus in this case, the limit measure 
% $\widehat Z_\infty\in \mathcal M_f(E)$.  
$Z_0^{(-\infty, 0]}\in \mathcal M_f(E)$.
% Denote the distribution of $\widehat Z_\infty$ under
 Denote the distribution of $Z_0^{(-\infty, 0]}$ under
%  $\widehat{\mathrm Q}_{\nu}$ by $\mathbf Q$,
  ${\mathrm Q}$ by $\mathbf Q$,
  which is also the limit distribution of $X_t$ under $\widetilde{\mathrm P}_\nu$ as $t\to\infty$.  Then when
$\int_E l(x)\nu(dx)<\infty$,
\[
\lim_{t\rightarrow\infty}e^{-\lambda t}\mathrm P_\nu\left(\exp\{-\langle f, X_t\rangle \};
%\zeta>t\right)=
X_t\neq \mathbf 0\right)=
\int_{{\mathcal M}_f(E)}\frac{1}{\mu(\phi)}e^{-\mu(f)}\mathbf Q(d\mu).
\]
In Theorem \ref{thm:E} we have shown that
\[
%\lim_{t\rightarrow\infty}e^{-\lambda t}\mathrm P_\nu(\zeta>t)=k
\lim_{t\rightarrow\infty}e^{-\lambda t}\mathrm P_\nu(X_t\neq \mathbf 0)=k
=\int_{{\mathcal M}_f(E)}\frac{1}{\mu(\phi)}\mathbf Q(d\mu)<\infty.
\]
Thus by the definition of the Yaglom distribution ${\mathbf Q}_\lambda $,
%\begin{eqnarray*}
\begin{align*}
%\mathbf Q_\lambda(\exp\{-\langle f, X_t\rangle \})&=&\lim_{t\rightarrow\infty}\mathrm P_\nu\left(\exp\{-\langle f, X_t\rangle \}\Big|\zeta>t\right)=\lim_{t\rightarrow\infty}\dfrac{\mathrm P_\nu\left(\exp\{-\langle f, X_t\rangle \};\zeta>t\right)}{\mathrm P_\nu(\zeta>t)}\\
%&=&\dfrac{\lim_{t\rightarrow\infty}e^{-\lambda t}\mathrm P_\nu\left(\exp\{-\langle f, X_t\rangle \};\zeta>t\right)}{\lim_{t\rightarrow\infty}e^{-\lambda t}\mathrm P_\nu(\zeta>t)}\\
%&=&\dfrac{\int_{{\mathcal M}_f(E)}\mu(\phi)^{-1}e^{-\mu(f)}\mathbf Q(d\mu)}{\int_{{\mathcal M}_f(E)}\mu(\phi)^{-1}\mathbf Q(d\mu)},
&\int_{\mathcal M_f(E)}\exp\{-\langle f, \mu\rangle \}\mathbf Q_\lambda(\mathrm d\mu)
=\lim_{t\rightarrow\infty}\mathrm P_\nu\left(\exp\{-\langle f, X_t\rangle \}\Big|X_t\neq \mathbf 0\right)\\
&=\lim_{t\rightarrow\infty}\dfrac{\mathrm P_\nu\left(\exp\{-\langle f, X_t\rangle \};X_t\neq \mathbf 0\right)}{\mathrm P_\nu(X_t\neq \mathbf 0)}\\
&=\dfrac{\lim_{t\rightarrow\infty}e^{-\lambda t}\mathrm P_\nu\left(\exp\{-\langle f, X_t\rangle \};X_t\neq \mathbf 0\right)}{\lim_{t\rightarrow\infty}e^{-\lambda t}\mathrm P_\nu(X_t\neq \mathbf 0)}\\
&=\dfrac{\int_{{\mathcal M}_f(E)}\mu(\phi)^{-1}e^{-\mu(f)}\mathbf Q(d\mu)}{\int_{{\mathcal M}_f(E)}\mu(\phi)^{-1}\mathbf Q(d\mu)},
%\end{eqnarray*}
\end{align*}
which says that the Yaglom distribution ${\mathbf Q}_\lambda$ can be written as
\begin{equation}\label{rep: yaglom}
%\mathbf Q_\lambda(\cdot)=\dfrac{1}{k}{\mathbf Q}\left(\dfrac{1}{\mu(\phi)}; \mu\in\cdot\right).
\int_{\mathcal M_f(E)}\mu(f)\mathbf Q_\lambda(\mathrm d\mu)=\dfrac{1}{k}\int_{\mathcal M_f(E)}\frac{\mu(f)}{\mu(\phi)}\mathbf Q(\mathrm d\mu).
\end{equation}
Consequently
\begin{equation}\label{ident: k}
%\mathbf Q_\lambda(\mu(\phi))=\dfrac{1}{k}{\mathbf Q}\left(\dfrac{\mu(\phi)}{\mu(\phi) }\right)=\dfrac{1}{k}<\infty.
\int_{\mathcal M_f(E)}\mu(\phi)\mathbf Q_\lambda(\mathrm d\mu)=1.
\end{equation}


(ii) If $\int_El(x)\nu(dx)=\infty$,
by Theorem \ref{thm:E}, 
%$\lim_{t\to\infty}e^{-\lambda t}\mathrm P_\nu(\zeta>t)=k=0$, 
$\lim_{t\to\infty}e^{-\lambda t}\mathrm P_\nu(X_t\neq \mathbf 0)=k=0$,
which implies that
 $\lim_{t\to\infty}e^{-\lambda t}\langle v(t),\nu\rangle=0$.  Since for any $s>0$,  $1-e^{-s}\leq s$, we have
$$G(\langle v(t),\nu\rangle\phi)\geq 1-e^{-G(\langle v(t),\nu\rangle\phi)}.$$
Also note that $\langle v(t),\nu\rangle
%\mathbf Q_\lambda(\mu(\phi))=G(\langle v(t),\nu\rangle\phi).$
\int_{\mathcal M_f(E)}\mu(\phi)\mathbf Q_\lambda(\mathrm d\mu)=G(\langle v(t),\nu\rangle\phi).$
Then  for $t>T_0$,
\[
%\mathbf Q_\lambda(\mu(\phi))
\int_{\mathcal M_f(E)}\mu(\phi)\mathbf Q_\lambda(\mathrm d\mu)
\geq \dfrac{1-e^{-G(\langle v(t),\nu\rangle\phi)}}{\langle v(t),\nu\rangle}.
\]
 Using \eqref{one point ratio limit} with $s=1$,
 there is some $T_0>0$ such that for $t>T_0$, $v(t+1,x)\leq 2\langle v(t),\nu\rangle\phi(x)$, $x\in E$.
 Therefore, as $t\to\infty$,
 \[
%\mathbf Q_\lambda(\mu(\phi))
\int_{\mathcal M_f(E)}\mu(\phi)\mathbf Q_\lambda(\mathrm d\mu)
\geq \dfrac{1-e^{-G(\frac{1}{2}v(t+1,x))}}{\langle v(t),\nu\rangle}\overset{\text{monotonicity}}\geq
\dfrac{1-e^{-G(v(t+1,x))}}{2\langle v(t),\nu\rangle}
=\dfrac{e^{\lambda(t+1)}}{2\langle v(t),\nu\rangle}\to\infty, 
\]
which implies that 
%$\mathbf Q_\lambda(\mu(\phi))=\infty$.
$\int_{\mathcal M_f(E)}\mu(\phi)\mathbf Q_\lambda(\mathrm d\mu)=\infty$.

(3)
%Let $Y_r=\langle\phi, X^{(r)}\rangle$, then $Y_r\geq 0$. 
Recall that $U^{(r)}$ stands for an $\mathcal M_f(E)$-valued random element with distribution $\mathbf Q_r$. Let $Y_r=\langle\phi, U^{(r)}\rangle$, then $Y_r\geq 0$.
Denote $g(u)=\mathbf Q_\lambda\left(e^{-uY_\lambda}\right),\, u\geq 0$. 
%It will be shown in Lemma \ref{prop:G} (2) that
It follows from Lemma \ref{prop:G} (2) that
$$
\mathbf Q_r\left(e^{-uY_r}\right)=1-[1-g(u)]^\alpha,\, u\geq 0.
$$
Applying the identity%, for any $x>0$,
\[
x^\alpha=c\int_0^\infty\dfrac{1-e^{-xt}}{t^{\alpha+1}}dt,
\quad x>0,
\]
where $c=\left(\int_0^\infty\dfrac{1-e^{-t}}{t^{\alpha+1}}dt\right)^{-1}$, and $\mathbf Q_r(\{{\bf 0}\})=0$, we obtain
\begin{eqnarray*}
\mathbf Q_r(\langle\phi, U^{(r)}\rangle^{\alpha})&=&\int_0^\infty x^\alpha \mathbf Q_r(Y_r\in dx)\\
&=&c\int_0^\infty \left(\int_0^\infty\dfrac{1-e^{-xt}}{t^{\alpha+1}}dt \right)\mathbf Q_r(Y_r\in dx)\\
&=&c\int_0^\infty\dfrac{1}{t^{\alpha+1}}dt\int_0^\infty\left(1-e^{-xt}\right)\mathbf Q_r(Y_r\in dx)\\
&=&c\int_0^\infty\dfrac{1-\mathbf Q_r\left(e^{-tY_r}\right)}{t^{\alpha+1}}dt\\
&=&c\int_0^\infty\dfrac{(1-g(t))^\alpha}{t^{\alpha+1}}dt.
\end{eqnarray*}
Since $\dfrac{(1-g(t))^\alpha}{t^{\alpha+1}}\leq t^{-(\alpha+1)}$, $\mathbf Q_r(\langle\phi, U^{(r)}\rangle^{\alpha})<\infty$ if and only if $\int_0^1\dfrac{(1-g(t))^\alpha}{t^{\alpha+1}}dt<\infty$.
For any $M>0$, $1-g(t)\geq 1-\mathbf Q_\lambda\left(e^{-t(Y_\lambda\wedge M)}\right)\sim t \mathbf Q_\lambda(Y_\lambda\wedge M)$ as $t\to 0$. Therefore,
$\dfrac{(1-g(t))^\alpha}{t^{\alpha+1}}\gtrsim t^{-1}$ when $t$ is sufficiently small.  Thanks to $\int_0^1 t^{-1}dt=\infty$,
$\int_0^1\dfrac{(1-g(t))^\alpha}{t^{\alpha+1}}dt=\infty$. The proof is complete.
\end{proof}

\subsection{Proof of Theorem \ref{thm:I}}
%new
Before we give the proof of Proof of Theorem \ref{thm:I}, we prove a lemma first.

\begin{lem}\label{l:new}
There exist $T_0$ and $\eta>0$ such that for all $T>T_0$ and $x\in E$,
$$
\mathrm P_x (X_T\neq\mathbf 0)\le \eta\phi(x)e^{-\lambda T}.
$$
\end{lem}
\begin{proof}
Note that
\begin{equation}\label{e:1}
P_x (X_T\neq\mathbf 0)=1-e^{-v_t(x)}\le v_t(x)
\end{equation}
By Lemma \ref{lem:ratio limit} (3), we have
$$
\lim_{t\to\infty}\sup_{x\in E}\Big|\frac{v_t(x)}{\nu(v_t)\phi(x)}-1\Big|=0.
$$
Thus there exists $t_0$ such that for all $t>t_0$ and $x\in E$,
\begin{equation}\label{e:2}
\frac{v_t(x)}{\nu(v_t)\phi(x)}\le 2.
\end{equation}
Combining \eqref{e:1} and \eqref{e:2} we get that  for all $t>t_0$ and $x\in E$,
\begin{equation}\label{e:3}
\mathrm P_x (X_t\neq\mathbf 0)\le 2 \nu(v_t)\phi(x).
\end{equation}
It follows from Theorem \ref{thm:E} there exist $t_1>0$ and $a>0$ such that 
$1-e^{-\nu(v_t)}\le a e^{\lambda t}$ for all $t\ge t_1$. By Lemma \ref{lem:extinc},
we have
$$
\lim_{t\to\infty}\frac{1-e^{-\nu(v_t)}}{\nu(v_t)}=1.
$$
Thus exists $t_2\ge t_1$ such that for $t\ge t_2$,
$$
\frac{1-e^{-\nu(v_t)}}{\nu(v_t)}\ge 2.
$$
Hence for $t\ge t_2$ we have $\nu(v_t)\le\frac{a}2e^{\lambda t}$.
Combining this with \eqref{e:3} we immediately get the desired result.
\end{proof}
%end new
%\begin{proof}
\begin{proof}[Proof of Theorem \ref{thm:I}]
According to the spine decomposition of $\{(X_t)_{t\geq 0}, \widetilde{\mathrm P}_\mu\}$  given by
\eqref{spine-decom2}, for any
$f\in\mathcal B_b(E,[0,\infty))$,
\[
\widetilde {\mathrm P}_{\mu}\left(e^{-\langle f, X_t\rangle }\right)=\mathrm Q_{\mu}\left(e^{-\langle f, X_t\rangle+\langle f, Z^{{\mathrm m},[0,t)}_t+Z^{{\mathrm n},[0,t)}_t\rangle }\right).
\]

(1) Suppose $\int_El(x)\nu(dx)<\infty$.
%When $\mu(dx)=\nu(dx)=\hat\phi(x)m(dx)$, 
When $\mu=\nu$, 
it has been shown in the proof of Theorem \ref{thm:L} that the distribution of
$X_t$ under $\widetilde {\mathrm P}_{\nu}$ converges weakly to $\mathbf Q$ as $t\to\infty$.
%It is also shown in \eqref{rep: yaglom} that $\mathbf Q$ is related to the Yaglom distribution:
%\begin{equation}\label{eq:2}
%\mathbf Q_\lambda(\cdot)=\dfrac{1}{k}{\mathbf Q}\left(\dfrac{1}{\mu(\phi) }; \mu\in\cdot\right).
%\end{equation}
It was also shown there that $\mathbf Q$ is related to the Yaglom distribution $\mathbf Q_\lambda$ via \eqref{rep: yaglom}.
Note that $k=[\mathbf Q_\lambda(\mu(\phi))]^{-1}$. 
%\eqref{eq:2} can be rewritten as
\eqref{rep: yaglom} can be rewritten as
\begin{equation}\label{eq size bias}
{\mathbf Q}\left(\mu\in\cdot\right)=
%\dfrac{\mathbf Q_\lambda(\mu(\phi); \mu\in \cdot)}{\mathbf Q_\lambda(\mu(\phi))}.
\dfrac{\int_{\mathcal M_f(E)}\mu(\phi)\mathbf 1_{\mu\in \cdot} \mathbf Q_\lambda(\mathrm d\mu)}{\int_{\mathcal M_f(E)}\mu(\phi)\mathbf Q_\lambda(\mathrm d\mu)}.
\end{equation}
We now prove that for all $\mu\in\mathcal M^o_f(E)$, the distribution of $X_t$ under $\widetilde {\mathrm P}_{\mu}$ converges weakly to $\mathbf Q$.  We do this in three steps.

{\bf Step 1}\quad For $f\in\mathcal B_b(E,[0,\infty))$, define
\begin{equation}\label{def: H}
H(x,t):={\mathrm Q}_x\left(e^{-\langle f, Z_{t}\rangle }\right)={\mathrm Q}_x\left(e^{-\langle f, Z^{\mathrm n}_{t} + Z^{\mathrm m}_{t}\rangle }\right),\quad x\in E.
\end{equation}
Set $\overline \eta(x):=\limsup_{t\to\infty}H(x,t), x\in E$.
In this step we prove that 
%$\overline \eta(\cdot)$ is a constant function.
$\overline\eta(\cdot)$ is $\nu$-almost surely a constant function.

Define  $\mathcal{H}_t=\sigma\big(\xi_s; s\leq t\big)$, $t\geq 0$, which is  the filtration generated by the spine process.  Then for $T,t>0$,
\begin{equation}\label{subcritical upper bound}
 \begin{aligned}
 &H(x,t+T)\\
 =&\mathrm Q_{x}\mathrm Q_{x}\Big[\exp\Big\{-\sum_{\sigma\in (0, t+T]\bigcap \mathcal D^{\mathrm m}}\langle f, X_{t+T-\sigma}^{{\mathrm m},\sigma}\rangle -\sum_{\tau\in (0, t+T]\bigcap \mathcal D^{\mathrm n}}\langle f, X_{t+T-\tau}^{{\mathrm n}, \tau}\rangle \Big\}\Big| \mathcal H_t\Big]\\
 \leq&\widetilde\Pi_x\mathrm Q_{x}\Big[\exp\Big\{-\sum_{\sigma\in (t, t+T]\bigcap \mathcal D^{\mathrm m}}\langle f, X_{t+T-\sigma}^{{\mathrm m},\sigma}\rangle -\sum_{\tau\in (t, t+T]\bigcap \mathcal D^{\mathrm n}}\langle f, X_{t+T-\tau}^{{\mathrm n}, \tau}\rangle \Big\}\Big| \mathcal H_t\Big]\\
 =&
   \widetilde\Pi_x\mathrm Q_{\xi_t}\Big[\exp\Big\{-\sum_{\sigma\in (0, T]\bigcap \mathcal D^{\mathrm m}}\langle f, X_{T-\sigma}^{{\mathrm m},\sigma}\rangle -\sum_{\tau\in (0, T]\bigcap \mathcal D^{\mathrm n}}\langle f, X_{T-\tau}^{{\mathrm n}, \tau}\rangle \Big\}\Big]\\
 =&\widetilde\Pi_x\left[ H(\xi_t, T)\right].
 \end{aligned}
 \end{equation}
% From \eqref{IU}, there are some constants $c,\rho>0$ such that when $t>1$,
By (H2), there are some constants $c,\rho>0$ such that when $t>1$,
\[
 H(x,t+T)\leq \widetilde\Pi_x\left[ H(\xi_t, T)\right]\leq 
 %(1+ce^{-\rho t})\int_E\phi(y)\widehat\phi(y)H(y,T)m(dy)<\infty.
 (1+C^{\eqref{asp:H2}}_{t, x, H(\xi_t, t)})\int_E\phi(y)H(y,T)\nu(\mathrm dy)<\infty.
 \]
For fixed  $T>0$, letting $t\to \infty$ in \eqref{subcritical upper bound}, we obtain that
\begin{equation}\label{sub super}
\overline\eta(x)\leq \int_E\phi(y)H(y,T)\nu(\mathrm dy).
\end{equation}
   Using Fatou's lemma, we get that, for any $x\in E$,
\begin{equation}\label{sup inequality}
\overline\eta(x)\leq
\limsup_{T\rightarrow\infty}\int_E\phi(y)H(y,T)\nu(\mathrm dy)
\leq \int_E\phi(y)\overline{\eta}(y)\nu(\mathrm dy).
\end{equation}
Using the fact that  $\overline{\eta}(\cdot)\leq 1$, 
%$\overline\eta(\cdot)$ is a constant function.
we see that $\overline\eta(\cdot)$ is $\nu$-almost surely a constant function.

{\bf Step 2}\quad
 Denote the $\nu$-\text{a.s.} value of the function $\overline\eta(\cdot)$ by $q(f)$.  In this step we prove that
 \begin{equation}\label{limit-H}
 \lim_{t\rightarrow\infty}H(x,t)=q(f),\qquad 
 %\mbox{for all}\,\, x\in E.
 \text{ for } \nu-\text{a.e.}x\in E.
 \end{equation}
If $q(f)= 0,$ then the above is true obviously. So in the
following, we assume $q(f)>0$.

We first claim that  as $T\to\infty$, $H(\cdot,T)$ converges to $q(f)$ in probability  
%under probability $\phi(x)\widehat{\phi}(x)m(dx)$. In fact,
with respect to the probability measure $\phi(x)\nu(\mathrm dx)$. 
%In fact, for any $\varepsilon_1>0$, let
For  any $\varepsilon_1\in (0, 1)$, let
$$
\mu_1(T)=\int_{\{x\in E;H(x,T)>(1+\varepsilon_1)q(f)\}}
\phi(x)\nu(\mathrm dx).
$$
Then $\limsup_{T\to\infty}H(x,T)=q(f)$ implies that $\lim_{T\rightarrow\infty}\mu_1(T)=0.$  %For any $\varepsilon_2>0$, let
For any $\varepsilon_2\in (0, 1)$, let
$$
\mu_2(T)=\int_{\{x\in E;H(x,T)<(1-\varepsilon_2)q(f)\}}
\phi(x)\nu(\mathrm dx).
$$
To prove the claim above, we only need to prove that  $\limsup_{T\rightarrow\infty}\mu_2(T)=0.$
It follows  from \eqref{sup inequality} that
\begin{eqnarray}\label{sublimitinprob}
q(f)&\leq&
(1-\varepsilon_2)q(f)\mu_2(T)+\mu_1(T)+(1+\varepsilon_1)q(f)(1-\mu_1(T)-\mu_2(T))\\
&\le
&(1+\varepsilon_1)q(f)-(\varepsilon_1+\varepsilon_2)q(f)\mu_2(T)+\mu_1(T).
\end{eqnarray}
 Hence
\begin{eqnarray*}\label{sublimitinequl}
q(f)&\leq&
\liminf_{T\rightarrow\infty}\left[(1+\varepsilon_1)q(f)-(\varepsilon_1+\varepsilon_2)\mu_2(T)%+C\mu_1(T)\right]\\
+\mu_1(T)\right]\\
&=&(1+\varepsilon_1)q(f)-(\varepsilon_1+\varepsilon_2)q(f)\limsup_{T\rightarrow\infty}\mu_2(T).
\end{eqnarray*}
Since $\varepsilon_1$ is an arbitrary positive constant.
\[
q(f)\leq q(f)-\varepsilon_2 q(f)\limsup_{T\rightarrow\infty}\mu_2(T).
\]
This is impossible unless $\limsup_{T\rightarrow\infty}\mu_2(T)=0.$
%new
Thus the claim above is valid.
%end


%Meanwhile, from the definition of $H$ given by \eqref{def: H}, we get the following domination:
By the definition of $H$ given by \eqref{def: H},  we have
\begin{equation}\label{subsub}
\begin{aligned}
     H(x,t+T)\geq& \mathrm Q_{x}\prod_{\sigma\leq t}I_{\{ X_{t+T-\sigma}^{{\mathrm m},\sigma}=0\}}\prod_{\tau\leq t}I_{\{ X_{t+T-\tau}^{{\mathrm n},\tau}=0\}}\\
&\cdot\mathrm Q_{\xi_t}\Big[\exp\Big\{-\sum_{\sigma\in (0, T]\bigcap \mathcal D^{\mathrm m}}\langle f, X_{T-\sigma}^{{\mathrm m},\sigma}\rangle -\sum_{\tau\in (0, T]\bigcap \mathcal D^{\mathrm n}}\langle f, X_{T-\tau}^{{\mathrm n},\tau}\rangle \Big\}\Big]\\
=& \mathrm Q_{x}\left[\prod_{\sigma\leq t}I_{\{ X_{t+T-\sigma}^{{\mathrm m},\sigma}=0\}}\prod_{\tau\leq t}I_{\{ X_{t+T-\tau}^{{\mathrm n},\tau}=0\}}H(\xi_t, T)\right].
\end{aligned}
\end{equation}
Note that
\begin{eqnarray*}
\mathrm Q_{x}\left(\prod_{\sigma\leq t}I_{\{ X_{t+T-\sigma}^{{\mathrm m},\sigma}=0\}}=1\right)
=\widetilde\Pi_x\exp\left\{-\int_0^t\mathrm ds\int_0^\infty r(1-\mathrm P_{r\delta_{xi_s}}(X_{T+t-s}\neq\mathbf 0))\pi(\xi_s,\mathrm dr)\right\}.
\end{eqnarray*}
Lemma \ref{lem:extinc} (2) tells us the $(\xi,\psi)$-superprocess starting from any finite measure becomes extinct in finite time.  Therefore
 by the dominated convergence theorem,
\begin{equation}\label{1infty limit}
\lim_{T\rightarrow\infty}\int_0^t\mathrm ds\int_1^\infty r(1-\mathrm P_{r\delta_{\xi_s}}(X_{T+t-s}\neq\mathbf 0))\pi(\xi_s,\mathrm dr)=0,\quad \widetilde\Pi_x-\mbox{a.s.}
\end{equation}
  Note that
\[
1-\mathrm P_{r\delta_{\xi_s}}(X_{T+t-s}\neq\mathbf 0)\leq 1-(1-\mathrm P_{\xi_s}(X_T\neq\mathbf 0))^r.
\]
%By Lemma \ref{lem:extinc}, 
By Lemma \ref{l:new},
there are $T_0>0$ and $\eta>0$ such that when $T>T_0$, 
\[
\mathrm P_x(X_T\neq \mathbf 0)\leq \eta \phi(x)e^{\lambda T}, \quad x\in E.
\]
%Since when $x\rightarrow 0+$, $1-(1-x)^r\sim rx$ for any $r>0$, we may take $T_0$  sufficiently large such that
Using elementary analysis one can easily check that 
$$
1-(1-a)^r\le 2 ra, \quad a\le (0, \frac12), r\in (0, 1].
$$
By taking $T_0$  sufficiently large we have 
%$\eta \phi(x)e^{\lambda T}$ is small enough so that $1-(1-\mathrm P_{Y_s}(\zeta>T))^r\leq 2r\eta \phi(Y_s)e^{\lambda T}$ for all  $T>T_0$ and $r\in(0,1]$.
$1-(1-\mathrm P_{\xi_s}(\zeta>T))^r\leq 2r\eta \phi(\xi_s)e^{\lambda T}$ for all  $T>T_0$ and $r\in(0,1]$.
Therefore,
\[
%\int_0^tds\int_0^1 r(1-\mathrm P_{r\delta_{Y_s}}(\zeta<T+t-s))\pi(Y_s,dr)\leq 2\eta %e^{\lambda T}\int_0^t\phi(Y_s)ds\int_0^1 r^2 \pi(Y_s,dr).
\int_0^tds\int_0^1 r(1-\mathrm P_{r\delta_{ \xi_s}}(X_{T+t-s}\neq \mathbf 0))\pi(\xi_s,dr)\leq 2\eta e^{\lambda T}\int_0^t\phi(\xi_s)ds\int_0^1 r^2 \pi(\xi_s,dr).
\]
%Then by the dominated convergence theorem,
Thus
\begin{equation}\label{01limit}
\lim_{T\rightarrow\infty}\int_0^t\mathrm ds\int_0^1 r(1-\mathrm P_{r\delta_{\xi_s}}(X_{T+t-s}\neq \mathbf 0))\pi(\xi_s,\mathrm dr)=0, \quad \widetilde\Pi_x-\mbox{a.s.}
\end{equation}
 Combining \eqref{1infty limit} and \eqref{01limit}, we get
\[
\lim_{T\rightarrow\infty}\mathrm Q_{x}\left(\prod_{\sigma\leq t}I_{\{ X_{t+T-\sigma}^{{\mathrm m},\sigma}=0\}}=1\right)=1.
\]
Similarly,
\begin{eqnarray*}
\mathrm Q_x\left(\prod_{\sigma\leq t}I_{\{ X_{t+T-\sigma}^{{\mathrm n},\sigma}=0\}}=1\right)
&=&\widetilde\Pi_x\exp\left\{-\int_0^t2\sigma(\xi_s)^2\mathrm N_{\xi_s}(X_{T+t-s}\neq \mathbf 0)\mathrm ds\right\}\\
&=&\widetilde\Pi_x\exp\left\{-\int_0^t2\sigma(\xi_s)^2v_{T+t-s}(\xi_s)\mathrm ds\right\}.
\end{eqnarray*}
We know that $v_{T+r}(x)$ is  bounded for $(r,x)\in (0,\infty)\times E$ when $T$ is large enough, and that  $\lim_{T\rightarrow\infty} v_{T+t-s}(x)=0$ for any $x$.  Thus by the bounded convergence theorem, 
\[
\lim_{T\rightarrow\infty}\mathrm Q_x\left(\prod_{\tau\leq t}I_{\{ X_{t+T-\tau}^{{\mathrm n},\tau}=0\}}=1\right)=1,
\]
%By the inequality \eqref{IU}, 
By (H2), 
for any $\varepsilon>0$ and $t>1$,
such that for any $x\in E$,
\begin{eqnarray*}
%&&\limsup_{T\rightarrow\infty}\widetilde\Pi_x\left(|H(Y_t, T)-q(f)|>\varepsilon\right)\\
%&\leq& \limsup_{T\rightarrow\infty}(1+ce^{-\rho t})\int_E\phi(y)\widehat\phi(y)m(dy)I_{\{|H(y, T)-q(f)|>\varepsilon\}}=0.
&&\limsup_{T\rightarrow\infty}\widetilde\Pi_x\left(|H(\xi_t, T)-q(f)|>\varepsilon\right)\\
&\leq& \limsup_{T\rightarrow\infty}(1+C^{\eqref{asp:H2}}_{t, x, H(\cdot, T)})\int_E\mathbf 1_{\{|H(y, T)-q(f)|>\varepsilon\}}\phi(y)\nu(\mathrm dy)=0.
\end{eqnarray*}
Then from the inequality \eqref{subsub}, we have for any $x\in E$,
\begin{eqnarray*}
\liminf_{T\rightarrow\infty}H(x, t+T)&\geq&  \liminf_{T\rightarrow\infty} \mathrm Q_x\left[\prod_{\sigma\leq t}I_{\{ X_{t+T-\sigma}^{{\mathrm m},\sigma}=0\}}\prod_{\tau\leq t}I_{\{ X_{t+T-\tau}^{{\mathrm n},\tau}=0\}}H(\xi_t, T)\right]\\
&\geq& q(f)=\limsup_{t\rightarrow\infty}H(x, t).
\end{eqnarray*}
 Therefore \eqref{limit-H} holds.

{\bf Step 3}\quad
Since $0\leq H(x,t)\leq 1$,
\begin{equation*}
q(f)
=\lim_{t\rightarrow\infty}\int_E\phi(x)H(x,t)\nu(\mathrm dx)
=\lim_{t\rightarrow\infty}\widetilde{\mathrm P}_{\nu}\left(e^{-\langle f, X_t\rangle }\right)
=\mathbf Q(e^{-\mu(f)}).
\end{equation*}
Therefore for any $\mu\in\mathcal M^o_f(E)$, and $f\in\mathcal B_b(E,[0,\infty))$,
\begin{eqnarray*}
\lim_{t\rightarrow\infty}\widetilde{\mathrm P}_\mu\left(e^{-\langle f, X_t\rangle}\right)&=&\lim_{t\rightarrow\infty}\mathrm P_\mu\left(e^{-\langle f, X_t\rangle}\right)
\lim_{t\to\infty}\dfrac{1}{\mu(\phi)}\int_E\phi(x)H(x, t)\mu(\mathrm dx)\\
&=&q(f)=\mathbf Q(e^{-\mu(f)}).
\end{eqnarray*}
This says $\mathbf Q$, the distribution limit of $X_t$ under $\widetilde{\mathrm P}_{\nu}$, is the  limit distribution of  $X_t$ under $\widetilde{\mathrm P}_{\mu}$ for any $\mu\in{\mathcal M}^o_f(E)$. Thus $\mathbf Q$ is the
equilibrium distribution of the $Q$-process, and is a size-biased distribution of the Yaglom probability $\mathbf Q_\lambda$ with weight function $\dfrac{\mu(\phi)}{\int_{{\mathcal M}_f(E)}\mu(\phi)\mathbf Q_\lambda(\mathrm d\mu)}$.  The proof of $(1)$ is complete.



(2) When $\int_El(x)\nu(dx)=\infty$,
% it is shown in Theorem \ref{thm:E} that
% $\sum_{s\in\mathcal D^{\mathrm m}} \langle \phi,\widehat X^{{\mathrm m},s}_s\rangle =\infty$, $\widehat{\mathrm Q}_\nu$ almost surely. 
%Thus
%\[
%\langle \phi, \widehat Z_{\infty}\rangle =\infty,\qquad \widehat{\mathrm Q}_\nu-{\mathrm a.s.}
%\]
%In this case, for any $\mu\in \mathcal M_f(E)\backslash\{0\}$, $\lim_{t\to\infty}\langle \phi, X_t\rangle =\infty$ in probability with respect to $\widetilde{\mathrm P}_\mu$. The  $Q$-process $\{(X_t)_{t\geq 0}; \widetilde{\mathrm P}_{\mu}\}$ does not have equilibrium distribution.
by Theorem \ref{thm:E}, for any $\mu\in \mathcal M^0_f(E)$, we have
$$
\lim_{to\infty}e^{-\lambda t}\mathrm P_\mu(X_t\neq\mathbf 0)=0.
$$
By \eqref{eq:M.3}, we have
$$
e^{-\lambda t}\mathrm P_\mu(X_t\neq\mathbf 0)=\mu(\phi)\widetilde{P}_\mu[X_t(\phi)^{-1}].
$$
Combining the two displays above we get that $X_t(\phi)^{-1}$ converges to zero in probability
with respect to $\mathrm P_\mu$, which is equivalent to the desired result.
\end{proof}

\appendix
\section{}

\subsection{Weak convergence and vague convergence}
\begin{lem} \label{thm:A.1}
	Let $E$ be a Polish space.
	Let $\{\mu\}\cup\{\mu_n:n\in \mathrm N\}$ be a sequence of finite Borel measures on $E$.
%	Suppose that $\mu_n$ convergence to $\mu$ vaguely when $n\to \infty$, and that $\mu_n(E)$ convergence to $\mu(E)$ when $n\to \infty$.
	Suppose that $\mu_n$ converges to $\mu$ vaguely as $n\to \infty$, and that $\mu_n(E)$ converges to $\mu(E)$ as $n\to \infty$.
%	Then $\mu_n$ convergence to $\mu$ weakly.
	Then $\mu_n$ converges to $\mu$ weakly.
\end{lem}
\begin{proof}
	Let $(\phi_j)_{j\in \mathrm N}$ be a sequence of function on $E$ such that
	(1) for each $j\in \mathrm N$, $\phi_j$ is compactly supported, $[0,1]$-valued, and continuous;
	(2) for each $x\in E$, $\phi_j(x)\xrightarrow[j\to \infty]{} 1$.
	Fix an arbitrary bounded continuous function $f$ on $E$.
	Then we can verify that for each $n$ and $j\in \mathrm N$,
 \begin{align}
 	&  |\mu_n(f) -\mu(f)|
 	= \big|\big[\mu_n(f\phi_j) + \mu_n\big(f(1-\phi_j)\big) \big]- \big[\mu(f\phi_j) + \mu\big(f(1-\phi_j)\big)\big]\big|
 	\\&\leq|\mu_n(f\phi_j) - \mu(f\phi_j)| + \big|\mu_n\big(f(1-\phi_j)\big)\big| + \big|\mu\big(f(1-\phi_j)\big)\big|
 	\\&\leq \big|\mu_n(f\phi_j) - \mu(f\phi_j)\big| + \|f\|_\infty \mu_n(1-\phi_j) + \|f\|_\infty\mu(1-\phi_j)
 	\\&\leq \big|\mu_n(f\phi_j) - \mu(f\phi_j)\big| + \|f\|_\infty [\mu_n(1-\phi_j)-\mu(1-\phi_j)] + 2\|f\|_\infty\mu\big(1-\phi_j\big)
 	\\&\leq |\mu_n(f\phi_j)-\mu(f\phi_j)| + \|f\|_\infty \big(|\mu_n(E)- \mu(E)| + |\mu_n(\phi_j)-\mu(\phi_j)|\big) + 2\|f\|_\infty \mu(1-\phi_j).
 \end{align}
 From this and the assumptions of the lemma, we have for each $j\in \mathrm N$,
 \begin{align}
 	\limsup_{n\to \infty} |\mu_n(f) - \mu(f)| \leq 2\|f\|_\infty \mu(1-\phi_j).
 \end{align}
 Finally, taking $j \to \infty$ above, using bounded convergence theorem, we have
 \[
 \lim_{n\to \infty} |\mu_n(f)-\mu(f)| = 0. \qedhere
 \]
\end{proof}

\subsection{Tauberian theorems}

In this subsection we collect some results on regularly varying functions and  Tauberian theorems, which are used in this paper.
The following lemma is from \cite[Appendix 13.6]{AH}.
\begin{lem}\label{lem:regu}
Let $f(x)$ be non-decreasing in $x\in (0,c)$. If
\[
\lim_{n\to\infty}\dfrac{f(\lambda\theta_n)}{f(\theta_n)}=\lambda^\alpha,\qquad \forall \lambda\in (0,1],
\]
for some $\alpha\in\mathbb R$ and some sequence $\{\theta_n\}$ of positive reals tending to $0$, as $n\to\infty$ in such a way that $\theta_n/\theta_{n+1}\leq c$ for $n\in\mathrm N$ and some $1<c<\infty$, then $f(x)$ is regularly varying with exponent $\alpha$.
\end{lem}


The following two lemmas are from \cite[Appendix 14]{AH}.
\begin{lem}\label{lem: tau}
Let $U(x)$ be a non-decreasing function on $[0,\infty)$ such that
\[
w(x)=\int_0^\infty e^{-xu} \mathrm dU(u)
\]
is finite for all $x>0$. If for some $\alpha\geq 0$, $w(x)\sim x^{-\alpha}L(1/x)$, $x\downarrow 0$, where $L$ is slowly varying at infinity, then
\[
U(x)\sim x^{\alpha}\dfrac{L(x)}{\Gamma(\alpha+1)},\qquad x\to\infty.
\]
 And if for some $\alpha\geq 0$, $w(x)\sim x^{-\alpha}L(x)$, $x\uparrow \infty$, then
\[
U(x)\sim x^{\alpha}\dfrac{L(1/x)}{\Gamma(\alpha+1)},\qquad x\downarrow 0.
\]
\end{lem}

\begin{lem}\label{lem:tail}
For $x\in [\beta,\infty)$, let
\[
U(x)=\int_\beta^xu(y)\mathrm dy,
\]
where $u(y)$ is ultimately monotone.  If for some $\alpha\geq 0$, $U(x)=x^\alpha L(x)$,  where $L$ is slowly varying at infinity, then
\[
\lim_{x\to\infty}\dfrac{xu(x)}{U(x)}=\alpha.
\]
\end{lem}

\subsection{Proof of Lemma   \ref{thm:E.4}}
\begin{proof}[Proof of Lemma \ref{thm:E.4}] 
(1) Suppose $\nu(l)<\infty$.
For any $N>0$, we have
\begin{equation}\label{sum}
\begin{array}{rl}
&\displaystyle\sum_{k\in \mathbb N}{\bf 1}_{s_k\in(-\infty,0]}e^{\epsilon s_k}y_k\phi(\xi_{s_k})\\
=&\displaystyle\sum_{k\in \mathbb N}e^{\epsilon s_k}y_k\phi(\xi_{s_k})
{\bf 1}_{\{\phi(\xi_{s_k})y_k\le e^{-Ns_k},s_k<0\}}
+\sum_{k\in \mathbb N}e^{\epsilon s_k}y_k\phi(\xi_{s_k}){\bf 1}_{\{\phi(\xi_{s_k})y_k>e^{-Ns_k},s_k<0\}}\\
=&I+II.
\end{array}
\end{equation}
Note that
\begin{eqnarray*}
&&\sum_{k\in\mathbb N} \mathrm Q\left(y_k\phi(\xi_{s_k}) > e^{-Ns_k},s_k<0\right)
=\sum_{k\in\mathbb N} \mathrm Q\left[\mathrm Q\left(y_k\phi(\xi_{s_k})>e^{-N s_k}, s_k<0\big|\sigma(\xi)\right)\right]\\
&=&\mathrm Q\left[\mathrm Q\left(\sum_{k\in\mathbb N}{\bf 1}_{\{y_k>e^{-N s_k}\phi(\xi_{s_k})^{-1},s_k<0\}}\Big|\sigma(\xi)\right)\right]=\widetilde{\nu\Pi}\left[\int_{-\infty}^0 \left(\int^{\infty}_{\phi(\xi_s)^{-1}e^{-N s}}r \pi(\xi_s, dr)\right)\mathrm ds\right]\\
&=&\int_{-\infty}^0  \mathrm ds\widetilde{\Pi}_\nu\left(\int^{\infty}_{\phi(\xi_s)^{-1}e^{-N s}}r \pi(\xi_s, \mathrm dr)\right)=\int_{-\infty}^0 \mathrm ds \int_E \phi(y)\nu(\mathrm dy)\int^{\infty}_{\phi(y)^{-1} e^{-N s}}r\pi(y, \mathrm dr)\\
&=& \int_E\phi(y)\nu(\mathrm dy)\int_{\phi(y)^{-1}}^\infty r\pi(y, \mathrm dr)\int^{\frac{\ln (r\phi(y))}{N}}_{0}\mathrm ds=\dfrac{\nu(l)}{N}.
\end{eqnarray*}
By the assumption that $\nu(l)<\infty$ and the Borel-Cantelli lemma,  we get
\begin{equation}\label{io}
\mathrm Q\Big(y_k\phi(\xi_{s_k})>e^{-N s_k},s_k<0,\mbox{ i. o.}\Big)=0
\end{equation}
for all $N>0$,  which implies that
\begin{equation}\label{big}
II<\infty.\quad \mathrm Q\mbox{-a.s.}
\end{equation}
Note also that
$$
\begin{array}{rl}
&\mathrm Q(I)=\displaystyle \mathrm Q\left(\mathrm Q\Big[\sum_{k\in\mathbb N} e^{\epsilon s_k}y_{s_k}\phi(\xi_{s_k}) {\bf 1}_{\{y_k\le e^{-N s_k}\phi(\xi_{s_k})^{-1}, s_k<0\}}\big|\sigma(\xi)\Big]\right)\\
&=\displaystyle \widetilde{\Pi}_\nu\int_{-\infty}^0 \mathrm dt e^{\epsilon t}\int_0^{\phi({\xi}_t)^{-1} e^{-N t}}\phi(\xi_t)r^2 \pi(\xi_t, \mathrm dr)\\
&\le \displaystyle\|\phi\|_{\infty}\widetilde{\Pi}_\nu\int_{-\infty}^0 \mathrm dt e^{\epsilon t}\int_0^1r^2 \pi(\xi_t, \mathrm dr)
+\widetilde{\Pi}_\nu\int_{-\infty}^0 \mathrm dt e^{(\epsilon-N)t}\int_1^{\infty}r \pi(\xi_t, \mathrm dr)\\
&= \displaystyle\|\phi\|_{\infty}\int_{-\infty}^0  e^{\epsilon t}\mathrm dt\int_E\phi(x)\nu(\mathrm dx)\int_0^1r^2 \pi(x, \mathrm dr)
+\int_{-\infty}^0  e^{(\epsilon-N)t}\mathrm dt\int_E\phi(x)\nu(\mathrm dx)\int_1^{\infty}r \pi(x, \mathrm dr),
\end{array}
$$
where in  the inequality we used the fact that $r\le\phi(\xi_t)^{-1} e^{-N t}$ implies that $r\phi(\xi_t)\le e^{-N t}$.
%By the assumption that $\sup_{x\in E}\int_0^\infty (r\wedge r^2) \pi(x, dr)<\infty$, and %choosing $N>\epsilon$, we have
Since $\sup_{x\in E}\int_0^\infty (r\wedge r^2) \pi(x, \mathrm dr)<\infty$, by choosing $N\in (0, \epsilon)$, we get
$\mathrm Q(I)<\infty$, which implies that
\begin{equation}\label{small}
I<\infty,\quad \mathrm Q\mbox{-a.s.}
\end{equation}
Combining \eqref{sum}, \eqref{big} and \eqref{small}, we see that
$$
\sum_{k\in \mathbb N}{\bf 1}_{s_k\in(-\infty,0]}e^{\epsilon s_k}y_k\phi(\xi_{s_k})<\infty,\qquad \mathrm Q\mbox{-a.s.}
$$


(2) Suppose that $\nu(l)=\infty$ and $\epsilon>0$.
Put $K_0:=1\vee(\max_{x\in E}\phi(x))$.
Then for $K\ge K_0$, $ K\inf_{x\in E}\phi^{-1}(x)\geq 1$.   We  only need to prove that for $K\ge K_0$,
\begin{equation}\label{inteqinfty'}
\int_{-\infty}^{0} \mathrm dt\int_{K\phi(\xi_t)^{-1}e^{-\epsilon t}}^\infty r\pi(\xi_t,\mathrm dr)
=\infty,\quad \mathrm Q\mbox{-a.s.}
\end{equation}
To prove \eqref{inteqinfty'}, we first prove that
\begin{equation}\label{mean=infty}
 \mathrm Q\left[\int_{-\infty}^{0} \mathrm dt\int_{K\phi(\xi_t)^{-1}
e^{-\epsilon t}}^\infty r \pi(\xi_t, \mathrm dr)\right]=\infty.
\end{equation}
Applying Fubini's theorem, we get
\begin{eqnarray*}
&&\mathrm Q\Big[\int_{-\infty}^{0} \mathrm dt\int_{K\phi(\xi_t)^{-1}e^{-\epsilon t}}^{\infty} r \pi(\xi_t, \mathrm dr)\Big]=\int_{-\infty}^{0} \mathrm dt \widetilde{\Pi}_\nu\int_{K\phi(\xi_t)^{-1}e^{-\epsilon t}}^{\infty} r\pi(\xi_t, \mathrm dr)\\
&=& \int_{-\infty}^{0} \mathrm dt\int_E\phi(y)\nu(\mathrm dy)\int_{K\phi(y)^{-1}e^{-\epsilon t}}^\infty r\pi(y, \mathrm dr)\\
&=& \int_E\phi(y)\nu(\mathrm dy)\int_{K\phi(y)^{-1}}^\infty r\pi(y, \mathrm dr)\int_{0}^{\frac{1}{\epsilon}\ln(\frac{r\phi(y)}{K})}\mathrm dt\\
&=&\frac{1}{\epsilon}\int_E\phi(y)\nu(\mathrm dy)\int_{K\phi(y)^{-1}}^\infty(\ln[r\phi(y)]-\ln K)r\pi(y, \mathrm dr)\\
&\ge&\frac{1}{\epsilon}\int_E\phi(y)\nu(\mathrm dy)\left[\int_{K\phi(y)^{-1}}^\infty r\ln[r\phi(y)]\pi(y, \mathrm dr)-A\right]\\
&=&\frac{1}{\epsilon}\int_E\nu(\mathrm dy) \int_{K}^\infty r\ln r\pi^\phi(y, \mathrm dr)-\frac{A}{\epsilon}\int_E\phi(y)\nu(\mathrm dy),
\end{eqnarray*}
for some positive constant $A$, where in the inequality we used the
facts that $K\phi(y)^{-1}>1$ for any
$y\in E$ and $\sup_{y\in E}\int^\infty_1r \pi(y,\mathrm dr)<\infty$.
Since
$$
\nu(l)=\int_E\nu(\mathrm dy)\int_1^\infty r\ln r \pi^\phi(y,\mathrm dr)=\infty,
$$
and
$$
\int_E\nu(\mathrm dy)\int_{1}^{K} r\ln r \pi^\phi(y, \mathrm dr)\leq K\log K\int_E \pi(y,[\|\phi\|_{\infty}^{-1},\infty))\nu(\mathrm dy)<\infty,
$$
we get that
$$
\int_E\nu(\mathrm dy)\int_{K}^\infty r\ln r \pi^\phi(y, \mathrm dr)=\infty,
$$
and therefore, \eqref{mean=infty} holds.


%When assumption \eqref{asp:H2} holds, 
By  \eqref{asp:H2},
there exists constant $c>0$ such that for any $t>c$, $\mu\in\mathcal M^\circ_f(E)$ and  $f\in \mathcal B_b(E,[0,\infty))$,
\begin{equation}\label{domi-p}
 \frac{1}{2}\nu(\phi f)\leq \widetilde{\Pi}_\mu f(\xi_t)\leq 2\nu(\phi f).
\end{equation}
For $T<-c$, we define
$$
 \eta_T=\int_{T}^0 \mathrm dt\int_{K\phi(\xi_t)^{-1}e^{-\epsilon t}}^\infty r\pi(\xi_t, \mathrm dr).
$$
Then the expectation of $\eta_T$ under $\mathrm Q$ is given by
$$
 \mathrm Q(\eta_T)=\int_0^{-T}\mathrm dt\int_E\phi(y)\nu(\mathrm dy)\int_{K e^{\epsilon t}\phi(y)^{-1}}^\infty r\pi(y, \mathrm dr),
$$
%which is finite for any finite $T\in (-\infty, 0]$
which is finite and psoitive  for any finite $T\in (-\infty, 0)$
thanks to the fact that $K e^{\epsilon t}\phi(y)^{-1}>1$ for all $y\in E$. 
% Since $\left\{\eta_{-\infty}=\infty\right\}$ is an invariant event, by
%the ergodic property of $\xi$ under $\widetilde{\nu\Pi}$, 
By (H2), $\xi=(\xi_t, \widetilde \Pi)$ is ergodic. Since $\left\{\eta_{-\infty}=\infty\right\}$ is an invariant event, 
to prove \eqref{inteqinfty'}  it suffices to show
\begin{equation}\label{positive-prob}
\mathrm Q\left(\eta_{-\infty}=\infty\right)=\widetilde{\Pi}_\nu\left(\eta_{-\infty}=\infty\right)>0.
\end{equation}
By  the Cauchy-Schwartz inequality, we have
\begin{equation}\label{Durrett-domi}
\mathrm Q\left(\eta_T\geq \frac{1}{2}\mathrm Q(\eta_T)\right)\geq \frac{(\mathrm Q(\eta_T))^2}{4\mathrm Q(\eta_T^2)}.
\end{equation}
If we can prove that there is a constant $\widehat C>0$ such that
for all $T<-c,$
\begin{equation}\label{uniform lower bound}
\dfrac{(\mathrm Q(\eta_T))^2}{4\mathrm Q(\eta_T^2)}\geq \widehat C.
\end{equation}
Then by \eqref{Durrett-domi} we would get
$$
\mathrm Q\left(\eta_T\geq \frac{1}{2}\mathrm Q(\eta_T)\right)\geq \widehat C,
$$
and therefore
\begin{eqnarray*}
\mathrm Q\left(\eta_\infty\geq \frac{1}{2}\mathrm Q(\eta_T)\right)\geq Q\left(\eta_T\geq \frac{1}{2}Q(\eta_T)\right)\ge\widehat C>0.
\end{eqnarray*}
Since $\lim_{T\rightarrow-\infty}\mathrm Q(\eta_T)=\mathrm Q(\eta_{-\infty})=\infty$ (see \eqref{mean=infty}), the above inequality implies
\eqref{positive-prob}.  Now we only need to prove
 \eqref{uniform lower bound}. For this purpose we first estimate
$\mathrm Q(\eta_T^2)$:
\begin{eqnarray*}
\mathrm Q(\eta_T^2)&=&\mathrm Q\int_T^0\mathrm dt\int_{K\phi(\xi_t)^{-1}e^{-\epsilon t}}^\infty r \pi(\xi_t, \mathrm dr)\int_T^0\mathrm ds\int_{K\phi(\xi_s)^{-1}e^{-\epsilon s}}^\infty u \pi(\xi_s, \mathrm du)\\
&=&2\mathrm Q\int_T^0\mathrm dt\int_{K\phi(\xi_t)^{-1} e^{-\epsilon t}}^\infty r \pi(\xi_t, \mathrm dr)\int_t^0\mathrm ds
\int_{K\phi(\xi_s)^{-1}e^{-\epsilon s}}^\infty u \pi(\xi_s, \mathrm du)\\
&=&2\mathrm Q\int_T^0\mathrm dt\int_{K\phi(\xi_t)^{-1}e^{-\epsilon t}}^\infty r \pi(\xi_t, \mathrm dr)
\int_t^{(t+c)\wedge 0}\mathrm ds\int_{K\phi(\xi_s)^{-1}e^{-\epsilon s}}^\infty u\, n(\xi_s, \mathrm du)\\
&&+2\mathrm Q\int_T^{0}\mathrm dt\int_{K \phi(\xi_t)^{-1}e^{-\epsilon t}}^\infty r \pi(\xi_t, \mathrm dr)
\int_{(t+c)\wedge 0}^0\mathrm ds\int_{K\phi(\xi_s)^{-1}e^{-\epsilon s}}^\infty u \pi(\xi_s, \mathrm du)\\
&=&III+IV,
\end{eqnarray*}
where
$$
III=\displaystyle 2\mathrm Q\int_T^0\mathrm dt\int_{K\phi(\xi_t)^{-1} e^{-\epsilon t}}^\infty r \pi(\xi_t, \mathrm dr)\int_t^{(t+c)\wedge 0}\mathrm ds\int_{K\phi(
\xi_s)^{-1}e^{-\epsilon s}}^\infty u \pi(\xi_s,\mathrm du)
$$
and
\begin{eqnarray*}
IV& =& 2\mathrm Q\int_T^0\mathrm dt \int_{K\phi(\xi_t)^{-1}e^{-\epsilon t}}^\infty r \pi(\xi_t, \mathrm dr)
\int_{(t+c)\wedge 0}^0\mathrm ds\int_{K\phi(\xi_s)^{-1}e^{-\epsilon s}}^\infty u \pi(\xi_s, \mathrm du)\\
&=& 2\int_T^0\mathrm dt\int_E\phi(y)\nu(\mathrm dy)\int_{K\phi(y)^{-1}e^{-\epsilon t}}^\infty r \pi(y, \mathrm dr)\times\\
&& \ \ \ \times \int_{(t+c)\wedge 0}^0\mathrm ds\widetilde\Pi_y\int_{K\phi(\xi_{s-t})^{-1}e^{-\epsilon s}}^\infty u \pi(\xi_{s-t},\mathrm du).
\end{eqnarray*}
By our assumption on the kernel $n$ we have that $ \|\int_1^\infty r\pi(\cdot,\mathrm dr)\|_{\infty}<\infty$. Since $K\inf_{x\in
E}\phi(x)^{-1}\ge 1$,  we have
$$
III\leq C_1 \mathrm Q(\eta_T),
$$
for some positive constant $ C_1$ which does not depend on $T$.
Using \eqref{domi-p} and the definition of $\pi^{\phi}$, we get from the fact $x\longmapsto\int_{K\phi(x)^{-1}e^{-\epsilon s}}^\infty u \pi(x,\mathrm du)\in\mathcal B_b(E,[0,\infty))$ that, for $t\in(T, 0)$,
$$
\begin{array}{rl}
&\displaystyle\int_{(t+c)\wedge 0}^0\mathrm ds\widetilde\Pi_y\int_{K\phi(\xi_{s-t})^{-1}e^{-\epsilon s}}^\infty u \pi(\xi_{s-t},\mathrm du)\\
\le&\displaystyle 2\int_{(t+c)\wedge 0}^0\mathrm ds\int_E\phi(z)\nu(\mathrm dz)\int_{K\phi(z)^{-1}e^{-\epsilon s}}^\infty u \pi(z, \mathrm du)\\
\le&\displaystyle 2\int_{T}^{0}\mathrm ds\int_E\phi(z)\nu(\mathrm dz)\int_{K\phi(z)^{-1}e^{-\epsilon s}}^\infty u \pi(z, \mathrm du)\\
=&\displaystyle 2\mathrm Q(\eta_T).
\end{array}
$$
Then we obtain
$$
IV\leq 4 (\mathrm Q(\eta_T))^2.
$$
Combining the estimates above on $III$ and $IV$, we get that there
exists a $C_2>0$ independent of $T$ such that for $T>c$,
$$
\mathrm Q(\eta_T^2)\le 4(\mathrm Q(\eta_T))^2+ C_1\mathrm Q(\eta_T)\le
C_2(\mathrm Q(\eta_T))^2.
$$
Then we have \eqref{uniform lower bound} with $\widehat C=1/C_2$.
The proof is complete. 
\end{proof}




\begin{thebibliography}{99}
	
\bibitem{AH}Asmussens, S. and Hering, H. :\emph{Branching Processes}. Birkhauser, Boston, 1983.

\bibitem{AthreyaNey1972Branching}
Athreya, K. B. and Ney, P. E.:
\emph{Branching processes.}
Die Grundlehren der mathematischen Wissenschaften, Band 196. Springer-Verlag, New York-Heidelberg, 1972. xi+287 pp.
\MR{0373040}

\bibitem{BigginsKyprianou2004Measure}
Biggins, J. D. and Kyprianou, A. E.:
\emph{Measure change in multitype branching.}
Adv. in Appl. Probab. \textbf{36} (2004), no. 2, 544--581.
\MR{2058149}

\bibitem{ChampagnatRoelly2008Limit}
Champagnat, N. and Roelly, S.:
\emph{Limit theorems for conditioned multitype Dawson-Watanabe processes and Feller diffusions.}
Electron. J. Probab. \textbf{13} (2008), no. 25, 777C810.
\MR{2399296}

\bibitem{ChampagnatVillemonais2018Convergence}
Champagnat, N. and Villemonais, D.:
\emph{Convergence of the Fleming-Viot process toward
theminimal quasi-stationary distribution.}
https://arxiv.org/pdf/1810.06849.pdf

\bibitem{ChenRenYang2017Skeleton}
Chen, Z.-Q., Ren, Y.-X. and Yang, T.:
\emph{Skeleton decomposition and law of large numbers for supercritical superprocesses.}
Acta Appl. Math. 159(1)(2019) 225-285

\bibitem{Dawson1992Infinitely}
Dawson, D. A.:
\emph{Infinitely divisible random measures and superprocesses.}Stochastic analysis and related topics (Silivri, 1990), 1--129,
Progr. Probab., 31, Birkh{\"a}user Boston, Boston, MA, 1992.
\MR{1203373}

\bibitem{DelmasHenard2013A-Williams}
Delmas, J.-F. and H\'enard, O.:
\emph{A Williams decomposition for spatially dependent super-processes. }
Electron. J. Probab. \textbf{18} (2013), no. 37, 43 pp.
\MR{3035765}

\bibitem{Dudley2002Real}
Dudley, R. M.:
\emph{Real analysis and probability.}
Revised reprint of the 1989 original. Cambridge Studies in Advanced Mathematics, 74. Cambridge University Press, Cambridge, 2002. x+555 pp.

\bibitem{Dynkin1993Superprocesses}
Dynkin, E. B.:
\emph{Superprocesses and partial differential equations.}
Ann. Probab. \textbf{21} (1993), no. 3, 1185--1262.
\MR{1235414}

\bibitem{EnglanderKyprianou2004Local}
Engl\"ander, J. and Kyprianou, A. E.:
\emph{Local extinction versus local exponential growth for spatial branching processes.}
Ann. Probab. \textbf{32} (2004), no. 1A, 78--99.
\MR{2040776}

\bibitem{Evans1993Two}
Evans, S. N.:
\emph{Two representations of a conditioned superprocess.}
Proc. Roy. Soc. Edinburgh Sect. A \textbf{123} (1993), no. 5, 959--971.
\MR{1249698}

%new
\bibitem{Fitz}
Fitzsimmons, P. J.: 
\emph{Construction and regularity of measure-valued Markov branching processes.} 
Israel J. Math. \textbf{64} (1988), no. 3, 337–-361.
%end new

\bibitem{Grey1974Asymptotic}
Grey, D. R.:
\emph{Asymptotic behaviour of continuous time, continuous state-space branching processes.}
J. Appl. Probability \textbf{11} (1974), 669--677.
\MR{0408016}

\bibitem{Heathcote}
Heathcote, R.,  Seneta, E.  and Vere-Jones, D.:
\emph{ A refinement of two theorems in the theory of branching processes}.
Theory Probab. Appl. 12 (1982), 297-301.

\bibitem{HeathcoteSenetaVere-Jones1967A-refinement}
Heathcote, C. R., Seneta, E. and Vere-Jones, D.:
\emph{A refinement of two theorems in the theory of branching processes.} (Russian summary)
Teor. Verojatnost. i Primenen. \textbf{12} 1967 341--346.
\MR{0217889}

\bibitem{Joffe}Joffe, A. and Waughw, A. O.:  Exact distributions of kin numbers in a Galton-Watson
process. J. Appl. Probab. 19 (1982), 767-775.

\bibitem{Kallenberg2002Foundations}	
	Kallenberg, O.:
	\emph{Foundations of Modern probability.}
	Second Edition.	Springer-Verlag New York, 2002.

\bibitem{KimSong2008Intrinsic}
Kim, P., Song, R.:
\emph{Intrinsic ultracontractivity of non-symmetric diffusion semigroups in bounded domains.}
Tohoku Math. J. (2) 60 (2008), no. 4, 527-547.
\MR{2487824}

\bibitem{KimSong2008Intrinsic2}
Kim, P. and Song, R.:
\emph{Intrinsic ultracontractivity of nonsymmetric diffusions with measure-valued drifts and potentials.}
Ann. Probab. \textbf{36} (2008), no. 5, 1904--1945.
\MR{2440927}

\bibitem{KimSong2009Intrinsic}
Kim, P. and Song, R.:
\emph{Intrinsic ultracontractivity for non-symmetric L\'evy processes.}
Forum Math. \textbf{21} (2009), no. 1, 43C66.
\MR{2494884}

\bibitem{Lambert2001Arbres}
Lambert, A.:
\emph{Arbres, excursions et processus de L\'evy completement asym\'etriques.}
Diss. Universit Pierre et Marie Curie-Paris VI, 2001.

\bibitem{Lambert2003Coalescence}
Lambert, A.:
\emph{Coalescence times for the branching process.}
Adv. in Appl. Probab. \textbf{35} (2003), no. 4, 1071--1089.
\MR{2014270}

\bibitem{Lambert2007Quasi-stationary}
Lambert, A.:
\emph{Quasi-stationary distributions and the continuous-state branching process conditioned to be never extinct.}
Electron. J. Probab. \textbf{12} (2007), no. 14, 420--446.
\MR{2299923}

\bibitem{Li00}
Li, Z.-H.:
\emph{Asymptotic behaviour of continuous time and state branching processes.}
J. Austral. Math. Soc. Ser. A \textbf{68} (2000), no. 1, 68--84.
\MR{1727226}

\bibitem{Li2011Measure-valued}
Li, Z.:
\emph{Measure-valued branching Markov processes.}
Probability and its Applications (New York). Springer, Heidelberg, 2011. xii+350 pp. ISBN: 978-3-642-15003-6
\MR{2760602}

\bibitem{LiuRenSong2009Llog}
Liu, R.-L., Ren, Y.-X. and Song, R.:
\emph{{$L \log L$} criterion for a class of superdiffusions.}
J. Appl. Probab. \textbf{46} (2009), no. 2, 479--496.
\MR{2535827}

\bibitem{LiuRenSongSun2020}
Liu, R.-L., Ren, Y.-X., Song, R. and Sun, Z.
\emph{Quasi-stationary distributions for subcritical superprocesses.}
Preprint.
ARXIV{2001.06697}

\bibitem{LyonsPemantlePeres1995Conceptual}
Lyons, R., Pemantle, R. and Peres, Y.:
\emph{Conceptual proofs of $L\log L$ criteria for mean behavior of branching processes.}
Ann. Probab. \textbf{23} (1995), no. 3, 1125--1138.
\MR{1349164}

\bibitem{MeleardVillemonais2012Quasi-stationary}
M\'el\'eard, S. and Villemonais, D.:
\emph{Quasi-stationary distributions and population processes.}
Probab. Surv. \textbf{9} (2012), 340C410.
\MR{2994898}

\bibitem{Nagasawa1964Time}
Nagasawa, M.:
\emph{Time reversions of Markov processes.}
Nagoya Math. J. \textbf{24} (1964), 177--204.
\MR{0169290}

\bibitem{Penisson2010Conditional}
P\'enisson, S.:
\emph{Conditional limit theorems for multitype branching processes and illustration in epidemiological risk analysis.}Diss. Universitt Potsdam, Universit Paris Sud-Paris XI, 2010.

\bibitem{RenSongSun2020Spine}
	Ren, Y.-X., Song, R., and Sun, Z.:
	\emph{Spine decompositions and limit theorems for a class of critical superprocesses.}
	Acta Appl. Math. \textbf{165} (2020): 91--131.

\bibitem{RenSongZhang2015Limit}
Ren, Y.-X., Song, R., and Zhang, R.:
\emph{Limit theorems for some critical superprocesses.}
Illinois J. Math. 59 (2015), no. 1, 235-276.

\bibitem{RenSongZhang2017Central}
Ren, Y.-X., Song, R., and Zhang, R.:
\emph{Central limit theorems for supercritical branching nonsymmetric Markov processes.}
Ann. Probab. 45 (2017), no. 1, 564-623.

\bibitem{RenSongZhang2018Williams}
Ren, Y.-X., Song, R. and Zhang, R.:
\emph{Williams decomposition for superprocesses.}
Electron. J. Probab. \textbf{23} (2018), Paper No. 23, 33 pp.
\MR{3771760}

\bibitem{RenSongYang2016Spine}
Ren, Y.-X., Song, R. and Yang, T.:
\emph{Spine decomposition and {$ L\log L $} criterion for superprocesses with non-local branching mechanisms.}
Preprint.
ARXIV{1609.02257}

\bibitem{RoellyRouault1989Processus}
Roelly, S. and Rouault, A.:
\emph{Processus de Dawson-Watanabe conditionn\'e par le futur lointain.} (French. English summary) [A Dawson-Watanabe process conditioned by the remote future]
C. R. Acad. Sci. Paris Sr. I Math. \textbf{309} (1989), no. 14, 867--872.
\MR{1055211}

\bibitem{Schaefer1974Banach}
Schaefer, H. H.:
\emph{Banach lattices and positive operators.}
Die Grundlehren der mathematischen Wissenschaften, Band 215. Springer-Verlag, New York-Heidelberg, 1974. xi+376 pp. \MR{0423039}

\bibitem{Yaglom1947}
Yaglom, A. M.:
\emph{Certain limit theorems of the theory of branching random processes.} (Russian)
Doklady Akad. Nauk SSSR (N.S.) \textbf{56} (1947), 795--798.
\MR{0022045}

\end{thebibliography}
\end{document}
