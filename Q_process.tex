\documentclass[12pt,a4paper]{amsart}
\setlength{\textwidth}{\paperwidth}
\addtolength{\textwidth}{-2in}
\calclayout
\numberwithin{equation}{section}
\allowdisplaybreaks
\theoremstyle{plain}
\newtheorem{thm}{Theorem}[section]
\newtheorem{assum}{Assumption}[section]
\newtheorem{lem}[thm]{Lemma}
\newtheorem{prop}[thm]{Proposition}
\newtheorem{cor}[thm]{Corollaray}
\newtheorem{fact}[thm]{Fact}
\newtheorem{remark}{Remark}
\newtheorem{claim}[thm]{Claim}
\theoremstyle{definition}
\newtheorem*{ack*}{Acknowledgment}
\theoremstyle{remark}
\newtheorem{exa}[thm]{Example}
\usepackage{amssymb}
\usepackage{mathtools}
\mathtoolsset{showonlyrefs}
\usepackage{mathrsfs}
\usepackage{comment}
\everymath{\displaystyle}
\usepackage{hyperref}
\usepackage[inline]{showlabels}
\begin{document}
%\title {Subcritical Superprocesses }
%\title {The $Q$ processes of Subcritical Superprocesses and their limit properties }
\title{TBD} %
\author[R. Liu, Y.-X. Ren, R. Song, and Z. Sun]{Rongli Liu, Yan-Xia Ren, Renming Song, and Zhenyao Sun}
\address{Yan-Xia Ren\\ LMAM School of Mathematical Sciences \& Center for
Statistical Science\\ Peking University\\ Beijing 100871\\ P. R. China}
\email{yxren@math.pku.edu.cn}
\thanks{The research of Yan-Xia Ren is supported in part by NSFC (Grant Nos. 11671017 and 11731009)  and LMEQF.}
\address{Rongli Liu\\ Mathematics and Applied Mathematics\\ Beijing jiaotong University\\ Beijing 100044\\ P. R. China}
\email{rlliu@bjtu.edu.cn}
\thanks{The research of Rongli Liu is supported in part by NSFC (Grant No. 11301261), and the Fundamental Research Funds for the Central Universities (Grant No.  2017RC007)}
\address{Renming Song\\ Department of Mathematics\\ University of Illinois at Urbana-Champaign \\ Urbana \\ IL 61801\\ USA}
\email{rsong@illinois.edu}
\address{Zhenyao Sun\\ Faculty of Industrial Engineering and Management \\ Technion, Isreal Institute of Technology \\ Haifa 3200003\\ Isreal}
\email{zhenyao.sun@gmail.com}
\begin{abstract}
\begin{comment}
Suppose that $E$ is a locally compact separable metric space, and  that  $X =\{(X_t)_{t\geq 0}; (\mathbb P_\mu)_{\mu \in \mathcal M_f(E)}\}$  is a subcritical superprocess, where $\mathcal M_f(E)$ is the space of all finite Borel measures on $E$. Under some conditions on the mean semigroup of $X$, we prove that the the $Q$ process of $X$ exists, and they have equilibrium probability if and only if the moment condition $\int_El(x)\nu(dx)<\infty$ is satisfied.
We also show that the equilibrium probability is a size-biased measure of the Yaglom distribution. 
\end{comment}
	TBD %
\end{abstract}
\maketitle
\section{Introduction}
\subsection{Background}
\begin{comment}
The study of the extinction of populations is of a great interest in biology. Conditioning on non-extinction can not only notably lead to a stationary behavior
of the process, but also provides a lot of information about the evolution of the population before extinction.  As far as the population dynamics which are extinct in finite time running over large amounts of time are concerned, special attentions are given to two conditional limits: one type of limits are the Yaglom distributions and the related quasi-stationary distributions, the other type of limits are the Q processes.   There are plenty of literatures investigating these properties of branching processes. See, for instance, Athreya and Ney \cite[pages 64-65]{AthreyaNey1972Branching},Grey\cite{Grey1974Asymptotic},Lyons, Pemantle and Peres \cite{LyonsPemantlePeres1995Conceptual}, Seneta and
Vere-Jones \cite{Heathcote},Heathcote,Joffe \cite{Joffe}, Yaglom \cite{Yaglom47} for Galton-Watson branching processes, Lambert\cite{Lambert2001Arbres, Lambert2003Coalescence},Li\cite[Theorem 4.3]{Li00} for continuous state and continuous time branching processes. Asmussen and Hering \cite{AH} studied limit behaviour of subcritical branching Markov processes, in which each particle  lives for exponential time, then give birth to random number of particles, and particles move as independent
Markov processes in between branching times and it is assume that  the life times, reproduction of different individuals is independent.

  In the authors' paper \cite{LSYS}, the Yaglom distributions of the superprocesses are investigated.  In
  this paper, we are mainly studying the properties of the Q processes for superprocesses and discover their relationships with the Yaglom distributions..
\end{comment}
TBD %
\subsection{Superprocess}\label{model}
	We first recall some basics about superprocesses. %
	% Suppose that $E$ is a locally compact separable metric space.
	Let $E$ be a Polish space. %
	% Suppose that $\partial$ is a separate point not contained in $E$.
	% We will use  $E_\partial$ to denote $E\cup\{\partial\} $.
	Let $\partial$ be an isolated point not contained in $E$ and $E_\partial := E \cup \{\partial\}$. %
	% We assume that  $\xi= \{(\xi_t)_{t\ge0}; (\Pi_x)_{x\in E}\}$ is a Hunt process on $E$ and $\tau:=\inf\{t>0: \xi_t=\partial\}$ is the lifetime of $\xi$.
	We assume that  $\xi= \{(\xi_t)_{t\ge0}; (\Pi_x)_{x\in E}\}$ is an $E_\partial$-valued Borel right process with $\partial$ as an absorbing state. Denote by $\tau:=\inf\{t>0: \xi_t=\partial\}$ the lifetime of $\xi$. %
	Let $\psi$ be a function on $E \times [0,\infty)$ given by
\begin{align}
	\psi(x,z)
	= -\beta(x) z + \sigma(x)^2 z^2 + \int_{(0,\infty)} (e^{-zu} -1 + zu) \pi(x,du),
	\quad x\in E, z\geq 0
\end{align}
	where $\beta, \sigma \in \mathcal B_b(E)$ and $(u \wedge u^2) \pi(x,du)$ is a bounded kernel from $E$ to $(0,\infty)$.
	Let $\mathcal M_f(E)$ denote the space of all finite Borel measures on $E$ equipped with topology of weak convergence.
	Denote by $\mathcal B(M_f(E))$ the Borel $\sigma$-field generated by this topology. For any $\mu \in \mathcal M_f(E)$ and $f\in \mathcal B(E,[0,\infty))$, we use $\mu(f)$ to denote the integration of $f$ with respect to $\mu$ whenever the integration is well-defined. %
	% For any $f \in \mathcal B_b^+(E)$, there is a unique locally bounded positive solution $(t,x)\mapsto V_tf(x)$ to the equation
	For any $f \in \mathcal B_b(E,[0,\infty))$, there is a unique locally bounded non-negative map $(t,x)\mapsto V_tf(x)$ on $[0,\infty)\times E$ such that %
\begin{equation} \label{eq: 1}
	% V_tf(x) + \int_0^t P_{s} \psi(\cdot, V_{t-s}f(\cdot)) (x)~ds = P_tf(x), \quad t\geq 0, x\in E,
	V_tf(x) + \Pi_x\Big[\int_0^{t\wedge \zeta} \psi(\xi_s, V_{t-s} f(\xi_s)) ds\Big] = \Pi_x[f(\xi_t) \mathbf 1_{t< \zeta}], \quad t\geq 0, x\in E. %
\end{equation}
	% where $(P_t)$ is the semigroup of $\xi$.  
	Here, the local boundedness of the map $(t,x) \mapsto V_tf(x)$ means that \[ \sup_{0\leq t\leq T, x\in E} V_tf(x) < \infty, \quad T>0.\] %
	% Then for any $\mu\in \mathcal M_f(E)\backslash\{0\}$, there exists an $\mathcal M_f(E)$-valued Hunt process $X =\{(X_t)_{t\geq 0}; (\mathbb P_\mu)_{\mu \in \mathcal M_f(E)}\}$ such that
	Then, there exists an $\mathcal M_f(E)$-valued Borel right process $X =\{(X_t)_{t\geq 0}; (\mathbb P_\mu)_{\mu \in \mathcal M_f(E)}\}$ such that
\begin{equation}
	\mathbb P_\mu[e^{- X_t(f)}]
	= e^{- \mu(V_tf)},
	% \quad t\geq 0,~ f \in \mathcal B_b^+(E).
	\quad t\geq 0, f \in \mathcal B_b(E,[0,\infty)).
\end{equation}
	We call $X$ a $(\xi, \psi)$-superprocess.  See \cite{Li2011MeasureValued} for more details.

	% Let $(P_t^\beta)_{t\geq 0}$ be the semigroup of operators on $\mathcal B_b(E)$ given by
	Let $(P_t^\beta)_{t\geq 0}$ be given by
\begin{align}
	P_t^\beta f(x)
	:= \Pi_x\Big[e^{\int_0^t \beta(\xi_r)dr }f(\xi_t) \mathbf 1_{\{t < \tau\}}\Big],
	\quad f\in \mathcal B_b(E), t\geq 0, x\in E.
\end{align}
	It is well-known (see \cite[Proposition 2.27]{Li2011MeasureValued}) that %it is the expectation semigroup of $X$, that is,
\begin{equation} \label{exp}
	\mathbb P_\mu[X_t(f)] = \mu (P_t^\beta f),
	\quad t\geq 0, f \in \mathcal B_b(E).
\end{equation}

% Suppose $\mu$ is some measure on $(E,\mathcal B(E))$.  Denote by $L_1^+(\mu)$ the collection of non-negative functions on $E$ which are integrable with respect to the measure $\mu$. 
	For any finite Borel measure $\mu$ on $E$, denote by $L_1^+(\mu)$ the collection of non-negative functions on $E$ which are integrable with respect to the measure $\mu$. %
In this paper, we will always assume $X$ satisfies the following assumptions.
\begin{assum}\label{assum}
\begin{itemize}
\item[(1)] there exist a $\lambda<0$, a positive bounded continuous function $\phi$ on $E$ and a probability measure $\nu(dx)=\hat\phi(x)m(dx)$ with full support on $E$ such that for each $t\geq 0$, $P_t^\beta \phi = e^{\lambda t}\phi$, $\nu P_t^\beta = e^{\lambda t} \nu$ and $\nu(\phi) = 1$.
\item[(2)] $P_t^\beta$ has density $p^\beta(t, x, y)$, and for any $t>0$, there exists
     a constant $c_t>0$ such that
    $$
    p^\beta(t, x, y)\le c_t\phi(x)\hat{\phi}(y), \quad \mbox{ for all }  (x, y)\in E\times E.
    $$
\item[(3)]There exists a $T\geq 0$ such that $\mathbb P_\nu(\|X_t\| = 0)>0$ for all $t> T$.
\end{itemize}
{\tt Let us try to work without the measure $m$.}
\end{assum}
Let  $\zeta=\inf\{t>0: \langle \mathbf 1_E,X_t\rangle=0\}$ be the extinction time of the superprocess $X$.  Assumption $(1)$ can assure from \eqref{exp} that
\[
\lim_{t\to\infty}\mathbb P_\mu[X_t(f)]=0, \qquad f \in \mathcal B_b^+(E).
\]
But it does not say
\begin{equation}\label{p1extinc}
	\mathbb P_\mu(\zeta<\infty)=1.
\end{equation}
The extinction probability also depend on the nonlinear part of the branching mechanism $\psi$. If the branching mechanism $\psi$ is independent of the location $x$, then \eqref{p1extinc} holds provided the following condition  is true.
\begin{equation}\label{extinc assump  for continuous}
	\int^\infty\frac{1}{\psi(\lambda)}d\lambda<\infty.
\end{equation}
When the branching mechanism depends on the location, we need add the assumption $(3)$ which can assure \eqref{p1extinc}. 
{\tt Z: I'm not sure this part is nessary.}

It is well known that if we define
\[
	v(t,x):= -\log \mathbb P_x(\zeta \leq t), \quad t > 0, x\in E,
\]
then we have
\[
	v(t,x)
	= \lim_{\theta \to \infty} V_t(\theta \mathbf 1_E)(x),
	\quad t>0, x\in E,
\]
	and
\begin{equation}
\label{eq: v and extinction}
	e^{-\langle v(t,\cdot), \mu \rangle}
	= \mathbb P_\mu(\zeta \leq t), \quad t>0, \mu \in \mathcal M_f(E).
\end{equation}
We will discuss the properties of $v(t,x)$ to get the extinction properties of $X$.
{\tt Z: I'm not sure this part is nessary.}

We say a probability ${\mathbf P}$ on $\mathcal M_f(E)$ is the \emph{Yaglom distribution} of the superprocess $X$ if for any $\mu\in \mathcal M_f(E)\setminus\{0\}$ we have
\[
  \mathbb P_\mu(X_t \in \cdot | \zeta > t) \xrightarrow[t\to \infty]{w} {\mathbf P}(\cdot).
\]
For any probability measure $\mathbf P$ on $\mathcal M_f(E)$, define $(\mathbf P\mathbb P)[\cdot] := \int_{\mathcal M_f(E)} \mathbb P_\mu[\cdot] \mathbf P(d\mu)$.  We say a probability measure $\mathbf Q$ on $\mathcal M_f(E)$ is a quasi-stationary distribution (QSD) of $X$, if
	\[
	(\mathbf Q \mathbb P) \left( X_t \in B \middle | \zeta>t \right) = \mathbf Q(B), \quad t\geq 0, B \in \mathcal B(\mathcal M_f(E)).
	\]
If the Yaglom distribution ${\mathbf P}$ of the superprocess $X$ exists, then it must be a QSD of $X$ (c.f. \cite{MeleardVillemonais2012Quasi-stationary}).
In \cite{LSRS}, we develop the properties of QSD as follows.
\begin{thm} \label{Yaglom}
	Suppose that $(X_t)$ satisfies assumption \ref{assum}.	Then there exists a probability measure $\mathbf Q_\lambda$ on $\mathcal M_f(E)$ such that
\begin{align}
 	\mathbb P_\mu \left(X_t \in \cdot \middle| \|X_t\| > 0 \right)
 	\xrightarrow[t\to \infty]{d} \mathbf Q_\lambda(\cdot),
 	\quad \forall \mu \in \mathcal M_f(E)\setminus \{0\}.
\end{align}
for each $r \in [\lambda, 0)$, there exists a unique QSD $\mathbf Q_r$ of $X$ with mass decay rate $r$. If $G$ is the log-Laplace functional of $\mathbf Q_\lambda$, then it satisfies
\begin{equation}\label{ll}
1-e^{-G(V_tf)}=e^{\lambda t}(1-e^{-G(f)}), \qquad f\in\mathcal B_b^+(E).
\end{equation}
And $1-\left(1-e^{-G(f)}\right)^\alpha$ is the Laplace functional of $\mathbf Q_r$, where $\alpha=r/\lambda$.  For any $u\in [0,1]$,
\begin{equation}\label{tail of L}
\lim_{t\to\infty}e^{-\lambda t}\left(1-e^{-G(uv(t))}\right)=u.
\end{equation}
	{\tt Z: Do we really want to introduce those result here?}
\end{thm}
In the next section, we start to investigate the properties of the superprocesses conditioned on non extinction called $Q$ processes.
\subsection{Main Results}
\hspace*{10pt}\\
Define a new kernel $n^\phi(x, dr)$ from $E$ to $(0,\infty)$ such that for any $f\in\mathcal B_b^+((0,\infty))$,
\begin{equation} \label{phi-change}
	\int_0^\infty f(r)n^\phi(x,dr)=\int_0^\infty f(r\phi(x))n(x, dr),
	\quad x\in E.
\end{equation}
	Our first result is about the asymptotic behavior of the extinction time $\zeta$, including proving \eqref{p1extinc} from assumption $(3)$. 
	{\tt delete this line.}

\begin{thm}\label{thm: distribution of zeta}
	Suppose that the superprocess $X$ satisfies Assumptions \ref{assum}. Then,
\begin{enumerate}
\item \label{subthm: extinct as sure}
	for each $\mu \in \mathcal M_f(E)$, we have  $\mathbb P_\mu(\zeta<\infty)=1$;
\item
 	for each $\mu,\tilde\mu\in \mathcal M_f(E)\setminus\{0\}$ and $s>0$, we have
 \[
 	\lim_{t\rightarrow\infty}\dfrac{\mathbb P_{\mu}(\zeta>t+s)}{\mathbb P_{\tilde\mu}(\zeta>t)}=\frac{\langle \phi,\mu\rangle }{\langle \phi,\tilde\mu\rangle }e^{\lambda s};
 \]
 \item
 	there exists a constant $k\in [0,\infty)$, such that for any $x\in E$,
\begin{equation}\label{decay rate}
	\lim_{t\rightarrow\infty} e^{-\lambda t}\mathbb P_x(\zeta>t)=k\phi(x).
\end{equation}
	Moreover, the constant $k>0$ if and only if $\int_E l(y)\nu(dy)<\infty$ where
\begin{equation}\label{def: m}
	l(y):=\int_1^\infty r\log r~n^\phi(y, dr),\quad y \in E.
\end{equation}
\end{enumerate}
\end{thm}

	{\tt Z: Only the point three is interesting.}
In particular, for any $x,y\in E$ and $s\geq 0$, the second result in the theorem above is written as
\begin{equation}\label{ratioresult}
 	\lim_{t\rightarrow\infty}\frac{\mathbb P_x(\zeta>t+s)}{\mathbb P_y(\zeta>t)}=\frac{\phi(x)}{\phi(y)}e^{\lambda s}.
\end{equation}


Since $\phi$ is an eigenfunction of $P_t^\beta$ that
\begin{equation}
	\mathbb P_\mu[X_t(\phi)]= \mu(P_t^\beta \phi)=e^{\lambda t}\mu(\phi).
\end{equation}
Define the process
\[
	M_t(\phi)=e^{-\lambda t}  \langle \phi, X_t\rangle, \quad t\geq 0, \mu\in\mathcal M_f(E).
\]
Then the process  $(M_t(\phi))_{t\geq 0}$ is a nonnegative martingale with respect to the natural filtration $(\mathscr F_t)_{t\geq 0}$ of the superprocess $X$.
For each $\mu \in \mathcal M_f(E)\backslash\{0\}$, let probability $\widetilde{\mathbb P}_\mu$ be the Doob's $h-$transform of $\mathbb P_\mu$ with respect to $M_t(\phi)$, such that
\begin{equation} \label{eq: martingale transformation}
	\frac{d\widetilde{\mathbb P}_\mu|_{\mathscr F_t}}{d\mathbb P_\mu|_{\mathscr F_t}}
	=\frac{M_t(\phi)}{\langle\phi,\mu\rangle },
	\quad t\geq 0.
\end{equation}
	This kind of martingale measure transformation for branching processes and measure-valued processes have been widely studied.
	We refer to the early papers \cite{EnglanderKyprianou2004Local,Evans1993Two,RoellyRouault1989Processus}, the thesis \cite{Penisson2010Conditional} and the references therein, and the recent papers \cite{ChampagnatRoelly2008Limit,RenSongSun2017Spine,RenSongZhang2018Williams}.
	It is well known that the process $\{(X_t)_{t\geq 0}; \widetilde{\mathbb P}_{\mu}\}$ can be characterized by the so called spine decomposition theorem.
	We will recall this decomposition in details for our model in section $2$.


Our second theorem says that $\{(X_t)_{t\geq 0}; \widetilde{\mathbb P}_{\mu}\}$ can be considered as the Q-processs of $X$, i.e. the process $\{(X_t)_{t\geq 0}; \mathbb P_{\mu}\}$ conditioned to be never extinct:

\begin{thm}\label{thm: Qprocess}
	Under the assumptions \ref{assum}, for each $\mu \in \mathcal M_f(E)\backslash\{0\}, t\geq 0$ and $A\in\mathscr F_t$, we have
$
	\lim_{s\rightarrow\infty}\mathbb P_\mu(A |\zeta>s)=\widetilde{\mathbb P}_\mu(A).
$
\end{thm}

It would be interesting to study the stationary states of this Q-process.  Same to the definition of $\mathbf Q\mathbb P$, for each probability $\mathbf Q$ on $\mathcal M_f(E)$, we define $\mathbf Q\widetilde{\mathbb P}$.  Then $\{(X_t)_{t\geq 0}; (\mathbf Q\widetilde{\mathbb P})\}$ can be considered as the Q-process with a random initial value $X_0$ whose distribution is $\mathbf Q$.  A probability $\mathbf Q$ on $\mathcal M_f(E)$ is an \emph{equilibrium probability} of the Q-process $\{(X_t)_{t\geq 0}; (\widetilde{\mathbb P}_\mu)_{\mu\in\mathcal M_f(E)}\}$ if
\[
	(\mathbf Q\widetilde{\mathbb P})(X_t \in \cdot ) =\mathbf Q(\cdot),	\quad t\geq 0.
\]
Our third theorem says the properties of the equilibrium probability $\mathbf Q$ of the Q-process.  Before stating the main result, we give a moment property of the QSD's.
\begin{prop}\label{prop: exp prop}
	Suppose that the superprocess $X$ satisfies Assumptions \ref{assum}.  Then
\begin{enumerate}
\item  For any $0<\gamma<r/\lambda$,
	$\mathbf Q_r[\langle\phi,\omega\rangle^\gamma]<\infty$ for each $r \in [\lambda, 0)$.
\item
	$\mathbf Q_\lambda[\langle\phi,\omega\rangle] < \infty$ if and only if $\int_E \widehat\phi(y)l(y)m(dy)<\infty$.
\end{enumerate}
	Moreover, if $\int_E \widehat\phi(y)l(y)m(dy)<\infty$ then the constant $k$ in \eqref{decay rate} is equal to $\mathbf Q_\lambda[\langle\phi,\omega\rangle]^{-1}$.
\end{prop}

{\tt This result should be considered as the LlogL result. And should be put above.}

\begin{thm}\label{thm: structure of Qprocess}
	Suppose Assumption \ref{assum} holds. $l(x), x\in E$ is the function defined in \eqref{def: m}.
\begin{enumerate}
\item
	If $\int_E l(x)\nu(dx)<\infty$, then $\{(X_t)_{t\geq 0};(\widetilde{\mathbb P}_\mu)_{\mu\in\mathcal M_f(E)\backslash\{0\}}\}$ has equilibrium probability ${\mathbf Q}$. In other words, for any $\mu\in\mathcal M_f(E)\backslash\{0\}$, we have
\[
	\widetilde{\mathbb P}_\mu(X_t \in \cdot ) \xrightarrow[t\to \infty]{w} {\mathbf Q}(\cdot).
\]
The equilibrium probability $\mathbf Q$ is the size-biased distribution of the quasi-stationary distribution $\mathbf Q_\lambda$. That is for any $f\in\mathcal B_b^+(E)$,
\[
  	\mathbf Q\left[e^{-\langle f,\omega\rangle}\right] = \dfrac{\mathbf Q_\lambda\left[\omega(\phi)e^{-\langle f,\omega\rangle}\right]}{\mathbf Q_\lambda[\omega(\phi)]}.
\]
\item
	If $\int_El(x)\nu(dx)=\infty$, then for each $\mu \in \mathcal M_f(E)\backslash\{0\}$, we have $\lim_{t\rightarrow\infty}\langle \phi, X_t\rangle =\infty$ in probability with respect to $\widetilde{\mathbb P}_\mu$. {\tt It is interesting to ask then what kind of convergence is this?}
\end{enumerate}
\end{thm}
%One important technique used to prove the above theorems is the ``spine-decomposition" for the superprocess $X$ under a martingale change of measure. This decomposition was used by Englander and Kyprianou in \cite{EnglanderKyprianou2004Local} to investigate the local extinction of super-diffusions, in which the branching mechanism is $\psi(x,z)=\alpha(x)^2z^2-\beta(x)z$. This technique is usually used to investigate the properties of supercritical superdiffusions ($\lambda>0$). Here we use it to analyze the case of $\lambda<0$.


\section{Preliminaries}
\subsection{Spine process and its time reverse}
\hspace*{10pt}\\
	Let $\{(\xi_t)_{t\geq 0}; (\Pi_x)_{x\in E}\}$ be the spatial motion introduced in Section $1$ satisfying assumptions $(2)$ and $(3)$.
For each $x\in E$, let the probability $\widetilde \Pi_{x}$ be Doob's $h$-transform of $\Pi_x$ such that
\begin{align}
	\dfrac{d\widetilde{\Pi}_x|_{\mathscr F^{\xi}_t}}{d\Pi_x|_{\mathscr F^{\xi}_t}}= \frac{e^{\int_0^t \beta(\xi_s)ds}\phi(\xi_t) \mathbf 1_{\{t<\tau\}}}{e^{\lambda t}\phi(x)},
	\quad t\geq 0,
\end{align}
	where $(\mathscr F_t^{\xi})_{t\geq 0}$ is the natural filtration of process $(\xi_t)_{t\geq 0}$. 	For each $\mu \in \mathcal M_f(E)\backslash\{0\}$, define
\[
	\Pi_{\mu}(\cdot)
	:= \mu(E)^{-1}\int_{E} \Pi_x(\cdot)\mu(dx),
\]
    and
\[
	\widetilde\Pi_{\mu}(\cdot):= \mu(E)^{-1} \int_E\widetilde\Pi_x(\cdot)\mu(dx).
\]
	For each function $f \in \mathcal B_b^+(E)$ and measure $\mu \in \mathcal M_f(E)$, define measure $(\phi \cdot\mu)$ such that
\[
    (\phi \cdot \mu)(dx)
    := \phi(x)\mu(dx),
    \quad x\in E.
\]
\begin{lem}
	For each $\mu\in \mathcal M_f(E)\backslash\{0\}$, we have
\[
	\dfrac{\widetilde \Pi_{\phi\cdot\mu}|_{\mathscr F_t^{\xi}}}{\Pi_{\mu}|_{\mathscr F_t^{\xi}}}
  	:= \frac{e^{\int_0^t \beta(\xi_s)ds}\phi(\xi_t) \mathbf 1_{\{t<\tau\}}}{\mu(E)^{-1}e^{\lambda t}\langle \phi,\mu\rangle},
  	\quad t\geq 0.
\]
\end{lem}
\begin{proof}
	Fix an arbitrary time $t\geq 0$. Fix an arbitrary event $A \in \mathscr F_t^{\xi}$.
	Then we have
\begin{align}
	&\widetilde{\Pi}_{\phi\cdot\mu}(A)
	= \langle\phi, \mu\rangle^{-1} \int_E \widetilde \Pi_x(A)\phi(x)\mu(dx)
	\\&=\langle\phi, \mu\rangle^{-1} \int_E  \Pi_x\Big[\frac{e^{\int_0^t \beta(\xi_s)ds}\phi(\xi_t) \mathbf 1_{\{t<\tau\}}}{e^{\lambda t}\phi(x)} \mathbf 1_A\Big]\phi(x)\mu(dx)
	\\&= \mu(E)^{-1}\int_E  \Pi_x\Big[\frac{e^{\int_0^t \beta(\xi_s)ds}\phi(\xi_t) \mathbf 1_{\{t<\tau\}}}{\mu(E)^{-1}e^{\lambda t}\langle \phi,\mu\rangle} \mathbf 1_A\Big]\mu(dx)
	\\&= \Pi_{\mu}\Big[\frac{e^{\int_0^t \beta(\xi_s)ds}\phi(\xi_t) \mathbf 1_{\{t<\tau\}}}{\mu(E)^{-1} e^{\lambda t}\langle \phi,\mu\rangle} \mathbf 1_A\Big].
	\qedhere
\end{align}
\end{proof}
	It can be verified (see \cite{KimSong2008Intrinsic} for example) that process $\{(\xi_t)_{t\geq 0}; (\widetilde\Pi_x)_{x\in E}\}$ is a time homogeneous Markov process.  Its transition density with respect to measure $m$ is given by
\begin{equation}
\label{eq: tilde p}
    \tilde p(t, x, y)
    :=\frac{\mbox{e}^{-\lambda t}}{\phi(x)}\ p^\beta(t, x, y)\phi(y),
    \quad x,y \in E,t>0.
\end{equation}
	It can also be verified that $\phi(y)\nu(dy)$ is the unique invariant measure of $\{(\xi_t)_{t\geq 0}; (\widetilde\Pi_x)_{x\in E}\}$.

	It follows from \cite[Theorem 2.7]{KimSong2008Intrinsic} that there exists $c, \rho > 0$ such that
\begin{equation}\label{IU}
	\sup_{x,y\in E}\Big|\frac{\tilde p(t,x,y)}{\phi(y) \hat\phi(y)}- 1\Big|
	=\sup_{x,y\in E}\Big|\frac{e^{-\lambda t}p^\beta(t,x,y)}{\phi(x) \hat\phi(y)}- 1\Big|
	\leq c\,e^{-\rho t},
	\quad t\geq 1.
\end{equation}

	Let $\{(\widehat{Y}_t)_{t\geq 0}; (\widehat{\Pi}_x)_{x\in E}\}$ be an $E$-valued Hunt process whose transition density with respect to measure $m$ is given by
\[
    \hat{p}(t,x,y)
    =e^{-\lambda t}p^\beta(t,y,x)\frac{{\hat\phi}(y)}{{\hat\phi}(x)}
    =\tilde p(t,y,x)\frac{\phi(y){\hat\phi}(y)}{\phi(x){\hat\phi}(x)},
    \quad x,y \in E,\,\, t> 0.
\]
	It is easy to check that $(\widehat Y_t)_{t\geq 0}$ has the unique invariant measure $\phi(x)\hat\phi(x)m(dx)$, and is exponentially ergodic in the sense that there exists $c, \rho > 0$ such that
\begin{equation}\label{IU'}
	\sup_{x,y\in E}\left|\frac{\hat{p}(t, x,y)}{\phi(y) \hat\phi(y)}- 1\right|\le c\,\mbox{e}^{-\rho t}, \quad t\geq 1.
\end{equation}
\begin{lem}
\label{lem:reverse of the spine}
	Let $\nu(dx):=\hat\phi(x)m(dx)$.
	For each $T > 0$, we have
\[
	\{(Y_{T-t})_{0\leq t\leq T}; \widetilde \Pi_{\phi \cdot \nu}\}
	\overset{d}{=} \{(\widehat Y_{t})_{0\leq t\leq T}; \widehat \Pi_{\phi \cdot \nu}\}
\]
\end{lem}
\begin{proof}
	Since $E$ is a Polish space, we only need to prove that for any fixed $T>0$, $n \in \mathbb N$ and $0= t_1\leq \dots \leq t_n \leq T$,
	 The following identity holds
\[
	\widetilde \Pi_{\phi \cdot \nu}\{Y_{T-t_i}\in B_i,\forall i=1,\dots, n\}
	\overset{d}{=}\widehat \Pi_{\phi \cdot \nu}\{\widehat Y_{t_i}\in B_i,\forall i=1,\dots, n\},
\]
for arbitrary $B_i \in \mathscr B(E)$,  $i=1,\dots, n$.
	In fact, on one hand, we have
\begin{align}
	&\widetilde \Pi_{\phi \cdot \nu}\{Y_{T-t_i}\in B_i,\forall i=1,\dots, n\}
	\\&= \int_{y_n\in B_n} \phi(y_n)\hat\phi(y_n) m(dy_n)\int_{y_{n-1}\in B_{n-1}} \tilde p_{t_n - t_{n-1}}(y_n,y_{n-1})m(dy_{n-1})
	\\& \qquad \dots \int_{y_1\in B_1} \tilde p_{t_2 - t_1}(y_2,y_1)m(dy_1)
	\\&= \int_{E^n} \Big(\prod_{i=1}^n \mathbf 1_{\{y_i\in B_i\}}\Big)\cdot\Big(\prod_{i=1}^{n-1} \tilde p_{t_{i+1}-t_i}(y_{i+1},y_i)\Big)\cdot\phi(y_n)\hat\phi(y_n)\cdot \Big(\prod_{i=1}^nm(dy_i)\Big).
\end{align}
	On the other hand, we have
\begin{align}
	&\widehat \Pi_{\phi\cdot\nu}\{Y_{t_i}\in B_i, \forall i = 1,\dots, n\}
	\\&= \int_{y_1\in B_1} \phi(y_1)\hat \phi(y_1)m(dy_1) \int_{y_2\in B_2} \hat p_{t_2-t_1}(y_1,y_2)m(dy_2)
	\\&\qquad \dots\int_{y_n\in B_n}\hat p_{t_n-t_{n-1}}(y_{n-1},y_n)m(dy_n)
	\\&= \int_{y_1\in  B_1} \phi(y_1)\hat \phi(y_1)m(dy_1) \int_{y_2\in B_2} \tilde p_{t_2-t_1}(y_2,y_1)\frac{\phi(y_2)\hat \phi(y_2)}{\phi(y_1)\hat \phi(y_1)}m(dy_2)
	\\&\qquad \dots\int_{y_n\in B_n}\tilde p_{t_n-t_{n-1}}(y_n,y_{n-1})\frac{\phi(y_n)\hat\phi(y_n)}{\phi(y_{n-1})\hat\phi(y_{n-1})}m(dy_n)
	\\&= \int_{E^n} \Big(\prod_{i=1}^n \mathbf 1_{\{y_i\in B_i\}}\Big)\cdot\Big(\prod_{i=1}^{n-1} \tilde p_{t_{i+1}-t_i}(y_{i+1},y_i)\Big)\cdot\phi(y_n)\hat\phi(y_n)\cdot \Big(\prod_{i=1}^n m(dy_i)\Big).
\qedhere
\end{align}
{\tt Ok, right now, it is not clear to me why this measure transformation is important for the proof.}
{\tt Just Put it in the Appendix.}
\end{proof}

\subsection{Kuznestuv measure}
	Suppose that $X$ is the superprocess introduced in Section \ref{model} which satisfies that $\mathbb P_{x}(\zeta < t)>0$ for each $x\in E$ and $t>0$. Denote by
\begin{align}
	\mathbb D &:=\{ w= (w_t)_{t\geq 0}: w \text{ is an $\mathcal M_f(E)$-valued c\`{a}dl\`{a}g function on $[0,\infty)$ }
	\\ &\qquad \text{ with the null measure as a trap} \}
\end{align}
	the Skorokhod space of measure-valued excursion paths.

	According to \cite[Section 8.4]{Li2011Measure-valued}, there is a unique family of $\sigma$-finite measures $(\mathbb N_x)_{x\in E}$ on $\mathbb D$ such that
\begin{itemize}
\item
    $\mathbb N_x \{\forall t > 0, w_t(\mathbf 1_E)=0\} =0$ for each $x\in E$;
\item
    $\mathbb N_x \{ w_0(\mathbf 1_E) > 0\} = 0$ for each $x\in E$;
\item
    For each $\mu \in \mathcal M(E)$, if $\mathcal N$ is a Poisson random measure on $\mathbb D$ with intensity
\[
	\mathbb N_\mu(dw):= \int_E \mathbb N_x(dw)\mu(dx), \quad w\in \mathbb D.
\]
	then
\[
	\{(X_t)_{t> 0};\mathbb P_\mu\}
	\overset{f.d.d.}{=} \left(\int_{\mathbb D} w_t~\mathcal N(dw)\right)_{t> 0}.
\]
\end{itemize}
	This family of measure $(\mathbb N_x)_{x\in E}$ is known as the \emph{Kuznetsov measures} of $X$.
	{\tt Z: It is not very clear, why we need this measure.}

	In the remainder of this paper, we will always use $w = (w_t)_{t\geq 0}$ to denote a generic element in $\mathbb D$.
	For each $f\in \mathcal B_b^+(D)$, from Campbell's formula it can be verified that for each $x\in E$ and $t>0$, we have
\begin{align}\label{eq: kuznetsov Laplace}
 	\mathbb N_x[1-e^{-w_t(f) }]
 	&=-\log \mathbb P_x[e^{-X_t(f)}] = V_t f(x),
 	\\ \mathbb N_x[w_t(f)]
 	&=P_t^{\beta}f(x),
 	\\
 \mathbb N_x(\zeta > t) &= - \log \mathbb P_x(\zeta \leq t).
\end{align}
	{\tt Z: It is also not very clear, why we need this here.}


\subsection{Spine decomposition}
    Suppose that $X$ is the superprocess introduced in Section \ref{model} which satisfies Assumption \ref{assum}.   Fix an arbitrary $\mu\in \mathcal M_f(E)$.  Define the probability $\widetilde {\mathbb P}_\mu$ using \eqref{eq: martingale transformation}.

    For each $\mu \in \mathcal M_f(E)\backslash\{0\}$,
    we say $\{(Y)_{t\geq 0}, (X^{\mathrm n, \sigma})_{\sigma\in \mathcal D^\mathrm n}, (X^{\mathrm m, \sigma})_{\sigma \in \mathcal D^\mathrm m}, (X_t)_{t\geq 0}; \mathbb Q_{\mu}\}$ is a \emph{spine representation} of $\{(X_t)_{t\geq 0}; \widetilde {\mathbb P}_\mu\}$ if the followings are true:
\begin{itemize}
\item
    The \emph{spine process} $\{(Y_t)_{t\geq 0}; \mathbb Q_\mu\}$ is a copy of $\{(Y_t)_{t\geq 0}; \widetilde \Pi_{\phi\cdot\mu}\}$.
\item
	Given $\{(Y_t)_{t\geq 0}; \mathbb Q_\mu\}$, \emph{the continuum immigration} $\{ (X^{\mathrm n,\sigma})_{\sigma \in \mathcal D^\mathrm n}; \mathbb Q_\mu(\cdot |Y)\}$ is a $\mathbb D$-valued point process such that
\[
	\mathrm n(ds,dw) := \sum_{\sigma\in \mathcal D^{\mathrm n}} \delta_{(\sigma, X^{\mathrm n,\sigma})}(ds,dw)
\]

	is a Poission random measure on $[0,T]\times \mathbb D$ with intensity
\[
	\mathbf n(ds,dw):= 2 \alpha(Y_s) ds \cdot \mathbb N_{Y_s}(dw).
\]
\item
	Given $\{(Y_t)_{t\geq 0}; \mathbb Q_\mu\}$, \emph{the discrete immigration} $\{(X^{\mathrm m,\sigma})_{\sigma\in \mathcal D^{\mathrm m}}; \mathbb Q_\mu(\cdot |Y)\}$ is a $\mathbb D$-valued point process such that
\[
	\mathrm m(ds,dw) := \sum_{\sigma\in \mathcal D^{\mathrm n}} \delta_{(\sigma, X^{\mathrm n,\sigma})}(ds,dw)
\]
	is a Poisson random measure on $[0,\infty ) \times \mathbb D$ with intensity
\begin{align}\label{eq:meanMeasImmigr}
	\mathbf m(ds,dw):= ds \cdot \int_{(0,\infty)} y \mathbb P_{y\delta_{Y_s}}(X\in dw) n(Y_s,dy);
\end{align}
\item
	Given $\{(Y_t)_{t\geq 0}; \mathbb Q_\mu\}$, the continuum immigration $(X^{\mathrm n,\sigma})_{\sigma \in \mathcal D^n}$ and the discrete immigration $(X^{\mathrm m,\sigma})_{\sigma\in \mathcal D^{\mathrm m}}$ are independent of each other.
\item
	%$\{(X_t)_{t\geq 0}; \mathbb Q_\mu\}$ is a copy of the superprocess $\{(X_t)_{t\geq 0}; \mathbb P_\mu\}$ which is independent of the spine process $(Y_t)_{t\geq 0}$, the continuum immigration $(X^{\mathrm n,\sigma})_{\sigma \in \mathcal D^n}$ and the discrete immigration $(X^{\mathrm m,\sigma})_{\sigma\in \mathcal D^{\mathrm m}}$.
	$\{(X_t)_{t\geq 0}; \mathbb Q_\mu\}$ is a copy of the superprocess $\{(X_t)_{t\geq 0}; \mathbb P_\mu\}$, and is independent of the spine process $(Y_t)_{t\geq 0}$, the continuum immigration $(X^{\mathrm n,\sigma})_{\sigma \in \mathcal D^\mathrm n}$ and the discrete immigration $(X^{\mathrm m,\sigma})_{\sigma\in \mathcal D^{\mathrm m}}$.
\end{itemize}

	%For each $\mu \in \mathcal M(E)$,
	To simplify the notations, for each $\mu \in \mathcal M(E)$,
	$t\geq 0$ and each $B \in \mathscr B([0,t))$, with respect to probability $\mathbb Q_\mu$, define the following random measures:
\begin{align}
	Z^{\mathrm n,B}_t
	&:= \int_{B\times \mathbb D} w_{t-s} ~\mathrm n (ds,dw)
	= \sum_{\sigma \in \mathcal D^\mathrm n \cap B} X^{\mathrm n,\sigma}_{t-\sigma},
	\\ Z^{\mathrm m,B}_t
	&:= \int_{B\times \mathbb D} w_{t-s} ~\mathrm m (ds,dw)
	= \sum_{\sigma \in \mathcal D^\mathrm m \cap B} X^{\mathrm m,\sigma}_{t-\sigma}.
\end{align}

	The spine decomposition theorem (see \cite{RenSongSun2017Spine} for the general cases) says that
\begin{lem}\label{spine structure}
	Suppose that $\{(Y)_{t\geq 0}, (X^{\mathrm n, \sigma})_{\sigma\in \mathcal D^\mathrm n}, (X^{\mathrm m, \sigma})_{\sigma \in \mathcal D^\mathrm m}, (X_t)_{t\geq 0}; \mathbb Q_{\mu}\}$ is a spine representation of $\{(X_t)_{t\geq 0}; \widetilde {\mathbb P}_\mu\}$. Then
\begin{align}
	\{(X_t)_{t\geq 0}; \widetilde{\mathbb P}_\mu\}
	\overset{f.d.d.}{=}
	\{(X_t + Z^{\mathrm n, [0,t)}_{t} + Z^{\mathrm m, [0,t)}_{t} )_{t\geq 0}; \mathbb Q_\mu\}.
\end{align}
\end{lem}
	{\tt First of all, this is just a repeat of the know result. It should be introduced if we really need it.}
\subsection{Reverse spine representation}
Suppose that $X$ is the superprocess introduced in Section \ref{model} which satisfies Assumption \ref{assum}.
	{\tt I'm not really a fan of this Reverse spine representation. It is in its natural the spine decomposition. We should be able proof things without it.
	The only key is that the spine process is, well, invariant under certain invariant measure.}

    Recall that $\nu := \hat \phi \cdot m \in \mathcal M_f(E)$. Define the probability $\widetilde {\mathbb P}_\nu$ using \eqref{eq: martingale transformation}.
    We say $\{(Y)_{t\geq 0}, (X^{\mathrm n, \sigma})_{\sigma\in \mathcal D^\mathrm n}, (X^{\mathrm m, \sigma})_{\sigma \in \mathcal D^\mathrm m}, (X_t)_{t\geq 0}; \widehat {\mathbb Q}_{\nu}\}$ is a \emph{reverse spine representation} of $\{(X_t)_{t\geq 0}; \widetilde {\mathbb P}_\nu\}$ if the followings are true:
\begin{itemize}
\item
    The \emph{reverse spine process} $\{(Y_t)_{t\geq 0}; \widehat {\mathbb Q}_\nu\}$ is a copy of $\{(Y_t)_{t\geq 0}; \widehat \Pi_{\phi\cdot\nu}\}$.
\item
    Conditioned on $\{(Y_t)_{t\geq 0}; \widehat{\mathbb Q}_\nu\}$, \emph{the reverse continuum immigration} $\{ (X^{\mathrm n,\sigma})_{\sigma \in \mathcal D^\mathrm n}; \widehat{\mathbb Q}_\nu(\cdot |Y)\}$ is a $\mathbb D$-valued point process such that
\[
    \mathrm n(ds,dw)
    = \sum_{\sigma\in \mathcal D^{\mathrm n}} \delta_{(\sigma, X^{\mathrm n,\sigma})}(ds,dw)
\]
	is a Poission random measure on $[0,T]\times \mathcal W$ with density
\[
	\mathbf n(ds,dw)= 2\alpha(Y_s) ds \cdot \mathbb N_{Y_s}(dw).
\]
\item
    Conditioned on $\{(Y_t)_{t\geq 0}; \widehat{\mathbb Q}_\nu\}$, \emph{the reverese discrete immigration} $\{(X^{\mathrm m,\sigma})_{\sigma\in \mathcal D^{\mathrm m}}; \widehat{\mathbb Q}_\nu(\cdot |Y)\}$ is a $\mathbb D$-valued point process such that
\[
    \mathrm m(ds,dw)
    = \sum_{\sigma\in \mathcal D^{\mathrm m}} \delta_{(\sigma, X^{\mathrm m,\sigma})}(ds,dw)
\]
	is a Poisson random measure on $[0,\infty ) \times \mathbb D$ with intensity
\begin{align}\label{eq:meanMeasImmigr}
	\mathbf m(ds,dw)= ds \cdot \int_{(0,\infty)} y \mathbb P_{y\delta_{Y_s}}(X\in dw) n(Y_s,dy);
\end{align}
\item
	Given $\{(Y_t)_{t\geq 0}; \widehat{\mathbb Q}_\nu\}$, the reverse continuum immigration $(X^{\mathrm n,\sigma})_{\sigma \in \mathcal D^n}$ and the reverse discrete immigration $(X^{\mathrm m,\sigma})_{\sigma\in \mathcal D^{\mathrm m}}$ are independent of each other.
\item
	$\{(X_t)_{t\geq 0}; \widehat {\mathbb Q}_\nu\}$ is a copy of the superprocess $\{(X_t)_{t\geq 0}; \mathbb P_\mu\}$ which is independent of the reverse spine process $(Y_t)_{t\geq 0}$, the reverse continuum immigration $(X^{\mathrm n,\sigma})_{\sigma \in \mathcal D^n}$ and the reverse discrete immigration $(X^{\mathrm m,\sigma})_{\sigma\in \mathcal D^{\mathrm m}}$.
\end{itemize}

	To simplify the notations, for each $t\geq 0$, with respect to probability $\widehat{\mathbb Q}_\nu$, define the following random measures:
\[\begin{split}
	\widehat Z^{\mathrm n}_t
	&:= \int_{[0,t)\times \mathbb D} w_{s} ~\mathrm n (ds,dw)
	= \sum_{\sigma \in \mathcal D^\mathrm n \cap [0,t)} X^{\mathrm n,\sigma}_{\sigma},
	\\ \widehat Z^{\mathrm m}_t
	&:= \int_{[0,t)\times \mathbb D} w_{s} ~\mathrm m (ds,dw)
	= \sum_{\sigma \in \mathcal D^\mathrm m \cap [0,t)} X^{\mathrm m,\sigma}_{\sigma}.
\end{split}\]
\begin{lem}
	Suppose that $\{X, Y, \mathrm n, \mathrm m; \widehat{\mathbb Q}_{\nu}\}$ is a reverse spine representation of $\{X; \widetilde {\mathbb P}_\nu\}$.
	Then for each $t\geq 0$,
$
	\{X_t; \widetilde{\mathbb P}_\nu\}
	\overset{d}{=}
	\{X_t + \widehat Z^{\mathrm n}_{t} + \widehat Z^{\mathrm m}_{t}; \widehat{\mathbb Q}_\nu\}.
$
\end{lem}
\begin{proof}
	Fix an arbitrary time $t\geq 0$. According to Lemma \ref{spine structure}, we only have to proof that
\[
	\{Z^{\mathrm n,[0,t)}_{t} + Z^{\mathrm m,[0,t)}_{t}; \mathbb Q_\nu\}
	\overset{d}{=}
	\{\widehat Z^{\mathrm n}_{t} + \widehat Z^{\mathrm m}_{t}; \widehat{\mathbb Q}_\nu\}.
\]
	In fact, for each $f\in \mathcal B_b^+(E)$, from campbell's formula, we have
\begin{align}
	&-\log \mathbb Q_\nu \left [\left. e^{-\langle f, Z^{\mathrm n,[0,t)}_{t} + Z^{\mathrm m,[0,t)}_{t}\rangle}\right |(Y_t)_{t\geq 0}\right]
	\\&= \int_{[0,t)\times \mathbb D} \left(1-e^{- \langle f, w_{t-s}\rangle}\right)\left(\mathbf n(ds,dw) + \mathbf m(ds,dw)\right)
	\\&= \int_{[0,t)} \left(2\alpha(Y_s) \cdot \mathbb N_{Y_s}\left(1-e^{-w_{t-s}(f)}\right) + \int_{(0,\infty)} y \mathbb P_{y\delta_{Y_s}}\left(1-e^{-X_{t-s}(f)}\right)n(Y_s,dy)\right) ds
	\\&= \int_{[0,t)} \left(2\alpha(Y_s) \cdot (V_{t-s}f)(Y_s) + \int_{(0,\infty)} y \left(1-e^{-y\cdot(V_{t-s}f)(Y_s)}\right)n(Y_s,dy)\right) ds
	\\&= \int_{[0,t)} \psi_0'\left( Y_s, V_{t-s}f(Y_s)\right)ds
\end{align}
	and
\begin{align}
	&-\log \widehat{\mathbb Q}_\nu \left [\left. e^{-(\widehat Z^{\mathrm n}_{t} + \widehat Z^{\mathrm m}_{t})(f)}\right |(Y_t)_{t\geq 0}\right]
	\\&= \int_{[0,t)\times \mathbb D} \left(1-e^{- w_s(f)}\right)\left(\mathbf n(ds,dw) + \mathbf m(ds,dw)\right)
	\\&= \int_{[0,t)} \left(2\alpha(Y_s) \cdot \mathbb N_{Y_s}\left(1-e^{-w_{s}(f)}\right) + \int_{(0,\infty)} y \mathbb P_{y\delta_{Y_s}}\left(1-e^{-X_{s}(f)}\right)n(Y_s,dy)\right) ds
	\\&= \int_{[0,t)} \left(2\alpha(Y_s) \cdot (V_{t}f)(Y_s) + \int_{(0,\infty)} y \left(1-e^{-y\cdot(V_{t}f)(Y_s)}\right)n(Y_s,dy)\right) ds
	\\&= \int_{[0,t)} \psi_0'\left( Y_s, V_{t}f(Y_s)\right)ds.
\end{align}
	Therefore, according to Lemma \ref{lem:reverse of the spine}, for each $f\in \mathcal B_b^+(E)$, we have
\begin{align}
  	&\mathbb Q_\nu  \big[e^{-(Z^{\mathrm n,[0,t)}_{t} + Z^{\mathrm m,[0,t)}_{t})(f)}\big]
  	= \widetilde \Pi_{\phi\cdot\nu} \big[e^{-\int_{[0,t)} \psi_0'( Y_s, V_{t-s}f(Y_s))ds}\big]
  	\\&= \widetilde \Pi_{\phi\cdot\nu} \big[e^{-\int_{[0,t)} \psi_0'( Y_{t-s}, V_{s}f(Y_{t-s}))ds}\big]
  	= \widehat \Pi_{\phi\cdot\nu} \big[e^{-\int_{[0,t)} \psi_0'( Y_{s}, V_{s}f(Y_{s}))ds}\big]
  	\\&= \widehat{\mathbb Q}_\nu \big [e^{-(\widehat Z^{\mathrm n}_{t} + \widehat Z^{\mathrm m}_{t})(f)}\big].
  	\qedhere
\end{align}
\end{proof}
The processes  $\{(Y)_{t\geq 0}, \widehat{\Pi}_x\}$ is ergodic and have the same invariant probability $\phi(x)\hat\phi(x)m(dx)$.
	Its transition probabilities have uniform convergence properties \eqref{IU'}.
	Therefore we can repeat the arguments for Lemma $3.2$ in \cite{LiuRenSong2009Llog} and obtain the following results. The proofs will be omitted.
\begin{lem}\label{import lemma}
	Let $\nu := \hat \phi \cdot m$.
	Suppose that \[\{(Y)_{t\geq 0}, (X^{\mathrm n, \sigma})_{\sigma\in \mathcal D^\mathrm n}, (X^{\mathrm m, \sigma})_{\sigma \in \mathcal D^\mathrm m}, (X_t)_{t\geq 0}; \widehat{\mathbb Q}_{\nu}\}\] is a reverse spine representation of $\{(X_t)_{t\geq 0}; \widetilde {\mathbb P}_\nu\}$.
	With respect to probability $\widehat{\mathbb Q}_\nu$, let $(m_\sigma)_{\sigma\in \mathcal D^{\mathrm m}}$ be the $\mathbb R^+$-valued point process defined by
\[
	m_\sigma
	= X^{\mathrm m, \sigma}_0(\mathbf 1_E),
	\quad \sigma \in \mathcal D^{\mathrm m}.
\]
And define the following sequence of random variables
\[
	\sigma_0=0,\quad \sigma_i=\inf\{s\in\mathcal D^{\mathrm m};\ s>\sigma_{i-1},\ m_s\phi(Y_s)>1\}, \quad\eta_i=m_{\sigma_i},\quad i=1,2,\cdots.
\]
Then for each $\varepsilon>0$, when $\int_E\widehat{\phi}(y)l(y)m(dy)<\infty$,
\[
	\sum_{s\in\mathcal D^{\mathrm m}} e^{-\varepsilon s}m_s\phi(Y_s)
	< \infty,
	\quad \widehat{\mathbb Q}_{\nu}\text{-a.s.};
\]
when $ \int_E\widehat{\phi}(y)l(y)m(dy)=\infty$,
\[
	\limsup_{i\rightarrow\infty} e^{-\varepsilon\sigma_i}\eta_i \phi(Y_{\sigma_i})
	=\infty,
	\quad \widehat{\mathbb Q}_{\nu}\text{-a.s.}.
\]
\end{lem}
%%%%%%%%%%%%%%%%%%%%%%%%%%%%%%%%%%%%%%%%%%%%%%%%%%%%%%%%%%%%%%%%%%%%%%%%%%%%%%%%%%%%%%%%%%%%%%%%%%%%%%%%%%%%%%%%%%%%%%%%%%%%%%%%%%
\section{Proofs of Main Results}
\subsection{Some lemmas}
\begin{lem}\label{lem:extinc}
	Suppose that Assumptions \ref{assum} hold. Then,
\begin{enumerate}
\item
	there exists $t_0>0$ such that
\[
	\langle v(t,\cdot),\mu\rangle <\infty, \quad t>t_0, \, \mu \in \mathcal M_f(E).
\]
\item	for each $\mu \in \mathcal M_f(E)$,
\[
	\lim_{t\rightarrow\infty}\langle v(t,\cdot),\mu\rangle=0.
\]
\item
	for each $s\geq 0$,
\begin{equation} \label{one point ratio limit}
	\lim_{t\to \infty} \sup_{x\in E}\Big|\frac{v(t+s,x)}{\langle v(t,\cdot),\nu\rangle\phi(x) } - e^{\lambda s} \Big|
	=0.
\end{equation}
\end{enumerate}
\end{lem}
\begin{remark}
 From the theorem we can see that $v(t,x)$  has the asymptotic behavior as $t\to\infty$
 \[
 \mathbb P_x(\zeta>t)\sim v(t,x)\sim \phi(x)a(t)e^{\lambda t},
 \]
 where $\lim_{t\rightarrow\infty}a(t+s)/a(t)=1$ for all $s>0$.  When $\int_E\widehat\phi(y)l(y)m(dy)<\infty$, $\lim_{t\rightarrow\infty}a(t)=k>0$, which will be given in \eqref{cons}.  When $\int_E\widehat\phi(y)l(y)m(dy)=\infty$, $\lim_{t\rightarrow\infty}a(t)=0$.
 \end{remark}

{\tt This remark is not necessary here.}

{\tt The proof should be simplified}

\begin{proof}
It is known that function $v(t,x)$ satisfies equation
\begin{equation}
\label{eq: equation for vt}
	v(t+s,x) + \int_0^sP^\beta_{s-u}\Big(\psi_0\big(\cdot, v(t+u,\cdot)\big)\Big)(x)~du
	=P^\beta_s\big(v(t,\cdot)\big)(x),
	\quad t,s > 0, x\in E.
\end{equation}

	We will proof the following claim which is stronger than (1): there exists $t_0 >0$ such that for each $t\geq t_0, s> 0$ and $x \in E$, the both side of the above equation are finite.
	In fact, according to Assumption \ref{assum} (2) and \eqref{eq: v and extinction}, we have $\langle v_t, \nu\rangle < \infty $ for $t>0$ large enough.
	According to Assumption \ref{assum} (3) and \eqref{eq: equation for vt}, there exists $t_0>0$ such that for all $t \geq t_0,s>0$ and $x\in E$ we have
\begin{equation}
\label{upp}
	v(t+s,x) \leq P^{\beta}_sv(t,x)\leq c_s\phi(x)\int_E\hat\phi(y)v(t,y)m(dy)\leq c_s\phi(x)\langle v(t,\cdot),\nu\rangle< \infty,
\end{equation}
	where $(c_s)_{s>0}$ are the constants appreared in Assumption \ref{assum} (3) and only depend on the choice of $s$.
	This implies the claim.

	According to \eqref{IU}, the constants $(c_s)_{s>0}$ appeared in Assumption \ref{assum} (3) can actually be chosen such that $c_s \xrightarrow[s\to \infty]{} 0$.
	In this case, according to \eqref{eq: equation for vt}, there exists $t_0>0$ such that for each $t \geq t_0, s>0$ and $\mu \in \mathcal M_f(E)$, we have
\[
	\langle v(t +s,\cdot),\mu\rangle
	\leq  \langle P^\beta_sv(t,\cdot),\mu\rangle
	\leq  c_s\langle \phi,\mu\rangle \langle v(t,\cdot), \nu \rangle
	\xrightarrow[s\to \infty]{} 0.
\]
	This implies $(2)$.  The result $(3)$ comes from Proposition $1.4$ and Lemma $2.1$ in our paper.
\end{proof}
The proof of Proposition $1.4$ is based on the following three results in Tauberian Theory.  We refer to \cite{AH} $A14$.

{\tt Great, I need to check if they are alright. Better to put everthing in the appendix though.}
\begin{lem}\label{lem: tau}
Let $U(x)$ be a monotone nondecreasing function on $[0,\infty)$ such that
\[
\omega(x)=\int_0^\infty e^{-xu} dU(u)
\]
is finite for all $x>0$. If for some $\alpha\geq 0$, $\omega(x)\sim x^{-\alpha}L(1/x)$, $x\downarrow 0$, where $L$ is slowly varying at infinity, then
\[
U(x)\sim x^{\alpha}\dfrac{L(x)}{\Gamma(\alpha+1)},\qquad x\to\infty.
\]
 And if for some $\alpha\geq 0$, $\omega(x)\sim x^{-\alpha}L(x)$, $x\uparrow \infty$, then
\[
U(x)\sim x^{\alpha}\dfrac{L(1/x)}{\Gamma(\alpha+1)},\qquad x\downarrow 0.
\]
\end{lem}
\begin{lem}\label{lem:regu}
Let $f(x)$ be monotone increasing for $x\in (0,c)$. If
\[
\lim_{n\to\infty}\dfrac{f(\lambda\theta_n)}{f(\theta_n)}=\lambda^\alpha,\qquad \forall \lambda\in (0,1],
\]
for some $\alpha\in\mathbb R$ and some sequence $\{\theta_n\}$ of positive reals tending to $0$, as $n\to\infty$ in such a way that $\theta_n/\theta_{n+1}\leq c$ for $n\in\mathbb N$ and some $1<c<\infty$, then $f(x)$ is regularly varying with exponent $\alpha$.
\end{lem}
\begin{lem}\label{lem:tail}
For $x\in [\beta,\infty)$, let
\[
U(x)=\int_\beta^xu(y)dy,
\]
where $u(y)$ is ultimately monotone.  If for some $\alpha\geq 0$,$U(x)=x^\alpha L(x)$,  where $L$ is slowly varying at infinity, then
\[
\lim_{x\to\infty}\dfrac{xu(x)}{U(x)}=\alpha.
\]
\end{lem}
\subsection{Proof of Theorem \ref{thm: distribution of zeta}}
\begin{proof}
According to Lemma \ref{lem:extinc}, we only need to prove results $(1)$ and $(3)$ hold for $\nu(dx)=\hat\phi(x)m(dx)$.  

{\tt Well, (3) has nothing to do with $\nu$. But I guess I what you mean.}

And the result $(2)$ comes from $(3)$ in Lemma \ref{lem:extinc}.  We are left to prove $(1)$ and $(3)$ under $\mathbb P_{\nu}$.  It is easy to see that for any $t\ge 0,$
\begin{eqnarray*}\label{subcritical equality}
 	&&e^{-\lambda t}\mathbb P_\nu(\zeta>t)=\widetilde{\mathbb P}_\nu\left(\frac{1}{\langle\phi, X_{t}\rangle }\right)\\
 &=&\mathbb Q_{\nu}\left[\frac{1}{\langle\phi, X_{t}\rangle +\sum_{\sigma\in(0, t]\bigcap\mathcal D^{\mathrm m}}\langle \phi, X_{t-\sigma}^{{\mathrm m},\sigma}\rangle +\sum_{\sigma\in (0, t]\bigcap \mathcal D^{\mathrm n}}\langle \phi, X_{t-\sigma}^{{\mathrm n}, \sigma}\rangle} \right]
\end{eqnarray*}
	
	{\tt Basically, I agree, but we probably want to mention $\widetilde {\mathbb P}$ here.}
	
When the initial value of $X$ is $\nu(dx)$, the initial distribution of the spine under $\mathbb Q_\nu$ is $\phi\cdot\nu$.
	From Lemma \ref{lem:reverse of the spine}, we have
\begin{eqnarray*}\label{duality}
    &&  \mathbb Q_{\nu}\left[\frac{1}{\langle\phi, X_{t}\rangle +\sum_{\sigma\in(0, t]\bigcap\mathcal D^{\mathrm m}}\langle \phi, X_{t-\sigma}^{{\mathrm m},\sigma}\rangle +\sum_{\sigma\in (0, t]\bigcap \mathcal D^{\mathrm n}}\langle \phi, X_{t-\sigma}^{{\mathrm n}, \sigma}\rangle }\right]\\
    &=&\widehat{\mathbb Q}_{\nu}\left[\frac{1}{\langle\phi, X_{t}\rangle +\sum_{\sigma\in(0, t]\bigcap\mathcal D^{\mathrm m}}\langle \phi, \widehat X_{\sigma}^{{\mathrm m},\sigma}\rangle +\sum_{\sigma\in (0, t]\bigcap \mathcal D^{\mathrm n}}\langle \phi, \widehat X_{\sigma}^{{\mathrm n},\sigma}\rangle }\right]
\end{eqnarray*}
    Since $\lim_{t\to\infty}\langle\phi, X_{t}\rangle=0$ and $\sum_{\sigma\in(0, t]\bigcap\mathcal D^{\mathrm m}}\langle \phi, \widehat X_{\sigma}^{{\mathrm m},\sigma}\rangle +\sum_{\sigma\in (0, t]\bigcap \mathcal D^{\mathrm n}}\langle \phi, \widehat X_{\sigma}^{{\mathrm n},\sigma}\rangle $ is increasing with respect to $t$ having almost sure limit as $t\to\infty$ as well.  
    
    {\tt We should be able to show the increaseness without the backward spine decomposition.}
    
    Therefore, by dominated convergence theory, if we let
\begin{equation}\label{cons}
	k=\widehat{\mathbb Q}_{\nu}\left[\frac{1}{\sum_{\sigma\in\mathcal D^{\mathrm m}}\langle \phi, \widehat X_{\sigma}^{{\mathrm m},\sigma}\rangle +\sum_{\sigma\in \mathcal D^{\mathrm n}}\langle \phi, \widehat X_{\sigma}^{{\mathrm n},\sigma}\rangle }\right],
\end{equation}
then $k<\infty$ and $\lim_{t\to\infty} e^{-\lambda t}\mathbb P_{\nu}(\zeta>t)=k$.  {\tt We should be using monotone convergence theorem, right?}

$(1)$ follows. {\tt We don't have to prove $(1)$ here.}

 Now let us analyze the cases in which $k>0$.	For the continuum immigration part,
\[
	\widehat{\mathbb Q}_{\nu}\left(\sum_{\sigma\in \mathcal D^{\mathrm n}}\langle \phi, \widehat X_{\sigma}^{{\mathrm n},\sigma}\rangle \right)=\int_0^\infty e^{\lambda s}\langle 2\alpha\phi, \phi\nu\rangle  ds=\frac{\langle 2\alpha\phi, \phi\nu\rangle}{-\lambda}<\infty.
\]
	Therefore, $\sum_{\sigma\in \mathcal D^{\mathrm n}}\langle \phi, \widehat X_{\sigma}^{{\mathrm n},\sigma}\rangle$ is finite almost surely.

	For the discrete immigration part, let $\mathcal G=\sigma(Y, ( m_\sigma)_{\sigma\in\mathcal D^{\mathrm m}})$.  When  $\int_E\hat{\phi}(y)l(y)m(dy)<\infty$, by Lemma \ref{import lemma},
\[
	\widehat{\mathbb Q}_{\nu}\left(\sum_{\sigma\in \mathcal D^{\mathrm m}}\langle \phi, \widehat X_{\sigma}^{{\mathrm m}}\rangle\Big|\mathcal G \right)
	=\sum_{\sigma<\infty}m_\sigma e^{\lambda \sigma}\phi(Y_{\sigma})<\infty,  \qquad\qquad \widehat{\mathbb Q}_{\nu}-{\mathrm a.s.}
\]
	Thus in this case, $k>0$.

	We claim that when $\int_D\hat\phi(y)l(y)m(dy)=\infty$,
\begin{equation}\label{infty}
	\sum_{\sigma\in \mathcal D^{\mathrm m}}\langle \phi, \widehat X_{\sigma}^{{\mathrm m},\sigma}\rangle =\infty,\quad\qquad  \widehat{\mathbb Q}_\nu-{\mathrm a.s.}
\end{equation}
	
	{\tt I think here, some details can be seperated.}
	
	In the proof of \cite[Lemma $3.2$]{LiuRenSong2009Llog}.  It is shown that for any $N>0$,
\begin{equation}\label{inf}
	\int_0^\infty dt\int_{\phi(Y_t)^{-1}e^{Nt}}^\infty rn(Y_t,dr)
	=\infty,\quad \widehat{\mathbb Q}_\nu-{\mathrm a.s.}
\end{equation}
	Fix a path of $(Y_t)$.  And define stochastic
	time sequence
\[
	\tau_0:=\left\{t>0; m_t>\phi(Y_t)^{-1}e^{Nt}\right\},\,
	\tau_{i+1}:=\left\{t>\tau_i;\, m_t>\phi(Y_t)^{-1}e^{Nt}\right\},\, i=0,1,\cdots
\]
	Then $\tau_i<\infty$, $i=1,2,\ldots$ almost surely from Lemma \ref{import lemma}.
	If we can prove that $\sum_{i=0}^\infty I_{\left\{\langle\phi, \widehat X^{{\mathrm m},\tau_i}_{\tau_i}\rangle  >\varepsilon\right\}}=\infty,$ for some $\varepsilon>0,$ then our claim holds.  Similar to the argument for the proof of the second assertion of \cite[Lemma $2.2$]{LiuRenSong2009Llog}, we just need to prove
\[
	\widehat{\mathbb Q}_\nu
	\left[\left.\sum_{i=0}^\infty I_{\left\{\langle\phi, \widehat X^{{\mathrm m},\tau_i}_{\tau_i}\rangle  >\varepsilon\right\}} \right| Y\right]=\infty,\quad \widehat\Pi_{\phi\cdot\nu}{\mathrm -a.s.}
\]
	Since given the spatial motion of the spine the immigration process is a Poisson point process, therefore,
\[
	\widehat{\mathbb Q}_\nu\left[\left.\sum_{i=0}^\infty I_{\left\{\langle\phi, \widehat X^{{\mathrm m},\tau_i}_{\tau_i}\rangle  >\varepsilon\right\}}\right| Y\right]=\int_0^\infty dt\int_{\phi(Y_t)^{-1}e^{Nt}}^\infty rn(Y_t, dr)\mathbb{P}_{r\delta_{Y_t}}\big(\langle\phi, X_t \rangle >\varepsilon\big).
\]
	If we can prove for some $\varepsilon>0$, there is some time $t_0>0$, and $\delta>0$, such that when $t>t_0$,
\begin{eqnarray}\label{last point}
	\inf_{r\geq \phi(x)^{-1}e^{Nt}, x\in E}\mathbb P_{r\delta_x}\big(\langle\phi, X_t
	\rangle >\varepsilon\big)>\delta,
\end{eqnarray}
	then from \eqref{inf}, \eqref{infty} is obtained.  By Chebyshev inequality, for any $\varepsilon>0$,
\begin{eqnarray*}
	&&\mathbb P_{r\delta_x}\big(\langle\phi, X_t\rangle >\varepsilon\big)=\mathbb P_{r\delta_x}\left(e^{-\langle\phi, X_t\rangle }<e^{-\varepsilon}\right)\\
	&=&1-\mathbb P_{r\delta_x}\left(e^{-\langle\phi, X_t
	\rangle }\geq e^{-\varepsilon}\right)\geq 1-e^{\varepsilon }\mathbb P_{r\delta_x}e^{-\langle\phi, X_t\rangle }\\
	&=&1-e^{\varepsilon }e^{-rV_t\phi(x)},
\end{eqnarray*}
	where, $V_t\phi(x)$ is a classical solution to evolution equation
\begin{equation}\label{eq diff}
\begin{cases}
	\dfrac{\partial U}{\partial t}=AU-\beta(x)U-\psi_0(x, U),& x\in E, t>0;\\
	U(0,x)=\phi(x),& x\in E.\\
\end{cases}
\end{equation}
	Define a function on $[0,\infty)\times E,$ $V(t,x)=e^{-Nt}\phi(x).$  Then
\[
	(A-\beta)V(t,x)-\psi_0(x, V)-\frac{\partial V(t,x)}{\partial t}=(\lambda +N)V(t,x)-\psi_0(x,V(t,x)).
\]
	It is obvious that $V$ is bounded on $[0,\infty)\times E$.  So we can choose $N$ large enough such that $(\lambda+N)V(t,x)-\psi(x,V)\geq 0$ on $[0,\infty)\times E.$
	Meanwhile, $V(0,x)=\phi(x)$.  Applying comparison theorem for semilinear equation, we obtain that $V_t\phi(x)\geq V(t,x)$.  So
\begin{eqnarray*}
	&&\inf_{r\geq \phi(x)^{-1}e^{Nt}, x\in E}\mathbb P_{r\delta_x}\big(\langle\phi, X_t \rangle >\varepsilon\big)\geq \inf_{r\geq \phi(x)^{-1}e^{Nt}, x\in E} 1-e^{\varepsilon}e^{-rV_t\phi(x)}\\
    &&\geq \inf_{r\geq \phi(x)^{-1}e^{Nt}, x\in E} 1-e^{\varepsilon}e^{-rV(t,x)}
     \geq 1-e^{\varepsilon-1}.
\end{eqnarray*}
	Choose $0<\varepsilon<1$, and let $\delta=1-e^{\varepsilon-1}$.  Then \eqref{last point} is obtained. And
\eqref{infty} follows. Thus $k=0$. The proof is finished.
\end{proof}
%%%%%%%%%%%%%%%%%%%%%%%%%%%%%%%%%%%%%%%%%%%%%%%%%%%%%%%%%%%%%%%%%%%%%%%%%%%%%%%%%%%%%%%%%%%%%%%%%%%%%%%%%%%%%%%%%%%%%%%%%%%%%%%%%%%%%%%%%%%%%%%%%%%%%%%%%%%%%%%%	
\subsection{Proof of Theorem \ref{thm: Qprocess}}
\begin{proof} 
	
	{\tt It seems that Theorem 1.3 is a simple result. Maybe it should be the first result!} 
	
Assume $s>t$.  For any $A\in\mathscr F_t$, by the Markov property of $X$,
\[
\mathbb P_\mu(A|\zeta>s)=\dfrac{\mathbb P_\mu(A, \zeta>s)}{\mathbb P_\mu(\zeta>s)}=\dfrac{\mathbb P_\mu\big(\mathbb P_{X_t}(\zeta>s-t);A\big)}{\mathbb P_\mu(\zeta>s)},
\]
 Note that
\begin{eqnarray*}
\lim_{s\rightarrow\infty}\dfrac{\mathbb P_{X_t}(\zeta>s-t)}{\mathbb P_\mu(\zeta>s)}
&=&\lim_{s\rightarrow\infty}\dfrac{1-e^{-\langle v(s-t,\cdot),X_t\rangle }}{1-e^{-\langle v(s,\cdot),\mu\rangle }}
=\lim_{s\rightarrow\infty}\dfrac{\langle v(s-t,\cdot),X_t\rangle }{\langle v(s,\cdot),\mu\rangle }\\
&=& \dfrac{e^{-\lambda t}\langle \phi, X_t\rangle }{\langle \phi,\mu\rangle }=\dfrac{M_t(\phi)}{\langle \phi,\mu\rangle }.
\end{eqnarray*}
The third identity follows from\eqref{one point ratio limit}.  From \eqref{upp} we can get that there is a constant $\widetilde C>0$, such that for any $s>T>0$ and $x\in E$,
\[
v(s,x)\leq \widetilde C\phi(x)e^{\lambda T}\langle v(s-T,\cdot),\nu\rangle .
\]
Using the fact that $\lim_{x\rightarrow 0+}\dfrac{1-e^{-x}}{x}=1$, we choose $s$ sufficiently large such that
\[
1-e^{-\langle v(s,\cdot),\mu\rangle }>\frac{1}{2}\langle v(s,\cdot),\mu\rangle .
\]
Meanwhile since $1-e^{-x}\leq x$ for $x>0$, choosing $0<T_0<s-t$
\[
\dfrac{1-e^{-\langle v(s-t,\cdot),X_t\rangle }}{1-e^{-\langle v(s,\cdot),\mu\rangle }}
\leq \dfrac{2\langle v(s-t,\cdot),X_t\rangle }{\langle v(s,\cdot),\mu\rangle }\leq \dfrac{2\widetilde C\langle \phi,X_t\rangle e^{\lambda T_0}\langle v(s-t-T_0,\cdot),\nu\rangle }{\langle v(s,\cdot),\mu\rangle }.
\]
We already show in \eqref{one point ratio limit} that
\[
\lim_{s\rightarrow\infty}\dfrac{\langle v(s-t-T_0,\cdot),\nu\rangle }{\langle v(s,\cdot),\mu\rangle }
=e^{-\lambda(t+T_0)}\langle \phi,\mu\rangle ^{-1}.
\]
And $\langle \phi,X_t\rangle $ is integrable with respect to $\mathbb P_\mu$.  Thus by dominated convergence theorem,
\[
\lim_{s\rightarrow\infty}\mathbb P_\mu(A|\zeta>s)=\mathbb P_\mu\left(\frac{M_t(\phi)}{\langle\phi,\mu\rangle };A\right)=\widetilde{\mathbb P}_\mu(A).
\]
\end{proof}
%%%%%%%%%%%%%%%%%%%%%%%%%%%%%%%%%%%%%%%%%%%%%%%%%%%%%%%%%%%%%%%%%%%%%%%%%%%%%%%%%%%%%%%%%%%%%%%%%%%%%%%%%%%%%%%%%%%%%%%%%%%%%%%%%%%
\subsection{Proof of Proposition \ref{prop: exp prop}}
\begin{proof}
From the limit \eqref{one point ratio limit}, for any $u\in (0,1)$ and $\varepsilon>0$, there is some $T_1>0$ such that when $t>T_1$,
\[
(1-\varepsilon)\langle v(t),\nu\rangle \phi(x)\leq v(t,x)\leq (1+\varepsilon)\langle v(t),\nu\rangle \phi(x),\qquad x\in E.
\]
Meanwhile since $1-e^{-G(f)}$ is a non-increasing function with respect to $f\in \mathcal B(E,[0,\infty])$, and $1-e^{-G(uf)}$ is a concave function with respect to $u$.  On one hand, we obtain that for $t>T_1$,
\begin{eqnarray}\label{lower}
&&\dfrac{1-e^{-G(u\langle v(t),\nu\rangle \phi)}}{1-e^{-G(\langle v(t),\nu\rangle \phi)}}\geq \dfrac{\dfrac{u}{1+\varepsilon}\left(1-e^{-G(v(t))}\right)}{1-e^{-G(\frac{1}{1-\varepsilon} v(t))}}=\dfrac{u}{1+\varepsilon}\left[e^{-\lambda t}\left(1-e^{-G(\frac{1}{1-\varepsilon} v(t))}\right)\right]^{-1}\nonumber\\
&&\geq \dfrac{u(1-\varepsilon)}{1+\varepsilon}\left[e^{-\lambda t}\left(1-e^{-G(v(t))}\right)\right]^{-1}=\dfrac{u(1-\varepsilon)}{1+\varepsilon}.
\end{eqnarray}
On the other hand, without lose of generality, we choose $\varepsilon>0$ satisfying $u/(1-\varepsilon)<1$, then
\begin{eqnarray}\label{upper}
&&\dfrac{1-e^{-G(u\langle v(t),\nu\rangle \phi)}}{1-e^{-G(\langle v(t),\nu\rangle \phi)}}\leq \dfrac{\dfrac{u}{1-\varepsilon}\left(1-e^{-G(v(t))}\right)}{1-e^{-G(\frac{1}{1+\varepsilon} v(t))}}=\dfrac{u}{1-\varepsilon}\left[e^{-\lambda t}\left(1-e^{-G(\frac{1}{1+\varepsilon} v(t))}\right)\right]^{-1}\\
&&\geq \dfrac{u(1+\varepsilon)}{1-\varepsilon}\left[e^{-\lambda t}\left(1-e^{-G(v(t))}\right)\right]^{-1}=\dfrac{u(1+\varepsilon)}{1-\varepsilon}.
\end{eqnarray}
Combining the two inequalities above, it follows that
\[
\lim_{t\to\infty}\dfrac{1-e^{-G(u\langle v(t),\nu\rangle \phi)}}{1-e^{-G(\langle v(t),\nu\rangle \phi)}}=u,\qquad u\in (0,1).
\]
And therefore, by Lemma \ref{lem:regu}, under the assumption \ref{assum},
\begin{equation}\label{eq regu}
1-e^{-G(u\phi)}\sim uL(u), u\rightarrow 0+,
\end{equation}
where $L$ is slowly varying at $0$.  Note that for any $u>0$, directing computation implies that
\[
\int_0^\infty e^{-us}{\mathbf Q}_\lambda(\langle\phi,\omega\rangle>s)ds=\dfrac{1-e^{-G(u\phi)}}{u}.
\]
From \eqref{eq regu}, Lemma\ref{lem: tau} and Lemma\ref{lem:tail}, there is some regularly varying function $L_1$ at infinity such that
\begin{equation}
{\mathbf Q}_r(\langle\phi,\omega\rangle>s)=\circ(s^{-\alpha}L_1(s)),\qquad s\to\infty, \alpha=\lambda/r
\end{equation}
Thus for any $0<\gamma<\alpha$,
\[
{\mathbf Q}_r\left(\langle\phi,\omega\rangle^{\gamma}\right)<\infty.
\]
The results in $(1)$ are obtained. \\
 Next let's prove $(2)$.  For any $f\in\mathcal B_b^+(E)$ and $t>0$, using the martingale change of probability $\mathbb P_\nu$, we obtain
\begin{eqnarray*}
&&\mathbb P_\nu\left(\exp\{-\langle f, X_t\rangle \};\zeta>t\right)\\
&=&\mathbb P_\nu\left(\dfrac{M_t(\phi)}{M_t(\phi)}\exp\{-\langle f, X_t\rangle \};\zeta>t\right)\\
&=&\widetilde{\mathbb P}_\nu\left(\dfrac{1}{M_t(\phi)}\exp\{-\langle f, X_t\rangle \}\right)\\
&=&e^{\lambda t}\mathbb Q_{\nu}\left(\dfrac{\exp\Big\{-\langle f, X_t\rangle -\langle f,  Z^{\mathrm m, [0,t)}_t+ Z^{\mathrm n, [0,t)}_t\rangle\Big \}}{\langle\phi, X_t\rangle +\langle\phi,  Z^{\mathrm m, [0,t)}_t+ Z^{\mathrm n, [0,t)}_t\rangle }\right)\\
&=&e^{\lambda t}\widehat{\mathbb Q}_{\nu}\left(\dfrac{\exp\Big\{-\langle f, X_t\rangle -\langle f,  \widehat Z^{\mathrm m}_t+ \widehat Z^{\mathrm n}_t\rangle\Big \}}{\langle\phi, X_t\rangle +\langle\phi,  \widehat Z^{\mathrm m}_t+ \widehat Z^{\mathrm n}_t\rangle }
\right).
\end{eqnarray*}
Since $\lim_{t\rightarrow\infty}X_t=0$ in probability and $\widehat Z^{\mathrm m}_t+ \widehat Z^{\mathrm n}_t$ is increasing having almost sure limit as $t\to\infty$.  Therefore, by dominated convergence theory,
\[
\lim_{t\to\infty}e^{-\lambda t}\mathbb P_\nu\left(\exp\{-\langle f, X_t\rangle \};\zeta>t\right)=\widehat{\mathbb Q}_{\nu}\left(\dfrac{\exp\left\{-\langle f,\sum_{\sigma\in\mathcal D^{\mathrm m}}\widehat X^{{\mathrm m},\sigma}_\sigma+\sum_{\tau\in\mathcal D^{\mathrm n}}\widehat X^{{\mathrm n},\tau}_\tau\rangle \right\}}{\langle \phi,\sum_{\sigma\in\mathcal D^{\mathrm m}}\widehat X^{{\mathrm m},\sigma}_\sigma+\sum_{\tau\in\mathcal D^{\mathrm n}}\widehat X^{{\mathrm n},\tau}_\tau\rangle}\right)
\]
Denote the random measure $\sum_{\sigma\in\mathcal D^{\mathrm m}}\widehat X^{{\mathrm m},\sigma}_\sigma+\sum_{\tau\in\mathcal D^{\mathrm n}}\widehat X^{{\mathrm n},\tau}_\tau$ by $X_\infty$.  Note that for the continuum immigration part,
\[
\widehat{\mathbb Q}_{\nu}\big(\sum_{\tau\in \mathcal D^{\mathrm n}}\langle f, \widehat X_{\tau}^{{\mathrm n},\tau} \rangle \big)=\int_0^\infty2\langle \alpha P^{\beta}_sf,\phi\nu\rangle ds
\leq 2\|\alpha\phi\|_\infty\dfrac{\langle f,\nu\rangle }{-\lambda}<\infty,
\]
and that for the discrete immigration part, when $\int_E\hat\phi(x)l(x)m(dx)<\infty$ by Lemma \ref{import lemma},
\begin{eqnarray*}
&&\widehat{\mathbb Q}_{\nu}\Big(\sum_{\sigma\in [1,\infty)\bigcap\mathcal D^{\mathrm m}}\langle f, \widehat X_{\sigma}^{{\mathrm m},\sigma} \rangle\Big|\mathcal G \Big)
=\sum_{t\in [1,\infty)\bigcap\mathcal D^{\mathrm m}}m_tP^{\beta}_tf( Y_t)\\
&&\leq \sum_{t\in \mathcal D^{\mathrm m}}(1+ce^{-\rho t})m_te^{\lambda t}
\phi(Y_t)\int_E\hat\phi(y)f(y)m(dy)\\
&&=\int_E\hat\phi(y)f(y)m(dy) \sum_{t\in \mathcal D^{\mathrm m}}(1+ce^{-\rho t})m_te^{\lambda t}\phi(Y_t)<\infty,\quad \widehat{\mathbb Q}_{\nu}-{\mathrm a.s.}
\end{eqnarray*}
 Thus in this case, the limit measure $X_\infty\in \mathcal M_f(E)$.  Denote the distribution of $X_\infty$ by $\mathbf Q$.  Then when $\int_E\hat\phi(x)l(x)m(dx)<\infty$,
\[
\lim_{t\rightarrow\infty}e^{-\lambda t}\mathbb P_\nu\left(\exp\{-\langle f, X_t\rangle \};\zeta>t\right)=\mathbf Q\left(\frac{1}{\langle\phi, \omega\rangle }\exp\{-\langle f, \omega\rangle \}\right).
\]
In Theorem \ref{thm: distribution of zeta} we have shown that
\[
\lim_{t\rightarrow\infty}e^{-\lambda t}\mathbb P_\nu(\zeta>t)=k={\mathbf Q} \left(\frac{1}{\langle\phi, \omega\rangle }\right)>0.
\]
Thus by the definition of the Yaglom distribution ${\mathbf Q}_\lambda $ of $(X,\mathbb P_\mu)$, when $\int_E\hat\phi(x)l(x)m(dx)<\infty$,
\begin{eqnarray*}
\mathbf Q_\lambda(\exp\{-\langle f, X_t\rangle \})&=&\lim_{t\rightarrow\infty}\mathbb P_\nu\left(\exp\{-\langle f, X_t\rangle \}\Big|\zeta>t\right)=\lim_{t\rightarrow\infty}\dfrac{\mathbb P_\nu\left(\exp\{-\langle f, X_t\rangle \};\zeta>t\right)}{\mathbb P_\nu(\zeta>t)}\\
&=&\dfrac{\lim_{t\rightarrow\infty}\widehat{\mathbb Q}_{\nu}\left(\dfrac{\exp\Big\{-\langle f, X_t\rangle -\langle f,  \widehat Z^{\mathrm m}_t+ \widehat Z^{\mathrm n}_t\rangle\Big \}}{\langle\phi, X_t\rangle +\langle\phi,  \widehat Z^{\mathrm m}_t+ \widehat Z^{\mathrm n}_t\rangle }
\right)}{\lim_{t\rightarrow\infty}e^{-\lambda t}\mathbb P_\mu(\zeta>t)}\\
&=&\dfrac{\mathbf Q\left(\dfrac{1}{\langle\phi, \omega\rangle }\exp\{-\langle f, \omega\rangle \}\right)}{{\mathbf Q}\left(\dfrac{1}{\langle\phi, \omega\rangle }\right)}.
\end{eqnarray*}
Or the Yaglom distribution ${\mathbf Q}_\lambda$ can be written as
\begin{equation}\label{rep: yaglom}
\mathbf Q_\lambda(\cdot)=\dfrac{1}{k}{\mathbf Q}\left(\dfrac{1}{\langle\phi, \omega\rangle }; \omega\in\cdot\right).
\end{equation}
Consequently
\begin{equation}\label{ident: k}
\mathbf Q_\lambda(\langle \phi,\omega\rangle)=\dfrac{1}{k}{\mathbf Q}\left(\dfrac{\langle \phi,\omega\rangle}{\langle \phi,\omega\rangle }\right)=\dfrac{1}{k}<\infty.
\end{equation}
When $\int_E\hat\phi(x)l(x)m(dx)=\infty$, it has been shown that $\lim_{t\to\infty}e^{-\lambda t}\langle v(t),\nu\rangle=k=0$.  And  recall that in \eqref{one point ratio limit}, it concluded the uniform convergence that
\[
\lim_{t\to\infty}\sup_{x\in E}\left|\dfrac{v(t+s,x)}{\langle v(t),\nu\rangle\phi(x)}-e^{\lambda s}\right|=0,\qquad \forall s\geq 0.
\]
Thus there is some $t_1>0$ such that when $t>t_0$, $v(t+1,x)\leq \langle v(t),\nu\rangle\phi(x)$, $x\in E$. From the basic inequality $1-e^{-s}\leq s$, for $s>0$, it follows that for $t>t_1$,
\[
\mathbf Q_\lambda(\langle\phi,\omega\rangle)\geq \dfrac{1-e^{-G(\langle v(t),\nu\rangle\phi)}}{\langle v(t),\nu\rangle}\geq \dfrac{1-e^{-G(v(t+1,x))}}{\langle v(t),\nu\rangle}=\dfrac{e^{\lambda(t+1)}}{\langle v(t),\nu\rangle}\to\infty, \quad {\rm as}\, t\to\infty.
\]
Therefore when   $\int_E\hat\phi(x)l(x)m(dx)=\infty$, $\mathbf Q_\lambda(\langle\phi,\omega\rangle)=\infty$.
\end{proof}
\begin{remark}
Note that the method to prove the infinity of the mean of $\langle\phi,X_\infty\rangle$ also can be used to prove its finiteness.  But the result we obtained  above is useful for the proof of next theorem.
\end{remark}
%%%%%%%%%%%%%%%%%%%%%%%%%%%%%%%%%%%%%%%%%%%%%%%%%%%%%%%%%%%%%%%%%%%%%%%%%%%%%%%%%%%%%%%%%%%%%%%%%%%%%%%%%%%%%%%%%%%%%%%%%%%%%%%%%%%
\subsection{Proof of Theorem \ref{thm: structure of Qprocess}}
\begin{proof}
It has been shown in section $2.3$, under $\widetilde{\mathbb P}_\mu$,  $X_t$ has a spine representation
(see Lemma \ref{spine structure}) for any $t>0$.  For any $f\in\mathcal B_b^+(E)$,
\[
\widetilde {\mathbb P}_{\mu}\left(e^{-\langle f, X_t\rangle }\right)=\mathbb Q_{\mu}\left(e^{-\langle f, X_t\rangle+\langle f, Z^{{\mathrm m},[0,t)}_t+Z^{{\mathrm n},[0,t)}_t\rangle }\right).
\]


When $\mu(dx)=\nu(dx)=\hat\phi(x)m(dx)$ and $\int_El(x)\nu(dx)<\infty$, it has been shown in the proof of Proposition \ref{prop: exp prop} that
$X_t$ under $\widetilde {\mathbb P}_{\nu}$ converges in distribution to $X_\infty\in\mathcal M_f(E)$ whose distribution is denoted by $\mathbf Q$.
It is also shown in \eqref{rep: yaglom} that $\mathbf Q$ is related to the Yaglom distribution, where
\begin{equation}\label{eq:2}
\mathbf Q_\lambda(\cdot)=\dfrac{1}{k}{\mathbf Q}\left(\dfrac{1}{\langle\phi, \omega\rangle }; \omega\in\cdot\right).
\end{equation}
Note that $k=[\mathbf Q_\lambda(\langle\phi, \omega\rangle)]^{-1}$. \eqref{eq:2} can be rewritten as
\begin{equation}\label{eq size bias}
{\mathbf Q}\left(\omega\in\cdot\right)=\dfrac{\mathbf Q_\lambda(\langle\phi, \omega\rangle; \omega\in \cdot)}{\mathbf Q_\lambda(\langle\phi, \omega\rangle)}.
\end{equation}
We are left to prove for all $\mu\in\mathcal M_f(E)\backslash\{0\}$, the distribution of $X_t$ under $\widetilde {\mathbb P}_{\nu}$ converges weakly to $\mathbf Q$.  For $f\in\mathcal B_b^+(E)$, define function
\begin{equation}\label{def: H}
H(x,t):={\mathbb Q}_x\left(e^{-\langle f, Z^{\mathrm n, [0,t)}_{t} + Z^{\mathrm m, [0,t)}_{t}\rangle }\right).
\end{equation}
Then
\begin{eqnarray*}
&&\mathbb Q_\nu\left(\exp\Big\{-\langle f, X_t\rangle-\langle f, Z^{{\mathrm m},[0,t)}_t+Z^{{\mathrm n},[0,t)}_t\rangle \Big\}\right)\\
&=&\mathbb P_\nu\left(e^{-\langle f, X_t\rangle}\right)\int_E\phi(y)\hat\phi(y)H(y,t)m(dy).
\end{eqnarray*}
%%%%%%%%%%%%%%%%%%%%%%%%%%%%%%%%%%%%%%%%%%%%%%%%%%%%%%%%%%%%%%%%%%%%%%%%%%%%%%%%%%%%%%%%%%%%%%%%%%%%%%%%%%%%%%%%%%%%%%%%%%%%%%%%%%%%%%%%%%%%%%%%%%%%%%%%%%%%%%%%%%%
Define the filtration generated by the spine process that $\mathcal{H}_t=\sigma\big(Y_s; s\leq t\big)$, $t\geq 0$.  Then for $T,t>0$,
\begin{equation}\label{subcritical upper bound}
 \begin{aligned}
 &H(x,t+T)\\
 =&\mathbb Q_{x}\mathbb Q_{x}\Big[\exp\Big\{-\sum_{\sigma\in (0, t+T]\bigcap \mathcal D^{\mathrm m}}\langle f, X_{t+T-\sigma}^{{\mathrm m},\sigma}\rangle -\sum_{\tau\in (0, t+T]\bigcap \mathcal D^{\mathrm n}}\langle f, X_{t+T-\tau}^{{\mathrm n}, \tau}\rangle \Big\}\Big| \mathcal H_t\Big]\\
 \leq&\widetilde\Pi_x\mathbb Q_{x}\Big[\exp\Big\{-\sum_{\sigma\in (t, t+T]\bigcap \mathcal D^{\mathrm m}}\langle f, X_{t+T-\sigma}^{{\mathrm m},\sigma}\rangle -\sum_{\tau\in (t, t+T]\bigcap \mathcal D^{\mathrm n}}\langle f, X_{t+T-\tau}^{{\mathrm n}, \tau}\rangle \Big\}\Big| \mathcal H_t\Big]\\
 =&
  \widetilde\Pi_x\mathbb Q_{Y_t}\Big[\exp\Big\{-\sum_{\sigma\in (0, T]\bigcap \mathcal D^{\mathrm m}}\langle \phi, X_{T-\sigma}^{{\mathrm m},\sigma}\rangle -\sum_{\tau\in (0, T]\bigcap \mathcal D^{\mathrm n}}\langle \phi, X_{T-\tau}^{{\mathrm n}, \tau}\rangle \Big\}\Big]\\
 =&\widetilde\Pi_x\left[ H(Y_t, T)\right].
 \end{aligned}
 \end{equation}
 From \eqref{IU}, there is some constants $c,\nu>0$ such that when $t>1$,
\[
 H(x,t+T)\leq \widetilde\Pi_x\left[ H(Y_t, T)\right]\leq (1+ce^{-\nu t})\int_E\phi(y)\hat\phi(y)H(y,T)m(dy)<\infty.
 \]
Since $H(x,t)\leq 1$.  Set $\overline \eta(x)$ to be the supremum limit of $H(x,t)$ as $t\to \infty$.
Fix time $T$ and let $t\to \infty$ in inequality \eqref{subcritical upper bound}. We can imply that
\begin{equation}\label{sub super}
\overline\eta(x)\leq \int_E\phi(y)\hat \phi(y)H(y,T)m(dy).
\end{equation}
   Using Fatou's lemma for supremum limit
in \eqref{sub super}, for any $x\in E$,
\begin{equation}\label{sup inequality}
\overline\eta(x)\leq \limsup_{T\rightarrow\infty}\int_E\phi(y)\hat \phi(y)H(y,T)m(dy)\leq \int_E\phi(y)\hat\phi(y)\overline{\eta}(y)m(dy).
\end{equation}
 Since $\overline{\eta}(\cdot)\leq 1$, $\overline\eta(\cdot)$ is a constant function by \eqref{sup inequality}.
 Denote $\overline\eta(\cdot)$ by $q(f)$.  If $q(f)\equiv 0,$ then
 \begin{equation}\label{limit}
 \lim_{t\rightarrow\infty}H(x,t)=q(f),\qquad \mbox{for all}\,\, x\in E.
 \end{equation}
  So in the
following, we assume $q(f)>0$.
 For any $\varepsilon_1>0$, let
$$
\mu_1(T)=\int_{\{x\in
E;H(x,T)>(1+\varepsilon_1)q(f)\}}\phi(x)\hat\phi(x)m(dx).
$$
Then $\lim_{T\rightarrow\infty}\mu_1(T)=0.$  For any $\varepsilon_2>0$, let
$$
\mu_2(T)=\int_{\{x\in
E;H(x,T)<(1-\varepsilon_2)q(f)\}}\phi(x)\hat\phi(x)m(dx).
$$
 Then we can deduce from \eqref{sup inequality} that
\begin{eqnarray}\label{sublimitinprob}
q(f)&\leq&
(1-\varepsilon_2)q(f)\mu_2(T)+\mu_1(T)+(1+\varepsilon_1)q(f)(1-\mu_1(T)-\mu_2(T))\\
&\le
&(1+\varepsilon_1)q(f)-(\varepsilon_1+\varepsilon_2)\mu_2(T)+C\mu_1(T),
\end{eqnarray}
where $C$ is some positive finite constant.  Hence
\begin{eqnarray*}\label{sublimitinequl}
q(f)&\leq&
\liminf_{T\rightarrow\infty}(1+\varepsilon_1)q(f)-(\varepsilon_1+\varepsilon_2)\mu_2(T)+C\mu_1(T)\\
&=&(1+\varepsilon_1)q(f)-(\varepsilon_1+\varepsilon_2)q(f)\limsup_{T\rightarrow\infty}\mu_1(T).
\end{eqnarray*}
Since $\varepsilon_1$ is an arbitrary positive constant.
\[
q(f)\leq q(f)-\varepsilon_1 q(f)\limsup_{T\rightarrow\infty}\mu_1(T).
\]
This is impossible unless $\limsup_{T\rightarrow\infty}\mu_1(T)=0.$
Therefore, $H(\cdot,T)$ converges to $q(f)$ in probability under probability $\phi(x)\hat{\phi}(x)m(dx)$ as $T\to\infty$.


Meanwhile, from the definition
\eqref{def: H} of function $H$, we can get the following inequality
\begin{equation}\label{subsub}
\begin{aligned}
     H(x,t+T)\geq& \mathbb Q_{x}\prod_{\sigma\leq t}I_{\{ X_{t+T-\sigma}^{{\mathrm m},\sigma}=0\}}\prod_{\tau\leq t}I_{\{ X_{t+T-\tau}^{{\mathrm n},\tau}=0\}}\\
&\cdot\mathbb Q_{Y_t}\Big[\exp\Big\{-\sum_{\sigma\in (0, T]\bigcap \mathcal D^{\mathrm m}}\langle \phi, X_{T-\sigma}^{{\mathrm m},\sigma}\rangle -\sum_{\tau\in (0, T]\bigcap \mathcal D^{\mathrm n}}\langle \phi, X_{T-\tau}^{{\mathrm n},\tau}\rangle \Big\}\Big]\\
=& \mathbb Q_{x}\left[\prod_{\sigma\leq t}I_{\{ X_{t+T-\sigma}^{{\mathrm m},\sigma}=0\}}\prod_{\tau\leq t}I_{\{ X_{t+T-\tau}^{{\mathrm n},\tau}=0\}}H(Y_t, T)\right].
\end{aligned}
\end{equation}
Consider the following probability,
\begin{eqnarray*}
\mathbb Q_{x}\left(\prod_{\sigma\leq t}I_{\{ X_{t+T-\sigma}^{{\mathrm m},\sigma}=0\}}=1\right)
=\widetilde\Pi_x\exp\left\{-\int_0^tds\int_0^\infty r(1-\mathbb P_{r\delta_{Y_s}}(\zeta<T+t-s))n(Y_s,dr)\right\}.
\end{eqnarray*}
Since in the case of $\lambda<0,$ the $(Y,\psi)$-superprocess starting from any finite measure is extinct in
finite time.  So by dominated convergence theorem,
\begin{equation}\label{1infty limit}
\lim_{T\rightarrow\infty}\int_0^tds\int_1^\infty r(1-\mathbb P_{r\delta_{Y_s}}(\zeta<T+t-s))n(Y_s,dr)=0,
\end{equation}
$\widehat\Pi_x$ almost surely.   Note that
\[
1-\mathbb P_{r\delta_{Y_s}}(\zeta<T+t-s)\leq 1-(1-\mathbb P_{Y_s}(\zeta>T))^r.
\]
And recall that there are $T_0>0$ and $\eta>0$ such that when $T>T_0$, for any $x\in E$,
\[
\mathbb P_x(\zeta>T)\leq \eta \phi(x)e^{\lambda T}.
\]
Since when $x\rightarrow 0+$, $1-(1-x)^r\sim rx$ for any $r>0$, we assume $T_0$ is sufficiently large such that
$\eta \phi(x)e^{\lambda T}$ is small enough so that $1-(1-\mathbb P_{Y_s}(\zeta>T))^r\leq 2r\eta \phi(Y_s)e^{\lambda T}$.
Therefore,
\[
\int_0^tds\int_0^1 r(1-\mathbb P_{r\delta_{ Y_s}}(\zeta<T+t-s))n(Y_s,dr)\leq 2\eta e^{\lambda T}\int_0^t\phi(Y_s)ds\int_0^1 r^2 n(Y_s,dr).
\]
As a result,
\begin{equation}\label{01limit}
\lim_{T\rightarrow\infty}\int_0^tds\int_0^1 r(1-\mathbb P_{r\delta_{Y_s}}(\zeta<T+t-s))n(Y_s,dr)=0,
\end{equation}
$\widehat\Pi_x$ almost surely.  Combining \eqref{1infty limit} and \eqref{01limit}, we get
\[
\lim_{T\rightarrow\infty}\mathbb Q_{x}\left(\prod_{\sigma\leq t}I_{\{ X_{t+T-\sigma}^{{\mathrm m},\sigma}=0\}}=1\right)=1.
\]
Meanwhile,
\begin{eqnarray*}
\mathbb Q_x\left(\prod_{\sigma\leq t}I_{\{ X_{t+T-\sigma}^{{\mathrm n},\sigma}=0\}}=1\right)
&=&\widetilde\Pi_x\exp\left\{-\int_0^t2\alpha(Y_s)\mathbb N_{Y_s}(\zeta<T+t-s)ds\right\}\\
&=&\widetilde\Pi_x\exp\left\{-\int_0^t2\alpha(Y_s)v(T+t-s,Y_s)ds\right\}.
\end{eqnarray*}
We also get
\[
\lim_{T\rightarrow\infty}\mathbb Q_x\left(\prod_{\tau\leq t}I_{\{ X_{t+T-\tau}^{{\mathrm n},\tau}=0\}}=1\right)=1,
\]
since $\lim_{T\rightarrow\infty} v(T+t-s,x)=0$ for any $x$, and it is bounded when time $T$ is sufficiently large.  By the inequality \eqref{IU}, for any $\varepsilon>0$ and $t>1$, there are $c>0$ and $\nu>0$, such that for any $x\in E$,
\begin{eqnarray*}
&&\limsup_{T\rightarrow\infty}\widetilde\Pi_x\left(|H(Y_t, T)-q(f)|>\varepsilon\right)\\
&\leq& \limsup_{T\rightarrow\infty}(1+ce^{-\nu t})\int_E\phi(y)\hat\phi(y)m(dy)I_{\{|H(y, T)-q(f)|>\varepsilon\}}=0.
\end{eqnarray*}
Then from the inequality \eqref{subsub}, we have for any $x\in E$,
\begin{eqnarray*}
\liminf_{T\rightarrow\infty}H(x, t+T)&\geq&  \liminf_{T\rightarrow\infty} \mathbb Q_x\left[\prod_{\sigma\leq t}I_{\{ X_{t+T-\sigma}^{{\mathrm m},\sigma}=0\}}\prod_{\tau\leq t}I_{\{ X_{t+T-\tau}^{{\mathrm n},\tau}=0\}}H(Y_t, T)\right]\\
&\geq& q(f)=\limsup_{t\rightarrow\infty}H(x, t).
\end{eqnarray*}
 Therefore  $\lim_{t\rightarrow\infty}H(x, t)=q(f)$ for any $x\in E$.
 %%%%%%%%%%%%%%%%%%%%%%%%%%%%%%%%%%%%%%%%%%%%%%%%%%%%%%%%%%%%%%%%%%%%%%%%%%%%%%%%%%%%%%%%%%%%%%%%%%%%%%%%%%%%%%%%%%%%%%%%%%%%%%%%%%%%%%%%%%%%%%%%%%%%%%%%%%%%%%%%%%%
Due to $0\leq H(x,t)\leq 1$,
\begin{equation*}
q(f)=\lim_{t\rightarrow\infty}\int_E\phi(x)\hat\phi(x)H(x,t)m(dx)
=\lim_{t\rightarrow\infty}\widetilde{\mathbb P}_{\nu}\left(e^{-\langle f, X_t\rangle }\right)
=\widehat{\mathbb Q}_{\nu}\left(e^{-\langle f, X_{\infty}\rangle }\right).
\end{equation*}
Therefore for any $\mu\in\mathcal M_f(E)\backslash\{0\}$, and $f\in\mathcal B_b^+(E)$,
\begin{eqnarray*}
\lim_{t\rightarrow\infty}\widetilde{\mathbb P}_\mu\left(e^{-\langle f, X_t\rangle}\right)&=&\lim_{t\rightarrow\infty}\mathbb P_\mu\left(e^{-\langle f, X_t\rangle}\right)
\lim_{t\to\infty}\dfrac{1}{\mu(\phi)}\int_E\phi(x)H(x, t)\mu(dx)=q(f)\\
&=&\widehat{\mathbb Q}_{\nu}\left(e^{-\langle f, X_{\infty}\rangle }\right).
\end{eqnarray*}
%%----------------------------------------------------------------------------------------------------------------------------
This says $\mathbf Q$ is the equilibrium probability of the $Q$ process.  And it is a size-biased distribution of the Yaglom probability $\mathbf Q_\lambda$ with weight function $\dfrac{\langle\phi,\omega\rangle}{\mathbf Q_\lambda(\langle\phi,\omega\rangle)}$.  The proof of $(1)$ is finished.


When $\int_E\hat\phi(x)l(x)m(dx)=\infty$, it is shown in theorem \ref{thm: distribution of zeta} that
 $\sum_{s\in\mathcal D^{\mathrm m}} \langle \phi,\widehat X^{{\mathrm m},s}_s\rangle =\infty$, $\widehat{\mathbb Q}_\nu$ almost surely. Thus
\[
\langle \phi, X_{\infty}\rangle =\infty,\qquad \widehat{\mathbb Q}_\nu-{\mathrm a.s.}
\]
In this case, the $Q$ process does not have equilibrium probability. And $\langle \phi, X_t\rangle $ converges to $\infty$ as $t\to\infty$ in probability with respect to $\widetilde{\mathbb P}_\mu$ for any $\mu\in \mathcal M_f(E)\backslash\{0\}$.
\end{proof}









\begin{thebibliography}{99}

%new added
\bibitem{AH}Asmussens, S. and Hering, H. :\emph{Branching Processes}. Birkhauser, Boston, 1983.

\bibitem{AthreyaNey1972Branching}
Athreya, K. B. and Ney, P. E.:
\emph{Branching processes.}
Die Grundlehren der mathematischen Wissenschaften, Band 196. Springer-Verlag, New York-Heidelberg, 1972. xi+287 pp.
\MR{0373040}
%end new

\bibitem{Dudley2002Real}
Dudley, R. M.:
\emph{Real analysis and probability.}
Revised reprint of the 1989 original. Cambridge Studies in Advanced Mathematics, 74. Cambridge University Press, Cambridge, 2002. x+555 pp.

\bibitem{KimSong2008Intrinsic}
Kim, P., Song, R.:
\emph{Intrinsic ultracontractivity of non-symmetric diffusion semigroups in bounded domains.}
Tohoku Math. J. (2) 60 (2008), no. 4, 527-547.

%new added

\bibitem{Heathcote}Heathcote, R.,  Seneta, E.  and Vere-Jones, D.:  \emph{ A refinement of two theorems in the
theory of branching processes}. Theory Probab. Appl. 12 (1982), 297-301.
\bibitem{Joffe}Joffe, A. and Waughw, A. O.:  Exact distributions of kin numbers in a Galton-Watson
process. J. Appl. Probab. 19 (1982), 767-775.

\bibitem{Li00} Li, Z.-H.:
\emph{Asymptotic behavior of continuous time and state branching processes}. J. Aus. Math. Soc.
Series A 68(2000), 68C84. MR1727226
%end new

\bibitem{Li2011MeasureValued}
Li, Z.:
\emph{Measure-valued branching Markov processes.}
Probability and its Applications (New York). Springer, Heidelberg, 2011. xii+350 pp.

%new added
\bibitem{MeleardVillemonais2012Quasi-stationary}
M\'el\'eard, S. and Villemonais, D.:
\emph{Quasi-stationary distributions and population processes.}
Probab. Surv. \textbf{9} (2012), 340C410.
\MR{2994898}
%end new

\bibitem{RenSongZhang2015Limit}
Ren, Y.-X., Song, R., and Zhang, R.:
\emph{Limit theorems for some critical superprocesses.}
Illinois J. Math. 59 (2015), no. 1, 235-276.

\bibitem{RenSongZhang2017Central}
Ren, Y.-X., Song, R., and Zhang, R.:
\emph{Central limit theorems for supercritical branching nonsymmetric Markov processes.}
Ann. Probab. 45 (2017), no. 1, 564-623.

\bibitem{Schaefer1974Banach}
Schaefer, H. H.:
\emph{Banach lattices and positive operators.}
Die Grundlehren der mathematischen Wissenschaften, Band 215. Springer-Verlag, New York-Heidelberg, 1974. xi+376 pp.


\bibitem{Yaglom47}Yaglom, A. M.: \emph{Certain limit theorems of the theory of branching processes}. Dokl. Acad.
Nauk. SSSR 56 (1947), 795--798.

\bibitem{AthreyaNey1972Branching}
Athreya, K. B. and Ney, P. E.:
\emph{Branching processes.}
Die Grundlehren der mathematischen Wissenschaften, Band 196. Springer-Verlag, New York-Heidelberg, 1972. xi+287 pp.
\MR{0373040}

\bibitem{BigginsKyprianou2004Measure}
Biggins, J. D. and Kyprianou, A. E.:
\emph{Measure change in multitype branching.}
Adv. in Appl. Probab. \textbf{36} (2004), no. 2, 544--581.
\MR{2058149}

\bibitem{ChampagnatRoelly2008Limit}
Champagnat, N. and Roelly, S.:
\emph{Limit theorems for conditioned multitype Dawson-Watanabe processes and Feller diffusions.}
Electron. J. Probab. \textbf{13} (2008), no. 25, 777C810.
\MR{2399296}

\bibitem{ChampagnatVillemonais2018Convergence}
Champagnat, N. and Villemonais, D.:
\emph{Convergence of the Fleming-Viot process toward
theminimal quasi-stationary distribution.}
https://arxiv.org/pdf/1810.06849.pdf

\bibitem{ChenRenYang2017Skeleton}
Chen, Z.-Q., Ren, Y.-X. and Yang, T.:
\emph{Skeleton decomposition and law of large numbers for supercritical superprocesses.}
Acta Appl. Math. (2017), 1--61.
ARXIV{1709.00847}

\bibitem{Dawson1992Infinitely}
Dawson, D. A.:
\emph{Infinitely divisible random measures and superprocesses.}Stochastic analysis and related topics (Silivri, 1990), 1--129,
Progr. Probab., 31, Birkh{\"a}user Boston, Boston, MA, 1992.
\MR{1203373}

\bibitem{DelmasHenard2013A-Williams}
Delmas, J.-F. and H\'enard, O.:
\emph{A Williams decomposition for spatially dependent super-processes. }
Electron. J. Probab. \textbf{18} (2013), no. 37, 43 pp.
\MR{3035765}

\bibitem{Dynkin1993Superprocesses}
Dynkin, E. B.:
\emph{Superprocesses and partial differential equations.}
Ann. Probab. \textbf{21} (1993), no. 3, 1185--1262.
\MR{1235414}

\bibitem{EnglanderKyprianou2004Local}
Engl\"ander, J. and Kyprianou, A. E.:
\emph{Local extinction versus local exponential growth for spatial branching processes.}
Ann. Probab. \textbf{32} (2004), no. 1A, 78--99.
\MR{2040776}

\bibitem{Evans1993Two}
Evans, S. N.:
\emph{Two representations of a conditioned superprocess.}
Proc. Roy. Soc. Edinburgh Sect. A \textbf{123} (1993), no. 5, 959--971.
\MR{1249698}

\bibitem{Grey1974Asymptotic}
Grey, D. R.:
\emph{Asymptotic behaviour of continuous time, continuous state-space branching processes.}
J. Appl. Probability \textbf{11} (1974), 669--677.
\MR{0408016}

\bibitem{HeathcoteSenetaVere-Jones1967A-refinement}
Heathcote, C. R., Seneta, E. and Vere-Jones, D.:
\emph{A refinement of two theorems in the theory of branching processes.} (Russian summary)
Teor. Verojatnost. i Primenen. \textbf{12} 1967 341--346.
\MR{0217889}

\bibitem{KimSong2008Intrinsic}
Kim, P. and Song, R.:
\emph{Intrinsic ultracontractivity of non-symmetric diffusion semigroups in bounded domains.}
Tohoku Math. J. (2) \textbf{60} (2008), no. 4, 527--547.
\MR{2487824}

\bibitem{KimSong2008Intrinsic2}
Kim, P. and Song, R.:
\emph{Intrinsic ultracontractivity of nonsymmetric diffusions with measure-valued drifts and potentials.}
Ann. Probab. \textbf{36} (2008), no. 5, 1904C1945.
\MR{2440927}

\bibitem{KimSong2009Intrinsic}
Kim, P. and Song, R.:
\emph{Intrinsic ultracontractivity for non-symmetric L\'evy processes.}
Forum Math. \textbf{21} (2009), no. 1, 43C66.
\MR{2494884}

\bibitem{Lambert2001Arbres}
Lambert, A.:
\emph{Arbres, excursions et processus de L\'evy completement asym\'etriques.}
Diss. Universit Pierre et Marie Curie-Paris VI, 2001.

\bibitem{Lambert2003Coalescence}
Lambert, A.:
\emph{Coalescence times for the branching process.}
Adv. in Appl. Probab. \textbf{35} (2003), no. 4, 1071--1089.
\MR{2014270}

\bibitem{Lambert2007Quasi-stationary}
Lambert, A.:
\emph{Quasi-stationary distributions and the continuous-state branching process conditioned to be never extinct.}
Electron. J. Probab. \textbf{12} (2007), no. 14, 420--446.
\MR{2299923}

\bibitem{Li2000Asymptotic}
Li, Z.-H.:
\emph{Asymptotic behaviour of continuous time and state branching processes.}
J. Austral. Math. Soc. Ser. A \textbf{68} (2000), no. 1, 68--84.
\MR{1727226}

\bibitem{Li2011Measure-valued}
Li, Z.:
\emph{Measure-valued branching Markov processes.}
Probability and its Applications (New York). Springer, Heidelberg, 2011. xii+350 pp. ISBN: 978-3-642-15003-6
\MR{2760602}

\bibitem{LiuRenSong2009Llog}
Liu, R.-L., Ren, Y.-X. and Song, R.:
\emph{{$L \log L$} criterion for a class of superdiffusions.}
J. Appl. Probab. \textbf{46} (2009), no. 2, 479C496.
\MR{2535827}

\bibitem{LyonsPemantlePeres1995Conceptual}
Lyons, R., Pemantle, R. and Peres, Y.:
\emph{Conceptual proofs of $L\log L$ criteria for mean behavior of branching processes.}
Ann. Probab. \textbf{23} (1995), no. 3, 1125--1138.
\MR{1349164}

\bibitem{MeleardVillemonais2012Quasi-stationary}
M\'el\'eard, S. and Villemonais, D.:
\emph{Quasi-stationary distributions and population processes.}
Probab. Surv. \textbf{9} (2012), 340C410.
\MR{2994898}

\bibitem{Nagasawa1964Time}
Nagasawa, M.:
\emph{Time reversions of Markov processes.}
Nagoya Math. J. \textbf{24} (1964), 177--204.
\MR{0169290}

\bibitem{RenSongSun2017Spine}
Ren, Y.-X., Song, R. and Sun, Z.:
\emph{Spine decompositions and limit theorems for a class of critical superprocesses.}
Preprint.
ARXIV{1711.09188}

\bibitem{RenSongYang2016Spine}
Ren, Y.-X., Song, R. and Yang, T.:
\emph{Spine decomposition and {$ L\log L $} criterion for superprocesses with non-local branching mechanisms.}
Preprint.
ARXIV{1609.02257}

\bibitem{RenSongZhang2015Limit}
Ren, Y.-X., Song, R. and Zhang, R.:
\emph{Limit theorems for some critical superprocesses.}
Illinois J. Math. \textbf{59} (2015), no. 1, 235C276.
\MR{3459635}

\bibitem{RenSongZhang2017Central}
Ren, Y.-X., Song, R. and Zhang, R.:
\emph{Central limit theorems for supercritical branching nonsymmetric Markov processes.}
Ann. Probab. \textbf{45} (2017), no. 1, 564C623.
\MR{3601657}

\bibitem{RenSongZhang2018Williams}
Ren, Y.-X., Song, R. and Zhang, R.:
\emph{Williams decomposition for superprocesses.}
Electron. J. Probab. \textbf{23} (2018), Paper No. 23, 33 pp.
\MR{3771760}

\bibitem{RoellyRouault1989Processus}
Roelly, S. and Rouault, A.:
\emph{Processus de Dawson-Watanabe conditionn\'e par le futur lointain.} (French. English summary) [A Dawson-Watanabe process conditioned by the remote future]
C. R. Acad. Sci. Paris Sr. I Math. \textbf{309} (1989), no. 14, 867--872.
\MR{1055211}

\bibitem{Penisson2010Conditional}
P\'enisson, S.:
\emph{Conditional limit theorems for multitype branching processes and illustration in epidemiological risk analysis.}Diss. Universitt Potsdam, Universit Paris Sud-Paris XI, 2010.

\bibitem{Schaefer1974Banach}
Schaefer, H. H.:
\emph{Banach lattices and positive operators.}
Die Grundlehren der mathematischen Wissenschaften, Band 215. Springer-Verlag, New York-Heidelberg, 1974.
\MR{0423039}

\bibitem{Yaglom1947Certain}
Yaglom, A. M.:
\emph{Certain limit theorems of the theory of branching random processes.} (Russian)
Doklady Akad. Nauk SSSR (N.S.) \textbf{56} (1947), 795--798.
\MR{0022045}
\end{thebibliography}
\end{document}
